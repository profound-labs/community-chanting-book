\documentclass[
  babelLanguage=portuguese,
  final,
  %showtrims,
  %showwirebinding,
  %webversion,
  a4Paper,
]{chantingbook}

% Sources:
%
% Bhikkhu aparihānīya 1 [Sattaka 1] AN iv 21
% https://suttacentral.net/pi/an7.23
%
% Bhikkhu-aparihaniya Sutta: Conditions for No Decline Among the Monks
% translated from the Pali by Thanissaro Bhikkhu
% https://www.accesstoinsight.org/tipitaka/an/an07/an07.021.than.html

\usepackage{local}

\title{Bhikkhu-aparihānīya-dhamma-sutta}
\subtitle{Seven Conditions for the Welfare of the Bhikkhus}

\begin{document}

\mainmatter

\eveningChapterSettings

\pagestyle{empty}

\chapter{Bhikkhu-aparihānīya-dhamma-sutta}

\emph{Seven Conditions for the Welfare of the Bhikkhus, AN 7.23}

\begin{leader}
  [Handa mayaṁ bhikkhu-aparihānīya-dhamma-sutta-pāṭhaṁ bhaṇāmase]
\end{leader}

[Evaṁ me sutaṁ.] Ekaṁ samayaṁ bhagavā rājagahe꜔꜒ viharati gijjhakūṭe pabbate.
Tatra kho꜔꜒ bhagavā bhikkhū꜔꜒ āmantesi: Satta vo, bhikkhave, aparihā꜔꜒niye dhamme
desessā꜔꜒mi. Taṁ suṇātha, sā꜔꜒dhukaṁ manasi karotha, bhāsissā꜔꜒mī'ti. Evaṁ, bhante'ti
kho꜔꜒ te bhikkhū꜔꜒ bhagavato paccasso꜔꜒su꜔꜒ṁ. Bhagavā etadavoca:

\begin{english}
  I have heard that on one occasion the Blessed One was staying in Rajagaha, on
  Vulture Peak. There he addressed the monks: “Monks, I will teach you the seven
  conditions that lead to no decline. Listen \& pay close attention. I will
  speak.” “Yes, lord,” the monks responded. The Blessed One said:
\end{english}

[1] Katame ca, bhikkhave, satta aparihā꜔꜒niyā dhammā? Yāvakīvañca, bhikkhave, bhikkhū꜔꜒
abhiṇha꜔꜒ṁ sa꜔꜒nnipātā bhavissa꜔꜒nti sa꜔꜒nnipātabahulā; vuddhiyeva, bhikkhave,
bhikkhū꜔꜒naṁ pāṭikaṅkhā꜔꜒, no parihā꜔꜒ni.

\begin{english}
  “And which seven are the conditions that lead to no decline? As long as the
  monks meet often, meet a great deal, their growth can be expected, not their
  decline.
\end{english}

[2] Yāvakīvañca, bhikkhave, bhikkhū꜔꜒ samaggā sa꜔꜒nnipatissa꜔꜒nti, samaggā
vuṭṭhahissa꜔꜒nti, samaggā sa꜔꜒ṅghakaraṇīyāni karissa꜔꜒nti; vuddhiyeva, bhikkhave,
bhikkhū꜔꜒naṁ pāṭikaṅkhā꜔꜒, no parihā꜔꜒ni.

\begin{english}
  “As long as the monks meet in harmony, adjourn from their meetings in harmony,
  and conduct Sangha business in harmony, their growth can be expected, not
  their decline.
\end{english}

[3] Yāvakīvañca, bhikkhave, bhikkhū꜔꜒ apaññattaṁ na paññāpessa꜔꜒nti, paññattaṁ na
samucchi꜔꜒ndissa꜔꜒nti, yathā꜔꜒paññattesu sikkhā꜔꜒padesu samādāya vattissa꜔꜒nti;
vuddhiyeva, bhikkhave, bhikkhū꜔꜒naṁ pāṭikaṅkhā꜔꜒, no parihā꜔꜒ni.

\begin{english}
  “As long as the monks neither decree what has been undecreed nor repeal what
  has been decreed, but practice undertaking the training rules as they have
  been decreed, their growth can be expected, not their decline.
\end{english}

[4] Yāvakīvañca, bhikkhave, bhikkhū꜔꜒ ye te bhikkhū꜔꜒ the꜔꜒rā rattaññū cirapabbajitā
sa꜔꜒ṅghapitaro sa꜔꜒ṅghapariṇāyakā te sakkarissa꜔꜒nti garuṁ karissa꜔꜒nti mānessa꜔꜒nti
pūjessa꜔꜒nti, tesa꜔꜒ñca so꜔꜒tabbaṁ maññissa꜔꜒nti; vuddhiyeva, bhikkhave, bhikkhū꜔꜒naṁ
pāṭikaṅkhā꜔꜒, no parihā꜔꜒ni.

\begin{english}
  “As long as the monks honor, respect, venerate, and do homage to the elder
  monks — those with seniority who have long been ordained, the fathers of the
  Sangha, leaders of the Sangha — regarding them as worth listening to, their
  growth can be expected, not their decline.
\end{english}

[5] Yāvakīvañca, bhikkhave, bhikkhū꜔꜒ uppannāya taṇhā꜔꜒ya ponobhavikāya na vasa꜔꜒ṁ
gacchissa꜔꜒nti; vuddhiyeva, bhikkhave, bhikkhū꜔꜒naṁ pāṭikaṅkhā꜔꜒, no parihā꜔꜒ni.

\begin{english}
  “As long as the monks do not submit to the power of any arisen craving that
  leads to further becoming, their growth can be expected, not their decline.
\end{english}

[6] Yāvakīvañca, bhikkhave, bhikkhū꜔꜒ āraññakesu se꜔꜒nāsanesu sā꜔꜒pekkhā꜔꜒ bhavissa꜔꜒nti;
vuddhiyeva, bhikkhave, bhikkhū꜔꜒naṁ pāṭikaṅkhā꜔꜒, no parihā꜔꜒ni.

\begin{english}
  “As long as the monks see their own benefit in wilderness dwellings, their
  growth can be expected, not their decline.
\end{english}

[7] Yāvakīvañca, bhikkhave, bhikkhū꜔꜒ paccattaññeva satiṁ upaṭṭhā꜔꜒pessa꜔꜒nti: Kinti
anāgatā ca pesalā sabrahmacārī āgacche꜔꜒yyuṁ, āgatā ca pesalā sabrahmacārī phā꜔꜒su꜔꜒ṁ
vihareyyun'ti; vuddhiyeva, bhikkhave, bhikkhū꜔꜒naṁ pāṭikaṅkhā꜔꜒, no parihā꜔꜒ni.

\begin{english}
  “As long as the monks each keep firmly in mind: `If there are any well-behaved
  fellow followers of the chaste life who have yet to come, may they come; and
  may the well-behaved fellow-followers of the chaste life who have come live in
  comfort,' their growth can be expected, not their decline.
\end{english}

Yāvakīvañca, bhikkhave, ime satta aparihā꜔꜒niyā dhammā bhikkhū꜔꜒su ṭhassa꜔꜒nti, imesu
ca sattasu aparihā꜔꜒niyesu dhammesu bhikkhū꜔꜒ sa꜔꜒ndississa꜔꜒nti; vuddhiyeva, bhikkhave,
bhikkhū꜔꜒naṁ pāṭikaṅkhā꜔꜒, no parihā꜔꜒nī'ti. Idam-avoca Bhagavā. Attamanā te bhikkhū꜔꜒
Bhagavato bhāsitaṁ abhinandun'ti.

\begin{english}
  “As long as the monks remain steadfast in these seven conditions, and as long
  as these seven conditions endure among the monks, the monks' growth can be
  expected, not their decline.” That is what the Blessed One said. Gratified,
  the monks delighted in the Blessed One's words.
\end{english}

\end{document}
