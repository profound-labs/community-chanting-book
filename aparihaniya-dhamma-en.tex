\documentclass[
  babelLanguage=portuguese,
  final,
  %showtrims,
  %showwirebinding,
  %webversion,
  a4Paper,
]{chantingbook}

% Sources:
%
% Bhikkhu aparihānīya 1 [Sattaka 1] AN iv 21
% https://suttacentral.net/pi/an7.23
%
% Bhikkhu-aparihaniya Sutta: Conditions for No Decline Among the Monks
% translated from the Pali by Thanissaro Bhikkhu
% https://www.accesstoinsight.org/tipitaka/an/an07/an07.021.than.html

\usepackage{local}

\title{Bhikkhu-aparihānīyā-dhammā Sutta}
\subtitle{Seven Conditions for the Welfare of the Bhikkhus}

\begin{document}

\mainmatter

\eveningChapterSettings

\pagestyle{empty}

\chapter{Bhikkhu-aparihānīyā-dhammā Sutta}

\emph{Seven Conditions for the Welfare of the Bhikkhus, AN 7.23}

\begin{leader}
  [Handa mayaṃ bhikkhu-aparihānīyā-dhammā-sutta-pāṭhaṃ bhaṇāmase]
\end{leader}

[Evaṃ me sutaṃ.] Ekaṃ samayaṃ bhagavā rājagahe viharati gijjhakūṭe pabbate.
Tatra kho bhagavā bhikkhū āmantesi: “satta vo, bhikkhave, aparihāniye dhamme
desessāmi. Taṃ suṇātha, sādhukaṃ manasi karotha, bhāsissāmī”ti. “Evaṃ, bhante”ti
kho te bhikkhū bhagavato paccassosuṃ. Bhagavā etadavoca:

\begin{english}
  I have heard that on one occasion the Blessed One was staying in Rajagaha, on
  Vulture Peak. There he addressed the monks: “Monks, I will teach you the seven
  conditions that lead to no decline. Listen \& pay close attention. I will
  speak.” “Yes, lord,” the monks responded. The Blessed One said:
\end{english}

[1] “Katame ca, bhikkhave, satta aparihāniyā dhammā? Yāvakīvañca, bhikkhave, bhikkhū
abhiṇhaṃ sannipātā bhavissanti sannipātabahulā; vuddhiyeva, bhikkhave,
bhikkhūnaṃ pāṭikaṅkhā, no parihāni.

\begin{english}
  “And which seven are the conditions that lead to no decline? As long as the
  monks meet often, meet a great deal, their growth can be expected, not their
  decline.
\end{english}

[2] “Yāvakīvañca, bhikkhave, bhikkhū samaggā sannipatissanti, samaggā
vuṭṭhahissanti, samaggā saṅghakaraṇīyāni karissanti; vuddhiyeva, bhikkhave,
bhikkhūnaṃ pāṭikaṅkhā, no parihāni.

\begin{english}
  “As long as the monks meet in harmony, adjourn from their meetings in harmony,
  and conduct Sangha business in harmony, their growth can be expected, not
  their decline.
\end{english}

[3] “Yāvakīvañca, bhikkhave, bhikkhū apaññattaṃ na paññāpessanti, paññattaṃ na
samucchindissanti, yathāpaññattesu sikkhāpadesu samādāya vattissanti;
vuddhiyeva, bhikkhave, bhikkhūnaṃ pāṭikaṅkhā, no parihāni.

\begin{english}
  “As long as the monks neither decree what has been undecreed nor repeal what
  has been decreed, but practice undertaking the training rules as they have
  been decreed, their growth can be expected, not their decline.
\end{english}

[4] “Yāvakīvañca, bhikkhave, bhikkhū ye te bhikkhū therā rattaññū cirapabbajitā
saṅghapitaro saṅghapariṇāyakā te sakkarissanti garuṃ karissanti mānessanti
pūjessanti, tesañca sotabbaṃ maññissanti; vuddhiyeva, bhikkhave, bhikkhūnaṃ
pāṭikaṅkhā, no parihāni.

\begin{english}
  “As long as the monks honor, respect, venerate, and do homage to the elder
  monks — those with seniority who have long been ordained, the fathers of the
  Sangha, leaders of the Sangha — regarding them as worth listening to, their
  growth can be expected, not their decline.
\end{english}

[5] “Yāvakīvañca, bhikkhave, bhikkhū uppannāya taṇhāya ponobhavikāya na vasaṃ
gacchissanti; vuddhiyeva, bhikkhave, bhikkhūnaṃ pāṭikaṅkhā, no parihāni.

\begin{english}
  “As long as the monks do not submit to the power of any arisen craving that
  leads to further becoming, their growth can be expected, not their decline.
\end{english}

[6] “Yāvakīvañca, bhikkhave, bhikkhū āraññakesu senāsanesu sāpekkhā bhavissanti;
vuddhiyeva, bhikkhave, bhikkhūnaṃ pāṭikaṅkhā, no parihāni.

\begin{english}
  “As long as the monks see their own benefit in wilderness dwellings, their
  growth can be expected, not their decline.
\end{english}

[7] “Yāvakīvañca, bhikkhave, bhikkhū paccattaññeva satiṃ upaṭṭhāpessanti: ‘kinti
anāgatā ca pesalā sabrahmacārī āgaccheyyuṃ, āgatā ca pesalā sabrahmacārī phāsuṃ
vihareyyun’ti; vuddhiyeva, bhikkhave, bhikkhūnaṃ pāṭikaṅkhā, no parihāni.

\begin{english}
  “As long as the monks each keep firmly in mind: `If there are any well-behaved
  fellow followers of the chaste life who have yet to come, may they come; and
  may the well-behaved fellow-followers of the chaste life who have come live in
  comfort,' their growth can be expected, not their decline.
\end{english}

“Yāvakīvañca, bhikkhave, ime satta aparihāniyā dhammā bhikkhūsu ṭhassanti, imesu
ca sattasu aparihāniyesu dhammesu bhikkhū sandississanti; vuddhiyeva, bhikkhave,
bhikkhūnaṃ pāṭikaṅkhā, no parihānī”ti. Idam-avoca Bhagavā. Attamanā te bhikkhū
Bhagavato bhāsitaṃ, abhinandun'ti.

\begin{english}
  “As long as the monks remain steadfast in these seven conditions, and as long
  as these seven conditions endure among the monks, the monks' growth can be
  expected, not their decline.” That is what the Blessed One said. Gratified,
  the monks delighted in the Blessed One's words.
\end{english}

\end{document}
