\documentclass[
  babelLanguage=english,
  final,
  webversion,
  %showtrims,
]{chantingbook}

\usepackage{local}

\makeatletter

\newcommand{\verseref}[1]{\sidepar{#1}}

\definecolor{titlecolor}{HTML}{1D296C}

\makeatother

\title{Sumedharama Monastic Chants}

\begin{document}

\frontmatter

\webcover{%
\centering

\includegraphics[width=130mm]{wave}
  
\vspace*{1cm} 

\resizebox{70mm}{!}{\color{titlecolor}\Calluna\textls{CÂNTICOS}}%

\vspace*{1cm} 

\includegraphics[width=130mm, angle=180]{wave}

\vspace*{1cm}
}

%\thispagestyle{empty}\mbox{}

\cleartorecto
\tableofcontents*

\mainmatter

\usePsMarksTitleOnly

\chapter[Incondicionado]{Reflexão sobre o Incondicionado}

\firstline{Atthi bhikkhave ajātaṁ abhūtaṁ akataṁ}

\begin{leader}
  [Ha꜓nda mayaṁ nibbāna-sutta-pāṭhaṁ bha꜕ṇāmase]
\end{leader}

Atthi bhi꜓kkha꜕ve a꜕jātaṁ a꜓bhūtaṁ a꜕kataṁ a꜕sa꜓ṅkh꜕ataṁ

\begin{english}
  Existe um Não-nascido, Não-originado, Incriado, Não-formado.
\end{english}

N꜕o cetaṁ bhi꜓kkha꜕ve a꜕bhavissa a꜕jātaṁ a꜓bhūtaṁ a꜕kataṁ a꜕sa꜓nkh꜕ataṁ

\begin{english}
 Se não existisse este Não-nascido, Não-originado, Incriado, Não-formado,
\end{english}

Na꜕ yidaṁ jātassa꜕ bhūtassa ka꜕tassa sa꜓ṅkh꜕atassa nissaraṇaṁ paññāye꜓tha

\begin{english}
  A libertação do mundo do nascido, originado, criado, formado, não seria possível.
\end{english}

Ya꜕smā ca kho bhi꜓kkh꜕ave atthi a꜕jātaṁ a꜓bhūtaṁ a꜕kataṁ a꜕sa꜓ṅkha꜕taṁ

\begin{english}
  Mas uma vez que existe um Não-nascido, Não-originado, Incriado, Não-formado,
\end{english}

Ta꜕smā jātass꜕a bhūtassa ka꜕tassa sa꜓ṅkha꜕tassa nissaraṇaṁ paññāyati

\begin{english}
  Assim é possível a libertação do mundo do nascido, originado, criado, formado.
\end{english}

\chapter[Quatro Requisitos]{Reflexão sobre os Quatro Requisitos}

\firstline{Paṭisaṅkhā yoniso}

\begin{leader}
  [Ha꜓nda mayaṁ taṅkhaṇika-paccave꜕kkhaṇa-pāṭhaṁ bhaṇāmase]
\end{leader}

[Paṭisaṅkhā] yoniso cīva꜕raṁ pa꜕ṭise꜓vāmi, yāvadeva sī꜓tassa꜕\\
pa꜕ṭighātāya, uṇhassa pa꜕ṭighātāya, ḍaṁsa-maka꜕sa꜕-vātāta꜕pa꜕-siriṁsapa-\\
-samphassānaṁ pa꜕ṭighātāya, yāvadeva hiri꜓kopina-pa꜕ṭicchāda꜕natthaṁ

\begin{english}
  Reflectindo sabiamente eu uso o manto: Somente por modéstia, para evitar o
  calor, o frio, as moscas, mosquitos, bichos rastejantes, o vento e as coisas
  que queimam.
\end{english}

[Paṭisaṅkhā] yoniso piṇḍa꜕pātaṁ pa꜕ṭise꜓vāmi, neva da꜕vāya, na ma꜕dāya, na maṇḍa꜕nāya, na꜕ vi꜓bhūsa꜕nāya, yāvadeva i꜓massa꜕ kāyassa꜕ ṭhi꜕tiyā, yāpa꜕nāya, vihiṁsū꜕para꜓ti꜕yā, brahmaca꜕ri꜓yānugga꜕hāya, iti purāṇañca꜕ veda꜓naṁ pa꜕ṭiha꜓ṅkhāmi, navañca꜕ veda꜓naṁ na uppādessāmi, yātrā ca꜕ me bhavissati a꜕navajjatā ca꜕ phāsuvihāro cā'ti

\begin{english}
  Reflectindo sabiamente eu uso a comida da mendicância: Não por diversão, não por
  prazer, não para engordar, não para me embelezar, mas somente para suster e
  nutrir este corpo, para o manter saudável, para ajudar à Vida Santa. Pensando
  desta forma: `Saciarei a fome sem comer demasiado, de forma
  a~continuar a viver sereno e sem remorsos.'
\end{english}

[Paṭisaṅkhā] yoniso senāsa꜕naṁ pa꜕ṭise꜓vāmi, yāvadeva sī꜓tassa꜕\\
pa꜕ṭighātāya, uṇhassa pa꜕ṭighātāya, ḍaṁsa-maka꜕sa꜕-vātāta꜕pa꜕-siriṁsapa-\\
-samphassānaṁ pa꜕ṭighātāya, yāvadeva utupa꜕rissaya vi꜕nodanaṁ pa꜕ṭisa꜓llānārāmatthaṁ

\begin{english}
  Reflectindo sabiamente eu uso o alojamento: Somente para evitar o frio, o calor,
  as moscas, mosquitos, bichos rastejantes, o vento e as coisas que
  queimam. Somente para me abrigar dos perigos da natureza e viver em
  recolhimento.
\end{english}

[Paṭisaṅkhā] yoniso gi꜕lāna-pacca꜕ya꜕-bhesajja-pa꜕rikkhāraṁ pa꜕ṭise꜓vāmi, yāvadeva uppa꜓nnānaṁ veyyābādhi꜕kānaṁ veda꜕nānaṁ pa꜕ṭighātāya, a꜕byāpajjha-pa꜕ramatāyā'ti

\begin{english}
  Reflectindo sabiamente eu uso o apoio necessário para medicamentos e
  enfermidades: Somente para aliviar as dores que tenham surgido, de forma a
  ficar o mais possível livre de doenças.
\end{english}

\chapter[Dez Temas]{Dez Temas para Recordar Frequentemente por Aqueles que Seguem o Caminho}

\firstline{Dasa ime bhikkhave}

\enlargethispage{\baselineskip}

\begin{leader}
  [Ha꜓nda mayaṁ pabbajita\hyp{}abhiṇha\hyp{}paccave꜕kkhaṇa\hyp{}pāṭhaṁ bhaṇāmase]
\end{leader}

[Dasa i꜕me bhikkhave] dhammā pabba꜕jitena a꜕bhiṇhaṁ pacca꜕vekkhi꜓tabbā, ka꜕ta꜕me dasa

\begin{english}
  Monges, existem dez dhammas acerca dos quais se deve reflectir frequentemente. \prul{Quais} são estes dez dhammas?
\end{english}

Vevaṇṇi꜕yamhi ajjhūpa꜕ga꜕to'ti pabba꜕jitena a꜕bhiṇhaṁ pacca꜕vekkhi꜓tabbaṁ

\begin{english}
  `Já não vivo segundo os valores e objectivos do mundo.'\\
  Quem perfaz o caminho\\
  deve reflectir sobre isto frequentemente.
\end{english}

Parapaṭi꜕baddhā me jīvi꜓kā'ti pabba꜕jitena a꜕bhiṇhaṁ pacca꜕vekkhi꜓tabbaṁ

\begin{english}
  `A minha própria vida é sustentada pela generosidade dos outros.'\\
  Quem perfaz o caminho\\
  deve reflectir sobre isto frequentemente.
\end{english}

Añño me ākappo ka꜕ra꜕ṇīyo'ti pabba꜕jitena a꜕bhiṇhaṁ pacca꜕vekkhi꜓tabbaṁ

\begin{english}
  `Devo esforçar-me por abandonar os meus hábitos antigos.'\\
  Quem perfaz o caminho\\
  deve reflectir sobre isto frequentemente.
\end{english}

\clearpage

Kacci nu꜕ kho me attā sīla꜕to na u꜕pavadatī'ti pabba꜕jitena a꜕bhiṇhaṁ pacca꜕vekkhi꜓tabbaṁ

\begin{english}
  `Surgem remorsos na minha mente em relação à minha conduta?'\\
  Quem perfaz o caminho\\
  deve reflectir sobre isto frequentemente.
\end{english}

Kacci nu꜕ kho maṁ a꜕nuvicca viññū sabrahma꜓cārī sīla꜕to na u꜕pavadantī'ti pabba꜕jitena a꜕bhiṇhaṁ pacca꜕vekkhi꜓tabbaṁ

\begin{english}
  `Será que os meus companheiros espirituais acham falhas na minha conduta?'\\
  Quem perfaz o caminho\\
  deve reflectir sobre isto frequentemente.
\end{english}

Sa꜕bbehi me pi꜕yehi ma꜕nāpehi꜕ nānābhāvo vi꜕nābhāvo'ti pabba꜕jitena abhiṇhaṁ pacca꜕vekkhi꜓tabbaṁ

\begin{english}
  `Tudo aquilo que é meu, que amo e prezo, tornar-se-á diferente, separar-se-á de mim.'\\
  Quem perfaz o caminho\\
  deve reflectir sobre isto frequentemente.
\end{english}

Kammassa꜕komhi kamma꜓dāyādo kamma꜕yoni kamma꜕bandhu kammapa꜕ṭisa꜓raṇo, yaṁ kammaṁ ka꜕rissāmi, kalyāṇaṁ vā pāpa꜕kaṁ vā, tassa꜕ dāyādo bha꜕vissāmī'ti pabba꜕jitena a꜕bhiṇhaṁ pacca꜕vekkhi꜓tabbaṁ

\enlargethispage{2\baselineskip}

\begin{english}
  `Sou o dono do meu Kamma, herdeiro do meu Kamma,\\
  nascido do meu Kamma, ligado ao meu Kamma,\\
  permaneço suportado pelo meu Kamma; seja qual Kamma eu criar,\\
  Para o bem ou para o mal, \prul{disso} serei o herdeiro.'\\
  Quem perfaz o caminho\\
  deve reflectir sobre isto frequentemente.
\end{english}

\clearpage

`Kathambhūtassa꜕ me rattindi꜕vā vīti꜕pa꜓tantī'ti pabba꜕jitena a꜕bhiṇhaṁ pacca꜕vekkhi꜓tabbaṁ

\begin{english}
  `Os dias e as noites passam continuamente; Como estou eu a usar\\ o meu tempo?'\\
 Quem perfaz o caminho\\
 deve reflectir sobre isto frequentemente.
\end{english}

Kacci nu꜕ kho'haṁ suññā꜓gāre abhira꜕māmī'ti pabba꜕jitena a꜕bhiṇhaṁ pacca꜕vekkhi꜓tabbaṁ

\begin{english}
  `Aprecio a solidão ou não?'\\
  Quem perfaz o caminho\\
  deve reflectir sobre isto frequentemente.
\end{english}

Atthi nu꜕ kho me uttari-ma꜕nussa-dhammā alamariya꜕-ñāṇa-dassana-viseso adhiga꜕to, so'haṁ pacchi꜓me kāle sa꜕brahmacārīhi꜕ puṭṭho na maṅku bha꜕vissāmī'ti pabba꜕jitena a꜕bhiṇhaṁ pacca꜕vekkhi꜓tabbaṁ

\begin{english}
  `Deu a minha prática frutos de compreensão e liberdade, de forma a que no fim da minha vida eu não me sinta envergonhado quando questionado pelos meus companheiros espirituais?'\\
  Quem perfaz o caminho\\
  deve reflectir sobre isto frequentemente.
\end{english}

Ime kho bhikkha꜓ve da꜕sa꜕ dhammā pabba꜕jitena a꜕bhiṇhaṁ pacca꜕vekkhitabbā'ti

\begin{english}
  Monges estes são dez Dhammas sobre os quais se deve reflectir frequentemente.
\end{english}

\cleartoverso

\chapter[Trinta-e-duas-Partes]{Reflexão sobre as Trinta-e-duas-Partes}

\firstline{Ayaṁ kho me kāyo}

\begin{leader}
  [Ha꜓nda mayaṁ dvattiṁsākāra-pāṭhaṁ bhaṇāmase]
\end{leader}

[Ayaṁ kho] me kāyo uddhaṁ pāda꜕ta꜕lā adho kesamatthakā\\
ta꜕ca꜕pa꜕ri꜕yanto pūro nānappa꜕kārassa꜕ a꜕su꜕ci꜕no

\begin{english}
  Isto, que é o meu corpo, das plantas dos pés para cima, e do topo da cabeça para baixo, é um saco de pele fechado cheio de coisas repugnantes.
\end{english}

Atthi imasmiṁ kāye

\begin{english}
  Neste corpo existem:
\end{english}

{\centering
\setArrayStrech{1}

\begin{tabular}{ r l }
kesā            & \tr{cabelo} \\
lomā            & \tr{pelos} \\
nakhā           & \tr{unhas} \\
dantā           & \tr{dentes} \\
taco            & \tr{pele} \\
maṁsaṁ          & \tr{carne} \\
nahārū          & \tr{tendões} \\
aṭṭhī           & \tr{ossos} \\
aṭṭhimiñjaṁ     & \tr{medula óssea} \\
vakkaṁ          & \tr{rins} \\
hadayaṁ         & \tr{coração} \\
yakanaṁ         & \tr{fígado} \\
kilomakaṁ       & \tr{membranas} \\
pihakaṁ         & \tr{baço} \\
papphāsaṁ       & \tr{pulmões} \\
\end{tabular}

\clearpage

\begin{tabular}{ r l }
antaṁ           & \tr{intestinos} \\
antaguṇaṁ       & \tr{tripas} \\
udariyaṁ        & \tr{comida não digerida} \\
karīsaṁ         & \tr{excremento} \\
pittaṁ          & \tr{bílis} \\
semhaṁ          & \tr{muco} \\
pubbo           & \tr{pus} \\% TODO: is this translated?
lohitaṁ         & \tr{sangue} \\
sedo            & \tr{suor} \\
medo            & \tr{gordura} \\
assu            & \tr{lágrimas} \\
vasā            & \tr{sebo} \\
kheḷo           & \tr{saliva} \\
siṅghāṇikā      & \tr{mucosidade} \\
lasikā          & \tr{lubrificante das articulações} \\
muttaṁ          & \tr{urina} \\
matthaluṅgan'ti & \tr{miolos} \\
\end{tabular}

\restoreArrayStretch
}

Evam-ayaṁ me kāyo uddhaṁ pāda꜕ta꜕lā adho kesamatthakā\\
ta꜕ca꜕pa꜕ri꜕yanto pūro nānappa꜕kārassa꜕ a꜕su꜕ci꜕no

\begin{english}
  Assim, isto que é o meu corpo, das plantas dos pés para cima, e do topo da cabeça para baixo, é um saco de pele fechado cheio de coisas repugnantes.
\end{english}

\chapter{The Teaching on Mindfulness of Breathing}

\begin{leader}
  [Ha꜓nda mayam ānāpānass꜕ati-sutta-pāṭhaṁ bha꜕ṇāmase]
\end{leader}

Ānāpāna꜓ssa꜕ti bhi꜓kkha꜕ve bhāvi꜓tā bahu꜕līka꜕tā\\
Mahappha꜕lā ho꜓ti mahā꜓nisa꜓ṁsā\\
Ānāpāna꜓ssa꜕ti bhi꜓kkha꜕ve bhāvi꜓tā bahu꜕līka꜕tā\\
Ca꜕ttāro sati꜓pa꜕ṭṭhāne pa꜕ri꜓pū꜕reti\\
Ca꜕ttāro sa꜕tipa꜕ṭṭhānā bhāvi꜓tā bahu꜕līka꜕tā\\
Sa꜕tta-bojjhaṅge pa꜕ri꜓pū꜕renti\\
Sa꜕tta-bojjhaṅgā bhāvi꜓tā bahu꜕līka꜕tā\\
Vijjā-vimuttiṁ pa꜕ri꜓pū꜕renti\\
Kathaṁ bhāvi꜓tā ca bhi꜓kkha꜕ve ānāpāna꜓ss꜕ati ka꜕thaṁ bahu꜕līka꜕tā\\
Mahappha꜕lā ho꜓ti mahā꜓nisa꜓ṁsā\\
Idha bhi꜓kkha꜕ve bhikkhu\\
Arañña꜓-ga꜕to vā\\
Rukkha-mūla꜓-ga꜕to vā\\
Suññāgāra꜓-ga꜕to vā\\
N꜕isīdati pallaṅkaṁ ābhuji꜓tv꜕ā\\
Ujuṁ kāyaṁ pa꜕ṇidhāya pa꜕rimukhaṁ sa꜕tiṁ u꜕paṭṭha꜕petvā\\
So sa꜕to'va a꜕ssasa꜕ti sa꜕to'va pa꜕ssa꜕sa꜕ti\\
Dīghaṁ vā assa꜕sa꜓nto dīghaṁ a꜕ssasā꜓mī'ti pa꜕jānāti\\
Dīghaṁ vā pa꜕ssa꜕santo dīghaṁ pa꜕ssasā꜓mī'ti pa꜕jānāti\\
Rassaṁ vā a꜕ssa꜕santo rassaṁ a꜕ssasā꜓mī'ti pa꜕jānāti\\
Rassaṁ vā pa꜕ssa꜕santo rassaṁ pa꜕ssasā꜓mī'ti pa꜕jānāti\\
Sabba꜕-kāya-paṭ꜕isa꜓ṁvedī a꜕ssasi꜕ssāmī'ti si꜕kkh꜕ati\\
Sabba꜕-kāya-paṭ꜕isa꜓ṁvedī pa꜕ssasi꜕ssāmī'ti si꜕kkh꜕ati

\clearpage

\enlargethispage{2\baselineskip}

Passa꜕mbhayaṁ kāya꜕-sa꜓ṅkhāraṁ a꜕ssasi꜕ssāmī'ti si꜕kkh꜕ati\\
Passa꜕mbhayaṁ kāya꜕-sa꜓ṅkhāraṁ pa꜕ssasi꜕ssāmī'ti si꜕kkh꜕ati\\
Pīti꜕-paṭi꜕sa꜓ṁvedī a꜕ssasi꜕ssāmī'ti si꜕kkh꜕ati\\
Pīti꜕-paṭi꜕sa꜓ṁvedī pa꜕ssasi꜕ssāmī'ti si꜕kkh꜕ati\\
Sukh꜕a-paṭi꜕sa꜓ṁvedī a꜕ssasi꜕ssāmī'ti si꜕kkh꜕ati\\
Sukh꜕a-paṭi꜕sa꜓ṁvedī pa꜕ssasi꜕ssāmī'ti si꜕kkh꜕ati\\
Citta꜕-sa꜓ṅkhāra-paṭi꜕sa꜓ṁvedī a꜕ssasi꜕ssāmī'ti si꜕kkh꜕ati\\
Citta꜕-sa꜓ṅkhāra-paṭi꜕sa꜓ṁvedī pa꜕ssasi꜕ssāmī'ti si꜕kkh꜕ati\\
Passa꜕mbhayaṁ citta꜕-sa꜓ṅkhāraṁ a꜕ssasi꜕ssāmī'ti si꜕kkh꜕ati\\
Passa꜕mbhayaṁ citt꜕a-sa꜓ṅkhāraṁ pa꜕ssasi꜕ssāmī'ti si꜕kkh꜕ati\\
Citta꜕-paṭi꜕sa꜓ṁvedī a꜕ssasi꜕ssāmī'ti si꜕kkh꜕ati\\
Citta꜕-paṭi꜕sa꜓ṁvedī pa꜕ssasi꜕ssāmī'ti si꜕kkh꜕ati\\
A꜕bhippa꜕moda꜓yaṁ cittaṁ a꜕ssasi꜕ssāmī'ti si꜕kkh꜕ati\\
A꜕bhippa꜕moda꜓yaṁ cittaṁ pa꜕ssasi꜕ssāmī'ti si꜕kkh꜕ati\\
Sa꜕māda꜓haṁ cittaṁ a꜕ssasi꜕ssāmī'ti si꜕kkh꜕ati\\
Sa꜕māda꜓haṁ cittaṁ pa꜕ssasi꜕ssāmī'ti si꜕kkh꜕ati\\
Vimoca꜓yaṁ cittaṁ a꜕ssasi꜕ssāmī'ti si꜕kkh꜕ati\\
Vimoca꜓yaṁ cittaṁ pa꜕ssasi꜕ssāmī'ti si꜕kkh꜕ati\\
Aniccānupa꜕ssī a꜕ssasi꜕ssāmī'ti si꜕kkh꜕ati\\
Aniccānupa꜕ssī pa꜕ssasi꜕ssāmī'ti si꜕kkh꜕ati\\
Virāgānupa꜕ssī a꜕ssasi꜕ssāmī'ti si꜕kkh꜕ati\\
Virāgānupa꜕ssī pa꜕ssasi꜕ssāmī'ti si꜕kkh꜕ati\\
Nirodhānupa꜕ssī a꜕ssasi꜕ssāmī'ti si꜕kkh꜕ati\\
Nirodhānupa꜕ssī pa꜕ssasi꜕ssāmī'ti si꜕kkh꜕ati\\
Pa꜕ṭiniss꜕aggānupa꜕ssī a꜕ssasi꜕ssāmī'ti si꜕kkh꜕ati\\
Pa꜕ṭinissa꜕ggānupa꜕ssī pa꜕ssasi꜕ssāmī'ti si꜕kkh꜕ati\\
Evaṁ bhāvi꜓tā kho bhi꜓kkha꜕ve ānāpāna꜓ss꜕ati evaṁ bahu꜕līka꜕tā\\
Mahappha꜕lā ho꜓ti mahā꜓nisa꜓ṁsā'ti

\cleartoverso

\chapter*{Sāriputta Sutta}

\begin{leader}
  [Ha꜓nda mayaṁ sāriputta-sutta-gāthā꜓yo bha꜕ṇāmase]
\end{leader}

\enlargethispage{\baselineskip}

\englishText

``Never before\\
have I seen or heard\\
from anyone\\
of a teacher with such lovely speech\\
come, together with his following\\
from Tusita heaven,\\
as the One with Eyes\\
who appears to the world with its devas\\
having dispelled all darkness\\
having arrived at delight\\
\vin all alone.

To that One Awakened —\\
\vin unentangled, Such, undeceptive,\\
\vin come with his following —\\
I have come with a question\\
on behalf of the many\\
here who are fettered.\\
For a monk disaffected,\\
frequenting a place that's remote —\\
\vin the root of a tree, a cemetery, in mountain caves\\
\vin various places to stay —\\
how many are the fears there\\
at which he shouldn't tremble\\
\vin — there in his noiseless abode —\\
how many the dangers in the world\\
for the monk going the direction\\
\vin \vin he never has gone\\
that he should transcend\\
there in his isolated abode?

\chapter{Sāriputta Sutta}

\enlargethispage{\baselineskip}

\paliText

\begin{leader}
  [Ha꜓nda mayaṁ sāriputta-sutta-gāthā꜓yo bha꜕ṇāmase]
\end{leader}

\verseref{1}%
\emph{na me diṭṭho ito pubbe\\
na suto uda kassaci\\
evaṁ vagguvado satthā\\
tusitā gaṇimāgato}

\verseref{2}%
\emph{sadevakassa lokassa\\
yathā dissati cakkhumā}\\
\emph{sabbaṁ tamaṁ vinodetvā\\
ekova ratimajjhagā}

\verseref{3}%
\emph{taṁ buddhaṁ asitaṁ tādiṁ\\
akuhaṁ gaṇimāgataṁ}\\
\emph{bahūnamidha baddhānaṁ\\
atthi pañhena āgamaṁ}

\verseref{4}%
\emph{bhikkhuno vijigucchato\\
bhajato rittamāsanaṁ}\\
\emph{rukkhamūlaṁ susānaṁ vā\\
pabbatānaṁ guhāsu vā}

\verseref{5}%
\emph{uccāvacesu sayanesu\\
kīvanto tattha bheravā}\\
\emph{yehi bhikkhu na vedheyya\\
nigghose sayanāsane}

\verseref{6}%
\emph{katī parissayā loke\\
gacchato agataṁ disaṁ}\\
\emph{ye bhikkhu abhisambhave\\
pantamhi sayanāsane}

\clearpage

\englishText

What should be the ways of his speech?\\
What should be his range there of action?\\
What should be a resolute monk's\\
\vin precepts \& practices?\\
Undertaking what training\\
\vin — alone, astute, \& mindful —\\
would he blow away\\
his own impurities\\
as a silver smith,\\
\vin those in molten silver?"

\sidepar{The Buddha:}%
``I will tell you as one who knows,\\
what is comfort for one disaffected\\
resorting to a remote place,\\
desiring self-awakening\\
in line with the Dhamma.\\
An enlightened monk,\\
\vin living circumscribed, mindful,\\
shouldn't fear the five fears:\\
of horseflies, mosquitoes, snakes,\\
human contact, four-footed beings;\\
shouldn't be disturbed\\
by those following another's teaching\\
even on seeing their manifold terrors;\\
should overcome still other\\
further dangers\\
as he seeks what is skillful.

Touched by the touch of discomforts, hunger,\\
he should endure cold \& inordinate heat.\\
He with no home,\\
in many ways touched by these things,\\
striving, should make firm his persistence.

\clearpage

\paliText

\verseref{7}%
\emph{kyāssa byappathayo assu\\
kyāssassu idha gocarā}\\
\emph{kāni sīlabbatānāssu\\
pahitattassa bhikkhuno}

\verseref{8}%
\emph{kaṁ so sikkhaṁ samādāya\\
ekodi nipako sato}\\
\emph{kammāro rajatasseva\\
niddhame malamattano}

\verseref{9}%
vijigucchamānassa yadidaṁ phāsu\\
rittāsanaṁ sayanaṁ sevato ce\\
sambodhikāmassa yathānudhammaṁ\\
taṁ te pavakkhāmi yathā pajānaṁ

\verseref{10}%
pañcannaṁ dhīro bhayānaṁ na bhāye\\
bhikkhu sato sapariyantacārī\\
ḍaṁsādhipātānaṁ sarīsapānaṁ\\
manussaphassānaṁ catuppadānaṁ

\verseref{11}%
paradhammikānampi na santaseyya\\
disvāpi tesaṁ bahubheravāni\\
athāparāni abhisambhaveyya\\
parissayāni kusalānuesī

\verseref{12}%
ātaṅkaphassena khudāya phuṭṭho\\
sītaṁ athuṇhaṁ adhivāsayeyya\\
so tehi phuṭṭho bahudhā anoko\\
vīriyaṁ parakkammadaḷhaṁ kareyya

\clearpage

\englishText

He shouldn't commit a theft, shouldn't speak a lie,\\
should touch with thoughts of good will\\
\vin beings firm \& infirm.\\
Conscious of when his mind is stirred up \& turbid,\\
he should dispel it:\\
\vin `It's on the Dark One's side.'

He shouldn't come under the sway of anger or pride.\\
Having dug up their root he would stand firm.\\
Then, when prevailing — yes —\\
he'd prevail over his sense of dear \& undear.\\
Yearning for discernment\\
enraptured with what's admirable,\\
he should overcome these dangers,\\
should conquer discontent in his isolated spot,\\
should conquer these four thoughts of lament:

\vin `What will I eat, or where will I eat.\\
\vin How badly I slept. Tonight where will I sleep?'

These lamenting thoughts he should subdue —\\
one under training, wandering without home.\\
Receiving food \& cloth at appropriate times,\\
he should have a sense of enough\\
for the sake of contentment.\\
Guarded in regard to these things\\
going restrained into a village,\\
even when harassed\\
he shouldn't say a harsh word.

With eyes downcast \& not footloose,\\
committed to jhana, he should be continually wakeful.\\
Strengthening equanimity, centered within,\\
he should cut off any penchant to conjecture or worry.

\clearpage

\paliText

\verseref{13}%
theyyaṁ na kāre na musā bhaṇeyya\\
mettāya phasse tasathāvarāni\\
yadāvilattaṁ manaso vijaññā\\
kaṇhassa pakkhoti vinodayeyya

\verseref{14}%
kodhātimānassa vasaṁ na gacche\\
mūlampi tesaṁ palikhañña tiṭṭhe\\
athappiyaṁ vā pana appiyaṁ vā\\
addhābhavanto abhisambhaveyya

\verseref{15}%
paññaṁ purakkhatvā kalyāṇapīti\\
vikkhambhaye tāni parissayāni\\
aratiṁ sahetha sayanamhi pante\\
caturo sahetha paridevadhamme

\verseref{16}%
kiṁsū asissāmi kuva vā asissaṁ\\
dukkhaṁ vata settha kvajja sessaṁ\\
ete vitakke paridevaneyye\\
vinayetha sekho aniketacārī

\verseref{17}%
annañca laddhā vasanañca kāle\\
mattaṁ so jaññā idha tosanatthaṁ\\
so tesu gutto yatacāri gāme\\
rusitopi vācaṁ pharusaṁ na vajjā

\verseref{18}%
okkhittacakkhu na ca pādalolo\\
jhānānuyutto bahujāgarassa\\
upekkhamārabbha samāhitatto\\
takkāsayaṁ kukkucciyūpachinde

\clearpage

\englishText

When reprimanded,\\
he should — mindful — rejoice;\\
should smash any stubbornness\\
toward his fellows in the holy life;\\
should utter skillful words\\
that are not untimely;\\
should give no mind\\
to the gossip people might say.

And then there are in the world\\
the five kinds of dust\\
for whose dispelling, mindful\\
he should train:\\
with regard to forms, sounds, tastes,\\
smells, \& tactile sensations\\
\vin he should conquer passion;\\
with regard to these things\\
\vin he should subdue his desire.

A monk, mindful,\\
his mind well-released,\\
contemplating the right Dhamma\\
at the right times,\\
\vin on coming to oneness\\
\vin should annihilate darkness,"

\vin \vin \vin the Blessed One said.\footnote{%
Sutta Nipāta, Aṭṭhaka Vagga, Chapter 16. ``Sāriputta Sutta: To Sāriputta'' (Sn 4.16), translated from the Pali by
Thanissaro Bhikkhu. Access to Insight (Legacy Edition), 30 November 2013.
}

\clearpage

\paliText

\verseref{19}%
cudito vacībhi satimābhinande\\
sabrahmacārīsu khilaṁ pabhinde\\
vācaṁ pamuñce kusalaṁ nātivelaṁ\\
janavādadhammāya na cetayeyya

\verseref{20}%
athāparaṁ pañca rajāni loke\\
yesaṁ satīmā vinayāya sikkhe\\
rūpesu saddesu atho rasesu\\
gandhesu phassesu sahetha rāgaṁ

\verseref{21}%
etesu dhammesu vineyya chandaṁ\\
bhikkhu satimā suvimuttacitto\\
kālena so sammā dhammaṁ parivīmaṁsamāno\\
ekodibhūto vihane tamaṁ soti

\end{document}
