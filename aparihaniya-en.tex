\documentclass[
  babelLanguage=british,
  final,
  toneMarksAbove,
  %showtrims,
  %showwirebinding,
  webversion,
]{chantingbook-a4}

\usepackage{local}

\makeatletter
% Ratio of font size / line height: 1.68
\def\BOOK@fontSizePt{12.5}
\def\BOOK@lineHeightPt{21}

\renewcommand{\normalsize}{%
  \@setfontsize\normalsize\BOOK@fontSizePt\BOOK@lineHeightPt
  \abovedisplayskip 12\p@ \@plus3\p@ \@minus7\p@
  \abovedisplayshortskip \z@ \@plus3\p@
  \belowdisplayshortskip 6.5\p@ \@plus3.5\p@ \@minus3\p@
  \belowdisplayskip \abovedisplayskip
  \color{textbody}
  \let\@listi\@listI}
\normalsize
\makeatother

\title{Aparihāniya-dhamma-sutta}

\begin{document}

\mainmatter

\chapter[Aparihāniya-dhamma-sutta]{Bhikkhu-aparihāniya-dhamma-sutta}

\emph{Seven conditions for non-decline of Bhikkhus, AN 7.23}

\begin{leader}
  [Handa mayaṁ bhikkhu-aparihāniya-dhamma-suttaṁ bhaṇāmase]
\end{leader}

[Evaṁ me sutaṁ.] Ekaṁ samayaṁ bhagavā rājagahe꜔꜒ viharati gijjhakūṭe pabbate.
Tatra kho꜔꜒ bhagavā bhikkhū꜔꜒ āmantesi: Satta vo, bhikkhave, aparihā꜔꜒niye dhamme
desessā꜔꜒mi. Taṁ suṇātha, sā꜔꜒dhukaṁ manasi karotha, bhāsissā꜔꜒mī'ti. Evaṁ, bhante'ti
kho꜔꜒ te bhikkhū꜔꜒ bhagavato paccasso꜔꜒su꜔꜒ṁ. Bhagavā etadavoca:

\begin{english}
	I have heard that on one occasion the Blessed One was staying in Rājagaha, on Vultures Peak Mountain. There he addressed the monks: 'Monks, I will teach you the seven conditions that lead to non-decline. Listen and pay close attention. I will speak.'
	'Yes, lord' the monks responded. The Blessed One said: 
 
\end{english}

[1] Katame ca, bhikkhave, satta aparihā꜔꜒niyā dhammā? Yāvakīvañca, bhikkhave, bhikkhū꜔꜒
abhiṇha꜔꜒ṁ sa꜔꜒nnipātā bhavissa꜔꜒nti sa꜔꜒nnipātabahulā; vuddhiyeva, bhikkhave,
bhikkhū꜔꜒naṁ pāṭikaṅkhā꜔꜒, no parihā꜔꜒ni.

\begin{english}
	And which seven are the conditions that lead to non-decline?
	As long as the monks meet often, meet a great deal, their growth can be expected, not their decline.

\end{english}

[2] Yāvakīvañca, bhikkhave, bhikkhū꜔꜒ samaggā sa꜔꜒nnipatissa꜔꜒nti, samaggā
vuṭṭhahissa꜔꜒nti, samaggā sa꜔꜒ṅghakaraṇīyāni karissa꜔꜒nti; vuddhiyeva, bhikkhave,
bhikkhū꜔꜒naṁ pāṭikaṅkhā꜔꜒, no parihā꜔꜒ni.

\begin{english}
  As long as the monks meet in harmony, adjourn from their meetings in harmony, and conduct Sangha business in harmony, their growth can be expected, not their decline.
\end{english}

[3] Yāvakīvañca, bhikkhave, bhikkhū꜔꜒ apaññattaṁ na paññāpessa꜔꜒nti, paññattaṁ na
samucchi꜔꜒ndissa꜔꜒nti, yathā꜔꜒paññattesu sikkhā꜔꜒padesu samādāya vattissa꜔꜒nti;
vuddhiyeva, bhikkhave, bhikkhū꜔꜒naṁ pāṭikaṅkhā꜔꜒, no parihā꜔꜒ni.

\begin{english}
  As long as the monks neither decree what has not been decreed nor repeal what has been decreed, but undertake and follow the training rules as they have been decreed, their growth can be expected, not their decline.
 
\end{english}



\break

[4] Yāvakīvañca, bhikkhave, bhikkhū꜔꜒ ye te bhikkhū꜔꜒ the꜔꜒rā rattaññū cirapabbajitā
sa꜔꜒ṅghapitaro sa꜔꜒ṅghapariṇāyakā te sakkarissa꜔꜒nti garuṁ karissa꜔꜒nti mānessa꜔꜒nti
pūjessa꜔꜒nti, tesa꜔꜒ñca so꜔꜒tabbaṁ maññissa꜔꜒nti; vuddhiyeva, bhikkhave, bhikkhū꜔꜒naṁ
pāṭikaṅkhā꜔꜒, no parihā꜔꜒ni.

\begin{english}
  As long as the monks honour, respect, venerate, and pay homage to the elder monks — those with seniority who have long been ordained, the fathers of the Sangha, leaders of the Sangha — regarding them as worth listening to, their growth can be expected, not their decline.
\end{english}

\enlargethispage{2\baselineskip}

[5] Yāvakīvañca, bhikkhave, bhikkhū꜔꜒ uppannāya taṇhā꜔꜒ya ponobhavikāya na vasa꜔꜒ṁ
gacchissa꜔꜒nti; vuddhiyeva, bhikkhave, bhikkhū꜔꜒naṁ pāṭikaṅkhā꜔꜒, no parihā꜔꜒ni.

\begin{english}
  As long as the monks do not submit to the power of any arisen craving that leads to further becoming, their growth can be expected, not their decline.
\end{english}

[6] Yāvakīvañca, bhikkhave, bhikkhū꜔꜒ āraññakesu se꜔꜒nāsanesu sā꜔꜒pekkhā꜔꜒ bhavissa꜔꜒nti;
vuddhiyeva, bhikkhave, bhikkhū꜔꜒naṁ pāṭikaṅkhā꜔꜒, no parihā꜔꜒ni.

\begin{english}
  As long as the monks see their own benefit in wilderness dwellings, their growth can be expected, not their decline.
\end{english}

[7] Yāvakīvañca, bhikkhave, bhikkhū꜔꜒ paccattaññeva satiṁ upaṭṭhā꜔꜒pessa꜔꜒nti: Kinti
anāgatā ca pesalā sabrahmacārī āgacche꜔꜒yyuṁ, āgatā ca pesalā sabrahmacārī phā꜔꜒su꜔꜒ṁ
vihareyyun'ti; vuddhiyeva, bhikkhave, bhikkhū꜔꜒naṁ pāṭikaṅkhā꜔꜒, no parihā꜔꜒ni.

\begin{english}
 As long as the monks each keep firmly in mind: 'If there are any well-behaved spiritual companions who have yet to come, may they come; and may the well-behaved spiritual companions who have come live in comfort,' their growth can be expected, not their decline.
\end{english}

Yāvakīvañca, bhikkhave, ime satta aparihā꜔꜒niyā dhammā bhikkhū꜔꜒su ṭhassa꜔꜒nti, imesu
ca sattasu aparihā꜔꜒niyesu dhammesu bhikkhū꜔꜒ sa꜔꜒ndississa꜔꜒nti; vuddhiyeva, bhikkhave,
bhikkhū꜔꜒naṁ pāṭikaṅkhā꜔꜒, no parihā꜔꜒nī'ti. Idam-avoca bhagavā. Attamanā te bhikkhū꜔꜒
bhagavato bhāsitaṁ abhinandun'ti.

\begin{english}
	'As long as the monks remain steadfast in these seven conditions, and as long as these seven conditions endure among the monks, the monks' growth can be expected, not their decline.'
	That is what the Blessed One said. Gratified, the monks delighted in the Blessed One's words.
\end{english}

\end{document}
