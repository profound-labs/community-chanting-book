\documentclass[
  babelLanguage=british,
  final,
  toneMarksAbove,
  %showtrims,
  %showwirebinding,
  webversion,
]{chantingbook-a4}

\usepackage{local}

\makeatletter
% Ratio of font size / line height: 1.68
\def\BOOK@fontSizePt{12.5}
\def\BOOK@lineHeightPt{21}

\renewcommand{\normalsize}{%
  \@setfontsize\normalsize\BOOK@fontSizePt\BOOK@lineHeightPt
  \abovedisplayskip 12\p@ \@plus3\p@ \@minus7\p@
  \abovedisplayshortskip \z@ \@plus3\p@
  \belowdisplayshortskip 6.5\p@ \@plus3.5\p@ \@minus3\p@
  \belowdisplayskip \abovedisplayskip
  \color{textbody}
  \let\@listi\@listI}
\normalsize
\makeatother

\title{Aparihāniya-dhamma-sutta}

\begin{document}

\mainmatter

\chapter[Aparihāniya-dhamma-sutta]{Bhikkhu-aparihāniya-dhamma-sutta}

\emph{Sete condições para a bem-aventurança dos bhikkhus, AN 7.23}

\begin{leader}
  [Handa mayaṁ bhikkhu-aparihāniya-dhamma-suttaṁ bhaṇāmase]
\end{leader}

[Evaṁ me sutaṁ.] Ekaṁ samayaṁ bhagavā rājagahe꜔꜒ viharati gijjhakūṭe pabbate.
Tatra kho꜔꜒ bhagavā bhikkhū꜔꜒ āmantesi: Satta vo, bhikkhave, aparihā꜔꜒niye dhamme
desessā꜔꜒mi. Taṁ suṇātha, sā꜔꜒dhukaṁ manasi karotha, bhāsissā꜔꜒mī'ti. Evaṁ, bhante'ti
kho꜔꜒ te bhikkhū꜔꜒ bhagavato paccasso꜔꜒su꜔꜒ṁ. Bhagavā etadavoca:

\begin{english}
  Eu ouvi que em certa ocasião o Excelso estava em Rajagaha, no Pico dos
  Abutres. Ali, ele dirigiu-se aos monges: `Monges, eu irei ensinar-vos as sete
  condições que não levam ao declínio. Ouçam e prestem muita atenção. Eu vou
  falar.' `Sim, senhor', responderam os monges. O Excelso disse:
\end{english}

[1] Katame ca, bhikkhave, satta aparihā꜔꜒niyā dhammā? Yāvakīvañca, bhikkhave, bhikkhū꜔꜒
abhiṇha꜔꜒ṁ sa꜔꜒nnipātā bhavissa꜔꜒nti sa꜔꜒nnipātabahulā; vuddhiyeva, bhikkhave,
bhikkhū꜔꜒naṁ pāṭikaṅkhā꜔꜒, no parihā꜔꜒ni.

\begin{english}
  `E quais são as sete condições que não levam ao declínio? Desde que os monges
  se reúnam com frequência, se reúnam assiduamente, o seu crescimento pode ser
  esperado, não o seu declínio.'

\end{english}

[2] Yāvakīvañca, bhikkhave, bhikkhū꜔꜒ samaggā sa꜔꜒nnipatissa꜔꜒nti, samaggā
vuṭṭhahissa꜔꜒nti, samaggā sa꜔꜒ṅghakaraṇīyāni karissa꜔꜒nti; vuddhiyeva, bhikkhave,
bhikkhū꜔꜒naṁ pāṭikaṅkhā꜔꜒, no parihā꜔꜒ni.

\begin{english}
  `Enquanto os monges se reunirem em harmonia, terminarem as suas reuniões em
  harmonia e conduzirem os assuntos da Saṅgha em harmonia, o seu crescimento
  pode ser esperado, não o seu declínio.'
\end{english}

[3] Yāvakīvañca, bhikkhave, bhikkhū꜔꜒ apaññattaṁ na paññāpessa꜔꜒nti, paññattaṁ na
samucchi꜔꜒ndissa꜔꜒nti, yathā꜔꜒paññattesu sikkhā꜔꜒padesu samādāya vattissa꜔꜒nti;
vuddhiyeva, bhikkhave, bhikkhū꜔꜒naṁ pāṭikaṅkhā꜔꜒, no parihā꜔꜒ni.

\begin{english}
  `Enquanto os monges nem decretarem o que não foi decretado, nem revogarem o que
  foi decretado, mas praticarem o cumprimento das regras de treino conforme
  foram decretadas, o seu crescimento pode ser esperado, não o seu declínio.'
\end{english}

[4] Yāvakīvañca, bhikkhave, bhikkhū꜔꜒ ye te bhikkhū꜔꜒ the꜔꜒rā rattaññū cirapabbajitā
sa꜔꜒ṅghapitaro sa꜔꜒ṅghapariṇāyakā te sakkarissa꜔꜒nti garuṁ karissa꜔꜒nti mānessa꜔꜒nti
pūjessa꜔꜒nti, tesa꜔꜒ñca so꜔꜒tabbaṁ maññissa꜔꜒nti; vuddhiyeva, bhikkhave, bhikkhū꜔꜒naṁ
pāṭikaṅkhā꜔꜒, no parihā꜔꜒ni.

\begin{english}
  `Enquanto os monges honrarem, respeitarem, venerarem e prestarem homenagem aos
  monges mais velhos -- aqueles com antiguidade que foram ordenados há muito
  tempo, os pais do Sangha, os líderes do Sangha -- considerando muito valioso
  ouvi-los, o seu crescimento pode ser esperado, não o seu declínio.'
\end{english}

\enlargethispage{2\baselineskip}

[5] Yāvakīvañca, bhikkhave, bhikkhū꜔꜒ uppannāya taṇhā꜔꜒ya ponobhavikāya na vasa꜔꜒ṁ
gacchissa꜔꜒nti; vuddhiyeva, bhikkhave, bhikkhū꜔꜒naṁ pāṭikaṅkhā꜔꜒, no parihā꜔꜒ni.

\begin{english}
  `Enquanto os monges não se submeterem ao poder de qualquer desejo que surja e
  que leve a um futuro nascimento, o seu crescimento pode ser esperado, não o
  seu declínio.'
\end{english}

[6] Yāvakīvañca, bhikkhave, bhikkhū꜔꜒ āraññakesu se꜔꜒nāsanesu sā꜔꜒pekkhā꜔꜒ bhavissa꜔꜒nti;
vuddhiyeva, bhikkhave, bhikkhū꜔꜒naṁ pāṭikaṅkhā꜔꜒, no parihā꜔꜒ni.

\begin{english}
  `Enquanto os monges regozijarem em viver nas florestas,
  o seu crescimento pode ser esperado, não o seu declínio.'
\end{english}

[7] Yāvakīvañca, bhikkhave, bhikkhū꜔꜒ paccattaññeva satiṁ upaṭṭhā꜔꜒pessa꜔꜒nti: Kinti
anāgatā ca pesalā sabrahmacārī āgacche꜔꜒yyuṁ, āgatā ca pesalā sabrahmacārī phā꜔꜒su꜔꜒ṁ
vihareyyun'ti; vuddhiyeva, bhikkhave, bhikkhū꜔꜒naṁ pāṭikaṅkhā꜔꜒, no parihā꜔꜒ni.

\begin{english}
  `Enquanto cada um dos monges mantiver firmemente em mente: ``Se houver
  companheiros bem-comportados, seguidores da vida casta que ainda estão por
  vir, que possam eles vir; e que os companheiros bem-comportados da vida casta
  que vierem possam viver à vontade'', pode-se esperar o seu crescimento, não o
  seu declínio.'
\end{english}

Yāvakīvañca, bhikkhave, ime satta aparihā꜔꜒niyā dhammā bhikkhū꜔꜒su ṭhassa꜔꜒nti, imesu
ca sattasu aparihā꜔꜒niyesu dhammesu bhikkhū꜔꜒ sa꜔꜒ndississa꜔꜒nti; vuddhiyeva, bhikkhave,
bhikkhū꜔꜒naṁ pāṭikaṅkhā꜔꜒, no parihā꜔꜒nī'ti. Idam-avoca bhagavā. Attamanā te bhikkhū꜔꜒
bhagavato bhāsitaṁ abhinandun'ti.

\begin{english}
  `Enquanto os monges permanecerem resolutos nestas sete condições, e enquanto
  estas sete condições persistirem entre os monges, pode-se esperar o seu
  crescimento, não o seu declínio.' Isto é o que o Excelso disse. Satisfeitos,
  os monges deleitaram-se com as palavras do Excelso.
\end{english}

\end{document}
