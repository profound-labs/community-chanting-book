\documentclass[
  babelLanguage=portuguese,
  final,
  %showtrims,
  %showwirebinding,
]{chantingbook}

\usepackage{local}

\title{Livro de Cânticos}
\subtitle{}

\begin{document}

\frontmatter

\thispagestyle{empty}

{\centering
\mbox{}
\vfill

\begin{minipage}{0.7\linewidth}

\parttitlefont\color{chaptertitle}
%\addfontfeature{LetterSpace=2.0}

Cânticos

\end{minipage}

\vspace*{4\baselineskip}

\vfill

\mbox{}
}

\mainmatter

\artopttrue

\clearpage
\chapter{Añjali}

Os cânticos e os pedidos formais são feitos com as mãos em añjali.
Este é um gesto de respeito, executado pondo as palmas das mãos juntas
directamente à frente do peito, com os dedos alinhados a apontar
para cima.

\chapter{Pedindo uma Palestra de Dhamma}

\begin{instruction}
  Depois de fazer a vénia três vezes, com as mãos unidas em añjali, recitar o seguinte:
\end{instruction}

Brahmā ca꜕ lokādhipa꜕tī sa꜕hampa꜕ti\\
Ka꜕tañja꜕lī a꜕nadhiva꜕raṁ a꜕yāca꜕tha

Santī꜓dha sa꜕ttāppa꜕ra꜕jakkha꜕-jātikā\\
Desetu꜕ dhammaṁ a꜕nu꜕kampi꜕maṁ pa꜕jaṁ

\begin{instruction}
  Fazer as três vénias outra vez.
\end{instruction}

\begin{english}
O deus Brahmā Sahampati, Senhor do mundo,\\
Com as palmas das mãos juntas em reverência, pediu um favor:

`Há seres aqui com apenas um pouco de pó nos olhos,\\
Por favor, por compaixão, ensina-lhes o Dhamma.'
\end{english}

\chapter{Reconhecendo o Ensinamento}

\enlargethispage{2\baselineskip}

\begin{tabular}{@{} ll @{}}
Uma pessoa: & Ha꜓nda mayaṁ dhammakathā꜓ya sā꜓dhukā꜕raṁ dadāmase \\
& \hspace*{1em}\tr{Expressemos agora  nossa aprovação}\\
& \hspace*{1em}\tr{deste Ensinamento do Dhamma.}\\
Resposta: & Sādhu, sādhu, sādhu, anu꜓modāmi \\
& \hspace*{1em}\tr{É bom, eu o valorizo.} \\
\end{tabular}

\clearpage
\chapter[Três Refúgios \& Oito Preceitos]{Pedido dos Três Refúgios\newline \& Oito Preceitos}

\begin{instruction}
  Após fazer três vénias, com as palmas\\
  das mão unidas em añjali, recita-se o pedido:
\end{instruction}

\subsection{Em grupo}

Mayaṁ bhante tisaraṇena sa꜕ha aṭṭha sī꜓lāni yā꜕cāma\\
Dutiyampi mayaṁ bhante tisaraṇena sa꜕ha aṭṭha sī꜓lāni yā꜕cāma\\
Tatiyampi mayaṁ bhante tisaraṇena sa꜕ha aṭṭha sī꜓lāni yā꜕cāma

\subsection{Individualmente}

Ahaṁ bhante tisaraṇena sa꜕ha aṭṭha sī꜓lāni yā꜕cāmi\\
Dutiyampi ahaṁ bhante tisaraṇena sa꜕ha aṭṭha sī꜓lāni yā꜕cāmi\\
Tatiyampi ahaṁ bhante tisaraṇena sa꜕ha aṭṭha sī꜓lāni yā꜕cāmi

\subsection{Tradução}

\begin{english}
  Pedimos/Peço, Venerável Mestre,\\
  \vin os Três Refúgios e os Oito Preceitos.

  Pela segunda vez, pedimos/peço, Venerável Mestre,\\
  \vin os Três Refúgios e os Oito Preceitos.

  Pela terceira vez, pedimos/peço, Venerável Mestre,\\
  \vin os Três Refúgios e os Oito Preceitos.
\end{english}

\clearpage
\chapter{Os Três Refúgios}

\begin{instruction}
  Repetir, depois de o líder ter\\
  cantado as primeiras três linhas
\end{instruction}

Namo tassa bhagavato arahato sammāsambuddhassa\\
Namo tassa bhagavato arahato sammāsambuddhassa\\
Namo tassa bhagavato arahato sammāsambuddhassa

\begin{english}
  Homenagem ao Excelso, Nobre e Perfeitamente Iluminado.\\
  Homenagem ao Excelso, Nobre e Perfeitamente Iluminado.\\
  Homenagem ao Excelso, Nobre e Perfeitamente Iluminado.
\end{english}

Buddhaṁ saraṇaṁ gacchāmi\\
Dhammaṁ saraṇaṁ gacchāmi\\
Saṅghaṁ saraṇaṁ gacchāmi

\begin{english}
  Tomo refúgio no Buddha.\\
  Tomo refúgio no Dhamma.\\
  Tomo refúgio no Sangha.
\end{english}

Dutiyampi buddhaṁ saraṇaṁ gacchāmi\\
Dutiyampi dhammaṁ saraṇaṁ gacchāmi\\
Dutiyampi saṅghaṁ saraṇaṁ gacchāmi

\begin{english}
  Pela segunda vez, tomo refúgio no Buddha.\\
  Pela segunda vez, tomo refúgio no Dhamma.\\
  Pela segunda vez, tomo refúgio no Sangha.
\end{english}

Tatiyampi buddhaṁ saraṇaṁ gacchāmi\\
Tatiyampi dhammaṁ saraṇaṁ gacchāmi\\
Tatiyampi saṅghaṁ saraṇaṁ gacchāmi

\clearpage

\begin{english}
  Pela terceira vez, tomo refúgio no Buddha.\\
  Pela terceira vez, tomo refúgio no Dhamma.\\
  Pela terceira vez, tomo refúgio no Sangha.
\end{english}

\begin{instruction}
  Líder:
\end{instruction}

[Tisaraṇa-gamanaṁ niṭṭhitaṁ]

\begin{english}
  Fica assim completo o Triplo Refúgio.
\end{english}

\begin{instruction}
  Resposta:
\end{instruction}

Āma bhante

\begin{english}
  Sim, Venerável Mestre.
\end{english}

\chapter{Os Oito Preceitos}

\begin{instruction}
  Repetir cada preceito depois do líder
\end{instruction}

\begin{precept}
  \setcounter{enumi}{0}
  \item Pāṇātipātā vera꜓maṇī sikkhā꜓padaṁ sa꜓mādi꜕yāmi
\end{precept}

\begin{english}
  Observo o preceito de me abster de matar qualquer criatura viva.
\end{english}

\begin{precept}
  \setcounter{enumi}{1}
  \item Adinnādānā vera꜓maṇī sikkhā꜓padaṁ sa꜓mādi꜕yāmi
\end{precept}

\begin{english}
  Observo o preceito de não tirar aquilo que não me for oferecido.
\end{english}

\begin{precept}
  \setcounter{enumi}{2}
  \item Abrahmacariyā vera꜓maṇī sikkhā꜓padaṁ sa꜓mādi꜕yāmi
\end{precept}

\begin{english}
  Observo o preceito de me abster de qualquer tipo de actividade sexual.
\end{english}

\begin{precept}
  \setcounter{enumi}{3}
  \item Musā꜓vādā vera꜓maṇī sikkhā꜓padaṁ sa꜓mādi꜕yāmi
\end{precept}

\begin{english}
  Observo o preceito de me abster de mentir.
\end{english}

\clearpage

\begin{precept}
  \setcounter{enumi}{4}
  \item Surāmeraya-majja-pamādaṭṭhā꜓nā vera꜓maṇī sikkhā꜓padaṁ sa꜓mādi꜕yāmi
\end{precept}

\begin{english}
  Observo o preceito de me abster de consumir bebidas\\
  e drogas intoxicantes que deturpem a mente.
\end{english}

\begin{precept}
  \setcounter{enumi}{5}
  \item Vikālabhojanā vera꜓maṇī sikkhā꜓padaṁ sa꜓mādi꜕yāmi.
\end{precept}

\begin{english}
  Observo o preceito de me abster de comer em alturas indevidas.
\end{english}

\begin{precept}
  \setcounter{enumi}{6}
  \item Nacca-gīta-vādita-visūkada꜓ssanā mālā-gandha-vilepana-dhāraṇa-maṇḍana-vibhūsanaṭṭhā꜓nā vera꜓maṇī sikkhā꜓padaṁ sa꜓mādi꜕yāmi.
\end{precept}

\begin{english}
  Observo o preceito de me abster de qualquer tipo de entretenimento, embelezamento e adornamento.
\end{english}

\begin{precept}
  \setcounter{enumi}{7}
  \item Uccāsayana-mahā꜓sayanā vera꜓maṇī sikkhā꜓padaṁ sa꜓mādi꜕yāmi.
\end{precept}

\begin{english}
  Observo o preceito de me abster de dormir em camas elevadas e luxuosas.
\end{english}

\begin{instruction}
  Líder:
\end{instruction}

[Imāni aṭṭha sikkhā꜓padāni\\
Sī꜓lena suga꜕tiṁ yanti\\
Sī꜓lena bhoga꜕sa꜓mpadā\\
Sī꜓lena nibbu꜕tiṁ yanti\\
Tasmā꜓ sī꜓laṁ viso꜓dhaye]

\clearpage

\begin{english}
  Estes são os Oito Preceitos;\\
  A virtude é fonte de felicidade,\\
  A virtude é fonte de verdadeira riqueza,\\
  A virtude é fonte de paz ---\\
  Que a virtude seja assim purificada.
\end{english}

\begin{instruction}
  Resposta:
\end{instruction}

Sādhu, sādhu, sādhu.

\begin{instruction}
  Fazer três vénias
\end{instruction}

\clearpage
\chapter[Três Refúgios \& Cinco Preceitos]{Pedido dos Três Refúgios\newline \& Cinco Preceitos}

\begin{instruction}
  Após fazer três vénias, com as palmas\\
  das mão unidas em añjali, recita-se o pedido:
\end{instruction}

\subsection{Em grupo}

Mayaṁ bhante tisaraṇena sa꜕ha pañca sī꜓lāni yā꜕cāma\\
Dutiyampi mayaṁ bhante tisaraṇena sa꜕ha pañca sī꜓lāni yā꜕cāma\\
Tatiyampi mayaṁ bhante tisaraṇena sa꜕ha pañca sī꜓lāni yā꜕cāma

\subsection{Individualmente}

Ahaṁ bhante tisaraṇena sa꜕ha pañca sī꜓lāni yā꜕cāmi\\
Dutiyampi ahaṁ bhante tisaraṇena sa꜕ha pañca sī꜓lāni yā꜕cāmi\\
Tatiyampi ahaṁ bhante tisaraṇena sa꜕ha pañca sī꜓lāni yā꜕cāmi

\subsection{Tradução}

\begin{english}
  Pedimos/Peço, Venerável Mestre,\\
  \vin os Três Refúgios e os Cinco Preceitos.

  Pela segunda vez, pedimos/peço, Venerável Mestre,\\
  \vin os Três Refúgios e os Cinco Preceitos.

  Pela terceira vez, pedimos/peço, Venerável Mestre,\\
  \vin os Três Refúgios e os Cinco Preceitos.
\end{english}

\clearpage
\chapter{Os Três Refúgios}

\begin{instruction}
  Repetir, depois de o líder ter\\
  cantado as primeiras três linhas
\end{instruction}

Namo tassa bhagavato arahato sammāsambuddhassa\\
Namo tassa bhagavato arahato sammāsambuddhassa\\
Namo tassa bhagavato arahato sammāsambuddhassa

\begin{english}
  Homenagem ao Excelso, Nobre e Perfeitamente Iluminado.\\
  Homenagem ao Excelso, Nobre e Perfeitamente Iluminado.\\
  Homenagem ao Excelso, Nobre e Perfeitamente Iluminado.
\end{english}

Buddhaṁ saraṇaṁ gacchāmi\\
Dhammaṁ saraṇaṁ gacchāmi\\
Saṅghaṁ saraṇaṁ gacchāmi

\begin{english}
  Tomo refúgio no Buddha.\\
  Tomo refúgio no Dhamma.\\
  Tomo refúgio no Sangha.
\end{english}

Dutiyampi buddhaṁ saraṇaṁ gacchāmi\\
Dutiyampi dhammaṁ saraṇaṁ gacchāmi\\
Dutiyampi saṅghaṁ saraṇaṁ gacchāmi

\begin{english}
  Pela segunda vez, tomo refúgio no Buddha.\\
  Pela segunda vez, tomo refúgio no Dhamma.\\
  Pela segunda vez, tomo refúgio no Sangha.
\end{english}

Tatiyampi buddhaṁ saraṇaṁ gacchāmi\\
Tatiyampi dhammaṁ saraṇaṁ gacchāmi\\
Tatiyampi saṅghaṁ saraṇaṁ gacchāmi

\clearpage

\begin{english}
  Pela terceira vez, tomo refúgio no Buddha.\\
  Pela terceira vez, tomo refúgio no Dhamma.\\
  Pela terceira vez, tomo refúgio no Sangha.
\end{english}

\begin{instruction}
  Líder:
\end{instruction}

[Tisaraṇa-gamanaṁ niṭṭhitaṁ]

\begin{english}
  Fica assim completo o Triplo Refúgio.
\end{english}

\begin{instruction}
  Resposta:
\end{instruction}

Āma bhante

\begin{english}
  Sim, Venerável Mestre.
\end{english}

\chapter{Os Cinco Preceitos}

\begin{instruction}
  Repetir cada preceito depois do líder
\end{instruction}

\begin{precept}
  \setcounter{enumi}{0}
  \item Pāṇātipātā vera꜓maṇī sikkhā꜓padaṁ sa꜓mādi꜕yāmi
\end{precept}

\begin{english}
  Observo o preceito de me abster de matar qualquer criatura viva.
\end{english}

\begin{precept}
  \setcounter{enumi}{1}
  \item Adinnādānā vera꜓maṇī sikkhā꜓padaṁ sa꜓mādi꜕yāmi
\end{precept}

\begin{english}
  Observo o preceito de não tirar aquilo que não me for oferecido.
\end{english}

\begin{precept}
  \setcounter{enumi}{2}
  \item Kāmesu micchā꜓cārā vera꜓maṇī sikkhā꜓padaṁ sa꜓mādi꜕yāmi
\end{precept}

\begin{english}
  Observo o preceito de me abster de ter uma conduta sexual imprórpia.
\end{english}

\begin{precept}
  \setcounter{enumi}{3}
  \item Musā꜓vādā vera꜓maṇī sikkhā꜓padaṁ sa꜓mādi꜕yāmi
\end{precept}

\enlargethispage{\baselineskip}

\begin{english}
  Observo o preceito de me abster de mentir.
\end{english}

\clearpage

\begin{precept}
  \setcounter{enumi}{4}
  \item Surāmeraya-majja-pamādaṭṭhā꜓nā vera꜓maṇī sikkhā꜓padaṁ sa꜓mādi꜕yāmi
\end{precept}

\begin{english}
  Observo o preceito de me abster de consumir bebidas\\
  e drogas intoxicantes que deturpem a mente.
\end{english}

\begin{instruction}
  Líder:
\end{instruction}

[Imāni pañca sikkhā꜓padāni\\
Sī꜓lena suga꜕tiṁ yanti\\
Sī꜓lena bhoga꜕sa꜓mpadā\\
Sī꜓lena nibbu꜕tiṁ yanti\\
Tasmā꜓ sī꜓laṁ viso꜓dhaye]

\begin{english}
  Estes são os Cinco Preceitos;\\
  A virtude é fonte de felicidade,\\
  A virtude é fonte de verdadeira riqueza,\\
  A virtude é fonte de paz ---\\
  Que a virtude seja assim purificada.
\end{english}

\begin{instruction}
  Resposta:
\end{instruction}

Sādhu, sādhu, sādhu

\begin{instruction}
  Fazer três vénias
\end{instruction}


\artoptfalse

\chapter{The Teaching on Mindfulness of Breathing}

\enlargethispage{2\baselineskip}

%\firstline{Ānāpānassati bhikkhave bhāvitā bahulī-katā}

\begin{leader}
  [Ha꜓nda mayam ānāpānass꜕ati-sutta-pāṭhaṁ bha꜕ṇāmase]
\end{leader}

Ānāpāna꜓ssa꜕ti bhi꜓kkha꜕ve bhāvi꜓tā bahu꜕līka꜕tā

\begin{english}
  Bhikkhus, wh꜕en mindfulness of bre꜓athing is de꜕veloped and cu꜕ltiva꜓ted
\end{english}

Mahappha꜕lā ho꜓ti mahā꜓nisa꜓ṁsā

\begin{english}
  It is of gre꜕at fruit and great be꜕nefit;
\end{english}

Ānāpāna꜓ssa꜕ti bhi꜓kkha꜕ve bhāvi꜓tā bahu꜕līka꜕tā

\begin{english}
  Wh꜕en mindfulness of bre꜓athing is de꜕veloped and cu꜕ltiva꜓ted
\end{english}

Ca꜕ttāro sati꜓pa꜕ṭṭhāne pa꜕ri꜓pū꜕reti

\begin{english}
  It fu꜕lfills the Four Foundations of Mi꜕ndfu꜕lness;
\end{english}

Ca꜕ttāro sa꜕tipa꜕ṭṭhānā bhāvi꜓tā bahu꜕līka꜕tā

\begin{english}
  When th꜕e Four Foundations of Mi꜓ndfulness are de꜕veloped and cu꜕ltiva꜓ted
\end{english}

Sa꜕tta-bojjhaṅge pa꜕ri꜓pū꜕renti

\begin{english}
  They fu꜕lfill the Seven Factors of Awa꜕kening;
\end{english}

Sa꜕tta-bojjhaṅgā bhāvi꜓tā bahu꜕līka꜕tā

\begin{english}
  When th꜕e Seven Factors of Awa꜓kening are de꜕veloped and cu꜕ltiva꜓ted
\end{english}

Vijjā-vimuttiṁ pa꜕ri꜓pū꜕renti

\begin{english}
  They fu꜕lfill true knowledge and deli꜕verance.
\end{english}

Kathaṁ bhāvi꜓tā ca bhi꜓kkha꜕ve ānāpāna꜓ss꜕ati ka꜕thaṁ bahu꜕līka꜕tā

\begin{english}
  An꜕d how, bhikkhus, is mindfulness of bre꜓athing de꜕veloped and cu꜕ltiva꜓ted
\end{english}

Mahappha꜕lā ho꜓ti mahā꜓nisa꜓ṁsā

\begin{english}
  So that it is of gre꜕at fruit and great be꜕nefit?
\end{english}

Idha bhi꜓kkha꜕ve bhikkhu

\begin{english}
  Here, bhikkhus, a bhi꜕kkhu,
\end{english}

Arañña꜓-ga꜕to vā

\begin{english}
  Gone to꜕ the fo꜓rest,
\end{english}

Rukkha-mūla꜓-ga꜕to vā

\begin{english}
  To the fo꜕ot o꜕f a꜕ tree
\end{english}

Suññāgāra꜓-ga꜕to vā

\begin{english}
  Or to an em꜓pty꜕ hut.
\end{english}

N꜕isīdati pallaṅkaṁ ābhuji꜓tv꜕ā

\begin{english}
  Si꜕ts down having cro꜕ssed hi꜕s legs,
\end{english}

Ujuṁ kāyaṁ pa꜕ṇidhāya pa꜕rimukhaṁ sa꜕tiṁ u꜕paṭṭha꜕petvā

\begin{english}
  Sets his bo꜕dy꜕ e꜕rect, having established mi꜓ndfulness in fro꜕nt o꜕f him.
\end{english}

So sa꜕to'va a꜕ssasa꜕ti sa꜕to'va pa꜕ssa꜕sa꜕ti

\begin{english}
  Ever mi꜓ndful he bre꜕athes in; mindful h꜕e bre꜕athes out.
\end{english}

Dīghaṁ vā assa꜕sa꜓nto dīghaṁ a꜕ssasā꜓mī'ti pa꜕jānāti

\begin{english}
  Breathing i꜓n long, he꜕ knows `I bre꜕athe i꜕n long';
\end{english}

Dīghaṁ vā pa꜕ssa꜕santo dīghaṁ pa꜕ssasā꜓mī'ti pa꜕jānāti

\begin{english}
  Breathing ou꜕t long, he꜕ knows `I bre꜕athe ou꜕t long';
\end{english}

Rassaṁ vā a꜕ssa꜕santo rassaṁ a꜕ssasā꜓mī'ti pa꜕jānāti

\begin{english}
  Breathing i꜓n short, h꜕e knows `I bre꜕athe i꜕n short';
\end{english}

Rassaṁ vā pa꜕ssa꜕santo rassaṁ pa꜕ssasā꜓mī'ti pa꜕jānāti

\begin{english}
  Breathing ou꜕t short, h꜕e knows `I bre꜕athe ou꜕t short'.
\end{english}

Sabba꜕-kāya-paṭ꜕isa꜓ṁvedī a꜕ssasi꜕ssāmī'ti si꜕kkh꜕ati

\begin{english}
  He tra꜕ins thus: `I shall breathe i꜓n experiencing the whole bo꜕dy'.
\end{english}

Sabba꜕-kāya-paṭ꜕isa꜓ṁvedī pa꜕ssasi꜕ssāmī'ti si꜕kkh꜕ati

\begin{english}
  He tra꜕ins thus: `I shall breathe ou꜕t e꜕xpe꜕ri꜕enci꜕ng th꜕e who꜕le bo꜕dy'.
\end{english}

Passa꜕mbhayaṁ kāya꜕-sa꜓ṅkhāraṁ a꜕ssasi꜕ssāmī'ti si꜕kkh꜕ati

\begin{english}
  He tra꜕ins thus: `I shall breathe i꜓n tranquillizing the bodily forma꜕tions'.
\end{english}

Passa꜕mbhayaṁ kāya꜕-sa꜓ṅkhāraṁ pa꜕ssasi꜕ssāmī'ti si꜕kkh꜕ati

\begin{english}
  He tra꜕ins thus: `I shall breathe ou꜕t tra꜕nqui꜕ll꜕izi꜕ng th꜕e bo꜕dily fo꜕rmations'.
\end{english}

Pīti꜕-paṭi꜕sa꜓ṁvedī a꜕ssasi꜕ssāmī'ti si꜕kkh꜕ati

\begin{english}
  He tra꜕ins thus: `I shall breathe i꜓n experiencing ra꜕pture'.
\end{english}

Pīti꜕-paṭi꜕sa꜓ṁvedī pa꜕ssasi꜕ssāmī'ti si꜕kkh꜕ati

\begin{english}
  He tra꜕ins thus: `I shall breathe ou꜕t e꜕xpe꜕ri꜕enci꜕ng ra꜕pture'.
\end{english}

Sukh꜕a-paṭi꜕sa꜓ṁvedī a꜕ssasi꜕ssāmī'ti si꜕kkh꜕ati

\begin{english}
  He tra꜕ins thus: `I shall breathe i꜓n experiencing ple꜕asure'
\end{english}

Sukh꜕a-paṭi꜕sa꜓ṁvedī pa꜕ssasi꜕ssāmī'ti si꜕kkh꜕ati

\begin{english}
  He tra꜕ins thus: `I shall breathe ou꜕t e꜕xpe꜕ri꜕enci꜕ng ple꜕asure'.
\end{english}

Citta꜕-sa꜓ṅkhāra-paṭi꜕sa꜓ṁvedī a꜕ssasi꜕ssāmī'ti si꜕kkh꜕ati

\begin{english}
  He tra꜕ins thus: `I shall breathe i꜓n experiencing the mental forma꜕tions'.
\end{english}

Citta꜕-sa꜓ṅkhāra-paṭi꜕sa꜓ṁvedī pa꜕ssasi꜕ssāmī'ti si꜕kkh꜕ati

\begin{english}
  He tra꜕ins thus: `I shall breathe ou꜕t e꜕xpe꜕ri꜕enci꜕ng th꜕e me꜕nta꜕l fo꜕rma꜕tions'.
\end{english}

Passa꜕mbhayaṁ citta꜕-sa꜓ṅkhāraṁ a꜕ssasi꜕ssāmī'ti si꜕kkh꜕ati

\begin{english}
  He tra꜕ins thus: `I shall breathe i꜓n tranquillizing the mental forma꜕tions'.
\end{english}

Passa꜕mbhayaṁ citt꜕a-sa꜓ṅkhāraṁ pa꜕ssasi꜕ssāmī'ti si꜕kkh꜕ati

\begin{english}
  He tra꜕ins thus: `I shall breathe ou꜕t tra꜕nqu꜕ill꜕izi꜕ng th꜕e me꜕nta꜕l fo꜕rma꜕tions'.
\end{english}

Citta꜕-paṭi꜕sa꜓ṁvedī a꜕ssasi꜕ssāmī'ti si꜕kkh꜕ati

\begin{english}
  He tra꜕ins thus: `I shall breathe i꜓n experiencing th꜕e mind'.
\end{english}

Citta꜕-paṭi꜕sa꜓ṁvedī pa꜕ssasi꜕ssāmī'ti si꜕kkh꜕ati

\begin{english}
  He tra꜕ins thus: `I shall breathe ou꜕t e꜕xpe꜕ri꜕enci꜕ng th꜕e mind'.
\end{english}

A꜕bhippa꜕moda꜓yaṁ cittaṁ a꜕ssasi꜕ssāmī'ti si꜕kkh꜕ati

\begin{english}
  He tra꜕ins thus: `I shall breathe i꜓n gladdening th꜕e mind'.
\end{english}

A꜕bhippa꜕moda꜓yaṁ cittaṁ pa꜕ssasi꜕ssāmī'ti si꜕kkh꜕ati

\begin{english}
  He tra꜕ins thus: `I shall breathe ou꜕t gl꜕adde꜕ni꜕ng th꜕e mind'.
\end{english}

Sa꜕māda꜓haṁ cittaṁ a꜕ssasi꜕ssāmī'ti si꜕kkh꜕ati

\begin{english}
  He tra꜕ins thus: `I shall breathe i꜓n concentrating th꜕e mind'
\end{english}

Sa꜕māda꜓haṁ cittaṁ pa꜕ssasi꜕ssāmī'ti si꜕kkh꜕ati

\begin{english}
  He tra꜕ins thus: `I shall breathe ou꜕t co꜕nce꜕ntr꜕ati꜕ng th꜕e mind'.
\end{english}

Vimoca꜓yaṁ cittaṁ a꜕ssasi꜕ssāmī'ti si꜕kkh꜕ati

\begin{english}
  He tra꜕ins thus: `I shall breathe i꜓n liberating th꜕e mind'.
\end{english}

Vimoca꜓yaṁ cittaṁ pa꜕ssasi꜕ssāmī'ti si꜕kkh꜕ati

\begin{english}
  He tra꜕ins thus: `I shall breathe ou꜕t li꜕be꜕ra꜕ti꜕ng th꜕e mind'.
\end{english}

Aniccānupa꜕ssī a꜕ssasi꜕ssāmī'ti si꜕kkh꜕ati

\begin{english}
  He tra꜕ins thus: `I shall breathe i꜓n contemplating impe꜕rmanence'.
\end{english}

Aniccānupa꜕ssī pa꜕ssasi꜕ssāmī'ti si꜕kkh꜕ati

\begin{english}
  He tra꜕ins thus: `I shall breathe ou꜕t co꜕nte꜕mpla꜕ti꜕ng i꜕mpe꜕rmanence'.
\end{english}

Virāgānupa꜕ssī a꜕ssasi꜕ssāmī'ti si꜕kkh꜕ati

\begin{english}
  He tra꜕ins thus: `I shall breathe i꜓n contemplating the fading away of pa꜕ssions'.
\end{english}

Virāgānupa꜕ssī pa꜕ssasi꜕ssāmī'ti si꜕kkh꜕ati

\begin{english}
  He tra꜕ins thus: `I shall breathe ou꜕t co꜕nte꜕mpl꜕ati꜕ng th꜕e fa꜕di꜕ng aw꜕ay o꜕f pa꜕ssions'.
\end{english}

Nirodhānupa꜕ssī a꜕ssasi꜕ssāmī'ti si꜕kkh꜕ati

\begin{english}
  He tra꜕ins thus: `I shall breathe i꜓n contemplating cessa꜕tion'.
\end{english}

Nirodhānupa꜕ssī pa꜕ssasi꜕ssāmī'ti si꜕kkh꜕ati

\begin{english}
  He tra꜕ins thus: `I shall breathe ou꜕t co꜕nte꜕mpl꜕ati꜕ng ce꜕ss꜕ation'.
\end{english}

Pa꜕ṭiniss꜕aggānupa꜕ssī a꜕ssasi꜕ssāmī'ti si꜕kkh꜕ati

\begin{english}
  He tra꜕ins thus: `I shall breathe i꜓n contemplating reli꜕nquishment'.
\end{english}

Pa꜕ṭinissa꜕ggānupa꜕ssī pa꜕ssasi꜕ssāmī'ti si꜕kkh꜕ati

\begin{english}
  He tra꜕ins thus: `I shall breathe ou꜕t co꜕nte꜕mpl꜕ati꜕ng re꜕li꜕nquishment'.
\end{english}

Evaṁ bhāvi꜓tā kho bhi꜓kkha꜕ve ānāpāna꜓ss꜕ati evaṁ bahu꜕līka꜕tā

\begin{english}
  Bhikkhus, that is ho꜕w mindfulness of bre꜓athing is de꜕veloped and cu꜕ltiva꜓ted
\end{english}

Mahappha꜕lā ho꜓ti mahā꜓nisa꜓ṁsā'ti

\begin{english}
  So that it is of gr꜕eat fruit and great be꜕nefit.
\end{english}

\chapter[Partilha de Bençãos]{Reflexões sobre a Partilha de Bençãos}

\enlargethispage{2\baselineskip}

\begin{leader}
  [Cantemos agora as Reflexões sobre a Partilha de Bençãos]
\end{leader}

\firstline{Através do bem que resulta da minha prática}

Através do bem que resulta da minha prática,\\
Que os meus mestres e guias espirituais de grande virtude,\\
A minha mãe, o meu pai e os meus familiares,\\
O Sol e a Lua, e todos os líderes virtuosos do mundo,\\
Que os Deuses mais elevados e as forças do mal,\\
Seres celestiais, espíritos guardiões da Terra e o Senhor da Morte,\\
Aqueles que são amigáveis, indiferentes ou hostis,\\
Que todos os seres recebam as bênçãos da minha vida.\\
Que brevemente cheguem à Tripla Bênção, e superem a morte.

Através do bem que resulta da minha prática,\\
E através desta partilha,\\
Que todos os desejos e apegos rapidamente cessem\\
Assim como os estados prejudiciais da mente.

Até realizar o Nibbana,\\
Em qualquer tipo de nascimento, que eu tenha uma mente justa,\\
Com consciência e sabedoria, austeridade e vigor.\\
Que as forças ilusórias não controlem,\\
nem enfraqueçam a minha decisão.

O Buddha é o meu excelente refúgio,\\
Insuperável é a proteção do Dhamma,\\
O Buddha solitário é o meu Nobre exemplo,\\
O Saṅgha é o meu maior suporte.

Que através desta supremacia\\
Desapareçam a escuridão e a ilusão.

%\clearpage
%\thispagestyle{empty}
%\mbox{}

\end{document}


