\documentclass[
  babelLanguage=english,
  final,
  webversion,
  %showtrims,
]{chantingbook}

\usepackage{local}

\makeatletter

\newcommand{\verseref}[1]{\sidepar{#1}}

\definecolor{titlecolor}{gray}{0.1}

\makeatother

\title{Sumedharama Monastic Chants}

\begin{document}

\frontmatter

% Cover page

\thispagestyle{empty}\mbox{}
\AddToShipoutPictureFG*{%
  \put(\LenToUnit{-3mm},\LenToUnit{-3mm}){%
    \begin{minipage}[b][\paperheight + 6mm][c]{\paperwidth + 6mm}%
      \centering%
      \includegraphics[width=\paperwidth+6mm]{inner-peace-crop}%
      \par%
      \vspace*{1cm}%
      \resizebox{70mm}{!}{\color{titlecolor}\Calluna\textls{CÂNTICOS}}%
      \par%
      \vspace*{2cm}%
    \end{minipage}%
  }%
}%
\clearpage

\cleartorecto
\tableofcontents*

\mainmatter

\usePsMarksTitleOnly

\chapter*[Partilha de Bençãos]{Reflexões sobre a Partilha de Bençãos}

\delegateSetUseNext

\begin{leader}
  [Ha꜓nda mayaṃ uddissanādhiṭṭhāna-gāthā꜓yo b꜕haṇāmase]
\end{leader}

\firstline{Iminā puññakammena upajjhāyā guṇuttarā}

[Iminā puñña꜕kammena] u꜕pajjhāyā gu꜕ṇutta꜕rā\\
Ācariyūpa꜕kārā ca꜕ mātāpitā ca꜕ ñāta꜕kā\\
Suriyo candimā rājā gu꜕ṇavantā na꜕rāpi꜕ ca꜕\\
Brahma-mārā ca꜕ indā ca꜕ loka꜕pālā ca꜕ deva꜕tā\\
Yamo mittā ma꜕nussā ca majjhattā veri꜕kāpi꜕ ca꜕\\
Sa꜕bbe sattā sukhī hontu puññāni pa꜕ka꜕tāni꜕ me\\
Sukhañca tividhaṃ dentu꜕ khippaṃ pāpetha꜕ voma꜕taṃ\\
Iminā puññakammena iminā uddi꜕ssena꜕ ca꜕\\
Khipp'āhaṃ su꜕la꜕bhe ceva taṇhūpādāna꜕-cheda꜕naṃ\\
Ye santāne hīnā dhammā yāva꜕ nibbāna꜕to ma꜕maṃ\\
Nassantu sabba꜕dā yeva yattha꜕ jāto bha꜕ve bha꜕ve\\
Ujucittaṃ sa꜕ti꜕paññā sallekho vi꜕ri꜕yamhinā\\
Mārā labhantu nokāsaṃ kātuñca vi꜕ri꜕yes꜕u me\\
Buddhādhipa꜕va꜕ro nātho dhammo nātho va꜕rutta꜕mo\\
Nātho pacceka꜕buddho ca꜕ saṅgho nāthotta꜕ro ma꜕maṃ\\
Tesottamānubhāvena mārokāsaṃ la꜕bhantu꜕ mā

\chapter[Partilha de Bençãos]{Reflexões sobre a Partilha de Bençãos}

\enlargethispage{2\baselineskip}

\begin{leader}
  [Cantemos agora as Reflexões sobre a Partilha de Bençãos]
\end{leader}

\firstline{Através do bem que resulta da minha prática}

Através do bem que resulta da minha prática,\\
Que os meus mestres e guias espirituais de grande virtude,\\
A minha mãe, o meu pai e os meus familiares,\\
O Sol e a Lua, e todos os líderes virtuosos do mundo,\\
Que os Deuses mais elevados e as forças do mal,\\
Seres celestiais, espíritos guardiões da Terra e o Senhor da Morte,\\
Aqueles que são amigáveis, indiferentes ou hostis,\\
Que todos os seres recebam as bênçãos da minha vida.\\
Que brevemente cheguem à Tripla Bênção, e superem a morte.

Através do bem que resulta da minha prática,\\
E através desta partilha,\\
Que todos os desejos e apegos rapidamente cessem\\
Assim como os estados prejudiciais da mente.

Até realizar o Nibbana,\\
Em qualquer tipo de nascimento, que eu tenha uma mente justa,\\
Com consciência e sabedoria, austeridade e vigor.\\
Que as forças ilusórias não controlem,\\
nem enfraqueçam a minha decisão.

O Buddha é o meu excelente refúgio,\\
Insuperável é a proteção do Dhamma,\\
O Buddha solitário é o meu Nobre exemplo,\\
O Saṅgha é o meu maior suporte.

Que através desta supremacia\\
Desapareçam a escuridão e a ilusão.

\chapter*[Metta Sutta]{Metta Sutta}

\delegateSetUseNext

\firstline{Karaṇīyam-attha-kusalena}

\begin{leader}
  [Ha꜓nda mayaṃ metta-sutta-gāthā꜓yo bha꜕ṇāmase]
\end{leader}

[Karaṇīyam-attha-kusalena]\\
Yan-taṃ santaṃ padaṃ abhisamecca\\
Sakko ujū ca suhujū ca\\
Suvaco c'assa mudu anatimānī

Santussako ca subharo ca\\
Appakicco ca sallahuka-vutti\\
Sant'indriyo ca nipako ca\\
Appagabbho kulesu ananugiddho

Na ca khuddaṃ samācare kiñci\\
Yena viññū pare upavadeyyuṃ\\
Sukhino vā khemino hontu\\
Sabbe sattā bhavantu sukhit'attā

Ye keci pāṇa-bhūt'atthi\\
Tasā vā thāvarā vā anavasesā\\
Dīghā vā ye mahantā vā\\
Majjhimā rassakā aṇuka-thūlā

Diṭṭhā vā ye ca adiṭṭhā\\
Ye ca dūre vasanti avidūre\\
Bhūtā vā sambhavesī vā\\
Sabbe sattā bhavantu sukhit'attā

\clearpage

Na paro paraṃ nikubbetha\\
Nātimaññetha katthaci naṃ kiñci\\
Byārosanā paṭighasaññā\\
Nāññam-aññassa dukkham-iccheyya

Mātā yathā niyaṃ puttaṃ\\
Āyusā eka-puttam-anurakkhe\\
Evam'pi sabba-bhūtesu\\
Mānasam-bhāvaye aparimāṇaṃ

Mettañca sabba-lokasmiṃ\\
Mānasam-bhāvaye aparimāṇaṃ\\
Uddhaṃ adho ca tiriyañca\\
Asambādhaṃ averaṃ asapattaṃ

Tiṭṭhañ-caraṃ nisinno vā\\
Sayāno vā yāvat'assa vigata-middho\\
Etaṃ satiṃ adhiṭṭheyya\\
Brahmam-etaṃ vihāraṃ idham-āhu

Diṭṭhiñca anupagamma\\
Sīlavā dassanena sampanno\\
Kāmesu vineyya gedhaṃ\\
Na hi jātu gabbha-seyyaṃ punaretī'ti

\chapter[Metta Sutta]{Metta Sutta}

\firstline{Eis o que se deve fazer}

\begin{leader}
  [Cantemos agora as palavras do Buddha\\ sobre o Amor e a Compaixão]
\end{leader}

Eis o que se deve fazer\\
Para cultivar a bondade\\
E seguir a via da paz:\\
Ser capaz e ser honesto,\\
Franco e gentil no falar.\\
Humilde e não arrogante,\\
Contente, facilmente satisfeito,\\
Aliviado de deveres e frugal no seu caminho.

Pacífico e sereno, sábio e inteligente,\\
Sem orgulho, sem exigência por natureza.\\
Que ele nada faça\\
Que os sábios possam vir a reprovar.\\
Desejando: Na alegria e na segurança,\\
Que todos os seres sejam felizes.\\
Quaisquer que sejam os seres vivos,\\
Fracos, fortes, sem excepção\\
Dos maiores aos mais pequenos,\\
Visíveis ou invisíveis,\\
Estejam perto ou estejam longe,\\
Nascidos ou por nascer ---\\
Que todos os seres sejam felizes!

\clearpage

Que ninguém engane ninguém,\\
Ou despreze alguém em que estado fôr.\\
Que ninguém por raiva ou má-fé,\\
Deseje mal a alguém.\\
Assim como uma Mãe protege o filho,\\
Com sua vida, seu único filho,\\
Assim de coração infinito,\\
Se deveria estimar todo o ser vivo;\\
Irradiando ternura por todo o mundo:\\
Acima ao mais alto céu,\\
E abaixo às profundezas;\\
Irradiante e sem limites,\\
Livre de ódio e má-fé.\\
Seja parado ou a andar,\\
Sentado ou deitado,\\
Livre de torpor,\\
Esta é uma lembrança a manter.

Diz-se esta ser a sublime permanência.\\
O puro de coração, com clareza de visão,\\
Ao não insistir em ideias fixas,\\
Liberto dos desejos dos sentidos,\\
Não voltará a nascer neste mundo.

\chapter*[Bem-Estar Universal]{Reflexão sobre o Bem-Estar Universal}

\delegateSetUseNext

\firstline{Ahaṃ sukhito homi}

\begin{leader}
[Ha꜓nda mayam mettāpharaṇaṃ ka꜕romase]
\end{leader}

[Aha꜓ṃ sukhito ho꜓mi]\\
Niddukkho ho꜓mi\\
A꜕vero ho꜓mi\\
A꜕byāpajjho ho꜓mi\\
A꜕nīgho ho꜓mi\\
Sukhī꜓ attānaṃ pa꜕riha꜓rāmi

Sa꜕bbe sa꜕ttā sukhitā ho꜓ntu\\
Sa꜕bbe sa꜕ttā averā ho꜓ntu\\
Sa꜕bbe sa꜕ttā abyāpajjhā ho꜓ntu\\
Sa꜕bbe sa꜕ttā anīghā ho꜓ntu\\
Sa꜕bbe sa꜕ttā sukhī꜓ a꜕ttānaṃ pa꜕riha꜓rantu

Sa꜕bbe sa꜕ttā sabbadukkhā pamucca꜓ntu

Sa꜕bbe sa꜕ttā laddha-sa꜓mpa꜕tti꜓to mā vigaccha꜓ntu

Sa꜕bbe sa꜕ttā kammassa꜕kā kamma꜓dāyādā kamma꜓yonī\\
\vin kamma꜓bandhū kammapa꜕ṭisa꜓ra꜕ṇā\\
Yaṃ kammaṃ ka꜕rissa꜓nti\\
Kalyāṇaṃ vā pāpa꜕kaṃ vā\\
Tassa꜕ dāyādā bha꜕vissa꜓nti

\chapter[Bem-Estar Universal]{Reflexão sobre o Bem-Estar Universal}

\firstline{Que eu mantenha bem-estar}

\begin{leader}
  [Cantemos agora as Reflexões sobre o Bem-estar Universal.]
\end{leader}

[Que eu mantenha bem-estar,]\\
Livre de aflição,\\
Livre de hostilidade,\\
Livre de má-fé,\\
Livre de ansiedade,\\
E possa eu \prul{manter} em mim bem-estar.

Que todos mantenham bem-estar,\\
Livres de hostilidade,\\
Livres de má-fé,\\
Livres de ansiedade, e possam eles\\
\prul{Manter} bem-estar em si próprios.

Possam \prul{todos} os seres se libertarem de todo o sofrimento.

E que todos não se separarem da \prul{boa fortuna} que alcançaram.

Quando agem com intenção,\\
\prul{Todos} os seres são os donos de sua acção e herdam seus resultados.\\
O seu futuro nasce de tal acção, companheiro de tal acção,\\
E os seus \prul{resultados} serão o seu lar.

\prul{Todas} as acções com intenção,\\
Sejam elas \prul{boas} ou más ---\\
De tais \prul{actos} eles serão os herdeiros.

\chapter[Incondicionado]{Reflexão sobre o Incondicionado}

\firstline{Atthi bhikkhave ajātaṃ abhūtaṃ akataṃ}

\begin{leader}
  [Ha꜓nda mayaṃ nibbāna-sutta-pāṭhaṃ bha꜕ṇāmase]
\end{leader}

Atthi bhi꜓kkha꜕ve a꜕jātaṃ a꜓bhūtaṃ a꜕kataṃ a꜕sa꜓ṅkh꜕ataṃ

\begin{english}
  Existe um Não-nascido, Não-originado, Incriado, Não-formado.
\end{english}

N꜕o cetaṃ bhi꜓kkha꜕ve a꜕bhavissa a꜕jātaṃ a꜓bhūtaṃ a꜕kataṃ a꜕sa꜓nkh꜕ataṃ

\begin{english}
 Se não existisse este Não-nascido, Não-originado, Incriado, Não-formado,
\end{english}

Na꜕ yidaṃ jātassa꜕ bhūtassa ka꜕tassa sa꜓ṅkh꜕atassa nissaraṇaṃ paññāye꜓tha

\begin{english}
  A libertação do mundo do nascido, originado, criado, formado, não seria possível.
\end{english}

Ya꜕smā ca kho bhi꜓kkh꜕ave atthi a꜕jātaṃ a꜓bhūtaṃ a꜕kataṃ a꜕sa꜓ṅkha꜕taṃ

\begin{english}
  Mas uma vez que existe um Não-nascido, Não-originado, Incriado, Não-formado,
\end{english}

Ta꜕smā jātass꜕a bhūtassa ka꜕tassa sa꜓ṅkha꜕tassa nissaraṇaṃ paññāyati

\begin{english}
  Assim é possível a libertação do mundo do nascido, originado, criado, formado.
\end{english}

\chapter[Quatro Requisitos]{Reflexão sobre os Quatro Requisitos}

\firstline{Paṭisaṅkhā yoniso}

\begin{leader}
  [Ha꜓nda mayaṃ taṅkhaṇika-paccave꜕kkhaṇa-pāṭhaṃ bhaṇāmase]
\end{leader}

[Paṭisaṅkhā] yoniso cīva꜕raṃ pa꜕ṭise꜓vāmi, yāvadeva sī꜓tassa꜕\\
pa꜕ṭighātāya, uṇhassa pa꜕ṭighātāya, ḍaṃsa-maka꜕sa꜕-vātāta꜕pa꜕-siriṃsapa-\\
-samphassānaṃ pa꜕ṭighātāya, yāvadeva hiri꜓kopina-pa꜕ṭicchāda꜕natthaṃ

\begin{english}
  Reflectindo sabiamente eu uso o manto: Somente por modéstia, para evitar o
  calor, o frio, as moscas, mosquitos, bichos rastejantes, o vento e as coisas
  que queimam.
\end{english}

[Paṭisaṅkhā] yoniso piṇḍa꜕pātaṃ pa꜕ṭise꜓vāmi, neva da꜕vāya, na ma꜕dāya, na maṇḍa꜕nāya, na꜕ vi꜓bhūsa꜕nāya, yāvadeva i꜓massa꜕ kāyassa꜕ ṭhi꜕tiyā, yāpa꜕nāya, vihiṃsū꜕para꜓ti꜕yā, brahmaca꜕ri꜓yānugga꜕hāya, iti purāṇañca꜕ veda꜓naṃ pa꜕ṭiha꜓ṅkhāmi, navañca꜕ veda꜓naṃ na uppādessāmi, yātrā ca꜕ me bhavissati a꜕navajjatā ca꜕ phāsuvihāro cā'ti

\begin{english}
  Reflectindo sabiamente eu uso a comida da mendicância: Não por diversão, não por
  prazer, não para engordar, não para me embelezar, mas somente para suster e
  nutrir este corpo, para o manter saudável, para ajudar à Vida Santa. Pensando
  desta forma: `Saciarei a fome sem comer demasiado, de forma
  a~continuar a viver sereno e sem remorsos.'
\end{english}

[Paṭisaṅkhā] yoniso senāsa꜕naṃ pa꜕ṭise꜓vāmi, yāvadeva sī꜓tassa꜕\\
pa꜕ṭighātāya, uṇhassa pa꜕ṭighātāya, ḍaṃsa-maka꜕sa꜕-vātāta꜕pa꜕-siriṃsapa-\\
-samphassānaṃ pa꜕ṭighātāya, yāvadeva utupa꜕rissaya vi꜕nodanaṃ pa꜕ṭisa꜓llānārāmatthaṃ

\begin{english}
  Reflectindo sabiamente eu uso o alojamento: Somente para evitar o frio, o calor,
  as moscas, mosquitos, bichos rastejantes, o vento e as coisas que
  queimam. Somente para me abrigar dos perigos da natureza e viver em
  recolhimento.
\end{english}

[Paṭisaṅkhā] yoniso gi꜕lāna-pacca꜕ya꜕-bhesajja-pa꜕rikkhāraṃ pa꜕ṭise꜓vāmi, yāvadeva uppa꜓nnānaṃ veyyābādhi꜕kānaṃ veda꜕nānaṃ pa꜕ṭighātāya, a꜕byāpajjha-pa꜕ramatāyā'ti

\begin{english}
  Reflectindo sabiamente eu uso o apoio necessário para medicamentos e
  enfermidades: Somente para aliviar as dores que tenham surgido, de forma a
  ficar o mais possível livre de doenças.
\end{english}

\chapter[Cinco Temas]{Cinco Temas para Recordar Frequentemente}

\firstline{Jarā-dhammomhi jaraṃ anatīto}

\begin{leader}
  [Ha꜓nda mayaṃ abhiṇha-paccave꜕kkhaṇa-pāṭhaṃ bhaṇāmase]
\end{leader}

\sidepar{Homens}%
[Jarā-dhammomhi꜕] jaraṃ a꜕na꜕tīto

\sidepar{Mulheres}%
[Jarā-dhammāmhi꜕] jaraṃ a꜕na꜕tītā

\begin{english}
  A minha natureza é envelhecer, ainda não fui além do envelhecimento.
\end{english}

\sidepar{h.}%
Byādhi꜓-dhammomhi꜕ byādhiṃ a꜕na꜕tīto

\sidepar{m.}%
Byādhi꜓-dhammāmhi꜕ byādhiṃ a꜕na꜕tītā

\begin{english}
  A minha natureza é adoecer, ainda não fui além da doença.
\end{english}

\sidepar{h.}%
Ma꜕raṇa-dhammomhi꜕ ma꜕raṇaṃ a꜕na꜕tīto

\sidepar{m.}%
Ma꜕raṇa-dhammāmhi꜕ ma꜕raṇaṃ a꜕na꜕tītā

\begin{english}
  A minha natureza é morrer, ainda não fui além da morte.
\end{english}

Sa꜕bbehi me pi꜕yehi ma꜕nāpehi꜕ nānābhāvo vi꜕nābhāvo

\begin{english}
  Tudo o que é meu, amado e agradável,\\
  ficará diferente, separar-se-á de mim.
\end{english}

\sidepar{h.}%
Kammassa꜕komhi kamma꜓dāyādo kamma꜕yoni kamma꜕bandhu kammapa꜕ṭisa꜓ra꜕ṇo\\
Yaṃ kammaṃ ka꜕rissāmi, kalyāṇaṃ vā pāpa꜕kaṃ vā, tassa꜕ dāyādo bha꜕vissāmi

\clearpage

\sidepar{m.}%
Kammassa꜕kāmhi kamma꜓dāyādā kamma꜕yoni kamma꜕bandhu kammapa꜕ṭisa꜓ra꜕ṇā\\
Yaṃ kammaṃ ka꜕rissāmi, kalyāṇaṃ vā pāpa꜕kaṃ vā, tassa꜕ dāyādā bha꜕vissāmi

\begin{english}
  Sou o dono do meu Kamma, herdeiro do meu Kamma,\\
  nascido do meu Kamma, ligado ao meu Kamma,\\
  permaneço suportado pelo meu Kamma; seja qual Kamma eu criar,\\
  Para o bem ou para o mal, \prul{disso} serei o herdeiro.
\end{english}

Evaṃ amhehi꜕ a꜕bhiṇhaṃ pacca꜕vekkhi꜓tabbaṃ

\begin{english}
  \prul{Assim} deveríamos frequentemente reflectir.
\end{english}

\chapter[Dez Temas]{Dez Temas para Recordar Frequentemente por Aqueles que Seguem o Caminho}

\firstline{Dasa ime bhikkhave}

\enlargethispage{\baselineskip}

\begin{leader}
  [Ha꜓nda mayaṃ pabbajita\hyp{}abhiṇha\hyp{}paccave꜕kkhaṇa\hyp{}pāṭhaṃ bhaṇāmase]
\end{leader}

[Dasa i꜕me bhikkhave] dhammā pabba꜕jitena a꜕bhiṇhaṃ pacca꜕vekkhi꜓tabbā, ka꜕ta꜕me dasa

\begin{english}
  Monges, existem dez dhammas acerca dos quais se deve reflectir frequentemente. \prul{Quais} são estes dez dhammas?
\end{english}

Vevaṇṇi꜕yamhi ajjhūpa꜕ga꜕to'ti pabba꜕jitena a꜕bhiṇhaṃ pacca꜕vekkhi꜓tabbaṃ

\begin{english}
  `Já não vivo segundo os valores e objectivos do mundo.'\\
  Quem perfaz o caminho\\
  deve reflectir sobre isto frequentemente.
\end{english}

Parapaṭi꜕baddhā me jīvi꜓kā'ti pabba꜕jitena a꜕bhiṇhaṃ pacca꜕vekkhi꜓tabbaṃ

\begin{english}
  `A minha própria vida é sustentada pela generosidade dos outros.'\\
  Quem perfaz o caminho\\
  deve reflectir sobre isto frequentemente.
\end{english}

Añño me ākappo ka꜕ra꜕ṇīyo'ti pabba꜕jitena a꜕bhiṇhaṃ pacca꜕vekkhi꜓tabbaṃ

\begin{english}
  `Devo esforçar-me por abandonar os meus hábitos antigos.'\\
  Quem perfaz o caminho\\
  deve reflectir sobre isto frequentemente.
\end{english}

\clearpage

Kacci nu꜕ kho me attā sīla꜕to na u꜕pavadatī'ti pabba꜕jitena a꜕bhiṇhaṃ pacca꜕vekkhi꜓tabbaṃ

\begin{english}
  `Surgem remorsos na minha mente em relação à minha conduta?'\\
  Quem perfaz o caminho\\
  deve reflectir sobre isto frequentemente.
\end{english}

Kacci nu꜕ kho maṃ a꜕nuvicca viññū sabrahma꜓cārī sīla꜕to na u꜕pavadantī'ti pabba꜕jitena a꜕bhiṇhaṃ pacca꜕vekkhi꜓tabbaṃ

\begin{english}
  `Será que os meus companheiros espirituais acham falhas na minha conduta?'\\
  Quem perfaz o caminho\\
  deve reflectir sobre isto frequentemente.
\end{english}

Sa꜕bbehi me pi꜕yehi ma꜕nāpehi꜕ nānābhāvo vi꜕nābhāvo'ti pabba꜕jitena abhiṇhaṃ pacca꜕vekkhi꜓tabbaṃ

\begin{english}
  `Tudo aquilo que é meu, que amo e prezo, tornar-se-á diferente, separar-se-á de mim.'\\
  Quem perfaz o caminho\\
  deve reflectir sobre isto frequentemente.
\end{english}

Kammassa꜕komhi kamma꜓dāyādo kamma꜕yoni kamma꜕bandhu kammapa꜕ṭisa꜓raṇo, yaṃ kammaṃ ka꜕rissāmi, kalyāṇaṃ vā pāpa꜕kaṃ vā, tassa꜕ dāyādo bha꜕vissāmī'ti pabba꜕jitena a꜕bhiṇhaṃ pacca꜕vekkhi꜓tabbaṃ

\enlargethispage{2\baselineskip}

\begin{english}
  `Sou o dono do meu Kamma, herdeiro do meu Kamma,\\
  nascido do meu Kamma, ligado ao meu Kamma,\\
  permaneço suportado pelo meu Kamma; seja qual Kamma eu criar,\\
  Para o bem ou para o mal, \prul{disso} serei o herdeiro.'\\
  Quem perfaz o caminho\\
  deve reflectir sobre isto frequentemente.
\end{english}

\clearpage

`Kathambhūtassa꜕ me rattindi꜕vā vīti꜕pa꜓tantī'ti pabba꜕jitena a꜕bhiṇhaṃ pacca꜕vekkhi꜓tabbaṃ

\begin{english}
  `Os dias e as noites passam continuamente; Como estou eu a usar\\ o meu tempo?'\\
 Quem perfaz o caminho\\
 deve reflectir sobre isto frequentemente.
\end{english}

Kacci nu꜕ kho'haṃ suññā꜓gāre abhira꜕māmī'ti pabba꜕jitena a꜕bhiṇhaṃ pacca꜕vekkhi꜓tabbaṃ

\begin{english}
  `Aprecio a solidão ou não?'\\
  Quem perfaz o caminho\\
  deve reflectir sobre isto frequentemente.
\end{english}

Atthi nu꜕ kho me uttari-ma꜕nussa-dhammā alamariya꜕-ñāṇa-dassana-viseso adhiga꜕to, so'haṃ pacchi꜓me kāle sa꜕brahmacārīhi꜕ puṭṭho na maṅku bha꜕vissāmī'ti pabba꜕jitena a꜕bhiṇhaṃ pacca꜕vekkhi꜓tabbaṃ

\begin{english}
  `Deu a minha prática frutos de compreensão e liberdade, de forma a que no fim da minha vida eu não me sinta envergonhado quando questionado pelos meus companheiros espirituais?'\\
  Quem perfaz o caminho\\
  deve reflectir sobre isto frequentemente.
\end{english}

Ime kho bhikkha꜓ve da꜕sa꜕ dhammā pabba꜕jitena a꜕bhiṇhaṃ pacca꜕vekkhitabbā'ti

\begin{english}
  Monges estes são dez Dhammas sobre os quais se deve reflectir frequentemente.
\end{english}

\chapter[Trinta-e-duas-Partes]{Reflexão sobre as Trinta-e-duas-Partes}

\firstline{Ayaṃ kho me kāyo}

\begin{leader}
  [Ha꜓nda mayaṃ dvattiṃsākāra-pāṭhaṃ bhaṇāmase]
\end{leader}

[Ayaṃ kho] me kāyo uddhaṃ pāda꜕ta꜕lā adho kesamatthakā\\
ta꜕ca꜕pa꜕ri꜕yanto pūro nānappa꜕kārassa꜕ a꜕su꜕ci꜕no

\begin{english}
  Isto, que é o meu corpo, das plantas dos pés para cima, e do topo da cabeça para baixo, é um saco de pele fechado cheio de coisas repugnantes.
\end{english}

Atthi imasmiṃ kāye

\begin{english}
  Neste corpo existem:
\end{english}

{\centering
\setArrayStrech{1}

\begin{tabular}{ r l }
kesā            & \tr{cabelo} \\
lomā            & \tr{pelos} \\
nakhā           & \tr{unhas} \\
dantā           & \tr{dentes} \\
taco            & \tr{pele} \\
maṃsaṃ          & \tr{carne} \\
nahārū          & \tr{tendões} \\
aṭṭhī           & \tr{ossos} \\
aṭṭhimiñjaṃ     & \tr{medula óssea} \\
vakkaṃ          & \tr{rins} \\
hadayaṃ         & \tr{coração} \\
yakanaṃ         & \tr{fígado} \\
kilomakaṃ       & \tr{membranas} \\
pihakaṃ         & \tr{baço} \\
papphāsaṃ       & \tr{pulmões} \\
\end{tabular}

\clearpage

\begin{tabular}{ r l }
antaṃ           & \tr{intestinos} \\
antaguṇaṃ       & \tr{tripas} \\
udariyaṃ        & \tr{comida não digerida} \\
karīsaṃ         & \tr{excremento} \\
pittaṃ          & \tr{bílis} \\
semhaṃ          & \tr{muco} \\
pubbo           & \tr{pus} \\% TODO: is this translated?
lohitaṃ         & \tr{sangue} \\
sedo            & \tr{suor} \\
medo            & \tr{gordura} \\
assu            & \tr{lágrimas} \\
vasā            & \tr{sebo} \\
kheḷo           & \tr{saliva} \\
siṅghāṇikā      & \tr{mucosidade} \\
lasikā          & \tr{lubrificante das articulações} \\
muttaṃ          & \tr{urina} \\
matthaluṅgan'ti & \tr{miolos} \\
\end{tabular}

\restoreArrayStretch
}

Evam-ayaṃ me kāyo uddhaṃ pāda꜕ta꜕lā adho kesamatthakā\\
ta꜕ca꜕pa꜕ri꜕yanto pūro nānappa꜕kārassa꜕ a꜕su꜕ci꜕no

\begin{english}
  Assim, isto que é o meu corpo, das plantas dos pés para cima, e do topo da cabeça para baixo, é um saco de pele fechado cheio de coisas repugnantes.
\end{english}

\chapter{The Teaching on Mindfulness of Breathing}

\begin{leader}
  [Ha꜓nda mayam ānāpānass꜕ati-sutta-pāṭhaṃ bha꜕ṇāmase]
\end{leader}

Ānāpāna꜓ssa꜕ti bhi꜓kkha꜕ve bhāvi꜓tā bahu꜕līka꜕tā\\
Mahappha꜕lā ho꜓ti mahā꜓nisa꜓ṃsā\\
Ānāpāna꜓ssa꜕ti bhi꜓kkha꜕ve bhāvi꜓tā bahu꜕līka꜕tā\\
Ca꜕ttāro sati꜓pa꜕ṭṭhāne pa꜕ri꜓pū꜕reti\\
Ca꜕ttāro sa꜕tipa꜕ṭṭhānā bhāvi꜓tā bahu꜕līka꜕tā\\
Sa꜕tta-bojjhaṅge pa꜕ri꜓pū꜕renti\\
Sa꜕tta-bojjhaṅgā bhāvi꜓tā bahu꜕līka꜕tā\\
Vijjā-vimuttiṃ pa꜕ri꜓pū꜕renti\\
Kathaṃ bhāvi꜓tā ca bhi꜓kkha꜕ve ānāpāna꜓ss꜕ati ka꜕thaṃ bahu꜕līka꜕tā\\
Mahappha꜕lā ho꜓ti mahā꜓nisa꜓ṃsā\\
Idha bhi꜓kkha꜕ve bhikkhu\\
Arañña꜓-ga꜕to vā\\
Rukkha-mūla꜓-ga꜕to vā\\
Suññāgāra꜓-ga꜕to vā\\
N꜕isīdati pallaṅkaṃ ābhuji꜓tv꜕ā\\
Ujuṃ kāyaṃ pa꜕ṇidhāya pa꜕rimukhaṃ sa꜕tiṃ u꜕paṭṭha꜕petvā\\
So sa꜕to'va a꜕ssasa꜕ti sa꜕to'va pa꜕ssa꜕sa꜕ti\\
Dīghaṃ vā assa꜕sa꜓nto dīghaṃ a꜕ssasā꜓mī'ti pa꜕jānāti\\
Dīghaṃ vā pa꜕ssa꜕santo dīghaṃ pa꜕ssasā꜓mī'ti pa꜕jānāti\\
Rassaṃ vā a꜕ssa꜕santo rassaṃ a꜕ssasā꜓mī'ti pa꜕jānāti\\
Rassaṃ vā pa꜕ssa꜕santo rassaṃ pa꜕ssasā꜓mī'ti pa꜕jānāti\\
Sabba꜕-kāya-paṭ꜕isa꜓ṃvedī a꜕ssasi꜕ssāmī'ti si꜕kkh꜕ati\\
Sabba꜕-kāya-paṭ꜕isa꜓ṃvedī pa꜕ssasi꜕ssāmī'ti si꜕kkh꜕ati

\clearpage

\enlargethispage{2\baselineskip}

Passa꜕mbhayaṃ kāya꜕-sa꜓ṅkhāraṃ a꜕ssasi꜕ssāmī'ti si꜕kkh꜕ati\\
Passa꜕mbhayaṃ kāya꜕-sa꜓ṅkhāraṃ pa꜕ssasi꜕ssāmī'ti si꜕kkh꜕ati\\
Pīti꜕-paṭi꜕sa꜓ṃvedī a꜕ssasi꜕ssāmī'ti si꜕kkh꜕ati\\
Pīti꜕-paṭi꜕sa꜓ṃvedī pa꜕ssasi꜕ssāmī'ti si꜕kkh꜕ati\\
Sukh꜕a-paṭi꜕sa꜓ṃvedī a꜕ssasi꜕ssāmī'ti si꜕kkh꜕ati\\
Sukh꜕a-paṭi꜕sa꜓ṃvedī pa꜕ssasi꜕ssāmī'ti si꜕kkh꜕ati\\
Citta꜕-sa꜓ṅkhāra-paṭi꜕sa꜓ṃvedī a꜕ssasi꜕ssāmī'ti si꜕kkh꜕ati\\
Citta꜕-sa꜓ṅkhāra-paṭi꜕sa꜓ṃvedī pa꜕ssasi꜕ssāmī'ti si꜕kkh꜕ati\\
Passa꜕mbhayaṃ citta꜕-sa꜓ṅkhāraṃ a꜕ssasi꜕ssāmī'ti si꜕kkh꜕ati\\
Passa꜕mbhayaṃ citt꜕a-sa꜓ṅkhāraṃ pa꜕ssasi꜕ssāmī'ti si꜕kkh꜕ati\\
Citta꜕-paṭi꜕sa꜓ṃvedī a꜕ssasi꜕ssāmī'ti si꜕kkh꜕ati\\
Citta꜕-paṭi꜕sa꜓ṃvedī pa꜕ssasi꜕ssāmī'ti si꜕kkh꜕ati\\
A꜕bhippa꜕moda꜓yaṃ cittaṃ a꜕ssasi꜕ssāmī'ti si꜕kkh꜕ati\\
A꜕bhippa꜕moda꜓yaṃ cittaṃ pa꜕ssasi꜕ssāmī'ti si꜕kkh꜕ati\\
Sa꜕māda꜓haṃ cittaṃ a꜕ssasi꜕ssāmī'ti si꜕kkh꜕ati\\
Sa꜕māda꜓haṃ cittaṃ pa꜕ssasi꜕ssāmī'ti si꜕kkh꜕ati\\
Vimoca꜓yaṃ cittaṃ a꜕ssasi꜕ssāmī'ti si꜕kkh꜕ati\\
Vimoca꜓yaṃ cittaṃ pa꜕ssasi꜕ssāmī'ti si꜕kkh꜕ati\\
Aniccānupa꜕ssī a꜕ssasi꜕ssāmī'ti si꜕kkh꜕ati\\
Aniccānupa꜕ssī pa꜕ssasi꜕ssāmī'ti si꜕kkh꜕ati\\
Virāgānupa꜕ssī a꜕ssasi꜕ssāmī'ti si꜕kkh꜕ati\\
Virāgānupa꜕ssī pa꜕ssasi꜕ssāmī'ti si꜕kkh꜕ati\\
Nirodhānupa꜕ssī a꜕ssasi꜕ssāmī'ti si꜕kkh꜕ati\\
Nirodhānupa꜕ssī pa꜕ssasi꜕ssāmī'ti si꜕kkh꜕ati\\
Pa꜕ṭiniss꜕aggānupa꜕ssī a꜕ssasi꜕ssāmī'ti si꜕kkh꜕ati\\
Pa꜕ṭinissa꜕ggānupa꜕ssī pa꜕ssasi꜕ssāmī'ti si꜕kkh꜕ati\\
Evaṃ bhāvi꜓tā kho bhi꜓kkha꜕ve ānāpāna꜓ss꜕ati evaṃ bahu꜕līka꜕tā\\
Mahappha꜕lā ho꜓ti mahā꜓nisa꜓ṃsā'ti

\end{document}
