\documentclass[
  babelLanguage=portuguese,
  final,
  %showtrims,
  %showwirebinding,
  %webversion,
]{chantingbook}

\usepackage{local}

\title{Ovāda-Pāṭimokkha}

\begin{document}

\mainmatter

\eveningChapterSettings

\chapter{Ovāda-Pāṭimokkha}

\thispagestyle{empty}

\enlargethispage{2\baselineskip}

\begin{leader}
  [Ha꜓nda mayaṁ ovāda-pā꜕ṭi꜕mokkha-gāthā꜓yo bha꜕ṇāmase]
\end{leader}

Kha꜓ntī pa꜕ramaṁ ta꜕po tīti꜕kkhā\\
Nibbānaṁ pa꜕ramaṁ va꜕dant꜕i buddhā\\
Na h꜕i pa꜕bbaji꜕to pa꜕rūpaghātī\\
Sa꜕maṇo ho꜓ti pa꜕raṁ vihe꜓ṭha꜕yanto

Sa꜕bb꜕a-pāpa꜕ss꜕a a꜕ka꜕ra꜓ṇaṁ\\
Ku꜕salassūpasa꜓mpa꜕dā\\
Sa꜕ci꜕tta-pa꜕ri꜓yoda꜓pa꜕naṁ\\
Etaṁ buddhāna sā꜓sa꜕naṁ

A꜕nūpa꜕vādo a꜕nūpa꜕ghāto\\
Pā꜕ṭimokkhe꜓ ca꜕ sa꜓ṁva꜕ro\\
Mattaññu꜕tā ca꜕ bhatta꜕smiṁ\\
Pa꜕ntañca꜕ saya꜓n'āsa꜕naṁ\\
A꜕dhici꜕tte ca꜕ āyogo\\
Etaṁ buddhāna sā꜓sa꜕naṁ

\bigskip

{\itshape

Permanecer paciente é a maior austeridade.\\
“Nibbāna é supremo”, dizem os Buddhas.\\
Não se é verdadeiramente monge quando se prejudica alguém,\\
nem verdadeiramente renunciante quando se oprime os outros.

Evitar todo o mal, cultivar o bem e purificar a mente --\\
Este é o ensinamento dos Buddhas.

Não ofender, não prejudicar,\\
conter-se de acordo com o código da disciplina monástica,\\
moderar-se na comida, viver solitário, devotar-se à consciência elevada --\\
Este é o ensinamento dos Buddhas.

}

\end{document}
