\newcommand\englishPage{%
  \clearpage%
  \englishText%
  %\markboth{\englishTitle}{\rightmark}%
}

\newcommand\paliPage{%
  \clearpage%
  \paliText%
  %\markboth{\paliTitle}{\rightmark}%
}

\renewcommand{\englishTitle}{The Foundations of Mindfulness}
\renewcommand{\paliTitle}{Mahāsatipaṭṭhāna Sutta}

\englishPage
\chapter{Introduction}

Thus have I heard.

On one occasion the Blessed One was in the Kuru country where there was a town
o̱f the̱ Ku̮ru̱s na̮me̱d Ka̱mmā̱sa̮dha̱mma̮. The̱re̱ the̱ Ble̱sse̱d O̱ne̱ a̱ddre̱sse̱d the̱ bhi̱kkhu͓s
thus: “Bhikkhus.” “Bhante,” the bhikkhus replied to the Blessed One. The Blessed
One said this:

“Bhikkhus, this is the one-way path for the purification of beings, for the
surmounting of sorrow and lamentation, for the passing away of pain and
de̱je̱cti̱o̱n, fo̱r the̱ a̱tta̱i̱nme̱nt o̱f the̱ tru̱e̱ wa̱y, fo̱r the̱ re̱a̮li̮sa̮ti̱o̱n o̱f Ni̱bbā̱na͓,
namely, the four foundations of mindfulness. What are the four?

Here, bhikkhus, a bhikkhu dwells contemplating the body in the body, ardent,
clearly comprehending, and mindful, having subdued longing and dejection in
regard to the world. He dwells contemplating feelings in feelings, ardent,
clearly comprehending, and mindful, having subdued longing and dejection in
regard to the world. He dwells contemplating mind in mind, ardent, clearly
comprehending, and mindful, having subdued longing and dejection in regard to
the world. He dwells contemplating phenomena in phenomena, ardent, clearly
comprehending, and mindful, having subdued longing and dejection in regard to
the world.

\instr{The Introduction is finished.}

\paliPage
\chapter*{Uddeso}

[E̱va̱ṁ me̱ su̮ta͓ṁ]

E̱ka̱ṁ sa̮ma̮ya̱ṁ bha̮ga̮vā̱ ku̮rū̱su̮ vi̮ha̮ra̮ti̮ ka̱mmā̱sa̮dha̱mma̱ṁ nā̱ma̮ ku̮rū̱na̱ṁ ni̮ga̮mo̱. Ta̱tra͓
kho̱꜔꜒ bha̮ga̮vā̱ bhi̱kkhū̱꜔꜒ ā̱ma̱nte̱si̮: “bhi̱kkha̮vo̱”ti̮. “Bha̱dda̱nte̱”ti̮ te̱ bhi̱kkhū̱꜔꜒ bha̮ga̮va̮to͓
pa̱cca̱sso̱꜔꜒su̱꜔꜒ṁ. Bha̮ga̮vā̱ e̱ta̮da̮vo̱ca͓:

“E̱kā̱ya̮no̱ a̮ya̱ṁ, bhi̱kkha̮ve̱, ma̱ggo̱ sa̱ttā̱na̱ṁ vi̮su̱ddhi̮yā̱, so̱꜔꜒ka̮-pa̮ri̮de̱vā̱na͓ṁ
sa̮ma̮ti̱kka̮mā̱ya̮ du̱kkha̮-do̱ma̮na̱ssā̱꜔꜒na̱ṁ a̱ttha̱꜔꜒ṅga̮mā̱ya̮ ñā̱ya̱ssa̮ a̮dhi̮ga̮mā̱ya̮ ni̱bbā̱na̱ssa͓
sa̱cchi̮ki̮ri̮yā̱ya̮, ya̮di̮da̱ṁ ca̱ttā̱ro̱ sa̮ti̮pa̱ṭṭhā̱꜔꜒nā͓.

Ka̮ta̮me̱ ca̱ttā̱ro̱? I̮dha̮, bhi̱kkha̮ve̱, bhi̱kkhu̮ kā̱ye̱ kā̱yā̱nu̮pa̱ssī̱꜔꜒ vi̮ha̮ra̮ti̮ ā̱tā̱pī͓
sa̱꜔꜒mpa̮jā̱no̱ sa̮ti̮mā̱ vi̮ne̱yya̮ lo̱ke̱ a̮bhi̱jjhā̱-do̱ma̮na̱ssa̱꜔꜒ṁ, ve̱da̮nā̱su̮ ve̱da̮nā̱nu̮pa̱ssī͓꜔꜒
vi̮ha̮ra̮ti̮ ā̱tā̱pī̱ sa̱꜔꜒mpa̮jā̱no̱ sa̮ti̮mā̱ vi̮ne̱yya̮ lo̱ke̱ a̮bhi̱jjhā̱-do̱ma̮na̱ssa̱꜔꜒ṁ, ci̱tte͓
ci̱ttā̱nu̮pa̱ssī̱꜔꜒ vi̮ha̮ra̮ti̮ ā̱tā̱pī̱ sa̱꜔꜒mpa̮jā̱no̱ sa̮ti̮mā̱ vi̮ne̱yya̮ lo̱ke̱ a̮bhi̱jjhā̱-do̱ma̮na̱ssa͓꜔꜒ṁ,
dha̱mme̱su̮ dha̱mmā̱nu̮pa̱ssī̱꜔꜒ vi̮ha̮ra̮ti̮ ā̱tā̱pī̱ sa̱꜔꜒mpa̮jā̱no̱ sa̮ti̮mā̱ vi̮ne̱yya̮ lo̱ke͓
a̮bhi̱jjhā̱-do̱ma̮na̱ssa͓꜔꜒ṁ.

\instr{Uddeso niṭṭhito.}

\englishPage
\chapter{Contemplation of the Body}

\section{Mindfulness of Breathing}

And how, bhikkhus, does a bhikkhu dwell contemplating\\
the body in the body?

Here, bhikkhus, a bhikkhu, gone to the forest, to the foot of a tree, or to an
empty hut, sits down; having folded his legs crosswise, straightened his body,
and established mindfulness in front of him.

Just mindful he breathes in, mindful he breathes out.\\
Breathing in long, he understands: ‘I breathe in long’;\\
or breathing out long, he understands: ‘I breathe out long.’\\
Breathing in short, he understands: ‘I breathe in short’;\\
or breathing out short, he understands: ‘I breathe out short.’\\
He trains thus: ‘I will breathe in experiencing the whole body’;\\
he trains thus:‘I will breathe out experiencing the whole body.’\\
He trains thus: ‘I will breathe in tranquilising the bodily formation’;\\
he trains thus: ‘I will breathe out tranquilising the bodily formation.’

Just as, bhikkhus, a skilled lathe-worker or his apprentice,\\
when making a long turn, understands: ‘I make a long turn’;\\
or, when making a short turn, understands: ‘I make a short turn’;\\
so too, bhikkhus, a bhikkhu\\
breathing in long, he understands: ‘I breathe in long’;\\
or breathing out long, he understands: ‘I breathe out long.’\\
Breathing in short, he understands: ‘I breathe in short’;\\
or breathing out short, he understands: ‘I breathe out short.’\\
He trains thus: ‘I will breathe in experiencing the whole body’;\\
he trains thus: ‘I will breathe out experiencing the whole body.’\\
He trains thus: ‘I will breathe in tranquilising the bodily formation’;\\
he trains thus: ‘I will breathe out tranquilising the bodily formation.’

\paliPage
\chapter*{Kāyānupassanā}

\section*{Ānāpānapabba}

Ka̮tha̱꜔꜒ñca̮ pa̮na̮, bhi̱kkha̮ve̱, bhi̱kkhu̮ kā̱ye̱ kā̱yā̱nu̮pa̱ssī̱꜔꜒ vi̮ha̮ra̮ti͓?

I̮dha̮, bhi̱kkha̮ve̱, bhi̱kkhu̮ a̮ra̱ñña̮ga̮to̱ vā̱ ru̱kkha̮mū̱la̮ga̮to̱ vā̱ su̱꜔꜒ññā̱gā̱ra̮ga̮to̱ vā͓
ni̮sī̱꜔꜒da̮ti̮ pa̱lla̱ṅka̱ṁ ā̱bhu̮ji̱tvā̱ u̮ju̱ṁ kā̱ya̱ṁ pa̮ṇi̮dhā̱ya̮ pa̮ri̮mu̮kha̱꜔꜒ṁ sa̮ti̱ṁ u̮pa̱ṭṭha̮pe̱tvā͓.
So꜔꜒ satova assasati, satova passasati.

Dī̱gha̱ṁ vā̱ a̱ssa̮sa̱꜔꜒nto̱ ‘dī̱gha̱ṁ a̱ssa̮sā̱꜔꜒mī̱’ti̮ pa̮jā̱nā̱ti͓,\\
dī̱gha̱ṁ vā̱ pa̱ssa̮sa̱꜔꜒nto̱ ‘dī̱gha̱ṁ pa̱ssa̮sā̱꜔꜒mī̱’ti̮ pa̮jā̱nā̱ti͓.\\
Ra̱ssa̱꜔꜒ṁ vā̱ a̱ssa̮sa̱꜔꜒nto̱ ‘ra̱ssa̱꜔꜒ṁ a̱ssa̮sā̱꜔꜒mī̱’ti̮ pa̮jā̱nā̱ti͓,\\
ra̱ssa̱꜔꜒ṁ vā̱ pa̱ssa̮sa̱꜔꜒nto̱ ‘ra̱ssa̱꜔꜒ṁ pa̱ssa̮sā̱꜔꜒mī̱’ti̮ pa̮jā̱nā̱ti͓.\\
‘Sa̱bba̮kā̱ya̮-pa̮ṭi̮sa̱꜔꜒ṁve̱dī̱ a̱ssa̮si̱ssā̱꜔꜒mī̱’ti̮ si̱kkha̮ti͓,\\
‘sa̱bba̮kā̱ya̮-pa̮ṭi̮sa̱꜔꜒ṁve̱dī̱ pa̱ssa̮si̱ssā̱꜔꜒mī̱’ti̮ si̱kkha̮ti͓.\\
‘Pa̱ssa̱꜔꜒mbha̮ya̱ṁ kā̱ya̮sa̱꜔꜒ṅkhā̱꜔꜒ra̱ṁ a̱ssa̮si̱ssā̱꜔꜒mī̱’ti̮ si̱kkha̮ti͓,\\
‘pa̱ssa̱꜔꜒mbha̮ya̱ṁ kā̱ya̮sa̱꜔꜒ṅkhā̱꜔꜒ra̱ṁ pa̱ssa̮si̱ssā̱꜔꜒mī̱’ti̮ si̱kkha̮ti͓.

Se̱꜔꜒yya̮thā̱꜔꜒pi̮, bhi̱kkha̮ve̱, da̱kkho̱꜔꜒ bha̮ma̮kā̱ro̱ vā̱ bha̮ma̮kā̱ra̱nte̱vā̱sī̱꜔꜒ vā͓\\
dī̱gha̱ṁ vā̱ a̱ñcha̱꜔꜒nto̱ ‘dī̱gha̱ṁ a̱ñchā̱꜔꜒mī̱’ti̮ pa̮jā̱nā̱ti͓,\\
ra̱ssa̱꜔꜒ṁ vā̱ a̱ñcha̱꜔꜒nto̱ ‘ra̱ssa̱꜔꜒ṁ a̱ñchā̱꜔꜒mī̱’ti̮ pa̮jā̱nā̱ti͓;\\
evameva kho꜔꜒, bhikkhave, bhikkhu\\
dī̱gha̱ṁ vā̱ a̱ssa̮sa̱꜔꜒nto̱ ‘dī̱gha̱ṁ a̱ssa̮sā̱꜔꜒mī̱’ti̮ pa̮jā̱nā̱ti͓,\\
dī̱gha̱ṁ vā̱ pa̱ssa̮sa̱꜔꜒nto̱ ‘dī̱gha̱ṁ pa̱ssa̮sā̱꜔꜒mī̱’ti̮ pa̮jā̱nā̱ti͓,\\
ra̱ssa̱꜔꜒ṁ vā̱ a̱ssa̮sa̱꜔꜒nto̱ ‘ra̱ssa̱꜔꜒ṁ a̱ssa̮sā̱꜔꜒mī̱’ti̮ pa̮jā̱nā̱ti͓,\\
ra̱ssa̱꜔꜒ṁ vā̱ pa̱ssa̮sa̱꜔꜒nto̱ ‘ra̱ssa̱꜔꜒ṁ pa̱ssa̮sā̱꜔꜒mī̱’ti̮ pa̮jā̱nā̱ti͓.\\
‘Sa̱bba̮kā̱ya̮-pa̮ṭi̮sa̱꜔꜒ṁve̱dī̱ a̱ssa̮si̱ssā̱꜔꜒mī̱’ti̮ si̱kkha̮ti͓,\\
‘sa̱bba̮kā̱ya̮-pa̮ṭi̮sa̱꜔꜒ṁve̱dī̱ pa̱ssa̮si̱ssā̱꜔꜒mī̱’ti̮ si̱kkha̮ti͓,\\
‘pa̱ssa̱꜔꜒mbha̮ya̱ṁ kā̱ya̮sa̱꜔꜒ṅkhā̱꜔꜒ra̱ṁ a̱ssa̮si̱ssā̱꜔꜒mī̱’ti̮ si̱kkha̮ti͓,\\
‘pa̱ssa̱꜔꜒mbha̮ya̱ṁ kā̱ya̮sa̱꜔꜒ṅkhā̱꜔꜒ra̱ṁ pa̱ssa̮si̱ssā̱꜔꜒mī̱’ti̮ si̱kkha̮ti͓.

\englishPage

In this way he dwells contemplating the body in the body internally, or he
dwells contemplating the body in the body externally, or he dwells contemplating
the body in the body both internally and externally. Or else he dwells
contemplating in the body its nature of arising, or he dwells contemplating in
the body its nature of vanishing, or he dwells contemplating in the body its
nature of both arising and vanishing. Or else mindfulness that ‘there is a body’
is simply established in him to the extent necessary for bare knowledge and
repeated mindfulness.

And he dwells independent, not clinging to anything in the world. That is how,
bhikkhus, a bhikkhu dwells contemplating the body in the body.

\instr{The section on Mindfulness of Breathing is finished.}

\section{The Four Postures}

Again, bhikkhus, a bhikkhu when walking, understands: ‘I am walking’; when
standing, he understands: ‘I am standing’; when sitting, he understands: ‘I am
sitting’; when lying down, he understands: ‘I am lying down’; or however his
body is disposed, he understands it accordingly.

In this way he dwells contemplating the body in the body internally, or he
dwells contemplating the body in the body externally, or he dwells contemplating
the body in the body both internally and externally. Or else he dwells
contemplating in the body its nature of arising, or he dwells contemplating in
the body its nature of vanishing, or he dwells contemplating in the body its
nature of both arising and vanishing. Or else mindfulness that ‘there is a body’
is simply established in him to the extent necessary for bare knowledge and
repeated mindfulness.

And he dwells independent, not clinging to anything in the world. That is how,
bhikkhus, a bhikkhu dwells contemplating the body in the body.

\instr{The section on the Four Postures is finished.}

\paliPage

I̮ti̮ a̱jjha̱tta̱ṁ vā̱ kā̱ye̱ kā̱yā̱nu̮pa̱ssī̱꜔꜒ vi̮ha̮ra̮ti̮, ba̮hi̱ddhā̱ vā̱ kā̱ye̱ kā̱yā̱nu̮pa̱ssī͓꜔꜒
vi̮ha̮ra̮ti̮, a̱jjha̱tta̮-ba̮hi̱ddhā̱ vā̱ kā̱ye̱ kā̱yā̱nu̮pa̱ssī̱꜔꜒ vi̮ha̮ra̮ti̮. Sa̮mu̮da̮ya̮-dha̱mmā̱nu̮pa̱ssī͓꜔꜒
vā̱ kā̱ya̱smi̱꜔꜒ṁ vi̮ha̮ra̮ti̮, va̮ya̮-dha̱mmā͓-\\
nu̮pa̱ssī̱꜔꜒ vā̱ kā̱ya̱smi̱꜔꜒ṁ vi̮ha̮ra̮ti̮, sa̮mu̮da̮ya̮-va̮ya̮-dha̱mmā̱nu̮pa̱ssī̱꜔꜒ vā̱ kā̱ya̱smi̱꜔꜒ṁ vi̮ha̮ra̮ti͓.
‘A̱tthi̮ kā̱yo̱’ti̮ vā̱ pa̮na̱ssa̮ sa̮ti̮ pa̱ccu̮pa̱ṭṭhi̮tā̱ ho̱꜔꜒ti̮ yā̱va̮de̱va̮ ñā̱ṇa̮ma̱ttā̱ya͓
pa̮ṭi̱ssa̮ti̮ma̱ttā̱ya̮ a̮ni̱ssi̮to̱ ca̮ vi̮ha̮ra̮ti̮, na̮ ca̮ ki̱ñci̮ lo̱ke̱ u̮pā̱di̮ya̮ti̮. E̱va̱mpi̮ kho͓꜔꜒,
bhi̱kkha̮ve̱, bhi̱kkhu̮ kā̱ye̱ kā̱yā̱nu̮pa̱ssī̱꜔꜒ vi̮ha̮ra̮ti͓.

\instr{Ānāpānapabbaṁ niṭṭhitaṁ.}

\section*{Iriyāpathapabba}

Pu̮na̮ ca̮pa̮ra̱ṁ, bhi̱kkha̮ve̱, bhi̱kkhu̮ ga̱ccha̱꜔꜒nto̱ vā̱ ‘ga̱cchā̱꜔꜒mī̱’ti̮ pa̮jā̱nā̱ti̮, ṭhi̮to̱ vā͓
‘ṭhi̮to̱mhī̱꜔꜒’ti̮ pa̮jā̱nā̱ti̮, ni̮si̱꜔꜒nno̱ vā̱ ‘ni̮si̱꜔꜒nno̱mhī̱꜔꜒’ti̮ pa̮jā̱nā̱ti̮, sa̮yā̱no̱ vā͓
‘sa̮yā̱no̱mhī̱꜔꜒’ti̮ pa̮jā̱nā̱ti̮, ya̮thā̱꜔꜒ ya̮thā̱꜔꜒ vā̱ pa̮na̱ssa̮ kā̱yo̱ pa̮ṇi̮hi̮to̱ ho̱꜔꜒ti̮ ta̮thā̱꜔꜒ ta̮thā͓꜔꜒
na̱ṁ pa̮jā̱nā̱ti͓.

I̮ti̮ a̱jjha̱tta̱ṁ vā̱ kā̱ye̱ kā̱yā̱nu̮pa̱ssī̱꜔꜒ vi̮ha̮ra̮ti̮, ba̮hi̱ddhā̱ vā̱ kā̱ye̱ kā̱yā̱nu̮pa̱ssī͓꜔꜒
vi̮ha̮ra̮ti̮, a̱jjha̱tta̮-ba̮hi̱ddhā̱ vā̱ kā̱ye̱ kā̱yā̱nu̮pa̱ssī̱꜔꜒ vi̮ha̮ra̮ti̮. Sa̮mu̮da̮ya̮-dha̱mmā̱nu̮pa̱ssī͓꜔꜒
vā̱ kā̱ya̱smi̱꜔꜒ṁ vi̮ha̮ra̮ti̮, va̮ya̮-dha̱mmā͓-\\
nu̮pa̱ssī̱꜔꜒ vā̱ kā̱ya̱smi̱꜔꜒ṁ vi̮ha̮ra̮ti̮, sa̮mu̮da̮ya̮-va̮ya̮-dha̱mmā̱nu̮pa̱ssī̱꜔꜒ vā̱ kā̱ya̱smi̱꜔꜒ṁ vi̮ha̮ra̮ti͓.
‘A̱tthi̮ kā̱yo̱’ti̮ vā̱ pa̮na̱ssa̮ sa̮ti̮ pa̱ccu̮pa̱ṭṭhi̮tā̱ ho̱꜔꜒ti̮ yā̱va̮de̱va̮ ñā̱ṇa̮ma̱ttā̱ya͓
pa̮ṭi̱ssa̮ti̮ma̱ttā̱ya̮ a̮ni̱ssi̮to̱ ca̮ vi̮ha̮ra̮ti̮, na̮ ca̮ ki̱ñci̮ lo̱ke̱ u̮pā̱di̮ya̮ti̮. E̱va̱mpi̮ kho͓꜔꜒,
bhi̱kkha̮ve̱, bhi̱kkhu̮ kā̱ye̱ kā̱yā̱nu̮pa̱ssī̱꜔꜒ vi̮ha̮ra̮ti͓.

\instr{Iriyāpathapabbaṁ niṭṭhitaṁ.}

\englishPage
\section{Clear Comprehension}

Again, bhikkhus, a bhikkhu is one who acts with clear comprehension when going
forward and returning, who acts with clear comprehension when looking ahead and
looking away; who acts with clear comprehension when bending and stretching his
limbs; who acts with clear comprehension when wearing his robes and carrying his
outer robe and bowl; who acts with clear comprehension when eating, drinking,
chewing, and tasting; who acts with clear comprehension when defecating and
urinating; who acts with clear comprehension when walking, standing, sitting,
falling asleep, waking up, talking and keeping silent.

In this way he dwells contemplating the body in the body internally, or he
dwells contemplating the body in the body externally, or he dwells contemplating
the body in the body both internally and externally. Or else he dwells
contemplating in the body its nature of arising, or he dwells contemplating in
the body its nature of vanishing, or he dwells contemplating in the body its
nature of both arising and vanishing. Or else mindfulness that ‘there is a body’
is simply established in him to the extent necessary for bare knowledge and
repeated mindfulness.

And he dwells independent, not clinging to anything in the world. That is how,
bhikkhus, a bhikkhu dwells contemplating the body in the body.

\instr{The section on Clear Comprehension is finished.}

\section{Unattractiveness of the Body}

Again, bhikkhus, a bhikkhu reviews this same body up from the soles of the feet
and down from the top of the hair, bounded by skin, as full of many kinds of
impurity thus:

\enlargethispage{2\baselineskip}

‘In this body there are head-hairs, body hairs, nails, teeth, skin, flesh,
sinews, bones, bone-marrow, kidneys, heart, liver, diaphragm, spleen, lungs,
intestines, mesentery, stomach, feces, bile, phlegm, pus, blood, sweat, fat,
tears, grease, spittle, snot, oil of the joints, and urine.’

\paliPage
\section*{Sampajānapabba}

Pu̮na̮ ca̮pa̮ra̱ṁ, bhi̱kkha̮ve̱, bhi̱kkhu̮ a̮bhi̱kka̱nte̱ pa̮ṭi̱kka̱nte̱ sa̱꜔꜒mpa̮jā̱na̮kā̱rī̱ ho̱꜔꜒ti͓,
ā̱lo̱ki̮te̱ vi̮lo̱ki̮te̱ sa̱꜔꜒mpa̮jā̱na̮kā̱rī̱ ho̱꜔꜒ti̮, sa̮mi̱ñji̮te̱ pa̮sā̱꜔꜒ri̮te̱ sa̱꜔꜒mpa̮jā̱na̮kā̱rī̱ ho̱꜔꜒ti͓,
sa̱꜔꜒ṅghā̱ṭi̮-pa̱tta̮-cī̱va̮ra̮-dhā̱ra̮ṇe̱ sa̱꜔꜒mpa̮jā̱na̮kā̱rī̱ ho̱꜔꜒ti̮, a̮si̮te̱ pī̱te̱ khā̱꜔꜒yi̮te̱ sā̱꜔꜒yi̮te͓
sa̱꜔꜒mpa̮jā̱na̮kā̱rī̱ ho̱꜔꜒ti̮, u̱ccā̱ra̮-pa̱ssā̱꜔꜒va̮-ka̱mme̱ sa̱꜔꜒mpa̮jā̱na̮kā̱rī̱ ho̱꜔꜒ti̮, ga̮te̱ ṭhi̮te̱ ni̮si̱꜔꜒nne͓
su̱tte̱ jā̱ga̮ri̮te̱ bhā̱si̮te̱ tu̱ṇhī̱꜔꜒bhā̱ve̱ sa̱꜔꜒mpa̮jā̱na̮kā̱rī̱ ho̱꜔꜒ti͓.

I̮ti̮ a̱jjha̱tta̱ṁ vā̱ kā̱ye̱ kā̱yā̱nu̮pa̱ssī̱꜔꜒ vi̮ha̮ra̮ti̮, ba̮hi̱ddhā̱ vā̱ kā̱ye̱ kā̱yā̱nu̮pa̱ssī͓꜔꜒
vi̮ha̮ra̮ti̮, a̱jjha̱tta̮-ba̮hi̱ddhā̱ vā̱ kā̱ye̱ kā̱yā̱nu̮pa̱ssī̱꜔꜒ vi̮ha̮ra̮ti̮. Sa̮mu̮da̮ya̮-dha̱mmā̱nu̮pa̱ssī͓꜔꜒
vā̱ kā̱ya̱smi̱꜔꜒ṁ vi̮ha̮ra̮ti̮, va̮ya̮-dha̱mmā͓-\\
nu̮pa̱ssī̱꜔꜒ vā̱ kā̱ya̱smi̱꜔꜒ṁ vi̮ha̮ra̮ti̮, sa̮mu̮da̮ya̮-va̮ya̮-dha̱mmā̱nu̮pa̱ssī̱꜔꜒ vā̱ kā̱ya̱smi̱꜔꜒ṁ vi̮ha̮ra̮ti͓.
‘A̱tthi̮ kā̱yo̱’ti̮ vā̱ pa̮na̱ssa̮ sa̮ti̮ pa̱ccu̮pa̱ṭṭhi̮tā̱ ho̱꜔꜒ti̮ yā̱va̮de̱va̮ ñā̱ṇa̮ma̱ttā̱ya͓
pa̮ṭi̱ssa̮ti̮ma̱ttā̱ya̮ a̮ni̱ssi̮to̱ ca̮ vi̮ha̮ra̮ti̮, na̮ ca̮ ki̱ñci̮ lo̱ke̱ u̮pā̱di̮ya̮ti̮. E̱va̱mpi̮ kho͓꜔꜒,
bhi̱kkha̮ve̱, bhi̱kkhu̮ kā̱ye̱ kā̱yā̱nu̮pa̱ssī̱꜔꜒ vi̮ha̮ra̮ti͓.

\instr{Sampajānapabbaṁ niṭṭhitaṁ.}

\section*{Paṭikūla-manasikārapabba}

Pu̮na̮ ca̮pa̮ra̱ṁ, bhi̱kkha̮ve̱, bhi̱kkhu̮ i̮ma̮me̱va̮ kā̱ya̱ṁ u̱ddha̱ṁ pā̱da̮ta̮lā̱ a̮dho̱ ke̱sa̮ma̱ttha̮kā͓
ta̮ca̮pa̮ri̮ya̱nta̱ṁ pū̱ra̱ṁ nā̱na̱ppa̮kā̱ra̱ssa̮ a̮su̮ci̮no̱ pa̱cca̮ve̱kkha̮ti͓:

‘A̱tthi̮ i̮ma̱smi̱꜔꜒ṁ kā̱ye̱ ke̱sā̱꜔꜒ lo̱mā̱ na̮khā̱꜔꜒ da̱ntā̱ ta̮co̱, ma̱ṁsa̱꜔꜒ṁ na̮hā̱꜔꜒rū̱ a̱ṭṭhī̱꜔꜒ a̱ṭṭhi̮mi̱ñja͓ṁ
va̱kka̱ṁ, ha̮da̮ya̱ṁ ya̮ka̮na̱ṁ ki̮lo̱ma̮ka̱ṁ pi̮ha̮ka̱ṁ pa̱pphā̱꜔꜒sa̱꜔꜒ṁ, a̱nta̱ṁ a̱nta̮gu̮ṇa̱ṁ u̮da̮ri̮ya͓ṁ
ka̮rī̱sa̱꜔꜒ṁ, pi̱tta̱ṁ se̱꜔꜒mha̱꜔꜒ṁ pu̱bbo̱ lo̱hi̮ta̱ṁ se̱꜔꜒do̱ me̱do̱, a̱ssu̮ va̮sā̱꜔꜒ khe̱꜔꜒ḷo̱ si̱꜔꜒ṅghā̱ṇi̮kā͓
la̮si̮kā̱ mu̱tta̱n’ti͓.

\englishPage

Just as though, bhikkhus, there were a bag with an opening at both ends full of
many sorts of grain, such as hill rice, red rice, beans, peas, millet, and white
rice, and a man with good eyes were to open it and review it thus:

‘This is hill rice, this is red rice, these are beans, these are peas, this is
millet, this is white rice’; so too, bhikkhus, a bhikkhu reviews this same body
up from the soles of the feet and down from the top of the hair, bounded by
skin, as full of many kinds of impurity thus:

'In this body there are head-hairs, body hairs, nails, teeth, skin, flesh,
sinews, bones, bone-marrow, kidneys, heart, liver, diaphragm, spleen, lungs,
intestines, mesentery, stomach, feces, bile, phlegm, pus, blood, sweat, fat,
tears, grease, spittle, snot, oil of the joints, and urine.'

In this way he dwells contemplating the body in the body internally, or he
dwells contemplating the body in the body externally, or he dwells contemplating
the body in the body both internally and externally. Or else he dwells
contemplating in the body its nature of arising, or he dwells contemplating in
the body its nature of vanishing, or he dwells contemplating in the body its
nature of both arising and vanishing. Or else mindfulness that ‘there is a body’
is simply established in him to the extent necessary for bare knowledge and
repeated mindfulness.

And he dwells independent, not clinging to anything in the world. That is how,
bhikkhus, a bhikkhu dwells contemplating the body in the body.

\instr{The section on Unattractiveness of the Body is finished.}

\paliPage

Se̱꜔꜒yya̮thā̱꜔꜒pi̮, bhi̱kkha̮ve̱, u̮bha̮to̱mu̮khā̱꜔꜒ pu̮to̱ḷi̮ pū̱rā̱ nā̱nā̱vi̮hi̮ta̱ssa̮ dha̱ñña̱ssa͓,
se̱꜔꜒yya̮thī̱꜔꜒da̱ṁ, sā̱꜔꜒lī̱na̱ṁ vī̱hī̱꜔꜒na̱ṁ mu̱ggā̱na̱ṁ mā̱sā̱꜔꜒na̱ṁ ti̮lā̱na̱ṁ ta̱ṇḍu̮lā̱na̱ṁ. Ta̮me̱na͓ṁ
ca̱kkhu̮mā̱ pu̮ri̮so̱꜔꜒ mu̱ñci̱tvā̱ pa̱cca̮ve̱kkhe̱꜔꜒yya͓:

‘I̮me̱ sā̱꜔꜒lī̱, i̮me̱ vī̱hī̱꜔꜒ i̮me̱ mu̱ggā̱ i̮me̱ mā̱sā̱꜔꜒ i̮me̱ ti̮lā̱ i̮me̱ ta̱ṇḍu̮lā̱’ti̮. E̱va̮me̱va̮ kho͓꜔꜒,
bhi̱kkha̮ve̱, bhi̱kkhu̮ i̮ma̮me̱va̮ kā̱ya̱ṁ u̱ddha̱ṁ pā̱da̮ta̮lā̱ a̮dho̱ ke̱sa̮ma̱ttha̮kā͓
ta̮ca̮pa̮ri̮ya̱nta̱ṁ pū̱ra̱ṁ nā̱na̱ppa̮kā̱ra̱ssa̮ a̮su̮ci̮no̱ pa̱cca̮ve̱kkha̮ti͓:

‘A̱tthi̮ i̮ma̱smi̱꜔꜒ṁ kā̱ye̱ ke̱sā̱꜔꜒ lo̱mā̱ na̮khā̱꜔꜒ da̱ntā̱ ta̮co̱, ma̱ṁsa̱꜔꜒ṁ na̮hā̱꜔꜒rū̱ a̱ṭṭhī̱꜔꜒ a̱ṭṭhi̮mi̱ñja͓ṁ
va̱kka̱ṁ, ha̮da̮ya̱ṁ ya̮ka̮na̱ṁ ki̮lo̱ma̮ka̱ṁ pi̮ha̮ka̱ṁ pa̱pphā̱꜔꜒sa̱꜔꜒ṁ, a̱nta̱ṁ a̱nta̮gu̮ṇa̱ṁ u̮da̮ri̮ya͓ṁ
ka̮rī̱sa̱꜔꜒ṁ, pi̱tta̱ṁ se̱꜔꜒mha̱꜔꜒ṁ pu̱bbo̱ lo̱hi̮ta̱ṁ se̱꜔꜒do̱ me̱do̱, a̱ssu̮ va̮sā̱꜔꜒ khe̱꜔꜒ḷo̱ si̱꜔꜒ṅghā̱ṇi̮kā͓
la̮si̮kā̱ mu̱tta̱n’ti͓.

I̮ti̮ a̱jjha̱tta̱ṁ vā̱ kā̱ye̱ kā̱yā̱nu̮pa̱ssī̱꜔꜒ vi̮ha̮ra̮ti̮, ba̮hi̱ddhā̱ vā̱ kā̱ye̱ kā̱yā̱nu̮pa̱ssī͓꜔꜒
vi̮ha̮ra̮ti̮, a̱jjha̱tta̮-ba̮hi̱ddhā̱ vā̱ kā̱ye̱ kā̱yā̱nu̮pa̱ssī̱꜔꜒ vi̮ha̮ra̮ti̮. Sa̮mu̮da̮ya̮-dha̱mmā̱nu̮pa̱ssī͓꜔꜒
vā̱ kā̱ya̱smi̱꜔꜒ṁ vi̮ha̮ra̮ti̮, va̮ya̮-dha̱mmā͓-\\
nu̮pa̱ssī̱꜔꜒ vā̱ kā̱ya̱smi̱꜔꜒ṁ vi̮ha̮ra̮ti̮, sa̮mu̮da̮ya̮-va̮ya̮-dha̱mmā̱nu̮pa̱ssī̱꜔꜒ vā̱ kā̱ya̱smi̱꜔꜒ṁ vi̮ha̮ra̮ti͓.
‘A̱tthi̮ kā̱yo̱’ti̮ vā̱ pa̮na̱ssa̮ sa̮ti̮ pa̱ccu̮pa̱ṭṭhi̮tā̱ ho̱꜔꜒ti̮ yā̱va̮de̱va̮ ñā̱ṇa̮ma̱ttā̱ya͓
pa̮ṭi̱ssa̮ti̮ma̱ttā̱ya̮ a̮ni̱ssi̮to̱ ca̮ vi̮ha̮ra̮ti̮, na̮ ca̮ ki̱ñci̮ lo̱ke̱ u̮pā̱di̮ya̮ti̮. E̱va̱mpi̮ kho͓꜔꜒,
bhi̱kkha̮ve̱, bhi̱kkhu̮ kā̱ye̱ kā̱yā̱nu̮pa̱ssī̱꜔꜒ vi̮ha̮ra̮ti͓.

\instr{Paṭikūla-manasikārapabbaṁ niṭṭhitaṁ.}

\englishPage
\section{Elements}

Again, bhikkhus, a bhikkhu reviews this same body, however it is placed, however
disposed, as consisting of elements thus: `In this body there are the earth
element, the water element, the fire element, and the air element.'

Just as though, bhikkhus, a skilled butcher or his apprentice had killed a cow
and were seated at the crossroads with it cut up into pieces; so too, bhikkhus,
a bhikkhu reviews this same body, however it is placed, however disposed, as
consisting of elements thus: `In this body there are the earth element, the
water element, the fire element, and the air element.'

In this way he dwells contemplating the body in the body internally, or he
dwells contemplating the body in the body externally, or he dwells contemplating
the body in the body both internally and externally. Or else he dwells
contemplating in the body its nature of arising, or he dwells contemplating in
the body its nature of vanishing, or he dwells contemplating in the body its
nature of both arising and vanishing. Or else mindfulness that ‘there is a body’
is simply established in him to the extent necessary for bare knowledge and
repeated mindfulness.

And he dwells independent, not clinging to anything in the world. That is how,
bhikkhus, a bhikkhu dwells contemplating the body in the body.

\instr{The section on Elements is finished.}

\section{Nine Charnel Ground Contemplations}

[1] Again, bhikkhus, as though he were to see a corpse thrown aside in a charnel
ground, one, two, or three days dead, bloated, livid, and oozing matter, a
bhikkhu compares this same body with it thus: 'This body too is of the same
nature, it will be like that, it is not exempt from that fate.'

\paliPage
\section*{Dhātu-manasikārapabba}

Pu̮na̮ ca̮pa̮ra̱ṁ, bhi̱kkha̮ve̱, bhi̱kkhu̮ i̮ma̮me̱va̮ kā̱ya̱ṁ ya̮thā̱꜔꜒ṭhi̮ta̱ṁ ya̮thā̱꜔꜒pa̮ṇi̮hi̮ta͓ṁ
dhā̱tu̮so̱꜔꜒ pa̱cca̮ve̱kkha̮ti̮: ‘a̱tthi̮ i̮ma̱smi̱꜔꜒ṁ kā̱ye̱ pa̮tha̮vī̱dhā̱tu̮ ā̱po̱dhā̱tu̮ te̱jo̱dhā̱tu͓
vā̱yo̱dhā̱tū̱’ti͓.

Se̱꜔꜒yya̮thā̱꜔꜒pi̮, bhi̱kkha̮ve̱, da̱kkho̱꜔꜒ go̱ghā̱ta̮ko̱ vā̱ go̱ghā̱ta̮ka̱nte̱vā̱sī̱꜔꜒ vā̱ gā̱vi̱ṁ va̮dhi̱tvā͓
cā̱tu̱mma̮hā̱꜔꜒pa̮the̱꜔꜒ bi̮la̮so̱꜔꜒ vi̮bha̮ji̱tvā̱ ni̮si̱꜔꜒nno̱ a̱ssa̮; e̱va̮me̱va̮ kho̱꜔꜒, bhi̱kkha̮ve̱, bhi̱kkhu͓
i̮ma̮me̱va̮ kā̱ya̱ṁ ya̮thā̱꜔꜒ṭhi̮ta̱ṁ ya̮thā̱꜔꜒pa̮ṇi̮hi̮ta̱ṁ dhā̱tu̮so̱꜔꜒ pa̱cca̮ve̱kkha̮ti̮: ‘a̱tthi̮ i̮ma̱smi͓꜔꜒ṁ
kā̱ye̱ pa̮tha̮vī̱dhā̱tu̮ ā̱po̱dhā̱tu̮ te̱jo̱dhā̱tu̮ vā̱yo̱dhā̱tū̱’ti͓.

I̮ti̮ a̱jjha̱tta̱ṁ vā̱ kā̱ye̱ kā̱yā̱nu̮pa̱ssī̱꜔꜒ vi̮ha̮ra̮ti̮, ba̮hi̱ddhā̱ vā̱ kā̱ye̱ kā̱yā̱nu̮pa̱ssī͓꜔꜒
vi̮ha̮ra̮ti̮, a̱jjha̱tta̮-ba̮hi̱ddhā̱ vā̱ kā̱ye̱ kā̱yā̱nu̮pa̱ssī̱꜔꜒ vi̮ha̮ra̮ti̮. Sa̮mu̮da̮ya̮-dha̱mmā̱nu̮pa̱ssī͓꜔꜒
vā̱ kā̱ya̱smi̱꜔꜒ṁ vi̮ha̮ra̮ti̮, va̮ya̮-dha̱mmā͓-\\
nu̮pa̱ssī̱꜔꜒ vā̱ kā̱ya̱smi̱꜔꜒ṁ vi̮ha̮ra̮ti̮, sa̮mu̮da̮ya̮-va̮ya̮-dha̱mmā̱nu̮pa̱ssī̱꜔꜒ vā̱ kā̱ya̱smi̱꜔꜒ṁ vi̮ha̮ra̮ti͓.
‘A̱tthi̮ kā̱yo̱’ti̮ vā̱ pa̮na̱ssa̮ sa̮ti̮ pa̱ccu̮pa̱ṭṭhi̮tā̱ ho̱꜔꜒ti̮ yā̱va̮de̱va̮ ñā̱ṇa̮ma̱ttā̱ya͓
pa̮ṭi̱ssa̮ti̮ma̱ttā̱ya̮ a̮ni̱ssi̮to̱ ca̮ vi̮ha̮ra̮ti̮, na̮ ca̮ ki̱ñci̮ lo̱ke̱ u̮pā̱di̮ya̮ti̮. E̱va̱mpi̮ kho͓꜔꜒,
bhi̱kkha̮ve̱, bhi̱kkhu̮ kā̱ye̱ kā̱yā̱nu̮pa̱ssī̱꜔꜒ vi̮ha̮ra̮ti͓.

\instr{Dhātu-manasikārapabbaṁ niṭṭhitaṁ.}

\section*{Navasivathikapabba}

[1] Pu̮na̮ ca̮pa̮ra̱ṁ, bhi̱kkha̮ve̱, bhi̱kkhu̮ se̱꜔꜒yya̮thā̱꜔꜒pi̮ pa̱sse̱꜔꜒yya̮ sa̮rī̱ra̱ṁ si̮va̮thi̮kā̱ya͓
cha̱ḍḍi̮ta̱ṁ e̱kā̱ha̮ma̮ta̱ṁ vā̱ dvī̱ha̮ma̮ta̱ṁ vā̱ tī̱ha̮ma̮ta̱ṁ vā̱ u̱ddhu̮mā̱ta̮ka̱ṁ vi̮nī̱la̮ka͓ṁ
vi̮pu̱bba̮ka̮jā̱ta̱ṁ. So̱꜔꜒ i̮ma̮me̱va̮ kā̱ya̱ṁ u̮pa̮sa̱꜔꜒ṁha̮ra̮ti̮: ‘a̮ya̱mpi̮ kho̱꜔꜒ kā̱yo̱ e̱va̱ṁ-dha̱mmo͓
e̱va̱ṁ-bhā̱vī̱ e̱va̮ṁ-a̮na̮tī̱to̱’ti͓.

\englishPage

In this way he dwells contemplating the body in the body internally, or he
dwells contemplating the body in the body externally, or he dwells contemplating
the body in the body both internally and externally. Or else he dwells
contemplating in the body its nature of arising, or he dwells contemplating in
the body its nature of vanishing, or he dwells contemplating in the body its
nature of both arising and vanishing. Or else mindfulness that ‘there is a body’
is simply established in him to the extent necessary for bare knowledge and
repeated mindfulness.

And he dwells independent, not clinging to anything in the world. That is how,
bhikkhus, a bhikkhu dwells contemplating the body in the body.

[2] Again, bhikkhus, as though he were to see a corpse thrown aside in a charnel
ground, being devoured by crows, being devoured by vultures, being devoured by
hawks, being devoured by dogs, being devoured by jackals, or being devoured by
various kinds of worms, a bhikkhu compares this same body with it thus: 'This
body too is of the same nature, it will be like that, it is not exempt from that
fate.'

In this way he dwells contemplating the body in the body internally, or he
dwells contemplating the body in the body externally, or he dwells contemplating
the body in the body both internally and externally. Or else he dwells
contemplating in the body its nature of arising, or he dwells contemplating in
the body its nature of vanishing, or he dwells contemplating in the body its
nature of both arising and vanishing. Or else mindfulness that ‘there is a body’
is simply established in him to the extent necessary for bare knowledge and
repeated mindfulness.

And he dwells independent, not clinging to anything in the world. That is how,
bhikkhus, a bhikkhu dwells contemplating the body in the body.

\paliPage

I̮ti̮ a̱jjha̱tta̱ṁ vā̱ kā̱ye̱ kā̱yā̱nu̮pa̱ssī̱꜔꜒ vi̮ha̮ra̮ti̮, ba̮hi̱ddhā̱ vā̱ kā̱ye̱ kā̱yā̱nu̮pa̱ssī͓꜔꜒
vi̮ha̮ra̮ti̮, a̱jjha̱tta̮-ba̮hi̱ddhā̱ vā̱ kā̱ye̱ kā̱yā̱nu̮pa̱ssī̱꜔꜒ vi̮ha̮ra̮ti̮. Sa̮mu̮da̮ya̮-dha̱mmā̱nu̮pa̱ssī͓꜔꜒
vā̱ kā̱ya̱smi̱꜔꜒ṁ vi̮ha̮ra̮ti̮, va̮ya̮-dha̱mmā͓-\\
nu̮pa̱ssī̱꜔꜒ vā̱ kā̱ya̱smi̱꜔꜒ṁ vi̮ha̮ra̮ti̮, sa̮mu̮da̮ya̮-va̮ya̮-dha̱mmā̱nu̮pa̱ssī̱꜔꜒ vā̱ kā̱ya̱smi̱꜔꜒ṁ vi̮ha̮ra̮ti͓.
‘A̱tthi̮ kā̱yo̱’ti̮ vā̱ pa̮na̱ssa̮ sa̮ti̮ pa̱ccu̮pa̱ṭṭhi̮tā̱ ho̱꜔꜒ti̮ yā̱va̮de̱va̮ ñā̱ṇa̮ma̱ttā̱ya͓
pa̮ṭi̱ssa̮ti̮ma̱ttā̱ya̮ a̮ni̱ssi̮to̱ ca̮ vi̮ha̮ra̮ti̮, na̮ ca̮ ki̱ñci̮ lo̱ke̱ u̮pā̱di̮ya̮ti̮. E̱va̱mpi̮ kho͓꜔꜒,
bhi̱kkha̮ve̱, bhi̱kkhu̮ kā̱ye̱ kā̱yā̱nu̮pa̱ssī̱꜔꜒ vi̮ha̮ra̮ti͓.

[2] Pu̮na̮ ca̮pa̮ra̱ṁ, bhi̱kkha̮ve̱, bhi̱kkhu̮ se̱꜔꜒yya̮thā̱꜔꜒pi̮ pa̱sse̱꜔꜒yya̮ sa̮rī̱ra̱ṁ si̮va̮thi̮kā̱ya͓
cha̱ḍḍi̮ta̱ṁ kā̱ke̱hi̮ vā̱ kha̱jja̮mā̱na̱ṁ ku̮la̮le̱hi̮ vā̱ kha̱jja̮mā̱na̱ṁ gi̱jjhe̱hi̮ vā̱ kha̱jja̮mā̱na͓ṁ
ka̱ṅke̱hi̮ vā̱ kha̱jja̮mā̱na̱ṁ su̮na̮khe̱꜔꜒hi̮ vā̱ kha̱jja̮mā̱na̱ṁ bya̱gghe̱hi̮ vā̱ kha̱jja̮mā̱na̱ṁ dī̱pī̱hi͓
vā̱ kha̱jja̮mā̱na̱ṁ si̱꜔꜒ṅgā̱le̱hi̮ vā̱ kha̱jja̮mā̱na̱ṁ vi̮vi̮dhe̱hi̮ vā̱ pā̱ṇa̮ka̮jā̱te̱hi̮ kha̱jja̮mā̱na͓ṁ.
So̱꜔꜒ i̮ma̮me̱va̮ kā̱ya̱ṁ u̮pa̮sa̱꜔꜒ṁha̮ra̮ti̮: ‘a̮ya̱mpi̮ kho̱꜔꜒ kā̱yo̱ e̱va̱ṁ-dha̱mmo̱ e̱va̱ṁ-bhā̱vī͓
e̱va̮ṁ-a̮na̮tī̱to̱’ti͓.

I̮ti̮ a̱jjha̱tta̱ṁ vā̱ kā̱ye̱ kā̱yā̱nu̮pa̱ssī̱꜔꜒ vi̮ha̮ra̮ti̮, ba̮hi̱ddhā̱ vā̱ kā̱ye̱ kā̱yā̱nu̮pa̱ssī͓꜔꜒
vi̮ha̮ra̮ti̮, a̱jjha̱tta̮-ba̮hi̱ddhā̱ vā̱ kā̱ye̱ kā̱yā̱nu̮pa̱ssī̱꜔꜒ vi̮ha̮ra̮ti̮. Sa̮mu̮da̮ya̮-dha̱mmā̱nu̮pa̱ssī͓꜔꜒
vā̱ kā̱ya̱smi̱꜔꜒ṁ vi̮ha̮ra̮ti̮, va̮ya̮-dha̱mmā͓-\\
nu̮pa̱ssī̱꜔꜒ vā̱ kā̱ya̱smi̱꜔꜒ṁ vi̮ha̮ra̮ti̮, sa̮mu̮da̮ya̮-va̮ya̮-dha̱mmā̱nu̮pa̱ssī̱꜔꜒ vā̱ kā̱ya̱smi̱꜔꜒ṁ vi̮ha̮ra̮ti͓.
‘A̱tthi̮ kā̱yo̱’ti̮ vā̱ pa̮na̱ssa̮ sa̮ti̮ pa̱ccu̮pa̱ṭṭhi̮tā̱ ho̱꜔꜒ti̮ yā̱va̮de̱va̮ ñā̱ṇa̮ma̱ttā̱ya͓
pa̮ṭi̱ssa̮ti̮ma̱ttā̱ya̮ a̮ni̱ssi̮to̱ ca̮ vi̮ha̮ra̮ti̮, na̮ ca̮ ki̱ñci̮ lo̱ke̱ u̮pā̱di̮ya̮ti̮. E̱va̱mpi̮ kho͓꜔꜒,
bhi̱kkha̮ve̱, bhi̱kkhu̮ kā̱ye̱ kā̱yā̱nu̮pa̱ssī̱꜔꜒ vi̮ha̮ra̮ti͓.

\englishPage

[3]~Again, bhikkhus, as though he were to see a corpse thrown aside in a
charnel ground, a skeleton with flesh and blood, held together with sinews~\ldots{}

[4]~a fleshless skeleton smeared with blood, held together with sinews~\ldots{}

[5]~a skeleton without flesh and blood, held together with sinews~\ldots{}

[6]~disconnected bones not held together with sinews scattered in all directions
-- here a hand-bone, there a foot bone, here a shin-bone, there a thigh-bone,
here a hip-bone, there a back-bone, here a rib-bone, there a chest-bone, here an
arm-bone, there a shoulder-bone, here a neck-bone, there a jaw-bone, here a
tooth-bone, there the skull -- a bhikkhu compares this same body with it thus:
`This body too is of the same nature, it will be like that, it is not exempt
from that fate.'

In this way he dwells contemplating the body in the body internally, or he
dwells contemplating the body in the body externally, or he dwells contemplating
the body in the body both internally and externally. Or else he dwells
contemplating in the body its nature of arising, or he dwells contemplating in
the body its nature of vanishing, or he dwells contemplating in the body its
nature of both arising and vanishing. Or else mindfulness that ‘there is a body’
is simply established in him to the extent necessary for bare knowledge and
repeated mindfulness.

And he dwells independent, not clinging to anything in the world. That is how,
bhikkhus, a bhikkhu dwells contemplating the body in the body.

[7] Again, bhikkhus, as though he were to see a corpse thrown aside in a charnel
ground, bones bleached white, the colour of shells~\ldots{}

[8]~bones heaped up, more than a year old~\ldots{}

\paliPage

[3]~Pu̮na̮ ca̮pa̮ra̱ṁ, bhi̱kkha̮ve̱, bhi̱kkhu̮ se̱꜔꜒yya̮thā̱꜔꜒pi̮ pa̱sse̱꜔꜒yya̮ sa̮rī̱ra̱ṁ si̮va̮thi̮kā̱ya͓
cha̱ḍḍi̮ta̱ṁ a̱ṭṭhi̮ka̮-sa̱꜔꜒ṅkha̮li̮ka̱ṁ sa̮ma̱ṁsa̮-lo̱hi̮ta̱ṁ na̮hā̱꜔꜒ru̮-sa̱꜔꜒mba̱ndha̱ṁ~\ldo͓ts{}

[4]~A̱ṭṭhi̮ka̮-sa̱꜔꜒ṅkha̮li̮ka̱ṁ ni̮ma̱ṁsa̮-lo̱hi̮ta̮-ma̱kkhi̮ta̱ṁ na̮hā̱꜔꜒ru̮-sa̱꜔꜒mba̱ndha̱ṁ~\ldo͓ts{}

[5]~A̱ṭṭhi̮ka̮-sa̱꜔꜒ṅkha̮li̮ka̱ṁ a̮pa̮ga̮ta̮-ma̱ṁsa̮-lo̱hi̮ta̱ṁ na̮hā̱꜔꜒ru̮-sa̱꜔꜒mba̱ndha̱ṁ~\ldo͓ts{}

[6]~A̱ṭṭhi̮kā̱ni̮ a̮pa̮ga̮ta̮-sa̱꜔꜒mba̱ndhā̱ni̮ di̮sā̱꜔꜒ vi̮di̮sā̱꜔꜒ vi̱kkhi̱ttā̱ni̮, a̱ññe̱na̮ ha̱tth'a̱ṭṭhi̮ka͓ṁ
a̱ññe̱na̮ pā̱d'a̱ṭṭhi̮ka̱ṁ a̱ññe̱na̮ go̱ppha̮k'a̱ṭṭhi̮ka̱ṁ a̱ññe̱na̮ ja̱ṅgh'a̱ṭṭhi̮ka̱ṁ a̱ññe̱na̮ ū̱r'u̱ṭṭhi̮ka͓ṁ
a̱ññe̱na̮ ka̮ṭ'i̱ṭṭhi̮ka̱ṁ a̱ññe̱na̮ phā̱꜔꜒su̮k'a̱ṭṭhi̮ka̱ṁ a̱ññe̱na̮ pi̱ṭṭh'i̱ṭṭhi̮ka̱ṁ a̱ññe̱na͓
kha̱꜔꜒ndh'a̱ṭṭhi̮ka̱ṁ a̱ññe̱na̮ gī̱v'a̱ṭṭhi̮ka̱ṁ a̱ññe̱na̮ ha̮nu̮k'a̱ṭṭhi̮ka̱ṁ a̱ññe̱na̮ da̱nt'a̱ṭṭhi̮ka͓ṁ
a̱ññe̱na̮ sī̱꜔꜒sa̮ka̮ṭā̱ha̱꜔꜒ṁ. So̱꜔꜒ i̮ma̮me̱va̮ kā̱ya̱ṁ u̮pa̮sa̱꜔꜒ṁha̮ra̮ti̮: ‘a̮ya̱mpi̮ kho̱꜔꜒ kā̱yo̱ e̱va̱ṁdha̱mmo͓
e̱va̱ṁbhā̱vī̱ e̱va̮ṁa̮na̮tī̱to̱’ti͓.

I̮ti̮ a̱jjha̱tta̱ṁ vā̱ kā̱ye̱ kā̱yā̱nu̮pa̱ssī̱꜔꜒ vi̮ha̮ra̮ti̮, ba̮hi̱ddhā̱ vā̱ kā̱ye̱ kā̱yā̱nu̮pa̱ssī͓꜔꜒
vi̮ha̮ra̮ti̮, a̱jjha̱tta̮-ba̮hi̱ddhā̱ vā̱ kā̱ye̱ kā̱yā̱nu̮pa̱ssī̱꜔꜒ vi̮ha̮ra̮ti̮. Sa̮mu̮da̮ya̮-dha̱mmā̱nu̮pa̱ssī͓꜔꜒
vā̱ kā̱ya̱smi̱꜔꜒ṁ vi̮ha̮ra̮ti̮, va̮ya̮-dha̱mmā͓-\\
nu̮pa̱ssī̱꜔꜒ vā̱ kā̱ya̱smi̱꜔꜒ṁ vi̮ha̮ra̮ti̮, sa̮mu̮da̮ya̮-va̮ya̮-dha̱mmā̱nu̮pa̱ssī̱꜔꜒ vā̱ kā̱ya̱smi̱꜔꜒ṁ vi̮ha̮ra̮ti͓.
‘A̱tthi̮ kā̱yo̱’ti̮ vā̱ pa̮na̱ssa̮ sa̮ti̮ pa̱ccu̮pa̱ṭṭhi̮tā̱ ho̱꜔꜒ti̮ yā̱va̮de̱va̮ ñā̱ṇa̮ma̱ttā̱ya͓
pa̮ṭi̱ssa̮ti̮ma̱ttā̱ya̮ a̮ni̱ssi̮to̱ ca̮ vi̮ha̮ra̮ti̮, na̮ ca̮ ki̱ñci̮ lo̱ke̱ u̮pā̱di̮ya̮ti̮. E̱va̱mpi̮ kho͓꜔꜒,
bhi̱kkha̮ve̱, bhi̱kkhu̮ kā̱ye̱ kā̱yā̱nu̮pa̱ssī̱꜔꜒ vi̮ha̮ra̮ti͓.

[7]~Pu̮na̮ ca̮pa̮ra̱ṁ, bhi̱kkha̮ve̱, bhi̱kkhu̮ se̱꜔꜒yya̮thā̱꜔꜒pi̮ pa̱sse̱꜔꜒yya̮ sa̮rī̱ra̱ṁ si̮va̮thi̮kā̱ya͓
cha̱ḍḍi̮ta̱ṁ a̱ṭṭhi̮kā̱ni̮ se̱꜔꜒tā̱ni̮ sa̱꜔꜒ṅkha̮-va̱ṇṇa̮-pa̮ṭi̮bhā̱gā̱ni̱~\ldo͓ts{}

[8]~A̱ṭṭhi̮kā̱ni̮ pu̱ñja̮ki̮tā̱ni̮ te̱ro̱va̱ssi̮kā̱ni̱~\ldo͓ts{}

\englishPage

[9]~bones rotted and crumbled to dust, a bhikkhu compares this same
body with it thus: ‘This body too is of the same nature, it will be like that,
it is not exempt from that fate.’

In this way he dwells contemplating the body in the body internally, or he
dwells contemplating the body in the body externally, or he dwells contemplating
the body in the body both internally and externally. Or else he dwells
contemplating in the body its nature of arising, or he dwells contemplating in
the body its nature of vanishing, or he dwells contemplating in the body its
nature of both arising and vanishing. Or else mindfulness that ‘there is a body’
is simply established in him to the extent necessary for bare knowledge and
repeated mindfulness.

And he dwells independent, not clinging to anything in the world. That is how,
bhikkhus, a bhikkhu dwells contemplating the body in the body.

\instr{The section on the Nine Charnel Ground Contemplations is finished.}

\instr{Contemplation of the Body is finished.}

\paliPage

[9]~A̱ṭṭhi̮kā̱ni̮ pū̱tī̱ni̮ cu̱ṇṇa̮ka̮jā̱tā̱ni̮. So̱꜔꜒ i̮ma̮me̱va̮ kā̱ya̱ṁ u̮pa̮sa̱꜔꜒ṁha̮ra̮ti̮: ‘a̮ya̱mpi̮ kho͓꜔꜒
kā̱yo̱ e̱va̱ṁ-dha̱mmo̱ e̱va̱ṁ-bhā̱vī̱ e̱va̮ṁ-a̮na̮tī̱to̱’ti͓.

I̮ti̮ a̱jjha̱tta̱ṁ vā̱ kā̱ye̱ kā̱yā̱nu̮pa̱ssī̱꜔꜒ vi̮ha̮ra̮ti̮, ba̮hi̱ddhā̱ vā̱ kā̱ye̱ kā̱yā̱nu̮pa̱ssī͓꜔꜒
vi̮ha̮ra̮ti̮, a̱jjha̱tta̮-ba̮hi̱ddhā̱ vā̱ kā̱ye̱ kā̱yā̱nu̮pa̱ssī̱꜔꜒ vi̮ha̮ra̮ti̮. Sa̮mu̮da̮ya̮-dha̱mmā̱nu̮pa̱ssī͓꜔꜒
vā̱ kā̱ya̱smi̱꜔꜒ṁ vi̮ha̮ra̮ti̮, va̮ya̮-dha̱mmā͓-\\
nu̮pa̱ssī̱꜔꜒ vā̱ kā̱ya̱smi̱꜔꜒ṁ vi̮ha̮ra̮ti̮, sa̮mu̮da̮ya̮-va̮ya̮-dha̱mmā̱nu̮pa̱ssī̱꜔꜒ vā̱ kā̱ya̱smi̱꜔꜒ṁ vi̮ha̮ra̮ti͓.
‘A̱tthi̮ kā̱yo̱’ti̮ vā̱ pa̮na̱ssa̮ sa̮ti̮ pa̱ccu̮pa̱ṭṭhi̮tā̱ ho̱꜔꜒ti̮ yā̱va̮de̱va̮ ñā̱ṇa̮ma̱ttā̱ya͓
pa̮ṭi̱ssa̮ti̮ma̱ttā̱ya̮ a̮ni̱ssi̮to̱ ca̮ vi̮ha̮ra̮ti̮, na̮ ca̮ ki̱ñci̮ lo̱ke̱ u̮pā̱di̮ya̮ti̮. E̱va̱mpi̮ kho͓꜔꜒,
bhi̱kkha̮ve̱, bhi̱kkhu̮ kā̱ye̱ kā̱yā̱nu̮pa̱ssī̱꜔꜒ vi̮ha̮ra̮ti͓.

\instr{Navasivathikapabbaṁ niṭṭhitaṁ.}

\instr{Kāyānupassanā niṭṭhitā.}

\englishPage
\chapter{Contemplation of Feelings}

And how, bhikkhus, does a bhikkhu dwell contemplating feelings in feelings?

Here, bhikkhus, when feeling a pleasant feeling, a bhikkhu understands:
`I~feel a pleasant feeling';
when feeling a painful feeling, he understands:
`I~feel a painful feeling';
when feeling a neither-painful-nor-pleasant feeling, he understands:
`I~feel a neither-painful-nor-pleasant feeling.'

When feeling a carnal pleasant feeling, he understands:
`I~feel a carnal pleasant feeling';
when feeling a spiritual pleasant feeling, he understands:
`I~feel a spiritual pleasant feeling';
when feeling a carnal painful feeling, he understands:
`I~feel a carnal painful feeling';
when feeling a spiritual painful feeling, he understands:
`I~feel a spiritual painful feeling';
when feeling a carnal neither-painful-nor-pleasant feeling, he understands:
`I~feel a carnal neither-painful-nor-pleasant feeling';
when feeling a spiritual neither-painful-nor-pleasant feeling, he understands:
`I~feel a spiritual neither-painful-nor-pleasant feeling.'

\paliPage
\chapter*{Vedanānupassanā}

Ka̮tha̱꜔꜒ñca̮ pa̮na̮, bhi̱kkha̮ve̱, bhi̱kkhu̮ ve̱da̮nā̱su̮ ve̱da̮nā̱nu̮pa̱ssī̱꜔꜒ vi̮ha̮ra̮ti͓?

Idha, bhikkhave, bhikkhu\\
su̮kha̱꜔꜒ṁ vā̱ ve̱da̮na̱ṁ ve̱da̮ya̮mā̱no͓\\
‘su̮kha̱꜔꜒ṁ ve̱da̮na̱ṁ ve̱da̮yā̱mī̱’ti̮ pa̮jā̱nā̱ti͓.\\
Du̱kkha̱꜔꜒ṁ vā̱ ve̱da̮na̱ṁ ve̱da̮ya̮mā̱no͓\\
‘du̱kkha̱꜔꜒ṁ ve̱da̮na̱ṁ ve̱da̮yā̱mī̱’ti̮ pa̮jā̱nā̱ti͓.\\
A̮du̱kkha̮ma̮su̮kha̱꜔꜒ṁ vā̱ ve̱da̮na̱ṁ ve̱da̮ya̮mā̱no͓\\
‘a̮du̱kkha̮ma̮su̮kha̱꜔꜒ṁ ve̱da̮na̱ṁ ve̱da̮yā̱mī̱’ti̮ pa̮jā̱nā̱ti͓.

Sā̱꜔꜒mi̮sa̱꜔꜒ṁ vā̱ su̮kha̱꜔꜒ṁ ve̱da̮na̱ṁ ve̱da̮ya̮mā̱no͓\\
‘sā̱꜔꜒mi̮sa̱꜔꜒ṁ su̮kha̱꜔꜒ṁ ve̱da̮na̱ṁ ve̱da̮yā̱mī̱’ti̮ pa̮jā̱nā̱ti͓.\\
Ni̮rā̱mi̮sa̱꜔꜒ṁ vā̱ su̮kha̱꜔꜒ṁ ve̱da̮na̱ṁ ve̱da̮ya̮mā̱no͓\\
‘ni̮rā̱mi̮sa̱꜔꜒ṁ su̮kha̱꜔꜒ṁ ve̱da̮na̱ṁ ve̱da̮yā̱mī̱’ti̮ pa̮jā̱nā̱ti͓.

Sā̱꜔꜒mi̮sa̱꜔꜒ṁ vā̱ du̱kkha̱꜔꜒ṁ ve̱da̮na̱ṁ ve̱da̮ya̮mā̱no͓\\
‘sā̱꜔꜒mi̮sa̱꜔꜒ṁ du̱kkha̱꜔꜒ṁ ve̱da̮na̱ṁ ve̱da̮yā̱mī̱’ti̮ pa̮jā̱nā̱ti͓.\\
Ni̮rā̱mi̮sa̱꜔꜒ṁ vā̱ du̱kkha̱꜔꜒ṁ ve̱da̮na̱ṁ ve̱da̮ya̮mā̱no͓\\
‘ni̮rā̱mi̮sa̱꜔꜒ṁ du̱kkha̱꜔꜒ṁ ve̱da̮na̱ṁ ve̱da̮yā̱mī̱’ti̮ pa̮jā̱nā̱ti͓.

Sā̱꜔꜒mi̮sa̱꜔꜒ṁ vā̱ a̮du̱kkha̮ma̮su̮kha̱꜔꜒ṁ ve̱da̮na̱ṁ ve̱da̮ya̮mā̱no͓\\
‘sā̱꜔꜒mi̮sa̱꜔꜒ṁ a̮du̱kkha̮ma̮su̮kha̱꜔꜒ṁ ve̱da̮na̱ṁ ve̱da̮yā̱mī̱’ti̮ pa̮jā̱nā̱ti͓.\\
Ni̮rā̱mi̮sa̱꜔꜒ṁ vā̱ a̮du̱kkha̮ma̮su̮kha̱꜔꜒ṁ ve̱da̮na̱ṁ ve̱da̮ya̮mā̱no͓\\
‘ni̮rā̱mi̮sa̱꜔꜒ṁ a̮du̱kkha̮ma̮su̮kha̱꜔꜒ṁ ve̱da̮na̱ṁ ve̱da̮yā̱mī̱’ti̮ pa̮jā̱nā̱ti͓.

\englishPage

In this way he dwells contemplating feelings in feelings internally, or he
dwells contemplating feelings in feelings externally, or he dwells contemplating
feelings in feelings both internally and externally. Or else he dwells
contemplating in feelings their nature of arising, or he dwells contemplating in
feelings their nature of vanishing, or he dwells contemplating in feelings their
nature of both arising and vanishing. Or else mindfulness that ‘there is
feeling’ is simply established in him to the extent necessary for bare knowledge
and repeated mindfulness.

And he dwells independent, not clinging to anything in the world. That is how,
bhikkhus, a bhikkhu dwells contemplating feelings in feelings.

\instr{The Contemplation of Feelings is finished.}

\chapter{Contemplation of Mind}

And how, bhikkhus, does a bhikkhu dwell contemplating mind in mind?

Here, bhikkhus, a bhikkhu\\
understands a mind with lust as a mind with lust,\\
and a mind without lust as a mind without lust.\\
He understands a mind with hatred as a mind with hatred,\\
and a mind without hatred as a mind without hatred.\\
He understands a mind with delusion as a mind with delusion,\\
and a mind without delusion as a mind without delusion.\\
He understands a contracted mind as contracted,\\
and a distracted mind as distracted.\\
He understands an exalted mind as exalted,\\
and an unexalted mind as unexalted.

\paliPage

I̮ti̮ a̱jjha̱tta̱ṁ vā̱ ve̱da̮nā̱su̮ ve̱da̮nā̱nu̮pa̱ssī̱꜔꜒ vi̮ha̮ra̮ti̮, ba̮hi̱ddhā̱ vā̱ ve̱da̮nā̱su͓
ve̱da̮nā̱nu̮pa̱ssī̱꜔꜒ vi̮ha̮ra̮ti̮, a̱jjha̱tta̮-ba̮hi̱ddhā̱ vā̱ ve̱da̮nā̱su̮ ve̱da̮nā̱nu̮pa̱ssī̱꜔꜒ vi̮ha̮ra̮ti͓.
Sa̮mu̮da̮ya̮-dha̱mmā̱nu̮pa̱ssī̱꜔꜒ vā̱ ve̱da̮nā̱su̮ vi̮ha̮ra̮ti̮, va̮ya̮-dha̱mmā̱nu̮pa̱ssī̱꜔꜒ vā̱ ve̱da̮nā̱su͓
viharati, samudaya-vaya-\\
dha̱mmā̱nu̮pa̱ssī̱꜔꜒ vā̱ ve̱da̮nā̱su̮ vi̮ha̮ra̮ti̮. ‘A̱tthi̮ ve̱da̮nā̱’ti̮ vā̱ pa̮na̱ssa̮ sa̮ti͓
pa̱ccu̮pa̱ṭṭhi̮tā̱ ho̱꜔꜒ti̮ yā̱va̮de̱va̮ ñā̱ṇa̮ma̱ttā̱ya̮ pa̮ṭi̱ssa̮ti̮ma̱ttā̱ya̮ a̮ni̱ssi̮to̱ ca̮ vi̮ha̮ra̮ti͓,
na̮ ca̮ ki̱ñci̮ lo̱ke̱ u̮pā̱di̮ya̮ti̮. E̱va̱mpi̮ kho̱꜔꜒, bhi̱kkha̮ve̱, bhi̱kkhu̮ ve̱da̮nā̱su͓
ve̱da̮nā̱nu̮pa̱ssī̱꜔꜒ vi̮ha̮ra̮ti͓.

\instr{Vedanānupassanā niṭṭhitā.}

\chapter*{Cittānupassanā}

Ka̮tha̱꜔꜒ñca̮ pa̮na̮, bhi̱kkha̮ve̱, bhi̱kkhu̮ ci̱tte̱ ci̱ttā̱nu̮pa̱ssī̱꜔꜒ vi̮ha̮ra̮ti͓?

Idha, bhikkhave, bhikkhu\\
sa̮rā̱ga̱ṁ vā̱ ci̱tta̱ṁ ‘sa̮rā̱ga̱ṁ ci̱tta̱n’ti̮ pa̮jā̱nā̱ti͓.\\
Vī̱ta̮rā̱ga̱ṁ vā̱ ci̱tta̱ṁ ‘vī̱ta̮rā̱ga̱ṁ ci̱tta̱n’ti̮ pa̮jā̱nā̱ti͓.\\
Sa̮do̱sa̱꜔꜒ṁ vā̱ ci̱tta̱ṁ ‘sa̮do̱sa̱꜔꜒ṁ ci̱tta̱n’ti̮ pa̮jā̱nā̱ti͓.\\
Vī̱ta̮do̱sa̱꜔꜒ṁ vā̱ ci̱tta̱ṁ ‘vī̱ta̮do̱sa̱꜔꜒ṁ ci̱tta̱n’ti̮ pa̮jā̱nā̱ti͓.\\
Sa̮mo̱ha̱꜔꜒ṁ vā̱ ci̱tta̱ṁ ‘sa̮mo̱ha̱꜔꜒ṁ ci̱tta̱n’ti̮ pa̮jā̱nā̱ti͓.\\
Vī̱ta̮mo̱ha̱꜔꜒ṁ vā̱ ci̱tta̱ṁ ‘vī̱ta̮mo̱ha̱꜔꜒ṁ ci̱tta̱n’ti̮ pa̮jā̱nā̱ti͓.\\
Sa̱꜔꜒ṅkhi̱tta̱ṁ vā̱ ci̱tta̱ṁ ‘sa̱꜔꜒ṅkhi̱tta̱ṁ ci̱tta̱n’ti̮ pa̮jā̱nā̱ti͓.\\
Vi̱kkhi̱tta̱ṁ vā̱ ci̱tta̱ṁ ‘vi̱kkhi̱tta̱ṁ ci̱tta̱n’ti̮ pa̮jā̱nā̱ti͓.\\
Ma̮ha̱gga̮ta̱ṁ vā̱ ci̱tta̱ṁ ‘ma̮ha̱gga̮ta̱ṁ ci̱tta̱n’ti̮ pa̮jā̱nā̱ti͓.\\
A̮ma̮ha̱gga̮ta̱ṁ vā̱ ci̱tta̱ṁ ‘a̮ma̮ha̱gga̮ta̱ṁ ci̱tta̱n’ti̮ pa̮jā̱nā̱ti͓.

\englishPage

He understands a surpassable mind as surpassable,\\
and an unsurpassable mind as unsurpassable.\\
He understands a concentrated mind as concentrated,\\
and an unconcentrated mind as unconcentrated.\\
He understands a liberated mind as liberated,\\
and an unliberated mind as unliberated.

In this way he dwells contemplating mind in mind internally, or he dwells
contemplating mind in mind externally, or he dwells contemplating mind in mind
both internally and externally. Or else he dwells contemplating in mind its
nature of arising, or he dwells contemplating in mind its nature of vanishing,
or he dwells contemplating in mind its nature of both arising and vanishing. Or
else mindfulness that ‘there is mind’ is simply established in him to the extent
necessary for bare knowledge and repeated mindfulness.

And he dwells independent, not clinging to anything in the world. That is how,
bhikkhus, a bhikkhu dwells contemplating mind in mind.

\instr{The Contemplation of Mind is finished.}

\paliPage

Sa̱u̱tta̮ra̱ṁ vā̱ ci̱tta̱ṁ ‘sa̱u̱tta̮ra̱ṁ ci̱tta̱n’ti̮ pa̮jā̱nā̱ti͓.\\
A̮nu̱tta̮ra̱ṁ vā̱ ci̱tta̱ṁ ‘a̮nu̱tta̮ra̱ṁ ci̱tta̱n’ti̮ pa̮jā̱nā̱ti͓.\\
Sa̮mā̱hi̮ta̱ṁ vā̱ ci̱tta̱ṁ ‘sa̮mā̱hi̮ta̱ṁ ci̱tta̱n’ti̮ pa̮jā̱nā̱ti͓.\\
A̮sa̮mā̱hi̮ta̱ṁ vā̱ ci̱tta̱ṁ ‘a̮sa̮mā̱hi̮ta̱ṁ ci̱tta̱n’ti̮ pa̮jā̱nā̱ti͓.\\
Vi̮mu̱tta̱ṁ vā̱ ci̱tta̱ṁ ‘vi̮mu̱tta̱ṁ ci̱tta̱n’ti̮ pa̮jā̱nā̱ti͓.\\
A̮vi̮mu̱tta̱ṁ vā̱ ci̱tta̱ṁ ‘a̮vi̮mu̱tta̱ṁ ci̱tta̱n’ti̮ pa̮jā̱nā̱ti͓.

I̮ti̮ a̱jjha̱tta̱ṁ vā̱ ci̱tte̱ ci̱ttā̱nu̮pa̱ssī̱꜔꜒ vi̮ha̮ra̮ti̮, ba̮hi̱ddhā̱ vā̱ ci̱tte̱ ci̱ttā̱nu̮pa̱ssī͓꜔꜒
vi̮ha̮ra̮ti̮, a̱jjha̱tta̮-ba̮hi̱ddhā̱ vā̱ ci̱tte̱ ci̱ttā̱nu̮pa̱ssī̱꜔꜒ vi̮ha̮ra̮ti͓.
Sa̮mu̮da̮ya̮-dha̱mmā̱nu̮pa̱ssī̱꜔꜒ vā̱ ci̱tta̱smi̱꜔꜒ṁ vi̮ha̮ra̮ti̮, va̮ya̮-dha̱mmā͓-\\
nu̮pa̱ssī̱꜔꜒ vā̱ ci̱tta̱smi̱꜔꜒ṁ vi̮ha̮ra̮ti̮, sa̮mu̮da̮ya̮-va̮ya̮-dha̱mmā̱nu̮pa̱ssī̱꜔꜒ vā̱ ci̱tta̱smi͓꜔꜒ṁ
vi̮ha̮ra̮ti̮. ‘A̱tthi̮ ci̱tta̱n’ti̮ vā̱ pa̮na̱ssa̮ sa̮ti̮ pa̱ccu̮pa̱ṭṭhi̮tā̱ ho̱꜔꜒ti̮ yā̱va̮de̱va͓
ñā̱ṇa̮ma̱ttā̱ya̮ pa̮ṭi̱ssa̮ti̮ma̱ttā̱ya̮ a̮ni̱ssi̮to̱ ca̮ vi̮ha̮ra̮ti̮, na̮ ca̮ ki̱ñci̮ lo̱ke̱ u̮pā̱di̮ya̮ti͓.
E̱va̱mpi̮ kho̱꜔꜒, bhi̱kkha̮ve̱, bhi̱kkhu̮ ci̱tte̱ ci̱ttā̱nu̮pa̱ssī̱꜔꜒ vi̮ha̮ra̮ti͓.

\instr{Cittānupassanā niṭṭhitā.}

\englishPage

\paliPage

\englishPage
\chapter{Contemplation of Phenomena}

\section{The Five Hindrances}

And how, bhikkhus, does a bhikkhu dwell contemplating phenomena in phenomena?

Here, bhikkhus, a bhikkhu dwells contemplating phenomena in phenomena in terms
of the five hindrances.

And how, bhikkhus, does a bhikkhu dwell contemplating phenomena in phenomena in
terms of the five hindrances?

Here, bhikkhus, a bhikkhu,
when there is sensual desire in him, understands:
`There is sensual desire in me';
or when there is no sensual desire in him, he understands:
`There is no sensual desire in me';
and he also understands how unarisen sensual desire arises,
and how arisen sensual desire is abandoned,
and how abandoned sensual desire does not arise again in the future.

When there is ill will in him, a bhikkhu understands:
`There is ill will in me';
or when there is no ill will in him, he understands:
`There is no ill will in me';
and he also understands how unarisen ill will arises,
and how arisen ill will is abandoned,
and how abandoned ill will does not arise again in the future.

When there is dullness and drowsiness in him, a bhikkhu understands:
`There is dullness and drowsiness in me';
or when there is no dullness and drowsiness in him, he understands:
`There is no dullness and drowsiness in me';
and he also understands how unarisen dullness and drowsiness arises,
and how arisen dullness and drowsiness is abandoned, and
how abandoned dullness and drowsiness does not arise again in the future.'

\paliPage
\chapter*{Dhammānupassanā}

\section*{Nīvaraṇapabba}

Ka̮tha̱꜔꜒ñca̮ pa̮na̮, bhi̱kkha̮ve̱, bhi̱kkhu̮ dha̱mme̱su̮ dha̱mmā̱nu̮pa̱ssī̱꜔꜒ vi̮ha̮ra̮ti͓?

I̮dha̮, bhi̱kkha̮ve̱, bhi̱kkhu̮ dha̱mme̱su̮ dha̱mmā̱nu̮pa̱ssī̱꜔꜒ vi̮ha̮ra̮ti̮ pa̱ñca̮su̮ nī̱va̮ra̮ṇe̱su͓.

Ka̮tha̱꜔꜒ñca̮ pa̮na̮, bhi̱kkha̮ve̱, bhi̱kkhu̮ dha̱mme̱su̮ dha̱mmā̱nu̮pa̱ssī̱꜔꜒ vi̮ha̮ra̮ti̮ pa̱ñca̮su͓
nī̱va̮ra̮ṇe̱su͓?

Idha, bhikkhave, bhikkhu
sa̱꜔꜒nta̱ṁ vā̱ a̱jjha̱tta̱ṁ kā̱ma̱ccha̱꜔꜒nda̱ṁ ‘a̱tthi̮ me̱ a̱jjha̱tta̱ṁ kā̱ma̱ccha̱꜔꜒ndo̱’ti̮ pa̮jā̱nā̱ti͓,
a̮sa̱꜔꜒nta̱ṁ vā̱ a̱jjha̱tta̱ṁ kā̱ma̱ccha̱꜔꜒nda̱ṁ ‘na̱tthi̮ me̱ a̱jjha̱tta̱ṁ kā̱ma̱ccha̱꜔꜒ndo̱’ti̮ pa̮jā̱nā̱ti͓,
ya̮thā̱꜔꜒ ca̮ a̮nu̱ppa̱nna̱ssa̮ kā̱ma̱ccha̱꜔꜒nda̱ssa̮ u̱ppā̱do̱ ho̱꜔꜒ti̮ ta̱ñca̮ pa̮jā̱nā̱ti͓,
ya̮thā̱꜔꜒ ca̮ u̱ppa̱nna̱ssa̮ kā̱ma̱ccha̱꜔꜒nda̱ssa̮ pa̮hā̱꜔꜒na̱ṁ ho̱꜔꜒ti̮ ta̱ñca̮ pa̮jā̱nā̱ti͓,
ya̮thā̱꜔꜒ ca̮ pa̮hī̱꜔꜒na̱ssa̮ kā̱ma̱ccha̱꜔꜒nda̱ssa̮ ā̱ya̮ti̱ṁ a̮nu̱ppā̱do̱ ho̱꜔꜒ti̮ ta̱ñca̮ pa̮jā̱nā̱ti͓.

Sa̱꜔꜒nta̱ṁ vā̱ a̱jjha̱tta̱ṁ byā̱pā̱da̱ṁ ‘a̱tthi̮ me̱ a̱jjha̱tta̱ṁ byā̱pā̱do̱’ti̮ pa̮jā̱nā̱ti͓,
a̮sa̱꜔꜒nta̱ṁ vā̱ a̱jjha̱tta̱ṁ byā̱pā̱da̱ṁ ‘na̱tthi̮ me̱ a̱jjha̱tta̱ṁ byā̱pā̱do̱’ti̮ pa̮jā̱nā̱ti͓,
ya̮thā̱꜔꜒ ca̮ a̮nu̱ppa̱nna̱ssa̮ byā̱pā̱da̱ssa̮ u̱ppā̱do̱ ho̱꜔꜒ti̮ ta̱ñca̮ pa̮jā̱nā̱ti͓,
ya̮thā̱꜔꜒ ca̮ u̱ppa̱nna̱ssa̮ byā̱pā̱da̱ssa̮ pa̮hā̱꜔꜒na̱ṁ ho̱꜔꜒ti̮ ta̱ñca̮ pa̮jā̱nā̱ti͓,
ya̮thā̱꜔꜒ ca̮ pa̮hī̱꜔꜒na̱ssa̮ byā̱pā̱da̱ssa̮ ā̱ya̮ti̱ṁ a̮nu̱ppā̱do̱ ho̱꜔꜒ti̮ ta̱ñca̮ pa̮jā̱nā̱ti͓.

\enlargethispage{\baselineskip}

Sa̱꜔꜒nta̱ṁ vā̱ a̱jjha̱tta̱ṁ thī̱꜔꜒na̮mi̱ddha̱ṁ ‘a̱tthi̮ me̱ a̱jjha̱tta̱ṁ thī̱꜔꜒na̮mi̱ddha̱n’ti̮ pa̮jā̱nā̱ti͓,
a̮sa̱꜔꜒nta̱ṁ vā̱ a̱jjha̱tta̱ṁ thī̱꜔꜒na̮mi̱ddha̱ṁ ‘na̱tthi̮ me̱ a̱jjha̱tta̱ṁ thī̱꜔꜒na̮mi̱ddha̱n’ti̮ pa̮jā̱nā̱ti͓,
ya̮thā̱꜔꜒ ca̮ a̮nu̱ppa̱nna̱ssa̮ thī̱꜔꜒na̮mi̱ddha̱ssa̮ u̱ppā̱do̱ ho̱꜔꜒ti̮ ta̱ñca̮ pa̮jā̱nā̱ti͓,
ya̮thā̱꜔꜒ ca̮ u̱ppa̱nna̱ssa̮ thī̱꜔꜒na̮mi̱ddha̱ssa̮ pa̮hā̱꜔꜒na̱ṁ ho̱꜔꜒ti̮ ta̱ñca̮ pa̮jā̱nā̱ti͓,
ya̮thā̱꜔꜒ ca̮ pa̮hī̱꜔꜒na̱ssa̮ thī̱꜔꜒na̮mi̱ddha̱ssa̮ ā̱ya̮ti̱ṁ a̮nu̱ppā̱do̱ ho̱꜔꜒ti̮ ta̱ñca̮ pa̮jā̱nā̱ti͓.

\englishPage

When there is restlessness and remorse in him, a bhikkhu understands:
`There is restlessness and remorse in me';
or when there is no restlessness and remorse in him, he understands:
`There is no restlessness and remorse in me';
and he also understands how unarisen restlessness and remorse arises,
and how arisen restlessness and remorse is abandoned,
and how abandoned restlessness and remorse does not arise again in the future.

When there is doubt in him, a bhikkhu understands:
`There is doubt in me';
or when there is no doubt in him, he understands:
`There is no doubt in me';
and he also understands how unarisen doubt arises,
and how arisen doubt is abandoned,
and how abandoned doubt does not arise again in the future.

In this way he dwells contemplating phenomena in phenomena internally, or he
dwells contemplating phenomena in phenomena externally, or he dwells
contemplating phenomena in phenomena both internally and externally. Or else he
dwells contemplating in phenomena its nature of arising, or he dwells
contemplating in phenomena its nature of vanishing, or he dwells contemplating
in phenomena its nature of both arising and vanishing. Or else mindfulness that
‘there are phenomena’ is simply established in him to the extent necessary for
bare knowledge and repeated mindfulness.

And he dwells independent, not clinging to anything in the world. That is how,
bhikkhus, a bhikkhu dwells contemplating phenomena in phenomena in terms of five
hindrances.

\instr{The section on the Five Hindrances is finished.}

\paliPage

Sa̱꜔꜒nta̱ṁ vā̱ a̱jjha̱tta̱ṁ u̱ddha̱cca̮-ku̱kku̱cca̱ṁ ‘a̱tthi̮ me̱ a̱jjha̱tta̱ṁ u̱ddha̱cca̮-ku̱kku̱cca̱n’ti̮ pa̮jā̱nā̱ti͓,
a̮sa̱꜔꜒nta̱ṁ vā̱ a̱jjha̱tta̱ṁ u̱ddha̱cca̮-ku̱kku̱cca̱ṁ ‘na̱tthi̮ me̱ a̱jjha̱tta̱ṁ u̱ddha̱cca̮-ku̱kku̱cca̱n’ti̮ pa̮jā̱nā̱ti͓,
ya̮thā̱꜔꜒ ca̮ a̮nu̱ppa̱nna̱ssa̮ u̱ddha̱cca̮-ku̱kku̱cca̱ssa̮ u̱ppā̱do̱ ho̱꜔꜒ti̮ ta̱ñca̮ pa̮jā̱nā̱ti͓,
ya̮thā̱꜔꜒ ca̮ u̱ppa̱nna̱ssa̮ u̱ddha̱cca̮-ku̱kku̱cca̱ssa̮ pa̮hā̱꜔꜒na̱ṁ ho̱꜔꜒ti̮ ta̱ñca̮ pa̮jā̱nā̱ti͓,
ya̮thā̱꜔꜒ ca̮ pa̮hī̱꜔꜒na̱ssa̮ u̱ddha̱cca̮-ku̱kku̱cca̱ssa̮ ā̱ya̮ti̱ṁ a̮nu̱ppā̱do̱ ho̱꜔꜒ti̮ ta̱ñca̮ pa̮jā̱nā̱ti͓.

Sa̱꜔꜒nta̱ṁ vā̱ a̱jjha̱tta̱ṁ vi̮ci̮ki̱ccha̱꜔꜒ṁ ‘a̱tthi̮ me̱ a̱jjha̱tta̱ṁ vi̮ci̮ki̱cchā̱꜔꜒’ti̮ pa̮jā̱nā̱ti͓,
a̮sa̱꜔꜒nta̱ṁ vā̱ a̱jjha̱tta̱ṁ vi̮ci̮ki̱ccha̱꜔꜒ṁ ‘na̱tthi̮ me̱ a̱jjha̱tta̱ṁ vi̮ci̮ki̱cchā̱꜔꜒’ti̮ pa̮jā̱nā̱ti͓,
ya̮thā̱꜔꜒ ca̮ a̮nu̱ppa̱nnā̱ya̮ vi̮ci̮ki̱cchā̱꜔꜒ya̮ u̱ppā̱do̱ ho̱꜔꜒ti̮ ta̱ñca̮ pa̮jā̱nā̱ti͓,
ya̮thā̱꜔꜒ ca̮ u̱ppa̱nnā̱ya̮ vi̮ci̮ki̱cchā̱꜔꜒ya̮ pa̮hā̱꜔꜒na̱ṁ ho̱꜔꜒ti̮ ta̱ñca̮ pa̮jā̱nā̱ti͓,
ya̮thā̱꜔꜒ ca̮ pa̮hī̱꜔꜒nā̱ya̮ vi̮ci̮ki̱cchā̱꜔꜒ya̮ ā̱ya̮ti̱ṁ a̮nu̱ppā̱do̱ ho̱꜔꜒ti̮ ta̱ñca̮ pa̮jā̱nā̱ti͓.

I̮ti̮ a̱jjha̱tta̱ṁ vā̱ dha̱mme̱su̮ dha̱mmā̱nu̮pa̱ssī̱꜔꜒ vi̮ha̮ra̮ti͓,
ba̮hi̱ddhā̱ vā̱ dha̱mme̱su̮ dha̱mmā̱nu̮pa̱ssī̱꜔꜒ vi̮ha̮ra̮ti͓,
a̱jjha̱tta̮-ba̮hi̱ddhā̱ vā̱ dha̱mme̱su̮ dha̱mmā̱nu̮pa̱ssī̱꜔꜒ vi̮ha̮ra̮ti͓.
Sa̮mu̮da̮ya̮-dha̱mmā̱nu̮pa̱ssī̱꜔꜒ vā̱ dha̱mme̱su̮ vi̮ha̮ra̮ti͓,
va̮ya̮-dha̱mmā̱nu̮pa̱ssī̱꜔꜒ vā̱ dha̱mme̱su̮ vi̮ha̮ra̮ti͓,
sa̮mu̮da̮ya̮-va̮ya̮-\\ dha̱mmā̱nu̮pa̱ssī̱꜔꜒ vā̱ dha̱mme̱su̮ vi̮ha̮ra̮ti͓.
‘A̱tthi̮ dha̱mmā̱’ti̮ vā̱ pa̮na̱ssa̮ sa̮ti̮ pa̱ccu̮pa̱ṭṭhi̮tā̱ ho̱꜔꜒ti͓
yā̱va̮de̱va̮ ñā̱ṇa̮ma̱ttā̱ya̮ pa̮ṭi̱ssa̮ti̮ma̱ttā̱ya̮, a̮ni̱ssi̮to̱ ca̮ vi̮ha̮ra̮ti͓,
na̮ ca̮ ki̱ñci̮ lo̱ke̱ u̮pā̱di̮ya̮ti̮. E̱va̱mpi̮ kho̱꜔꜒, bhi̱kkha̮ve̱, bhi̱kkhu͓
dha̱mme̱su̮ dha̱mmā̱nu̮pa̱ssī̱꜔꜒ vi̮ha̮ra̮ti̮ pa̱ñca̮su̮ nī̱va̮ra̮ṇe̱su͓.

\instr{Nīvaraṇapabbaṁ niṭṭhitaṁ.}

\englishPage
\section{The Five Aggregates}

Again, bhikkhus, a bhikkhu dwells contemplating phenomena in phenomena in terms
of the five aggregates subject to clinging.

And how, bhikkhus, does a bhikkhu dwell contemplating phenomena in phenomena in
terms of the five aggregates affected by clinging?

Here, bhikkhus, a bhikkhu understands:
`Such is form, such its origin, such its passing away;
such is feeling, such its origin, such its passing away;
such is perception, such its origin, such its passing away;
such are the volitional formations, such their origin, such their passing away;
such is consciousness, such its origin, such its passing away.'

In this way he dwells contemplating phenomena in phenomena internally, or he
dwells contemplating phenomena in phenomena externally, or he dwells
contemplating phenomena in phenomena both internally and externally. Or else he
dwells contemplating in phenomena its nature of arising, or he dwells
contemplating in phenomena its nature of vanishing, or he dwells contemplating
in phenomena its nature of both arising and vanishing. Or else mindfulness that
‘there are phenomena’ is simply established in him to the extent necessary for
bare knowledge and repeated mindfulness.

And he dwells independent, not clinging to anything in the world. That is how,
bhikkhus, a bhikkhu dwells contemplating phenomena in phenomena in terms of the
five aggregates subject to clinging.

\instr{The section on the Five Aggregates is finished.}

\paliPage
\section*{Khandhapabba}

Pu̮na̮ ca̮pa̮ra̱ṁ, bhi̱kkha̮ve̱, bhi̱kkhu̮ dha̱mme̱su̮ dha̱mmā̱nu̮pa̱ssī̱꜔꜒ vi̮ha̮ra̮ti̮ pa̱ñca̮su͓
u̮pā̱dā̱na̱-kkha̱꜔꜒ndhe̱su͓.

Ka̮tha̱꜔꜒ñca̮ pa̮na̮, bhi̱kkha̮ve̱, bhi̱kkhu̮ dha̱mme̱su̮ dha̱mmā̱nu̮pa̱ssī̱꜔꜒ vi̮ha̮ra̮ti̮ pa̱ñca̮su͓
u̮pā̱dā̱na̱-kkha̱꜔꜒ndhe̱su͓?

Idha, bhikkhave, bhikkhu:
‘i̮ti̮ rū̱pa̱ṁ, i̮ti̮ rū̱pa̱ssa̮ sa̮mu̮da̮yo̱, i̮ti̮ rū̱pa̱ssa̮ a̱ttha̱꜔꜒ṅga̮mo͓;
i̮ti̮ ve̱da̮nā̱, i̮ti̮ ve̱da̮nā̱ya̮ sa̮mu̮da̮yo̱, i̮ti̮ ve̱da̮nā̱ya̮ a̱ttha̱꜔꜒ṅga̮mo͓;
i̮ti̮ sa̱꜔꜒ññā̱, i̮ti̮ sa̱꜔꜒ññā̱ya̮ sa̮mu̮da̮yo̱, i̮ti̮ sa̱꜔꜒ññā̱ya̮ a̱ttha̱꜔꜒ṅga̮mo͓;
i̮ti̮ sa̱꜔꜒ṅkhā̱꜔꜒rā̱, i̮ti̮ sa̱꜔꜒ṅkhā̱꜔꜒rā̱na̱ṁ sa̮mu̮da̮yo̱, i̮ti̮ sa̱꜔꜒ṅkhā̱꜔꜒rā̱na̱ṁ a̱ttha̱꜔꜒ṅga̮mo͓;
i̮ti̮ vi̱ññā̱ṇa̱ṁ, i̮ti̮ vi̱ññā̱ṇa̱ssa̮ sa̮mu̮da̮yo̱, i̮ti̮ vi̱ññā̱ṇa̱ssa̮ a̱ttha̱꜔꜒ṅga̮mo̱’ti͓.

I̮ti̮ a̱jjha̱tta̱ṁ vā̱ dha̱mme̱su̮ dha̱mmā̱nu̮pa̱ssī̱꜔꜒ vi̮ha̮ra̮ti͓,
ba̮hi̱ddhā̱ vā̱ dha̱mme̱su̮ dha̱mmā̱nu̮pa̱ssī̱꜔꜒ vi̮ha̮ra̮ti͓,
a̱jjha̱tta̮-ba̮hi̱ddhā̱ vā̱ dha̱mme̱su̮ dha̱mmā̱nu̮pa̱ssī̱꜔꜒ vi̮ha̮ra̮ti͓.
Sa̮mu̮da̮ya̮-dha̱mmā̱nu̮pa̱ssī̱꜔꜒ vā̱ dha̱mme̱su̮ vi̮ha̮ra̮ti͓,
va̮ya̮-dha̱mmā̱nu̮pa̱ssī̱꜔꜒ vā̱ dha̱mme̱su̮ vi̮ha̮ra̮ti͓,
sa̮mu̮da̮ya̮-va̮ya̮-\\ dha̱mmā̱nu̮pa̱ssī̱꜔꜒ vā̱ dha̱mme̱su̮ vi̮ha̮ra̮ti͓.
‘A̱tthi̮ dha̱mmā̱’ti̮ vā̱ pa̮na̱ssa̮ sa̮ti̮ pa̱ccu̮pa̱ṭṭhi̮tā̱ ho̱꜔꜒ti͓
yā̱va̮de̱va̮ ñā̱ṇa̮ma̱ttā̱ya̮ pa̮ṭi̱ssa̮ti̮ma̱ttā̱ya̮, a̮ni̱ssi̮to̱ ca̮ vi̮ha̮ra̮ti͓,
na̮ ca̮ ki̱ñci̮ lo̱ke̱ u̮pā̱di̮ya̮ti̮. E̱va̱mpi̮ kho̱꜔꜒, bhi̱kkha̮ve̱, bhi̱kkhu͓
dha̱mme̱su̮ dha̱mmā̱nu̮pa̱ssī̱꜔꜒ vi̮ha̮ra̮ti̮ pa̱ñca̮su̮ u̮pā̱dā̱na̱-kkha̱꜔꜒ndhe̱su͓.

\instr{Khandhapabbaṁ niṭṭhitaṁ.}

\englishPage
\section{The Six Sense Bases}

Again, bhikkhus, a bhikkhu dwells contemplating phenomena in phenomena in terms
of the six internal and external sense bases.

And how, bhikkhus, does a bhikkhu dwell contemplating phenomena in phenomena in
terms of the six internal and external sense bases?

Here, bhikkhus, a bhikkhu understands the eye, he understands forms, and he
understands the fetter that arises dependent on both; and he also understands
how the unarisen fetter arises, and how the arisen fetter is abandoned, and how
the abandoned fetter does not arise again in the future.

He understands the ear, he understands sounds, and he understands the fetter
that arises dependent on both; and he also understands how the unarisen fetter
arises, and how the arisen fetter is abandoned, and how the abandoned fetter
does not arise again in the future.

He understands the nose, he understands odours, and he understands the fetter
that arises dependent on both; and he also understands how the unarisen fetter
arises, and how the arisen fetter is abandoned, and how the abandoned fetter
does not arise again in the future.

He understands the tongue, he understands flavours, and he understands the
fetter that arises dependent on both; and he also understands how the unarisen
fetter arises, and how the arisen fetter is abandoned, and how the abandoned
fetter does not arise again in the future.

\paliPage
\section*{Āyatanapabba}

Pu̮na̮ ca̮pa̮ra̱ṁ, bhi̱kkha̮ve̱, bhi̱kkhu̮ dha̱mme̱su̮ dha̱mmā̱nu̮pa̱ssī̱꜔꜒ vi̮ha̮ra̮ti̮ cha̮su͓
a̱jjha̱tti̮ka̮-bā̱hi̮re̱su̮ ā̱ya̮ta̮ne̱su͓.

Ka̮tha̱꜔꜒ñca̮ pa̮na̮, bhi̱kkha̮ve̱, bhi̱kkhu̮ dha̱mme̱su̮ dha̱mmā̱nu̮pa̱ssī̱꜔꜒ vi̮ha̮ra̮ti̮ cha̮su͓
a̱jjha̱tti̮ka̮-bā̱hi̮re̱su̮ ā̱ya̮ta̮ne̱su͓?

Idha, bhikkhave, bhikkhu
ca̱kkhu̱꜔꜒ñca̮ pa̮jā̱nā̱ti͓,
rū̱pe̱ ca̮ pa̮jā̱nā̱ti͓,
ya̱ñca̮ ta̮du̮bha̮ya̱ṁ pa̮ṭi̱cca̮ u̱ppa̱jja̮ti̮ sa̱꜔꜒ṁyo̱ja̮na̱ṁ ta̱ñca̮ pa̮jā̱nā̱ti͓,
ya̮thā̱꜔꜒ ca̮ a̮nu̱ppa̱nna̱ssa̮ sa̱꜔꜒ṁyo̱ja̮na̱ssa̮ u̱ppā̱do̱ ho̱꜔꜒ti̮ ta̱ñca̮ pa̮jā̱nā̱ti͓,
ya̮thā̱꜔꜒ ca̮ u̱ppa̱nna̱ssa̮ sa̱꜔꜒ṁyo̱ja̮na̱ssa̮ pa̮hā̱꜔꜒na̱ṁ ho̱꜔꜒ti̮ ta̱ñca̮ pa̮jā̱nā̱ti͓,
ya̮thā̱꜔꜒ ca̮ pa̮hī̱꜔꜒na̱ssa̮ sa̱꜔꜒ṁyo̱ja̮na̱ssa̮ ā̱ya̮ti̱ṁ a̮nu̱ppā̱do̱ ho̱꜔꜒ti̮ ta̱ñca̮ pa̮jā̱nā̱ti͓.

So̱꜔꜒ta̱ñca̮ pa̮jā̱nā̱ti͓,
sa̱dde̱ ca̮ pa̮jā̱nā̱ti͓,
ya̱ñca̮ ta̮du̮bha̮ya̱ṁ pa̮ṭi̱cca̮ u̱ppa̱jja̮ti̮ sa̱꜔꜒ṁyo̱ja̮na̱ṁ ta̱ñca̮ pa̮jā̱nā̱ti͓,
ya̮thā̱꜔꜒ ca̮ a̮nu̱ppa̱nna̱ssa̮ sa̱꜔꜒ṁyo̱ja̮na̱ssa̮ u̱ppā̱do̱ ho̱꜔꜒ti̮ ta̱ñca̮ pa̮jā̱nā̱ti͓,
ya̮thā̱꜔꜒ ca̮ u̱ppa̱nna̱ssa̮ sa̱꜔꜒ṁyo̱ja̮na̱ssa̮ pa̮hā̱꜔꜒na̱ṁ ho̱꜔꜒ti̮ ta̱ñca̮ pa̮jā̱nā̱ti͓,
ya̮thā̱꜔꜒ ca̮ pa̮hī̱꜔꜒na̱ssa̮ sa̱꜔꜒ṁyo̱ja̮na̱ssa̮ ā̱ya̮ti̱ṁ a̮nu̱ppā̱do̱ ho̱꜔꜒ti̮ ta̱ñca̮ pa̮jā̱nā̱ti͓.

Ghā̱na̱ñca̮ pa̮jā̱nā̱ti͓,
ga̱ndhe̱ ca̮ pa̮jā̱nā̱ti͓,
ya̱ñca̮ ta̮du̮bha̮ya̱ṁ pa̮ṭi̱cca̮ u̱ppa̱jja̮ti̮ sa̱꜔꜒ṁyo̱ja̮na̱ṁ ta̱ñca̮ pa̮jā̱nā̱ti͓,
ya̮thā̱꜔꜒ ca̮ a̮nu̱ppa̱nna̱ssa̮ sa̱꜔꜒ṁyo̱ja̮na̱ssa̮ u̱ppā̱do̱ ho̱꜔꜒ti̮ ta̱ñca̮ pa̮jā̱nā̱ti͓,
ya̮thā̱꜔꜒ ca̮ u̱ppa̱nna̱ssa̮ sa̱꜔꜒ṁyo̱ja̮na̱ssa̮ pa̮hā̱꜔꜒na̱ṁ ho̱꜔꜒ti̮ ta̱ñca̮ pa̮jā̱nā̱ti͓,
ya̮thā̱꜔꜒ ca̮ pa̮hī̱꜔꜒na̱ssa̮ sa̱꜔꜒ṁyo̱ja̮na̱ssa̮ ā̱ya̮ti̱ṁ a̮nu̱ppā̱do̱ ho̱꜔꜒ti̮ ta̱ñca̮ pa̮jā̱nā̱ti͓.

Ji̱vha̱꜔꜒ñca̮ pa̮jā̱nā̱ti͓,
ra̮se̱꜔꜒ ca̮ pa̮jā̱nā̱ti͓,
ya̱ñca̮ ta̮du̮bha̮ya̱ṁ pa̮ṭi̱cca̮ u̱ppa̱jja̮ti̮ sa̱꜔꜒ṁyo̱ja̮na̱ṁ ta̱ñca̮ pa̮jā̱nā̱ti͓,
ya̮thā̱꜔꜒ ca̮ a̮nu̱ppa̱nna̱ssa̮ sa̱꜔꜒ṁyo̱ja̮na̱ssa̮ u̱ppā̱do̱ ho̱꜔꜒ti̮ ta̱ñca̮ pa̮jā̱nā̱ti͓,
ya̮thā̱꜔꜒ ca̮ u̱ppa̱nna̱ssa̮ sa̱꜔꜒ṁyo̱ja̮na̱ssa̮ pa̮hā̱꜔꜒na̱ṁ ho̱꜔꜒ti̮ ta̱ñca̮ pa̮jā̱nā̱ti͓,
ya̮thā̱꜔꜒ ca̮ pa̮hī̱꜔꜒na̱ssa̮ sa̱꜔꜒ṁyo̱ja̮na̱ssa̮ ā̱ya̮ti̱ṁ a̮nu̱ppā̱do̱ ho̱꜔꜒ti̮ ta̱ñca̮ pa̮jā̱nā̱ti͓.

\englishPage

He understands the body, he understands tactile objects, and he understands the
fetter that arises dependent on both; and he also understands how the unarisen
fetter arises, and how the arisen fetter is abandoned, and how the abandoned
fetter does not arise again in the future.

He understands the mind, he understands phenomena, and he understands the fetter
that arises dependent on both; and he also understands how the unarisen fetter
arises, and how the arisen fetter is abandoned, and how the abandoned fetter
does not arise again in the future.

In this way he dwells contemplating phenomena in phenomena internally, or he
dwells contemplating phenomena in phenomena externally, or he dwells
contemplating phenomena in phenomena both internally and externally. Or else he
dwells contemplating in phenomena its nature of arising, or he dwells
contemplating in phenomena its nature of vanishing, or he dwells contemplating
in phenomena its nature of both arising and vanishing. Or else mindfulness that
‘there are phenomena’ is simply established in him to the extent necessary for
bare knowledge and repeated mindfulness.

And he dwells independent, not clinging to anything in the world. That is how,
bhikkhus, a bhikkhu dwells contemplating phenomena in phenomena in terms of the
six internal and external sense bases.

\instr{The section on the six sense bases is finished.}

\section{The Seven Enlightenment Factors}

Again, bhikkhus, a bhikkhu dwells contemplating phenomena in phenomena in terms
of the seven enlightenment factors.

And how, bhikkhus, does a bhikkhu dwell contemplating phenomena in phenomena in
terms of the seven enlightenment factors?

\paliPage

Kā̱ya̱ñca̮ pa̮jā̱nā̱ti͓,
pho̱ṭṭha̱bbe̱ ca̮ pa̮jā̱nā̱ti͓,
ya̱ñca̮ ta̮du̮bha̮ya̱ṁ pa̮ṭi̱cca̮ u̱ppa̱jja̮ti̮ sa̱꜔꜒ṁyo̱ja̮na̱ṁ ta̱ñca̮ pa̮jā̱nā̱ti͓,
ya̮thā̱꜔꜒ ca̮ a̮nu̱ppa̱nna̱ssa̮ sa̱꜔꜒ṁyo̱ja̮na̱ssa̮ u̱ppā̱do̱ ho̱꜔꜒ti̮ ta̱ñca̮ pa̮jā̱nā̱ti͓,
ya̮thā̱꜔꜒ ca̮ u̱ppa̱nna̱ssa̮ sa̱꜔꜒ṁyo̱ja̮na̱ssa̮ pa̮hā̱꜔꜒na̱ṁ ho̱꜔꜒ti̮ ta̱ñca̮ pa̮jā̱nā̱ti͓,
ya̮thā̱꜔꜒ ca̮ pa̮hī̱꜔꜒na̱ssa̮ sa̱꜔꜒ṁyo̱ja̮na̱ssa̮ ā̱ya̮ti̱ṁ a̮nu̱ppā̱do̱ ho̱꜔꜒ti̮ ta̱ñca̮ pa̮jā̱nā̱ti͓.

Ma̮na̱ñca̮ pa̮jā̱nā̱ti͓,
dha̱mme̱ ca̮ pa̮jā̱nā̱ti͓,
ya̱ñca̮ ta̮du̮bha̮ya̱ṁ pa̮ṭi̱cca̮ u̱ppa̱jja̮ti̮ sa̱꜔꜒ṁyo̱ja̮na̱ṁ ta̱ñca̮ pa̮jā̱nā̱ti͓,
ya̮thā̱꜔꜒ ca̮ a̮nu̱ppa̱nna̱ssa̮ sa̱꜔꜒ṁyo̱ja̮na̱ssa̮ u̱ppā̱do̱ ho̱꜔꜒ti̮ ta̱ñca̮ pa̮jā̱nā̱ti͓,
ya̮thā̱꜔꜒ ca̮ u̱ppa̱nna̱ssa̮ sa̱꜔꜒ṁyo̱ja̮na̱ssa̮ pa̮hā̱꜔꜒na̱ṁ ho̱꜔꜒ti̮ ta̱ñca̮ pa̮jā̱nā̱ti͓,
ya̮thā̱꜔꜒ ca̮ pa̮hī̱꜔꜒na̱ssa̮ sa̱꜔꜒ṁyo̱ja̮na̱ssa̮ ā̱ya̮ti̱ṁ a̮nu̱ppā̱do̱ ho̱꜔꜒ti̮ ta̱ñca̮ pa̮jā̱nā̱ti͓.

I̮ti̮ a̱jjha̱tta̱ṁ vā̱ dha̱mme̱su̮ dha̱mmā̱nu̮pa̱ssī̱꜔꜒ vi̮ha̮ra̮ti͓,
ba̮hi̱ddhā̱ vā̱ dha̱mme̱su̮ dha̱mmā̱nu̮pa̱ssī̱꜔꜒ vi̮ha̮ra̮ti͓,
a̱jjha̱tta̮-ba̮hi̱ddhā̱ vā̱ dha̱mme̱su̮ dha̱mmā̱nu̮pa̱ssī̱꜔꜒ vi̮ha̮ra̮ti͓.
Sa̮mu̮da̮ya̮-dha̱mmā̱nu̮pa̱ssī̱꜔꜒ vā̱ dha̱mme̱su̮ vi̮ha̮ra̮ti͓,
va̮ya̮-dha̱mmā̱nu̮pa̱ssī̱꜔꜒ vā̱ dha̱mme̱su̮ vi̮ha̮ra̮ti͓,
sa̮mu̮da̮ya̮-va̮ya̮-\\ dha̱mmā̱nu̮pa̱ssī̱꜔꜒ vā̱ dha̱mme̱su̮ vi̮ha̮ra̮ti͓.
‘A̱tthi̮ dha̱mmā̱’ti̮ vā̱ pa̮na̱ssa̮ sa̮ti̮ pa̱ccu̮pa̱ṭṭhi̮tā̱ ho̱꜔꜒ti͓
yā̱va̮de̱va̮ ñā̱ṇa̮ma̱ttā̱ya̮ pa̮ṭi̱ssa̮ti̮ma̱ttā̱ya̮, a̮ni̱ssi̮to̱ ca̮ vi̮ha̮ra̮ti͓,
na̮ ca̮ ki̱ñci̮ lo̱ke̱ u̮pā̱di̮ya̮ti̮. E̱va̱mpi̮ kho̱꜔꜒, bhi̱kkha̮ve̱, bhi̱kkhu͓
dha̱mme̱su̮ dha̱mmā̱nu̮pa̱ssī̱꜔꜒ vi̮ha̮ra̮ti̮ cha̮su̮ a̱jjha̱tti̮ka̮-bā̱hi̮re̱su̮ ā̱ya̮ta̮ne̱su͓.

\instr{Āyatanapabbaṁ niṭṭhitaṁ.}

\section*{Bojjhaṅgapabba}

Pu̮na̮ ca̮pa̮ra̱ṁ, bhi̱kkha̮ve̱, bhi̱kkhu̮ dha̱mme̱su̮ dha̱mmā̱nu̮pa̱ssī̱꜔꜒ vi̮ha̮ra̮ti̮ sa̱tta̮su͓
bo̱jjha̱ṅge̱su͓.

Ka̮tha̱꜔꜒ñca̮ pa̮na̮, bhi̱kkha̮ve̱, bhi̱kkhu̮ dha̱mme̱su̮ dha̱mmā̱nu̮pa̱ssī̱꜔꜒ vi̮ha̮ra̮ti̮ sa̱tta̮su͓
bo̱jjha̱ṅge̱su͓?

\englishPage

Here, bhikkhus, when there is the mindfulness enlightenment factor in him, a
bhikkhu understands: `There is the mindfulness enlightenment factor in me'; or
when there is no mindfulness enlightenment factor in him, he understands: `There
is no mindfulness enlightenment factor in me'; and he also understands how the
unarisen mindfulness enlightenment factor arises, and how the arisen mindfulness
enlightenment factor comes to fulfillment by development.

When there is the discrimination of phenomena enlightenment factor in him, a
bhikkhu understands: `There is the discrimination of phenomena enlightenment
factor in me'; or when there is no discrimination of phenomena enlightenment
factor in him, he understands: `There is no discrimination of phenomena
enlightenment factor in me'; and he also understands how the unarisen
discrimination of phenomena enlightenment factor arises, and how the arisen
discrimination of phenomena enlightenment factor comes to fulfillment by
development.

When there is the energy enlightenment factor in him, a bhikkhu understands:
`There is the energy enlightenment factor in me'; or when there is no energy
enlightenment factor in him, he understands: `There is no energy enlightenment
factor in me'; and he also understands how the unarisen energy enlightenment
factor arises, and how the arisen energy enlightenment factor comes to
fulfillment by development.

When there is the rapture enlightenment factor in him, a bhikkhu understands:
`There is the rapture enlightenment factor in me'; or when there is no rapture
enlightenment factor in him, he understands: `There is no rapture enlightenment
factor in me'; and he also understands how the unarisen rapture enlightenment
factor arises, and how the arisen rapture enlightenment factor comes to
fulfillment by development.

\paliPage

Idha, bhikkhave, bhikkhu
sa̱꜔꜒nta̱ṁ vā̱ a̱jjha̱tta̱ṁ sa̮ti̮-sa̱꜔꜒mbo̱jjha̱ṅga̱ṁ ‘a̱tthi̮ me̱ a̱jjha̱tta̱ṁ sa̮ti̮-sa̱꜔꜒mbo̱jjha̱ṅgo̱’ti̮ pa̮jā̱nā̱ti͓,
a̮sa̱꜔꜒nta̱ṁ vā̱ a̱jjha̱tta̱ṁ sa̮ti̮-sa̱꜔꜒mbo̱jjha̱ṅga̱ṁ ‘na̱tthi̮ me̱ a̱jjha̱tta̱ṁ sa̮ti̮-sa̱꜔꜒mbo̱jjha̱ṅgo̱’ti̮ pa̮jā̱nā̱ti͓,
ya̮thā̱꜔꜒ ca̮ a̮nu̱ppa̱nna̱ssa̮ sa̮ti̮-sa̱꜔꜒mbo̱jjha̱ṅga̱ssa̮ u̱ppā̱do̱ ho̱꜔꜒ti̮ ta̱ñca̮ pa̮jā̱nā̱ti͓,
ya̮thā̱꜔꜒ ca̮ u̱ppa̱nna̱ssa̮ sa̮ti̮-sa̱꜔꜒mbo̱jjha̱ṅga̱ssa̮ bhā̱va̮nā̱ya̮ pā̱ri̮pū̱rī̱ ho̱꜔꜒ti̮ ta̱ñca̮ pa̮jā̱nā̱ti͓.

Sa̱꜔꜒nta̱ṁ vā̱ a̱jjha̱tta̱ṁ dha̱mma̮vi̮ca̮ya̮-sa̱꜔꜒mbo̱jjha̱ṅga̱ṁ ‘a̱tthi̮ me̱ a̱jjha̱tta̱ṁ dha̱mma̮vi̮ca̮ya̮-sa̱꜔꜒mbo̱jjha̱ṅgo̱’ti̮ pa̮jā̱nā̱ti͓,
a̮sa̱꜔꜒nta̱ṁ vā̱ a̱jjha̱tta̱ṁ dha̱mma̮vi̮ca̮ya̮-sa̱꜔꜒mbo̱jjha̱ṅga̱ṁ ‘na̱tthi̮ me̱ a̱jjha̱tta̱ṁ dha̱mma̮vi̮ca̮ya̮-sa̱꜔꜒mbo̱jjha̱ṅgo̱’ti̮ pa̮jā̱nā̱ti͓,
ya̮thā̱꜔꜒ ca̮ a̮nu̱ppa̱nna̱ssa̮ dha̱mma̮vi̮ca̮ya̮-sa̱꜔꜒mbo̱jjha̱ṅga̱ssa̮ u̱ppā̱do̱ ho̱꜔꜒ti̮ ta̱ñca̮ pa̮jā̱nā̱ti͓,
ya̮thā̱꜔꜒ ca̮ u̱ppa̱nna̱ssa̮ dha̱mma̮vi̮ca̮ya̮-sa̱꜔꜒mbo̱jjha̱ṅga̱ssa̮ bhā̱va̮nā̱ya̮ pā̱ri̮pū̱rī̱ ho̱꜔꜒ti̮ ta̱ñca̮ pa̮jā̱nā̱ti͓.

Sa̱꜔꜒nta̱ṁ vā̱ a̱jjha̱tta̱ṁ vi̮ri̮ya̮-sa̱꜔꜒mbo̱jjha̱ṅga̱ṁ ‘a̱tthi̮ me̱ a̱jjha̱tta̱ṁ vi̮ri̮ya̮-sa̱꜔꜒mbo̱jjha̱ṅgo̱’ti̮ pa̮jā̱nā̱ti͓,
a̮sa̱꜔꜒nta̱ṁ vā̱ a̱jjha̱tta̱ṁ vi̮ri̮ya̮-sa̱꜔꜒mbo̱jjha̱ṅga̱ṁ ‘na̱tthi̮ me̱ a̱jjha̱tta̱ṁ vi̮ri̮ya̮-sa̱꜔꜒mbo̱jjha̱ṅgo̱’ti̮ pa̮jā̱nā̱ti͓,
ya̮thā̱꜔꜒ ca̮ a̮nu̱ppa̱nna̱ssa̮ vi̮ri̮ya̮-sa̱꜔꜒mbo̱jjha̱ṅga̱ssa̮ u̱ppā̱do̱ ho̱꜔꜒ti̮ ta̱ñca̮ pa̮jā̱nā̱ti͓,
ya̮thā̱꜔꜒ ca̮ u̱ppa̱nna̱ssa̮ vi̮ri̮ya̮-sa̱꜔꜒mbo̱jjha̱ṅga̱ssa̮ bhā̱va̮nā̱ya̮ pā̱ri̮pū̱rī̱ ho̱꜔꜒ti̮ ta̱ñca̮ pa̮jā̱nā̱ti͓.

Sa̱꜔꜒nta̱ṁ vā̱ a̱jjha̱tta̱ṁ pī̱ti̮-sa̱꜔꜒mbo̱jjha̱ṅga̱ṁ ‘a̱tthi̮ me̱ a̱jjha̱tta̱ṁ pī̱ti̮-sa̱꜔꜒mbo̱jjha̱ṅgo̱’ti̮ pa̮jā̱nā̱ti͓,
a̮sa̱꜔꜒nta̱ṁ vā̱ a̱jjha̱tta̱ṁ pī̱ti̮-sa̱꜔꜒mbo̱jjha̱ṅga̱ṁ ‘na̱tthi̮ me̱ a̱jjha̱tta̱ṁ pī̱ti̮-sa̱꜔꜒mbo̱jjha̱ṅgo̱’ti̮ pa̮jā̱nā̱ti͓,
ya̮thā̱꜔꜒ ca̮ a̮nu̱ppa̱nna̱ssa̮ pī̱ti̮-sa̱꜔꜒mbo̱jjha̱ṅga̱ssa̮ u̱ppā̱do̱ ho̱꜔꜒ti̮ ta̱ñca̮ pa̮jā̱nā̱ti͓,
ya̮thā̱꜔꜒ ca̮ u̱ppa̱nna̱ssa̮ pī̱ti̮-sa̱꜔꜒mbo̱jjha̱ṅga̱ssa̮ bhā̱va̮nā̱ya̮ pā̱ri̮pū̱rī̱ ho̱꜔꜒ti̮ ta̱ñca̮ pa̮jā̱nā̱ti͓.

\englishPage

When there is the tranquillity enlightenment factor in him, a bhikkhu
understands: `There is the tranquillity enlightenment factor in me'; or when
there is no tranquillity enlightenment factor in him, he understands: `There is
no tranquillity enlightenment factor in me'; and he also understands how the
unarisen tranquillity enlightenment factor arises, and how the arisen
tranquillity enlightenment factor comes to fulfillment by development.

When there is the concentration enlightenment factor in him, a bhikkhu
understands: `There is the concentration enlightenment factor in me'; or when
there is no concentration enlightenment factor in him, he understands: `There is
no concentration enlightenment factor in me'; and he also understands how the
unarisen concentration enlightenment factor arises, and how the arisen
concentration enlightenment factor comes to fulfillment by development.

When there is the equanimity enlightenment factor in him, a bhikkhu understands:
`There is the equanimity enlightenment factor in me'; or when there is no
equanimity enlightenment factor in him, he understands: `There is no equanimity
enlightenment factor in me'; and he also understands how the unarisen equanimity
enlightenment factor arises, and how the arisen equanimity enlightenment factor
comes to fulfillment by development.

\enlargethispage{2\baselineskip}

In this way he dwells contemplating phenomena in phenomena \ldots{} or else
mindfulness that ‘there are phenomena’ is simply established in him to the
extent necessary for bare knowledge and repeated mindfulness.

And he dwells independent, not clinging to anything in the world. That is how,
bhikkhus, a bhikkhu dwells contemplating phenomena in phenomena in terms of the
seven enlightenment factors.

\instr{The section on the seven enlightenment factors is finished.}

\paliPage

Sa̱꜔꜒nta̱ṁ vā̱ a̱jjha̱tta̱ṁ pa̱ssa̱ddhi̮-sa̱꜔꜒mbo̱jjha̱ṅga̱ṁ ‘a̱tthi̮ me̱ a̱jjha̱tta̱ṁ pa̱ssa̱ddhi̮-sa̱꜔꜒mbo̱jjha̱ṅgo̱’ti̮ pa̮jā̱nā̱ti͓,
a̮sa̱꜔꜒nta̱ṁ vā̱ a̱jjha̱tta̱ṁ pa̱ssa̱ddhi̮-sa̱꜔꜒mbo̱jjha̱ṅga̱ṁ ‘na̱tthi̮ me̱ a̱jjha̱tta̱ṁ pa̱ssa̱ddhi̮-\\ sa̱꜔꜒mbo̱jjha̱ṅgo̱’ti̮ pa̮jā̱nā̱ti͓,
ya̮thā̱꜔꜒ ca̮ a̮nu̱ppa̱nna̱ssa̮ pa̱ssa̱ddhi̮-\\ sa̱꜔꜒mbo̱jjha̱ṅga̱ssa̮ u̱ppā̱do̱ ho̱꜔꜒ti̮ ta̱ñca̮ pa̮jā̱nā̱ti͓,
ya̮thā̱꜔꜒ ca̮ u̱ppa̱nna̱ssa̮ pa̱ssa̱ddhi̮-sa̱꜔꜒mbo̱jjha̱ṅga̱ssa̮ bhā̱va̮nā̱ya̮ pā̱ri̮pū̱rī̱ ho̱꜔꜒ti̮ ta̱ñca̮ pa̮jā̱nā̱ti͓.

Sa̱꜔꜒nta̱ṁ vā̱ a̱jjha̱tta̱ṁ sa̮mā̱dhi̮-sa̱꜔꜒mbo̱jjha̱ṅga̱ṁ ‘a̱tthi̮ me̱ a̱jjha̱tta̱ṁ sa̮mā̱dhi̮-sa̱꜔꜒mbo̱jjha̱ṅgo̱’ti̮ pa̮jā̱nā̱ti͓,
a̮sa̱꜔꜒nta̱ṁ vā̱ a̱jjha̱tta̱ṁ sa̮mā̱dhi̮-sa̱꜔꜒mbo̱jjha̱ṅga̱ṁ ‘na̱tthi̮ me̱ a̱jjha̱tta̱ṁ sa̮mā̱dhi̮-\\ sa̱꜔꜒mbo̱jjha̱ṅgo̱’ti̮ pa̮jā̱nā̱ti͓,
ya̮thā̱꜔꜒ ca̮ a̮nu̱ppa̱nna̱ssa̮ sa̮mā̱dhi̮-\\ sa̱꜔꜒mbo̱jjha̱ṅga̱ssa̮ u̱ppā̱do̱ ho̱꜔꜒ti̮ ta̱ñca̮ pa̮jā̱nā̱ti͓,
ya̮thā̱꜔꜒ ca̮ u̱ppa̱nna̱ssa̮ sa̮mā̱dhi̮-sa̱꜔꜒mbo̱jjha̱ṅga̱ssa̮ bhā̱va̮nā̱ya̮ pā̱ri̮pū̱rī̱ ho̱꜔꜒ti̮ ta̱ñca̮ pa̮jā̱nā̱ti͓.

Sa̱꜔꜒nta̱ṁ vā̱ a̱jjha̱tta̱ṁ u̮pe̱kkhā̱꜔꜒-sa̱꜔꜒mbo̱jjha̱ṅga̱ṁ ‘a̱tthi̮ me̱ a̱jjha̱tta̱ṁ u̮pe̱kkhā̱꜔꜒-sa̱꜔꜒mbo̱jjha̱ṅgo̱’ti̮ pa̮jā̱nā̱ti͓,
a̮sa̱꜔꜒nta̱ṁ vā̱ a̱jjha̱tta̱ṁ u̮pe̱kkhā̱꜔꜒-sa̱꜔꜒mbo̱jjha̱ṅga̱ṁ ‘na̱tthi̮ me̱ a̱jjha̱tta̱ṁ u̮pe̱kkhā̱꜔꜒-\\ sa̱꜔꜒mbo̱jjha̱ṅgo̱’ti̮ pa̮jā̱nā̱ti͓,
ya̮thā̱꜔꜒ ca̮ a̮nu̱ppa̱nna̱ssa̮ u̮pe̱kkhā̱꜔꜒-\\ sa̱꜔꜒mbo̱jjha̱ṅga̱ssa̮ u̱ppā̱do̱ ho̱꜔꜒ti̮ ta̱ñca̮ pa̮jā̱nā̱ti͓,
ya̮thā̱꜔꜒ ca̮ u̱ppa̱nna̱ssa̮ u̮pe̱kkhā̱꜔꜒-sa̱꜔꜒mbo̱jjha̱ṅga̱ssa̮ bhā̱va̮nā̱ya̮ pā̱ri̮pū̱rī̱ ho̱꜔꜒ti̮ ta̱ñca̮ pa̮jā̱nā̱ti͓.

I̮ti̮ a̱jjha̱tta̱ṁ vā̱ dha̱mme̱su̮ dha̱mmā̱nu̮pa̱ssī̱꜔꜒ vi̮ha̮ra̮ti͓,
ba̮hi̱ddhā̱ vā̱ dha̱mme̱su̮ dha̱mmā̱nu̮pa̱ssī̱꜔꜒ vi̮ha̮ra̮ti͓,
a̱jjha̱tta̮-ba̮hi̱ddhā̱ vā̱ dha̱mme̱su̮ dha̱mmā̱nu̮pa̱ssī̱꜔꜒ vi̮ha̮ra̮ti͓.
Sa̮mu̮da̮ya̮-dha̱mmā̱nu̮pa̱ssī̱꜔꜒ vā̱ dha̱mme̱su̮ vi̮ha̮ra̮ti͓,
va̮ya̮-dha̱mmā̱nu̮pa̱ssī̱꜔꜒ vā̱ dha̱mme̱su̮ vi̮ha̮ra̮ti͓,
sa̮mu̮da̮ya̮-va̮ya̮-\\ dha̱mmā̱nu̮pa̱ssī̱꜔꜒ vā̱ dha̱mme̱su̮ vi̮ha̮ra̮ti͓.
‘A̱tthi̮ dha̱mmā̱’ti̮ vā̱ pa̮na̱ssa̮ sa̮ti̮ pa̱ccu̮pa̱ṭṭhi̮tā̱ ho̱꜔꜒ti͓
yā̱va̮de̱va̮ ñā̱ṇa̮ma̱ttā̱ya̮ pa̮ṭi̱ssa̮ti̮ma̱ttā̱ya̮, a̮ni̱ssi̮to̱ ca̮ vi̮ha̮ra̮ti͓,
na̮ ca̮ ki̱ñci̮ lo̱ke̱ u̮pā̱di̮ya̮ti̮. E̱va̱mpi̮ kho̱꜔꜒, bhi̱kkha̮ve̱, bhi̱kkhu͓
dha̱mme̱su̮ dha̱mmā̱nu̮pa̱ssī̱꜔꜒ vi̮ha̮ra̮ti̮ sa̱tta̮su̮ bo̱jjha̱ṅge̱su͓.

\enlargethispage{\baselineskip}

\instr{Bojjhaṅgapabbaṁ niṭṭhitaṁ.}

\englishPage
\section*{The Truths}% starred, to save an entry in the TOC

Again, bhikkhus, a bhikkhu dwells contemplating phenomena in phenomena in terms
of the Four Noble Truths.

And how, bhikkhus, does a bhikkhu dwell contemplating phenomena in phenomena in
terms of the Four Noble Truths?

Here, bhikkhus, a bhikkhu understands as it really is: `This is suffering. The
is the origin of suffering. This is the cessation of suffering. This is the way
leading to the cessation of suffering.'

\section{Suffering}

And what, bhikkhus, is the noble truth of suffering? Birth is suffering; ageing
is suffering; death is suffering; sorrow, lamentation, pain, grief, and despair
are suffering; union with what is displeasing is suffering; separation from what
is pleasing is suffering; not to get what one wants is suffering; in brief, the
five aggregates subject to clinging are suffering.

And what, bhikkhus, is birth? The birth of beings into the vaious orders of
beings, their coming to birth, precipitation [in a womb], generation, the
manifestation of the aggregates, obtaining the bases for contact -- this is
called birth.

And what, bhikkhus, is ageing? The ageing of beings in the various orders of
beings, their old age, brokenness of teeth, greyness of hair, wrinkling of skin,
deline of life, weakness of faculties -- this is called ageing.

\paliPage

\section*{Saccapabba}

Pu̮na̮ ca̮pa̮ra̱ṁ, bhi̱kkha̮ve̱, bhi̱kkhu̮ dha̱mme̱su̮ dha̱mmā̱nu̮pa̱ssī̱꜔꜒ vi̮ha̮ra̮ti̮ ca̮tū̱su͓
ariyasaccesu.

Ka̮tha̱꜔꜒ñca̮ pa̮na̮, bhi̱kkha̮ve̱, bhi̱kkhu̮ dha̱mme̱su̮ dha̱mmā̱nu̮pa̱ssī̱꜔꜒ vi̮ha̮ra̮ti̮ ca̮tū̱su͓
ariyasaccesu?

I̮dha̮, bhi̱kkha̮ve̱, bhi̱kkhu̮ ‘i̮da̱ṁ du̱kkha̱꜔꜒n’ti̮ ya̮thā̱꜔꜒bhū̱ta̱ṁ pa̮jā̱nā̱ti̮, ‘a̮ya͓ṁ
du̱kkha̮-sa̮mu̮da̮yo̱’ti̮ ya̮thā̱꜔꜒bhū̱ta̱ṁ pa̮jā̱nā̱ti̮, ‘a̮ya̱ṁ du̱kkha̮-ni̮ro̱dho̱’ti̮ ya̮thā̱꜔꜒bhū̱ta͓ṁ
pa̮jā̱nā̱ti̮, ‘a̮ya̱ṁ du̱kkha̮-ni̮ro̱dha̮-gā̱mi̮nī̱ pa̮ṭi̮pa̮dā̱’ti̮ ya̮thā̱꜔꜒bhū̱ta̱ṁ pa̮jā̱nā̱ti͓.

\section*{Dukkha-sacca}

Ka̮ta̮ma̱ñca̮, bhi̱kkha̮ve̱, du̱kkha̱꜔꜒ṁ a̮ri̮ya̮sa̱cca̱ṁ? Jā̱ti̮pi̮ du̱kkhā̱꜔꜒, ja̮rā̱pi̮ du̱kkhā͓꜔꜒,
ma̮ra̮ṇa̱mpi̮ du̱kkha̱꜔꜒ṁ, so̱꜔꜒ka̮-pa̮ri̮de̱va̮-du̱kkha̮-\\ do̱ma̮na̱ss'u̮pā̱yā̱sā̱꜔꜒pi̮ du̱kkhā̱꜔꜒, a̱ppi̮ye̱hi͓
sa̱꜔꜒mpa̮yo̱go̱ du̱kkho̱꜔꜒, pi̮ye̱hi̮ vi̱ppa̮yo̱go̱ du̱kkho̱꜔꜒, ya̱mpi̱ccha̱꜔꜒ṁ na̮ la̮bha̮ti̮ ta̱mpi̮ du̱kkha͓꜔꜒ṁ,
sa̱꜔꜒ṅkhi̱tte̱na̮ pa̱ñcu̮pā̱dā̱na̱-kkha̱꜔꜒ndhā̱ du̱kkhā͓꜔꜒.

Ka̮ta̮mā̱ ca̮, bhi̱kkha̮ve̱, jā̱ti̮? Yā̱ te̱sa̱꜔꜒ṁ te̱sa̱꜔꜒ṁ sa̱ttā̱na̱ṁ ta̱mhi̮ ta̱mhi̮ sa̱tta̮ni̮kā̱ye̱ jā̱ti͓
sa̱꜔꜒ñjā̱ti̮ o̱kka̱nti̮ a̮bhi̮ni̱bba̱tti̮ kha̱꜔꜒ndhā̱na̱ṁ pā̱tu̮bhā̱vo̱ ā̱ya̮ta̮nā̱na̱ṁ pa̮ṭi̮lā̱bho̱, a̮ya͓ṁ
vu̱cca̮ti̮, bhi̱kkha̮ve̱, jā̱ti͓.

Ka̮ta̮mā̱ ca̮, bhi̱kkha̮ve̱, ja̮rā̱? Yā̱ te̱sa̱꜔꜒ṁ te̱sa̱꜔꜒ṁ sa̱ttā̱na̱ṁ ta̱mhi̮ ta̱mhi̮ sa̱tta̮ni̮kā̱ye̱ ja̮rā͓
jī̱ra̮ṇa̮tā̱ kha̱꜔꜒ṇḍi̱cca̱ṁ pā̱li̱cca̱ṁ va̮li̱tta̮ca̮tā̱ ā̱yu̮no̱ sa̱꜔꜒ṁhā̱꜔꜒ni̮ i̱ndri̮yā̱na̱ṁ pa̮ri̮pā̱ko̱, a̮ya͓ṁ
vu̱cca̮ti̮, bhi̱kkha̮ve̱, ja̮rā͓.

\englishPage

And what, bhikkhus, is death? The passing of beings out of the various orders of
beings, their passing away, dissolution, disappearance, dying, death, completion
of time, dissolution of aggregates, laying down of the body, the cutting off of
the life faculty -- this is called death.

And what, bhikkhus, is sorrow? Bhikkhus, for one who has encountered some
misfortune or is affected by some painful state, there is sorrow, sorrowing,
sorrowfulness, inner sorrow, inner sorriness -- this is called sorrow.

And what, bhikkhus, is lamentation? Bhikkhus, for one who has encountered some
misfortune or is affected by some painful state, there is wail and lament,
wailing and lamenting, bewailing and lamentation -- this is called lamentation.

And what, bhikkhus, is pain? That, bhikkhus, which is bodily pain, bodily
discomfort, painful, uncomfortable feeling born of bodily contact -- this is
called pain.

And what, bhikkhus, is grief? That, bhikkhus, which is mental pain, mental
discomfort, painful, uncomfortable feeling born of mental contact -- this is
called grief.

And what, bhikkhus, is despair? Bhikkhus, for one who has encountered some
misfortune or is affected by some painful state, there is the trouble and
despair, the tribulation and desperation -- this is called despair.

And what, bhikkhus, is `union with what is displeasing is suffering'? Here,
bhikkhus, for one who has undesireable, unlovely, disagreeable forms, sounds,
ordours, tastes, and tactile objects;

\paliPage

Ka̮ta̮ma̱ñca̮, bhi̱kkha̮ve̱, ma̮ra̮ṇa̱ṁ? Ya̱ṁ te̱sa̱꜔꜒ṁ te̱sa̱꜔꜒ṁ sa̱ttā̱na̱ṁ ta̱mhā̱꜔꜒ ta̱mhā̱꜔꜒ sa̱tta̮ni̮kā̱yā͓
cu̮ti̮ ca̮va̮na̮tā̱ bhe̱do̱ a̱nta̮ra̮dhā̱na̱ṁ ma̱ccu̮ ma̮ra̮ṇa̱ṁ kā̱la̮ki̮ri̮yā̱ kha̱꜔꜒ndhā̱na̱ṁ bhe̱do͓
ka̱ḷe̱va̮ra̱ssa̮ ni̱kkhe̱꜔꜒po̱ jī̱vi̮ti̱ndri̮ya̱ss'u̮pa̱cche̱꜔꜒do̱, i̮da̱ṁ vu̱cca̮ti̮, bhi̱kkha̮ve̱, ma̮ra̮ṇa͓ṁ.

Ka̮ta̮mo̱ ca̮, bhi̱kkha̮ve̱, so̱꜔꜒ko̱? Yo̱ kho̱꜔꜒, bhi̱kkha̮ve̱, a̱ñña̮ta̮r'a̱ñña̮ta̮re̱na̮ bya̮sa̮ne̱na͓
sa̮ma̱nnā̱ga̮ta̱ssa̮ a̱ñña̮ta̮r'a̱ñña̮ta̮re̱na̮ du̱kkha̮-dha̱mme̱na̮ phu̱ṭṭha̱ssa̮ so̱꜔꜒ko̱ so̱꜔꜒ca̮nā͓
so̱꜔꜒ci̮ta̱tta̱ṁ a̱nto̱so̱꜔꜒ko̱ a̱nto̱pa̮ri̮so̱꜔꜒ko̱, a̮ya̱ṁ vu̱cca̮ti̮, bhi̱kkha̮ve̱, so̱꜔꜒ko͓.

Ka̮ta̮mo̱ ca̮, bhi̱kkha̮ve̱, pa̮ri̮de̱vo̱? Yo̱ kho̱꜔꜒, bhi̱kkha̮ve̱, a̱ñña̮ta̮r'a̱ñña̮ta̮re̱na͓
bya̮sa̮ne̱na̮ sa̮ma̱nnā̱ga̮ta̱ssa̮ a̱ñña̮ta̮r'a̱ñña̮ta̮re̱na̮ du̱kkha̮-dha̱mme̱na̮ phu̱ṭṭha̱ssa͓
ā̱de̱vo̱ pa̮ri̮de̱vo̱ ā̱de̱va̮nā̱ pa̮ri̮de̱va̮nā̱ ā̱de̱vi̮ta̱tta̱ṁ pa̮ri̮de̱vi̮ta̱tta̱ṁ, a̮ya͓ṁ
vuccati, bhikkhave, paridevo.

Ka̮ta̮ma̱ñca̮, bhi̱kkha̮ve̱, du̱kkha̱꜔꜒ṁ? Ya̱ṁ kho̱꜔꜒, bhi̱kkha̮ve̱, kā̱yi̮ka̱ṁ du̱kkha̱꜔꜒ṁ kā̱yi̮ka͓ṁ
a̮sā̱꜔꜒ta̱ṁ kā̱ya̮-sa̱꜔꜒mpha̱ssa̮ja̱ṁ du̱kkha̱꜔꜒ṁ a̮sā̱꜔꜒ta̱ṁ ve̱da̮yi̮ta̱ṁ, i̮da̱ṁ vu̱cca̮ti̮, bhi̱kkha̮ve͓,
du̱kkha͓꜔꜒ṁ.

Ka̮ta̮ma̱ñca̮, bhi̱kkha̮ve̱, do̱ma̮na̱ssa̱꜔꜒ṁ? Ya̱ṁ kho̱꜔꜒, bhi̱kkha̮ve̱, ce̱ta̮si̮ka̱ṁ du̱kkha͓꜔꜒ṁ
ce̱ta̮si̮ka̱ṁ a̮sā̱꜔꜒ta̱ṁ ma̮no̱-sa̱꜔꜒mpha̱ssa̮ja̱ṁ du̱kkha̱꜔꜒ṁ a̮sā̱꜔꜒ta̱ṁ ve̱da̮yi̮ta̱ṁ, i̮da̱ṁ vu̱cca̮ti͓,
bhi̱kkha̮ve̱, do̱ma̮na̱ssa͓꜔꜒ṁ.

Ka̮ta̮mo̱ ca̮, bhi̱kkha̮ve̱, u̮pā̱yā̱so̱꜔꜒? Yo̱ kho̱꜔꜒, bhi̱kkha̮ve̱, a̱ñña̮ta̮r'a̱ñña̮ta̮re̱na̮ bya̮sa̮ne̱na͓
sa̮ma̱nnā̱ga̮ta̱ssa̮ a̱ñña̮ta̮r'a̱ñña̮ta̮re̱na̮ du̱kkha̮-dha̱mme̱na̮ phu̱ṭṭha̱ssa̮ ā̱yā̱so̱꜔꜒ u̮pā̱yā̱so͓꜔꜒
ā̱yā̱si̮ta̱tta̱ṁ u̮pā̱yā̱si̮ta̱tta̱ṁ, a̮ya̱ṁ vu̱cca̮ti̮, bhi̱kkha̮ve̱, u̮pā̱yā̱so͓꜔꜒.

Ka̮ta̮mo̱ ca̮, bhi̱kkha̮ve̱, a̱ppi̮ye̱hi̮ sa̱꜔꜒mpa̮yo̱go̱ du̱kkho̱꜔꜒? I̮dha̮ ya̱ssa̮ te̱ ho̱꜔꜒nti̮ a̮ni̱ṭṭhā͓꜔꜒
a̮ka̱ntā̱ a̮ma̮nā̱pā̱ rū̱pā̱ sa̱ddā̱ ga̱ndhā̱ ra̮sā̱꜔꜒ pho̱ṭṭha̱bbā̱ dha̱mmā͓,

\englishPage

or for one who has those who do not desire his welfare, his benefit, his
comfort, and his security -- (and then) having meetings, assembly, connection,
and mixing with them: this, bhikkhus, is called `union with what is displeasing
is suffering'.

And what, bhikkhus, is `separation from what is pleasing is suffering'? Here,
bhikkhus, for one who has desirable, lovely, agreeable forms, sounds, ordours,
tastes, and tactile objects; or, for one who has those who do desire his
welfare, his benefit, his comfort and his security -- mothers, fathers,
brothers, or sisters; friends, companions, or blood relatives -- (and then) not
having meetings, assembly, connection, and mixing with them: this, bhikkhus, is
called `separation from what is pleasing is suffering'?

And what, bhikkhus, is `not to get what one wants is suffeing'? To beings
subject to birth there comes the wish: `Oh, that we were not subject to birth!
That birth would not come to us!' But this is not to be obtained by wishing;
this is `not to get what one wants is suffering.'

To beings subject to ageing there comes the wish: `Oh, that we were not subject
to ageing! That ageing would not come to us!' But this is not to be obtained by
wishing; this is `not to get what one wants is suffering.'

To beings subject to sickness there comes the wish: `Oh, that we were not
subject to sickness! That sickness would not come to us!' But this is not to be
obtained by wishing; this is `not to get what one wants is suffering.'

To beings subject to death there comes the wish: `Oh, that we were not subject
to death! That death would not come to us!' But this is not to be obtained by
wishing; this is `not to get what one wants is suffering.'

\paliPage

ye̱ vā̱ pa̮na̱ssa̮ te̱ ho̱꜔꜒nti̮ a̮na̱ttha̮kā̱mā̱ a̮hi̮ta̮kā̱mā̱ a̮phā̱꜔꜒su̮ka̮kā̱mā̱ a̮yo̱ga̱kkhe̱꜔꜒ma̮kā̱mā̱, yā͓
te̱hi̮ sa̱ddhi̱ṁ sa̱꜔꜒ṅga̮ti̮ sa̮mā̱ga̮mo̱ sa̮mo̱dhā̱na̱ṁ mi̱ssī̱꜔꜒bhā̱vo̱, a̮ya̱ṁ vu̱cca̮ti̮, bhi̱kkha̮ve͓,
appiyehi sa꜔꜒mpayogo dukkho꜔꜒.

Ka̮ta̮mo̱ ca̮, bhi̱kkha̮ve̱, pi̮ye̱hi̮ vi̱ppa̮yo̱go̱ du̱kkho̱꜔꜒? I̮dha̮ ya̱ssa̮ te̱ ho̱꜔꜒nti̮ i̱ṭṭhā̱꜔꜒ ka̱ntā͓
ma̮nā̱pā̱ rū̱pā̱ sa̱ddā̱ ga̱ndhā̱ ra̮sā̱꜔꜒ pho̱ṭṭha̱bbā̱ dha̱mmā̱, ye̱ vā̱ pa̮na̱ssa̮ te̱ ho̱꜔꜒nti͓
a̱ttha̮kā̱mā̱ hi̮ta̮kā̱mā̱ phā̱꜔꜒su̮ka̮kā̱mā̱ yo̱ga̱kkhe̱꜔꜒ma̮kā̱mā̱ mā̱tā̱ vā̱ pi̮tā̱ vā̱ bhā̱tā̱ vā̱ bha̮gi̮nī͓
vā̱ mi̱ttā̱ vā̱ a̮ma̱ccā̱ vā̱ ñā̱ti̮sā̱꜔꜒lo̱hi̮tā̱ vā̱, yā̱ te̱hi̮ sa̱ddhi̱ṁ a̮sa̱꜔꜒ṅga̮ti̮ a̮sa̮mā̱ga̮mo͓
a̮sa̮mo̱dhā̱na̱ṁ a̮mi̱ssī̱꜔꜒bhā̱vo̱, a̮ya̱ṁ vu̱cca̮ti̮, bhi̱kkha̮ve̱, pi̮ye̱hi̮ vi̱ppa̮yo̱go̱ du̱kkho͓꜔꜒.

Ka̮ta̮ma̱ñca̮, bhi̱kkha̮ve̱, ya̱mpi̱ccha̱꜔꜒ṁ na̮ la̮bha̮ti̮ ta̱mpi̮ du̱kkha̱꜔꜒ṁ? Jā̱ti̮dha̱mmā̱na͓ṁ,
bhi̱kkha̮ve̱, sa̱ttā̱na̱ṁ e̱va̱ṁ i̱cchā̱꜔꜒ u̱ppa̱jja̮ti̮: ‘a̮ho̱꜔꜒ va̮ta̮ ma̮ya̱ṁ na̮ jā̱ti̮dha̱mmā̱ a̱ssā̱꜔꜒ma͓,
na̮ ca̮ va̮ta̮ no̱ jā̱ti̮ ā̱ga̱cche̱꜔꜒yyā̱’ti̮. Na̮ kho̱꜔꜒ pa̮ne̱ta̱ṁ i̱cchā̱꜔꜒ya̮ pa̱tta̱bba̱ṁ, i̮da̱mpi͓
ya̱mpi̱ccha̱꜔꜒ṁ na̮ la̮bha̮ti̮ ta̱mpi̮ du̱kkha͓꜔꜒ṁ.

Ja̮rā̱dha̱mmā̱na̱ṁ, bhi̱kkha̮ve̱, sa̱ttā̱na̱ṁ e̱va̱ṁ i̱cchā̱꜔꜒ u̱ppa̱jja̮ti̮: ‘a̮ho̱꜔꜒ va̮ta̮ ma̮ya̱ṁ na͓
ja̮rā̱dha̱mmā̱ a̱ssā̱꜔꜒ma̮, na̮ ca̮ va̮ta̮ no̱ ja̮rā̱ ā̱ga̱cche̱꜔꜒yyā̱’ti̮. Na̮ kho̱꜔꜒ pa̮ne̱ta̱ṁ i̱cchā̱꜔꜒ya͓
pa̱tta̱bba̱ṁ, i̮da̱mpi̮ ya̱mpi̱ccha̱꜔꜒ṁ na̮ la̮bha̮ti̮ ta̱mpi̮ du̱kkha͓꜔꜒ṁ.

Byā̱dhi̮dha̱mmā̱na̱ṁ, bhi̱kkha̮ve̱, sa̱ttā̱na̱ṁ e̱va̱ṁ i̱cchā̱꜔꜒ u̱ppa̱jja̮ti̮ ‘a̮ho̱꜔꜒ va̮ta̮ ma̮ya̱ṁ na͓
byā̱dhi̮dha̱mmā̱ a̱ssā̱꜔꜒ma̮, na̮ ca̮ va̮ta̮ no̱ byā̱dhi̮ ā̱ga̱cche̱꜔꜒yyā̱’ti̮. Na̮ kho̱꜔꜒ pa̮ne̱ta̱ṁ i̱cchā̱꜔꜒ya͓
pa̱tta̱bba̱ṁ, i̮da̱mpi̮ ya̱mpi̱ccha̱꜔꜒ṁ na̮ la̮bha̮ti̮ ta̱mpi̮ du̱kkha͓꜔꜒ṁ.

Ma̮ra̮ṇa̮dha̱mmā̱na̱ṁ, bhi̱kkha̮ve̱, sa̱ttā̱na̱ṁ e̱va̱ṁ i̱cchā̱꜔꜒ u̱ppa̱jja̮ti̮ ‘a̮ho̱꜔꜒ va̮ta̮ ma̮ya̱ṁ na͓
ma̮ra̮ṇa̮dha̱mmā̱ a̱ssā̱꜔꜒ma̮, na̮ ca̮ va̮ta̮ no̱ ma̮ra̮ṇa̱ṁ ā̱ga̱cche̱꜔꜒yyā̱’ti̮. Na̮ kho̱꜔꜒ pa̮ne̱ta̱ṁ i̱cchā̱꜔꜒ya͓
pa̱tta̱bba̱ṁ, i̮da̱mpi̮ ya̱mpi̱ccha̱꜔꜒ṁ na̮ la̮bha̮ti̮ ta̱mpi̮ du̱kkha͓꜔꜒ṁ.

\englishPage

To beings subject to sorrow, lamentation, pain, grief, and despair, there comes
the wish: `Oh, that we were not subject to sorrow, lamentation, pain, grief, and
despair! That sorrow, lamentation, pain, grief, and despair would not come to
us!' But this is not to be obtained by wishing; this is `not to get what one
wants is suffering.'

And what, bhikkhus, are the five aggregates subject to clinging that, in brief,
are suffering? They are: the material form aggregate subject to clinging, the
feeling aggregate subject to clinging, the perception aggregate subject to
clinging, the volitional formations aggregate subject to clinging, the
consciousness aggregate subject to clinging. These are the five aggregates
subject to clinging that, in brief, are suffering.

This, bhikkhus, is called the noble truth of suffering.

\section{Origin}

And what, bhikkhus, is the noble truth of the origin of suffering? It is this
craving, which leads to renewed existence, accompanied by delight and lust,
seeking delight in this and that; that is, craving for sensual pleasures,
craving for existence, and craving for non-existence.

Now where, bhikkhus, does that craving when it is arising arise? When settling
where does it settle? That in the world which is pleasant and agreeable -- here
this craving when it is arising arises, here when settling it settles.

In the world what is pleasant and agreeable? In the world the eye \ldots{} the
ear \ldots{} the nose \ldots{} the tongue \ldots{} the body \ldots{} the mind is
likeable and pleasing -- here this craving when it is arising arises, here when
settling it settles.

In the world the forms \ldots{} the sounds \ldots{} the ordours \ldots{} the
tastes \ldots{} the tactile objects \ldots{} the mental phenomena is likeable
and pleasing -- here this craving when it is arising arises, here when settling
it settles.

\paliPage

So̱꜔꜒ka̮-pa̮ri̮de̱va̮-du̱kkha̮-do̱ma̮na̱ss'u̮pā̱yā̱sa̮-dha̱mmā̱na̱ṁ, bhi̱kkha̮ve̱, sa̱ttā̱na̱ṁ e̱va̱ṁ i̱cchā͓꜔꜒
u̱ppa̱jja̮ti̮ ‘a̮ho̱꜔꜒ va̮ta̮ ma̮ya̱ṁ na̮ so̱꜔꜒ka̮-pa̮ri̮de̱va͓-\\
du̱kkha̮-do̱ma̮na̱ss'u̮pā̱yā̱sa̮-dha̱mmā̱ a̱ssā̱꜔꜒ma̮, na̮ ca̮ va̮ta̮ no̱ so̱꜔꜒ka͓-\\
pa̮ri̮de̱va̮-du̱kkha̮-do̱ma̮na̱ss'u̮pā̱yā̱sā̱꜔꜒ ā̱ga̱cche̱꜔꜒yyu̱n’ti̮. Na̮ kho̱꜔꜒ pa̮ne̱ta̱ṁ i̱cchā̱꜔꜒ya͓
pa̱tta̱bba̱ṁ, i̮da̱mpi̮ ya̱mpi̱ccha̱꜔꜒ṁ na̮ la̮bha̮ti̮ ta̱mpi̮ du̱kkha͓꜔꜒ṁ.

Ka̮ta̮me̱ ca̮, bhi̱kkha̮ve̱, sa̱꜔꜒ṅkhi̱tte̱na̮ pa̱ñcu̮pā̱dā̱na̱-kkha̱꜔꜒ndhā̱ du̱kkhā̱꜔꜒? Se̱꜔꜒yya̮thī̱꜔꜒da͓ṁ,
rū̱pū̱pā̱dā̱na̱-kkha̱꜔꜒ndho̱, ve̱da̮nū̱pā̱dā̱na̱-kkha̱꜔꜒ndho̱, sa̱꜔꜒ññū̱pā̱dā̱na̱-kkha̱꜔꜒ndho͓,
sa̱꜔꜒ṅkhā̱꜔꜒rū̱pā̱dā̱na̱-kkha̱꜔꜒ndho̱, vi̱ññā̱ṇū̱pā̱dā̱na̱-kkha̱꜔꜒ndho̱. I̮me̱ vu̱cca̱nti̮, bhi̱kkha̮ve͓,
sa̱꜔꜒ṅkhi̱tte̱na̮ pa̱ñcu̮pā̱dā̱na̱-kkha̱꜔꜒ndhā̱ du̱kkhā͓꜔꜒.

I̮da̱ṁ vu̱cca̮ti̮, bhi̱kkha̮ve̱, du̱kkha̱꜔꜒ṁ a̮ri̮ya̮sa̱cca͓ṁ.

\section*{Samudaya-sacca}

Ka̮ta̮ma̱ñca̮, bhi̱kkha̮ve̱, du̱kkha̮-sa̮mu̮da̮ya̱ṁ a̮ri̮ya̮sa̱cca̱ṁ? Yā̱ya̱ṁ ta̱ṇhā̱꜔꜒ po̱no̱bbha̮vi̮kā͓
na̱ndi̮-rā̱ga̮-sa̮ha̮ga̮tā̱ ta̱tra̮-ta̱trā̱bhi̮na̱ndi̮nī̱, se̱꜔꜒yya̮thī̱꜔꜒da̱ṁ, kā̱ma̮ta̱ṇhā̱꜔꜒ bha̮va̮ta̱ṇhā͓꜔꜒
vi̮bha̮va̮ta̱ṇhā͓꜔꜒.

Sā̱꜔꜒ kho̱꜔꜒ pa̮ne̱sā̱꜔꜒, bhi̱kkha̮ve̱, ta̱ṇhā̱꜔꜒ ka̱ttha̮ u̱ppa̱jja̮mā̱nā̱ u̱ppa̱jja̮ti̮, ka̱ttha̮ ni̮vī̱sa̮mā̱nā͓
ni̮vī̱sa̮ti̮? Ya̱ṁ lo̱ke̱ pi̮ya̮rū̱pa̱ṁ sā̱꜔꜒ta̮rū̱pa̱ṁ, e̱tthe̱꜔꜒sā̱꜔꜒ ta̱ṇhā̱꜔꜒ u̱ppa̱jja̮mā̱nā̱ u̱ppa̱jja̮ti͓,
e̱ttha̮ ni̮vī̱sa̮mā̱nā̱ ni̮vī̱sa̮ti͓.

\enlargethispage{\baselineskip}

Ki̱ñca̮ lo̱ke̱ pi̮ya̮rū̱pa̱ṁ sā̱꜔꜒ta̮rū̱pa̱ṁ? Ca̱kkhu̮ lo̱ke̱ pi̮ya̮rū̱pa̱ṁ sā̱꜔꜒ta̮rū̱pa̱ṁ, e̱tthe̱꜔꜒sā̱꜔꜒ ta̱ṇhā͓꜔꜒
u̱ppa̱jja̮mā̱nā̱ u̱ppa̱jja̮ti̮, e̱ttha̮ ni̮vī̱sa̮mā̱nā̱ ni̮vī̱sa̮ti̮. So̱꜔꜒ta̱ṁ lo̱ke͓
\ldots{} ghānaṁ loke \ldots{} jivhā꜔꜒ loke \ldots{} kāyo loke \ldots{} mano loke
pi̮ya̮rū̱pa̱ṁ sā̱꜔꜒ta̮rū̱pa̱ṁ, e̱tthe̱꜔꜒sā̱꜔꜒ ta̱ṇhā̱꜔꜒ u̱ppa̱jja̮mā̱nā̱ u̱ppa̱jja̮ti̮, e̱ttha̮ ni̮vī̱sa̮mā̱nā͓
ni̮vī̱sa̮ti͓.

Rū̱pā̱ lo̱ke̱ \ldo̱ts{} sa̱ddā̱ lo̱ke̱ \ldo̱ts{} ga̱ndhā̱ lo̱ke̱ \ldo̱ts{} ra̮sā̱꜔꜒ lo̱ke̱ \ldo͓ts{}
pho̱ṭṭha̱bbā̱ lo̱ke̱ \ldo̱ts{} dha̱mmā̱ lo̱ke̱ pi̮ya̮rū̱pa̱ṁ sā̱꜔꜒ta̮rū̱pa̱ṁ, e̱tthe̱꜔꜒sā̱꜔꜒ ta̱ṇhā͓꜔꜒
u̱ppa̱jja̮mā̱nā̱ u̱ppa̱jja̮ti̮, e̱ttha̮ ni̮vī̱sa̮mā̱nā̱ ni̮vī̱sa̮ti͓.

\englishPage

In the world the eye-consciousness \ldots{} ear-consciousness \ldots{} nose-consciousness
 \ldots{} tongue-consciousness \ldots{} body-consciousness \ldots{} mind-consciousness is likeable
and pleasing -- here this craving when it is arising arises, here when settling
it settles.

In the world the eye-contact \ldots{} ear-contact \ldots{} nose-contact \ldots{}
tongue-contact \ldots{} body-contact \ldots{} mind-contact is likeable and
pleasing -- here this craving when it is arising arises, here when settling it
settles.

In the world feeling born of eye-contact \ldots{} feeling born of ear-contact
\ldots{} feeling born of nose-contact \ldots{} feeling born of tongue-contact
\ldots{} feeling born of body-contact \ldots{} feeling born of mind-contact is
likeable and pleasing -- here this craving when it is arising arises, here when
settling it settles.

In the world perception of forms \ldots{} perception of sounds \ldots{}
perception of odours \ldots{} perception of tastes \ldots{} perception of
tactile objects \ldots{} perception of mental phenomena is likeable and pleasing
-- here this craving when it is arising arises, here when settling it settles.

In the world volition regarding forms \ldots{} volition regarding sounds
\ldots{} volition regarding odours \ldots{} volition regarding tastes \ldots{}
volition regarding tactile objects \ldots{} volition regarding mental phenomena
is likeable and pleasing -- here this craving when it is arising arises, here
when settling it settles.

In the world craving for forms \ldots{} craving for sounds \ldots{} craving for
odours \ldots{} craving for tastes \ldots{} craving for tactile objects \ldots{}
craving for mental phenomena is likeable and pleasing -- here this craving when
it is arising arises, here when settling it settles.

\paliPage

Ca̱kkhu̮-vi̱ññā̱ṇa̱ṁ lo̱ke̱ \ldo̱ts{} so̱꜔꜒ta̮-vi̱ññā̱ṇa̱ṁ lo̱ke̱ \ldo̱ts{} ghā̱na̮-vi̱ññā̱ṇa̱ṁ lo̱ke͓
\ldots{} jivhā꜔꜒-viññāṇaṁ loke \ldots{} kāya-viññāṇaṁ loke \ldots{} mano-viññāṇaṁ
lo̱ke̱ pi̮ya̮rū̱pa̱ṁ sā̱꜔꜒ta̮rū̱pa̱ṁ, e̱tthe̱꜔꜒sā̱꜔꜒ ta̱ṇhā̱꜔꜒ u̱ppa̱jja̮mā̱nā̱ u̱ppa̱jja̮ti̮, e̱ttha̮ ni̮vī̱sa̮mā̱nā͓
ni̮vī̱sa̮ti͓.

Ca̱kkhu̮-sa̱꜔꜒mpha̱sso̱꜔꜒ lo̱ke̱ \ldo̱ts{} so̱꜔꜒ta̮-sa̱꜔꜒mpha̱sso̱꜔꜒ lo̱ke̱ \ldo̱ts{} ghā̱na̮-sa̱꜔꜒mpha̱sso̱꜔꜒ lo̱ke͓
\ldots{} jivhā꜔꜒-sa꜔꜒mphasso꜔꜒ loke \ldots{} kāya-sa꜔꜒mphasso꜔꜒ loke \ldots{} mano-sa꜔꜒mphasso꜔꜒
lo̱ke̱ pi̮ya̮rū̱pa̱ṁ sā̱꜔꜒ta̮rū̱pa̱ṁ, e̱tthe̱꜔꜒sā̱꜔꜒ ta̱ṇhā̱꜔꜒ u̱ppa̱jja̮mā̱nā̱ u̱ppa̱jja̮ti̮, e̱ttha̮ ni̮vī̱sa̮mā̱nā͓
ni̮vī̱sa̮ti͓.

Ca̱kkhu̮-sa̱꜔꜒mpha̱ssa̮jā̱ ve̱da̮nā̱ lo̱ke̱ \ldo̱ts{} so̱꜔꜒ta̮-sa̱꜔꜒mpha̱ssa̮jā̱ ve̱da̮nā̱ lo̱ke̱ \ldo͓ts{}
ghā̱na̮-sa̱꜔꜒mpha̱ssa̮jā̱ ve̱da̮nā̱ lo̱ke̱ \ldo̱ts{} ji̱vhā̱꜔꜒-sa̱꜔꜒mpha̱ssa̮jā̱ ve̱da̮nā̱ lo̱ke̱ \ldo͓ts{}
kā̱ya̮-sa̱꜔꜒mpha̱ssa̮jā̱ ve̱da̮nā̱ lo̱ke̱ \ldo̱ts{} ma̮no̱-sa̱꜔꜒mpha̱ssa̮jā̱ ve̱da̮nā̱ lo̱ke̱ pi̮ya̮rū̱pa͓ṁ
sā̱꜔꜒ta̮rū̱pa̱ṁ, e̱tthe̱꜔꜒sā̱꜔꜒ ta̱ṇhā̱꜔꜒ u̱ppa̱jja̮mā̱nā̱ u̱ppa̱jja̮ti̮, e̱ttha̮ ni̮vī̱sa̮mā̱nā̱ ni̮vī̱sa̮ti͓.

Rū̱pa̮-sa̱꜔꜒ññā̱ lo̱ke̱ \ldo̱ts{} sa̱dda̮-sa̱꜔꜒ññā̱ lo̱ke̱ \ldo̱ts{} ga̱ndha̮-sa̱꜔꜒ññā̱ lo̱ke̱ \ldo͓ts{}
ra̮sa̮-sa̱꜔꜒ññā̱ lo̱ke̱ \ldo̱ts{} pho̱ṭṭha̱bba̮-sa̱꜔꜒ññā̱ lo̱ke̱ \ldo̱ts{} dha̱mma̮-sa̱꜔꜒ññā̱ lo̱ke̱ pi̮ya̮rū̱pa͓ṁ
sā̱꜔꜒ta̮rū̱pa̱ṁ, e̱tthe̱꜔꜒sā̱꜔꜒ ta̱ṇhā̱꜔꜒ u̱ppa̱jja̮mā̱nā̱ u̱ppa̱jja̮ti̮, e̱ttha̮ ni̮vī̱sa̮mā̱nā̱ ni̮vī̱sa̮ti͓.

Rū̱pa̮-sa̱꜔꜒ñce̱ta̮nā̱ lo̱ke̱ \ldo̱ts{} sa̱dda̮-sa̱꜔꜒ñce̱ta̮nā̱ lo̱ke̱ \ldo̱ts{} ga̱ndha̮-sa̱꜔꜒ñce̱ta̮nā̱ lo̱ke͓
\ldots{} rasa-sa꜔꜒ñcetanā loke \ldots{} phoṭṭhabba-sa꜔꜒ñcetanā loke \ldots{}
dha̱mma̮-sa̱꜔꜒ñce̱ta̮nā̱ lo̱ke̱ pi̮ya̮rū̱pa̱ṁ sā̱꜔꜒ta̮rū̱pa̱ṁ, e̱tthe̱꜔꜒sā̱꜔꜒ ta̱ṇhā̱꜔꜒ u̱ppa̱jja̮mā̱nā̱ u̱ppa̱jja̮ti͓,
e̱ttha̮ ni̮vī̱sa̮mā̱nā̱ ni̮vī̱sa̮ti͓.

Rū̱pa̮-ta̱ṇhā̱꜔꜒ lo̱ke̱ \ldo̱ts{} sa̱dda̮-ta̱ṇhā̱꜔꜒ lo̱ke̱ \ldo̱ts{} ga̱ndha̮-ta̱ṇhā̱꜔꜒ lo̱ke̱ \ldo͓ts{}
ra̮sa̮-ta̱ṇhā̱꜔꜒ lo̱ke̱ \ldo̱ts{} pho̱ṭṭha̱bba̮-ta̱ṇhā̱꜔꜒ lo̱ke̱ \ldo̱ts{} dha̱mma̮-ta̱ṇhā̱꜔꜒ lo̱ke̱ pi̮ya̮rū̱pa͓ṁ
sā̱꜔꜒ta̮rū̱pa̱ṁ, e̱tthe̱꜔꜒sā̱꜔꜒ ta̱ṇhā̱꜔꜒ u̱ppa̱jja̮mā̱nā̱ u̱ppa̱jja̮ti̮, e̱ttha̮ ni̮vī̱sa̮mā̱nā̱ ni̮vī̱sa̮ti͓.

\englishPage

In the world thought about forms \ldots{} thought about sounds \ldots{} thought
about odours \ldots{} thought about tastes \ldots{} thought about tactile
objects \ldots{} thought about mental phenomena is likeable and pleasing -- here
this craving when it is arising arises, here when settling it settles.

In the world examination of forms \ldots{} examination of sounds \ldots{}
examination of odours \ldots{} examination of tastes \ldots{} examination of
tactile objects \ldots{} examination of mental phenomena is likeable and
pleasing -- here this craving when it is arising arises, here when settling it
settles.

This, bhikkhus, is called the noble truth of the origin of suffering.

\section{Cessation}

And what, bhikkhus, is the noble truth of the cessation of suffering? It is the
remainderless fading away and cessation of that same craving, the giving up and
relinquishing of it, freedom from it, non-reliance on it.

Now where, bhikkhus, is that craving when it is being abandoned, abandoned? When
ceasing where does it cease? That in the world which is pleasant and agreeable
-- here this craving when it is being abandoned, abandoned, here when ceasing it
ceases.

In the world what is pleasant and agreeable? In the world the eye \ldots{} the
ear \ldots{} the nose \ldots{} the tongue \ldots{} the body \ldots{} the mind is
likeable and pleasing -- here this craving when it is being abandoned, abandoned,
here when ceasing it ceases.

In the world the forms \ldots{} the sounds \ldots{} the ordours \ldots{} the
tastes \ldots{} the tactile objects \ldots{} the mental phenomena is likeable
and pleasing -- here this craving when it is being abandoned, abandoned, here
when ceasing it ceases.

\paliPage

Rū̱pa̮-vi̮ta̱kko̱ lo̱ke̱ \ldo̱ts{} sa̱dda̮-vi̮ta̱kko̱ lo̱ke̱ \ldo̱ts{} ga̱ndha̮-vi̮ta̱kko̱ lo̱ke̱ \ldo͓ts{}
ra̮sa̮-vi̮ta̱kko̱ lo̱ke̱ \ldo̱ts{} pho̱ṭṭha̱bba̮-vi̮ta̱kko̱ lo̱ke̱ \ldo̱ts{} dha̱mma̮-vi̮ta̱kko̱ lo̱ke͓
pi̮ya̮rū̱pa̱ṁ sā̱꜔꜒ta̮rū̱pa̱ṁ, e̱tthe̱꜔꜒sā̱꜔꜒ ta̱ṇhā̱꜔꜒ u̱ppa̱jja̮mā̱nā̱ u̱ppa̱jja̮ti̮, e̱ttha̮ ni̮vī̱sa̮mā̱nā͓
ni̮vī̱sa̮ti͓.

Rū̱pa̮-vi̮cā̱ro̱ lo̱ke̱ \ldo̱ts{} sa̱dda̮-vi̮cā̱ro̱ lo̱ke̱ \ldo̱ts{} ga̱ndha̮-vi̮cā̱ro̱ lo̱ke̱ \ldo͓ts{}
ra̮sa̮-vi̮cā̱ro̱ lo̱ke̱ \ldo̱ts{} pho̱ṭṭha̱bba̮-vi̮cā̱ro̱ lo̱ke̱ \ldo̱ts{} dha̱mma̮-vi̮cā̱ro̱ lo̱ke͓
pi̮ya̮rū̱pa̱ṁ sā̱꜔꜒ta̮rū̱pa̱ṁ, e̱tthe̱꜔꜒sā̱꜔꜒ ta̱ṇhā̱꜔꜒ u̱ppa̱jja̮mā̱nā̱ u̱ppa̱jja̮ti̮, e̱ttha̮ ni̮vī̱sa̮mā̱nā͓
ni̮vī̱sa̮ti͓.

I̮da̱ṁ vu̱cca̮ti̮, bhi̱kkha̮ve̱, du̱kkha̮-sa̮mu̮da̮ya̱ṁ a̮ri̮ya̮sa̱cca͓ṁ.

\section*{Nirodha-sacca}

Ka̮ta̮ma̱ñca̮, bhi̱kkha̮ve̱, du̱kkha̮-ni̮ro̱dha̱ṁ a̮ri̮ya̮sa̱cca̱ṁ? Yo̱ ta̱ssā̱꜔꜒ye̱va̮ ta̱ṇhā̱꜔꜒ya͓
a̮se̱꜔꜒sa̮-vi̮rā̱ga̮-ni̮ro̱dho̱ cā̱go̱ pa̮ṭi̮ni̱ssa̱ggo̱ mu̱tti̮ a̮nā̱la̮yo͓.

Sā̱꜔꜒ kho̱꜔꜒ pa̮ne̱sā̱꜔꜒, bhi̱kkha̮ve̱, ta̱ṇhā̱꜔꜒ ka̱ttha̮ pa̮hī̱꜔꜒ya̮mā̱nā̱ pa̮hī̱꜔꜒ya̮ti̮, ka̱ttha̮ ni̮ru̱jjha̮mā̱nā͓
ni̮ru̱jjha̮ti̮? Ya̱ṁ lo̱ke̱ pi̮ya̮rū̱pa̱ṁ sā̱꜔꜒ta̮rū̱pa̱ṁ, e̱tthe̱꜔꜒sā̱꜔꜒ ta̱ṇhā̱꜔꜒ pa̮hī̱꜔꜒ya̮mā̱nā̱ pa̮hī̱꜔꜒ya̮ti͓,
e̱ttha̮ ni̮ru̱jjha̮mā̱nā̱ ni̮ru̱jjha̮ti͓.

Ki̱ñca̮ lo̱ke̱ pi̮ya̮rū̱pa̱ṁ sā̱꜔꜒ta̮rū̱pa̱ṁ? Ca̱kkhu̮ lo̱ke̱ pi̮ya̮rū̱pa̱ṁ sā̱꜔꜒ta̮rū̱pa̱ṁ, e̱tthe̱꜔꜒sā̱꜔꜒ ta̱ṇhā͓꜔꜒
pa̮hī̱꜔꜒ya̮mā̱nā̱ pa̮hī̱꜔꜒ya̮ti̮, e̱ttha̮ ni̮ru̱jjha̮mā̱nā̱ ni̮ru̱jjha̮ti̮. So̱꜔꜒ta̱ṁ lo̱ke̱ \ldo͓ts{}
ghā̱na̱ṁ lo̱ke̱ \ldo̱ts{} ji̱vhā̱꜔꜒ lo̱ke̱ \ldo̱ts{} kā̱yo̱ lo̱ke̱ \ldo̱ts{} ma̮no̱ lo̱ke͓
pi̮ya̮rū̱pa̱ṁ sā̱꜔꜒ta̮rū̱pa̱ṁ, e̱tthe̱꜔꜒sā̱꜔꜒ ta̱ṇhā̱꜔꜒ pa̮hī̱꜔꜒ya̮mā̱nā̱ pa̮hī̱꜔꜒ya̮ti̮, e̱ttha̮ ni̮ru̱jjha̮mā̱nā͓
nirujjhati.

Rū̱pā̱ lo̱ke̱ \ldo̱ts{} sa̱ddā̱ lo̱ke̱ \ldo̱ts{} ga̱ndhā̱ lo̱ke̱ \ldo̱ts{} ra̮sā̱꜔꜒ lo̱ke̱ \ldo͓ts{}
pho̱ṭṭha̱bbā̱ lo̱ke̱ \ldo̱ts{} dha̱mmā̱ lo̱ke̱ pi̮ya̮rū̱pa̱ṁ sā̱꜔꜒ta̮rū̱pa̱ṁ, e̱tthe̱꜔꜒sā̱꜔꜒ ta̱ṇhā͓꜔꜒
pa̮hī̱꜔꜒ya̮mā̱nā̱ pa̮hī̱꜔꜒ya̮ti̮, e̱ttha̮ ni̮ru̱jjha̮mā̱nā̱ ni̮ru̱jjha̮ti͓.

\englishPage

In the world the eye-consciousness \ldots{} ear-consciousness \ldots{}
nose-consciousness \ldots{} tongue-consciousness \ldots{} body-consciousness
\ldots{} mind-consciousness is likeable and pleasing -- here this craving when
it is being abandoned, abandoned, here when ceasing it ceases.

In the world the eye-contact \ldots{} ear-contact \ldots{} nose-contact \ldots{}
tongue-contact \ldots{} body-contact \ldots{} mind-contact is likeable and
pleasing -- here this craving when it is being abandoned, abandoned, here when
ceasing it ceases.

In the world feeling born of eye-contact \ldots{} feeling born of ear-contact
\ldots{} feeling born of nose-contact \ldots{} feeling born of tongue-contact
\ldots{} feeling born of body-contact \ldots{} feeling born of mind-contact is
likeable and pleasing -- here this craving when it is being abandoned, abandoned,
here when ceasing it ceases.

In the world perception of forms \ldots{} perception of sounds \ldots{}
perception of odours \ldots{} perception of tastes \ldots{} perception of
tactile objects \ldots{} perception of mental phenomena is likeable and pleasing
-- here this craving when it is being abandoned, abandoned, here when ceasing it
ceases.

In the world volition regarding forms \ldots{} volition regarding sounds
\ldots{} volition regarding odours \ldots{} volition regarding tastes \ldots{}
volition regarding tactile objects \ldots{} volition regarding mental phenomena
is likeable and pleasing -- here this craving when it is being abandoned,
abandoned, here when ceasing it ceases.

In the world craving for forms \ldots{} craving for sounds \ldots{} craving for
odours \ldots{} craving for tastes \ldots{} craving for tactile objects \ldots{}
craving for mental phenomena is likeable and pleasing -- here this craving when
it is being abandoned, abandoned, here when ceasing it ceases.

\paliPage

Ca̱kkhu̮-vi̱ññā̱ṇa̱ṁ lo̱ke̱ \ldo̱ts{} so̱꜔꜒ta̮-vi̱ññā̱ṇa̱ṁ lo̱ke̱ \ldo̱ts{} ghā̱na̮-vi̱ññā̱ṇa̱ṁ lo̱ke͓
\ldots{} jivhā꜔꜒-viññāṇaṁ loke \ldots{} kāya-viññāṇaṁ loke \ldots{} mano-viññāṇaṁ
lo̱ke̱ pi̮ya̮rū̱pa̱ṁ sā̱꜔꜒ta̮rū̱pa̱ṁ, e̱tthe̱꜔꜒sā̱꜔꜒ ta̱ṇhā̱꜔꜒ pa̮hī̱꜔꜒ya̮mā̱nā̱ pa̮hī̱꜔꜒ya̮ti̮, e̱ttha̮ ni̮ru̱jjha̮mā̱nā͓
nirujjhati.

Ca̱kkhu̮-sa̱꜔꜒mpha̱sso̱꜔꜒ lo̱ke̱ \ldo̱ts{} so̱꜔꜒ta̮-sa̱꜔꜒mpha̱sso̱꜔꜒ lo̱ke̱ \ldo̱ts{} ghā̱na̮-sa̱꜔꜒mpha̱sso̱꜔꜒ lo̱ke͓
\ldots{} jivhā꜔꜒-sa꜔꜒mphasso꜔꜒ loke \ldots{} kāya-sa꜔꜒mphasso꜔꜒ loke \ldots{} mano-sa꜔꜒mphasso꜔꜒
lo̱ke̱ pi̮ya̮rū̱pa̱ṁ sā̱꜔꜒ta̮rū̱pa̱ṁ, e̱tthe̱꜔꜒sā̱꜔꜒ ta̱ṇhā̱꜔꜒ pa̮hī̱꜔꜒ya̮mā̱nā̱ pa̮hī̱꜔꜒ya̮ti̮, e̱ttha̮ ni̮ru̱jjha̮mā̱nā͓
nirujjhati.

Ca̱kkhu̮-sa̱꜔꜒mpha̱ssa̮jā̱ ve̱da̮nā̱ lo̱ke̱ \ldo̱ts{} so̱꜔꜒ta̮-sa̱꜔꜒mpha̱ssa̮jā̱ ve̱da̮nā̱ lo̱ke̱ \ldo͓ts{}
ghā̱na̮-sa̱꜔꜒mpha̱ssa̮jā̱ ve̱da̮nā̱ lo̱ke̱ \ldo̱ts{} ji̱vhā̱꜔꜒-sa̱꜔꜒mpha̱ssa̮jā̱ ve̱da̮nā̱ lo̱ke̱ \ldo͓ts{}
kā̱ya̮-sa̱꜔꜒mpha̱ssa̮jā̱ ve̱da̮nā̱ lo̱ke̱ \ldo̱ts{} ma̮no̱-sa̱꜔꜒mpha̱ssa̮jā̱ ve̱da̮nā̱ lo̱ke̱ pi̮ya̮rū̱pa͓ṁ
sā̱꜔꜒ta̮rū̱pa̱ṁ, e̱tthe̱꜔꜒sā̱꜔꜒ ta̱ṇhā̱꜔꜒ pa̮hī̱꜔꜒ya̮mā̱nā̱ pa̮hī̱꜔꜒ya̮ti̮, e̱ttha̮ ni̮ru̱jjha̮mā̱nā̱ ni̮ru̱jjha̮ti͓.

Rū̱pa̮-sa̱꜔꜒ññā̱ lo̱ke̱ \ldo̱ts{} sa̱dda̮-sa̱꜔꜒ññā̱ lo̱ke̱ \ldo̱ts{} ga̱ndha̮-sa̱꜔꜒ññā̱ lo̱ke̱ \ldo͓ts{}
ra̮sa̮-sa̱꜔꜒ññā̱ lo̱ke̱ \ldo̱ts{} pho̱ṭṭha̱bba̮-sa̱꜔꜒ññā̱ lo̱ke̱ \ldo̱ts{} dha̱mma̮-sa̱꜔꜒ññā̱ lo̱ke̱ pi̮ya̮rū̱pa͓ṁ
sā̱꜔꜒ta̮rū̱pa̱ṁ, e̱tthe̱꜔꜒sā̱꜔꜒ ta̱ṇhā̱꜔꜒ pa̮hī̱꜔꜒ya̮mā̱nā̱ pa̮hī̱꜔꜒ya̮ti̮, e̱ttha̮ ni̮ru̱jjha̮mā̱nā̱ ni̮ru̱jjha̮ti͓.

Rū̱pa̮-sa̱꜔꜒ñce̱ta̮nā̱ lo̱ke̱ \ldo̱ts{} sa̱dda̮-sa̱꜔꜒ñce̱ta̮nā̱ lo̱ke̱ \ldo̱ts{} ga̱ndha̮-sa̱꜔꜒ñce̱ta̮nā̱ lo̱ke͓
\ldots{} rasa-sa꜔꜒ñcetanā loke \ldots{} phoṭṭhabba-sa꜔꜒ñcetanā loke \ldots{}
dha̱mma̮-sa̱꜔꜒ñce̱ta̮nā̱ lo̱ke̱ pi̮ya̮rū̱pa̱ṁ sā̱꜔꜒ta̮rū̱pa̱ṁ, e̱tthe̱꜔꜒sā̱꜔꜒ ta̱ṇhā̱꜔꜒ pa̮hī̱꜔꜒ya̮mā̱nā̱ pa̮hī̱꜔꜒ya̮ti͓,
e̱ttha̮ ni̮ru̱jjha̮mā̱nā̱ ni̮ru̱jjha̮ti͓.

Rū̱pa̮-ta̱ṇhā̱꜔꜒ lo̱ke̱ \ldo̱ts{} sa̱dda̮-ta̱ṇhā̱꜔꜒ lo̱ke̱ \ldo̱ts{} ga̱ndha̮-ta̱ṇhā̱꜔꜒ lo̱ke̱ \ldo͓ts{}
ra̮sa̮-ta̱ṇhā̱꜔꜒ lo̱ke̱ \ldo̱ts{} pho̱ṭṭha̱bba̮-ta̱ṇhā̱꜔꜒ lo̱ke̱ \ldo̱ts{} dha̱mma̮-ta̱ṇhā̱꜔꜒ lo̱ke̱ pi̮ya̮rū̱pa͓ṁ
sā̱꜔꜒ta̮rū̱pa̱ṁ, e̱tthe̱꜔꜒sā̱꜔꜒ ta̱ṇhā̱꜔꜒ pa̮hī̱꜔꜒ya̮mā̱nā̱ pa̮hī̱꜔꜒ya̮ti̮, e̱ttha̮ ni̮ru̱jjha̮mā̱nā̱ ni̮ru̱jjha̮ti͓.

\englishPage

In the world thought about forms \ldots{} thought about sounds \ldots{} thought
about odours \ldots{} thought about tastes \ldots{} thought about tactile
objects \ldots{} thought about mental phenomena is likeable and pleasing -- here
this craving when it is being abandoned, abandoned, here when ceasing it ceases.

In the world examination of forms \ldots{} examination of sounds \ldots{}
examination of odours \ldots{} examination of tastes \ldots{} examination of
tactile objects \ldots{} examination of mental phenomena is likeable and
pleasing -- here this craving when it is being abandoned, abandoned, here when
ceasing it ceases.

This, bhikkhus, is called the noble truth of the cessation of suffering.

\section{The Way}

And what, bhikkhus, is the noble truth of the way leading to the cessation of
suffering? It is this Noble Eightfold Path; that is right view, right intention,
right speech, right action, right livelihood, right effort, right mindfulness,
right concentration.

And what, bhikkhus, is right view? Bhikkhus, the knowledge of suffering,
knowledge of the origin of suffering, knowledge of the cessation of suffering,
knowledge of the way leading to the cessation of suffering: this, bhikkhus, is
called right view.

And what, bhikkhus, is right intention? Intention of renunciation, intention of
non-ill will, intention of harmlessness: this, bhikkhus, is called right
intention.

And what, bhikkhus, is right speech? Abstinence from false speech, abstinence
from divisive sppech, abstinence from harsh speech, abstinence from idle
chatter: this, bhikkhus, is called right speech.

\paliPage

Rū̱pa̮-vi̮ta̱kko̱ lo̱ke̱ \ldo̱ts{} sa̱dda̮-vi̮ta̱kko̱ lo̱ke̱ \ldo̱ts{} ga̱ndha̮-vi̮ta̱kko̱ lo̱ke̱ \ldo͓ts{}
ra̮sa̮-vi̮ta̱kko̱ lo̱ke̱ \ldo̱ts{} pho̱ṭṭha̱bba̮-vi̮ta̱kko̱ lo̱ke̱ \ldo̱ts{} dha̱mma̮-vi̮ta̱kko̱ lo̱ke͓
pi̮ya̮rū̱pa̱ṁ sā̱꜔꜒ta̮rū̱pa̱ṁ, e̱tthe̱꜔꜒sā̱꜔꜒ ta̱ṇhā̱꜔꜒ pa̮hī̱꜔꜒ya̮mā̱nā̱ pa̮hī̱꜔꜒ya̮ti̮, e̱ttha̮ ni̮ru̱jjha̮mā̱nā͓
nirujjhati.

Rū̱pa̮-vi̮cā̱ro̱ lo̱ke̱ \ldo̱ts{} sa̱dda̮-vi̮cā̱ro̱ lo̱ke̱ \ldo̱ts{} ga̱ndha̮-vi̮cā̱ro̱ lo̱ke̱ \ldo͓ts{}
ra̮sa̮-vi̮cā̱ro̱ lo̱ke̱ \ldo̱ts{} pho̱ṭṭha̱bba̮-vi̮cā̱ro̱ lo̱ke̱ \ldo̱ts{} dha̱mma̮-vi̮cā̱ro̱ lo̱ke͓
pi̮ya̮rū̱pa̱ṁ sā̱꜔꜒ta̮rū̱pa̱ṁ, e̱tthe̱꜔꜒sā̱꜔꜒ ta̱ṇhā̱꜔꜒ pa̮hī̱꜔꜒ya̮mā̱nā̱ pa̮hī̱꜔꜒ya̮ti̮, e̱ttha̮ ni̮ru̱jjha̮mā̱nā͓
nirujjhati.

I̮da̱ṁ vu̱cca̮ti̮, bhi̱kkha̮ve̱, du̱kkha̮ni̮ro̱dha̱ṁ a̮ri̮ya̮sa̱cca͓ṁ.

\section*{Magga-sacca}

Ka̮ta̮ma̱ñca̮, bhi̱kkha̮ve̱, du̱kkha̮-ni̮ro̱dha̮-gā̱mi̮nī̱ pa̮ṭi̮pa̮dā̱ a̮ri̮ya̮sa̱cca̱ṁ? A̮ya̮me̱va̮ a̮ri̮yo͓
a̱ṭṭha̱꜔꜒ṅgi̮ko̱ ma̱ggo̱ se̱꜔꜒yya̮thī̱꜔꜒da̱ṁ, sa̱꜔꜒mmā̱-di̱ṭṭhi̮ sa̱꜔꜒mmā̱-sa̱꜔꜒ṅka̱ppo̱ sa̱꜔꜒mmā̱-vā̱cā͓
sa̱꜔꜒mmā̱-ka̱mma̱nto̱ sa̱꜔꜒mmā̱-ā̱jī̱vo̱ sa̱꜔꜒mmā̱-vā̱yā̱mo̱ sa̱꜔꜒mmā̱-sa̮ti̮ sa̱꜔꜒mmā̱-sa̮mā̱dhi͓.

Ka̮ta̮mā̱ ca̮, bhi̱kkha̮ve̱, sa̱꜔꜒mmā̱-di̱ṭṭhi̮? Ya̱ṁ kho̱꜔꜒, bhi̱kkha̮ve̱, du̱kkhe̱꜔꜒ ñā̱ṇa͓ṁ,
du̱kkha̮-sa̮mu̮da̮ye̱ ñā̱ṇa̱ṁ, du̱kkha̮-ni̮ro̱dhe̱ ñā̱ṇa̱ṁ, du̱kkha̮-ni̮ro̱dha̮-gā̱mi̮ni̮yā̱ pa̮ṭi̮pa̮dā̱ya͓
ñā̱ṇa̱ṁ. A̮ya̱ṁ vu̱cca̮ti̮, bhi̱kkha̮ve̱, sa̱꜔꜒mmā̱-di̱ṭṭhi͓.

Ka̮ta̮mo̱ ca̮, bhi̱kkha̮ve̱, sa̱꜔꜒mmā̱-sa̱꜔꜒ṅka̱ppo̱? Ne̱kkha̱꜔꜒mma̮-sa̱꜔꜒ṅka̱ppo̱ a̮byā̱pā̱da̮-sa̱꜔꜒ṅka̱ppo͓
a̮vi̮hi̱꜔꜒ṁsā̱꜔꜒-sa̱꜔꜒ṅka̱ppo̱. A̮ya̱ṁ vu̱cca̮ti̮, bhi̱kkha̮ve̱, sa̱꜔꜒mmā̱-sa̱꜔꜒ṅka̱ppo͓.

Ka̮ta̮mā̱ ca̮, bhi̱kkha̮ve̱, sa̱꜔꜒mmā̱-vā̱cā̱? Mu̮sā̱꜔꜒vā̱dā̱ ve̱ra̮ma̮ṇī̱ pi̮su̮ṇā̱ya̮ vā̱cā̱ya̮ ve̱ra̮ma̮ṇī͓
pha̮ru̮sā̱꜔꜒ya̮ vā̱cā̱ya̮ ve̱ra̮ma̮ṇī̱ sa̱꜔꜒mpha̱ppa̮lā̱pā̱ ve̱ra̮ma̮ṇī̱. A̮ya̱ṁ vu̱cca̮ti̮, bhi̱kkha̮ve͓,
sa̱꜔꜒mmā̱-vā̱cā͓.

\englishPage

And what, bhikkhus, is right action? Abstinence from the destruction of life,
abstinence from taking what is not given, abstinence from sexual misconduct:
this, bhikkhus, is called right action.

And what, bhikkhus, is right livelihood? Here, bhikkhus, a noble disciple,
having abandoned a wrong mode of livelihood, earns his living by a right
livelihood: this, bhikkhus, is called right livelihood.

And what, bhikkhus, is right effort? Here, bhikkhus, a bhikkhu, for the
nonarising of unarisen evil unwholesome states; he generates desire, makes an
effort, arouses energy, applies his mind, and strives. For the abandoning of
arisen evil unwholesome states; he generates desire, makes an effort, arouses
energy, applies his mind, and strives. For the arising of unarisen wholesome
states; he generates desire, makes an effort, arouses energy, applies his mind,
and strives. For the maintenance of arisen wholesome states, for their nondecay,
increase, expansion, and fulfilment by development; he generates desire, makes
an effort, arouses energy, applies his mind, and strives. This, bhikkhus, is
called right effort.

And what, bhikkhus, is right mindfulness? Here, bhikkhus, a bhikkhu dwells
contemplating the body in the body, ardent, clearly comprehending, and mindful,
having subdued longing and dejection in regard to the world.

He dwells contemplating feelings in feelings, ardent, clearly comprehending, and
mindful, having subdued longing and dejection in regard to the world.

He dwells contemplating mind in mind, ardent, clearly comprehending, and
mindful, having subdued longing and dejection in regard to the world.

He dwells contemplating phenomena in phenomena, ardent, clearly comprehending,
and mindful, having subdued longing and dejection in regard to the world. This,
bhikkhus, is called right mindfulness.

\paliPage

Ka̮ta̮mo̱ ca̮, bhi̱kkha̮ve̱, sa̱꜔꜒mmā̱-ka̱mma̱nto̱? Pā̱ṇā̱ti̮pā̱tā̱ ve̱ra̮ma̮ṇī̱ a̮di̱nnā̱dā̱nā̱ ve̱ra̮ma̮ṇī͓
kā̱me̱su̮-mi̱cchā̱꜔꜒cā̱rā̱ ve̱ra̮ma̮ṇī̱. A̮ya̱ṁ vu̱cca̮ti̮, bhi̱kkha̮ve̱, sa̱꜔꜒mmā̱-ka̱mma̱nto͓.

Ka̮ta̮mo̱ ca̮, bhi̱kkha̮ve̱, sa̱꜔꜒mmā̱-ā̱jī̱vo̱? I̮dha̮, bhi̱kkha̮ve̱, a̮ri̮ya̮sā̱꜔꜒va̮ko̱ mi̱cchā̱꜔꜒-ā̱jī̱va͓ṁ
pa̮hā̱꜔꜒ya̮ sa̱꜔꜒mmā̱-ā̱jī̱ve̱na̮ jī̱vi̮ta̱ṁ ka̱ppe̱ti̮. A̮ya̱ṁ vu̱cca̮ti̮, bhi̱kkha̮ve̱, sa̱꜔꜒mmā̱-ā̱jī̱vo͓.

Ka̮ta̮mo̱ ca̮, bhi̱kkha̮ve̱, sa̱꜔꜒mmā̱-vā̱yā̱mo̱? I̮dha̮, bhi̱kkha̮ve̱, bhi̱kkhu͓
a̮nu̱ppa̱nnā̱na̱ṁ pā̱pa̮kā̱na̱ṁ a̮ku̮sa̮lā̱na̱ṁ dha̱mmā̱na͓ṁ
a̮nu̱ppā̱dā̱ya̮ cha̱꜔꜒nda̱ṁ ja̮ne̱ti̮ vā̱ya̮ma̮ti̮ vi̮ri̮ya̱ṁ ā̱ra̮bha̮ti̮ ci̱tta̱ṁ pa̱gga̱ṇhā̱꜔꜒ti̮ pa̮da̮ha̮ti͓;

u̱ppa̱nnā̱na̱ṁ pā̱pa̮kā̱na̱ṁ a̮ku̮sa̮lā̱na̱ṁ dha̱mmā̱na͓ṁ
pa̮hā̱꜔꜒nā̱ya̮ cha̱꜔꜒nda̱ṁ ja̮ne̱ti̮ vā̱ya̮ma̮ti̮ vi̮ri̮ya̱ṁ ā̱ra̮bha̮ti̮ ci̱tta̱ṁ pa̱gga̱ṇhā̱꜔꜒ti̮ pa̮da̮ha̮ti͓;

a̮nu̱ppa̱nnā̱na̱ṁ ku̮sa̮lā̱na̱ṁ dha̱mmā̱na͓ṁ
u̱ppā̱dā̱ya̮ cha̱꜔꜒nda̱ṁ ja̮ne̱ti̮ vā̱ya̮ma̮ti̮ vi̮ri̮ya̱ṁ ā̱ra̮bha̮ti̮ ci̱tta̱ṁ pa̱gga̱ṇhā̱꜔꜒ti̮ pa̮da̮ha̮ti͓;

u̱ppa̱nnā̱na̱ṁ ku̮sa̮lā̱na̱ṁ dha̱mmā̱na͓ṁ
ṭhi̮ti̮yā̱ a̮sa̱꜔꜒mmo̱sā̱꜔꜒ya̮ bhi̱yyo̱bhā̱vā̱ya̮ ve̱pu̱llā̱ya͓
bhā̱va̮nā̱ya̮ pā̱ri̮pū̱ri̮yā̱ cha̱꜔꜒nda̱ṁ ja̮ne̱ti̮ vā̱ya̮ma̮ti̮ vi̮ri̮ya̱ṁ ā̱ra̮bha̮ti̮ ci̱tta̱ṁ pa̱gga̱ṇhā̱꜔꜒ti͓
pa̮da̮ha̮ti̮. A̮ya̱ṁ vu̱cca̮ti̮, bhi̱kkha̮ve̱, sa̱꜔꜒mmā̱-vā̱yā̱mo͓.

Ka̮ta̮mā̱ ca̮, bhi̱kkha̮ve̱, sa̱꜔꜒mmā̱-sa̮ti̮? I̮dha̮, bhi̱kkha̮ve̱, bhi̱kkhu̮ kā̱ye̱ kā̱yā̱nu̮pa̱ssī͓꜔꜒
vi̮ha̮ra̮ti̮ ā̱tā̱pī̱ sa̱꜔꜒mpa̮jā̱no̱ sa̮ti̮mā̱ vi̮ne̱yya̮ lo̱ke̱ a̮bhi̱jjhā̱-do̱ma̮na̱ssa͓꜔꜒ṁ;

ve̱da̮nā̱su̮ ve̱da̮nā̱nu̮pa̱ssī̱꜔꜒ vi̮ha̮ra̮ti̮ ā̱tā̱pī̱ sa̱꜔꜒mpa̮jā̱no̱ sa̮ti̮mā̱ vi̮ne̱yya̮ lo̱ke͓
a̮bhi̱jjhā̱-do̱ma̮na̱ssa͓꜔꜒ṁ;

\enlargethispage{\baselineskip}

ci̱tte̱ ci̱ttā̱nu̮pa̱ssī̱꜔꜒ vi̮ha̮ra̮ti̮ ā̱tā̱pī̱ sa̱꜔꜒mpa̮jā̱no̱ sa̮ti̮mā̱ vi̮ne̱yya̮ lo̱ke͓
a̮bhi̱jjhā̱-do̱ma̮na̱ssa͓꜔꜒ṁ;

dha̱mme̱su̮ dha̱mmā̱nu̮pa̱ssī̱꜔꜒ vi̮ha̮ra̮ti̮ ā̱tā̱pī̱ sa̱꜔꜒mpa̮jā̱no̱ sa̮ti̮mā̱ vi̮ne̱yya̮ lo̱ke͓
a̮bhi̱jjhā̱-do̱ma̮na̱ssa̱꜔꜒ṁ. A̮ya̱ṁ vu̱cca̮ti̮, bhi̱kkha̮ve̱, sa̱꜔꜒mmā̱-sa̮ti͓.

\englishPage

And what, bhikkhus, is right concentration? Here, bhikkhus, secluded from
sensual pleasures, secluded from unwholesome states, accompanied by thought and
examination, with rapture and happiness born of seclusion, a bhikkhu enters and
dwe̱lls i̱n the̱ fi̱rst jhā̱na͓.

With the subsiding of thought and examination, with internal confidence and
unification of mind, being without thought and examination, having the rapture
a̱nd ha̱ppi̮ne̱ss bo̱rn o̱f co̱nce̱ntra̮ti̱o̱n, he̱ e̱nte̱rs a̱nd dwe̱lls i̱n the̱ se̱co̱nd jhā̱na͓.

With the fading away as well of rapture, he dwells equanimous and, mindful and
clearly comprehending, he experiences happiness with the body; that which the
noble ones declare: `He is equanimous, mindul, one who dwells happily', he
e̱nte̱rs a̱nd dwe̱lls i̱n the̱ thi̱rd jhā̱na͓.

With the abandoning of pleasure, with the abandoning of pain, with the previous
passing away of joy and displeasure, which is neither painful nor pleasant and
includes the purification of mindfulness by equanimity, he enters and dwells in
the̱ fo̱u̱rth jhā̱na̮. Thi̱s, bhi̱kkhu̱s, i̱s ca̱lle̱d ri̱ght co̱nce̱ntra̮ti̱o͓n.

This, bhikkhus, is called the noble truth of the way leading to the cessation of
suffering.

In this way he dwells contemplating phenomena in phenomena \ldots{} or else
mindfulness that ‘there are phenomena’ is simply established in him to the
extent necessary for bare knowledge and repeated mindfulness.

And he dwells independent, not clinging to anything in the world. That is how,
bhikkhus, a bhikkhu dwells contemplating phenomena in phenomena in terms of the
Four Noble Truths.

\instr{The section on Truths is finished.}

\instr{The Contemplation of Phenomena is finished.}

\paliPage

Ka̮ta̮mo̱ ca̮, bhi̱kkha̮ve̱, sa̱꜔꜒mmā̱-sa̮mā̱dhi̮? I̮dha̮, bhi̱kkha̮ve̱, bhi̱kkhu̮ vi̮vi̱cce̱va̮ kā̱me̱hi͓
vi̮vi̱cca̮ a̮ku̮sa̮le̱hi̮ dha̱mme̱hi̮ sa̮vi̮ta̱kka̱ṁ sa̮vi̮cā̱ra̱ṁ vi̮ve̱ka̮ja̱ṁ pī̱ti̮su̮kha̱꜔꜒ṁ pa̮ṭha̮ma͓ṁ
jhā̱na̱ṁ u̮pa̮sa̱꜔꜒mpa̱jja̮ vi̮ha̮ra̮ti͓.

Vi̮ta̱kka̮-vi̮cā̱rā̱na̱ṁ vū̱pa̮sa̮mā̱ a̱jjha̱tta̱ṁ sa̱꜔꜒mpa̮sā̱꜔꜒da̮na̱ṁ ce̱ta̮so̱꜔꜒ e̱ko̱di̮bhā̱va̱ṁ a̮vi̮ta̱kka͓ṁ
a̮vi̮cā̱ra̱ṁ sa̮mā̱dhi̮ja̱ṁ pī̱ti̮su̮kha̱꜔꜒ṁ du̮ti̮ya̱ṁ jhā̱na̱ṁ u̮pa̮sa̱꜔꜒mpa̱jja̮ vi̮ha̮ra̮ti͓.

Pī̱ti̮yā̱ ca̮ vi̮rā̱gā̱ u̮pe̱kkha̮ko̱ ca̮ vi̮ha̮ra̮ti̮, sa̮to̱ ca̮ sa̱꜔꜒mpa̮jā̱no̱, su̮kha̱꜔꜒ñca̮ kā̱ye̱na͓
pa̮ṭi̮sa̱꜔꜒ṁve̱de̱ti̮, ya̱ṁ ta̱ṁ a̮ri̮yā̱ ā̱ci̱kkha̱꜔꜒nti̮ ‘u̮pe̱kkha̮ko̱ sa̮ti̮mā̱ su̮kha̮vi̮hā̱꜔꜒rī̱’ti̮ ta̮ti̮ya͓ṁ
jhā̱na̱ṁ u̮pa̮sa̱꜔꜒mpa̱jja̮ vi̮ha̮ra̮ti͓.

Su̮kha̱ssa̮ ca̮ pa̮hā̱꜔꜒nā̱ du̱kkha̱ssa̮ ca̮ pa̮hā̱꜔꜒nā̱ pu̱bbe̱va̮ so̱꜔꜒ma̮na̱ssa̮-do̱ma̮na̱ssā̱꜔꜒na̱ṁ a̱ttha̱꜔꜒ṅga̮mā͓
a̮du̱kkha̮ma̮su̮kha̱꜔꜒ṁ u̮pe̱kkhā̱꜔꜒-sa̮ti̮pā̱ri̮su̱ddhi̱ṁ ca̮tu̱ttha̱꜔꜒ṁ jhā̱na̱ṁ u̮pa̮sa̱꜔꜒mpa̱jja̮ vi̮ha̮ra̮ti͓.
A̮ya̱ṁ vu̱cca̮ti̮, bhi̱kkha̮ve̱, sa̱꜔꜒mmā̱-sa̮mā̱dhi͓.

I̮da̱ṁ vu̱cca̮ti̮, bhi̱kkha̮ve̱, du̱kkha̮-ni̮ro̱dha̮-gā̱mi̮nī̱ pa̮ṭi̮pa̮dā̱ a̮ri̮ya̮sa̱cca͓ṁ.

I̮ti̮ a̱jjha̱tta̱ṁ vā̱ dha̱mme̱su̮ dha̱mmā̱nu̮pa̱ssī̱꜔꜒ vi̮ha̮ra̮ti͓,
ba̮hi̱ddhā̱ vā̱ dha̱mme̱su̮ dha̱mmā̱nu̮pa̱ssī̱꜔꜒ vi̮ha̮ra̮ti͓,
a̱jjha̱tta̮-ba̮hi̱ddhā̱ vā̱ dha̱mme̱su̮ dha̱mmā̱nu̮pa̱ssī̱꜔꜒ vi̮ha̮ra̮ti͓.
Sa̮mu̮da̮ya̮-dha̱mmā̱nu̮pa̱ssī̱꜔꜒ vā̱ dha̱mme̱su̮ vi̮ha̮ra̮ti͓,
va̮ya̮-dha̱mmā̱nu̮pa̱ssī̱꜔꜒ vā̱ dha̱mme̱su̮ vi̮ha̮ra̮ti͓,
sa̮mu̮da̮ya̮-va̮ya̮-\\ dha̱mmā̱nu̮pa̱ssī̱꜔꜒ vā̱ dha̱mme̱su̮ vi̮ha̮ra̮ti͓.
‘A̱tthi̮ dha̱mmā̱’ti̮ vā̱ pa̮na̱ssa̮ sa̮ti̮ pa̱ccu̮pa̱ṭṭhi̮tā̱ ho̱꜔꜒ti͓
yā̱va̮de̱va̮ ñā̱ṇa̮ma̱ttā̱ya̮ pa̮ṭi̱ssa̮ti̮ma̱ttā̱ya̮ a̮ni̱ssi̮to̱ ca̮ vi̮ha̮ra̮ti͓,
na̮ ca̮ ki̱ñci̮ lo̱ke̱ u̮pā̱di̮ya̮ti̮. E̱va̱mpi̮ kho̱꜔꜒, bhi̱kkha̮ve̱, bhi̱kkhu͓
dha̱mme̱su̮ dha̱mmā̱nu̮pa̱ssī̱꜔꜒ vi̮ha̮ra̮ti̮ ca̮tū̱su̮ a̮ri̮ya̮sa̱cce̱su͓.

\instr{Saccapabbaṁ niṭṭhitaṁ.}

\instr{Dhammānupassanā niṭṭhitā.}

\englishPage
\chapter{Conclusion}

Bhikkhus, if anyone should develop these four foundations of mindfulness in
such a way for seven years, one of two fruits could be expected for him: either
final knowledge here and now, or if there is a trace of clinging left,
nonreturning.

Let alone seven years, bhikkhus. If anyone should develop these four
foundations of mindfulness in such a way for six years \ldots{} five years
\ldots{} four years \ldots{} three years \ldots{} two years \ldots{} one year,
Let alone one year, bhikkhus. If anyone should develop these four foundations
of mindfulness in such a way for seven months, one of two fruits could be
expected for him: either final knowledge here and now, or if there is a trace of
clinging left, nonreturning. Let alone seven months, bhikkhus. If anyone should
develop these four foundations of mindfulness in such a way for six months
\ldots{} five months \ldots{} four months \ldots{} three months \ldots{} two
months \ldots{} one month \ldots{} half a month, Let alone half a month,
bhikkhus.

If anyone should develop these four foundations of mindfulness in such a way
for seven days, one of two fruits could be expected for him: eitherfinal
knowledge here and now, or if there is a trace of clinging left, nonreturning.

`Bhikkhus, this is the direct path for the purification of beings, for the
surmounting of sorrow and lamentation, for the passing away of pain and
de̱je̱cti̱o̱n, fo̱r the̱ a̱tta̱i̱nme̱nt o̱f the̱ tru̱e̱ wa̱y, fo̱r the̱ re̱a̮li̮sa̮ti̱o̱n o̱f Ni̱bbā̱na͓,
namely, the four foundations of mindfulness.'

It was with reference to this that it was said. This is what the Blessed One
said. The bhikkhus were satisfied and delighted in the Blessed One's words.

\bigskip

{\centering\instructionFont\color{instruction}\upshape

  The Greater Discourse on the\\
  Foundations of Mindfulness is finished.

}

\paliPage
\chapter*{Conclusion}

Yo̱ hi̮ ko̱ci̮, bhi̱kkha̮ve̱, i̮me̱ ca̱ttā̱ro̱ sa̮ti̮pa̱ṭṭhā̱꜔꜒ne̱ e̱va̱ṁ bhā̱ve̱yya̮ sa̱tta̮va̱ssā̱꜔꜒ni͓,
ta̱ssa̮ dvi̱nna̱ṁ pha̮lā̱na̱ṁ a̱ñña̮ta̮ra̱ṁ pha̮la̱ṁ pā̱ṭi̮ka̱ṅkha̱꜔꜒ṁ di̱ṭṭhe̱꜔꜒va̮ dha̱mme̱ a̱ññā̱; sa̮ti͓
vā̱ u̮pā̱di̮se̱꜔꜒se̱꜔꜒ a̮nā̱gā̱mi̮tā͓.

Ti̱ṭṭha̱꜔꜒ntu̮, bhi̱kkha̮ve̱, sa̱tta̮va̱ssā̱꜔꜒ni̮. Yo̱ hi̮ ko̱ci̮, bhi̱kkha̮ve̱, i̮me̱ ca̱ttā̱ro͓
sa̮ti̮pa̱ṭṭhā̱꜔꜒ne̱ e̱va̱ṁ bhā̱ve̱yya̮ cha̮ va̱ssā̱꜔꜒ni̮ \ldo̱ts{} pa̱ñca̮ va̱ssā̱꜔꜒ni͓
\ldots{} cattāri vassā꜔꜒ni \ldots{} tīṇi vassā꜔꜒ni \ldots{} dve vassā꜔꜒ni \ldots{}
e̱ka̱ṁ va̱ssa̱꜔꜒ṁ \ldo̱ts{} ti̱ṭṭha̮tu̮, bhi̱kkha̮ve̱, e̱ka̱ṁ va̱ssa̱꜔꜒ṁ. Yo̱ hi̮ ko̱ci̮, bhi̱kkha̮ve͓,
i̮me̱ ca̱ttā̱ro̱ sa̮ti̮pa̱ṭṭhā̱꜔꜒ne̱ e̱va̱ṁ bhā̱ve̱yya̮ sa̱tta̮mā̱sā̱꜔꜒ni̮, ta̱ssa̮ dvi̱nna̱ṁ pha̮lā̱na͓ṁ
a̱ñña̮ta̮ra̱ṁ pha̮la̱ṁ pā̱ṭi̮ka̱ṅkha̱꜔꜒ṁ di̱ṭṭhe̱꜔꜒va̮ dha̱mme̱ a̱ññā̱; sa̮ti̮ vā̱ u̮pā̱di̮se̱꜔꜒se̱꜔꜒ a̮nā̱gā̱mi̮tā͓.

Ti̱ṭṭha̱꜔꜒ntu̮, bhi̱kkha̮ve̱, sa̱tta̮ mā̱sā̱꜔꜒ni̮. Yo̱ hi̮ ko̱ci̮, bhi̱kkha̮ve̱, i̮me̱ ca̱ttā̱ro͓
sa̮ti̮pa̱ṭṭhā̱꜔꜒ne̱ e̱va̱ṁ bhā̱ve̱yya̮ cha̮ mā̱sā̱꜔꜒ni̮ \ldo̱ts{} pa̱ñca̮ mā̱sā̱꜔꜒ni̮ \ldo͓ts{}
ca̱ttā̱ri̮ mā̱sā̱꜔꜒ni̮ \ldo̱ts{} tī̱ṇi̮ mā̱sā̱꜔꜒ni̮ \ldo̱ts{} dve̱ mā̱sā̱꜔꜒ni̮ \ldo̱ts{} e̱ka̱ṁ mā̱sa͓꜔꜒ṁ
\ldots{} aḍḍhamāsa꜔꜒ṁ \ldots{} tiṭṭhatu, bhikkhave, aḍḍhamāso꜔꜒. Yo hi koci,
bhi̱kkha̮ve̱, i̮me̱ ca̱ttā̱ro̱ sa̮ti̮pa̱ṭṭhā̱꜔꜒ne̱ e̱va̱ṁ bhā̱ve̱yya̮ sa̱ttā̱ha̱꜔꜒ṁ, ta̱ssa̮ dvi̱nna͓ṁ
pha̮lā̱na̱ṁ a̱ñña̮ta̮ra̱ṁ pha̮la̱ṁ pā̱ṭi̮ka̱ṅkha̱꜔꜒ṁ di̱ṭṭhe̱꜔꜒va̮ dha̱mme̱ a̱ññā̱; sa̮ti̮ vā̱ u̮pā̱di̮se̱꜔꜒se͓꜔꜒
a̮nā̱gā̱mi̮tā͓.

E̱kā̱ya̮no̱ a̮ya̱ṁ, bhi̱kkha̮ve̱, ma̱ggo̱ sa̱ttā̱na̱ṁ vi̮su̱ddhi̮yā̱ so̱꜔꜒ka̮-pa̮ri̮de̱vā̱na̱ṁ sa̮ma̮ti̱kka̮mā̱ya͓
du̱kkha̮-do̱ma̮na̱ssā̱꜔꜒na̱ṁ a̱ttha̱꜔꜒ṅga̮mā̱ya̮ ñā̱ya̱ssa̮ a̮dhi̮ga̮mā̱ya̮ ni̱bbā̱na̱ssa̮ sa̱cchi̮ki̮ri̮yā̱ya͓
ya̮di̮da̱ṁ ca̱ttā̱ro̱ sa̮ti̮pa̱ṭṭhā̱꜔꜒nā̱ti̮. I̮ti̮ ya̱ṁ ta̱ṁ vu̱tta̱ṁ, i̮da̮me̱ta̱ṁ pa̮ṭi̱cca̮ vu̱tta̱n'ti͓.

I̮da̮ma̮vo̱ca̮ bha̮ga̮vā̱. A̱tta̮ma̮nā̱ te̱ bhi̱kkhū̱꜔꜒ bha̮ga̮va̮to̱ bhā̱si̮ta̱ṁ a̮bhi̮na̱ndu̱nti͓.

\bigskip

{\centering\instructionFont\color{instruction}\upshape

  Ma̮hā̱sa̮ti̮pa̱ṭṭhā̱na̮su̱tta̱ṁ ni̱ṭṭhi̮ta͓ṁ.

}

\resumeNormalText
