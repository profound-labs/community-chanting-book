\chapterTocDelegatePageNumber
\chapter{Setting in Motion the Wheel of Dhamma}

\setTocDelegatedPageNumber
\englishText
\renewcommand{\englishTitle}{Setting in Motion the Wheel of Dhamma}

\begin{leader}
\soloinstr{Solo introduction}

Thi̱s i̱s the̱ fi̱rst te̱a̮chi̱ng o̱f the̱ Ta̮thā̱ga̮ta̮ o̱n a̱tta̱i̮ni̱ng to̱ u̮ne̱xce̱lle͓d,
perfect enlightenment.

Here is the perfect turning of the incomparable wheel of Truth,
inestimable wherever it is expounded in the world.

Disclosed here are the two extremes, and the Middle Way, with the Four Noble
Truths and the purified knowledge and vision pointed out by the Lord of
Dhamma.

Let us chant together this Sutta proclaiming the supreme, independent
enlightenment that is widely renowned as ‘The~Turning of the Wheel of
the Dhamma.’

\end{leader}

Thus have I heard.

Once when the Blessed One was staying in the deer sanctuary at
Isipatana, near Benares, he spoke to the group of five bhikkhus:

‘These two extremes, bhikkhus, should not be followed by one who has
gone forth: sensual indulgence, which is low, coarse, vulgar, ignoble,
and unprofitable; and self-torture, which is painful, ignoble, and
unprofitable.

‘Bhi̱kkhu̱s, by a̮vo̱i̮di̱ng the̱se̱ two̱ e̱xtre̱me̱s, the̱ Ta̮thā̱ga̮ta̮ ha̱s re̱a̮li̱ze͓d
the Middle Way, which gives vision and understanding, which leads to
ca̱lm, pe̱ne̱tra̮ti̱o̱n, e̱nli̱ghte̱nme̱nt, to̱ Ni̱bbā̱na͓.

‘A̱nd wha̱t, bhi̱kkhu̱s, i̱s the̱ Mi̱ddle̱ Wa̱y re̱a̮li̱ze̱d by the̱ Ta̮thā̱ga̮ta̮, whi͓ch
gives vision and understanding, which leads to calm, penetration,
e̱nli̱ghte̱nme̱nt, to̱ Ni̱bbā̱na͓?

\chapterTocSubIndentTrue
\chapter{Dhammacakkappavattana Sutta}

\paliText
\renewcommand{\paliTitle}{Dhammacakkappavattana Sutta}

\begin{leader}
\soloinstr{Solo introduction}

\begin{solotwochants}
A̮nu̱tta̮ra̱ṁ a̮bhi̮sa̱mbo̱dhi̱ṁ & sa̱mbu̱jjhi̱tvā̱ ta̮thā̱ga̮to͓\\
Pa̮tha̮ma̱ṁ ya̱ṁ a̮de̱se̱si̮ & dha̱mma̮ca̱kka̱ṁ a̮nu̱tta̮ra͓ṁ\\
Sa̱mma̮de̱va̮ pa̮va̱tte̱nto̱ & lo̱ke̱ a̱ppa̮ti̮va̱tti̮ya͓ṁ\\
Ya̱tthā̱kkhā̱tā̱ u̮bho̱ a̱ntā̱ & pa̮ṭi̮pa̱tti̮ ca̮ ma̱jjhi̮mā͓\\
Ca̮tū̱svā̱ri̮ya̮sa̱cce̱su̮ & vi̮su̱ddha̱ṁ ñā̱ṇa̮da̱ssa̮na͓ṁ\\
De̱si̮ta̱ṁ dha̱mma̮rā̱je̱na̮ & sa̱mmā̱sa̱mbo̱dhi̮ki̱tta̮na͓ṁ\\
Nā̱me̱na̮ vi̱ssu̮ta̱ṁ su̱tta̱ṁ & dha̱mma̮ca̱kka̱ppa̮va̱tta̮na͓ṁ\\
Ve̱yyā̱ka̮ra̮ṇa̮pā̱the̱na̮ & sa̱ṅgī̱ta̱nta̱m bha̮ṇā̱ma̮ se͓\\
\end{solotwochants}
\end{leader}

[E̱va̱ṁ me̱ su̮ta͓ṁ]

E̱ka̱ṁ sa̮ma̮ya̱ṁ bha̮ga̮vā̱ bā̱rā̱ṇa̮si̮ya̱ṁ vi̮ha̮ra̮ti̮ i̮si̮pa̮ta̮ne̱ mi̮ga̮dā̱ye̱. Ta̱tra̮ kho͓
bha̮ga̮vā̱ pa̱ñca̮va̱ggi̮ye̱ bhi̱kkhū̱ ā̱ma̱nte̱si͓:

Dve̱'me̱, bhi̱kkha̮ve̱, a̱ntā̱ pa̱bba̮ji̮te̱na̮ na̮ se̱vi̮ta̱bbā̱: yo̱ cā̱ya̱ṁ kā̱me̱su͓
kā̱ma̮-su̮kh'a̱lli̮kā̱nu̮yo̱go̱; hī̱no̱, ga̱mmo̱, po̱thu̱jja̮ni̮ko̱, a̮na̮ri̮yo͓,
a̮na̱ttha̮-sa̱ñhi̮to̱; yo̱ cā̱ya̱ṁ a̱tta̮-ki̮la̮ma̮thā̱nu̮yo̱go̱; du̱kkho̱, a̮na̮ri̮yo͓,
a̮na̱ttha̮-sa̱ñhi̮to͓.

E̱te̱ te̱, bhi̱kkha̮ve̱, u̮bho̱ a̱nte̱ a̮nu̮pa̮ga̱mma̮ ma̱jjhi̮mā̱ pa̮ṭi̮pa̮dā̱ ta̮thā̱ga̮te̱na͓
a̮bhi̮sa̱mbu̱ddhā̱ ca̱kkhu̮ka̮ra̮ṇī̱, ñā̱ṇa̮ka̮ra̮ṇī̱, u̮pa̮sa̮mā̱ya̮, a̮bhi̱ññā̱ya͓,
sa̱mbo̱dhā̱ya̮, ni̱bbā̱nā̱ya̮ sa̱ṁva̱tta̮ti͓.

Ka̮ta̮mā̱ ca̮ sā̱, bhi̱kkha̮ve̱, ma̱jjhi̮mā̱ pa̮ṭi̮pa̮dā̱ ta̮thā̱ga̮te̱na̮ a̮bhi̮sa̱mbu̱ddhā͓
ca̱kkhu̮ka̮ra̮ṇī̱ ñā̱ṇa̮ka̮ra̮ṇī̱, u̮pa̮sa̮mā̱ya̮, a̮bhi̱ññā̱ya̮, sa̱mbo̱dhā̱ya̮, ni̱bbā̱nā̱ya͓
sa̱ṁva̱tta̮ti͓.

\clearpage

\englishText
\markboth{\englishTitle}{\rightmark}

‘It is just this Noble Eightfold Path, namely:

‘Right View, Right Intention, Right Speech, Right Action, Right
Livelihood, Right Effort, Right Mindfulness, and Right Concentration.

‘Tru̱ly, bhi̱kkhu̱s, thi̱s Mi̱ddle̱ Wa̱y u̱nde̱rsto̱o̱d by the̱ Ta̮thā̱ga̮ta̮ pro̱du̮ce͓s
vision, produces knowledge, and leads to calm, penetration,
e̱nli̱ghte̱nme̱nt, to̱ Ni̱bbā̱na͓.

‘This, bhikkhus, is the Noble Truth of dukkha:

‘Birth is dukkha, ageing is dukkha, death is dukkha, grief,
lamentation, pain, sorrow and despair are dukkha, association with the
disliked is dukkha, separation from the liked is dukkha, not to get what
one wants is dukkha. In brief, clinging to the five khandhas is dukkha.

‘This, bhikkhus, is the Noble Truth of the cause of dukkha:

‘The craving which causes rebirth and is bound up with pleasure and
lust, ever seeking fresh delight, now here, now there; namely, craving
for sense pleasure, craving for existence, and craving for annihilation.

‘This, bhikkhus, is the Noble Truth of the cessation of dukkha:

‘The complete cessation, giving up, abandonment of that craving,
complete release from that craving, and complete detachment from it.

‘This, bhikkhus, is the Noble Truth of the way leading to the cessation
of dukkha:

‘Only this Noble Eightfold Path; namely, Right View, Right Intention,
Right Speech, Right Action, Right Livelihood, Right Effort, Right
Mindfulness, and Right Concentration.

‘With the thought, “This is the Noble Truth of dukkha,” there arose in
me, bhikkhus, vision, knowledge, insight, wisdom, light, concerning
things unknown before.

\clearpage

\paliText
\markboth{\paliTitle}{\rightmark}

A̮ya̮m-e̱va̮ a̮ri̮yo̱ a̱ṭṭha̱ṅgi̮ko̱ ma̱ggo̱ se̱yya̮thī̱da͓ṁ:

Sa̱mmā̱-di̱ṭṭhi̮, sa̱mmā̱-sa̱ṅka̱ppo̱, sa̱mmā̱-vā̱cā̱, sa̱mmā̱-ka̱mma̱nto̱, sa̱mmā̱-ā̱jī̱vo͓,
sa̱mmā̱-vā̱yā̱mo̱, sa̱mmā̱-sa̮ti̮, sa̱mmā̱-sa̮mā̱dhi͓.

A̮ya̱ṁ kho̱ sā̱, bhi̱kkha̮ve̱, ma̱jjhi̮mā̱ pa̮ṭi̮pa̮dā̱ ta̮thā̱ga̮te̱na̮ a̮bhi̮sa̱mbu̱ddhā͓
ca̱kkhu̮ka̮ra̮ṇī̱ ñā̱ṇa̮ka̮ra̮ṇī̱, u̮pa̮sa̮mā̱ya̮, a̮bhi̱ññā̱ya̮, sa̱mbo̱dhā̱ya̮, ni̱bbā̱nā̱ya͓
sa̱ṁva̱tta̮ti͓.

I̮da̱ṁ kho̱ pa̮na̮, bhi̱kkha̮ve̱, du̱kkha̱ṁ a̮ri̮ya̮-sa̱cca͓ṁ:

Jā̱ti̮pi̮ du̱kkhā̱, ja̮rā̱pi̮ du̱kkhā̱, ma̮ra̮na̱mpi̮ du̱kkha͓ṁ,
so̱ka̮-pa̮ri̮de̱va̮-du̱kkha̮-do̱ma̮na̱ss'u̮pā̱yā̱sā̱pi̮ du̱kkhā̱, a̱ppi̮ye̱hi̮ sa̱mpa̮yo̱go͓
du̱kkho̱, pi̮ye̱hi̮ vi̱ppa̮yo̱go̱ du̱kkho̱, ya̱mp'i̱ccha̱ṁ na̮ la̮bha̮ti̮ ta̱mpi̮ du̱kkha͓ṁ,
sa̱ṅkhi̱tte̱na̮ pa̱ñcu̮pā̱dā̱na̱kkha̱ndhā̱ du̱kkhā͓.

I̮da̱ṁ kho̱ pa̮na̮, bhi̱kkha̮ve̱, du̱kkha̮-sa̮mu̮da̮yo̱ a̮ri̮ya̮-sa̱cca͓ṁ:

Yā̱'ya̱ṁ ta̱ṇhā̱ po̱no̱bbha̮vi̮kā̱ na̱ndi̮-rā̱ga̮-sa̮ha̮ga̮tā̱ ta̱tra̮-ta̱trā̱bhi̮na̱ndi̮nī͓
se̱yya̮thī̱da̱ṁ: kā̱ma̮-ta̱ṇhā̱, bha̮va̮-ta̱ṇhā̱, vi̮bha̮va̮-ta̱ṇhā͓.

I̮da̱ṁ kho̱ pa̮na̮, bhi̱kkha̮ve̱, du̱kkha̮-ni̮ro̱dho̱ a̮ri̮ya̮-sa̱cca͓ṁ:

Yo̱ ta̱ssā̱ ye̱va̮ ta̱ṇhā̱ya̮ a̮se̱sa̮-vi̮rā̱ga̮-ni̮ro̱dho̱, cā̱go̱, pa̮ṭi̮ni̱ssa̱ggo̱, mu̱tti͓,
a̮nā̱la̮yo͓.

I̮da̱ṁ kho̱ pa̮na̮, bhi̱kkha̮ve̱, du̱kkha̮-ni̮ro̱dha̮-gā̱mi̮nī̱ pa̮ṭi̮pa̮dā̱ a̮ri̮ya̮-sa̱cca͓ṁ:

A̮ya̮m-e̱va̮ a̮ri̮yo̱ a̱ṭṭha̱ṅgi̮ko̱ ma̱ggo̱ se̱yya̮thī̱da̱m: sa̱mmā̱-di̱ṭṭhi͓,
sa̱mmā̱-sa̱ṅka̱ppo̱, sa̱mmā̱-vā̱cā̱, sa̱mmā̱-ka̱mma̱nto̱, sa̱mmā̱-ā̱jī̱vo̱, sa̱mmā̱-vā̱yā̱mo͓,
sa̱mmā̱-sa̮ti̮, sa̱mmā̱-sa̮mā̱dhi͓.

\enlargethispage{\baselineskip}

[I̮da̱ṁ du̱kkha̱ṁ] a̮ri̮ya̮-sa̱cca̱n'ti̮ me̱ bhi̱kkha̮ve̱, pu̱bbe̱ a̮na̮nu̱ssu̮te̱su̮ dha̱mme̱su͓
ca̱kkhu̱ṁ u̮da̮pā̱di̮, ñā̱ṇa̱ṁ u̮da̮pā̱di̮, pa̱ññā̱ u̮da̮pā̱di̮, vi̱jjā̱ u̮da̮pā̱di̮, ā̱lo̱ko͓
u̮da̮pā̱di͓.

\clearpage

\englishText
\markboth{\englishTitle}{\rightmark}

‘With the thought, “This is the Noble Truth of dukkha, and this dukkha
has to be understood,” there arose in me, bhikkhus, vision, knowledge,
insight, wisdom, light, concerning things unknown before.

‘With the thought, “This is the Noble Truth of dukkha, and this dukkha
has been understood,” there arose in me, bhikkhus, vision, knowledge,
insight, wisdom, light, concerning things unknown before.

‘With the thought, “This is the Noble Truth of the cause of dukkha,”
there arose in me, bhikkhus, vision, knowledge, insight, wisdom, light,
concerning things unknown before.

‘With the thought, “This is the Noble Truth of the cause of dukkha, and
this cause of dukkha has to be abandoned,” there arose in me, bhikkhus,
vision, knowledge, insight, wisdom, light, concerning things unknown
before.

‘With the thought, “This is the Noble Truth of the cause of dukkha, and
this cause of dukkha has been abandoned,” there arose in me, bhikkhus,
vision, knowledge, insight, wisdom, light, concerning things unknown
before.

‘With the thought, “This is the Noble Truth of the cessation of dukkha,”
there arose in me, bhikkhus, vision, knowledge, insight, wisdom, light,
concerning things unknown before.

‘With the thought, “This is the Noble Truth of the cessation of dukkha,
and this cessation of dukkha has to be realized,” there arose in me,
bhikkhus, vision, knowledge, insight, wisdom, light, concerning things
unknown before.

‘With the thought, “This is the Noble Truth of the cessation of dukkha,
and this cessation of dukkha has been realized,” there arose in me,
bhikkhus, vision, knowledge, insight, wisdom, light, concerning things
unknown before.

\clearpage

\paliText
\markboth{\paliTitle}{\rightmark}

Ta̱ṁ kho̱ pa̮n'i̮da̱ṁ du̱kkha̱ṁ a̮ri̮ya̮-sa̱cca̱ṁ pa̮ri̱ññe̱yya̱n'ti̮ me̱ bhi̱kkha̮ve̱, pu̱bbe͓
a̮na̮nu̱ssu̮te̱su̮ dha̱mme̱su̮ ca̱kkhu̱ṁ u̮da̮pā̱di̮, ñā̱ṇa̱ṁ u̮da̮pā̱di̮, pa̱ññā̱ u̮da̮pā̱di͓,
vi̱jjā̱ u̮da̮pā̱di̮, ā̱lo̱ko̱ u̮da̮pā̱di͓.

Ta̱ṁ kho̱ pa̮n'i̮da̱ṁ du̱kkha̱ṁ a̮ri̮ya̮-sa̱cca̱ṁ pa̮ri̱ññā̱ta̱n'ti̮ me̱ bhi̱kkha̮ve̱, pu̱bbe͓
a̮na̮nu̱ssu̮te̱su̮ dha̱mme̱su̮ ca̱kkhu̱ṁ u̮da̮pā̱di̮, ñā̱ṇa̱ṁ u̮da̮pā̱di̮, pa̱ññā̱ u̮da̮pā̱di͓,
vi̱jjā̱ u̮da̮pā̱di̮, ā̱lo̱ko̱ u̮da̮pā̱di͓.

I̮da̱ṁ du̱kkha̮-sa̮mu̮da̮yo̱ a̮ri̮ya̮-sa̱cca̱n'ti̮ me̱ bhi̱kkha̮ve̱, pu̱bbe̱ a̮na̮nu̱ssu̮te̱su͓
dha̱mme̱su̮ ca̱kkhu̱ṁ u̮da̮pā̱di̮, ñā̱ṇa̱ṁ u̮da̮pā̱di̮, pa̱ññā̱ u̮da̮pā̱di̮, vi̱jjā̱ u̮da̮pā̱di͓,
ā̱lo̱ko̱ u̮da̮pā̱di͓.

Ta̱ṁ kho̱ pa̮n'i̮da̱ṁ du̱kkha̮-sa̮mu̮da̮yo̱ a̮ri̮ya̮sa̱cca̱ṁ pa̮hā̱ta̱bba̱n'ti̮ me̱ bhi̱kkha̮ve͓,
pu̱bbe̱ a̮na̮nu̱ssu̮te̱su̮ dha̱mme̱su̮ ca̱kkhu̱ṁ u̮da̮pā̱di̮, ñā̱ṇa̱ṁ u̮da̮pā̱di̮, pa̱ññā͓
u̮da̮pā̱di̮, vi̱jjā̱ u̮da̮pā̱di̮, ā̱lo̱ko̱ u̮da̮pā̱di͓.

Ta̱ṁ kho̱ pa̮n'i̮da̱ṁ du̱kkha̮-sa̮mu̮da̮yo̱ a̮ri̮ya̮-sa̱cca̱ṁ pa̮hī̱na̱n'ti̮ me̱ bhi̱kkha̮ve̱, pu̱bbe͓
a̮na̮nu̱ssu̮te̱su̮ dha̱mme̱su̮ ca̱kkhu̱ṁ u̮da̮pā̱di̮, ñā̱ṇa̱ṁ u̮da̮pā̱di̮, pa̱ññā̱ u̮da̮pā̱di͓,
vi̱jjā̱ u̮da̮pā̱di̮, ā̱lo̱ko̱ u̮da̮pā̱di͓.

I̮da̱ṁ du̱kkha̮-ni̮ro̱dho̱ a̮ri̮ya̮-sa̱cca̱n'ti̮ me̱ bhi̱kkha̮ve̱, pu̱bbe̱ a̮na̮nu̱ssu̮te̱su͓
dha̱mme̱su̮ ca̱kkhu̱ṁ u̮da̮pā̱di̮, ñā̱ṇa̱ṁ u̮da̮pā̱di̮, pa̱ññā̱ u̮da̮pā̱di̮, vi̱jjā̱ u̮da̮pā̱di͓,
ā̱lo̱ko̱ u̮da̮pā̱di͓.

Ta̱ṁ kho̱ pa̮n'i̮da̱ṁ du̱kkha̮-ni̮ro̱dho̱ a̮ri̮ya̮-sa̱cca̱ṁ sa̱cchi̮kā̱ta̱bba̱n'ti̮ me̱ bhi̱kkha̮ve͓,
pu̱bbe̱ a̮na̮nu̱ssu̮te̱su̮ dha̱mme̱su̮ ca̱kkhu̱ṁ u̮da̮pā̱di̮, ñā̱ṇa̱ṁ u̮da̮pā̱di̮, pa̱ññā͓
u̮da̮pā̱di̮, vi̱jjā̱, u̮da̮pā̱di̮ ā̱lo̱ko̱ u̮da̮pā̱di͓.

Ta̱ṁ kho̱ pa̮n'i̮da̱ṁ du̱kkha̮-ni̮ro̱dho̱ a̮ri̮ya̮-sa̱cca̱ṁ sa̱cchi̮ka̮ta̱n'ti̮ me̱ bhi̱kkha̮ve͓,
pu̱bbe̱ a̮na̮nu̱ssu̮te̱su̮ dha̱mme̱su̮ ca̱kkhu̱ṁ u̮da̮pā̱di̮, ñā̱ṇa̱ṁ u̮da̮pā̱di̮, pa̱ññā͓
u̮da̮pā̱di̮, vi̱jjā̱ u̮da̮pā̱di̮, ā̱lo̱ko̱ u̮da̮pā̱di͓.

\clearpage

\englishText
\markboth{\englishTitle}{\rightmark}

‘With the thought, “This is the Noble Truth of the way leading to the
cessation of dukkha,” there arose in me, bhikkhus, vision, knowledge,
insight, wisdom, light, concerning things unknown before.

‘With the thought, “This Noble Truth of the way leading to the cessation
of dukkha has to be developed,” there arose in me, bhikkhus, vision,
knowledge, insight, wisdom, light, concerning things unknown before.

‘With the thought, “This Noble Truth of the way leading to the cessation
of dukkha has been developed,” there arose in me, bhikkhus, vision,
knowledge, insight, wisdom, light, concerning things unknown before.

‘So long, bhikkhus, as my knowledge and vision of reality regarding
these Four Noble Truths, in their three phases and twelve aspects, was
not fully clear to me, I did not declare to the world of spirits,
demons, and gods, with its seekers and sages, celestial and human
beings, the realization of incomparable, perfect enlightenment.

‘But when, bhikkhus, my knowledge and vision of reality regarding these
Four Noble Truths, in their three phases and twelve aspects, was fully
clear to me, I declared to the world of spirits, demons, and gods, with
its seekers and sages, celestial and human beings, that I had realized
incomparable, perfect enlightenment.

‘Knowledge and vision arose: “Unshakeable is my deliverance; this is
the last birth, there will be no more renewal of being.”\thinspace ’

Thus spoke the Blessed One. Glad at heart, the group of five bhikkhus
approved of the words of the Blessed One.

\clearpage

\paliText
\markboth{\paliTitle}{\rightmark}

I̮da̱ṁ du̱kkha̮-ni̮ro̱dha̮-gā̱mi̮nī̱ pa̮ṭi̮pa̮dā̱ a̮ri̮ya̮-sa̱cca̱n'ti̮ me̱ bhi̱kkha̮ve̱, pu̱bbe͓
a̮na̮nu̱ssu̮te̱su̮ dha̱mme̱su̮ ca̱kkhu̱ṁ u̮da̮pā̱di̮, ñā̱ṇa̱ṁ u̮da̮pā̱di̮, pa̱ññā̱ u̮da̮pā̱di͓,
vi̱jjā̱ u̮da̮pā̱di̮, ā̱lo̱ko̱ u̮da̮pā̱di͓.

Ta̱ṁ kho̱ pa̮n'i̮da̱ṁ du̱kkha̮-ni̮ro̱dha̮-gā̱mi̮nī̱ pa̮ṭi̮pa̮dā̱ a̮ri̮ya̮-sa̱cca̱ṁ bhā̱ve̱ta̱bba̱n'ti͓
me̱ bhi̱kkha̮ve̱, pu̱bbe̱ a̮na̮nu̱ssu̮te̱su̮ dha̱mme̱su̮ ca̱kkhu̱ṁ u̮da̮pā̱di̮, ñā̱ṇa͓ṁ
u̮da̮pā̱di̮, pa̱ññā̱ u̮da̮pā̱di̮, vi̱jjā̱ u̮da̮pā̱di̮, ā̱lo̱ko̱ u̮da̮pā̱di͓.

Ta̱ṁ kho̱ pa̮n'i̮da̱ṁ du̱kkha̮-ni̮ro̱dha̮-gā̱mi̮nī̱ pa̮ṭi̮pa̮dā̱ a̮ri̮ya̮-sa̱cca̱ṁ bhā̱vi̮ta̱n'ti̮ me͓
bhi̱kkha̮ve̱, pu̱bbe̱ a̮na̮nu̱ssu̮te̱su̮ dha̱mme̱su̮ ca̱kkhu̱ṁ u̮da̮pā̱di̮, ñā̱ṇa̱ṁ u̮da̮pā̱di͓,
pa̱ññā̱ u̮da̮pā̱di̮, vi̱jjā̱ u̮da̮pā̱di̮, ā̱lo̱ko̱ u̮da̮pā̱di͓.

[Yā̱va̮ kī̱va̱ñca̮ me̱ bhi̱kkha̮ve̱,] i̮me̱su̮ ca̮tū̱su̮ a̮ri̮ya̮-sa̱cce̱su̮ e̱va̱n-ti̮-pa̮ri̮va̱ṭṭa͓ṁ
dvā̱da̮s'ā̱kā̱ra̱ṁ ya̮thā̱-bhū̱ta̱ṁ ñā̱ṇa̮-da̱ssa̮na̱ṁ na̮ su̮vi̮su̱ddha̱ṁ a̮ho̱si̮, n'e̱va̮ tā̱v'ā̱ha͓ṁ
bhi̱kkha̮ve̱, sa̮de̱va̮ke̱ lo̱ke̱ sa̮mā̱ra̮ke̱ sa̮bra̱hma̮ke̱ sa̱ssa̮ma̮ṇa̮-brā̱hma̮ṇi̮yā̱ pa̮jā̱ya͓
sa̮de̱va̮-ma̮nu̱ssā̱ya̮ a̮nu̱tta̮ra̱ṁ sa̱mmā̱-sa̱mbo̱dhi̱ṁ a̮bhi̮sa̱mbu̱ddho̱ pa̱cca̱ññā̱si͓ṁ.

Ya̮to̱ ca̮ kho̱ me̱ bhi̱kkha̮ve̱, i̮me̱su̮ ca̮tū̱su̮ a̮ri̮ya̮-sa̱cce̱su̮ e̱va̱n-ti̮-pa̮ri̮va̱ṭṭa͓ṁ
dvā̱da̮s'ā̱kā̱ra̱ṁ ya̮thā̱-bhū̱ta̱ṁ ñā̱ṇa̮-da̱ssa̮na̱ṁ su̮vi̮su̱ddha̱m a̮ho̱si̮, a̮th'ā̱ha͓m
bhi̱kkha̮ve̱, sa̮de̱va̮ke̱ lo̱ke̱ sa̮mā̱ra̮ke̱ sa̮bra̱hma̮ke̱ sa̱ssa̮ma̮ṇa̮-brā̱hma̮ṇi̮yā̱ pa̮jā̱ya͓
sa̮de̱va̮-ma̮nu̱ssā̱ya̮ a̮nu̱tta̮ra̱ṁ sa̱mmā̱-sa̱mbo̱dhi̱ṁ a̮bhi̮sa̱mbu̱ddho̱ pa̱cca̱ññā̱si͓ṁ.

Ñā̱ṇa̱ñca̮ pa̮na̮ me̱ da̱ssa̮na̱ṁ u̮da̮pā̱di̮, a̮ku̱ppā̱ me̱ vi̮mu̱tti̮ a̮ya̮m-a̱nti̮mā̱ jā̱ti͓,
na̱tthi̮ dā̱ni̮ pu̮na̱bbha̮vo̱'ti͓.

I̮da̮m-a̮vo̱ca̮ bha̮ga̮vā̱. A̱tta̮ma̮nā̱ pa̱ñca̮va̱ggi̮yā̱ bhi̱kkhū̱ bha̮ga̮va̮to̱ bhā̱si̮ta͓ṁ
a̮bhi̮na̱ndu͓ṁ.

\clearpage

\englishText
\markboth{\englishTitle}{\rightmark}

As this exposition was proceeding, the spotless, immaculate vision of
the̱ Dha̱mma̮ a̱ppe̱a̮re̱d to̱ the̱ Ve̱ne̱ra̱ble̱ Ko̱ṇḍa̱ñña̮ a̱nd he̱ kne̱w: ‘E̱ve̱rythi͓ng
that has the nature to arise has the nature to cease.’

When the Blessed One had set in motion the Wheel of Dhamma, the
Earthbound devas proclaimed with one voice,

‘The incomparable Wheel of Dhamma has been set in motion by the Blessed
One in the deer sanctuary at Isipatana, near Benares, and no seeker,
brahmin, celestial being, demon, god, or any other being in the world
can stop it.’

Having heard what the Earthbound devas said, the devas of the Four Great
Kings proclaimed with one voice\ldots

Having heard what the devas of the Four Great Kings said, the devas of
the Thirty-three proclaimed with one voice\ldots

Ha̮vi̱ng he̱a̱rd wha̱t the̱ de̱va̱s o̱f the̱ Thi̱rty-thre̱e̱ sa̱i̱d, the̱ Yā̱ma̮ de̱va͓s
proclaimed with one voice\ldots

Ha̮vi̱ng he̱a̱rd wha̱t the̱ Yā̱ma̮ de̱va̱s sa̱i̱d, the̱ De̱va̱s o̱f De̱li̱ght pro̱cla̱i̮me͓d
with one voice\ldots

Having heard what the Devas of Delight said, the Devas Who Delight in
Creating, proclaimed with one voice\ldots

Having heard what the Devas Who Delight in Creating said, the Devas Who
Delight in the Creations of Others proclaimed with one voice\ldots

Having heard what the Devas Who Delight in the Creations of Others said,
the Brahma gods proclaimed in one voice,

‘The incomparable Wheel of Dhamma has been set in motion by the Blessed
One in the deer sanctuary at Isipatana, near Benares, and no seeker,
brahmin, celestial being, demon, god, or any other being in the world
can stop it.’

\clearpage

\paliText
\markboth{\paliTitle}{\rightmark}

I̮ma̱smi̱ñca̮ pa̮na̮ ve̱yyā̱ka̮ra̮ṇa̱smi̱ṁ bha̱ñña̮mā̱ne̱ ā̱ya̱sma̮to̱ ko̱ṇḍa̱ñña̱ssa̮ vi̮ra̮ja͓ṁ
vī̱ta̮ma̮la̱ṁ dha̱mma̮ca̱kkhu̱ṁ u̮da̮pā̱di̮: ya̱ṁ ki̱ñci̮ sa̮mu̮da̮ya̮-dha̱mma̱ṁ sa̱bba̱n-ta͓ṁ
nirodha-dhamman'ti.

[Pa̮va̱tti̮te̱ ca̮ bha̮ga̮va̮tā̱] dha̱mma̮ca̱kke̱ bhu̱mmā̱ de̱vā̱ sa̱dda̮ma̮nu̱ssā̱ve̱su͓ṁ:

E̱ta̱ṁ bha̮ga̮va̮tā̱ bā̱rā̱ṇa̮si̮ya̱ṁ i̮si̮pa̮ta̮ne̱ mi̮ga̮dā̱ye̱ a̮nu̱tta̮ra̱ṁ dha̱mma̮ca̱kka͓ṁ
pa̮va̱tti̮ta̱ṁ a̱ppa̮ṭi̮va̱tti̮ya̱ṁ sa̮ma̮ṇe̱na̮ vā̱ brā̱hma̮ṇe̱na̮ vā̱ de̱ve̱na̮ vā̱ mā̱re̱na̮ vā͓
bra̱hmu̮nā̱ vā̱ ke̱na̮ci̮ vā̱ lo̱ka̱smi̱n'ti͓.

Bhu̱mmā̱na̱ṁ de̱vā̱na̱ṁ sa̱dda̱ṁ su̱tvā̱, cā̱tu̱mma̮hā̱rā̱ji̮kā̱ de̱vā͓
sa̱dda̮ma̮nu̱ssā̱ve̱su̱ṁ\ldo͓ts

Cā̱tu̱mma̮hā̱rā̱ji̮kā̱na̱ṁ de̱vā̱na̱ṁ sa̱dda̱ṁ su̱tvā̱, tā̱va̮ti̱ṁsā̱ de̱vā͓
sa̱dda̮ma̮nu̱ssā̱ve̱su̱ṁ\ldo͓ts

Tā̱va̮ti̱ṁsā̱na̱ṁ de̱vā̱na̱ṁ sa̱dda̱ṁ su̱tvā̱, yā̱mā̱ de̱vā̱ sa̱dda̮ma̮nu̱ssā̱ve̱su̱ṁ\ldo͓ts

Yā̱mā̱na̱ṁ de̱vā̱na̱ṁ sa̱dda̱ṁ su̱tvā̱, tu̮si̮tā̱ de̱vā̱ sa̱dda̮ma̮nu̱ssā̱ve̱su̱ṁ\ldo͓ts

Tu̮si̮tā̱na̱ṁ de̱vā̱na̱ṁ sa̱dda̱ṁ su̱tvā̱, ni̱mmā̱na̮ra̮tī̱ de̱vā̱ sa̱dda̮ma̮nu̱ssā̱ve̱su̱m\ldo͓ts

Ni̱mmā̱na̮ra̮tī̱na̱ṁ de̱vā̱na̱ṁ sa̱dda̱ṁ su̱tvā̱, pa̮ra̮ni̱mmi̮ta̮va̮sa̮va̱ttī̱ de̱vā͓
sa̱dda̮ma̮nu̱ssā̱ve̱su̱ṁ\ldo͓ts

Pa̮ra̮ni̱mmi̮ta̮va̮sa̮va̱ttī̱na̱ṁ de̱vā̱na̱ṁ sa̱dda̱ṁ su̱tvā̱, bra̱hma̮kā̱yi̮kā̱ de̱vā͓
sa̱dda̮ma̮nu̱ssā̱ve̱su͓ṁ:

E̱ta̱ṁ bha̮ga̮va̮tā̱ bā̱rā̱ṇa̮si̮ya̱ṁ i̮si̮pa̮ta̮ne̱ mi̮ga̮dā̱ye̱ a̮nu̱tta̮ra̱ṁ dha̱mma̮ca̱kka͓ṁ
pa̮va̱tti̮ta̱ṁ a̱ppa̮ṭi̮va̱tti̮ya̱ṁ sa̮ma̮ṇe̱na̮ vā̱ brā̱hma̮ṇe̱na̮ vā̱ de̱ve̱na̮ vā̱ mā̱re̱na̮ vā͓
bra̱hmu̮nā̱ vā̱ ke̱na̮ci̮ vā̱ lo̱ka̱smi̱n'ti͓.

\clearpage

\englishText
\markboth{\englishTitle}{\rightmark}

Thus in a moment, an instant, a flash, word of the Setting in Motion of
the Wheel of Dhamma went forth up to the Brahma world, and the
ten-thousandfold universal system trembled and quaked and shook, and a
boundless, sublime radiance surpassing the power of devas appeared on
earth.

Then the Blessed One made the utterance,

‘Tru̱ly, Ko̱ṇḍa̱ñña̮ ha̱s u̱nde̱rsto̱o̱d, Ko̱ṇḍa̱ñña̮ ha̱s u̱nde̱rsto̱o̱d!’ Thu̱s i̱t wa͓s
tha̱t the̱ Ve̱ne̱ra̱ble̱ Ko̱ṇḍa̱ñña̮ go̱t the̱ na̮me̱ A̱ññā̱-Ko̱ṇḍa̱ñña̮: ‘Ko̱ṇḍa̱ñña̮ Who͓
Understands.’

Thus ends the discourse on Setting in Motion the Wheel of Dhamma.

\clearpage

\paliText
\markboth{\paliTitle}{\rightmark}

I̮ti̮'ha̮ te̱na̮ kha̮ṇe̱na̮, te̱na̮ mu̮hu̱tte̱na̮, yā̱va̮ bra̱hma̮lo̱kā̱ sa̱ddo̱ a̱bbhu̱gga̱cchi͓.
A̮ya̱ñca̮ da̮sa̮-sa̮ha̱ssī̱ lo̱ka̮dhā̱tu̮ sa̱ṅka̱mpi̮ sa̱mpa̮ka̱mpi̮ sa̱mpa̮ve̱dhi̮, a̱ppa̮mā̱ṇo̱ ca͓
o̱ḷā̱ro̱ o̱bhā̱so̱ lo̱ke̱ pā̱tu̮ra̮ho̱si̮ a̮ti̱kka̱mme̱va̮ de̱vā̱na̱ṁ de̱vā̱nu̮bhā̱va͓ṁ.

A̮tha̮ kho̱ bha̮ga̮vā̱ u̮dā̱na̱ṁ u̮dā̱ne̱si͓:

A̱ññā̱si̮ va̮ta̮ bho̱ ko̱ṇḍa̱ñño̱, a̱ññā̱si̮ va̮ta̮ bho̱ ko̱ṇḍa̱ñño̱'ti̮. I̮ti̮ hi̮da̱ṁ ā̱ya̱sma̮to͓
ko̱ṇḍa̱ñña̱ssa̮ a̱ññā̱-ko̱ṇḍa̱ñño̱ tve̱va̮ nā̱ma̱ṁ a̮ho̱sī̱'ti͓.

Dha̱mma̮ca̱kka̱ppa̮va̱tta̮na̮-su̱tta̱ṁ ni̱ṭṭhi̮ta͓ṁ.

\chapterTocDelegatePageNumber
\chapter{The Characteristic of Not-Self}

\setTocDelegatedPageNumber
\englishText
\renewcommand{\englishTitle}{The Characteristic of Not-Self}

\begin{leader}
\soloinstr{Solo introduction}

All beings should take pains to understand the characteristic of
not-self, which provides matchless deliverance from self-view and
self-perception, as taught by the supreme Buddha.

This teaching is given so that those who meditate on experienceable
realities may arrive at perfect comprehension;

It is for the development of perfect understanding of these phenomena,
and for the investigation of all defiled mind-moments.

The consequence of this practice is total deliverance, so, desirous of
bringing this teaching forth with its great benefit, let us now recite
this Sutta.

\end{leader}

Thus have I heard.

At one time the Blessed One was dwelling at Benares in the deer park.
There he addressed the group of five bhikkhus:

‘Form, bhikkhus, is not-self. If, bhikkhus, form were self, then form
would not lead to affliction, and one might be able to say in regard to
form, “Let my form be thus, let my form not be thus.” But since,
bhikkhus, form is not-self, form therefore leads to affliction, and one
is not able to say in regard to form, “Let my form be thus, let my form
not be thus.”

‘Feeling is not-self. If, bhikkhus, feeling were self, feeling would
not lead to affliction, and one might be able to say in regard to
feeling, “Let my feeling be thus, let my feeling not be thus.” But
since, bhikkhus, feeling is not-self, feeling therefore leads to
affliction, and one is not able to say in regard to feeling, “Let my
feeling be thus, let my feeling not be thus.”

\chapterTocSubIndentTrue
\chapter{Anatta-lakkhaṇa Sutta}

\paliText
\renewcommand{\paliTitle}{Anatta-lakkhaṇa Sutta}

\begin{leader}
\soloinstr{Solo introduction}

{\setlength{\tabcolsep}{0.9em}
\begin{solotwochants}
Ya̱nta̱ṁ sa̱tte̱hi̮ du̱kkhe̱na̮ & ñe̱yya̱ṁ a̮na̱tta̮la̱kkha̮ṇa͓ṁ\\
A̱tta̮vā̱da̱tta̮sa̱ññā̱ṇa̱ṁ  & sa̱mma̮de̱va̮ vi̮mo̱ca̮na͓ṁ\\
Sa̱mbu̱ddho̱ ta̱ṁ pa̮kā̱se̱si̮ & di̱ṭṭha̮sa̱ccā̱na̮ yo̱gi̮na͓ṁ\\
U̱tta̮ri̱ṁ pa̮ṭi̮ve̱dhā̱ya̮ & bhā̱ve̱tu̱ṁ ñā̱ṇa̮mu̱tta̮ma͓ṁ\\
Ya̱nte̱sa̱ṁ di̱ṭṭha̮dha̱mmā̱na̱m & ñā̱ṇe̱nu̮pa̮pa̮ri̱kkha̮ta͓ṁ\\
Sa̱bbā̱sa̮ve̱hi̮ ci̱ttā̱ni̮ & vi̮mu̱cci̱ṁsu̮ a̮se̱sa̮to͓\\
Ta̮thā̱ ñā̱ṇā̱nu̱ssā̱re̱na̮ & sā̱sa̮na̱ṁ kā̱tu̮mi̱ccha̮ta͓ṁ\\
Sā̱dhū̱na̱ṁ a̱ttha̮si̱ddha̱ttha̱ṁ & ta̱ṁ su̱tta̱nta̱ṁ bha̮ṇā̱ma̮ se͓\\
\end{solotwochants}
}
\end{leader}

[E̱va̱ṁ me̱ su̮ta͓ṁ]

E̱ka̱ṁ sa̮ma̮ya̱ṁ bha̮ga̮vā̱ bā̱rā̱ṇa̮si̮ya̱ṁ vi̮ha̮ra̮ti̮ i̮si̮pa̮ta̮ne̱ mi̮ga̮dā̱ye̱. Ta̱tra̮ kho͓
bha̮ga̮vā̱ pa̱ñca̮va̱ggi̮ye̱ bhi̱kkhū̱ ā̱ma̱nte̱si͓:

Rū̱pa̱ṁ bhi̱kkha̮ve̱ a̮na̱ttā̱, rū̱pa̱ñca̮ hi̮da̱ṁ bhi̱kkha̮ve̱ a̱ttā̱ a̮bha̮vi̱ssa̮, na̮yi̮da̱ṁ rū̱pa͓ṁ
ā̱bā̱dhā̱ya̮ sa̱ṁva̱tte̱yya̮, la̱bbhe̱tha̮ ca̮ rū̱pe̱, e̱va̱ṁ me̱ rū̱pa̱ṁ ho̱tu̮, e̱va̱ṁ me̱ rū̱pa̱ṁ mā͓
a̮ho̱sī̱'ti̮. Ya̱smā̱ ca̮ kho̱ bhi̱kkha̮ve̱ rū̱pa̱ṁ a̮na̱ttā̱, ta̱smā̱ rū̱pa̱ṁ ā̱bā̱dhā̱ya̮ sa̱ṁva̱tta̮ti͓,
na̮ ca̮ la̱bbha̮ti̮ rū̱pe̱, e̱va̱ṁ me̱ rū̱pa̱ṁ ho̱tu̮, e̱va̱ṁ me̱ rū̱pa̱ṁ mā̱ a̮ho̱sī̱'ti͓.

Ve̱da̮nā̱ a̮na̱ttā̱, ve̱da̮nā̱ ca̮ hi̮da̱ṁ bhi̱kkha̮ve̱ a̱ttā̱ a̮bha̮vi̱ssa̮, na̮yi̮da̱ṁ ve̱da̮nā̱ ā̱bā̱dhā̱ya͓
sa̱ṁva̱tte̱yya̮, la̱bbhe̱tha̮ ca̮ ve̱da̮nā̱ya̮, e̱va̱ṁ me̱ ve̱da̮nā̱ ho̱tu̮, e̱va̱ṁ me̱ ve̱da̮nā̱ mā͓
a̮ho̱sī̱'ti̮. Ya̱smā̱ ca̮ kho̱ bhi̱kkha̮ve̱ ve̱da̮nā̱ a̮na̱ttā̱, ta̱smā̱ ve̱da̮nā̱ ā̱bā̱dhā̱ya͓
sa̱ṁva̱tta̮ti̮, na̮ ca̮ la̱bbha̮ti̮ ve̱da̮nā̱ya̮, e̱va̱ṁ me̱ ve̱da̮nā̱ ho̱tu̮, e̱va̱ṁ me̱ ve̱da̮nā̱ mā͓
a̮ho̱sī̱'ti͓.

\clearpage

\englishText
\markboth{\englishTitle}{\rightmark}

‘Perception is not-self. If, bhikkhus, perception were self, perception
would not lead to affliction, and one might be able to say in regard to
perception, “Let my perception be thus, let my perception not be thus.”
But since, bhikkhus, perception is not-self, perception therefore leads
to affliction, and one is not able to say in regard to perception, “Let
my perception be thus, let my perception not be thus.”

‘Mental formations are not-self. If, bhikkhus, mental formations were
self, mental formations would not lead to affliction, and one might be
able to say in regard to mental formations, “Let my mental formations be
thus, let my mental formations not be thus.” But since, bhikkhus, mental
formations are not-self, mental formations therefore lead to affliction,
and one is not able to say in regard to mental formations, “Let my
mental formations be thus, let my mental formations not be thus.”

‘Consciousness is not-self. If, bhikkhus, consciousness were self,
consciousness would not lead to affliction, and one might be able to say
in regard to consciousness, “Let my consciousness be thus, let my
consciousness not be thus.” But since, bhikkhus, consciousness is
not-self, consciousness therefore leads to affliction, and one is not
able to say in regard to consciousness, “Let my consciousness be thus,
let my consciousness not be thus.”

‘What do you think about this, bhikkhus? Is form permanent or
impermanent?’

‘Impermanent, Venerable Sir.’

‘But is that which is impermanent painful or pleasurable?’

‘Painful, Venerable Sir.’

‘But is it fit to consider that which is impermanent, painful, of a
nature to change, as “This is mine, I am this, this is my self”?’

‘It is not, Venerable Sir.’

\clearpage

\paliText
\markboth{\paliTitle}{\rightmark}

Sa̱ññā̱ a̮na̱ttā̱, sa̱ññā̱ ca̮ hi̮da̱ṁ bhi̱kkha̮ve̱ a̱ttā̱ a̮bha̮vi̱ssa̮, na̮yi̮da̱ṁ sa̱ññā̱ ā̱bā̱dhā̱ya͓
sa̱ṁva̱tte̱yya̮, la̱bbhe̱tha̮ ca̮ sa̱ññā̱ya̮, e̱va̱ṁ me̱ sa̱ññā̱ ho̱tu̮, e̱va̱ṁ me̱ sa̱ññā̱ mā͓
a̮ho̱sī̱'ti̮. Ya̱smā̱ ca̮ kho̱ bhi̱kkha̮ve̱ sa̱ññā̱ a̮na̱ttā̱, ta̱smā̱, sa̱ññā̱ ā̱bā̱dhā̱ya̮ sa̱ṁva̱tta̮ti͓,
na̮ ca̮ la̱bbha̮ti̮ sa̱ññā̱ya̮, e̱va̱ṁ me̱ sa̱ññā̱ ho̱tu̮, e̱va̱ṁ me̱ sa̱ññā̱ mā̱ a̮ho̱sī̱'ti͓.

Sa̱ṅkhā̱rā̱ a̮na̱ttā̱, sa̱ṅkhā̱rā̱ ca̮ hi̮da̱ṁ bhi̱kkha̮ve̱ a̱ttā̱ a̮bha̮vi̱ssa̱ṁsu̮, na̮yi̮da̱ṁ sa̱ṅkhā̱rā͓
ā̱bā̱dhā̱ya̮ sa̱ṁva̱tte̱yyu̱ṁ, la̱bbhe̱tha̮ ca̮ sa̱ṅkhā̱re̱su̮, e̱va̱ṁ me̱ sa̱ṅkhā̱rā̱ ho̱ntu̮, e̱va̱ṁ me͓
sa̱ṅkhā̱rā̱ mā̱ a̮he̱su̱n'ti̮. Ya̱smā̱ ca̮ kho̱ bhi̱kkha̮ve̱ sa̱ṅkhā̱rā̱ a̮na̱ttā̱, ta̱smā̱ sa̱ṅkhā̱rā͓
ā̱bā̱dhā̱ya̮ sa̱ṁva̱tta̱nti̮, na̮ ca̮ la̱bbha̮ti̮ sa̱ṅkhā̱re̱su̮, e̱va̱ṁ me̱ sa̱ṅkhā̱rā̱ ho̱ntu̮, e̱va̱ṁ me͓
sa̱ṅkhā̱rā̱ mā̱ a̮he̱su̱n'ti͓.

Vi̱ññā̱ṇa̱ṁ a̮na̱ttā̱, vi̱ññā̱ṇa̱ñca̮ hi̮da̱ṁ bhi̱kkha̮ve̱ a̱ttā̱ a̮bha̮vi̱ssa̮, na̮yi̮da̱ṁ vi̱ññā̱na͓m
ā̱bā̱dhā̱ya̮ sa̱ṁva̱tte̱yya̮, la̱bbhe̱tha̮ ca̮ vi̱ññā̱ne̱ e̱va̱ṁ me̱ vi̱ññā̱ṇa̱ṁ ho̱tu̮, e̱va̱ṁ me͓
vi̱ññā̱ṇa̱ṁ mā̱ a̮ho̱sī̱'ti̮. Ya̱smā̱ ca̮ kho̱ bhi̱kkha̮ve̱ vi̱ññā̱ṇa̱ṁ a̮na̱ttā̱, ta̱smā̱ vi̱ññā̱ṇa͓ṁ
ā̱bā̱dhā̱ya̮ sa̱ṁva̱tta̮ti̮, na̮ ca̮ la̱bbha̮ti̮ vi̱ññā̱ne̱, e̱va̱ṁ me̱ vi̱ññā̱ṇa̱ṁ ho̱tu̮, e̱va̱ṁ me͓
vi̱ññā̱ṇa̱ṁ mā̱ a̮ho̱sī̱'ti͓.

[Ta̱ṁ ki̱ṁ ma̱ñña̮tha̮ bhi̱kkha̮ve̱,] rū̱pa̱m ni̱cca̱ṁ vā̱ a̮ni̱cca̱ṁ vā̱'ti͓.

A̮ni̱cca̱ṁ bha̱nte͓.

Ya̱m pa̮nā̱ni̱cca̱ṁ, du̱kkha̱ṁ vā̱ ta̱ṁ su̮kha̱ṁ vā̱'ti͓.

Du̱kkha̱ṁ bha̱nte͓.

Ya̱m pa̮nā̱ni̱cca̱ṁ du̱kkha̱ṁ vi̮pa̮ri̮nā̱ma̮-dha̱mma̱ṁ, ka̱lla̱ṁ nu̮ ta̱ṁ sa̮ma̮nu̮pa̱ssi̮tu͓ṁ,
e̱ta̱ṁ ma̮ma̮, e̱so̱ha̮m'a̱smi̮, e̱so̱ me̱ a̱ttā̱'ti͓.

No̱ he̱ta̱ṁ bha̱nte͓.

\clearpage

\englishText
\markboth{\englishTitle}{\rightmark}

‘What do you think about this, bhikkhus? Is feeling permanent or
impermanent?’

‘Impermanent, Venerable Sir.’

‘But is that which is impermanent painful or pleasurable?’

‘Painful, Venerable Sir.’

‘But is it fit to consider that which is impermanent, painful, of a
nature to change, as “This is mine, I am this, this is my self”?’

‘It is not, Venerable Sir.’

‘What do you think about this, bhikkhus? Is perception permanent or
impermanent?’

‘Impermanent, Venerable Sir.’

‘But is that which is impermanent painful or pleasurable?’

‘Painful, Venerable Sir.’

‘But is it fit to consider that which is impermanent, painful, of a
nature to change, as “This is mine, I am this, this is my self”?’

‘It is not, Venerable Sir.’

‘What do you think about this, bhikkhus? Are mental formations
permanent or impermanent?’

‘Impermanent, Venerable Sir.’

‘But is that which is impermanent painful or pleasurable?’

‘Painful, Venerable Sir.’

‘But is it fit to consider that which is impermanent, painful, of a
nature to change, as “This is mine, I am this, this is my self”?’

‘It is not, Venerable Sir.’

\clearpage

\paliText
\markboth{\paliTitle}{\rightmark}

Ta̱ṁ ki̱ṁ ma̱ñña̮tha̮ bhi̱kkha̮ve̱, ve̱da̮nā̱ ni̱ccā̱ vā̱ a̮ni̱ccā̱ vā̱'ti͓.

A̮ni̱ccā̱ bha̱nte͓.

Ya̱m pa̮nā̱ni̱cca̱ṁ, du̱kkha̱ṁ vā̱ ta̱ṁ su̮kha̱ṁ vā̱'ti͓.

Du̱kkha̱ṁ bha̱nte͓.

Ya̱m pa̮nā̱ni̱cca̱ṁ du̱kkha̱ṁ vi̮pa̮ri̮nā̱ma̮-dha̱mma̱ṁ, ka̱lla̱ṁ nu̮ ta̱ṁ sa̮ma̮nu̮pa̱ssi̮tu͓ṁ,
e̱ta̱ṁ ma̮ma̮, e̱so̱ha̮m'a̱smi̮, e̱so̱ me̱ a̱ttā̱'ti͓.

No̱ he̱ta̱ṁ bha̱nte͓.

Ta̱ṁ ki̱ṁ ma̱ñña̮tha̮ bhi̱kkha̮ve̱, sa̱ññā̱ ni̱ccā̱ vā̱ a̮ni̱ccā̱ vā̱'ti͓.

A̮ni̱ccā̱ bha̱nte͓.

Ya̱m pa̮nā̱ni̱cca̱ṁ, du̱kkha̱ṁ vā̱ ta̱ṁ su̮kha̱ṁ vā̱'ti͓.

Du̱kkha̱ṁ bha̱nte͓.

Ya̱m pa̮nā̱ni̱cca̱ṁ du̱kkha̱ṁ vi̮pa̮ri̮nā̱ma̮-dha̱mma̱ṁ, ka̱lla̱ṁ nu̮ ta̱ṁ sa̮ma̮nu̮pa̱ssi̮tu͓ṁ,
e̱ta̱ṁ ma̮ma̮, e̱so̱ha̮m'a̱smi̮, e̱so̱ me̱ a̱ttā̱'ti͓.

No̱ he̱ta̱ṁ bha̱nte͓.

Ta̱ṁ ki̱ṁ ma̱ñña̮tha̮ bhi̱kkha̮ve̱, sa̱ṅkhā̱rā̱ ni̱ccā̱ vā̱ a̮ni̱ccā̱ vā̱'ti͓.

A̮ni̱ccā̱ bha̱nte͓.

Ya̱m pa̮nā̱ni̱cca̱ṁ, du̱kkha̱ṁ vā̱ ta̱ṁ su̮kha̱ṁ vā̱'ti͓.

Du̱kkha̱ṁ bha̱nte͓.

Ya̱m pa̮nā̱ni̱cca̱ṁ du̱kkha̱ṁ vi̮pa̮ri̮nā̱ma̮-dha̱mma̱ṁ, ka̱lla̱ṁ nu̮ ta̱ṁ sa̮ma̮nu̮pa̱ssi̮tu͓ṁ,
e̱ta̱ṁ ma̮ma̮, e̱so̱ha̮m'a̱smi̮, e̱so̱ me̱ a̱ttā̱'ti͓.

No̱ he̱ta̱ṁ bha̱nte͓.

\clearpage

\englishText
\markboth{\englishTitle}{\rightmark}

‘What do you think about this, bhikkhus? Is consciousness permanent or
impermanent?’

‘Impermanent, Venerable Sir.’

‘But is that which is impermanent painful or pleasurable?’

‘Painful, Venerable Sir.’

‘But is it fit to consider that which is impermanent, painful, of a
nature to change, as “This is mine, I am this, this is my self”?’

‘It is not, Venerable Sir.’

‘Wherefore, bhikkhus, whatever form there is, past, future, present,
internal or external, gross or subtle, inferior or superior, whether it
is far or near, all form should, by means of right wisdom, be seen as it
really is, thus: “This is not mine, I am not this, this is not my self.”

‘Whatever feeling there is, past, future, present, internal or
external, gross or subtle, inferior or superior, whether it is far or
near, all feeling should, by means of right wisdom, be seen as it really
is, thus: “This is not mine, I am not this, this is not my self.”

‘Whatever perception there is, past, future, present, internal or
external, gross or subtle, inferior or superior, whether it is far or
near, all perception should, by means of right wisdom, be seen as it really
is, thus: “This is not mine, I am not this, this is not my self.”

‘Whatever mental formations there are, past, future, present, internal
or external, gross or subtle, inferior or superior, whether they are far
or near, all mental formations should, by means of right wisdom, be seen
as they really are, thus: “This is not mine, I am not this, this is not
my self.”

\clearpage

\paliText
\markboth{\paliTitle}{\rightmark}

Ta̱ṁ ki̱ṁ ma̱ñña̮tha̮ bhi̱kkha̮ve̱, vi̱ññā̱ṇa̱ṁ ni̱cca̱ṁ vā̱ a̮ni̱cca̱ṁ vā̱'ti͓.

A̮ni̱cca̱ṁ bha̱nte͓.

Ya̱m pa̮nā̱ni̱cca̱ṁ, du̱kkha̱ṁ vā̱ ta̱ṁ su̮kha̱ṁ vā̱'ti͓.

Du̱kkha̱ṁ bha̱nte͓.

Ya̱m pa̮nā̱ni̱cca̱ṁ du̱kkha̱ṁ vi̮pa̮ri̮nā̱ma̮-dha̱mma̱ṁ, ka̱lla̱ṁ nu̮ ta̱ṁ sa̮ma̮nu̮pa̱ssi̮tu͓ṁ
e̱ta̱ṁ ma̮ma̮, e̱so̱ha̮m'a̱smi̮, e̱so̱ me̱ a̱ttā̱'ti͓.

No̱ he̱ta̱ṁ bha̱nte͓.

[Ta̱smā̱ ti̮ha̮ bhi̱kkha̮ve̱] ya̱ṁ ki̱ñci̮ rū̱pa̱ṁ a̮tī̱tā̱nā̱ga̮ta̮-pa̱ccu̱ppa̱nna̱ṁ a̱jjha̱tta͓ṁ
vā̱ ba̮hi̱ddhā̱ vā̱ o̱ḷā̱ri̮ka̱ṁ vā̱ su̮khu̮ma̱ṁ vā̱ hī̱na̱ṁ vā̱ pa̮ṇī̱ta̱ṁ vā̱ ya̱ndū̱re͓
sa̱nti̮ke̱ vā̱, sa̱bba̱ṁ rū̱pa̱ṁ ne̱ta̱ṁ ma̮ma̮, ne̱so̱ha̮m'a̱smi̮, na̮ me̱ so̱ a̱ttā̱'ti͓,
e̱va̮me̱ta̱ṁ ya̮thā̱bhū̱ta̱ṁ sa̱mma̱ppa̱ññā̱ya̮ da̱ṭṭha̱bba͓ṁ.

Yā̱ kā̱ci̮ ve̱da̮nā̱ a̮tī̱tā̱nā̱ga̮ta̮-pa̱ccu̱ppa̱nnā̱ a̱jjha̱ttā̱ vā̱ ba̮hi̱ddhā̱ vā̱ o̱ḷā̱ri̮kā͓
vā̱ su̮khu̮mā̱ vā̱ hī̱nā̱ vā̱ pa̮ṇī̱tā̱ vā̱ yā̱ dū̱re̱ sa̱nti̮ke̱ vā̱, sa̱bbā̱ ve̱da̮nā̱ ne̱ta͓ṁ
ma̮ma̮, ne̱so̱ha̮m'a̱smi̮, na̮ me̱ so̱ a̱ttā̱'ti̮, e̱va̮me̱ta̱ṁ ya̮thā̱bhū̱ta̱ṁ sa̱mma̱ppa̱ññā̱ya͓
da̱ṭṭha̱bba͓ṁ.

Yā̱ kā̱ci̮ sa̱ññā̱ a̮tī̱tā̱nā̱ga̮ta̮-pa̱ccu̱ppa̱nnā̱ a̱jjha̱ttā̱ vā̱ ba̮hi̱ddhā̱ vā̱ o̱ḷā̱ri̮kā̱ vā͓
su̮khu̮mā̱ vā̱ hī̱nā̱ vā̱ pa̮ṇī̱tā̱ vā̱ yā̱ dū̱re̱ sa̱nti̮ke̱ vā̱, sa̱bbā̱ sa̱ññā̱ ne̱ta̱ṁ ma̮ma͓,
ne̱so̱ha̮m'a̱smi̮, na̮ me̱ so̱ a̱ttā̱'ti̮, e̱va̮me̱ta̱ṁ ya̮thā̱bhū̱ta̱ṁ sa̱mma̱ppa̱ññā̱ya͓
da̱ṭṭha̱bba͓ṁ.

Ye̱ ke̱ci̮ sa̱ṅkhā̱rā̱ a̮tī̱tā̱nā̱ga̮ta̮-pa̱ccu̱ppa̱nnā̱ a̱jjha̱ttā̱ vā̱ ba̮hi̱ddhā̱ vā̱ o̱ḷā̱ri̮kā͓
vā̱ su̮khu̮mā̱ vā̱ hī̱nā̱ vā̱ pa̮ṇī̱tā̱ vā̱ ye̱ dū̱re̱ sa̱nti̮ke̱ vā̱, sa̱bbe̱ sa̱ṅkhā̱rā̱ ne̱ta͓ṁ
ma̮ma̮, ne̱so̱ha̮m'a̱smi̮, na̮ me̱ so̱ a̱ttā̱'ti̮, e̱va̮me̱ta̱ṁ ya̮thā̱bhū̱ta̱ṁ sa̱mma̱ppa̱ññā̱ya͓
da̱ṭṭha̱bba͓ṁ.

\clearpage

\englishText
\markboth{\englishTitle}{\rightmark}

‘Whatever consciousness there is, past, future, present, internal or
external, gross or subtle, inferior or superior, whether far or near,
all consciousness should, by means of right wisdom, be seen as it really
is, thus: “This is not mine, I am not this, this is not my self.”

‘Seeing in this way, bhikkhus, the wise noble disciple becomes
disenchanted with form, becomes disenchanted with feeling, becomes
disenchanted with perception, becomes disenchanted with mental
formations, becomes disenchanted with consciousness. Becoming
disenchanted, their passions fade away; with the fading of passion the
heart is liberated; with liberation there comes the knowledge: “It is
liberated,” and they know: “Destroyed is birth, the Holy Life has been
lived out, done is what had to be done, there is no more coming into any
state of being.”\thinspace ’

Thus spoke the Blessed One. Delighted, the group of five bhikkhus
rejoiced in what the Blessed One had said. Moreover, while this discourse was
being delivered, the minds of the five bhikkhus were freed from the
defilements, through clinging no more.

Thus ends the discourse on The Characteristic of Not-self.

\clearpage

\paliText
\markboth{\paliTitle}{\rightmark}

Ya̱ṁ ki̱ñci̮ vi̱ññā̱ṇa̱ṁ a̮tī̱tā̱nā̱ga̮ta̮-pa̱ccu̱ppa̱nna̱ṁ a̱jjha̱tta̱ṁ vā̱ ba̮hi̱ddhā̱ vā͓
o̱ḷā̱ri̮ka̱ṁ vā̱ su̮khu̮ma̱ṁ vā̱ hī̱na̱ṁ vā̱ pa̮ṇī̱ta̱ṁ vā̱ ya̱ndū̱re̱ sa̱nti̮ke̱ vā̱, sa̱bba͓ṁ
vi̱ññā̱ṇa̱ṁ ne̱ta̱ṁ ma̮ma̮, ne̱so̱ha̮m'a̱smi̮, na̮ me̱ so̱ a̱ttā̱'ti̮, e̱va̮me̱ta̱ṁ ya̮thā̱bhū̱ta͓ṁ
sa̱mma̱ppa̱ññā̱ya̮ da̱ṭṭha̱bba͓ṁ.

[E̱va̱ṁ pa̱ssa̱ṁ bhi̱kkha̮ve̱] su̱tvā̱ a̮ri̮ya̮sā̱va̮ko̱ rū̱pa̱smi̱m pi̮ ni̱bbi̱nda̮ti̮, ve̱da̮nā̱ya͓
pi̮ ni̱bbi̱nda̮ti̮, sa̱ññā̱ya̮ pi̮ ni̱bbi̱nda̮ti̮, sa̱ṅkhā̱re̱su̮ pi̮ ni̱bbi̱nda̮ti͓,
vi̱ññā̱ṇa̱smi̱m pi̮ ni̱bbi̱nda̮ti̮, ni̱bbi̱nda̱ṁ vi̮ra̱jja̮ti̮, vi̮rā̱gā̱ vi̮mu̱cca̮ti͓,
vi̮mu̱tta̱smi̱ṁ vi̮mu̱tta̱m i̮ti̮ ñā̱ṇa̱ṁ ho̱ti̮, khī̱ṇā̱ jā̱ti̮, vu̮si̮ta̱ṁ bra̱hma̮ca̮ri̮ya͓ṁ,
ka̮ta̱ṁ ka̮ra̮ṇī̱ya̱ṁ, nā̱pa̮ra̱ṁ i̱ttha̱ttā̱yā̱'ti̮ pa̮jā̱nā̱tī̱'ti͓.

[I̮da̮m-a̮vo̱ca̮ bha̮ga̮vā̱.] A̱tta̮ma̮nā̱ pa̱ñca̮va̱ggi̮yā̱ bhi̱kkhū̱ bha̮ga̮va̮to̱ bhā̱si̮ta͓ṁ
a̮bhi̮na̱ndu̱ṁ. I̮ma̱smi̱ñca̮ pa̮na̮ ve̱yyā̱ka̮ra̮ṇa̱smi̱ṁ bha̱ñña̮mā̱ne̱ pa̱ñca̮va̱ggi̮yā̱na͓ṁ
bhi̱kkhū̱na̱ṁ a̮nu̮pā̱dā̱ya̮ ā̱sa̮ve̱hi̮ ci̱ttā̱ni̮ vi̮mu̱cci̱ṁsū̱'ti͓.

A̮na̱tta̮la̱kkha̮ṇa̮-su̱tta̱ṁ ni̱ṭṭhi̮ta͓ṁ.

\chapterTocDelegatePageNumber
\chapter{The Fire Sermon}

\setTocDelegatedPageNumber
\englishText
\renewcommand{\englishTitle}{The Fire Sermon}

\begin{leader}
\soloinstr{Solo introduction}

With his skill in training the trainable, the All-transcendent Buddha,
lucid speaker, teacher of the highest knowledge,

He who expounds to the people the Dhamma and Vinaya that is fitting and
worthy, teaching with this wonderful parable about fire, meditators of
the highest skill;

He has liberated those who listen with the liberation that is utterly
complete, through true investigation, with wisdom\\ and attention.

Let us now recite this Sutta which describes the characteristics\\ of dukkha.

\end{leader}

Thus have I heard.

A̱t o̱ne̱ ti̮me̱ the̱ Ble̱sse̱d O̱ne̱ wa̱s sta̮yi̱ng ne̱a̱r Ga̮yā̱ a̱t Ga̮yā̱ He̱a̱d to̱ge̱the͓r
with a thousand bhikkhus. There the Blessed One addressed the bhikkhus
thus:

‘Bhikkhus, everything is burning. And what, bhikkhus, is everything
that is burning?

‘The eye, bhikkhus, is burning, forms are burning, eye consciousness is
burning, eye contact is burning, the feeling that arises from eye
contact, whether it is pleasant, painful, or neutral, that too is
burning. With what is it burning? I declare that it is burning with the
fires of passion, hatred, and delusion; it is burning with birth,
ageing, and death, with sorrow, lamentation, pain, grief, and despair.

\enlargethispage{2\baselineskip}

‘The ear is burning, sounds are burning, ear consciousness is burning,
ear contact is burning, the feeling that arises from ear contact,
whether it is pleasant, painful, or neutral, that too is burning. With
what is it burning? I declare that it is burning with the fires of
passion, hatred, and delusion; it is burning with birth, ageing, and
death, with sorrow, lamentation, pain, grief, and despair.

\chapterTocSubIndentTrue
\chapter{Āditta-pariyāya Sutta}

\paliText
\renewcommand{\paliTitle}{Āditta-pariyāya Sutta}

\begin{leader}
\soloinstr{Solo introduction}

\begin{solotwochants}
Ve̱ne̱yya̮da̮ma̮no̱pā̱ye̱  & sa̱bba̮so̱ pā̱ra̮mi̱ṁ ga̮to͓\\
A̮mo̱gha̮va̮ca̮no̱ bu̱ddho̱ & a̮bhi̱ññā̱yā̱nu̮sā̱sa̮ko͓\\
Ci̱ṇṇā̱nu̮rū̱pa̮to̱ cā̱pi̮ & dha̱mme̱na̮ vi̮na̮ya̱ṁ pa̮ja͓ṁ\\
Ci̱ṇṇā̱ggi̮pā̱ri̮ca̮ri̮yā̱na̱ṁ & sa̱mbo̱jjhā̱ra̮ha̮yo̱gi̮na͓ṁ\\
Ya̮mā̱di̱tta̮pa̮ri̮yā̱ya̱ṁ & de̱sa̮ya̱nto̱ ma̮no̱ha̮ra͓ṁ\\
Te̱ so̱tā̱ro̱ vi̮mo̱ce̱si̮ & a̮se̱kkhā̱ya̮ vi̮mu̱tti̮yā͓\\
Ta̮the̱vo̱pa̮pa̮ri̱kkhā̱ya̮ & vi̱ññū̱ṇa̱ṁ so̱tu̮mi̱ccha̮ta͓ṁ\\
Du̱kkha̮tā̱la̱kkha̮ṇo̱pā̱ya̱ṁ & ta̱ṁ su̱tta̱nta̱ṁ bha̮ṇā̱ma̮ se͓\\
\end{solotwochants}
\end{leader}

[E̱va̱ṁ me̱ su̮ta͓ṁ]

E̱ka̱ṁ sa̮ma̮ya̱ṁ bha̮ga̮vā̱ ga̮yā̱ya̱ṁ vi̮ha̮ra̮ti̮ ga̮yā̱sī̱se̱ sa̱ddhi̱ṁ bhi̱kkhu̮-sa̮ha̱sse̱na͓.
Ta̱tra̮ kho̱ bha̮ga̮vā̱ bhi̱kkhū̱ ā̱ma̱nte̱si͓:

Sa̱bba̱ṁ bhi̱kkha̮ve̱ ā̱di̱tta̱ṁ. Ki̱ñca̮ bhi̱kkha̮ve̱ sa̱bba̱ṁ ā̱di̱tta͓ṁ.

Ca̱kkhu̱ṁ bhi̱kkha̮ve̱ ā̱di̱tta̱ṁ, rū̱pā̱ ā̱di̱ttā̱, ca̱kkhu̮vi̱ññā̱ṇa̱ṁ ā̱di̱tta͓ṁ,
ca̱kkhu̮sa̱mpha̱sso̱ ā̱di̱tto̱, ya̱mpi̮da̱ṁ ca̱kkhu̮sa̱mpha̱ssa̮pa̱cca̮yā̱ u̱ppa̱jja̮ti͓
ve̱da̮yi̮ta̱ṁ su̮kha̱ṁ vā̱ du̱kkha̱ṁ vā̱ a̮du̱kkha̮ma̮su̮kha̱ṁ vā̱ ta̱m pi̮ ā̱di̱tta̱ṁ. Ke̱na͓
ā̱di̱tta̱ṁ. Ā̱di̱tta̱ṁ rā̱ga̱ggi̮nā̱ do̱sa̱ggi̮nā̱ mo̱ha̱ggi̮nā̱, ā̱di̱tta̱ṁ jā̱ti̮yā͓
ja̮rā̱ma̮ra̮ṇe̱na̮ so̱ke̱hi̮ pa̮ri̮de̱ve̱hi̮ du̱kkhe̱hi̮ do̱ma̮na̱sse̱hi̮ u̮pā̱yā̱se̱hi̮ ā̱di̱tta̱n'ti͓
va̮dā̱mi͓.

So̱ta̱ṁ ā̱di̱tta̱ṁ, sa̱ddā̱ ā̱di̱ttā̱, so̱ta̮vi̱ññā̱ṇa̱ṁ ā̱di̱tta̱ṁ, so̱ta̮sa̱mpha̱sso̱ ā̱di̱tto͓,
ya̱mpi̮da̱ṁ so̱ta̮sa̱mpha̱ssa̮pa̱cca̮yā̱ u̱ppa̱jja̮ti̮ ve̱da̮yi̮ta̱ṁ su̮kha̱ṁ vā̱ du̱kkha̱ṁ vā͓
a̮du̱kkha̮ma̮su̮kha̱ṁ vā̱ ta̱m pi̮ ā̱di̱tta̱ṁ. Ke̱na̮ ā̱di̱tta̱ṁ. Ā̱di̱tta̱ṁ rā̱ga̱ggi̮nā͓
do̱sa̱ggi̮nā̱ mo̱ha̱ggi̮nā̱, ā̱di̱tta̱ṁ jā̱ti̮yā̱ ja̮rā̱ma̮ra̮ṇe̱na̮ so̱ke̱hi̮ pa̮ri̮de̱ve̱hi͓
du̱kkhe̱hi̮ do̱ma̮na̱sse̱hi̮ u̮pā̱yā̱se̱hi̮ ā̱di̱tta̱n'ti̮ va̮dā̱mi͓.

\clearpage

\englishText
\markboth{\englishTitle}{\rightmark}

‘The nose is burning, odours are burning, nose consciousness is burning,
nose contact is burning, the feeling that arises from nose contact,
whether it is pleasant, painful, or neutral, that too is burning. With
what is it burning? I declare that it is burning with the fires of
passion, hatred, and delusion; it is burning with birth, ageing, and
death, with sorrow, lamentation, pain, grief, and despair.

‘The tongue is burning, tastes are burning, tongue consciousness is
burning, tongue contact is burning, the feeling that arises from tongue
contact, whether it is pleasant, painful, or neutral, that too is
burning. With what is it burning? I declare that it is burning with the
fires of passion, hatred, and delusion; it is burning with birth,
ageing, and death, with sorrow, lamentation, pain, grief, and despair.

‘The body is burning, tangible objects are burning, body consciousness
is burning, body contact is burning, the feeling that arises from body
contact, whether it is pleasant, painful, or neutral, that too is
burning. With what is it burning? I declare that it is burning with the
fires of passion, hatred, and delusion; it is burning with birth,
ageing, and death, with sorrow, lamentation, pain, grief, and despair.

‘The mind is burning, mental states are burning, mind consciousness is
burning, mind contact is burning, the feeling that arises through mind
contact, whether it is pleasant, painful, or neutral, that too is
burning. With what is it burning? I declare that it is burning with the
fires of passion, hatred, and delusion; it is burning with birth,
ageing, and death, with sorrow, lamentation, pain, grief, and despair.

‘Seeing thus, bhikkhus, the wise noble disciple becomes disenchanted
with the eye, disenchanted with forms, disenchanted with eye
consciousness, disenchanted with eye contact, and the feeling that
arises from eye contact --- whether it is pleasant, painful, or
neutral --- that too they become disenchanted with.

\clearpage

\paliText
\markboth{\paliTitle}{\rightmark}

Ghā̱na̱ṁ ā̱di̱tta̱ṁ, ga̱ndhā̱ ā̱di̱ttā̱, ghā̱na̮vi̱ññā̱ṇa̱ṁ ā̱di̱tta̱ṁ, ghā̱na̮sa̱mpha̱sso͓
ā̱di̱tto̱, ya̱mpi̮da̱ṁ ghā̱na̮sa̱mpha̱ssa̮pa̱cca̮yā̱ u̱ppa̱jja̮ti̮ ve̱da̮yi̮ta̱ṁ su̮kha̱ṁ vā͓
du̱kkha̱ṁ vā̱ a̮du̱kkha̮ma̮su̮kha̱ṁ vā̱ ta̱m pi̮ ā̱di̱tta̱ṁ. Ke̱na̮ ā̱di̱tta̱ṁ. Ā̱di̱tta͓ṁ
rā̱ga̱ggi̮nā̱ do̱sa̱ggi̮nā̱ mo̱ha̱ggi̮nā̱, ā̱di̱tta̱ṁ jā̱ti̮yā̱ ja̮rā̱ma̮ra̮ṇe̱na̮ so̱ke̱hi͓
pa̮ri̮de̱ve̱hi̮ du̱kkhe̱hi̮ do̱ma̮na̱sse̱hi̮ u̮pā̱yā̱se̱hi̮ ā̱di̱tta̱n'ti̮ va̮dā̱mi͓.

Ji̱vhā̱ ā̱di̱ttā̱, ra̮sā̱ ā̱di̱ttā̱, ji̱vhā̱vi̱ññā̱ṇa̱m ā̱di̱tta̱ṁ, ji̱vhā̱sa̱mpha̱sso̱ ā̱di̱tto͓,
ya̱mpi̮da̱ṁ ji̱vhā̱sa̱mpha̱ssa̮pa̱cca̮yā̱ u̱ppa̱jja̮ti̮ ve̱da̮yi̮ta̱ṁ su̮kha̱ṁ vā̱ du̱kkha̱ṁ vā͓
a̮du̱kkha̮ma̮su̮kha̱ṁ vā̱ ta̱m pi̮ ā̱di̱tta̱ṁ. Ke̱na̮ ā̱di̱tta̱ṁ. Ā̱di̱tta̱ṁ rā̱ga̱ggi̮nā͓
do̱sa̱ggi̮nā̱ mo̱ha̱ggi̮nā̱, ā̱di̱tta̱ṁ jā̱ti̮yā̱ ja̮rā̱ma̮ra̮ṇe̱na̮ so̱ke̱hi̮ pa̮ri̮de̱ve̱hi͓
du̱kkhe̱hi̮ do̱ma̮na̱sse̱hi̮ u̮pā̱yā̱se̱hi̮ ā̱di̱tta̱n'ti̮ va̮dā̱mi͓.

Kā̱yo̱ ā̱di̱tto̱, pho̱ṭṭha̱bbā̱ ā̱di̱ttā̱, kā̱ya̮vi̱ññā̱ṇa̱ṁ ā̱di̱tta̱ṁ, kā̱ya̮sa̱mpha̱sso͓
ā̱di̱tto̱, ya̱mpi̮da̱ṁ kā̱ya̮sa̱mpha̱ssa̮pa̱cca̮yā̱ u̱ppa̱jja̮ti̮ ve̱da̮yi̮ta̱ṁ su̮kha̱ṁ vā͓
du̱kkha̱ṁ vā̱ a̮du̱kkha̮ma̮su̮kha̱ṁ vā̱ ta̱m pi̮ ā̱di̱tta̱ṁ. Ke̱na̮ ā̱di̱tta̱ṁ. Ā̱di̱tta͓ṁ
rā̱ga̱ggi̮nā̱ do̱sa̱ggi̮nā̱ mo̱ha̱ggi̮nā̱, ā̱di̱tta̱ṁ jā̱ti̮yā̱ ja̮rā̱ma̮ra̮ṇe̱na̮ so̱ke̱hi͓
pa̮ri̮de̱ve̱hi̮ du̱kkhe̱hi̮ do̱ma̮na̱sse̱hi̮ u̮pā̱yā̱se̱hi̮ ā̱di̱tta̱n'ti̮ va̮dā̱mi͓.

Ma̮no̱ ā̱di̱tto̱, dha̱mmā̱ ā̱di̱ttā̱, ma̮no̱vi̱ññā̱ṇa̱ṁ ā̱di̱tta̱ṁ, ma̮no̱sa̱mpha̱sso̱ ā̱di̱tto͓,
ya̱mpi̮da̱ṁ ma̮no̱sa̱mpha̱ssa̮pa̱cca̮yā̱ u̱ppa̱jja̮ti̮ ve̱da̮yi̮ta̱ṁ su̮kha̱ṁ vā̱ du̱kkha̱ṁ vā͓
a̮du̱kkha̮ma̮su̮kha̱ṁ vā̱ ta̱m pi̮ ā̱di̱tta̱ṁ. Ke̱na̮ ā̱di̱tta̱ṁ. Ā̱di̱tta̱ṁ rā̱ga̱ggi̮nā͓
do̱sa̱ggi̮nā̱ mo̱ha̱ggi̮nā̱, ā̱di̱tta̱ṁ jā̱ti̮yā̱ ja̮rā̱ma̮ra̮ṇe̱na̮ so̱ke̱hi̮ pa̮ri̮de̱ve̱hi͓
du̱kkhe̱hi̮ do̱ma̮na̱sse̱hi̮ u̮pā̱yā̱se̱hi̮ ā̱di̱tta̱n'ti̮ va̮dā̱mi͓.

\enlargethispage{2\baselineskip}

[E̱va̱ṁ pa̱ssa̱ṁ bhi̱kkha̮ve̱] su̱tvā̱ a̮ri̮ya̮sā̱va̮ko̱ ca̱kkhu̱smi̱ṁ pi̮ ni̱bbi̱nda̮ti͓,
rū̱pe̱su̮ pi̮ ni̱bbi̱nda̮ti̮, ca̱kkhu̮vi̱ññā̱ṇe̱ pi̮ ni̱bbi̱nda̮ti̮, ca̱kkhu̮sa̱mpha̱sse̱pi͓
ni̱bbi̱nda̮ti̮, ya̱mpi̮da̱ṁ ca̱kkhu̮sa̱mpha̱ssa̮pa̱cca̮yā̱ u̱ppa̱jja̮ti̮ ve̱da̮yi̮ta̱ṁ su̮kha͓ṁ
vā̱ du̱kkha̱ṁ vā̱ a̮du̱kkha̮ma̮su̮kha̱ṁ vā̱ ta̱smi̱ṁ pi̮ ni̱bbi̱nda̮ti͓.

\clearpage

\englishText
\markboth{\englishTitle}{\rightmark}

‘They become disenchanted with the ear, disenchanted with sounds,
disenchanted with ear consciousness, disenchanted with ear contact, and
the feeling that arises from ear contact --- whether it is pleasant,
painful, or neutral --- that too they become disenchanted with.

‘They become disenchanted with the nose, disenchanted with odours,
disenchanted with nose consciousness, disenchanted with nose contact,
and the feeling that arises from nose contact --- whether it is pleasant,
painful, or neutral --- that too they become disenchanted with.

‘They become disenchanted with the tongue, disenchanted with tastes,
disenchanted with tongue consciousness, disenchanted with tongue
contact, and the feeling that arises from tongue contact --- whether it is
pleasant, painful, or neutral --- that too they become disenchanted with.

‘They become disenchanted with the body, disenchanted with tangible
objects, disenchanted with body consciousness, disenchanted with body
contact, and the feeling that arises from body contact --- whether it is
pleasant, painful, or neutral --- that too they become disenchanted with.

‘They become disenchanted with the mind, disenchanted with mental
states, disenchanted with mind consciousness, disenchanted with mind
contact, and the feeling that arises from mind contact --- whether it is
pleasant, painful, or neutral --- that too they become disenchanted with.

‘Becoming disenchanted, their passions fade away; with the fading of
passion the heart is liberated; with liberation there comes the
knowledge: “It is liberated,” and they know: “Destroyed is birth, the
Holy Life has been lived out, done is what had to be done, there is no
more coming into any state of being.”\thinspace ’

\enlargethispage{\baselineskip}

Thus spoke the Blessed One; delighted, the bhikkhus rejoiced in what the
Blessed One had said. Moreover, while this discourse was being uttered, the
minds of those thousand bhikkhus were freed from the defilements,
without any further attachment.

Thus ends The Fire Sermon.

\clearpage

\paliText
\markboth{\paliTitle}{\rightmark}

So̱ta̱smi̱ṁ pi̮ ni̱bbi̱nda̮ti̮, sa̱dde̱su̮ pi̮ ni̱bbi̱nda̮ti̮, so̱ta̮vi̱ññā̱ṇe̱ pi͓
ni̱bbi̱nda̮ti̮, so̱ta̮sa̱mpha̱sse̱pi̮ ni̱bbi̱nda̮ti̮, ya̱mpi̮da̱ṁ so̱ta̮sa̱mpha̱ssa̮pa̱cca̮yā͓
u̱ppa̱jja̮ti̮ ve̱da̮yi̮ta̱ṁ su̮kha̱ṁ vā̱ du̱kkha̱ṁ vā̱ a̮du̱kkha̮ma̮su̮kha̱ṁ vā̱ ta̱smi̱ṁ pi͓
nibbindati.

Ghā̱na̱smi̱ṁ pi̮ ni̱bbi̱nda̮ti̮, ga̱ndhe̱su̮ pi̮ ni̱bbi̱nda̮ti̮, ghā̱na̮vi̱ññā̱ṇe̱ pi͓
ni̱bbi̱nda̮ti̮, ghā̱na̮sa̱mpha̱sse̱pi̮ ni̱bbi̱nda̮ti̮, ya̱mpi̮da̱ṁ ghā̱na̮sa̱mpha̱ssa̮pa̱cca̮yā͓
u̱ppa̱jja̮ti̮ ve̱da̮yi̮ta̱ṁ su̮kha̱ṁ vā̱ du̱kkha̱ṁ vā̱ a̮du̱kkha̮ma̮su̮kha̱ṁ vā̱ ta̱smi̱ṁ pi͓
nibbindati.

Ji̱vhā̱ya̮ pi̮ ni̱bbi̱nda̮ti̮, ra̮se̱su̮ pi̮ ni̱bbi̱nda̮ti̮, ji̱vhā̱vi̱ññā̱ṇe̱ pi̮ ni̱bbi̱nda̮ti͓,
ji̱vhā̱sa̱mpha̱sse̱pi̮ ni̱bbi̱nda̮ti̮, ya̱mpi̮da̱ṁ ji̱vhā̱sa̱mpha̱ssa̮pa̱cca̮yā̱ u̱ppa̱jja̮ti͓
ve̱da̮yi̮ta̱ṁ su̮kha̱ṁ vā̱ du̱kkha̱ṁ vā̱ a̮du̱kkha̮ma̮su̮kha̱ṁ vā̱ ta̱smi̱ṁ pi̮ ni̱bbi̱nda̮ti͓.

Kā̱ya̱smi̱ṁ pi̮ ni̱bbi̱nda̮ti̮, pho̱ṭṭha̱bbe̱su̮ pi̮ ni̱bbi̱nda̮ti̮, kā̱ya̮vi̱ññā̱ṇe̱ pi͓
ni̱bbi̱nda̮ti̮, kā̱ya̮sa̱mpha̱sse̱pi̮ ni̱bbi̱nda̮ti̮, ya̱mpi̮da̱ṁ kā̱ya̮sa̱mpha̱ssa̮pa̱cca̮yā͓
u̱ppa̱jja̮ti̮ ve̱da̮yi̮ta̱ṁ su̮kha̱ṁ vā̱ du̱kkha̱ṁ vā̱ a̮du̱kkha̮ma̮su̮kha̱ṁ vā̱ ta̱smi̱ṁ pi͓
nibbindati.

Ma̮na̱smi̱ṁ pi̮ ni̱bbi̱nda̮ti̮, dha̱mme̱su̮ pi̮ ni̱bbi̱nda̮ti̮, ma̮no̱vi̱ññā̱ṇe̱ pi͓
ni̱bbi̱nda̮ti̮, ma̮no̱sa̱mpha̱sse̱pi̮ ni̱bbi̱nda̮ti̮, ya̱mpi̮da̱ṁ ma̮no̱sa̱mpha̱ssa̮pa̱cca̮yā͓
u̱ppa̱jja̮ti̮ ve̱da̮yi̮ta̱ṁ su̮kha̱ṁ vā̱ du̱kkha̱ṁ vā̱ a̮du̱kkha̮ma̮su̮kha̱ṁ vā̱ ta̱smi̱ṁ pi͓
nibbindati.

Ni̱bbi̱nda̱ṁ vi̮ra̱jja̮ti̮, vi̮rā̱gā̱ vi̮mu̱cca̮ti̮, vi̮mu̱tta̱smi̱ṁ, vi̮mu̱tta̱m i̮ti̮ ñā̱ṇa͓ṁ
ho̱ti̮, khī̱ṇā̱ jā̱ti̮, vu̮si̮ta̱ṁ bra̱hma̮ca̮ri̮ya̱ṁ, ka̮ta̱ṁ ka̮ra̮ṇī̱ya̱ṁ, nā̱pa̮ra͓ṁ
i̱ttha̱ttā̱yā̱'ti̮ pa̮jā̱nā̱tī̱'ti͓.

\enlargethispage{\baselineskip}

[I̮da̮m-a̮vo̱ca̮ bha̮ga̮vā̱.] A̱tta̮ma̮nā̱ te̱ bhi̱kkhū̱ bha̮ga̮va̮to̱ bhā̱si̮ta̱ṁ a̮bhi̮na̱ndu͓ṁ.
I̮ma̱smi̱ñca̮ pa̮na̮ ve̱yyā̱ka̮ra̮ṇa̱smi̱ṁ bha̱ñña̮mā̱ne̱ ta̱ssa̮ bhi̱kkhu̮-sa̮ha̱ssa̱ssa͓
a̮nu̮pā̱dā̱ya̮ ā̱sa̮ve̱hi̮ ci̱ttā̱ni̮ vi̮mu̱cci̱ṁsū̱'ti͓.

Ā̱di̱tta̮pa̮ri̮yā̱ya̮-su̱tta̱ṁ ni̱ṭṭhi̮ta͓ṁ.

\resumeNormalText

% End of suttas.tex
