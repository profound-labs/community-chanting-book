\chapter{Dedicación de Ofrendas}

[Yo so] bha꜕gavā a꜕rahaṁ sammāsambuddho

\begin{english}
Al Señor, el Maestro, que totalmente alcanzó la iluminación perfecta,
\end{english}

Svākkhā꜓to yena bha꜕gava꜓tā dhammo

\begin{english}
A la enseñanza, tan bien explicada por Él,
\end{english}

Supaṭi꜕panno yassa bha꜕gava꜕to sāvaka꜕saṅgho

\begin{english}
A los discípulos del Señor, que tan bien practicaron,
\end{english}

Tam-ma꜓yaṁ bha꜕gavantaṁ sa꜕dhammaṁ sa꜕saṅghaṁ

\begin{english}
A estos – al Buddha, al Dhamma y a la Saṅgha ---
\end{english}

Imehi꜓ sakkārehi꜕ yathārahaṁ āropi꜕tehi a꜕bhi꜓pūja꜕yāma

\begin{english}
Presentamos el debido homenaje con ofrendas.
\end{english}

Sādhu꜓ no bhante bha꜕gavā su꜕cira-parinibbu꜕topi

\begin{english}
Es beneficioso para nosotros, que habiéndose liberado el Señor,
\end{english}

Pacchi꜓mā-ja꜕na꜓tānu꜓kampa꜕-mānasā

\begin{english}
Haya tenido compasión por las generaciones futuras.
\end{english}

Ime sakkāre dugga꜕ta꜕-paṇṇākāra꜓-bhūte pa꜕ṭiggaṇhātu

\begin{english}
Que estas simples ofrendas sean aceptadas
\end{english}

Amhā꜓kaṁ dīgha꜕rattaṁ hi꜕tāya su꜕khāya

\begin{english}
Por nuestro duradero beneficio y por la felicidad que nos da.
\end{english}

\clearpage

[Arahaṁ] sammāsambuddho bha꜕gavā

\begin{english}
Al Maestro, el perfectamente Iluminado y Excelso ---
\end{english}

Buddhaṁ bha꜕gavantaṁ a꜕bhi꜓vādemi

\begin{english}
  Al Buddha, el Excelso, yo rindo homenaje.
  \instr{Reverencia}
\end{english}

[Svākkhā꜓to] bha꜕gava꜓tā dhammo

\begin{english}
 A la enseñanza, tan bien explicada por Él ---
\end{english}

Dhammaṁ namassāmi

\begin{english}
  Al Dhamma, yo rindo homenaje.
  \instr{Reverencia}
\end{english}

[Supaṭi꜕panno] bha꜕gava꜕to sāvaka꜕saṅgho

\begin{english}
A los discípulos del Señor que tan bien practicaron ---
\end{english}

Sa꜓ṅghaṁ na꜕māmi

\begin{english}
  A la Saṅgha, yo rindo homenaje.
  \instr{Reverencia}
\end{english}

\chapter{Homenaje Preliminar}

\begin{leader}
  [Ha꜓nda mayaṁ buddhassa꜕ bha꜕gavato\\ pubbabhāga-namakā꜕raṁ karomase]
\end{leader}

\begin{english}
  [Rindamos ahora homenaje preliminar al Buddha.]
\end{english}

\vspace{\baselineskip}

Namo tassa bha꜕gava꜕to araha꜕to sa꜓mmāsa꜓mbuddha꜕ssa

\instr{Tres veces}

\begin{english}
  Homenaje al Señor, Noble y Perfectamente Iluminado.

  \instr{Tres veces}
\end{english}

\clearpage

\chapter{Homenaje al Buddha}

\begin{leader}
  [Ha꜓nda mayaṁ buddhābhi꜕tthu꜕tiṁ karomase]
\end{leader}

\begin{english}
  [Cantemos ahora en elogio al Buddha.]
\end{english}

Yo so tathā꜓ga꜕to a꜕rahaṁ sammāsambuddho

\begin{english}
  El Tathāgata es puro y perfectamente iluminado.
\end{english}

Vijjāca꜕raṇa꜓-sampanno

\begin{english}
  Impecable en conducta y comprensión,
\end{english}

Su꜕ga꜕to

\begin{english}
  Realizado,
\end{english}

Loka꜕vi꜓dū

\begin{english}
  Conocedor de los mundos.
\end{english}

Anu꜓tta꜕ro purisa꜕damma-sārathi

\begin{english}
  Él entrena perfectamente aquellos que desean entrenarse.
\end{english}

Satthā deva-ma꜕nussānaṁ

\begin{english}
  Él es Maestro de dioses y humanos.
\end{english}

Buddho bha꜕gavā

\begin{english}
  Él es despierto y sagrado.
\end{english}

Yo imaṁ lokaṁ sa꜕devakaṁ sa꜕mārakaṁ sa꜕brahma꜕kaṁ

\begin{english}
  En este mundo \pause\ con sus dioses, demonios e espíritus gentiles,
\end{english}

Sassa꜓maṇa-brāhmaṇiṁ pa꜕jaṁ sa꜕deva-ma꜕nussa꜓ṁ sa꜕yaṁ a꜕bhiññā sacchika꜕tv꜓ā pa꜕vedesi

\begin{english}
  Sus buscadores y sabios, seres celestiales y humanos,\\ Él reveló la verdad a través de una comprensión profunda.
\end{english}

Yo dhammaṁ dese꜓si ā꜕di꜓-kalyāṇaṁ majjhe꜓-ka꜕lyāṇaṁ \\pa꜕riyosāna-k꜕alyāṇaṁ

\begin{english}
  Él explicó el Dhamma: Sublime al principio,\\ Sublime en el medio y Sublime al final.
\end{english}

Sāttha꜓ṁ sa꜕byañjanaṁ kevala-pa꜕ripuṇṇaṁ pa꜕risuddhaṁ \\brahma-ca꜕ri꜓yaṁ pa꜕kāsesi

\begin{english}
  Él explicó la vida espiritual de completa pureza,\\En su esencia y convenciones.
\end{english}

Tam-aha꜓ṁ bha꜕gavantaṁ a꜕bhi꜓pūja꜕yāmi tam-aha꜓ṁ bha꜕gavantaṁ \\si꜕rasā꜓ na꜕māmi

\begin{english}
  Yo canto mi elogio al Señor, yo saludo respetuosamente \\al Señor.
  \instr{Reverencia}
\end{english}

\clearpage

\chapter{Homenaje al Dhamma}

\begin{leader}
  [Ha꜓nda mayaṁ dhammābhi꜕tthu꜕tiṁ karomase]
\end{leader}

\begin{english}
  [Cantemos ahora en elogio al Dhamma.]
\end{english}

Yo so svākkhā꜓to bha꜕gava꜓tā dhammo

\begin{english}
  El Dhamma, tan bien explicado por el Señor,
\end{english}

Sa꜓ndiṭṭhi꜕ko

\begin{english}
  Presente aquí y ahora,
\end{english}

A꜕kāli꜕ko

\begin{english}
  Intemporal,
\end{english}

Ehi꜕passi꜕ko

\begin{english}
  Incentivando a investigar,
\end{english}

Opanayi꜕ko

\begin{english}
  Guiando al interior,
\end{english}

Pa꜕cca꜕ttaṁ vedi꜓ta꜕bbo viññūhi

\begin{english}
  Para ser experimentado individualmente por los sabios.
\end{english}

Tam-aha꜓ṁ dhammaṁ a꜕bhi꜓pūja꜕yāmi tam-aha꜓ṁ dhammaṁ \\si꜕rasā꜓ na꜕māmi

\begin{english}
  Yo canto mi elogio a esta enseñanza,\\
  yo saludo respetuosamente esta verdad.
  \instr{Reverencia}
\end{english}

\clearpage

\chapter{Homenaje a la Saṅgha}

\begin{leader}
  [Ha꜓nda mayaṁ saṅghābhi꜕tthu꜕tiṁ karomase]
\end{leader}

\begin{english}
  [Cantemos ahora en elogio a la Saṅgha.]
\end{english}

Yo so supaṭi꜕panno bha꜕gava꜕to sāvaka꜕saṅgho

\begin{english}
  Son los discípulos del Señor que practicaron correctamente,
\end{english}

Ujupaṭi꜕panno bha꜕gava꜕to sāvaka꜕saṅgho

\begin{english}
  Que practicaron directamente,
\end{english}

Ñāyapaṭi꜕panno bha꜕gava꜕to sāvaka꜕saṅgho

\begin{english}
  Que practicaron con reflexión,
\end{english}

Sā꜓mīci꜕pa꜕ṭi꜕panno bha꜕gava꜕to sāvaka꜕saṅgho

\begin{english}
  Aquellos que practicaron con integridad ---
\end{english}

Yadidaṁ cattāri purisa꜕yugāni aṭṭha꜓ purisa꜕pugga꜕lā

\begin{english}
  Es decir, los cuatro pares, los ocho tipos de Seres Nobles ---
\end{english}

Esa bha꜕gava꜕to sāvaka꜕saṅgho

\begin{english}
 Estos son los discípulos del Señor.
\end{english}

Āhu꜕neyyo

\begin{english}
  Tales discípulos son merecedores de presentes,
\end{english}

Pāhu꜕neyyo

\begin{english}
  Merecedores de hospitalidad,
\end{english}

\clearpage

Dakkhi꜕ṇeyyo

\begin{english}
  Merecedores de ofrendas,
\end{english}

Añja꜕li-ka꜕ra꜓ṇīyo

\begin{english}
  Merecedores de respeto;
\end{english}

Anu꜓tta꜕raṁ puññakkhe꜕ttaṁ lokassa

\begin{english}
  Ellos promueven el surgimiento \pause\ de un bien incomparable\\ en el mundo.
\end{english}

Tam-aha꜓ṁ saṅghaṁ a꜕bhi꜓pūja꜕yāmi tam-aha꜓ṁ saṅghaṁ\\ si꜕rasā꜓ na꜕māmi

\begin{english}
  Yo canto mi elogio a esta Saṅgha,\\
  yo saludo respetuosamente a esta Saṅgha.
  \instr{Reverencia}
\end{english}

\clearpage

\chapter{Salutación a la Joya Triple}

\begin{leader}
  [Ha꜓nda mayaṁ ratanattaya-paṇāma-gāthā꜓yo c'eva\\
  sa꜓ṁvega-parikittana-pāṭhañca꜕ bhaṇāmase]
\end{leader}

\begin{english}
  [Cantemos ahora nuestra salutación a la Joya Tripe \pause\ y el pasaje\\ que estimula el sentido de urgencia.]
\end{english}

Buddho su꜕suddho ka꜕ruṇā-maha꜓ṇṇavo

\begin{english}
  El Buddha \pause\ absolutamente puro, con compasión como un Océano,
\end{english}

Yo'ccanta꜕-suddhabba꜕ra-ñāṇa꜕-loca꜕no

\begin{english}
 Poseyendo la visión clara de Sabiduría,
\end{english}

Lokassa꜕ pāpūpa꜕ki꜓lesa꜕-ghāta꜕ko

\begin{english}
  Destructor de los defectos humanos mundanos ---
\end{english}

Vandāmi꜓ buddhaṁ a꜕ha꜓m-āda꜕rena꜕ taṁ

\begin{english}
  En plena devoción, ese Buddha yo venero.
\end{english}

Dhammo pa꜕dīpo vi꜕ya tassa꜕ satthu꜕no

\begin{english}
  La enseñanza del Maestro, como una luz,
\end{english}

Yo magga꜓-pākāma꜕ta꜕-bheda꜕-bhinna꜕ko

\begin{english}
  Iluminando el camino y su fruto: la Realidad Inmortal,
\end{english}

Lokuttaro yo ca꜕ ta꜕d-attha꜕-dīpa꜕no

\begin{english}
  Aquello que está más allá del mundo condicionado ---
\end{english}

Vandāmi꜓ dhammaṁ a꜕ha꜓m-āda꜕rena꜕ taṁ

\begin{english}
  En plena devoción, ese Dhamma yo venero.
\end{english}

Sa꜓ṅgho su꜕khettābhyati-khe꜕tta-sa꜓ññito

\begin{english}
  La Saṅgha, el mejor terreno para cultivo,
\end{english}

Yo diṭṭha꜓-santo su꜕ga꜕tānu꜕bodha꜕ko

\begin{english}
  Aquellos que realizaron la paz, despertando después del\\ Maestro,
\end{english}

Lolappa꜕hīno a꜕ri꜓yo su꜕medha꜕so

\begin{english}
  Nobles y Sabios, habiendo abandonado todo anhelo, ---
\end{english}

Vandāmi꜓ saṅghaṁ a꜕ha꜓m-āda꜕rena꜕ taṁ

\begin{english}
 En plena devoción, esa Saṅgha yo venero.
\end{english}

Iccevam-ekanta꜕bhi꜓pūja꜕-neyya꜕kaṁ vatthuttayaṁ \\vanda꜕ya꜕tābhi꜕saṅkha꜕taṁ

\begin{english}
 Esta salutación debe ser hecha \pause\ a lo que tiene valor.
\end{english}

Puññaṁ ma꜕yā yaṁ ma꜕ma꜕ sabbu꜕padda꜕vā mā ho꜓ntu꜕ ve tassa꜕ pa꜕bhāva꜕-siddhi꜕yā

\begin{english}
  A través del poder de esta acción benéfica, que puedan todos los obstáculos ser vencidos.
\end{english}

Idha tathā꜓ga꜕to loke u꜕ppanno a꜕rahaṁ sammāsambuddho

\begin{english}
  Aquel que conoce las cosas como son, vino a este mundo \pause\ y es un Arahant, un ser perfectamente despierto.
\end{english}

Dhammo ca꜕ desi꜕to niyyāni꜕ko u꜕pa꜕sa꜕miko pa꜕rinibbāni꜕ko sa꜓mbodha꜕gāmī su꜕ga꜕tappa꜕vedi꜕to

\enlargethispage{\baselineskip}

\begin{english}
  Purificando la vía que libera de la ilusión, tranquilizando y dirigiéndose hacia la paz perfecta, guiando a la Iluminación --- Este Camino Él dio a conocer.
\end{english}

Ma꜓yan-taṁ dhammaṁ su꜕tvā evaṁ jānāma

\begin{english}
  Habiendo oído la Enseñanza sabemos lo siguiente:
\end{english}

Jātipi꜕ dukkhā

\begin{english}
  El nacimiento es dukkha,
\end{english}

Jarāpi꜕ dukkhā

\begin{english}
  El envejecer es dukkha,
\end{english}

Ma꜕raṇampi꜕ dukkhaṁ

\begin{english}
  La muerte es dukkha;
\end{english}

So꜓ka-pa꜕rideva-dukkha꜕-domanass'u꜕pāyāsā꜓pi꜕ dukkhā

\begin{english}
  Tristeza, lamentación, dolor, angustia y desespero son dukkha;
\end{english}

Appiyehi꜕ sa꜓mpa꜕yogo dukkho

\begin{english}
  Asociación con lo que no gusta es dukkha;
\end{english}

Piyehi꜕ vi꜓ppa꜕yogo dukkho

\begin{english}
  Separación de lo que gusta es dukkha;
\end{english}

Yamp'iccha꜓ṁ na꜕ labhati tampi꜕ dukkhaṁ

\begin{english}
  No alcanzar aquello que se quiere es dukkha.
\end{english}

Sa꜓ṅkhittena pañcu꜕pādānakkha꜓ndhā dukkhā

\begin{english}
  Resumiendo, los cinco grupos de identificación son dukkha.
\end{english}

Seyya꜕thīdaṁ

\begin{english}
  Estos son:
\end{english}

\clearpage

Rūpūpādāna꜕kkha꜓ndho

\begin{english}
  Apego a forma,
\end{english}

Vedanūpādāna꜕kkha꜓ndho

\begin{english}
  Apego a sensación,
\end{english}

Sa꜓ññūpādāna꜕kkha꜓ndho

\begin{english}
  Apego a percepción,
\end{english}

Sa꜓ṅkhā꜓rūpādāna꜕kkha꜓ndho

\begin{english}
  Apego a las formaciones mentales,
\end{english}

Viññāṇūpādāna꜕kkha꜓ndho

\begin{english}
  Apego a cognición.
\end{english}

Yesaṁ pa꜕riññāya

\begin{english}
  Para esta total comprensión,
\end{english}

Dha꜕ramāno so꜓ bha꜕gavā evaṁ ba꜕hulaṁ sā꜓va꜕ke vi꜕neti

\begin{english}
  Frequentemente durante su vida, el Excelso instruyó así a sus\\ discípulos.
\end{english}

Evaṁ bhāgā ca꜕ panassa bha꜕gava꜕to sā꜓va꜕kesu a꜕nusā꜓sa꜕nī ba꜕hulā pa꜕vatta꜕ti

\begin{english}
  Más allá de eso, Él instruyó:
\end{english}


Rūpaṁ a꜕niccaṁ

\begin{english}
  La forma es impermanente,
\end{english}

Vedanā a꜕niccā

\begin{english}
  La sensación es impermanente,
\end{english}

Sa꜓ññā a꜕niccā

\begin{english}
  La percepción es impermanente,
\end{english}

Sa꜓ṅkhā꜓rā a꜕niccā

\begin{english}
  Las formaciones mentales son impermanentes,
\end{english}

Viññāṇaṁ a꜕niccaṁ

\begin{english}
  La cognición es impermanente;
\end{english}

Rūpaṁ a꜕nattā

\begin{english}
  La forma no es yo,
\end{english}

Vedanā a꜕nattā

\begin{english}
  La sensación no es yo,
\end{english}

Sa꜓ññā a꜕nattā

\begin{english}
  La percepción no es yo,
\end{english}

Sa꜓ṅkhā꜓rā a꜕nattā

\begin{english}
  Las formaciones mentales no son yo,
\end{english}

Viññāṇaṁ a꜕nattā

\begin{english}
  La cognición no es yo;
\end{english}

Sa꜕bbe sa꜓ṅkhā꜓rā a꜕niccā

\begin{english}
  Todas las condiciones son transitorias,
\end{english}

Sa꜕bbe dhammā a꜕nattā'ti

\begin{english}
  Todo el Dhamma no es yo.
\end{english}

\clearpage

Te ma꜓yaṁ otiṇṇāmha jāti꜕yā ja꜕rā-maraṇena

\begin{english}
  Todos nosotros nos vemos arrastados \pause\ por el nacimiento, el envejecimiento y la muerte,
\end{english}

So꜓kehi꜕ pa꜕ridevehi꜕ dukkhe꜓hi꜕ domanassehi꜕ u꜕pāyāsehi

\begin{english}
  Por la tristeza, lamentación, dolor, angustia y desespero,
\end{english}

Dukkho꜓tiṇṇā dukkha꜕-pa꜕retā

\begin{english}
  Arrastados por dukkha y obstruidos por dukkha.
\end{english}

Appeva nāmi꜓massa꜕ kevalassa꜕ dukkha-kkha꜓ndhassa꜕ anta꜕kiri꜓yā \\paññāyethā'ti

\begin{english}
  Que alcancemos el fin de toda esta masa de sufrimiento.
\end{english}

\begin{instruction}
  La parte que sigue es cantada solamente por los monjes.
\end{instruction}

Ci꜓ra꜓-pari꜕nibbutampi꜓ taṁ bha꜕gava꜓ntaṁ uddissa a꜕raha꜓ntaṁ sammāsambuddhaṁ

\begin{english}
  Recordando al Excelso, el Noble Maestro, el Perfectamente Iluminado, que hace mucho alcanzó el Parinibbāna,
\end{english}

Saddhā a꜕gārasmā anagāri꜓yaṁ pabba꜕ji꜕tā

\begin{english}
  Partimos con confianza \pause\ del hogar hacia la vida monástica.
\end{english}

Tasmi꜓ṁ bha꜕gavati brahma-ca꜕ri꜓yaṁ ca꜕rāma

\begin{english}
  Y tal como el Iluminado, practicamos la Vida Sagrada,
\end{english}

Bhikkhū꜓naṁ si꜓kkhāsā꜕jīva꜕-samāpannā

\begin{english}
  Completamente equipados con el sistema de entrenamiento de los Bhikkhus.
\end{english}

\clearpage

Taṁ no brahma-ca꜕ri꜓yaṁ imassa꜕ kevalassa꜕ dukkha-kkha꜓ndhassa꜕ anta꜕kiri꜓yāya sa꜓ṁva꜓tta꜕tu

\begin{english}
  Que esta vida purificada \pause\ pueda conducirnos \pause\ al término de toda esta masa de sufrimiento.
\end{english}

\begin{instruction}
  Una versión alternativa de la sección anterior, que puede ser también cantada por los laicos.
\end{instruction}

Ci꜓ra꜓-pari꜕nibbutampi꜓ taṁ bha꜕gava꜓ntaṁ saraṇaṁ ga꜕tā

\begin{english}
  El Excelso, que hace mucho alcanzó el Parinibbāna, es nuestro refugio.
\end{english}

Dha꜓mmañca sa꜓ṅghañca

\begin{english}
  Así como el Dhamma y la Saṅgha.
\end{english}

Tassa bha꜕gavato sā꜓sanaṁ yathā꜓-sati yathā꜓-balaṁ manasika꜕roma a꜕nupaṭipa꜓jjāma

\begin{english}
  Atentamente, seguimos el camino de aquel Excelso, con toda\\ nuestra fuerza y conciencia.
\end{english}

Sā꜓ sā꜓ no pa꜕ṭi꜓patti

\begin{english}
  Que el cultivo de esta práctica
\end{english}

Imassa꜕ kevalassa꜕ dukkha-kkha꜓ndhassa꜕ anta꜕kiri꜓yāya sa꜓ṁva꜓tta꜕tu

\begin{english}
 Pueda conducirnos al término \pause\ de todo tipo de sufrimiento.
\end{english}

\clearpage

\chapter{Homenaje de Cierre}

[Arahaṁ] sammāsambuddho bha꜕gavā

\begin{english}
  Al Maestro, el perfectamente Iluminado y Excelso ---
\end{english}

Buddhaṁ bha꜕gavantaṁ a꜕bhi꜓vādemi

\begin{english}
  Al Buddha, el Excelso, yo rindo homenaje.
  \instr{Reverencia}
\end{english}

[Svākkhā꜓to] bha꜕gava꜓tā dhammo

\begin{english}
  A la Enseñanza, tan bien explicada por Él ---
\end{english}

Dhammaṁ namassāmi

\begin{english}
  Al Dhamma, yo rindo homenaje.
  \instr{Reverencia}
\end{english}


[Supaṭi꜕panno] bha꜕gava꜕to sāvaka꜕saṅgho

\begin{english}
  A los discípulos del Excelso que tan bien practicaron ---
\end{english}

Sa꜓ṅghaṁ na꜕māmi

\begin{english}
  A la Saṅgha, yo rindo homenaje.
  \instr{Reverencia}
\end{english}
