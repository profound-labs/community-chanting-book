\chapterTocDelegatePageNumber
\chapter{Setting in Motion the Wheel of Dhamma}

\setTocDelegatedPageNumber
\englishText
\renewcommand{\englishTitle}{Setting in Motion the Wheel of Dhamma}

\begin{leader}
\soloinstr{Solo introduction}

This is the first teaching of the Tathāgata on attaining to unexcelled,
perfect enlightenment.

Here is the perfect turning of the incomparable wheel of Truth,
inestimable wherever it is expounded in the world.

Disclosed here are the two extremes, and the Middle Way, with the Four Noble
Truths and the purified knowledge and vision pointed out by the Lord of
Dhamma.

Let us chant together this Sutta proclaiming the supreme, independent
enlightenment that is widely renowned as ‘The~Turning of the Wheel of
the Dhamma.’

\end{leader}

Thus have I heard.

Once when the Blessed One was staying in the deer sanctuary at
Isipatana, near Benares, he spoke to the group of five bhikkhus:

‘These two extremes, bhikkhus, should not be followed by one who has
gone forth: sensual indulgence, which is low, coarse, vulgar, ignoble,
and unprofitable; and self-torture, which is painful, ignoble, and
unprofitable.

‘Bhikkhus, by avoiding these two extremes, the Tathāgata has realized
the Middle Way, which gives vision and understanding, which leads to
calm, penetration, enlightenment, to Nibbāna.

‘And what, bhikkhus, is the Middle Way realized by the Tathāgata, which
gives vision and understanding, which leads to calm, penetration,
enlightenment, to Nibbāna?

\chapterTocSubIndentTrue
\chapter{Dhammacakkappavattana Sutta}

\paliText
\renewcommand{\paliTitle}{Dhammacakkappavattana Sutta}

\begin{leader}
\soloinstr{Solo introduction}

\begin{solotwochants}
Anuttaraṁ abhisambodhiṁ & sambujjhitvā tathāgato\\
Pathamaṁ yaṁ adesesi & dhammacakkaṁ anuttaraṁ\\
Sammadeva pavattento & loke appativattiyaṁ\\
Yatthākkhātā ubho antā & paṭipatti ca majjhimā\\
Catūsvāriyasaccesu & visuddhaṁ ñāṇadassanaṁ\\
Desitaṁ dhammarājena & sammāsambodhikittanaṁ\\
Nāmena vissutaṁ suttaṁ & dhammacakkappavattanaṁ\\
Veyyākaraṇapāthena & saṅgītantam bhaṇāma se\\
\end{solotwochants}
\end{leader}

[Evaṁ me sutaṁ]

Ekaṁ samayaṁ bhagavā bārāṇasiyaṁ viharati isipatane migadāye. Tatra kho
bhagavā pañcavaggiye bhikkhū āmantesi:

Dve'me, bhikkhave, antā pabbajitena na sevitabbā: yo cāyaṁ kāmesu
kāma-sukh'allikānuyogo; hīno, gammo, pothujjaniko, anariyo,
anattha-sañhito; yo cāyaṁ atta-kilamathānuyogo; dukkho, anariyo,
anattha-sañhito.

Ete te, bhikkhave, ubho ante anupagamma majjhimā paṭipadā tathāgatena
abhisambuddhā cakkhukaraṇī, ñāṇakaraṇī, upasamāya, abhiññāya,
sambodhāya, nibbānāya saṁvattati.

Katamā ca sā, bhikkhave, majjhimā paṭipadā tathāgatena abhisambuddhā
cakkhukaraṇī ñāṇakaraṇī, upasamāya, abhiññāya, sambodhāya, nibbānāya
saṁvattati.

\clearpage

\englishText
\markboth{\englishTitle}{\rightmark}

‘It is just this Noble Eightfold Path, namely:

‘Right View, Right Intention, Right Speech, Right Action, Right
Livelihood, Right Effort, Right Mindfulness, and Right Concentration.

‘Truly, bhikkhus, this Middle Way understood by the Tathāgata produces
vision, produces knowledge, and leads to calm, penetration,
enlightenment, to Nibbāna.

‘This, bhikkhus, is the Noble Truth of dukkha:

‘Birth is dukkha, ageing is dukkha, death is dukkha, grief,
lamentation, pain, sorrow and despair are dukkha, association with the
disliked is dukkha, separation from the liked is dukkha, not to get what
one wants is dukkha. In brief, clinging to the five khandhas is dukkha.

‘This, bhikkhus, is the Noble Truth of the cause of dukkha:

‘The craving which causes rebirth and is bound up with pleasure and
lust, ever seeking fresh delight, now here, now there; namely, craving
for sense pleasure, craving for existence, and craving for annihilation.

‘This, bhikkhus, is the Noble Truth of the cessation of dukkha:

‘The complete cessation, giving up, abandonment of that craving,
complete release from that craving, and complete detachment from it.

‘This, bhikkhus, is the Noble Truth of the way leading to the cessation
of dukkha:

‘Only this Noble Eightfold Path; namely, Right View, Right Intention,
Right Speech, Right Action, Right Livelihood, Right Effort, Right
Mindfulness, and Right Concentration.

‘With the thought, “This is the Noble Truth of dukkha,” there arose in
me, bhikkhus, vision, knowledge, insight, wisdom, light, concerning
things unknown before.

\clearpage

\paliText
\markboth{\paliTitle}{\rightmark}

Ayam-eva ariyo aṭṭhaṅgiko maggo seyyathīdaṁ:

Sammā-diṭṭhi, sammā-saṅkappo, sammā-vācā, sammā-kammanto, sammā-ājīvo,
sammā-vāyāmo, sammā-sati, sammā-samādhi.

Ayaṁ kho sā, bhikkhave, majjhimā paṭipadā tathāgatena abhisambuddhā
cakkhukaraṇī ñāṇakaraṇī, upasamāya, abhiññāya, sambodhāya, nibbānāya
saṁvattati.

Idaṁ kho pana, bhikkhave, dukkhaṁ ariya-saccaṁ:

Jātipi dukkhā, jarāpi dukkhā, maranampi dukkhaṁ,
soka-parideva-dukkha-domanass'upāyāsāpi dukkhā, appiyehi sampayogo
dukkho, piyehi vippayogo dukkho, yamp'icchaṁ na labhati tampi dukkhaṁ,
saṅkhittena pañcupādānakkhandhā dukkhā.

Idaṁ kho pana, bhikkhave, dukkha-samudayo ariya-saccaṁ:

Yā'yaṁ taṇhā ponobbhavikā nandi-rāga-sahagatā tatra-tatrābhinandinī
seyyathīdaṁ: kāma-taṇhā, bhava-taṇhā, vibhava-taṇhā.

Idaṁ kho pana, bhikkhave, dukkha-nirodho ariya-saccaṁ:

Yo tassā yeva taṇhāya asesa-virāga-nirodho, cāgo, paṭinissaggo, mutti,
anālayo.

Idaṁ kho pana, bhikkhave, dukkha-nirodha-gāminī paṭipadā ariya-saccaṁ:

Ayam-eva ariyo aṭṭhaṅgiko maggo seyyathīdam: sammā-diṭṭhi,
sammā-saṅkappo, sammā-vācā, sammā-kammanto, sammā-ājīvo, sammā-vāyāmo,
sammā-sati, sammā-samādhi.

\enlargethispage{\baselineskip}

[Idaṁ dukkhaṁ] ariya-saccan'ti me bhikkhave, pubbe ananussutesu dhammesu
cakkhuṁ udapādi, ñāṇaṁ udapādi, paññā udapādi, vijjā udapādi, āloko
udapādi.

\clearpage

\englishText
\markboth{\englishTitle}{\rightmark}

‘With the thought, “This is the Noble Truth of dukkha, and this dukkha
has to be understood,” there arose in me, bhikkhus, vision, knowledge,
insight, wisdom, light, concerning things unknown before.

‘With the thought, “This is the Noble Truth of dukkha, and this dukkha
has been understood,” there arose in me, bhikkhus, vision, knowledge,
insight, wisdom, light, concerning things unknown before.

‘With the thought, “This is the Noble Truth of the cause of dukkha,”
there arose in me, bhikkhus, vision, knowledge, insight, wisdom, light,
concerning things unknown before.

‘With the thought, “This is the Noble Truth of the cause of dukkha, and
this cause of dukkha has to be abandoned,” there arose in me, bhikkhus,
vision, knowledge, insight, wisdom, light, concerning things unknown
before.

‘With the thought, “This is the Noble Truth of the cause of dukkha, and
this cause of dukkha has been abandoned,” there arose in me, bhikkhus,
vision, knowledge, insight, wisdom, light, concerning things unknown
before.

‘With the thought, “This is the Noble Truth of the cessation of dukkha,”
there arose in me, bhikkhus, vision, knowledge, insight, wisdom, light,
concerning things unknown before.

‘With the thought, “This is the Noble Truth of the cessation of dukkha,
and this cessation of dukkha has to be realized,” there arose in me,
bhikkhus, vision, knowledge, insight, wisdom, light, concerning things
unknown before.

‘With the thought, “This is the Noble Truth of the cessation of dukkha,
and this cessation of dukkha has been realized,” there arose in me,
bhikkhus, vision, knowledge, insight, wisdom, light, concerning things
unknown before.

\clearpage

\paliText
\markboth{\paliTitle}{\rightmark}

Taṁ kho pan'idaṁ dukkhaṁ ariya-saccaṁ pariññeyyan'ti me bhikkhave, pubbe
ananussutesu dhammesu cakkhuṁ udapādi, ñāṇaṁ udapādi, paññā udapādi,
vijjā udapādi, āloko udapādi.

Taṁ kho pan'idaṁ dukkhaṁ ariya-saccaṁ pariññātan'ti me bhikkhave, pubbe
ananussutesu dhammesu cakkhuṁ udapādi, ñāṇaṁ udapādi, paññā udapādi,
vijjā udapādi, āloko udapādi.

Idaṁ dukkha-samudayo ariya-saccan'ti me bhikkhave, pubbe ananussutesu
dhammesu cakkhuṁ udapādi, ñāṇaṁ udapādi, paññā udapādi, vijjā udapādi,
āloko udapādi.

Taṁ kho pan'idaṁ dukkha-samudayo ariyasaccaṁ pahātabban'ti me bhikkhave,
pubbe ananussutesu dhammesu cakkhuṁ udapādi, ñāṇaṁ udapādi, paññā
udapādi, vijjā udapādi, āloko udapādi.

Taṁ kho pan'idaṁ dukkha-samudayo ariya-saccaṁ pahīnan'ti me bhikkhave, pubbe
ananussutesu dhammesu cakkhuṁ udapādi, ñāṇaṁ udapādi, paññā udapādi,
vijjā udapādi, āloko udapādi.

Idaṁ dukkha-nirodho ariya-saccan'ti me bhikkhave, pubbe ananussutesu
dhammesu cakkhuṁ udapādi, ñāṇaṁ udapādi, paññā udapādi, vijjā udapādi,
āloko udapādi.

Taṁ kho pan'idaṁ dukkha-nirodho ariya-saccaṁ sacchikātabban'ti me bhikkhave,
pubbe ananussutesu dhammesu cakkhuṁ udapādi, ñāṇaṁ udapādi, paññā
udapādi, vijjā, udapādi āloko udapādi.

Taṁ kho pan'idaṁ dukkha-nirodho ariya-saccaṁ sacchikatan'ti me bhikkhave,
pubbe ananussutesu dhammesu cakkhuṁ udapādi, ñāṇaṁ udapādi, paññā
udapādi, vijjā udapādi, āloko udapādi.

\clearpage

\englishText
\markboth{\englishTitle}{\rightmark}

‘With the thought, “This is the Noble Truth of the way leading to the
cessation of dukkha,” there arose in me, bhikkhus, vision, knowledge,
insight, wisdom, light, concerning things unknown before.

‘With the thought, “This Noble Truth of the way leading to the cessation
of dukkha has to be developed,” there arose in me, bhikkhus, vision,
knowledge, insight, wisdom, light, concerning things unknown before.

‘With the thought, “This Noble Truth of the way leading to the cessation
of dukkha has been developed,” there arose in me, bhikkhus, vision,
knowledge, insight, wisdom, light, concerning things unknown before.

‘So long, bhikkhus, as my knowledge and vision of reality regarding
these Four Noble Truths, in their three phases and twelve aspects, was
not fully clear to me, I did not declare to the world of spirits,
demons, and gods, with its seekers and sages, celestial and human
beings, the realization of incomparable, perfect enlightenment.

‘But when, bhikkhus, my knowledge and vision of reality regarding these
Four Noble Truths, in their three phases and twelve aspects, was fully
clear to me, I declared to the world of spirits, demons, and gods, with
its seekers and sages, celestial and human beings, that I had realized
incomparable, perfect enlightenment.

‘Knowledge and vision arose: “Unshakeable is my deliverance; this is
the last birth, there will be no more renewal of being.”\thinspace ’

Thus spoke the Blessed One. Glad at heart, the group of five bhikkhus
approved of the words of the Blessed One.

\clearpage

\paliText
\markboth{\paliTitle}{\rightmark}

Idaṁ dukkha-nirodha-gāminī paṭipadā ariya-saccan'ti me bhikkhave, pubbe
ananussutesu dhammesu cakkhuṁ udapādi, ñāṇaṁ udapādi, paññā udapādi,
vijjā udapādi, āloko udapādi.

Taṁ kho pan'idaṁ dukkha-nirodha-gāminī paṭipadā ariya-saccaṁ bhāvetabban'ti
me bhikkhave, pubbe ananussutesu dhammesu cakkhuṁ udapādi, ñāṇaṁ
udapādi, paññā udapādi, vijjā udapādi, āloko udapādi.

Taṁ kho pan'idaṁ dukkha-nirodha-gāminī paṭipadā ariya-saccaṁ bhāvitan'ti me
bhikkhave, pubbe ananussutesu dhammesu cakkhuṁ udapādi, ñāṇaṁ udapādi,
paññā udapādi, vijjā udapādi, āloko udapādi.

[Yāva kīvañca me bhikkhave,] imesu catūsu ariya-saccesu evan-ti-parivaṭṭaṁ
dvādas'ākāraṁ yathā-bhūtaṁ ñāṇa-dassanaṁ na suvisuddhaṁ ahosi, n'eva tāv'āhaṁ
bhikkhave, sadevake loke samārake sabrahmake sassamaṇa-brāhmaṇiyā pajāya
sadeva-manussāya anuttaraṁ sammā-sambodhiṁ abhisambuddho paccaññāsiṁ.

Yato ca kho me bhikkhave, imesu catūsu ariya-saccesu evan-ti-parivaṭṭaṁ
dvādas'ākāraṁ yathā-bhūtaṁ ñāṇa-dassanaṁ suvisuddham ahosi, ath'āham
bhikkhave, sadevake loke samārake sabrahmake sassamaṇa-brāhmaṇiyā pajāya
sadeva-manussāya anuttaraṁ sammā-sambodhiṁ abhisambuddho paccaññāsiṁ.

Ñāṇañca pana me dassanaṁ udapādi, akuppā me vimutti ayam-antimā jāti,
natthi dāni punabbhavo'ti.

Idam-avoca bhagavā. Attamanā pañcavaggiyā bhikkhū bhagavato bhāsitaṁ
abhinanduṁ.

\clearpage

\englishText
\markboth{\englishTitle}{\rightmark}

As this exposition was proceeding, the spotless, immaculate vision of
the Dhamma appeared to the Venerable Koṇḍañña and he knew: ‘Everything
that has the nature to arise has the nature to cease.’

When the Blessed One had set in motion the Wheel of Dhamma, the
Earthbound devas proclaimed with one voice,

‘The incomparable Wheel of Dhamma has been set in motion by the Blessed
One in the deer sanctuary at Isipatana, near Benares, and no seeker,
brahmin, celestial being, demon, god, or any other being in the world
can stop it.’

Having heard what the Earthbound devas said, the devas of the Four Great
Kings proclaimed with one voice\ldots

Having heard what the devas of the Four Great Kings said, the devas of
the Thirty-three proclaimed with one voice\ldots

Having heard what the devas of the Thirty-three said, the Yāma devas
proclaimed with one voice\ldots

Having heard what the Yāma devas said, the Devas of Delight proclaimed
with one voice\ldots

Having heard what the Devas of Delight said, the Devas Who Delight in
Creating, proclaimed with one voice\ldots

Having heard what the Devas Who Delight in Creating said, the Devas Who
Delight in the Creations of Others proclaimed with one voice\ldots

Having heard what the Devas Who Delight in the Creations of Others said,
the Brahma gods proclaimed in one voice,

‘The incomparable Wheel of Dhamma has been set in motion by the Blessed
One in the deer sanctuary at Isipatana, near Benares, and no seeker,
brahmin, celestial being, demon, god, or any other being in the world
can stop it.’

\clearpage

\paliText
\markboth{\paliTitle}{\rightmark}

Imasmiñca pana veyyākaraṇasmiṁ bhaññamāne āyasmato koṇḍaññassa virajaṁ
vītamalaṁ dhammacakkhuṁ udapādi: yaṁ kiñci samudaya-dhammaṁ sabban-taṁ
nirodha-dhamman'ti.

[Pavattite ca bhagavatā] dhammacakke bhummā devā saddamanussāvesuṁ:

Etaṁ bhagavatā bārāṇasiyaṁ isipatane migadāye anuttaraṁ dhammacakkaṁ
pavattitaṁ appaṭivattiyaṁ samaṇena vā brāhmaṇena vā devena vā mārena vā
brahmunā vā kenaci vā lokasmin'ti.

Bhummānaṁ devānaṁ saddaṁ sutvā, cātummahārājikā devā
saddamanussāvesuṁ\ldots

Cātummahārājikānaṁ devānaṁ saddaṁ sutvā, tāvatiṁsā devā
saddamanussāvesuṁ\ldots

Tāvatiṁsānaṁ devānaṁ saddaṁ sutvā, yāmā devā saddamanussāvesuṁ\ldots

Yāmānaṁ devānaṁ saddaṁ sutvā, tusitā devā saddamanussāvesuṁ\ldots

Tusitānaṁ devānaṁ saddaṁ sutvā, nimmānaratī devā saddamanussāvesum\ldots

Nimmānaratīnaṁ devānaṁ saddaṁ sutvā, paranimmitavasavattī devā
saddamanussāvesuṁ\ldots

Paranimmitavasavattīnaṁ devānaṁ saddaṁ sutvā, brahmakāyikā devā
saddamanussāvesuṁ:

Etaṁ bhagavatā bārāṇasiyaṁ isipatane migadāye anuttaraṁ dhammacakkaṁ
pavattitaṁ appaṭivattiyaṁ samaṇena vā brāhmaṇena vā devena vā mārena vā
brahmunā vā kenaci vā lokasmin'ti.

\clearpage

\englishText
\markboth{\englishTitle}{\rightmark}

Thus in a moment, an instant, a flash, word of the Setting in Motion of
the Wheel of Dhamma went forth up to the Brahma world, and the
ten-thousandfold universal system trembled and quaked and shook, and a
boundless, sublime radiance surpassing the power of devas appeared on
earth.

Then the Blessed One made the utterance,

‘Truly, Koṇḍañña has understood, Koṇḍañña has understood!’ Thus it was
that the Venerable Koṇḍañña got the name Aññā-Koṇḍañña: ‘Koṇḍañña Who
Understands.’

Thus ends the discourse on Setting in Motion the Wheel of Dhamma.

\clearpage

\paliText
\markboth{\paliTitle}{\rightmark}

Iti'ha tena khaṇena, tena muhuttena, yāva brahmalokā saddo abbhuggacchi.
Ayañca dasa-sahassī lokadhātu saṅkampi sampakampi sampavedhi, appamāṇo ca
oḷāro obhāso loke pāturahosi atikkammeva devānaṁ devānubhāvaṁ.

Atha kho bhagavā udānaṁ udānesi:

Aññāsi vata bho koṇḍañño, aññāsi vata bho koṇḍañño'ti. Iti hidaṁ āyasmato
koṇḍaññassa aññā-koṇḍañño tveva nāmaṁ ahosī'ti.

Dhammacakkappavattana-suttaṁ niṭṭhitaṁ.

\chapterTocDelegatePageNumber
\chapter{The Characteristic of Not-Self}

\setTocDelegatedPageNumber
\englishText
\renewcommand{\englishTitle}{The Characteristic of Not-Self}

\begin{leader}
\soloinstr{Solo introduction}

All beings should take pains to understand the characteristic of
not-self, which provides matchless deliverance from self-view and
self-perception, as taught by the supreme Buddha.

This teaching is given so that those who meditate on experienceable
realities may arrive at perfect comprehension;

It is for the development of perfect understanding of these phenomena,
and for the investigation of all defiled mind-moments.

The consequence of this practice is total deliverance, so, desirous of
bringing this teaching forth with its great benefit, let us now recite
this Sutta.

\end{leader}

Thus have I heard.

At one time the Blessed One was dwelling at Benares in the deer park.
There he addressed the group of five bhikkhus:

‘Form, bhikkhus, is not-self. If, bhikkhus, form were self, then form
would not lead to affliction, and one might be able to say in regard to
form, “Let my form be thus, let my form not be thus.” But since,
bhikkhus, form is not-self, form therefore leads to affliction, and one
is not able to say in regard to form, “Let my form be thus, let my form
not be thus.”

‘Feeling is not-self. If, bhikkhus, feeling were self, feeling would
not lead to affliction, and one might be able to say in regard to
feeling, “Let my feeling be thus, let my feeling not be thus.” But
since, bhikkhus, feeling is not-self, feeling therefore leads to
affliction, and one is not able to say in regard to feeling, “Let my
feeling be thus, let my feeling not be thus.”

\chapterTocSubIndentTrue
\chapter{Anatta-lakkhaṇa Sutta}

\paliText
\renewcommand{\paliTitle}{Anatta-lakkhaṇa Sutta}

\begin{leader}
\soloinstr{Solo introduction}

{\setlength{\tabcolsep}{0.9em}
\begin{solotwochants}
Yantaṁ sattehi dukkhena & ñeyyaṁ anattalakkhaṇaṁ\\
Attavādattasaññāṇaṁ  & sammadeva vimocanaṁ\\
Sambuddho taṁ pakāsesi & diṭṭhasaccāna yoginaṁ\\
Uttariṁ paṭivedhāya & bhāvetuṁ ñāṇamuttamaṁ\\
Yantesaṁ diṭṭhadhammānam & ñāṇenupaparikkhataṁ\\
Sabbāsavehi cittāni & vimucciṁsu asesato\\
Tathā ñāṇānussārena & sāsanaṁ kātumicchataṁ\\
Sādhūnaṁ atthasiddhatthaṁ & taṁ suttantaṁ bhaṇāma se\\
\end{solotwochants}
}
\end{leader}

[Evaṁ me sutaṁ]

Ekaṁ samayaṁ bhagavā bārāṇasiyaṁ viharati isipatane migadāye. Tatra kho
bhagavā pañcavaggiye bhikkhū āmantesi:

Rūpaṁ bhikkhave anattā, rūpañca hidaṁ bhikkhave attā abhavissa, nayidaṁ rūpaṁ
ābādhāya saṁvatteyya, labbhetha ca rūpe, evaṁ me rūpaṁ hotu, evaṁ me rūpaṁ mā
ahosī'ti. Yasmā ca kho bhikkhave rūpaṁ anattā, tasmā rūpaṁ ābādhāya saṁvattati,
na ca labbhati rūpe, evaṁ me rūpaṁ hotu, evaṁ me rūpaṁ mā ahosī'ti.

Vedanā anattā, vedanā ca hidaṁ bhikkhave attā abhavissa, nayidaṁ vedanā ābādhāya
saṁvatteyya, labbhetha ca vedanāya, evaṁ me vedanā hotu, evaṁ me vedanā mā
ahosī'ti. Yasmā ca kho bhikkhave vedanā anattā, tasmā vedanā ābādhāya
saṁvattati, na ca labbhati vedanāya, evaṁ me vedanā hotu, evaṁ me vedanā mā
ahosī'ti.

\clearpage

\englishText
\markboth{\englishTitle}{\rightmark}

‘Perception is not-self. If, bhikkhus, perception were self, perception
would not lead to affliction, and one might be able to say in regard to
perception, “Let my perception be thus, let my perception not be thus.”
But since, bhikkhus, perception is not-self, perception therefore leads
to affliction, and one is not able to say in regard to perception, “Let
my perception be thus, let my perception not be thus.”

‘Mental formations are not-self. If, bhikkhus, mental formations were
self, mental formations would not lead to affliction, and one might be
able to say in regard to mental formations, “Let my mental formations be
thus, let my mental formations not be thus.” But since, bhikkhus, mental
formations are not-self, mental formations therefore lead to affliction,
and one is not able to say in regard to mental formations, “Let my
mental formations be thus, let my mental formations not be thus.”

‘Consciousness is not-self. If, bhikkhus, consciousness were self,
consciousness would not lead to affliction, and one might be able to say
in regard to consciousness, “Let my consciousness be thus, let my
consciousness not be thus.” But since, bhikkhus, consciousness is
not-self, consciousness therefore leads to affliction, and one is not
able to say in regard to consciousness, “Let my consciousness be thus,
let my consciousness not be thus.”

‘What do you think about this, bhikkhus? Is form permanent or
impermanent?’

‘Impermanent, Venerable Sir.’

‘But is that which is impermanent painful or pleasurable?’

‘Painful, Venerable Sir.’

‘But is it fit to consider that which is impermanent, painful, of a
nature to change, as “This is mine, I am this, this is my self”?’

‘It is not, Venerable Sir.’

\clearpage

\paliText
\markboth{\paliTitle}{\rightmark}

Saññā anattā, saññā ca hidaṁ bhikkhave attā abhavissa, nayidaṁ saññā ābādhāya
saṁvatteyya, labbhetha ca saññāya, evaṁ me saññā hotu, evaṁ me saññā mā
ahosī'ti. Yasmā ca kho bhikkhave saññā anattā, tasmā, saññā ābādhāya saṁvattati,
na ca labbhati saññāya, evaṁ me saññā hotu, evaṁ me saññā mā ahosī'ti.

Saṅkhārā anattā, saṅkhārā ca hidaṁ bhikkhave attā abhavissaṁsu, nayidaṁ saṅkhārā
ābādhāya saṁvatteyyuṁ, labbhetha ca saṅkhāresu, evaṁ me saṅkhārā hontu, evaṁ me
saṅkhārā mā ahesun'ti. Yasmā ca kho bhikkhave saṅkhārā anattā, tasmā saṅkhārā
ābādhāya saṁvattanti, na ca labbhati saṅkhāresu, evaṁ me saṅkhārā hontu, evaṁ me
saṅkhārā mā ahesun'ti.

Viññāṇaṁ anattā, viññāṇañca hidaṁ bhikkhave attā abhavissa, nayidaṁ viññānam
ābādhāya saṁvatteyya, labbhetha ca viññāne evaṁ me viññāṇaṁ hotu, evaṁ me
viññāṇaṁ mā ahosī'ti. Yasmā ca kho bhikkhave viññāṇaṁ anattā, tasmā viññāṇaṁ
ābādhāya saṁvattati, na ca labbhati viññāne, evaṁ me viññāṇaṁ hotu, evaṁ me
viññāṇaṁ mā ahosī'ti.

[Taṁ kiṁ maññatha bhikkhave,] rūpam niccaṁ vā aniccaṁ vā'ti.

Aniccaṁ bhante.

Yam panāniccaṁ, dukkhaṁ vā taṁ sukhaṁ vā'ti.

Dukkhaṁ bhante.

Yam panāniccaṁ dukkhaṁ viparināma-dhammaṁ, kallaṁ nu taṁ samanupassituṁ,
etaṁ mama, esoham'asmi, eso me attā'ti.

No hetaṁ bhante.

\clearpage

\englishText
\markboth{\englishTitle}{\rightmark}

‘What do you think about this, bhikkhus? Is feeling permanent or
impermanent?’

‘Impermanent, Venerable Sir.’

‘But is that which is impermanent painful or pleasurable?’

‘Painful, Venerable Sir.’

‘But is it fit to consider that which is impermanent, painful, of a
nature to change, as “This is mine, I am this, this is my self”?’

‘It is not, Venerable Sir.’

‘What do you think about this, bhikkhus? Is perception permanent or
impermanent?’

‘Impermanent, Venerable Sir.’

‘But is that which is impermanent painful or pleasurable?’

‘Painful, Venerable Sir.’

‘But is it fit to consider that which is impermanent, painful, of a
nature to change, as “This is mine, I am this, this is my self”?’

‘It is not, Venerable Sir.’

‘What do you think about this, bhikkhus? Are mental formations
permanent or impermanent?’

‘Impermanent, Venerable Sir.’

‘But is that which is impermanent painful or pleasurable?’

‘Painful, Venerable Sir.’

‘But is it fit to consider that which is impermanent, painful, of a
nature to change, as “This is mine, I am this, this is my self”?’

‘It is not, Venerable Sir.’

\clearpage

\paliText
\markboth{\paliTitle}{\rightmark}

Taṁ kiṁ maññatha bhikkhave, vedanā niccā vā aniccā vā'ti.

Aniccā bhante.

Yam panāniccaṁ, dukkhaṁ vā taṁ sukhaṁ vā'ti.

Dukkhaṁ bhante.

Yam panāniccaṁ dukkhaṁ viparināma-dhammaṁ, kallaṁ nu taṁ samanupassituṁ,
etaṁ mama, esoham'asmi, eso me attā'ti.

No hetaṁ bhante.

Taṁ kiṁ maññatha bhikkhave, saññā niccā vā aniccā vā'ti.

Aniccā bhante.

Yam panāniccaṁ, dukkhaṁ vā taṁ sukhaṁ vā'ti.

Dukkhaṁ bhante.

Yam panāniccaṁ dukkhaṁ viparināma-dhammaṁ, kallaṁ nu taṁ samanupassituṁ,
etaṁ mama, esoham'asmi, eso me attā'ti.

No hetaṁ bhante.

Taṁ kiṁ maññatha bhikkhave, saṅkhārā niccā vā aniccā vā'ti.

Aniccā bhante.

Yam panāniccaṁ, dukkhaṁ vā taṁ sukhaṁ vā'ti.

Dukkhaṁ bhante.

Yam panāniccaṁ dukkhaṁ viparināma-dhammaṁ, kallaṁ nu taṁ samanupassituṁ,
etaṁ mama, esoham'asmi, eso me attā'ti.

No hetaṁ bhante.

\clearpage

\englishText
\markboth{\englishTitle}{\rightmark}

‘What do you think about this, bhikkhus? Is consciousness permanent or
impermanent?’

‘Impermanent, Venerable Sir.’

‘But is that which is impermanent painful or pleasurable?’

‘Painful, Venerable Sir.’

‘But is it fit to consider that which is impermanent, painful, of a
nature to change, as “This is mine, I am this, this is my self”?’

‘It is not, Venerable Sir.’

‘Wherefore, bhikkhus, whatever form there is, past, future, present,
internal or external, gross or subtle, inferior or superior, whether it
is far or near, all form should, by means of right wisdom, be seen as it
really is, thus: “This is not mine, I am not this, this is not my self.”

‘Whatever feeling there is, past, future, present, internal or
external, gross or subtle, inferior or superior, whether it is far or
near, all feeling should, by means of right wisdom, be seen as it really
is, thus: “This is not mine, I am not this, this is not my self.”

‘Whatever perception there is, past, future, present, internal or
external, gross or subtle, inferior or superior, whether it is far or
near, all perception should, by means of right wisdom, be seen as it really
is, thus: “This is not mine, I am not this, this is not my self.”

‘Whatever mental formations there are, past, future, present, internal
or external, gross or subtle, inferior or superior, whether they are far
or near, all mental formations should, by means of right wisdom, be seen
as they really are, thus: “This is not mine, I am not this, this is not
my self.”

\clearpage

\paliText
\markboth{\paliTitle}{\rightmark}

Taṁ kiṁ maññatha bhikkhave, viññāṇaṁ niccaṁ vā aniccaṁ vā'ti.

Aniccaṁ bhante.

Yam panāniccaṁ, dukkhaṁ vā taṁ sukhaṁ vā'ti.

Dukkhaṁ bhante.

Yam panāniccaṁ dukkhaṁ viparināma-dhammaṁ, kallaṁ nu taṁ samanupassituṁ
etaṁ mama, esoham'asmi, eso me attā'ti.

No hetaṁ bhante.

[Tasmā tiha bhikkhave] yaṁ kiñci rūpaṁ atītānāgata-paccuppannaṁ ajjhattaṁ
vā bahiddhā vā oḷārikaṁ vā sukhumaṁ vā hīnaṁ vā paṇītaṁ vā yandūre
santike vā, sabbaṁ rūpaṁ netaṁ mama, nesoham'asmi, na me so attā'ti,
evametaṁ yathābhūtaṁ sammappaññāya daṭṭhabbaṁ.

Yā kāci vedanā atītānāgata-paccuppannā ajjhattā vā bahiddhā vā oḷārikā
vā sukhumā vā hīnā vā paṇītā vā yā dūre santike vā, sabbā vedanā netaṁ
mama, nesoham'asmi, na me so attā'ti, evametaṁ yathābhūtaṁ sammappaññāya
daṭṭhabbaṁ.

Yā kāci saññā atītānāgata-paccuppannā ajjhattā vā bahiddhā vā oḷārikā vā
sukhumā vā hīnā vā paṇītā vā yā dūre santike vā, sabbā saññā netaṁ mama,
nesoham'asmi, na me so attā'ti, evametaṁ yathābhūtaṁ sammappaññāya
daṭṭhabbaṁ.

Ye keci saṅkhārā atītānāgata-paccuppannā ajjhattā vā bahiddhā vā oḷārikā
vā sukhumā vā hīnā vā paṇītā vā ye dūre santike vā, sabbe saṅkhārā netaṁ
mama, nesoham'asmi, na me so attā'ti, evametaṁ yathābhūtaṁ sammappaññāya
daṭṭhabbaṁ.

\clearpage

\englishText
\markboth{\englishTitle}{\rightmark}

‘Whatever consciousness there is, past, future, present, internal or
external, gross or subtle, inferior or superior, whether far or near,
all consciousness should, by means of right wisdom, be seen as it really
is, thus: “This is not mine, I am not this, this is not my self.”

‘Seeing in this way, bhikkhus, the wise noble disciple becomes
disenchanted with form, becomes disenchanted with feeling, becomes
disenchanted with perception, becomes disenchanted with mental
formations, becomes disenchanted with consciousness. Becoming
disenchanted, their passions fade away; with the fading of passion the
heart is liberated; with liberation there comes the knowledge: “It is
liberated,” and they know: “Destroyed is birth, the Holy Life has been
lived out, done is what had to be done, there is no more coming into any
state of being.”\thinspace ’

Thus spoke the Blessed One. Delighted, the group of five bhikkhus
rejoiced in what the Blessed One had said. Moreover, while this discourse was
being delivered, the minds of the five bhikkhus were freed from the
defilements, through clinging no more.

Thus ends the discourse on The Characteristic of Not-self.

\clearpage

\paliText
\markboth{\paliTitle}{\rightmark}

Yaṁ kiñci viññāṇaṁ atītānāgata-paccuppannaṁ ajjhattaṁ vā bahiddhā vā
oḷārikaṁ vā sukhumaṁ vā hīnaṁ vā paṇītaṁ vā yandūre santike vā, sabbaṁ
viññāṇaṁ netaṁ mama, nesoham'asmi, na me so attā'ti, evametaṁ yathābhūtaṁ
sammappaññāya daṭṭhabbaṁ.

[Evaṁ passaṁ bhikkhave] sutvā ariyasāvako rūpasmim pi nibbindati, vedanāya
pi nibbindati, saññāya pi nibbindati, saṅkhāresu pi nibbindati,
viññāṇasmim pi nibbindati, nibbindaṁ virajjati, virāgā vimuccati,
vimuttasmiṁ vimuttam iti ñāṇaṁ hoti, khīṇā jāti, vusitaṁ brahmacariyaṁ,
kataṁ karaṇīyaṁ, nāparaṁ itthattāyā'ti pajānātī'ti.

[Idam-avoca bhagavā.] Attamanā pañcavaggiyā bhikkhū bhagavato bhāsitaṁ
abhinanduṁ. Imasmiñca pana veyyākaraṇasmiṁ bhaññamāne pañcavaggiyānaṁ
bhikkhūnaṁ anupādāya āsavehi cittāni vimucciṁsū'ti.

Anattalakkhaṇa-suttaṁ niṭṭhitaṁ.

\chapterTocDelegatePageNumber
\chapter{The Fire Sermon}

\setTocDelegatedPageNumber
\englishText
\renewcommand{\englishTitle}{The Fire Sermon}

\begin{leader}
\soloinstr{Solo introduction}

With his skill in training the trainable, the All-transcendent Buddha,
lucid speaker, teacher of the highest knowledge,

He who expounds to the people the Dhamma and Vinaya that is fitting and
worthy, teaching with this wonderful parable about fire, meditators of
the highest skill;

He has liberated those who listen with the liberation that is utterly
complete, through true investigation, with wisdom\\ and attention.

Let us now recite this Sutta which describes the characteristics\\ of dukkha.

\end{leader}

Thus have I heard.

At one time the Blessed One was staying near Gayā at Gayā Head together
with a thousand bhikkhus. There the Blessed One addressed the bhikkhus
thus:

‘Bhikkhus, everything is burning. And what, bhikkhus, is everything
that is burning?

‘The eye, bhikkhus, is burning, forms are burning, eye consciousness is
burning, eye contact is burning, the feeling that arises from eye
contact, whether it is pleasant, painful, or neutral, that too is
burning. With what is it burning? I declare that it is burning with the
fires of passion, hatred, and delusion; it is burning with birth,
ageing, and death, with sorrow, lamentation, pain, grief, and despair.

\enlargethispage{2\baselineskip}

‘The ear is burning, sounds are burning, ear consciousness is burning,
ear contact is burning, the feeling that arises from ear contact,
whether it is pleasant, painful, or neutral, that too is burning. With
what is it burning? I declare that it is burning with the fires of
passion, hatred, and delusion; it is burning with birth, ageing, and
death, with sorrow, lamentation, pain, grief, and despair.

\chapterTocSubIndentTrue
\chapter{Āditta-pariyāya Sutta}

\paliText
\renewcommand{\paliTitle}{Āditta-pariyāya Sutta}

\begin{leader}
\soloinstr{Solo introduction}

\begin{solotwochants}
Veneyyadamanopāye  & sabbaso pāramiṁ gato\\
Amoghavacano buddho & abhiññāyānusāsako\\
Ciṇṇānurūpato cāpi & dhammena vinayaṁ pajaṁ\\
Ciṇṇāggipāricariyānaṁ & sambojjhārahayoginaṁ\\
Yamādittapariyāyaṁ & desayanto manoharaṁ\\
Te sotāro vimocesi & asekkhāya vimuttiyā\\
Tathevopaparikkhāya & viññūṇaṁ sotumicchataṁ\\
Dukkhatālakkhaṇopāyaṁ & taṁ suttantaṁ bhaṇāma se\\
\end{solotwochants}
\end{leader}

[Evaṁ me sutaṁ]

Ekaṁ samayaṁ bhagavā gayāyaṁ viharati gayāsīse saddhiṁ bhikkhu-sahassena.
Tatra kho bhagavā bhikkhū āmantesi:

Sabbaṁ bhikkhave ādittaṁ. Kiñca bhikkhave sabbaṁ ādittaṁ.

Cakkhuṁ bhikkhave ādittaṁ, rūpā ādittā, cakkhuviññāṇaṁ ādittaṁ,
cakkhusamphasso āditto, yampidaṁ cakkhusamphassapaccayā uppajjati
vedayitaṁ sukhaṁ vā dukkhaṁ vā adukkhamasukhaṁ vā tam pi ādittaṁ. Kena
ādittaṁ. Ādittaṁ rāgagginā dosagginā mohagginā, ādittaṁ jātiyā
jarāmaraṇena sokehi paridevehi dukkhehi domanassehi upāyāsehi ādittan'ti
vadāmi.

Sotaṁ ādittaṁ, saddā ādittā, sotaviññāṇaṁ ādittaṁ, sotasamphasso āditto,
yampidaṁ sotasamphassapaccayā uppajjati vedayitaṁ sukhaṁ vā dukkhaṁ vā
adukkhamasukhaṁ vā tam pi ādittaṁ. Kena ādittaṁ. Ādittaṁ rāgagginā
dosagginā mohagginā, ādittaṁ jātiyā jarāmaraṇena sokehi paridevehi
dukkhehi domanassehi upāyāsehi ādittan'ti vadāmi.

\clearpage

\englishText
\markboth{\englishTitle}{\rightmark}

‘The nose is burning, odours are burning, nose consciousness is burning,
nose contact is burning, the feeling that arises from nose contact,
whether it is pleasant, painful, or neutral, that too is burning. With
what is it burning? I declare that it is burning with the fires of
passion, hatred, and delusion; it is burning with birth, ageing, and
death, with sorrow, lamentation, pain, grief, and despair.

‘The tongue is burning, tastes are burning, tongue consciousness is
burning, tongue contact is burning, the feeling that arises from tongue
contact, whether it is pleasant, painful, or neutral, that too is
burning. With what is it burning? I declare that it is burning with the
fires of passion, hatred, and delusion; it is burning with birth,
ageing, and death, with sorrow, lamentation, pain, grief, and despair.

‘The body is burning, tangible objects are burning, body consciousness
is burning, body contact is burning, the feeling that arises from body
contact, whether it is pleasant, painful, or neutral, that too is
burning. With what is it burning? I declare that it is burning with the
fires of passion, hatred, and delusion; it is burning with birth,
ageing, and death, with sorrow, lamentation, pain, grief, and despair.

‘The mind is burning, mental states are burning, mind consciousness is
burning, mind contact is burning, the feeling that arises through mind
contact, whether it is pleasant, painful, or neutral, that too is
burning. With what is it burning? I declare that it is burning with the
fires of passion, hatred, and delusion; it is burning with birth,
ageing, and death, with sorrow, lamentation, pain, grief, and despair.

‘Seeing thus, bhikkhus, the wise noble disciple becomes disenchanted
with the eye, disenchanted with forms, disenchanted with eye
consciousness, disenchanted with eye contact, and the feeling that
arises from eye contact --- whether it is pleasant, painful, or
neutral --- that too they become disenchanted with.

\clearpage

\paliText
\markboth{\paliTitle}{\rightmark}

Ghānaṁ ādittaṁ, gandhā ādittā, ghānaviññāṇaṁ ādittaṁ, ghānasamphasso
āditto, yampidaṁ ghānasamphassapaccayā uppajjati vedayitaṁ sukhaṁ vā
dukkhaṁ vā adukkhamasukhaṁ vā tam pi ādittaṁ. Kena ādittaṁ. Ādittaṁ
rāgagginā dosagginā mohagginā, ādittaṁ jātiyā jarāmaraṇena sokehi
paridevehi dukkhehi domanassehi upāyāsehi ādittan'ti vadāmi.

Jivhā ādittā, rasā ādittā, jivhāviññāṇam ādittaṁ, jivhāsamphasso āditto,
yampidaṁ jivhāsamphassapaccayā uppajjati vedayitaṁ sukhaṁ vā dukkhaṁ vā
adukkhamasukhaṁ vā tam pi ādittaṁ. Kena ādittaṁ. Ādittaṁ rāgagginā
dosagginā mohagginā, ādittaṁ jātiyā jarāmaraṇena sokehi paridevehi
dukkhehi domanassehi upāyāsehi ādittan'ti vadāmi.

Kāyo āditto, phoṭṭhabbā ādittā, kāyaviññāṇaṁ ādittaṁ, kāyasamphasso
āditto, yampidaṁ kāyasamphassapaccayā uppajjati vedayitaṁ sukhaṁ vā
dukkhaṁ vā adukkhamasukhaṁ vā tam pi ādittaṁ. Kena ādittaṁ. Ādittaṁ
rāgagginā dosagginā mohagginā, ādittaṁ jātiyā jarāmaraṇena sokehi
paridevehi dukkhehi domanassehi upāyāsehi ādittan'ti vadāmi.

Mano āditto, dhammā ādittā, manoviññāṇaṁ ādittaṁ, manosamphasso āditto,
yampidaṁ manosamphassapaccayā uppajjati vedayitaṁ sukhaṁ vā dukkhaṁ vā
adukkhamasukhaṁ vā tam pi ādittaṁ. Kena ādittaṁ. Ādittaṁ rāgagginā
dosagginā mohagginā, ādittaṁ jātiyā jarāmaraṇena sokehi paridevehi
dukkhehi domanassehi upāyāsehi ādittan'ti vadāmi.

\enlargethispage{2\baselineskip}

[Evaṁ passaṁ bhikkhave] sutvā ariyasāvako cakkhusmiṁ pi nibbindati,
rūpesu pi nibbindati, cakkhuviññāṇe pi nibbindati, cakkhusamphassepi
nibbindati, yampidaṁ cakkhusamphassapaccayā uppajjati vedayitaṁ sukhaṁ
vā dukkhaṁ vā adukkhamasukhaṁ vā tasmiṁ pi nibbindati.

\clearpage

\englishText
\markboth{\englishTitle}{\rightmark}

‘They become disenchanted with the ear, disenchanted with sounds,
disenchanted with ear consciousness, disenchanted with ear contact, and
the feeling that arises from ear contact --- whether it is pleasant,
painful, or neutral --- that too they become disenchanted with.

‘They become disenchanted with the nose, disenchanted with odours,
disenchanted with nose consciousness, disenchanted with nose contact,
and the feeling that arises from nose contact --- whether it is pleasant,
painful, or neutral --- that too they become disenchanted with.

‘They become disenchanted with the tongue, disenchanted with tastes,
disenchanted with tongue consciousness, disenchanted with tongue
contact, and the feeling that arises from tongue contact --- whether it is
pleasant, painful, or neutral --- that too they become disenchanted with.

‘They become disenchanted with the body, disenchanted with tangible
objects, disenchanted with body consciousness, disenchanted with body
contact, and the feeling that arises from body contact --- whether it is
pleasant, painful, or neutral --- that too they become disenchanted with.

‘They become disenchanted with the mind, disenchanted with mental
states, disenchanted with mind consciousness, disenchanted with mind
contact, and the feeling that arises from mind contact --- whether it is
pleasant, painful, or neutral --- that too they become disenchanted with.

‘Becoming disenchanted, their passions fade away; with the fading of
passion the heart is liberated; with liberation there comes the
knowledge: “It is liberated,” and they know: “Destroyed is birth, the
Holy Life has been lived out, done is what had to be done, there is no
more coming into any state of being.”\thinspace ’

\enlargethispage{\baselineskip}

Thus spoke the Blessed One; delighted, the bhikkhus rejoiced in what the
Blessed One had said. Moreover, while this discourse was being uttered, the
minds of those thousand bhikkhus were freed from the defilements,
without any further attachment.

Thus ends The Fire Sermon.

\clearpage

\paliText
\markboth{\paliTitle}{\rightmark}

Sotasmiṁ pi nibbindati, saddesu pi nibbindati, sotaviññāṇe pi
nibbindati, sotasamphassepi nibbindati, yampidaṁ sotasamphassapaccayā
uppajjati vedayitaṁ sukhaṁ vā dukkhaṁ vā adukkhamasukhaṁ vā tasmiṁ pi
nibbindati.

Ghānasmiṁ pi nibbindati, gandhesu pi nibbindati, ghānaviññāṇe pi
nibbindati, ghānasamphassepi nibbindati, yampidaṁ ghānasamphassapaccayā
uppajjati vedayitaṁ sukhaṁ vā dukkhaṁ vā adukkhamasukhaṁ vā tasmiṁ pi
nibbindati.

Jivhāya pi nibbindati, rasesu pi nibbindati, jivhāviññāṇe pi nibbindati,
jivhāsamphassepi nibbindati, yampidaṁ jivhāsamphassapaccayā uppajjati
vedayitaṁ sukhaṁ vā dukkhaṁ vā adukkhamasukhaṁ vā tasmiṁ pi nibbindati.

Kāyasmiṁ pi nibbindati, phoṭṭhabbesu pi nibbindati, kāyaviññāṇe pi
nibbindati, kāyasamphassepi nibbindati, yampidaṁ kāyasamphassapaccayā
uppajjati vedayitaṁ sukhaṁ vā dukkhaṁ vā adukkhamasukhaṁ vā tasmiṁ pi
nibbindati.

Manasmiṁ pi nibbindati, dhammesu pi nibbindati, manoviññāṇe pi
nibbindati, manosamphassepi nibbindati, yampidaṁ manosamphassapaccayā
uppajjati vedayitaṁ sukhaṁ vā dukkhaṁ vā adukkhamasukhaṁ vā tasmiṁ pi
nibbindati.

Nibbindaṁ virajjati, virāgā vimuccati, vimuttasmiṁ, vimuttam iti ñāṇaṁ
hoti, khīṇā jāti, vusitaṁ brahmacariyaṁ, kataṁ karaṇīyaṁ, nāparaṁ
itthattāyā'ti pajānātī'ti.

\enlargethispage{\baselineskip}

[Idam-avoca bhagavā.] Attamanā te bhikkhū bhagavato bhāsitaṁ abhinanduṁ.
Imasmiñca pana veyyākaraṇasmiṁ bhaññamāne tassa bhikkhu-sahassassa
anupādāya āsavehi cittāni vimucciṁsū'ti.

Ādittapariyāya-suttaṁ niṭṭhitaṁ.

\resumeNormalText

% End of suttas.tex
