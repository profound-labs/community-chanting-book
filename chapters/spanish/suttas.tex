\chapterTocDelegatePageNumber
\chapter{La puesta en marcha de la rueda del Dhamma}

\setTocDelegatedPageNumber
\englishText
\renewcommand{\englishTitle}{La puesta en marcha de la rueda del Dhamma}

\begin{leader}
\soloinstr{Solo introduction}

Esta es la ensenanza del  Tathāgata on attaining to unexcelled,
perfect enlightenment.

Aqui es la primera vuelta de tuerca del Truth,
inestimable wherever it is expounded in the world.

Disclosed here are the two extremes, and the Middle Way, with the Four Noble
Truths and the purified knowledge and vision pointed out by the Lord of
Dhamma.

Let us chant together this Sutta proclaiming the supreme, independent
enlightenment that is widely renowned as ‘The~Turning of the Wheel of
the Dhamma.’

\end{leader}

%%Sutta Central SN56.11
%%https://suttacentral.net/sn56.11/es/baron?lang=es&reference=none&highlight=false

Esto es lo que he oído. En una ocasión el Sublime estaba residiendo cerca de Benares, en Isipatana, en el Parque de los Venados. Allí el Sublime se dirigió al grupo de los cinco monjes.

Estos dos extremos, oh monjes, no deberían ser seguidos por un renunciante. ¿Cuáles son éstos dos? Complacencia en los placeres sensuales, esto es bajo, vulgar, ordinario, innoble y sin beneficio; y adicción a la mortificación, esto es doloroso, innoble y sin beneficio. No siguiendo estos dos extremos el Tathagata ha penetrado el camino medio que genera la visión, que genera el entendimiento, que conduce a la paz, que conduce a la sabiduría, que conduce a la iluminación y que conduce al Nibbana.

¿Cuál, oh monjes, es el camino medio que el Tathagata ha penetrado que genera la visión, que genera el entendimiento, que conduce a la paz, que conduce a la sabiduría, que conduce a la iluminación y al Nibbana? 



\chapterTocSubIndentTrue
\chapter{Dhammacakkappavattana Sutta}

\paliText
\renewcommand{\paliTitle}{Dhammacakkappavattana Sutta}

\begin{leader}
\soloinstr{Solo introduction}

\begin{solotwochants}
Anuttaraṁ abhisambodhiṁ & sambujjhitvā tathāgato\\
Pathamaṁ yaṁ adesesi & dhammacakkaṁ anuttaraṁ\\
Sammadeva pavattento & loke appativattiyaṁ\\
Yatthākkhātā ubho antā & paṭipatti ca majjhimā\\
Catūsvāriyasaccesu & visuddhaṁ ñāṇadassanaṁ\\
Desitaṁ dhammarājena & sammāsambodhikittanaṁ\\
Nāmena vissutaṁ suttaṁ & dhammacakkappavattanaṁ\\
Veyyākaraṇapāthena & saṅgītantam bhaṇāma se\\
\end{solotwochants}
\end{leader}

[Evaṁ me sutaṁ]

Ekaṁ samayaṁ bhagavā bārāṇasiyaṁ viharati isipatane migadāye. Tatra kho
bhagavā pañcavaggiye bhikkhū āmantesi:

Dve'me, bhikkhave, antā pabbajitena na sevitabbā: yo cāyaṁ kāmesu
kāma-sukh'allikānuyogo; hīno, gammo, pothujjaniko, anariyo,
anattha-sañhito; yo cāyaṁ atta-kilamathānuyogo; dukkho, anariyo,
anattha-sañhito.

Ete te, bhikkhave, ubho ante anupagamma majjhimā paṭipadā tathāgatena
abhisambuddhā cakkhukaraṇī, ñāṇakaraṇī, upasamāya, abhiññāya,
sambodhāya, nibbānāya saṁvattati.

Katamā ca sā, bhikkhave, majjhimā paṭipadā tathāgatena abhisambuddhā
cakkhukaraṇī ñāṇakaraṇī, upasamāya, abhiññāya, sambodhāya, nibbānāya
saṁvattati.

\clearpage

\englishText
\markboth{\englishTitle}{\rightmark}

Simplemente este Óctuple Noble Sendero; es decir, Recto Entendimiento, Recto Pensamiento, Recto Lenguaje, Recta Acción, Recta Vida, Recto Esfuerzo, Recta Atención y Recta Concentración. Éste, oh monjes, es ese camino medio que el Tathagata ha penetrado que genera la visión, que genera el entendimiento, que conduce a la paz, que conduce a la sabiduría, que conduce a la iluminación y al Nibbana.

Ésta, oh monjes, es la Noble Verdad del Sufrimiento. El nacimiento es sufrimiento, la vejez es sufrimiento, la enfermedad es sufrimiento, la muerte es sufrimiento, asociarse con lo indeseable es sufrimiento, separarse de lo deseable es sufrimiento, no obtener lo deseado es sufrimiento. En breve, los de la adherencia son sufrimiento.

Ésta, oh monjes, es la Noble Verdad del Origen del Sufrimiento. Es este deseo que genera nueva existencia, que asociado con placer y pasión se deleita aquí y allí. Es decir, el deseo sensual, el deseo por la existencia y el deseo por la no existencia.

Ésta, oh monjes, es la Noble Verdad de la Cesación del Sufrimiento. Es la total extinción y cesación de ese mismo deseo, su abandono, su descarte, liberación, no dependencia.

Ésta, oh monjes, es la Noble Verdad del Sendero que conduce a la Cesación del Sufrimiento. Simplemente este Óctuple Noble Sendero; es decir, Recto Entendimiento, Recto Pensamiento, Recto Lenguaje, Recta Acción, Recta Vida, Recto Esfuerzo, Recta Atención y Recta Concentración.

Ésta es la Noble Verdad del Sufrimiento. Así, oh monjes, con relación a cosas desconocidas por mi anteriormente, surgió la visión, surgió el entendimiento, surgió la sabiduría, surgió la penetración y surgió la luz.

\clearpage

\paliText
\markboth{\paliTitle}{\rightmark}

Ayam-eva ariyo aṭṭhaṅgiko maggo seyyathīdaṁ:

Sammā-diṭṭhi, sammā-saṅkappo, sammā-vācā, sammā-kammanto, sammā-ājīvo,
sammā-vāyāmo, sammā-sati, sammā-samādhi.

Ayaṁ kho sā, bhikkhave, majjhimā paṭipadā tathāgatena abhisambuddhā
cakkhukaraṇī ñāṇakaraṇī, upasamāya, abhiññāya, sambodhāya, nibbānāya
saṁvattati.

Idaṁ kho pana, bhikkhave, dukkhaṁ ariya-saccaṁ:

Jātipi dukkhā, jarāpi dukkhā, maranampi dukkhaṁ,
soka-parideva-dukkha-domanass'upāyāsāpi dukkhā, appiyehi sampayogo
dukkho, piyehi vippayogo dukkho, yamp'icchaṁ na labhati tampi dukkhaṁ,
saṅkhittena pañcupādānakkhandhā dukkhā.

Idaṁ kho pana, bhikkhave, dukkha-samudayo ariya-saccaṁ:

Yā'yaṁ taṇhā ponobbhavikā nandi-rāga-sahagatā tatra-tatrābhinandinī
seyyathīdaṁ: kāma-taṇhā, bhava-taṇhā, vibhava-taṇhā.

Idaṁ kho pana, bhikkhave, dukkha-nirodho ariya-saccaṁ:

Yo tassā yeva taṇhāya asesa-virāga-nirodho, cāgo, paṭinissaggo, mutti,
anālayo.

Idaṁ kho pana, bhikkhave, dukkha-nirodha-gāminī paṭipadā ariya-saccaṁ:

Ayam-eva ariyo aṭṭhaṅgiko maggo seyyathīdam: sammā-diṭṭhi,
sammā-saṅkappo, sammā-vācā, sammā-kammanto, sammā-ājīvo, sammā-vāyāmo,
sammā-sati, sammā-samādhi.

\enlargethispage{\baselineskip}

[Idaṁ dukkhaṁ] ariya-saccan'ti me bhikkhave, pubbe ananussutesu dhammesu
cakkhuṁ udapādi, ñāṇaṁ udapādi, paññā udapādi, vijjā udapādi, āloko
udapādi.

\clearpage

\englishText
\markboth{\englishTitle}{\rightmark}

Ésta Noble Verdad del Sufrimiento debe ser completamente comprendida. Así, oh monjes, con relación a cosas desconocidas por mi anteriormente, surgió la visión, surgió el entendimiento, surgió la sabiduría, surgió la penetración y surgió la luz.

Ésta Noble Verdad del Sufrimiento ha sido completamente comprendida. Así, oh monjes, con relación a cosas desconocidas por mi anteriormente, surgió la visión, surgió el entendimiento, surgió la sabiduría, surgió la penetración y surgió la luz.

Ésta es la Noble Verdad del Origen del Sufrimiento. Así, oh monjes, con relación a cosas desconocidas por mi anteriormente, surgió la visión, surgió el entendimiento, surgió la sabiduría, surgió la penetración y surgió la luz.

Ésta Noble Verdad del Origen del Sufrimiento debe ser erradicada. Así, oh monjes, con relación a cosas desconocidas por mi anteriormente, surgió la visión, surgió el entendimiento, surgió la sabiduría, surgió la penetración y surgió la luz.

Ésta Noble Verdad del Origen del Sufrimiento ha sido erradicada. Así, oh monjes, con relación a cosas desconocidas por mi anteriormente, surgió la visión, surgió el entendimiento, surgió la sabiduría, surgió la penetración y surgió la luz.

Ésta es la Noble Verdad de la Cesación del Sufrimiento. Así, oh monjes, con relación a cosas desconocidas por mi anteriormente, surgió la visión, surgió el entendimiento, surgió la sabiduría, surgió la penetración y surgió la luz.

Ésta Noble Verdad de la Cesación del Sufrimiento debe ser realizada. Así, oh monjes, con relación a cosas desconocidas por mi anteriormente, surgió la visión, surgió el entendimiento, surgió la sabiduría, surgió la penetración y surgió la luz.

\enlargethispage*{2\baselineskip}

Ésta Noble Verdad de la Cesación del Sufrimiento ha sido realizada. Así, oh monjes, con relación a cosas desconocidas por mi anteriormente, surgió la visión, surgió el entendimiento, surgió la sabiduría, surgió la penetración y surgió la luz.

\clearpage

\paliText
\markboth{\paliTitle}{\rightmark}

Taṁ kho pan'idaṁ dukkhaṁ ariya-saccaṁ pariññeyyan'ti me bhikkhave, pubbe
ananussutesu dhammesu cakkhuṁ udapādi, ñāṇaṁ udapādi, paññā udapādi,
vijjā udapādi, āloko udapādi.

Taṁ kho pan'idaṁ dukkhaṁ ariya-saccaṁ pariññātan'ti me bhikkhave, pubbe
ananussutesu dhammesu cakkhuṁ udapādi, ñāṇaṁ udapādi, paññā udapādi,
vijjā udapādi, āloko udapādi.

Idaṁ dukkha-samudayo ariya-saccan'ti me bhikkhave, pubbe ananussutesu
dhammesu cakkhuṁ udapādi, ñāṇaṁ udapādi, paññā udapādi, vijjā udapādi,
āloko udapādi.

Taṁ kho pan'idaṁ dukkha-samudayo ariyasaccaṁ pahātabban'ti me bhikkhave,
pubbe ananussutesu dhammesu cakkhuṁ udapādi, ñāṇaṁ udapādi, paññā
udapādi, vijjā udapādi, āloko udapādi.

Taṁ kho pan'idaṁ dukkha-samudayo ariya-saccaṁ pahīnan'ti me bhikkhave, pubbe
ananussutesu dhammesu cakkhuṁ udapādi, ñāṇaṁ udapādi, paññā udapādi,
vijjā udapādi, āloko udapādi.

Idaṁ dukkha-nirodho ariya-saccan'ti me bhikkhave, pubbe ananussutesu
dhammesu cakkhuṁ udapādi, ñāṇaṁ udapādi, paññā udapādi, vijjā udapādi,
āloko udapādi.

Taṁ kho pan'idaṁ dukkha-nirodho ariya-saccaṁ sacchikātabban'ti me bhikkhave,
pubbe ananussutesu dhammesu cakkhuṁ udapādi, ñāṇaṁ udapādi, paññā
udapādi, vijjā, udapādi āloko udapādi.

Taṁ kho pan'idaṁ dukkha-nirodho ariya-saccaṁ sacchikatan'ti me bhikkhave,
pubbe ananussutesu dhammesu cakkhuṁ udapādi, ñāṇaṁ udapādi, paññā
udapādi, vijjā udapādi, āloko udapādi.

\clearpage

\englishText
\markboth{\englishTitle}{\rightmark}
Ésta es la Noble Verdad del Sendero que conduce a la Cesación del Sufrimiento. Así, oh monjes, con relación a cosas desconocidas por mi anteriormente, surgió la visión, surgió el entendimiento, surgió la sabiduría, surgió la penetración y surgió la luz.

Ésta Noble Verdad del Sendero que conduce a la Cesación del Sufrimiento debe ser desarrollada. Así, oh monjes, con relación a cosas desconocidas por mi anteriormente, surgió la visión, surgió el entendimiento, surgió la sabiduría, surgió la penetración y surgió la luz.

Ésta Noble Verdad del Sendero que conduce a la Cesación del Sufrimiento ha sido desarrollada. Así, oh monjes, con relación a cosas desconocidas por mi anteriormente, surgió la visión, surgió el entendimiento, surgió la sabiduría, surgió la penetración y surgió la luz.

Mientras, oh monjes, el entendimiento y la visión con respecto a estas Cuatro Nobles Verdades de acuerdo con la realidad bajo sus tres modos y doce aspectos no fue totalmente puro en mí, no admití al mundo con sus divinidades, Maras y Brahmas, a la humanidad con sus ascéticos, brahmanes y hombres, que había realizado correctamente por mí mismo la incomparable iluminación.

Cuando, oh monjes, el entendimiento y la visión con respecto a estas Cuatro Nobles Verdades de acuerdo con la realidad bajo sus tres modos y doce aspectos fue totalmente puro en mí, entonces admití al mundo con sus divinidades, Maras y Brahmas, a la humanidad con sus ascéticos, brahmanes y hombres, que había realizado correctamente por mí mismo la incomparable iluminación. Y surgió en mí el entendimiento y la visión: ‘Inconmovible es mi liberación. Éste es el último nacimiento. Ahora no hay nueva existencia’.

Esto dijo el Sublime. Los cinco monjes se regocijaron de las palabras del Sublime.


\clearpage

\paliText
\markboth{\paliTitle}{\rightmark}

Idaṁ dukkha-nirodha-gāminī paṭipadā ariya-saccan'ti me bhikkhave, pubbe
ananussutesu dhammesu cakkhuṁ udapādi, ñāṇaṁ udapādi, paññā udapādi,
vijjā udapādi, āloko udapādi.

Taṁ kho pan'idaṁ dukkha-nirodha-gāminī paṭipadā ariya-saccaṁ bhāvetabban'ti
me bhikkhave, pubbe ananussutesu dhammesu cakkhuṁ udapādi, ñāṇaṁ
udapādi, paññā udapādi, vijjā udapādi, āloko udapādi.

Taṁ kho pan'idaṁ dukkha-nirodha-gāminī paṭipadā ariya-saccaṁ bhāvitan'ti me
bhikkhave, pubbe ananussutesu dhammesu cakkhuṁ udapādi, ñāṇaṁ udapādi,
paññā udapādi, vijjā udapādi, āloko udapādi.

[Yāva kīvañca me bhikkhave,] imesu catūsu ariya-saccesu evan-ti-parivaṭṭaṁ
dvādas'ākāraṁ yathā-bhūtaṁ ñāṇa-dassanaṁ na suvisuddhaṁ ahosi, n'eva tāv'āhaṁ
bhikkhave, sadevake loke samārake sabrahmake sassamaṇa-brāhmaṇiyā pajāya
sadeva-manussāya anuttaraṁ sammā-sambodhiṁ abhisambuddho paccaññāsiṁ.

Yato ca kho me bhikkhave, imesu catūsu ariya-saccesu evan-ti-parivaṭṭaṁ
dvādas'ākāraṁ yathā-bhūtaṁ ñāṇa-dassanaṁ suvisuddham ahosi, ath'āham
bhikkhave, sadevake loke samārake sabrahmake sassamaṇa-brāhmaṇiyā pajāya
sadeva-manussāya anuttaraṁ sammā-sambodhiṁ abhisambuddho paccaññāsiṁ.

Ñāṇañca pana me dassanaṁ udapādi, akuppā me vimutti ayam-antimā jāti,
natthi dāni punabbhavo'ti.

Idam-avoca bhagavā. Attamanā pañcavaggiyā bhikkhū bhagavato bhāsitaṁ
abhinanduṁ.

\clearpage

\englishText
\markboth{\englishTitle}{\rightmark}

Cuando esta exposición se estaba impartiendo surgió en el Venerable Kondañña la pura e inmaculada visión del Dhamma: ‘Todo aquello que está sujeto a un surgir está sujeto a un cesar’.

Cuando el Sublime puso en movimiento la rueda de la doctrina, la divinidades terrestres hicieron oír este sonido: ‘Esta excelente rueda de la doctrina ha sido puesta en movimiento por el Sublime cerca de Benares, en Isipatana, en el Parque de los Venados, y no puede ser detenida por ningún ascético, brahmán, divinidad, Mara, Brahma, o ningún ser en el universo’.

Habiendo escuchado esto de las divinidades terrestres, las divinidades de Càtumahàràjika hicieron oír este sonido\ldots

Habiendo escuchado esto de las divinidades de Càtumahàràjika, las divinidades de Tàvatiçsa hicieron oír este sonido\ldots

Habiendo escuchado esto de las divinidades de Tàvatiçsa, las divinidades de Yàma hicieron oír este sonido\ldots  

Habiendo escuchado esto de las divinidades de Yàma, las divinidades de Tusita hicieron oír este sonido\ldots

Habiendo escuchado esto de las divinidades de Tusita, las divinidades de Nimmànaratì hicieron oír este sonido\ldots

Habiendo escuchado esto de las divinidades de Nimmànarati, las divinidades de Paranimmitavasavattì hicieron oír este sonido\ldots

Habiendo escuchado esto de las divinidades de Paranimmitavasavattì, las divinidades del mundo de los Brahmas hicieron oír este sonido: ‘Esta excelente rueda de la doctrina ha sido puesta en movimiento por el Sublime cerca de Benares, en Isipatana, en el Parque de los Venados, y no puede ser detenida por ningún ascético, brahmán, divinidad, Mara, Brahma, o ningún ser en el universo’.



\clearpage

\paliText
\markboth{\paliTitle}{\rightmark}

Imasmiñca pana veyyākaraṇasmiṁ bhaññamāne āyasmato koṇḍaññassa virajaṁ
vītamalaṁ dhammacakkhuṁ udapādi: yaṁ kiñci samudaya-dhammaṁ sabban-taṁ
nirodha-dhamman'ti.

[Pavattite ca bhagavatā] dhammacakke bhummā devā saddamanussāvesuṁ:

Etaṁ bhagavatā bārāṇasiyaṁ isipatane migadāye anuttaraṁ dhammacakkaṁ
pavattitaṁ appaṭivattiyaṁ samaṇena vā brāhmaṇena vā devena vā mārena vā
brahmunā vā kenaci vā lokasmin'ti.

Bhummānaṁ devānaṁ saddaṁ sutvā, cātummahārājikā devā
saddamanussāvesuṁ\ldots

Cātummahārājikānaṁ devānaṁ saddaṁ sutvā, tāvatiṁsā devā
saddamanussāvesuṁ\ldots

Tāvatiṁsānaṁ devānaṁ saddaṁ sutvā, yāmā devā saddamanussāvesuṁ\ldots

Yāmānaṁ devānaṁ saddaṁ sutvā, tusitā devā saddamanussāvesuṁ\ldots

Tusitānaṁ devānaṁ saddaṁ sutvā, nimmānaratī devā saddamanussāvesum\ldots

Nimmānaratīnaṁ devānaṁ saddaṁ sutvā, paranimmitavasavattī devā
saddamanussāvesuṁ\ldots

Paranimmitavasavattīnaṁ devānaṁ saddaṁ sutvā, brahmakāyikā devā
saddamanussāvesuṁ:

Etaṁ bhagavatā bārāṇasiyaṁ isipatane migadāye anuttaraṁ dhammacakkaṁ
pavattitaṁ appaṭivattiyaṁ samaṇena vā brāhmaṇena vā devena vā mārena vā
brahmunā vā kenaci vā lokasmin'ti.

\clearpage

\englishText
\markboth{\englishTitle}{\rightmark}

Y en ese segundo, en ese momento, en ese instante, esa exclamación se extendió hasta el mundo de los Brahmas. Y los diez mil universos se estremecieron, se sacudieron y temblaron violentamente. Una espléndida e ilimitada luminosidad, sobrepasando la refulgencia de las divinidades, se manifestó en el mundo.

Después el Sublime pronunció esta expresión de alegría: ‘Amigos, Kondañña realmente ha comprendido. Amigos, Kondañña realmente ha comprendido’. Y el Venerable Kondañña fue llamado Añña-Kondañña.

Y el Venerable Añña-Kondañña, habiendo visto el Dhamma, alcanzado el Dhamma, conocido el Dhamma, penetrado el Dhamma, trascendido la duda, sin incertidumbre, sereno y no dependiendo de otro en la enseñanza del Maestro, se dirigió al Sublime: ‘Venerable Señor, yo deseo recibir la ordenación en la presencia del Sublime, deseo recibir la alta ordenación’. ‘Venga monje,’ dijo el Sublime. ‘Bien expuesta está la Doctrina. Practique la vida noble para completamente poner fin al sufrimiento’. Y ésa simplemente fue la ordenación del Venerable.



Así termina el discurso de la puesta en marcha de la rueda del Dhamma.

\clearpage

\paliText
\markboth{\paliTitle}{\rightmark}

Iti'ha tena khaṇena, tena muhuttena, yāva brahmalokā saddo abbhuggacchi.
Ayañca dasa-sahassī lokadhātu saṅkampi sampakampi sampavedhi, appamāṇo ca
oḷāro obhāso loke pāturahosi atikkammeva devānaṁ devānubhāvaṁ.

Atha kho bhagavā udānaṁ udānesi:

Aññāsi vata bho koṇḍañño, aññāsi vata bho koṇḍañño'ti. Iti hidaṁ āyasmato
koṇḍaññassa aññā-koṇḍañño tveva nāmaṁ ahosī'ti.

Dhammacakkappavattana-suttaṁ niṭṭhitaṁ.

\chapterTocDelegatePageNumber
\chapter{Las Características de Anatta}

% Sutta Central SN22.59
%%https://suttacentral.net/sn22.59/es/baron?lang=es&reference=none&highlight=false
\setTocDelegatedPageNumber
\englishText
\renewcommand{\englishTitle}{Las Características de Anatta}

\begin{leader}
\soloinstr{Solo introduction}

All beings should take pains to understand the characteristic of
not-self, which provides matchless deliverance from self-view and
self-perception, as taught by the supreme Buddha.

This teaching is given so that those who meditate on experienceable
realities may arrive at perfect comprehension;

It is for the development of perfect understanding of these phenomena,
and for the investigation of all defiled mind-moments.

The consequence of this practice is total deliverance, so, desirous of
bringing this teaching forth with its great benefit, let us now recite
this Sutta.

\end{leader}

Esto he escuchado:

En una ocasión el Bienaventurado estaba morando en el Parque de los Venados de Isipatana cerca de Baranasi. Estando allí, el Bienaventurado se dirigió al grupo de los cinco : “Monjes”.—“Sí, venerable señor”, respondieron los monjes y el Bienaventurado dijo:

“Monjes, la forma no es un 'yo'. Porque, monjes, si la forma fuera un 'yo', no conduciría a la aflicción y sería posible conseguir eso de la forma: ‘que la forma sea de esa manera o que la forma sea de otra manera’. Pero como la forma no es un 'yo', conduce a la aflicción y no es posible conseguir eso de la forma: ‘que la forma sea de esa manera o que la forma sea de otra manera’.

“La sensación no es 'yo'. Porque, monjes, si la sensación fuera 'yo', no conduciría a la aflicción y sería posible conseguir eso de la sensación: ‘que la sensación sea de esa manera o que la sensación sea de otra manera’. Pero como la sensación no es un 'yo', conduce a la aflicción y no es posible conseguir eso de la sensación: ‘que la sensación sea de esa manera o que la sensación sea de otra manera’.

\chapterTocSubIndentTrue
\chapter{Anatta-lakkhaṇa Sutta}

\paliText
\renewcommand{\paliTitle}{Anatta-lakkhaṇa Sutta}

\begin{leader}
\soloinstr{Solo introduction}

{\setlength{\tabcolsep}{0.9em}
\begin{solotwochants}
Yantaṁ sattehi dukkhena & ñeyyaṁ anattalakkhaṇaṁ\\
Attavādattasaññāṇaṁ  & sammadeva vimocanaṁ\\
Sambuddho taṁ pakāsesi & diṭṭhasaccāna yoginaṁ\\
Uttariṁ paṭivedhāya & bhāvetuṁ ñāṇamuttamaṁ\\
Yantesaṁ diṭṭhadhammānam & ñāṇenupaparikkhataṁ\\
Sabbāsavehi cittāni & vimucciṁsu asesato\\
Tathā ñāṇānussārena & sāsanaṁ kātumicchataṁ\\
Sādhūnaṁ atthasiddhatthaṁ & taṁ suttantaṁ bhaṇāma se\\
\end{solotwochants}
}
\end{leader}

[Evaṁ me sutaṁ]

Ekaṁ samayaṁ bhagavā bārāṇasiyaṁ viharati isipatane migadāye. Tatra kho
bhagavā pañcavaggiye bhikkhū āmantesi:

Rūpaṁ bhikkhave anattā, rūpañca hidaṁ bhikkhave attā abhavissa, nayidaṁ rūpaṁ
ābādhāya saṁvatteyya, labbhetha ca rūpe, evaṁ me rūpaṁ hotu, evaṁ me rūpaṁ mā
ahosī'ti. Yasmā ca kho bhikkhave rūpaṁ anattā, tasmā rūpaṁ ābādhāya saṁvattati,
na ca labbhati rūpe, evaṁ me rūpaṁ hotu, evaṁ me rūpaṁ mā ahosī'ti.

Vedanā anattā, vedanā ca hidaṁ bhikkhave attā abhavissa, nayidaṁ vedanā ābādhāya
saṁvatteyya, labbhetha ca vedanāya, evaṁ me vedanā hotu, evaṁ me vedanā mā
ahosī'ti. Yasmā ca kho bhikkhave vedanā anattā, tasmā vedanā ābādhāya
saṁvattati, na ca labbhati vedanāya, evaṁ me vedanā hotu, evaṁ me vedanā mā
ahosī'ti.

\clearpage

\englishText
\markboth{\englishTitle}{\rightmark}

“La percepción no es 'yo'. Porque, monjes, si la percepción fuera un 'yo', no conduciría a la aflicción y sería posible conseguir eso de la percepción: ‘que la percepción sea de esa manera o que la percepción sea de otra manera’. Pero como la percepción no es un 'yo', conduce a la aflicción y no es posible conseguir eso de la percepción: ‘que la percepción sea de esa manera o que la percepción sea de otra manera’.

“Las formaciones mentales son 'yo'. Porque, monjes, si las formaciones mentales fueran un 'yo', no conducirían a la aflicción y sería posible conseguir eso de las formaciones mentales: ‘que las formaciones mentales sean de esa manera o que las formaciones mentales sean de otra manera’. Pero como las formaciones mentales no son un 'yo', conducen a la aflicción y no es posible conseguir eso de las formaciones mentales: ‘que las formaciones mentales sean de esa manera o que las formaciones mentales sean de otra manera’.

“La conciencia no es 'yo'. Porque, monjes, si la conciencia fuera un 'yo', no conduciría a la aflicción y sería posible conseguir eso de la conciencia: ‘que la conciencia sea de esa manera o que la conciencia sea de otra manera’. Pero como la conciencia no es un 'yo', conduce a la aflicción y no es posible conseguir eso de la conciencia: ‘que la conciencia sea de esa manera o que la conciencia sea de otra manera’.

“¿Qué opináis, monjes, es la forma permanente o transitoria?”
—“Transitoria, venerable señor”.
—“Y, lo que es transitorio, ¿es insatisfacción o felicidad?”
—“Insatisfacción, venerable señor”.
—“Y, lo que es transitorio, insatisfactorio y sujeto a cambio, ¿puede ser considerado de esa manera: ‘eso es mío, eso soy yo, eso es mi ser’?”
—“No, venerable señor”.



\clearpage

\paliText
\markboth{\paliTitle}{\rightmark}

Saññā anattā, saññā ca hidaṁ bhikkhave attā abhavissa, nayidaṁ saññā ābādhāya
saṁvatteyya, labbhetha ca saññāya, evaṁ me saññā hotu, evaṁ me saññā mā
ahosī'ti. Yasmā ca kho bhikkhave saññā anattā, tasmā, saññā ābādhāya saṁvattati,
na ca labbhati saññāya, evaṁ me saññā hotu, evaṁ me saññā mā ahosī'ti.

Saṅkhārā anattā, saṅkhārā ca hidaṁ bhikkhave attā abhavissaṁsu, nayidaṁ saṅkhārā
ābādhāya saṁvatteyyuṁ, labbhetha ca saṅkhāresu, evaṁ me saṅkhārā hontu, evaṁ me
saṅkhārā mā ahesun'ti. Yasmā ca kho bhikkhave saṅkhārā anattā, tasmā saṅkhārā
ābādhāya saṁvattanti, na ca labbhati saṅkhāresu, evaṁ me saṅkhārā hontu, evaṁ me
saṅkhārā mā ahesun'ti.

Viññāṇaṁ anattā, viññāṇañca hidaṁ bhikkhave attā abhavissa, nayidaṁ viññānam
ābādhāya saṁvatteyya, labbhetha ca viññāne evaṁ me viññāṇaṁ hotu, evaṁ me
viññāṇaṁ mā ahosī'ti. Yasmā ca kho bhikkhave viññāṇaṁ anattā, tasmā viññāṇaṁ
ābādhāya saṁvattati, na ca labbhati viññāne, evaṁ me viññāṇaṁ hotu, evaṁ me
viññāṇaṁ mā ahosī'ti.

[Taṁ kiṁ maññatha bhikkhave,] rūpam niccaṁ vā aniccaṁ vā'ti.

Aniccaṁ bhante.

Yam panāniccaṁ, dukkhaṁ vā taṁ sukhaṁ vā'ti.

Dukkhaṁ bhante.

Yam panāniccaṁ dukkhaṁ viparināma-dhammaṁ, kallaṁ nu taṁ samanupassituṁ,
etaṁ mama, esoham'asmi, eso me attā'ti.

No hetaṁ bhante.

\clearpage

\englishText
\markboth{\englishTitle}{\rightmark}
“¿Es la sensación permanente o transitoria?”—“Transitoria, venerable señor”.—“Y, lo que es transitorio, ¿es insatisfacción o felicidad?”—“Insatisfacción, venerable señor”.—“Y, lo que es transitorio, insatisfactorio y sujeto a cambio, ¿puede ser considerado de esa manera: ‘eso es mío, eso soy yo, eso es mi ser’?”—“No, venerable señor”.

“¿Es la percepción permanente o transitoria?”—“Transitoria, venerable señor”.—“Y, lo que es transitorio, ¿es insatisfacción o felicidad?”—“Insatisfacción, venerable señor”.—“Y, lo que es transitorio, insatisfactorio y sujeto a cambio, ¿puede ser considerado de esa manera: ‘eso es mío, eso soy yo, eso es mi ser’?”—“No, venerable señor”.

“¿Son las formaciones mentales permanentes o transitorias?”—“Transitorias, venerable señor”.—“Y, lo que es transitorio, ¿es insatisfacción o felicidad?”—“Insatisfacción, venerable señor”.—“Y, lo que es transitorio, insatisfactorio y sujeto a cambio, ¿puede ser considerado de esa manera: ‘eso es mío, eso soy yo, eso es mi ser’?”—“No, venerable señor”.



\clearpage

\paliText
\markboth{\paliTitle}{\rightmark}

Taṁ kiṁ maññatha bhikkhave, vedanā niccā vā aniccā vā'ti.

Aniccā bhante.

Yam panāniccaṁ, dukkhaṁ vā taṁ sukhaṁ vā'ti.

Dukkhaṁ bhante.

Yam panāniccaṁ dukkhaṁ viparināma-dhammaṁ, kallaṁ nu taṁ samanupassituṁ,
etaṁ mama, esoham'asmi, eso me attā'ti.

No hetaṁ bhante.

Taṁ kiṁ maññatha bhikkhave, saññā niccā vā aniccā vā'ti.

Aniccā bhante.

Yam panāniccaṁ, dukkhaṁ vā taṁ sukhaṁ vā'ti.

Dukkhaṁ bhante.

Yam panāniccaṁ dukkhaṁ viparināma-dhammaṁ, kallaṁ nu taṁ samanupassituṁ,
etaṁ mama, esoham'asmi, eso me attā'ti.

No hetaṁ bhante.

Taṁ kiṁ maññatha bhikkhave, saṅkhārā niccā vā aniccā vā'ti.

Aniccā bhante.

Yam panāniccaṁ, dukkhaṁ vā taṁ sukhaṁ vā'ti.

Dukkhaṁ bhante.

Yam panāniccaṁ dukkhaṁ viparināma-dhammaṁ, kallaṁ nu taṁ samanupassituṁ,
etaṁ mama, esoham'asmi, eso me attā'ti.

No hetaṁ bhante.

\clearpage

\englishText
\markboth{\englishTitle}{\rightmark}

“¿Es la conciencia permanente o transitoria?”—“Transitoria, venerable señor”.—“Y, lo que es transitorio, ¿es insatisfacción o felicidad?”—“Insatisfacción, venerable señor”.—“Y, lo que es transitorio, insatisfactorio y sujeto a cambio, ¿puede ser considerado de esa manera: ‘eso es mío, eso soy yo, eso es mi ser’?”—“No, venerable señor”.

“Por eso, monjes, cualquier clase de forma, sea del pasado, futuro o presente, interna o externa, vulgar o sutil, inferior o superior, lejana o cercana, toda forma debería ser vista tal como realmente es con la correcta sabiduría así: ‘eso no es mío, eso no soy yo, eso no es mi ser’.

“Cualquier clase de sensación, sea del pasado, futuro o presente, interna o externa, vulgar o sutil, inferior o superior, lejana o cercana, toda sensación debería ser vista tal como realmente es con la correcta sabiduría así: ‘eso no es mío, eso no soy yo, eso no es mi ser’.

“Cualquier clase de percepción, sea del pasado, futuro o presente, interna o externa, vulgar o sutil, inferior o superior, lejana o cercana, toda percepción debería ser vista tal como realmente es con la correcta sabiduría así: ‘eso no es mío, eso no soy yo, eso no es mi ser’.

“Cualquier clase de formaciones mentales, sea del pasado, futuro o presente, interna o externa, vulgar o sutil, inferior o superior, lejana o cercana, toda formación mental debería ser vista tal como realmente es con la correcta sabiduría así: ‘eso no es mío, eso no soy yo, eso no es mi ser’.



\clearpage

\paliText
\markboth{\paliTitle}{\rightmark}

Taṁ kiṁ maññatha bhikkhave, viññāṇaṁ niccaṁ vā aniccaṁ vā'ti.

Aniccaṁ bhante.

Yam panāniccaṁ, dukkhaṁ vā taṁ sukhaṁ vā'ti.

Dukkhaṁ bhante.

Yam panāniccaṁ dukkhaṁ viparināma-dhammaṁ, kallaṁ nu taṁ samanupassituṁ
etaṁ mama, esoham'asmi, eso me attā'ti.

No hetaṁ bhante.

[Tasmā tiha bhikkhave] yaṁ kiñci rūpaṁ atītānāgata-paccuppannaṁ ajjhattaṁ
vā bahiddhā vā oḷārikaṁ vā sukhumaṁ vā hīnaṁ vā paṇītaṁ vā yandūre
santike vā, sabbaṁ rūpaṁ netaṁ mama, nesoham'asmi, na me so attā'ti,
evametaṁ yathābhūtaṁ sammappaññāya daṭṭhabbaṁ.

Yā kāci vedanā atītānāgata-paccuppannā ajjhattā vā bahiddhā vā oḷārikā
vā sukhumā vā hīnā vā paṇītā vā yā dūre santike vā, sabbā vedanā netaṁ
mama, nesoham'asmi, na me so attā'ti, evametaṁ yathābhūtaṁ sammappaññāya
daṭṭhabbaṁ.

Yā kāci saññā atītānāgata-paccuppannā ajjhattā vā bahiddhā vā oḷārikā vā
sukhumā vā hīnā vā paṇītā vā yā dūre santike vā, sabbā saññā netaṁ mama,
nesoham'asmi, na me so attā'ti, evametaṁ yathābhūtaṁ sammappaññāya
daṭṭhabbaṁ.

Ye keci saṅkhārā atītānāgata-paccuppannā ajjhattā vā bahiddhā vā oḷārikā
vā sukhumā vā hīnā vā paṇītā vā ye dūre santike vā, sabbe saṅkhārā netaṁ
mama, nesoham'asmi, na me so attā'ti, evametaṁ yathābhūtaṁ sammappaññāya
daṭṭhabbaṁ.

\clearpage

\englishText
\markboth{\englishTitle}{\rightmark}

“Cualquier clase de conciencia, sea del pasado, futuro o presente, interna o externa, vulgar o sutil, inferior o superior, lejana o cercana, toda conciencia debería ser vista tal como realmente es con la correcta sabiduría así: ‘eso no es mío, eso no soy yo, eso no es mi ser’.

“Viendo de esta manera, monjes, el instruido noble discípulo experimenta repugnancia hacia la forma, repugnancia hacia la sensación, repugnancia hacia la percepción, repugnancia hacia las formaciones mentales y repugnancia hacia la conciencia. Experimentando repugnancia, llega a estar desapasionado. Mediante el desapasionamiento [su mente] es liberada. Cuando se libera, llega este conocimiento: ‘esta es la liberación’, y comprende eso: ‘destruido está el nacimiento, la vida santa ha sido vivida, lo que había que hacer se ha realizado y he aquí no hay más futuros estados de existencia’”.

Esto es lo que dije el Bienaventurado y aquellos monjes fueron satisfechos y se deleitaron en las palabras del Bienaventurado. Y mientras se pronunciaba ese discurso las mentes de los monjes del grupo de los cinco fueron liberadas mediante el no apego.

Asi termina el discurso de las caracteristicas de anatta.

\clearpage

\paliText
\markboth{\paliTitle}{\rightmark}

Yaṁ kiñci viññāṇaṁ atītānāgata-paccuppannaṁ ajjhattaṁ vā bahiddhā vā
oḷārikaṁ vā sukhumaṁ vā hīnaṁ vā paṇītaṁ vā yandūre santike vā, sabbaṁ
viññāṇaṁ netaṁ mama, nesoham'asmi, na me so attā'ti, evametaṁ yathābhūtaṁ
sammappaññāya daṭṭhabbaṁ.

[Evaṁ passaṁ bhikkhave] sutvā ariyasāvako rūpasmim pi nibbindati, vedanāya
pi nibbindati, saññāya pi nibbindati, saṅkhāresu pi nibbindati,
viññāṇasmim pi nibbindati, nibbindaṁ virajjati, virāgā vimuccati,
vimuttasmiṁ vimuttam iti ñāṇaṁ hoti, khīṇā jāti, vusitaṁ brahmacariyaṁ,
kataṁ karaṇīyaṁ, nāparaṁ itthattāyā'ti pajānātī'ti.

[Idam-avoca bhagavā.] Attamanā pañcavaggiyā bhikkhū bhagavato bhāsitaṁ
abhinanduṁ. Imasmiñca pana veyyākaraṇasmiṁ bhaññamāne pañcavaggiyānaṁ
bhikkhūnaṁ anupādāya āsavehi cittāni vimucciṁsū'ti.

Anattalakkhaṇa-suttaṁ niṭṭhitaṁ.

\chapterTocDelegatePageNumber
\chapter{El Sermon del fuego}

\setTocDelegatedPageNumber
\englishText
\renewcommand{\englishTitle}{The Fire Sermon}

\begin{leader}
\soloinstr{Solo introduction}

With his skill in training the trainable, the All-transcendent Buddha,
lucid speaker, teacher of the highest knowledge,

He who expounds to the people the Dhamma and Vinaya that is fitting and
worthy, teaching with this wonderful parable about fire, meditators of
the highest skill;

He has liberated those who listen with the liberation that is utterly
complete, through true investigation, with wisdom\\ and attention.

Let us now recite this Sutta which describes the characteristics\\ of dukkha.

\end{leader}

Esto he escuchado. En una ocasión, el Bienaventurado estaba en Gaya, en la Cabeza de Gaya, junto a un grupo de mil monjes. Estando ahí, se dirigió a ellos con estas palabras:

“Monjes, todo está ardiendo. ¿Y qué es este ‘todo' que está ardiendo?

“El ojo está ardiendo, las formas están ardiendo, la conciencia del ojo está ardiendo, el contacto del ojo está ardiendo, también toda la sensación placentera o dolorosa, o la que no es ni placentera ni dolorosa dependiente del ojo como su condición indispensable, está ardiendo. ¿Ardiendo con qué? Ardiendo con el fuego de la pasion, con el fuego del odio, con el fuego de la falsa ilusión; ardiendo con el nacimiento, el envejecimiento y la muerte, con las penas, lamentaciones y dolores, con angustia y desesperación, declaro yo.


\enlargethispage{2\baselineskip}

“El oido está ardiendo, los sonidos están ardiendo, la conciencia del oido está ardiendo, el contacto del oido está ardiendo, también toda la sensación placentera o dolorosa, o la que no es ni placentera ni dolorosa dependiente del ojo como su condición indispensable, está ardiendo. ¿Ardiendo con qué? Ardiendo con el fuego de la pasion, con el fuego del odio, con el fuego de la falsa ilusión; ardiendo con el nacimiento, el envejecimiento y la muerte, con las penas, lamentaciones y dolores, con angustia y desesperación, declaro yo.




\chapterTocSubIndentTrue
\chapter{Āditta-pariyāya Sutta}

\paliText
\renewcommand{\paliTitle}{Āditta-pariyāya Sutta}

\begin{leader}
\soloinstr{Solo introduction}

\begin{solotwochants}
Veneyyadamanopāye  & sabbaso pāramiṁ gato\\
Amoghavacano buddho & abhiññāyānusāsako\\
Ciṇṇānurūpato cāpi & dhammena vinayaṁ pajaṁ\\
Ciṇṇāggipāricariyānaṁ & sambojjhārahayoginaṁ\\
Yamādittapariyāyaṁ & desayanto manoharaṁ\\
Te sotāro vimocesi & asekkhāya vimuttiyā\\
Tathevopaparikkhāya & viññūṇaṁ sotumicchataṁ\\
Dukkhatālakkhaṇopāyaṁ & taṁ suttantaṁ bhaṇāma se\\
\end{solotwochants}
\end{leader}

[Evaṁ me sutaṁ]

Ekaṁ samayaṁ bhagavā gayāyaṁ viharati gayāsīse saddhiṁ bhikkhu-sahassena.
Tatra kho bhagavā bhikkhū āmantesi:

Sabbaṁ bhikkhave ādittaṁ. Kiñca bhikkhave sabbaṁ ādittaṁ.

Cakkhuṁ bhikkhave ādittaṁ, rūpā ādittā, cakkhuviññāṇaṁ ādittaṁ,
cakkhusamphasso āditto, yampidaṁ cakkhusamphassapaccayā uppajjati
vedayitaṁ sukhaṁ vā dukkhaṁ vā adukkhamasukhaṁ vā tam pi ādittaṁ. Kena
ādittaṁ. Ādittaṁ rāgagginā dosagginā mohagginā, ādittaṁ jātiyā
jarāmaraṇena sokehi paridevehi dukkhehi domanassehi upāyāsehi ādittan'ti
vadāmi.

Sotaṁ ādittaṁ, saddā ādittā, sotaviññāṇaṁ ādittaṁ, sotasamphasso āditto,
yampidaṁ sotasamphassapaccayā uppajjati vedayitaṁ sukhaṁ vā dukkhaṁ vā
adukkhamasukhaṁ vā tam pi ādittaṁ. Kena ādittaṁ. Ādittaṁ rāgagginā
dosagginā mohagginā, ādittaṁ jātiyā jarāmaraṇena sokehi paridevehi
dukkhehi domanassehi upāyāsehi ādittan'ti vadāmi.

\clearpage

\englishText
\markboth{\englishTitle}{\rightmark}
“El olfato está ardiendo, los olores están ardiendo, la conciencia del olfato está ardiendo, el contacto del oido está ardiendo, también toda la sensación placentera o dolorosa, o la que no es ni placentera ni dolorosa dependiente del ojo como su condición indispensable, está ardiendo. ¿Ardiendo con qué? Ardiendo con el fuego de la pasion, con el fuego del odio, con el fuego de la falsa ilusión; ardiendo con el nacimiento, el envejecimiento y la muerte, con las penas, lamentaciones y dolores, con angustia y desesperación, declaro yo.

“La lengua está ardiendo, los sabores están ardiendo, la conciencia del olfato está ardiendo, el contacto del oido está ardiendo, también toda la sensación placentera o dolorosa, o la que no es ni placentera ni dolorosa dependiente del ojo como su condición indispensable, está ardiendo. ¿Ardiendo con qué? Ardiendo con el fuego de la pasion, con el fuego del odio, con el fuego de la falsa ilusión; ardiendo con el nacimiento, el envejecimiento y la muerte, con las penas, lamentaciones y dolores, con angustia y desesperación, declaro yo.

“La lengua está ardiendo, los sabores están ardiendo, la conciencia del olfato está ardiendo, el contacto del oido está ardiendo, también toda la sensación placentera o dolorosa, o la que no es ni placentera ni dolorosa dependiente del ojo como su condición indispensable, está ardiendo. ¿Ardiendo con qué? Ardiendo con el fuego de la pasion, con el fuego del odio, con el fuego de la falsa ilusión; ardiendo con el nacimiento, el envejecimiento y la muerte, con las penas, lamentaciones y dolores, con angustia y desesperación, declaro yo.

"El cuerpo está ardiendo, lo +++ están ardiendo, la conciencia del cuerpo está ardiendo, el contacto del cuerpo está ardiendo, también toda la sensación placentera o dolorosa, o la que no es ni placentera ni dolorosa dependiente del ojo como su condición indispensable, está ardiendo. ¿Ardiendo con qué? Ardiendo con el fuego de la pasion, con el fuego del odio, con el fuego de la falsa ilusión; ardiendo con el nacimiento, el envejecimiento y la muerte, con las penas, lamentaciones y dolores, con angustia y desesperación, declaro yo.

"La mente está ardiendo, las ideas están ardiendo, la conciencia de la mente está ardiendo, el contacto de la mente está ardiendo, también toda la sensación placentera o dolorosa, o la que no es ni placentera ni dolorosa dependiente del ojo como su condición indispensable, está ardiendo. ¿Ardiendo con qué? Ardiendo con el fuego de la pasion, con el fuego del odio, con el fuego de la falsa ilusión; ardiendo con el nacimiento, el envejecimiento y la muerte, con las penas, lamentaciones y dolores, con angustia y desesperación, declaro yo.

Monjes, viendo esto, el bien instruido noble discípulo experimenta desencanto hacia el ojo, hacia las formas, hacia la conciencia del ojo, hacia el contacto del ojo y hacia toda la sensación placentera o dolorosa, o cualquier sensacion neutra dependiente del ojo como su condición indispensable. 



\clearpage

\paliText
\markboth{\paliTitle}{\rightmark}

Ghānaṁ ādittaṁ, gandhā ādittā, ghānaviññāṇaṁ ādittaṁ, ghānasamphasso
āditto, yampidaṁ ghānasamphassapaccayā uppajjati vedayitaṁ sukhaṁ vā
dukkhaṁ vā adukkhamasukhaṁ vā tam pi ādittaṁ. Kena ādittaṁ. Ādittaṁ
rāgagginā dosagginā mohagginā, ādittaṁ jātiyā jarāmaraṇena sokehi
paridevehi dukkhehi domanassehi upāyāsehi ādittan'ti vadāmi.

Jivhā ādittā, rasā ādittā, jivhāviññāṇam ādittaṁ, jivhāsamphasso āditto,
yampidaṁ jivhāsamphassapaccayā uppajjati vedayitaṁ sukhaṁ vā dukkhaṁ vā
adukkhamasukhaṁ vā tam pi ādittaṁ. Kena ādittaṁ. Ādittaṁ rāgagginā
dosagginā mohagginā, ādittaṁ jātiyā jarāmaraṇena sokehi paridevehi
dukkhehi domanassehi upāyāsehi ādittan'ti vadāmi.

Kāyo āditto, phoṭṭhabbā ādittā, kāyaviññāṇaṁ ādittaṁ, kāyasamphasso
āditto, yampidaṁ kāyasamphassapaccayā uppajjati vedayitaṁ sukhaṁ vā
dukkhaṁ vā adukkhamasukhaṁ vā tam pi ādittaṁ. Kena ādittaṁ. Ādittaṁ
rāgagginā dosagginā mohagginā, ādittaṁ jātiyā jarāmaraṇena sokehi
paridevehi dukkhehi domanassehi upāyāsehi ādittan'ti vadāmi.

Mano āditto, dhammā ādittā, manoviññāṇaṁ ādittaṁ, manosamphasso āditto,
yampidaṁ manosamphassapaccayā uppajjati vedayitaṁ sukhaṁ vā dukkhaṁ vā
adukkhamasukhaṁ vā tam pi ādittaṁ. Kena ādittaṁ. Ādittaṁ rāgagginā
dosagginā mohagginā, ādittaṁ jātiyā jarāmaraṇena sokehi paridevehi
dukkhehi domanassehi upāyāsehi ādittan'ti vadāmi.

\enlargethispage{2\baselineskip}

[Evaṁ passaṁ bhikkhave] sutvā ariyasāvako cakkhusmiṁ pi nibbindati,
rūpesu pi nibbindati, cakkhuviññāṇe pi nibbindati, cakkhusamphassepi
nibbindati, yampidaṁ cakkhusamphassapaccayā uppajjati vedayitaṁ sukhaṁ
vā dukkhaṁ vā adukkhamasukhaṁ vā tasmiṁ pi nibbindati.

\clearpage

\englishText
\markboth{\englishTitle}{\rightmark}

Monjes, viendo esto, el bien instruido noble discípulo experimenta desencanto con el oido, desencanto con los sonidos, desencanto con la conciencia del oido, desencanto con el contacto del oido y con toda sensación placentera o dolorosa, o cualquier sensacion neutra dependiente del oido como su condición indispensable.

Monjes, viendo esto, el bien instruido noble discípulo experimenta desencanto con el olfato, desencanto con los olores, desencanto con la conciencia del olfato, desencanto con el contacto del olfato y con toda sensación placentera o dolorosa, o cualquier sensacion neutra dependiente del olfato como su condición indispensable.

Monjes, viendo esto, el bien instruido noble discípulo experimenta desencanto con la lengua, desencanto con los sabores, desencanto con la conciencia de la lengua, desencanto con el contacto de la lengua y con toda sensación placentera o dolorosa, o cualquier sensacion neutra dependiente de la lengua como su condición indispensable.

Monjes, viendo esto, el bien instruido noble discípulo experimenta desencanto con el cuerpo, desencanto con los objetos tangibles, desencanto con la conciencia del cuerpo, desencanto con el contacto del cuerpo y con toda sensación placentera o dolorosa, o cualquier sensacion neutra dependiente del cuerpo como su condición indispensable.


Monjes, viendo esto, el bien instruido noble discípulo experimenta desencanto con la mente, desencanto con las ideas, desencanto con la conciencia mental, desencanto con el contacto mental y con toda sensación placentera o dolorosa, o cualquier sensacion neutra dependiente de la mente como su condición indispensable.

“Y experimentando el desencanto, se vuelve desapasionado. Mediante el desapasionamiento, su mente es liberada. Cuando es liberado, aparece en él este conocimiento: ‘Ésta es la liberación’. Entonces entiende que ‘el nacimiento está destruido, la vida santa ha sido realizada, la tarea ha concluido. No queda más nada por hacer’”.

\enlargethispage{\baselineskip}

Esto dijo el Bienaventurado y aquellos monjes fueron elevados y se deleitaron en las palabras del Bienaventurado.

Y durante este discurso, las mentes de estos mil monjes fueron plenamente liberadas de las contaminaciones a través del no-apego.


Asi termina el sermon del fuego.

\clearpage

\paliText
\markboth{\paliTitle}{\rightmark}

Sotasmiṁ pi nibbindati, saddesu pi nibbindati, sotaviññāṇe pi
nibbindati, sotasamphassepi nibbindati, yampidaṁ sotasamphassapaccayā
uppajjati vedayitaṁ sukhaṁ vā dukkhaṁ vā adukkhamasukhaṁ vā tasmiṁ pi
nibbindati.

Ghānasmiṁ pi nibbindati, gandhesu pi nibbindati, ghānaviññāṇe pi
nibbindati, ghānasamphassepi nibbindati, yampidaṁ ghānasamphassapaccayā
uppajjati vedayitaṁ sukhaṁ vā dukkhaṁ vā adukkhamasukhaṁ vā tasmiṁ pi
nibbindati.

Jivhāya pi nibbindati, rasesu pi nibbindati, jivhāviññāṇe pi nibbindati,
jivhāsamphassepi nibbindati, yampidaṁ jivhāsamphassapaccayā uppajjati
vedayitaṁ sukhaṁ vā dukkhaṁ vā adukkhamasukhaṁ vā tasmiṁ pi nibbindati.

Kāyasmiṁ pi nibbindati, phoṭṭhabbesu pi nibbindati, kāyaviññāṇe pi
nibbindati, kāyasamphassepi nibbindati, yampidaṁ kāyasamphassapaccayā
uppajjati vedayitaṁ sukhaṁ vā dukkhaṁ vā adukkhamasukhaṁ vā tasmiṁ pi
nibbindati.

Manasmiṁ pi nibbindati, dhammesu pi nibbindati, manoviññāṇe pi
nibbindati, manosamphassepi nibbindati, yampidaṁ manosamphassapaccayā
uppajjati vedayitaṁ sukhaṁ vā dukkhaṁ vā adukkhamasukhaṁ vā tasmiṁ pi
nibbindati.

Nibbindaṁ virajjati, virāgā vimuccati, vimuttasmiṁ, vimuttam iti ñāṇaṁ
hoti, khīṇā jāti, vusitaṁ brahmacariyaṁ, kataṁ karaṇīyaṁ, nāparaṁ
itthattāyā'ti pajānātī'ti.

\enlargethispage{\baselineskip}

[Idam-avoca bhagavā.] Attamanā te bhikkhū bhagavato bhāsitaṁ abhinanduṁ.
Imasmiñca pana veyyākaraṇasmiṁ bhaññamāne tassa bhikkhu-sahassassa
anupādāya āsavehi cittāni vimucciṁsū'ti.

Ādittapariyāya-suttaṁ niṭṭhitaṁ.

\resumeNormalText

% End of suttas.tex
