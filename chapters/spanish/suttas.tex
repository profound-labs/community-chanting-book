\chapterTocDelegatePageNumber
\chapter{La puesta en marcha de la rueda del Dhamma}

\setTocDelegatedPageNumber
\englishText
\renewcommand{\englishTitle}{La puesta en marcha de la rueda del Dhamma}

\begin{leader}
\soloinstr{Introduccción}

Esta es la primera enseñanza del Tathāgata sobre la obtención de la insuperable perfecta iluminación.
Aquí se pone en marcha la incomparable rueda de la verdad, inestimable dondequiera que es expuesta en este mundo.
Aquí se desvelan los dos extremos, y el camino medio, con las cuatro verdades nobles y el conocimiento y visión purificados, señalados por el Señor del Dhamma.
Cantemos juntos este Sutta proclamando la iluminación suprema e independiente que es ampliamente conocida como ‘La puesta en marcha de la rueda del Dhamma’.


\end{leader}

%%Sutta Central SN56.11
%%https://suttacentral.net/sn56.11/es/baron?lang=es&reference=none&highlight=false

Esto es lo que he oído. En una ocasión el Excelso estaba residiendo cerca de Benares, en Isipatana, en el Parque de los Ciervos. Allí el Excelso se dirigió al grupo de los cinco monjes:

‘Estos dos extremos, oh monjes, no deberían ser seguidos por un renunciante. Complacencia en los placeres sensuales, esto es bajo, vulgar, ordinario, innoble y sin beneficio; y adicción a la mortificación, esto es doloroso, innoble y sin beneficio.’

‘Monjes, evitando estos dos extremos, el Tathāgata ha penetrado el camino medio que genera la visión, que genera el entendimiento, que conduce a la paz, que conduce a la sabiduría, que conduce a la iluminación y que conduce al Nibbana.’

‘¿Cuál, oh monjes, es el camino medio que el Tathāgata ha penetrado que genera la visión, que genera el entendimiento, que conduce a la paz, que conduce a la sabiduría, que conduce a la iluminación y al Nibbana?’



\chapterTocSubIndentTrue
\chapter{Dhammacakkappavattana Sutta}

\paliText
\renewcommand{\paliTitle}{Dhammacakkappavattana Sutta}

\begin{leader}
\soloinstr{Introducción}

\begin{solotwochants}
Anuttaraṁ abhisambodhiṁ & sambujjhitvā tathāgato\\
Pathamaṁ yaṁ adesesi & dhammacakkaṁ anuttaraṁ\\
Sammadeva pavattento & loke appativattiyaṁ\\
Yatthākkhātā ubho antā & paṭipatti ca majjhimā\\
Catūsvāriyasaccesu & visuddhaṁ ñāṇadassanaṁ\\
Desitaṁ dhammarājena & sammāsambodhikittanaṁ\\
Nāmena vissutaṁ suttaṁ & dhammacakkappavattanaṁ\\
Veyyākaraṇapāthena & saṅgītantam bhaṇāma se\\
\end{solotwochants}
\end{leader}

[Evaṁ me sutaṁ]

Ekaṁ samayaṁ bhagavā bārāṇasiyaṁ viharati isipatane migadāye. Tatra kho
bhagavā pañcavaggiye bhikkhū āmantesi:

Dve'me, bhikkhave, antā pabbajitena na sevitabbā: yo cāyaṁ kāmesu
kāma-sukh'allikānuyogo; hīno, gammo, pothujjaniko, anariyo,
anattha-sañhito; yo cāyaṁ atta-kilamathānuyogo; dukkho, anariyo,
anattha-sañhito.

Ete te, bhikkhave, ubho ante anupagamma majjhimā paṭipadā tathāgatena
abhisambuddhā cakkhukaraṇī, ñāṇakaraṇī, upasamāya, abhiññāya,
sambodhāya, nibbānāya saṁvattati.

Katamā ca sā, bhikkhave, majjhimā paṭipadā tathāgatena abhisambuddhā
cakkhukaraṇī ñāṇakaraṇī, upasamāya, abhiññāya, sambodhāya, nibbānāya
saṁvattati.

\clearpage

\englishText
\markboth{\englishTitle}{\rightmark}

‘Es simplemente este Óctuple Noble Sendero; es decir, Visión Correcta, Intención Correcta, Lenguaje Correcto, Acción Correcta, Medio de Vida Correcto, Esfuerzo Correcto, Sati Correcto y Concentración Correcta. Éste, oh monjes, es ese camino medio que el Tathāgata ha penetrado que genera la visión, que genera el entendimiento, que conduce a la paz, que conduce a la sabiduría, que conduce a la iluminación y al Nibbana.’

‘Ésta, oh monjes, es la Noble Verdad de Dukkha:’

‘El nacimiento es dukkha, el envejecimiento es dukkha, la muerte es dukkha, asociarse con lo indeseable es dukkha, separarse de lo deseable es dukkha, no obtener lo que uno quiere es dukkha. Resumiendo, el apego a las cinco khandhas es dukkha.’

‘Ésta, oh monjes, es la Noble Verdad del Origen de Dukkha:'
‘Es este deseo que genera nueva existencia, que asociado con placer y pasión se deleita aquí y allá. Es decir, el deseo sensual, el deseo por la existencia y el deseo por la no existencia.’

‘Ésta, oh monjes, es la Noble Verdad del Cese de Dukkha:'
‘Es la total extinción y cese de ese mismo deseo, su abandono, su descarte, liberación, no dependencia.’

‘Ésta, oh monjes, es la Noble Verdad del Sendero que conduce al Cese de Dukkha:’
‘Es simplemente este Óctuple Noble Sendero; es decir, Visión Correcta, Intención Correcta, Lenguaje Correcto, Acción Correcta, Medio de Vida Correcto, Esfuerzo Correcto, Sati Correcto y Concentración Correcta.’

‘Con el pensamiento, ‘Ésta es la Noble Verdad de Dukkha', oh monjes, surgió en mí la visión, surgió el entendimiento, surgió la sabiduría, surgió la penetración y surgió la luz en relación con cosas desconocidas por mí anteriormente.’

\clearpage

\paliText
\markboth{\paliTitle}{\rightmark}

Ayam-eva ariyo aṭṭhaṅgiko maggo seyyathīdaṁ:

Sammā-diṭṭhi, sammā-saṅkappo, sammā-vācā, sammā-kammanto, sammā-ājīvo,
sammā-vāyāmo, sammā-sati, sammā-samādhi.

Ayaṁ kho sā, bhikkhave, majjhimā paṭipadā tathāgatena abhisambuddhā
cakkhukaraṇī ñāṇakaraṇī, upasamāya, abhiññāya, sambodhāya, nibbānāya
saṁvattati.

Idaṁ kho pana, bhikkhave, dukkhaṁ ariya-saccaṁ:

Jātipi dukkhā, jarāpi dukkhā, maranampi dukkhaṁ,
soka-parideva-dukkha-domanass'upāyāsāpi dukkhā, appiyehi sampayogo
dukkho, piyehi vippayogo dukkho, yamp'icchaṁ na labhati tampi dukkhaṁ,
saṅkhittena pañcupādānakkhandhā dukkhā.

Idaṁ kho pana, bhikkhave, dukkha-samudayo ariya-saccaṁ:

Yā'yaṁ taṇhā ponobbhavikā nandi-rāga-sahagatā tatra-tatrābhinandinī
seyyathīdaṁ: kāma-taṇhā, bhava-taṇhā, vibhava-taṇhā.

Idaṁ kho pana, bhikkhave, dukkha-nirodho ariya-saccaṁ:

Yo tassā yeva taṇhāya asesa-virāga-nirodho, cāgo, paṭinissaggo, mutti,
anālayo.

Idaṁ kho pana, bhikkhave, dukkha-nirodha-gāminī paṭipadā ariya-saccaṁ:

Ayam-eva ariyo aṭṭhaṅgiko maggo seyyathīdam: sammā-diṭṭhi,
sammā-saṅkappo, sammā-vācā, sammā-kammanto, sammā-ājīvo, sammā-vāyāmo,
sammā-sati, sammā-samādhi.

\enlargethispage{\baselineskip}

[Idaṁ dukkhaṁ] ariya-saccan'ti me bhikkhave, pubbe ananussutesu dhammesu
cakkhuṁ udapādi, ñāṇaṁ udapādi, paññā udapādi, vijjā udapādi, āloko
udapādi.

\clearpage

\englishText
\markboth{\englishTitle}{\rightmark}

‘Con el pensamiento, ‘Ésta Noble Verdad de dukkha debe ser completamente comprendida', oh monjes, surgió en mí la visión, surgió el entendimiento, surgió la sabiduría, surgió la penetración y surgió la luz en relación con cosas desconocidas por mí anteriormente.’

‘Con el pensamiento, ‘Ésta Noble Verdad de dukkha ha sido completamente comprendida', oh monjes, surgió en mí la visión, surgió el entendimiento, surgió la sabiduría, surgió la penetración y surgió la luz en relación con cosas desconocidas por mí anteriormente.’

‘Con el pensamiento, ‘Ésta es la Noble Verdad del origen de dukkha', oh monjes, surgió en mí la visión, surgió el entendimiento, surgió la sabiduría, surgió la penetración y surgió la luz en relación con cosas desconocidas por mí anteriormente.’

‘Con el pensamiento, ‘Ésta Noble Verdad del origen de dukkha debe ser abandonada', oh monjes, surgió en mí la visión, surgió el entendimiento, surgió la sabiduría, surgió la penetración y surgió la luz en relación con cosas desconocidas por mí anteriormente.’

‘Con el pensamiento, ‘Ésta Noble Verdad del origen de dukkha ha sido abandonada', oh monjes, surgió en mí la visión, surgió el entendimiento, surgió la sabiduría, surgió la penetración y surgió la luz en relación con cosas desconocidas por mí anteriormente.’

‘Con el pensamiento, ‘Ésta es la Noble Verdad del cese de dukkha', oh monjes, surgió en mí la visión, surgió el entendimiento, surgió la sabiduría, surgió la penetración y surgió la luz en relación con cosas desconocidas por mí anteriormente.’

\enlargethispage*{2\baselineskip}

‘Con el pensamiento, ‘Ésta Noble Verdad del cese de dukkha debe ser realizada', oh monjes, surgió en mí la visión, surgió el entendimiento, surgió la sabiduría, surgió la penetración y surgió la luz en relación con cosas desconocidas por mí anteriormente.’

‘Con el pensamiento, ‘Ésta Noble Verdad del cese de dukkha ha sido realizada', oh monjes, surgió en mí la visión, surgió el entendimiento, surgió la sabiduría, surgió la penetración y surgió la luz en relación con cosas desconocidas por mí anteriormente.’


\clearpage

\paliText
\markboth{\paliTitle}{\rightmark}

Taṁ kho pan'idaṁ dukkhaṁ ariya-saccaṁ pariññeyyan'ti me bhikkhave, pubbe
ananussutesu dhammesu cakkhuṁ udapādi, ñāṇaṁ udapādi, paññā udapādi,
vijjā udapādi, āloko udapādi.

Taṁ kho pan'idaṁ dukkhaṁ ariya-saccaṁ pariññātan'ti me bhikkhave, pubbe
ananussutesu dhammesu cakkhuṁ udapādi, ñāṇaṁ udapādi, paññā udapādi,
vijjā udapādi, āloko udapādi.

Idaṁ dukkha-samudayo ariya-saccan'ti me bhikkhave, pubbe ananussutesu
dhammesu cakkhuṁ udapādi, ñāṇaṁ udapādi, paññā udapādi, vijjā udapādi,
āloko udapādi.

Taṁ kho pan'idaṁ dukkha-samudayo ariyasaccaṁ pahātabban'ti me bhikkhave,
pubbe ananussutesu dhammesu cakkhuṁ udapādi, ñāṇaṁ udapādi, paññā
udapādi, vijjā udapādi, āloko udapādi.

Taṁ kho pan'idaṁ dukkha-samudayo ariya-saccaṁ pahīnan'ti me bhikkhave, pubbe
ananussutesu dhammesu cakkhuṁ udapādi, ñāṇaṁ udapādi, paññā udapādi,
vijjā udapādi, āloko udapādi.

Idaṁ dukkha-nirodho ariya-saccan'ti me bhikkhave, pubbe ananussutesu
dhammesu cakkhuṁ udapādi, ñāṇaṁ udapādi, paññā udapādi, vijjā udapādi,
āloko udapādi.

Taṁ kho pan'idaṁ dukkha-nirodho ariya-saccaṁ sacchikātabban'ti me bhikkhave,
pubbe ananussutesu dhammesu cakkhuṁ udapādi, ñāṇaṁ udapādi, paññā
udapādi, vijjā, udapādi āloko udapādi.

Taṁ kho pan'idaṁ dukkha-nirodho ariya-saccaṁ sacchikatan'ti me bhikkhave,
pubbe ananussutesu dhammesu cakkhuṁ udapādi, ñāṇaṁ udapādi, paññā
udapādi, vijjā udapādi, āloko udapādi.

\clearpage

\englishText
\markboth{\englishTitle}{\rightmark}
‘Con el pensamiento, ‘Ésta es la Noble Verdad del Sendero que conduce al cese de dukkha', oh monjes, surgió en mí la visión, surgió el entendimiento, surgió la sabiduría, surgió la penetración y surgió la luz en relación con cosas desconocidas por mí anteriormente.’

‘Con el pensamiento, ‘Ésta Noble Verdad del Sendero que conduce al cese de dukkha debe ser desarrollada', oh monjes, surgió en mí la visión, surgió el entendimiento, surgió la sabiduría, surgió la penetración y surgió la luz en relación con cosas desconocidas por mí anteriormente.’

‘Con el pensamiento, ‘Ésta Noble Verdad del Sendero que conduce al cese de dukkha ha sido desarrollada', oh monjes, surgió en mí la visión, surgió el entendimiento, surgió la sabiduría, surgió la penetración y surgió la luz en relación con cosas desconocidas por mí anteriormente.’

‘Mientras, oh monjes, el entendimiento y la visión con respecto a estas Cuatro Nobles Verdades de acuerdo con la realidad bajo sus tres modos y doce aspectos no fueron totalmente purificados en mí, no pude declarar al mundo con sus divinidades, Maras y Brahmas, a la humanidad con sus ascéticos, brahmanes y hombres, la realización de la incomparable, perfecta iluminación.’

‘Pero cuando, oh monjes, el entendimiento y la visión con respecto a estas Cuatro Nobles Verdades de acuerdo con la realidad bajo sus tres modos y doce aspectos fueron totalmente purificados en mí, entonces pude declarar al mundo con sus divinidades, Maras y Brahmas, a la humanidad con sus ascéticos, brahmanes y hombres, que había realizado la incomparable, perfecta iluminación.’
‘Y en mí surgió el entendimiento y la visión: ‘Inconmovible es mi liberación. Éste es mi último nacimiento. Ahora ya no habrá nueva existencia’.

Así hablo el Excelso. Los cinco monjes se regocijaron en las palabras del Excelso.


\clearpage

\paliText
\markboth{\paliTitle}{\rightmark}

Idaṁ dukkha-nirodha-gāminī paṭipadā ariya-saccan'ti me bhikkhave, pubbe
ananussutesu dhammesu cakkhuṁ udapādi, ñāṇaṁ udapādi, paññā udapādi,
vijjā udapādi, āloko udapādi.

Taṁ kho pan'idaṁ dukkha-nirodha-gāminī paṭipadā ariya-saccaṁ bhāvetabban'ti
me bhikkhave, pubbe ananussutesu dhammesu cakkhuṁ udapādi, ñāṇaṁ
udapādi, paññā udapādi, vijjā udapādi, āloko udapādi.

Taṁ kho pan'idaṁ dukkha-nirodha-gāminī paṭipadā ariya-saccaṁ bhāvitan'ti me
bhikkhave, pubbe ananussutesu dhammesu cakkhuṁ udapādi, ñāṇaṁ udapādi,
paññā udapādi, vijjā udapādi, āloko udapādi.

[Yāva kīvañca me bhikkhave,] imesu catūsu ariya-saccesu evan-ti-parivaṭṭaṁ
dvādas'ākāraṁ yathā-bhūtaṁ ñāṇa-dassanaṁ na suvisuddhaṁ ahosi, n'eva tāv'āhaṁ
bhikkhave, sadevake loke samārake sabrahmake sassamaṇa-brāhmaṇiyā pajāya
sadeva-manussāya anuttaraṁ sammā-sambodhiṁ abhisambuddho paccaññāsiṁ.

Yato ca kho me bhikkhave, imesu catūsu ariya-saccesu evan-ti-parivaṭṭaṁ
dvādas'ākāraṁ yathā-bhūtaṁ ñāṇa-dassanaṁ suvisuddham ahosi, ath'āham
bhikkhave, sadevake loke samārake sabrahmake sassamaṇa-brāhmaṇiyā pajāya
sadeva-manussāya anuttaraṁ sammā-sambodhiṁ abhisambuddho paccaññāsiṁ.

Ñāṇañca pana me dassanaṁ udapādi, akuppā me vimutti ayam-antimā jāti,
natthi dāni punabbhavo'ti.

Idam-avoca bhagavā. Attamanā pañcavaggiyā bhikkhū bhagavato bhāsitaṁ
abhinanduṁ.

\clearpage

\englishText
\markboth{\englishTitle}{\rightmark}

Cuando esta exposición se estaba impartiendo surgió en el Venerable Kondañño la pura e inmaculada visión del Dhamma: ‘Todo aquello que está sujeto a surgir, está sujeto a cesar’.

Cuando el Excelso puso en movimiento la rueda de la doctrina, la divinidades terrestres hicieron oír este sonido: ‘Esta excelente rueda del Dhamma ha sido puesta en movimiento por el Excelso cerca de Benares, en Isipatana, en el Parque de los Ciervos, y no puede ser detenida por ningún ascético, brahmán, divinidad, Māra, Brāhma, o ningún ser en el mundo’.

Habiendo escuchado esto de las divinidades terrestres, las divinidades de Cātumahārājika hicieron oír este sonido\ldots

Habiendo escuchado esto de las divinidades de Cātumahārājika, las divinidades de Tāvatiṁsa hicieron oír este sonido\ldots

Habiendo escuchado esto de las divinidades de Tāvatiṁsa, las divinidades de Yāma hicieron oír este sonido\ldots  

Habiendo escuchado esto de las divinidades de Yāma, las divinidades de Tusita hicieron oír este sonido\ldots

Habiendo escuchado esto de las divinidades de Tusita, las divinidades de Nimmānaratì hicieron oír este sonido\ldots

Habiendo escuchado esto de las divinidades de Nimmānarati, las divinidades de Paranimmitavasavattì hicieron oír este sonido\ldots

Habiendo escuchado esto de las divinidades de Paranimmitavasavattì, las divinidades del mundo de los Brahmas hicieron oír este sonido: ‘Esta excelente rueda del Dhamma ha sido puesta en movimiento por el Excelso cerca de Benares, en Isipatana, en el Parque de los Ciervos, y no puede ser detenida por ningún ascético, brahmán, divinidad, Māra, Brāhma, o ningún ser en el mundo’.



\clearpage

\paliText
\markboth{\paliTitle}{\rightmark}

Imasmiñca pana veyyākaraṇasmiṁ bhaññamāne āyasmato koṇḍaññassa virajaṁ
vītamalaṁ dhammacakkhuṁ udapādi: yaṁ kiñci samudaya-dhammaṁ sabban-taṁ
nirodha-dhamman'ti.

[Pavattite ca bhagavatā] dhammacakke bhummā devā saddamanussāvesuṁ:

Etaṁ bhagavatā bārāṇasiyaṁ isipatane migadāye anuttaraṁ dhammacakkaṁ
pavattitaṁ appaṭivattiyaṁ samaṇena vā brāhmaṇena vā devena vā mārena vā
brahmunā vā kenaci vā lokasmin'ti.

Bhummānaṁ devānaṁ saddaṁ sutvā, cātummahārājikā devā
saddamanussāvesuṁ\ldots

Cātummahārājikānaṁ devānaṁ saddaṁ sutvā, tāvatiṁsā devā
saddamanussāvesuṁ\ldots

Tāvatiṁsānaṁ devānaṁ saddaṁ sutvā, yāmā devā saddamanussāvesuṁ\ldots

Yāmānaṁ devānaṁ saddaṁ sutvā, tusitā devā saddamanussāvesuṁ\ldots

Tusitānaṁ devānaṁ saddaṁ sutvā, nimmānaratī devā saddamanussāvesum\ldots

Nimmānaratīnaṁ devānaṁ saddaṁ sutvā, paranimmitavasavattī devā
saddamanussāvesuṁ\ldots

Paranimmitavasavattīnaṁ devānaṁ saddaṁ sutvā, brahmakāyikā devā
saddamanussāvesuṁ:

Etaṁ bhagavatā bārāṇasiyaṁ isipatane migadāye anuttaraṁ dhammacakkaṁ
pavattitaṁ appaṭivattiyaṁ samaṇena vā brāhmaṇena vā devena vā mārena vā
brahmunā vā kenaci vā lokasmin'ti.

\clearpage

\englishText
\markboth{\englishTitle}{\rightmark}

Y en ese segundo, en ese momento, en ese instante, esa exclamación se extendió hasta el mundo de los Brahmas. Y los diez mil universos se estremecieron, se sacudieron y temblaron violentamente. Una espléndida e ilimitada luminosidad, sobrepasando la refulgencia de las divinidades, se manifestó en el mundo.

Después el Excelso pronunció esta expresión de alegría: ‘Amigos, Kondañño realmente ha comprendido. Amigos, Kondañño realmente ha comprendido’. Y el Venerable Kondañño fue llamado Añña-Kondañño.


Así termina el discurso de la puesta en marcha de la rueda del Dhamma.

\clearpage

\paliText
\markboth{\paliTitle}{\rightmark}

Iti'ha tena khaṇena, tena muhuttena, yāva brahmalokā saddo abbhuggacchi.
Ayañca dasa-sahassī lokadhātu saṅkampi sampakampi sampavedhi, appamāṇo ca
oḷāro obhāso loke pāturahosi atikkammeva devānaṁ devānubhāvaṁ.

Atha kho bhagavā udānaṁ udānesi:

Aññāsi vata bho koṇḍañño, aññāsi vata bho koṇḍañño'ti. Iti hidaṁ āyasmato
koṇḍaññassa aññā-koṇḍañño tveva nāmaṁ ahosī'ti.

Dhammacakkappavattana-suttaṁ niṭṭhitaṁ.

\chapterTocDelegatePageNumber
\chapter{Las Características de Anatta}

% Sutta Central SN22.59
%%https://suttacentral.net/sn22.59/es/baron?lang=es&reference=none&highlight=false
\setTocDelegatedPageNumber
\englishText
\renewcommand{\englishTitle}{Las Características de Anatta}

\begin{leader}
\soloinstr{Introducción}

Todos los seres deberían conocer las características de Anatta a través de dukkha, que provee de percepción y visión, como fue enseñada por el Buddha supremo.
Esta enseñanza es dada para que aquellos que meditan sobre las realidades experimentales puedan alcanzar la perfecta comprensión.
Es para el desarrollo del perfecto entendimiento de este fenómeno y para la investigación de todos los momentos mentales sucios.
La consecuencia de esta práctica es la total liberación. Así, deseosos de llevar esta enseñanza a buen término con gran beneficio, recitemos ahora este discurso.

\end{leader}

Esto he escuchado:

En una ocasión el Excelso estaba residiendo en Benares, en el Parque de los Ciervos de Isipatana. Allí, se dirigió al grupo de los cinco: 

‘Monjes, la forma no es ‘yo’. Porque, monjes, si la forma fuera ‘yo’, no conduciría a la aflicción y sería posible conseguir esto de la forma: ‘que la forma sea así’ o ‘que la forma no sea así’. Pero como la forma no es ‘yo’, conduce a la aflicción y no es posible conseguir esto de la forma: ‘que la forma sea así’ o ‘que la forma no sea así’.

‘La sensación no es ‘yo’. Porque, monjes, si la sensación fuera ‘yo’, no conduciría a la aflicción y sería posible conseguir esto de la sensación: ‘que la sensación sea así’ o ‘que la sensación no sea así’. Pero como la sensación no es ‘yo’, conduce a la aflicción y no es posible conseguir esto de la sensación: ‘que la sensación sea así’ o ‘que la sensación no sea así’.

\chapterTocSubIndentTrue
\chapter{Anatta-lakkhaṇa Sutta}

\paliText
\renewcommand{\paliTitle}{Anatta-lakkhaṇa Sutta}

\begin{leader}
\soloinstr{Introducción}

{\setlength{\tabcolsep}{0.9em}
\begin{solotwochants}
Yantaṁ sattehi dukkhena & ñeyyaṁ anattalakkhaṇaṁ\\
Attavādattasaññāṇaṁ  & sammadeva vimocanaṁ\\
Sambuddho taṁ pakāsesi & diṭṭhasaccāna yoginaṁ\\
Uttariṁ paṭivedhāya & bhāvetuṁ ñāṇamuttamaṁ\\
Yantesaṁ diṭṭhadhammānam & ñāṇenupaparikkhataṁ\\
Sabbāsavehi cittāni & vimucciṁsu asesato\\
Tathā ñāṇānussārena & sāsanaṁ kātumicchataṁ\\
Sādhūnaṁ atthasiddhatthaṁ & taṁ suttantaṁ bhaṇāma se\\
\end{solotwochants}
}
\end{leader}

[Evaṁ me sutaṁ]

Ekaṁ samayaṁ bhagavā bārāṇasiyaṁ viharati isipatane migadāye. Tatra kho
bhagavā pañcavaggiye bhikkhū āmantesi:

Rūpaṁ bhikkhave anattā, rūpañca hidaṁ bhikkhave attā abhavissa, nayidaṁ rūpaṁ
ābādhāya saṁvatteyya, labbhetha ca rūpe, evaṁ me rūpaṁ hotu, evaṁ me rūpaṁ mā
ahosī'ti. Yasmā ca kho bhikkhave rūpaṁ anattā, tasmā rūpaṁ ābādhāya saṁvattati,
na ca labbhati rūpe, evaṁ me rūpaṁ hotu, evaṁ me rūpaṁ mā ahosī'ti.

Vedanā anattā, vedanā ca hidaṁ bhikkhave attā abhavissa, nayidaṁ vedanā ābādhāya
saṁvatteyya, labbhetha ca vedanāya, evaṁ me vedanā hotu, evaṁ me vedanā mā
ahosī'ti. Yasmā ca kho bhikkhave vedanā anattā, tasmā vedanā ābādhāya
saṁvattati, na ca labbhati vedanāya, evaṁ me vedanā hotu, evaṁ me vedanā mā
ahosī'ti.

\clearpage

\englishText
\markboth{\englishTitle}{\rightmark}

‘La percepción no es ‘yo’. Porque, monjes, si la percepción fuera ‘yo’, no conduciría a la aflicción y sería posible conseguir esto de la percepción: ‘que la percepción sea así’ o ‘que la percepción no sea así’. Pero como la percepción no es ‘yo’, conduce a la aflicción y no es posible conseguir esto de la percepción: ‘que la percepción sea así’ o ‘que la percepción no sea así’.

‘Las formaciones mentales no son ‘yo’. Porque, monjes, si las formaciones mentales fueran ‘yo’, no conducirían a la aflicción y sería posible conseguir esto de las formaciones mentales: ‘que las formaciones mentales sean así’ o ‘que las formaciones mentales no sean así’. Pero como las formaciones mentales no son ‘yo’, conducen a la aflicción y no es posible conseguir esto de las formaciones mentales: ‘que las formaciones mentales sea así’ o ‘que las formaciones mentales no sean así’.

‘La cognición no es ‘yo’. Porque, monjes, si la cognición fuera ‘yo’, no conduciría a la aflicción y sería posible conseguir esto de la cognición: ‘que la cognición sea así’ o ‘que la cognición no sea así’. Pero como la cognición no es ‘yo’, conduce a la aflicción y no es posible conseguir esto de la cognición: ‘que la cognición sea así’ o ‘que la cognición no sea así’.

‘¿Qué opináis, monjes, es la forma permanente o impermanente?’

‘Impermanente, venerable señor’

‘Y, lo que es impermanente, ¿es doloroso o placentero?’

‘Doloroso, venerable señor’

‘Y, lo que es impermanente, es doloroso y sujeto al cambio, ¿puede ser considerado de esta manera: ‘eso es mío, eso soy yo, eso es mi ser’?’

‘No, venerable señor.’

\clearpage

\paliText
\markboth{\paliTitle}{\rightmark}

Saññā anattā, saññā ca hidaṁ bhikkhave attā abhavissa, nayidaṁ saññā ābādhāya
saṁvatteyya, labbhetha ca saññāya, evaṁ me saññā hotu, evaṁ me saññā mā
ahosī'ti. Yasmā ca kho bhikkhave saññā anattā, tasmā, saññā ābādhāya saṁvattati,
na ca labbhati saññāya, evaṁ me saññā hotu, evaṁ me saññā mā ahosī'ti.

Saṅkhārā anattā, saṅkhārā ca hidaṁ bhikkhave attā abhavissaṁsu, nayidaṁ saṅkhārā
ābādhāya saṁvatteyyuṁ, labbhetha ca saṅkhāresu, evaṁ me saṅkhārā hontu, evaṁ me
saṅkhārā mā ahesun'ti. Yasmā ca kho bhikkhave saṅkhārā anattā, tasmā saṅkhārā
ābādhāya saṁvattanti, na ca labbhati saṅkhāresu, evaṁ me saṅkhārā hontu, evaṁ me
saṅkhārā mā ahesun'ti.

Viññāṇaṁ anattā, viññāṇañca hidaṁ bhikkhave attā abhavissa, nayidaṁ viññānam
ābādhāya saṁvatteyya, labbhetha ca viññāne evaṁ me viññāṇaṁ hotu, evaṁ me
viññāṇaṁ mā ahosī'ti. Yasmā ca kho bhikkhave viññāṇaṁ anattā, tasmā viññāṇaṁ
ābādhāya saṁvattati, na ca labbhati viññāne, evaṁ me viññāṇaṁ hotu, evaṁ me
viññāṇaṁ mā ahosī'ti.

[Taṁ kiṁ maññatha bhikkhave,] rūpam niccaṁ vā aniccaṁ vā'ti.

Aniccaṁ bhante.

Yam panāniccaṁ, dukkhaṁ vā taṁ sukhaṁ vā'ti.

Dukkhaṁ bhante.

Yam panāniccaṁ dukkhaṁ viparināma-dhammaṁ, kallaṁ nu taṁ samanupassituṁ,
etaṁ mama, esoham'asmi, eso me attā'ti.

No hetaṁ bhante.

\clearpage

\englishText
\markboth{\englishTitle}{\rightmark}
‘¿Qué opináis, monjes, es la sensación permanente o impermanente?’

‘Impermanente, venerable señor’

‘Y, lo que es impermanente, ¿es doloroso o placentero?’

‘Doloroso, venerable señor’

‘Y, lo que es impermanente, es doloroso y sujeto al cambio, ¿puede ser considerado de esta manera: ‘eso es mío, eso soy yo, eso es mi ser’?’

‘No, venerable señor’

‘¿Qué opináis, monjes, es la percepción permanente o impermanente?’

‘Impermanente, venerable señor’

‘Y, lo que es impermanente, ¿es doloroso o placentero?’

‘Doloroso, venerable señor’

‘Y, lo que es impermanente, es doloroso y sujeto al cambio, ¿puede ser considerado de esta manera: ‘eso es mío, eso soy yo, eso es mi ser’?’

‘No, venerable señor’


‘¿Qué opináis, monjes, son las formaciones mentales permanentes o impermanentes?’

‘Impermanentes, venerable señor’

‘Y, lo que es impermanente, ¿es doloroso o placentero?’

‘Doloroso, venerable señor’

‘Y, lo que es impermanente, es doloroso y sujeto al cambio, ¿puede ser considerado de esta manera: ‘eso es mío, eso soy yo, eso es mi ser’?’

‘No, venerable señor’



\clearpage

\paliText
\markboth{\paliTitle}{\rightmark}

Taṁ kiṁ maññatha bhikkhave, vedanā niccā vā aniccā vā'ti.

Aniccā bhante.

Yam panāniccaṁ, dukkhaṁ vā taṁ sukhaṁ vā'ti.

Dukkhaṁ bhante.

Yam panāniccaṁ dukkhaṁ viparināma-dhammaṁ, kallaṁ nu taṁ samanupassituṁ,
etaṁ mama, esoham'asmi, eso me attā'ti.

No hetaṁ bhante.

Taṁ kiṁ maññatha bhikkhave, saññā niccā vā aniccā vā'ti.

Aniccā bhante.

Yam panāniccaṁ, dukkhaṁ vā taṁ sukhaṁ vā'ti.

Dukkhaṁ bhante.

Yam panāniccaṁ dukkhaṁ viparināma-dhammaṁ, kallaṁ nu taṁ samanupassituṁ,
etaṁ mama, esoham'asmi, eso me attā'ti.

No hetaṁ bhante.

Taṁ kiṁ maññatha bhikkhave, saṅkhārā niccā vā aniccā vā'ti.

Aniccā bhante.

Yam panāniccaṁ, dukkhaṁ vā taṁ sukhaṁ vā'ti.

Dukkhaṁ bhante.

Yam panāniccaṁ dukkhaṁ viparināma-dhammaṁ, kallaṁ nu taṁ samanupassituṁ,
etaṁ mama, esoham'asmi, eso me attā'ti.

No hetaṁ bhante.

\clearpage

\englishText
\markboth{\englishTitle}{\rightmark}

‘¿Qué opináis, monjes, es la cognición permanente o impermanente?’

‘Impermanente, venerable señor’

‘Y, lo que es impermanente, ¿es doloroso o placentero?’

‘Doloroso, venerable señor’

‘Y, lo que es impermanente, es doloroso y sujeto al cambio, ¿puede ser considerado de esta manera: ‘eso es mío, eso soy yo, eso es mi ser’?’

‘No, venerable señor’

‘Por eso, monjes, cualquier tipo de forma, sea del pasado, futuro o presente, interna o externa, vulgar o sutil, inferior o superior, lejana o cercana, toda forma debería ser vista tal como realmente es con la correcta sabiduría así: ‘eso no es mío, eso no soy yo, eso no es mi ser’.

‘Cualquier tipo de sensación, sea del pasado, futuro o presente, interna o externa, vulgar o sutil, inferior o superior, lejana o cercana, toda sensación debería ser vista tal como realmente es con la correcta sabiduría así: ‘eso no es mío, eso no soy yo, eso no es mi ser’.

‘Cualquier tipo de percepción, sea del pasado, futuro o presente, interna o externa, vulgar o sutil, inferior o superior, lejana o cercana, toda percepción debería ser vista tal como realmente es con la correcta sabiduría así: ‘eso no es mío, eso no soy yo, eso no es mi ser’.

‘Cualquier tipo de formaciones mentales, sea del pasado, futuro o presente, interna o externa, vulgar o sutil, inferior o superior, lejana o cercana, toda formación mental debería ser vista tal como realmente es con la correcta sabiduría así: ‘eso no es mío, eso no soy yo, eso no es mi ser’.



\clearpage

\paliText
\markboth{\paliTitle}{\rightmark}

Taṁ kiṁ maññatha bhikkhave, viññāṇaṁ niccaṁ vā aniccaṁ vā'ti.

Aniccaṁ bhante.

Yam panāniccaṁ, dukkhaṁ vā taṁ sukhaṁ vā'ti.

Dukkhaṁ bhante.

Yam panāniccaṁ dukkhaṁ viparināma-dhammaṁ, kallaṁ nu taṁ samanupassituṁ
etaṁ mama, esoham'asmi, eso me attā'ti.

No hetaṁ bhante.

[Tasmā tiha bhikkhave] yaṁ kiñci rūpaṁ atītānāgata-paccuppannaṁ ajjhattaṁ
vā bahiddhā vā oḷārikaṁ vā sukhumaṁ vā hīnaṁ vā paṇītaṁ vā yandūre
santike vā, sabbaṁ rūpaṁ netaṁ mama, nesoham'asmi, na me so attā'ti,
evametaṁ yathābhūtaṁ sammappaññāya daṭṭhabbaṁ.

Yā kāci vedanā atītānāgata-paccuppannā ajjhattā vā bahiddhā vā oḷārikā
vā sukhumā vā hīnā vā paṇītā vā yā dūre santike vā, sabbā vedanā netaṁ
mama, nesoham'asmi, na me so attā'ti, evametaṁ yathābhūtaṁ sammappaññāya
daṭṭhabbaṁ.

Yā kāci saññā atītānāgata-paccuppannā ajjhattā vā bahiddhā vā oḷārikā vā
sukhumā vā hīnā vā paṇītā vā yā dūre santike vā, sabbā saññā netaṁ mama,
nesoham'asmi, na me so attā'ti, evametaṁ yathābhūtaṁ sammappaññāya
daṭṭhabbaṁ.

Ye keci saṅkhārā atītānāgata-paccuppannā ajjhattā vā bahiddhā vā oḷārikā
vā sukhumā vā hīnā vā paṇītā vā ye dūre santike vā, sabbe saṅkhārā netaṁ
mama, nesoham'asmi, na me so attā'ti, evametaṁ yathābhūtaṁ sammappaññāya
daṭṭhabbaṁ.

\clearpage

\englishText
\markboth{\englishTitle}{\rightmark}

‘Cualquier tipo de conciencia, sea del pasado, futuro o presente, interna o externa, vulgar o sutil, inferior o superior, lejana o cercana, toda conciencia debería ser vista tal como realmente es con la correcta sabiduría así: ‘eso no es mío, eso no soy yo, eso no es mi ser’.

‘Viendo de esta manera, monjes, el instruido noble discípulo experimenta desencanto hacia la forma, desencanto hacia la sensación, desencanto hacia la percepción, desencanto hacia las formaciones mentales y desencanto hacia la conciencia. Experimentando desencanto, llega a estar desapasionado. Mediante el desapasionamiento [su mente] es liberada. Cuando se libera, llega al conocimiento: ‘esta es la liberación’, y comprende esto: ‘destruido está el nacimiento, la vida santa ha sido vivida, lo que había que hacer se ha hecho y ya no hay más futuros estados de existencia’.

Esto es lo que dijo el Excelso y los monjes, satisfechos, se deleitaron en las palabras del Excelso. Y mientras se pronunciaba este discurso las mentes de los monjes del grupo de los cinco fueron liberadas de las impurezas, a través del desapego.

Así termina el discurso de las características de Anatta.

\clearpage

\paliText
\markboth{\paliTitle}{\rightmark}

Yaṁ kiñci viññāṇaṁ atītānāgata-paccuppannaṁ ajjhattaṁ vā bahiddhā vā
oḷārikaṁ vā sukhumaṁ vā hīnaṁ vā paṇītaṁ vā yandūre santike vā, sabbaṁ
viññāṇaṁ netaṁ mama, nesoham'asmi, na me so attā'ti, evametaṁ yathābhūtaṁ
sammappaññāya daṭṭhabbaṁ.

[Evaṁ passaṁ bhikkhave] sutvā ariyasāvako rūpasmim pi nibbindati, vedanāya
pi nibbindati, saññāya pi nibbindati, saṅkhāresu pi nibbindati,
viññāṇasmim pi nibbindati, nibbindaṁ virajjati, virāgā vimuccati,
vimuttasmiṁ vimuttam iti ñāṇaṁ hoti, khīṇā jāti, vusitaṁ brahmacariyaṁ,
kataṁ karaṇīyaṁ, nāparaṁ itthattāyā'ti pajānātī'ti.

[Idam-avoca bhagavā.] Attamanā pañcavaggiyā bhikkhū bhagavato bhāsitaṁ
abhinanduṁ. Imasmiñca pana veyyākaraṇasmiṁ bhaññamāne pañcavaggiyānaṁ
bhikkhūnaṁ anupādāya āsavehi cittāni vimucciṁsū'ti.

Anattalakkhaṇa-suttaṁ niṭṭhitaṁ.

\chapterTocDelegatePageNumber
\chapter{El Sermón del fuego}

\setTocDelegatedPageNumber
\englishText
\renewcommand{\englishTitle}{El Sermón del fuego}

\begin{leader}
\soloinstr{Introducción}

Con su capacidad de entrenar al entrenable, el Buddha todo trascendente, orador lucido, maestro del conocimiento superior,
Él, quien expone a la gente el Dhamma y Vinaya que es apropiado y merece la pena, enseña en esta maravillosa parábola sobre el fuego, meditadores de capacidad suprema;
Él ha liberado aquellos que escuchan con la liberación que es últimamente completa, a través de investigación real, con sabiduría y atención.
Recitemos ahora este Sutta que describe las características de dukkha.



\end{leader}

Esto he escuchado. En una ocasión, el Excelso estaba en Gaya, en la Cabeza de Gaya, junto a un grupo de mil monjes. Estando allí, se dirigió a ellos con estas palabras:

‘Monjes, todo está ardiendo. ¿Y qué es este ‘todo’ que está ardiendo?’

‘El ojo está ardiendo, las formas están ardiendo, la conciencia del ojo está ardiendo, el contacto del ojo está ardiendo, también toda la sensación placentera o dolorosa, o la que no es ni placentera ni dolorosa dependiente del ojo como su condición indispensable, está ardiendo. ¿Ardiendo con qué? Ardiendo con el fuego de la pasión, con el fuego del odio, con el fuego de la ilusión; ardiendo con el nacimiento, el envejecimiento y la muerte, con las penas, lamentaciones y dolores, con angustia y desesperación, declaro yo.’


\enlargethispage{2\baselineskip}

‘El oído está ardiendo, los sonidos están ardiendo, la conciencia del oído está ardiendo, el contacto del oído está ardiendo, también toda la sensación placentera o dolorosa, o la que no es ni placentera ni dolorosa dependiente del ojo como su condición indispensable, está ardiendo. ¿Ardiendo con qué? Ardiendo con el fuego de la pasión, con el fuego del odio, con el fuego de la ilusión; ardiendo con el nacimiento, el envejecimiento y la muerte, con las penas, lamentaciones y dolores, con angustia y desesperación, declaro yo.’




\chapterTocSubIndentTrue
\chapter{Āditta-pariyāya Sutta}

\paliText
\renewcommand{\paliTitle}{Āditta-pariyāya Sutta}

\begin{leader}
\soloinstr{Introducción}

\begin{solotwochants}
Veneyyadamanopāye  & sabbaso pāramiṁ gato\\
Amoghavacano buddho & abhiññāyānusāsako\\
Ciṇṇānurūpato cāpi & dhammena vinayaṁ pajaṁ\\
Ciṇṇāggipāricariyānaṁ & sambojjhārahayoginaṁ\\
Yamādittapariyāyaṁ & desayanto manoharaṁ\\
Te sotāro vimocesi & asekkhāya vimuttiyā\\
Tathevopaparikkhāya & viññūṇaṁ sotumicchataṁ\\
Dukkhatālakkhaṇopāyaṁ & taṁ suttantaṁ bhaṇāma se\\
\end{solotwochants}
\end{leader}

[Evaṁ me sutaṁ]

Ekaṁ samayaṁ bhagavā gayāyaṁ viharati gayāsīse saddhiṁ bhikkhu-sahassena.
Tatra kho bhagavā bhikkhū āmantesi:

Sabbaṁ bhikkhave ādittaṁ. Kiñca bhikkhave sabbaṁ ādittaṁ.

Cakkhuṁ bhikkhave ādittaṁ, rūpā ādittā, cakkhuviññāṇaṁ ādittaṁ,
cakkhusamphasso āditto, yampidaṁ cakkhusamphassapaccayā uppajjati
vedayitaṁ sukhaṁ vā dukkhaṁ vā adukkhamasukhaṁ vā tam pi ādittaṁ. Kena
ādittaṁ. Ādittaṁ rāgagginā dosagginā mohagginā, ādittaṁ jātiyā
jarāmaraṇena sokehi paridevehi dukkhehi domanassehi upāyāsehi ādittan'ti
vadāmi.

Sotaṁ ādittaṁ, saddā ādittā, sotaviññāṇaṁ ādittaṁ, sotasamphasso āditto,
yampidaṁ sotasamphassapaccayā uppajjati vedayitaṁ sukhaṁ vā dukkhaṁ vā
adukkhamasukhaṁ vā tam pi ādittaṁ. Kena ādittaṁ. Ādittaṁ rāgagginā
dosagginā mohagginā, ādittaṁ jātiyā jarāmaraṇena sokehi paridevehi
dukkhehi domanassehi upāyāsehi ādittan'ti vadāmi.

\clearpage

\englishText
\markboth{\englishTitle}{\rightmark}
‘El olfato está ardiendo, los olores están ardiendo, la conciencia del olfato está ardiendo, el contacto del oído está ardiendo, también toda la sensación placentera o dolorosa, o la que no es ni placentera ni dolorosa dependiente del ojo como su condición indispensable, está ardiendo. ¿Ardiendo con qué? Ardiendo con el fuego de la pasión, con el fuego del odio, con el fuego de la ilusión; ardiendo con el nacimiento, el envejecimiento y la muerte, con las penas, lamentaciones y dolores, con angustia y desesperación, declaro yo.’

‘La lengua está ardiendo, los sabores están ardiendo, la conciencia de la lengua está ardiendo, el contacto de la lengua está ardiendo, también toda la sensación placentera o dolorosa, o la que no es ni placentera ni dolorosa dependiente de la lengua como su condición indispensable, está ardiendo. ¿Ardiendo con qué? Ardiendo con el fuego de la pasión, con el fuego del odio, con el fuego de la ilusión...’


‘El cuerpo está ardiendo, los objetos tangibles están ardiendo, la conciencia del cuerpo está ardiendo, el contacto del cuerpo está ardiendo, también toda la sensación placentera o dolorosa, o la que no es ni placentera ni dolorosa dependiente del ojo como su condición indispensable, está ardiendo. ¿Ardiendo con qué? Ardiendo con el fuego de la pasión, con el fuego del odio, con el fuego de la ilusión...’

‘La mente está ardiendo, las ideas están ardiendo, la conciencia de la mente está ardiendo, el contacto de la mente está ardiendo, también toda la sensación placentera o dolorosa, o la que no es ni placentera ni dolorosa dependiente del ojo como su condición indispensable, está ardiendo. ¿Ardiendo con qué? Ardiendo con el fuego de la pasión, con el fuego del odio, con el fuego de la ilusión; ardiendo con el nacimiento, el envejecimiento y la muerte, con las penas, lamentaciones y dolores, con angustia y desesperación, declaro yo.’
\enlargethispage{2\baselineskip}

‘Monjes, viendo esto, el bien instruido noble discípulo experimenta desencanto con el ojo, desencanto con las formas, desencanto con la conciencia del ojo, desencanto con el contacto del ojo y con toda sensación placentera o dolorosa, o cualquier sensación neutra dependiente del ojo como su condición indispensable. ’



\clearpage

\paliText
\markboth{\paliTitle}{\rightmark}

Ghānaṁ ādittaṁ, gandhā ādittā, ghānaviññāṇaṁ ādittaṁ, ghānasamphasso
āditto, yampidaṁ ghānasamphassapaccayā uppajjati vedayitaṁ sukhaṁ vā
dukkhaṁ vā adukkhamasukhaṁ vā tam pi ādittaṁ. Kena ādittaṁ. Ādittaṁ
rāgagginā dosagginā mohagginā, ādittaṁ jātiyā jarāmaraṇena sokehi
paridevehi dukkhehi domanassehi upāyāsehi ādittan'ti vadāmi.

Jivhā ādittā, rasā ādittā, jivhāviññāṇam ādittaṁ, jivhāsamphasso āditto,
yampidaṁ jivhāsamphassapaccayā uppajjati vedayitaṁ sukhaṁ vā dukkhaṁ vā
adukkhamasukhaṁ vā tam pi ādittaṁ. Kena ādittaṁ. Ādittaṁ rāgagginā
dosagginā mohagginā, ādittaṁ jātiyā jarāmaraṇena sokehi paridevehi
dukkhehi domanassehi upāyāsehi ādittan'ti vadāmi.

Kāyo āditto, phoṭṭhabbā ādittā, kāyaviññāṇaṁ ādittaṁ, kāyasamphasso
āditto, yampidaṁ kāyasamphassapaccayā uppajjati vedayitaṁ sukhaṁ vā
dukkhaṁ vā adukkhamasukhaṁ vā tam pi ādittaṁ. Kena ādittaṁ. Ādittaṁ
rāgagginā dosagginā mohagginā, ādittaṁ jātiyā jarāmaraṇena sokehi
paridevehi dukkhehi domanassehi upāyāsehi ādittan'ti vadāmi.

Mano āditto, dhammā ādittā, manoviññāṇaṁ ādittaṁ, manosamphasso āditto,
yampidaṁ manosamphassapaccayā uppajjati vedayitaṁ sukhaṁ vā dukkhaṁ vā
adukkhamasukhaṁ vā tam pi ādittaṁ. Kena ādittaṁ. Ādittaṁ rāgagginā
dosagginā mohagginā, ādittaṁ jātiyā jarāmaraṇena sokehi paridevehi
dukkhehi domanassehi upāyāsehi ādittan'ti vadāmi.

\enlargethispage{2\baselineskip}

[Evaṁ passaṁ bhikkhave] sutvā ariyasāvako cakkhusmiṁ pi nibbindati,
rūpesu pi nibbindati, cakkhuviññāṇe pi nibbindati, cakkhusamphassepi
nibbindati, yampidaṁ cakkhusamphassapaccayā uppajjati vedayitaṁ sukhaṁ
vā dukkhaṁ vā adukkhamasukhaṁ vā tasmiṁ pi nibbindati.

\clearpage

\englishText
\markboth{\englishTitle}{\rightmark}

‘Experimenta desencanto con el oído, desencanto con los sonidos, desencanto con la cognición del oído, desencanto con el contacto del oído y con toda sensación placentera o dolorosa, o cualquier sensación neutra dependiente del oído como su condición indispensable.’

‘Experimenta desencanto con el olfato, desencanto con los olores, desencanto con la cognición del olfato, desencanto con el contacto del olfato y con toda sensación placentera o dolorosa, o cualquier sensación neutra dependiente del olfato como su condición indispensable.’

‘Experimenta desencanto con la lengua, desencanto con los sabores, desencanto con la cognición de la lengua, desencanto con el contacto de la lengua y con toda sensación placentera o dolorosa, o cualquier sensación neutra dependiente de la lengua como su condición indispensable.’

‘Experimenta desencanto con el cuerpo, desencanto con los objetos tangibles, desencanto con la cognición del cuerpo, desencanto con el contacto del cuerpo y con toda sensación placentera o dolorosa, o cualquier sensación neutra dependiente del cuerpo como su condición indispensable.’

‘Experimenta desencanto con la mente, desencanto con las ideas, desencanto con la cognición mental, desencanto con el contacto mental y con toda sensación placentera o dolorosa, o cualquier sensación neutra dependiente de la mente como su condición indispensable.’

‘Y experimentando desencanto, se vuelve desapasionado. Mediante el desapasionamiento, su mente es liberada. Cuando es liberada, aparece en él el conocimiento: ‘Ésta es la liberación’. Entonces entiende que ‘el nacimiento está destruido, la vida santa ha sido vivida, se ha hecho lo que tenía que hacerse. No hay vuelta a cualquier otro estado de existencia’.

\enlargethispage{2\baselineskip}

Esto dijo el Excelso y los monjes, satisfechos, se deleitaron en las palabras del Excelso. Y durante este discurso, las mentes de estos mil monjes fueron completamente liberadas de las corrupciones a través del desapego.


Así termina el sermón del fuego.

\clearpage

\paliText
\markboth{\paliTitle}{\rightmark}

Sotasmiṁ pi nibbindati, saddesu pi nibbindati, sotaviññāṇe pi
nibbindati, sotasamphassepi nibbindati, yampidaṁ sotasamphassapaccayā
uppajjati vedayitaṁ sukhaṁ vā dukkhaṁ vā adukkhamasukhaṁ vā tasmiṁ pi
nibbindati.

Ghānasmiṁ pi nibbindati, gandhesu pi nibbindati, ghānaviññāṇe pi
nibbindati, ghānasamphassepi nibbindati, yampidaṁ ghānasamphassapaccayā
uppajjati vedayitaṁ sukhaṁ vā dukkhaṁ vā adukkhamasukhaṁ vā tasmiṁ pi
nibbindati.

Jivhāya pi nibbindati, rasesu pi nibbindati, jivhāviññāṇe pi nibbindati,
jivhāsamphassepi nibbindati, yampidaṁ jivhāsamphassapaccayā uppajjati
vedayitaṁ sukhaṁ vā dukkhaṁ vā adukkhamasukhaṁ vā tasmiṁ pi nibbindati.

Kāyasmiṁ pi nibbindati, phoṭṭhabbesu pi nibbindati, kāyaviññāṇe pi
nibbindati, kāyasamphassepi nibbindati, yampidaṁ kāyasamphassapaccayā
uppajjati vedayitaṁ sukhaṁ vā dukkhaṁ vā adukkhamasukhaṁ vā tasmiṁ pi
nibbindati.

Manasmiṁ pi nibbindati, dhammesu pi nibbindati, manoviññāṇe pi
nibbindati, manosamphassepi nibbindati, yampidaṁ manosamphassapaccayā
uppajjati vedayitaṁ sukhaṁ vā dukkhaṁ vā adukkhamasukhaṁ vā tasmiṁ pi
nibbindati.

Nibbindaṁ virajjati, virāgā vimuccati, vimuttasmiṁ, vimuttam iti ñāṇaṁ
hoti, khīṇā jāti, vusitaṁ brahmacariyaṁ, kataṁ karaṇīyaṁ, nāparaṁ
itthattāyā'ti pajānātī'ti.

\enlargethispage{\baselineskip}

[Idam-avoca bhagavā.] Attamanā te bhikkhū bhagavato bhāsitaṁ abhinanduṁ.
Imasmiñca pana veyyākaraṇasmiṁ bhaññamāne tassa bhikkhu-sahassassa
anupādāya āsavehi cittāni vimucciṁsū'ti.

Ādittapariyāya-suttaṁ niṭṭhitaṁ.

\resumeNormalText

% End of suttas.tex
