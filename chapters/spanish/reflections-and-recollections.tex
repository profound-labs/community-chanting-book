\chapter*[Compartir Mérito]{Versos de Dedicación de Mérito}

\delegateSetUseNext

\begin{leader}
  [Ha꜓nda mayaṁ uddissanādhiṭṭhāna-gāthā꜓yo b꜕haṇāmase]
\end{leader}

\firstline{Iminā puññakammena upajjhāyā guṇuttarā}

[Iminā puñña꜕kammena] u꜕pajjhāyā gu꜕ṇutta꜕rā\\
Ācariyūpa꜕kārā ca꜕ mātāpitā ca꜕ ñāta꜕kā\\
Suriyo candimā rājā gu꜕ṇavantā na꜕rāpi꜕ ca꜕\\
Brahma-mārā ca꜕ indā ca꜕ loka꜕pālā ca꜕ deva꜕tā\\
Yamo mittā ma꜕nussā ca majjhattā veri꜕kāpi꜕ ca꜕\\
Sa꜕bbe sattā sukhī hontu puññāni pa꜕ka꜕tāni꜕ me\\
Sukhañca tividhaṁ dentu꜕ khippaṁ pāpetha꜕ voma꜕taṁ\\
Iminā puññakammena iminā uddi꜕ssena꜕ ca꜕\\
Khipp'āhaṁ su꜕la꜕bhe ceva taṇhūpādāna꜕-cheda꜕naṁ\\
Ye santāne hīnā dhammā yāva꜕ nibbāna꜕to ma꜕maṁ\\
Nassantu sabba꜕dā yeva yattha꜕ jāto bha꜕ve bha꜕ve\\
Ujucittaṁ sa꜕ti꜕paññā sallekho vi꜕ri꜕yamhinā\\
Mārā labhantu nokāsaṁ kātuñca vi꜕ri꜕yes꜕u me\\
Buddhādhipa꜕va꜕ro nātho dhammo nātho va꜕rutta꜕mo\\
Nātho pacceka꜕buddho ca꜕ saṅgho nāthotta꜕ro ma꜕maṁ\\
Tesottamānubhāvena mārokāsaṁ la꜕bhantu꜕ mā

\chapter[Compartir Mérito]{Versos de Dedicación de Mérito}

\enlargethispage{2\baselineskip}

\begin{leader}
  [Cantemos ahora los versos para dedicar mérito.]
\end{leader}

\firstline{A través del bien que resulta de mi práctica}

A través del bien que resulta de mi práctica,\\
Que mis maestros y guías espirituales de gran virtud,\\
Mi madre, mi padre y mis familiares,\\
El Sol y la Luna, y todos los líderes virtuosos del mundo,\\
Que los Dioses más elevados y las fuerzas del mal,\\
Seres celestiales, espíritus guardianes de la Tierra y el Señor de la Muerte,\\
Aquellos que son amigables, indiferentes u hostiles,\\
Que todos los seres reciban las bendiciones de mi vida.\\
Que brevemente lleguen a la Triple Bendición, y superen la muerte.\\
A través del mérito que resulta de mi práctica,\\
Y con esta dedicación,\\
Que todos los deseos y apegos rápidamente cesen,\\
Así como los estados prejudiciales de la mente.\\
Hasta realizar Nibbāna,\\
En cualquier tipo de nacimiento, que yo tenga una mente justa,\\
Con consciencia y sabiduría, austeridad y vigor.\\
Que las fuerzas ilusorias no controlen,\\
ni enflaquezcan mi decisión.

El Buddha es mi excelente refugio,\\
Insuperable es la protección del Dhamma,\\
El Buddha solitario es mi Noble ejemplo,\\
La Saṅgha es mi mayor soporte.

Que a través de esta supremacía\\
Desaparezcan la oscuridad y la ilusión.

\chapter[Beneficio de la Dádiva]{Versos sobre el Beneficio de la Dádiva}

\firstline{Puññass'idāni katassa yān'aññāni katāni me}

\begin{leader}
	[Ha꜓nda mayaṁ sa꜕bba-patti-dāna-gāthā꜓yo bha꜕ṇāmase]
\end{leader}

%\suttaref{Trad.}%
Puññass'i꜕dāni꜓ ka꜕ta꜕ssa yān'aññāni꜓ ka꜕tāni꜓ me\\
Tesa꜓ñca꜕ bhāgi꜕no ho꜓ntu꜕ sa꜕ttānantā꜕ppa꜕māṇa꜓kā

\begin{english}
	Que todos los seres,\\
	Sin límite y sin fin,\\
	Tomen parte de todo el mérito,\\
	de mis buenas acciones.
\end{english}

Ye pi꜕yā gu꜓ṇavantā ca꜕ mayhaṁ mātā-pi꜕tāda꜓yo\\
Diṭṭhā me cāpy꜕adiṭṭhā vā aññe majjh꜓att꜕a-veri꜓no

\begin{english}
	Aquellos queridos y llenos de bondad,\\
	mis amados madre y padre,\\
	Seres vistos y nunca vistos,\\
	hostiles o indiferentes,
\end{english}

Sa꜕ttā tiṭṭha꜓nti꜕ lokasmiṁ te-bhummā ca꜕tu꜕-yoni꜓kā\\
Pañc'eka꜕-ca꜕tu꜕-vokārā sa꜓ṁsa꜕rantā bh꜕avābha꜕ve

\begin{english}
	Seres establecidos en el mundo,\\
	De tres planos y cuatro formas de nacer,\\
	Con cinco agregados o uno o cuatro,\\
	Vagando de mundo en mundo,
\end{english}

Ñātaṁ ye pa꜕tti꜓-dānam-me a꜕nu꜓modantu꜕ te sa꜕yaṁ\\
Ye c'imaṁ nappa꜕jānanti devā tesa꜓ṁ ni꜕veda꜕yuṁ

\clearpage

\begin{english}
	Aquellos que mi acto de dedicación conocen,\\
	Que todos se alegren con él,\\
	y para aquellos que no lo saben,\\
	que los devas se lo hagan saber.
\end{english}

Ma꜓yā dinnāna-puññānaṁ a꜕nu꜓moda꜕na-he꜓tu꜕nā\\
Sa꜕bbe sa꜕ttā sa꜕dā ho꜓ntu꜕ a꜕verā su꜕kh꜕a-jīvi꜓no\\
Kh꜓ema꜓ppa꜕dañca꜕ pa꜕ppontu꜕ tesā꜓sā꜓ si꜕jjha꜕taṁ su꜕bhā

\begin{english}
	Al regocijarse en mi compartir,\\
	Qué todos los seres vivan feliz,\\
	De hostilidad sean libres,\\
	Sus buenos deseos se puedan cumplir,\\
	Y alcancen seguridad todos.
\end{english}

\chapter[Mettā Sutta]{Mettā Sutta}

\delegateSetUseNext

\firstline{Karaṇīyam-attha-kusalena}

\begin{leader}
  [Ha꜓nda mayaṁ metta-sutta-gāthā꜓yo bha꜕ṇāmase]
\end{leader}

[Karaṇīyam-attha-kusalena]\\
Yan-taṁ santaṁ padaṁ abhisamecca\\
Sakko ujū ca suhujū ca\\
Suvaco c'assa mudu anatimānī

Santussako ca subharo ca\\
Appakicco ca sallahuka-vutti\\
Sant'indriyo ca nipako ca\\
Appagabbho kulesu ananugiddho

Na ca khuddaṁ samācare kiñci\\
Yena viññū pare upavadeyyuṁ\\
Sukhino vā khemino hontu\\
Sabbe sattā bhavantu sukhit'attā

Ye keci pāṇa-bhūt'atthi\\
Tasā vā thāvarā vā anavasesā\\
Dīghā vā ye mahantā vā\\
Majjhimā rassakā aṇuka-thūlā

Diṭṭhā vā ye ca adiṭṭhā\\
Ye ca dūre vasanti avidūre\\
Bhūtā vā sambhavesī vā\\
Sabbe sattā bhavantu sukhit'attā

\chapter*[Mettā Sutta]{Mettā Sutta}

\firstline{Esto es lo que se debe hacer }

\begin{leader}
  [Cantemos ahora las palabras de Buddha \\ sobre el Amor y la Compasión.]
\end{leader}

Esto es lo que se debe hacer\\
para cultivar la bondad\\
y seguir la vía de la paz:\\
Ser capaz y ser honesto,\\
franco y gentil al hablar.\\
Humilde y no arrogante,\\
contento, fácilmente satisfecho,\\
liberado de deberes y frugal en su camino.

Pacífico y sereno, sabio e inteligente,\\
sin orgullo, sin exigencias por naturaleza.\\
Que él no haga nada que\\
los sabios puedan reprender.\\
Deseando: En alegría y seguridad,\\
Que todos los seres sean felices.\\
Cualesquiera que sean los seres vivos,\\
débiles, fuertes, sin excepción.\\
Del más grande al más pequeño,\\
visibles o invisibles,\\
ya sea cerca o sea lejos,\\
nacidos o por nacer: \\
¡Que todos los seres sean felices!

\clearpage

Na paro paraṁ nikubbetha\\
Nātimaññetha katthaci naṁ kiñci\\
Byārosanā paṭighasaññā\\
Nāññam-aññassa dukkham-iccheyya

Mātā yathā niyaṁ puttaṁ\\
Āyusā eka-puttam-anurakkhe\\
Evam'pi sabba-bhūtesu\\
Mānasam-bhāvaye aparimāṇaṁ

Mettañca sabba-lokasmiṁ\\
Mānasam-bhāvaye aparimāṇaṁ\\
Uddhaṁ adho ca tiriyañca\\
Asambādhaṁ averaṁ asapattaṁ

Tiṭṭhañ-caraṁ nisinno vā\\
Sayāno vā yāvat'assa vigata-middho\\
Etaṁ satiṁ adhiṭṭheyya\\
Brahmam-etaṁ vihāraṁ idham-āhu

Diṭṭhiñca anupagamma\\
Sīlavā dassanena sampanno\\
Kāmesu vineyya gedhaṁ\\
Na hi jātu gabbha-seyyaṁ punaretī'ti

\clearpage

Que nadie engañe a nadie,\\
ni desprecie a otro en ninguna condición.\\
Que nadie, por ira o mala fe,\\
desee el mal a nadie.

Igual que una madre protege a su hijo,\\
con su vida, su único hijo.\\
Así, de corazón infinito,\\
se debería estimar todo ser vivo;\\
irradiando ternura por todo el mundo:\\
Arriba al más alto cielo,\\
y abajo hasta el abismo;\\
radiante y sin límites,\\
libre de odio y mala fe.\\
Sea parado o andando,\\
sentado o reclinado,\\
libre de torpor,\\
Esta es una memoria a mantener,\\
la verdadera casa divina.\\

Puro de corazón, con claridad de visión,\\
sin insistir en ideas fijas,\\
liberado de los deseos sensuales,\\
no volverá a nacer en este mundo.

\chapter[Once Beneficios]{Once Beneficios de la Práctica de Mettā}

\emph{Mettānisaṁsa Sutta, AN 11.15}

% https://suttacentral.net/an11.15/pli/ms

\begin{leader}
  [Handa mayaṁ mettānisaṁsa-suttaṁ bhaṇāmase]
\end{leader}

Mettāya, bhikkhave, cetovimuttiyā āsevitāya bhāvitāya bahulīkatāya yānīkatāya vatthukatāya anuṭṭhitāya paricitāya susamāraddhāya ekādasānisaṁsā pāṭikaṅkhā. Katame ekādasa?

\begin{english}
  Monjes, hay once beneficios que se pueden esperar como resultado de la liberación del corazón por la práctica de Mettā, conociéndola, cultivándola, desarrollándola, teniéndola como guía, estimándola, siguiéndola, consolidándola e implementándola.\\
  ¿Cuales son los once?
\end{english}

Sukhaṁ supati.\\
Sukhaṁ paṭibujjhati.\\
Na pāpakaṁ supinaṁ passati.\\
Manussānaṁ piyo hoti.\\
Amanussānaṁ piyo hoti.\\
Devatā rakkhanti.\\
Nāssa aggi vā visaṁ vā satthaṁ vā kamati.\\
Tuvaṭaṁ cittaṁ samādhiyati.\\
Mukhavaṇṇo vippasīdati.\\
Asammūḷho kālaṁ karoti.\\
Uttari appaṭivijjhanto brahmalokūpago hoti.

\clearpage

\begin{english}
  Dormir bien.\\
  Despertar contento.\\
  No tener pesadillas.\\
  Ser querido por los seres humanos.\\
  Ser querido por los seres no humanos.\\
  Ser protegido por los seres celestiales.\\
  No verse afectado ni por fuego, ni por veneno, ni por armas.\\
  Poder concentrar la mente rápidamente.\\
  Tener un rostro sereno.\\
  Morir sin confusión alguna.\\
  En caso de no obtener un estado superior de iluminación, renacer en el mundo de Brahma.
\end{english}

Mettāya, bhikkhave, cetovimuttiyā āsevitāya bhāvitāya bahulīkatāya yānīkatāya vatthukatāya anuṭṭhitāya paricitāya susamāraddhāya ime ekādasānisaṁsā pāṭikaṅkhā'ti.

\begin{english}
  Monjes, estos once beneficios se pueden esperar como resultado de la liberación del corazón por la práctica de Mettā, conociéndola, cultivándola, desarrollándola, teniéndola como guía, estimándola, siguiéndola, consolidándola e implementándola.
\end{english}

\chapter[Irradiando los Estados Divinos]{Irradiando los Estados Divinos}

\delegateSetUseNext

\firstline{Mettā-sahagatena}

\begin{leader}
  [Ha꜓nda mayaṁ caturappamaññā-obhāsanaṁ karomase]
\end{leader}

[Mettā-sa꜕ha꜕ga꜕tena] cetasā ekaṁ disaṁ pha꜕ri꜕tv꜕ā viha꜕ra꜕ti\\
Ta꜕thā dutiyaṁ ta꜕thā tatiyaṁ ta꜕thā ca꜕tutthaṁ\\
Iti uddhamadho tiriyaṁ sabba꜕dhi꜕ sabbatta꜕tāya\\
Sabbāvantaṁ lokaṁ mettā-sa꜕ha꜕ga꜕tena cetasā\\
Vipulena mahagga꜕tena appa꜕māṇena a꜕verena a꜕byāpajjhena pha꜕ri꜕tv꜕ā viha꜕ra꜕ti

\enlargethispage{3\baselineskip}

Karuṇā-sa꜕ha꜕ga꜕tena cetasā ekaṁ disaṁ pha꜕ri꜕tv꜕ā viha꜕ra꜕ti\\
Ta꜕thā dutiyaṁ ta꜕thā tatiyaṁ ta꜕thā ca꜕tutthaṁ\\
Iti uddhamadho tiriyaṁ sabba꜕dhi꜕ sabbatta꜕tāya\\
Sabbāvantaṁ lokaṁ ka꜕ru꜕ṇā-sa꜕ha꜕ga꜕tena cetasā\\
Vipulena mahagga꜕tena appa꜕māṇena a꜕verena a꜕byāpajjhena pha꜕ri꜕tv꜕ā viha꜕ra꜕ti

Muditā-sa꜕ha꜕ga꜕tena cetasā ekaṁ disaṁ pha꜕ri꜕tv꜕ā viha꜕ra꜕ti\\
Ta꜕thā dutiyaṁ ta꜕thā tatiyaṁ ta꜕thā ca꜕tutthaṁ\\
Iti uddhamadho tiriyaṁ sabba꜕dhi꜕ sabbatta꜕tāya\\
Sabbāvantaṁ lokaṁ mu꜕di꜕tā-sa꜕ha꜕ga꜕tena cetasā\\
Vipulena mahagga꜕tena appa꜕māṇena a꜕verena a꜕byāpajjhena pha꜕ri꜕tv꜕ā viha꜕ra꜕ti

Upekkhā-saha꜕ga꜕te꜕na cetasā ekaṁ disaṁ pha꜕ri꜕tv꜕ā viha꜕ra꜕ti\\
Ta꜕thā dutiyaṁ ta꜕thā tatiyaṁ ta꜕thā ca꜕tutthaṁ\\
Iti uddhamadho tiriyaṁ sabba꜕dhi꜕ sabbatta꜕tāya\\
Sabbāvantaṁ lokaṁ u꜕pe꜕kkhā-sa꜕ha꜕ga꜕tena cetasā\\
Vipulena mahagga꜕tena appa꜕māṇena a꜕verena a꜕byāpajjhena pha꜕ri꜕tv꜕ā viha꜕ra꜕tī'ti

\chapter*[Irradiando los Estados  Divinos]{Irradiando los Estados Divinos}

\enlargethispage{\baselineskip}

\firstline{Llenaré un cuarto}

\begin{leader}
  \vspace*{-\baselineskip}
  \mbox{[Irradiemos ahora las Cuatro Cualidades Inmensurables.]}
\end{leader}

[Llenaré un cuarto] del mundo con un corazón de amistad;\\
Así el segundo, así el tercero, y así el cuarto;\\
Arriba, abajo, alrededor, en todas direcciones;\\
Tanto para todos, como para mí.\\
Llenaré el mundo entero con un corazón de amistad;\\ abundante, inmensurable, excelso, sin hostilidad y sin mala fe.

[Llenaré un cuarto] del mundo con un corazón de compasión;\\
Así el segundo, así el tercero, y así el cuarto;\\
Arriba, abajo, alrededor, en todas direcciones;\\
Tanto para todos, como para mí.\\
Llenaré el mundo entero con un corazón de compasión;\\ abundante, inmensurable, excelso, sin hostilidad y sin mala fe.

[Llenaré un cuarto] del mundo con un corazón empático;\\
Así el segundo, así el tercero, y así el cuarto;\\
Arriba, abajo, alrededor, en todas direcciones;\\
Tanto para todos, como para mí.\\
Llenaré el mundo entero con un corazón empático;\\ abundante, inmensurable, excelso, sin hostilidad y sin mala fe.

[Llenaré un cuarto] del mundo con un corazón ecuánime;\\
Así el segundo, así el tercero, y así el cuarto;\\
Arriba, abajo, alrededor, en todas direcciones;\\
Tanto para todos, como para mí.\\
Llenaré el mundo entero con un corazón ecuánime;\\ abundante, inmensurable, excelso, sin hostilidad y sin mala fe.


\clearpage

\chapter{Las Supremas Bendiciones}

\firstline{Así he oído, que el Excelso}

\begin{leader}
  [Cantemos ahora los versos sobre las Supremas Bendiciones.]
\end{leader}

[Así he oído, que el Excelso]\\
Se encontraba en Savatthi,\\
En el Bosque de Jeta residiendo\\
En el Parque de Anāthapiṇḍika.

Entonces en la oscuridad de la noche, una deva radiante\\
Todo el Parque de Jeta iluminó.\\
Se inclinó venerando al Excelso\\
Dijo, después de ponerse en pie:

`Los Devas desean la felicidad\\
Y buscan Paz continuamente.\\
Lo mismo, de los humanos es verdad.\\
Así, ¿cuáles son las supremas bendiciones?'

`Evitar a los necios,\\
Asociarse con los sabios,\\
Y honrar lo que debe ser honrado.\\
Estas son las supremas bendiciones.

`Vivir en lugares adecuados,\\
Con los frutos de méritos pasados,\\
Guiado por determinación correcta.\\
Estas son las supremas bendiciones.

\clearpage

`Competente en estudios y artes,\\
Con disciplina entrenada bien,\\
Y un hablar agradable.\\
Estas son las supremas bendiciones.

`Cuidar de madre y padre,\\
Sustentar a la familia,\\
Actuando sin obstruir.\\
Estas son las supremas bendiciones.

`Generosidad y una vida de Dhamma,\\
Ayudar a amigos y familiares,\\
Con actos irreprochables.\\
Estas son las supremas bendiciones.

`Refrenarse, eludiendo el mal,\\
Evitar intoxicantes que destruyen,\\
Y diligente en todas las cosas.\\
Estas son las supremas bendiciones.

`Respeto y humildad,\\
Satisfacción y gratitud,\\
Escuchar el Dhamma a su propio tiempo.\\
Estas son las supremas bendiciones.

`Paciencia y ser fácil de amonestar,\\
Encontrar venerables renunciantes,\\
Compartir apropiadamente el Dhamma.\\
Estas son las supremas bendiciones.

\clearpage

`Dedicarse con ardor a la Vida Santa,\\
Las Nobles Verdades entender\\
Y realizar el Nibbāna.\\
Estas son las supremas bendiciones.

`En contacto con el mundo,\\
El corazón no se agita,\\
sin pena, sin mancha, seguro.\\
Estas son las supremas bendiciones.

`Aquellos que siguen este camino,\\
son victoriosos donde quiera que van,\\
Y cualquier lugar es seguro.\\
Estas son las supremas bendiciones'.

\chapter*[Bienestar Universal]{Reflexión sobre el Bienestar Universal}

\delegateSetUseNext

\firstline{Ahaṁ sukhito homi}

\begin{leader}
  [Ha꜓nda mayam mettāpharaṇaṁ ka꜕romase]
\end{leader}

[Aha꜓ṁ sukhito ho꜓mi]\\
Niddukkho ho꜓mi\\
A꜕vero ho꜓mi\\
A꜕byāpajjho ho꜓mi\\
A꜕nīgho ho꜓mi\\
Sukhī꜓ attānaṁ pa꜕riha꜓rāmi

Sa꜕bbe sa꜕ttā sukhitā ho꜓ntu\\
Sa꜕bbe sa꜕ttā averā ho꜓ntu\\
Sa꜕bbe sa꜕ttā abyāpajjhā ho꜓ntu\\
Sa꜕bbe sa꜕ttā anīghā ho꜓ntu\\
Sa꜕bbe sa꜕ttā sukhī꜓ a꜕ttānaṁ pa꜕riha꜓rantu

Sa꜕bbe sa꜕ttā sabbadukkhā pamucca꜓ntu

Sa꜕bbe sa꜕ttā laddha-sa꜓mpa꜕tti꜓to mā vigaccha꜓ntu

Sa꜕bbe sa꜕ttā kammassa꜕kā kamma꜓dāyādā kamma꜓yonī\\
\vin kamma꜓bandhū kammapa꜕ṭisa꜓ra꜕ṇā\\
Yaṁ kammaṁ ka꜕rissa꜓nti\\
Kalyāṇaṁ vā pāpa꜕kaṁ vā\\
Tassa꜕ dāyādā bha꜕vissa꜓nti

\chapter[Bienestar Universal]{Reflexión sobre el Bienestar Universal}

\firstline{Que yo mantenga bienestar}

\begin{leader}
  [Cantemos ahora la Reflexión sobre el Bienestar Universal.]
\end{leader}

[Pueda yo vivir feliz,]\\
Libre de aflicción,\\
Libre de hostilidad,\\
Libre de mala fe,\\
Libre de ansiedad,\\
Y que mantenga mi felicidad.

Que todos los seres vivan felices,\\
Libres de hostilidad,\\
Libres de mala fe,\\
Libres de ansiedad,\\
Y que mantengan su felicidad.

Que todos los seres se liberen de tod
o dukkha.

Y que no pierdan la buena fortuna que alcanzaron.

Cuando con intención actúan,\\
Todos los seres son dueños de sus actos, \\ 
y sus resultados heredarán.\\
Su futuro nace de tales actos, \\
compañero de tales actos,\\
Y sus resultados su hogar serán.
\enlargethispage{3\baselineskip}

Todas los actos con intención,\\
Sean buenos o malos ---\\
De estos actos, ellos los herederos serán.

\chapter[Cuatro Requisitos]{Reflexión sobre los Cuatro Requisitos}

\firstline{Paṭisaṅkhā yoniso}

\begin{leader}
  [Ha꜓nda mayaṁ taṅkhaṇika-paccave꜕kkhaṇa-pāṭhaṁ bhaṇāmase]
\end{leader}

[Paṭisaṅkhā] yoniso cīva꜕raṁ pa꜕ṭise꜓vāmi, \pause\\
yāvadeva sī꜓tassa꜕ pa꜕ṭighātāya, \pause\ uṇhassa pa꜕ṭighātāya, \pause\\
ḍaṁsa-maka꜕sa꜕-vātāta꜕pa꜕-siriṁsapa-samphassānaṁ pa꜕ṭighātāya, \pause\\
yāvadeva hiri꜓kopina-pa꜕ṭicchāda꜕natthaṁ

\begin{english}
  Reflexionando sabiamente \pause\ yo uso la túnica: \pause\ solamente por modestia, \pause\
  para evitar el calor, \pause\ el frío, \pause\ las moscas, \pause\ mosquitos,
  \pause\ bichos que se arrastran, \pause\ el viento y las cosas que queman.
\end{english}

[Paṭisaṅkhā] yoniso piṇḍa꜕pātaṁ pa꜕ṭise꜓vāmi, \pause\\
neva da꜕vāya, na ma꜕dāya, na maṇḍa꜕nāya, na꜕ vi꜓bhūsa꜕nāya, \pause\\
yāvadeva i꜓massa꜕ kāyassa꜕ ṭhi꜕tiyā, \pause\ yāpa꜕nāya, vihiṁsū꜕para꜓ti꜕yā, \pause\\
brahmaca꜕ri꜓yānugga꜕hāya, \pause\ iti purāṇañca꜕ veda꜓naṁ pa꜕ṭiha꜓ṅkhāmi,
navañca꜕ veda꜓naṁ na uppādessāmi, \pause\ yātrā ca꜕ me bhavissati a꜕navajjatā
ca꜕ phāsuvihāro cā'ti

\begin{english}
  Reflexionando sabiamente \pause\ yo uso la comida de mendicidad: \pause\ no por
  diversión, \pause\ no por placer, \pause\ no para engordar, \pause\ no para
  embellecerme, \pause\ pero solamente para sostener y nutrir este cuerpo, \pause\
  para mantenerlo saludable, \pause\ para ayudar a la Vida Santa. \pause\ Pensando
  de esta forma: \pause\ ‘Saciaré el hambre sin comer demasiado, \pause\ para continuar viviendo sereno y sin remordimientos.’ 
\end{english}

\clearpage

[Paṭisaṅkhā] yoniso senāsa꜕naṁ pa꜕ṭise꜓vāmi, \pause\\
yāvadeva sī꜓tassa꜕ pa꜕ṭighātāya, \pause\ uṇhassa pa꜕ṭighātāya, \pause\\
ḍaṁsa-maka꜕sa꜕-vātāta꜕pa꜕-siriṁsapa-samphassānaṁ pa꜕ṭighātāya, \pause\\
yāvadeva utupa꜕rissaya vi꜕nodanaṁ \pause\ pa꜕ṭisa꜓llānārāmatthaṁ

\begin{english}
  Reflexionando sabiamente \pause\ uso el alojamiento: \pause\ solamente para evitar el
  frío, \pause\ el calor, \pause\ las moscas, \pause\ mosquitos, \pause\ bichos
  que se arrastran, \pause\ el viento y las cosas que queman. \pause\ Solamente para
  protegerme de los peligros de la naturaleza \pause\ y vivir en recogimiento.
\end{english}

[Paṭisaṅkhā] yoniso gi꜕lāna-pacca꜕ya꜕-bhesajja-pa꜕rikkhāraṁ\\
pa꜕ṭise꜓vāmi, \pause\ yāvadeva uppa꜓nnānaṁ veyyābādhi꜕kānaṁ veda꜕nānaṁ
pa꜕ṭighātāya, \pause\ a꜕byāpajjha-pa꜕ramatāyā'ti

\begin{english}
  Reflexionando sabiamente \pause\ hago uso de las medicinas: \pause\ solamente para aliviar los dolores que surjan,
  \pause\ para permanecer lo más posible libre de dolencias.
\end{english}

\chapter[Treinta y dos Partes]{Reflexión sobre las Treinta y dos Partes}

\firstline{Ayaṁ kho me kāyo}

\begin{leader}
  [Ha꜓nda mayaṁ dvattiṁsākāra-pāṭhaṁ bhaṇāmase]
\end{leader}

[Ayaṁ kho] me kāyo uddhaṁ pāda꜕ta꜕lā adho kesamatthakā\\
ta꜕ca꜕pa꜕ri꜕yanto pūro nānappa꜕kārassa꜕ a꜕su꜕ci꜕no

\begin{english}
  Esto, que es mi cuerpo, de la planta de los pies para arriba, y de la coronilla para abajo, es un saco de piel cerrado, lleno de cosas sucias.
\end{english}

Atthi imasmi꜔꜒ṁ kāye

\begin{english}
  En este cuerpo existen:
\end{english}

{\centering
\setArrayStrech{1}

\begin{tabular}{ r l }
kesā꜔꜒           & \tr{cabello} \\
lomā            & \tr{bello} \\
nakhā꜔꜒          & \tr{uñas} \\
dantā           & \tr{dientes} \\
taco            & \tr{piel} \\
maṁsa꜔꜒ṁ        & \tr{carne} \\
nahā꜔꜒rū         & \tr{tendones} \\
aṭṭhī꜔꜒           & \tr{huesos} \\
aṭṭhimiñjaṁ     & \tr{médula ósea} \\
vakkaṁ          & \tr{riñones} \\
hadayaṁ         & \tr{corazón} \\
yakanaṁ         & \tr{hígado} \\
kilomakaṁ       & \tr{membranas} \\
pihakaṁ         & \tr{bazo} \\
papphā꜔꜒sa꜔꜒ṁ    & \tr{pulmones} \\
\end{tabular}

\clearpage

\begin{tabular}{ r l }
antaṁ           & \tr{intestino grueso} \\
antaguṇaṁ       & \tr{intestino delgado} \\
udariyaṁ        & \tr{comida no digerida} \\
karīsa꜔꜒ṁ        & \tr{excremento} \\
pittaṁ          & \tr{bilis} \\
se꜔꜒mha꜔꜒ṁ       & \tr{flema} \\
pubbo           & \tr{pus} \\
lohitaṁ         & \tr{sangre} \\
se꜔꜒do           & \tr{sudor} \\
medo            & \tr{grasa} \\
assu            & \tr{lágrimas} \\
vasā꜔꜒           & \tr{sebo} \\
khe꜔꜒ḷo          & \tr{saliva} \\
si꜔꜒ṅghāṇikā    & \tr{moco} \\
lasikā          & \tr{lubricante de las articulaciones} \\
muttaṁ          & \tr{orina} \\
matthaluṅgan’ti & \tr{cerebro} \\
\end{tabular}

\restoreArrayStretch
}

Evam-ayaṁ me kāyo uddhaṁ pāda꜕ta꜕lā adho kesamatthakā\\
ta꜕ca꜕pa꜕ri꜕yanto pūro nānappa꜕kārassa꜕ a꜕su꜕ci꜕no

\begin{english}
  Así, esto que es mi cuerpo, de la planta de los pies para arriba, y de la coronilla para abajo, es un saco de piel cerrado, lleno de cosas sucias.
\end{english}

\clearpage


\chapter[Cualidades Repugnantes de nuestros Requisitos]{Reflexión sobre las Cualidades Repugnantes de nuestros Requisitos}

\firstline{Yathā paccayaṁ pavattamānaṁ}

\begin{leader}
	[Ha꜓nda mayaṁ dhātu-paṭikūla-paccavekkhaṇa-pāṭhaṁ bhaṇāmase]
\end{leader}

%\suttaref{Trad.}%
[Yathā꜓ pa꜕ccayaṁ] pava꜓tt꜕amānaṁ dhātu꜕-ma꜓tta꜕m-ev'etaṁ

\trline{Compuestas de elementos según causas y condiciones}

Yad i꜓daṁ cī꜓varaṁ ta꜕d upa꜕bhuñja꜓ko c꜕a pu꜕gga꜕lo

\trline{Son estas túnicas y la persona que se las pone;}

Dhātu-ma꜓tta꜕ko

\trline{Tan solo elementos,}

Ni꜓ssa꜕tto

\trline{No un ser,}

Ni꜓jjīvo

\trline{Sin alma}

Su꜓ñño

\trline{Y vacío de esencia.}

S꜕abbāni pa꜕na imāni cī꜓varāni a꜕jigu꜓ccha꜕nīyāni

\trline{Ninguna de estas túnicas es de por sí repugnante}

\clearpage

Imaṁ pūti꜓-kāyaṁ pa꜕tvā

\trline{Pero al tocar este cuerpo podrido}

A꜕tiviya jigu꜓ccha꜕nīyāni jāyanti

\trline{Se vuelven realmente asquerosas.}

Yathā꜓ pa꜕ccayaṁ pava꜓tt꜕amānaṁ dhātu꜕-ma꜓tta꜕m-ev'etaṁ

\trline{Compuesta de elementos según causas y condiciones}

Yad i꜓daṁ piṇḍa꜓pāto ta꜕d upa꜕bhuñja꜓ko c꜕a pu꜕gga꜕lo

\trline{Es esta comida de mendicidad y la persona que la consume;}

Dhātu-ma꜓tta꜕ko

\trline{Tan solo elementos,}

Ni꜓ssa꜕tto

\trline{No un ser}

Ni꜓jjīvo

\trline{Sin alma}

Su꜓ñño

\trline{Y vacío de esencia.}

S꜕abbo pa꜕nāyaṁ piṇḍa꜓pāto a꜕jigu꜓ccha꜕nīyo

\trline{Ninguno de estos alimentos es de por sí repugnante}

\clearpage

Imaṁ pūti꜓-kāyaṁ pa꜕tvā

\trline{Pero al tocar este cuerpo podrido}

A꜕tiviya jigu꜓ccha꜕nīyo jāyati

\trline{Se vuelven realmente asquerosos.}

Yathā꜓ pa꜕ccayaṁ pava꜓tt꜕amānaṁ dhātu꜕-ma꜓tta꜕m-ev'etaṁ

\trline{Compuesto de elementos según causas y condiciones}

Yad i꜓daṁ senā꜓sanaṁ ta꜕d upa꜕bhuñja꜓ko c꜕a pu꜕gga꜕lo

\trline{Es este alojamiento y la persona que lo habita;}

Dhātu-ma꜓tta꜕ko

\trline{Tan solo elementos,}

Ni꜓ssa꜕tto

\trline{No un ser}

Ni꜓jjīvo

\trline{Sin alma}

Su꜓ñño

\trline{Y vacío de esencia.}

S꜕abbāni pa꜕na imāni senā꜓sanāni a꜕jigu꜓ccha꜕nīyāni

\trline{Ninguno de estos alojamientos es de por sí repugnante}

\clearpage

Imaṁ pūti꜓-kāyaṁ pa꜕tvā

\trline{Pero al tocar este cuerpo podrido}

A꜕tiviya jigu꜓ccha꜕nīyāni jāyanti

\trline{Se vuelven realmente asquerosos.}

Yathā꜓ pa꜕ccayaṁ pava꜓tt꜕amānaṁ dhātu꜕-ma꜓tta꜕m-ev'etaṁ

\trline{Compuestas de elementos según causas y condiciones}

Yad i꜓daṁ gi꜕lāna-pacca꜕ya꜕-bhesajja-pa꜕rikkhāro ta꜕d upa꜕bhuñja꜓ko c꜕a pu꜕gga꜕lo

\trline{Son estas medicinas y la persona que las usa;}

Dhātu-ma꜓tta꜕ko

\trline{Tan solo elementos,}

Ni꜓ssa꜕tto

\trline{No un ser,}

Ni꜓jjīvo

\trline{Sin alma}

Su꜓ñño

\trline{Y vacío de esencia.}

S꜕abbo pa꜕nāyaṁ gi꜕lāna-pacca꜕ya꜕-bhesajja-pa꜕rikkhāro a꜕jigu꜓ccha꜕nīyo

\trline{Ninguna de estas medicinas es de por sí repugnante}

Imaṁ pūti꜓-kāyaṁ pa꜕tvā

\trline{Pero al tocar este cuerpo podrido}

A꜕tiviya jigu꜓ccha꜕nīyo jāyati

\trline{Se vuelven realmente asquerosas.}

%\artopttrue
\clearpage



\chapter[Cinco Temas]{Cinco Temas para Recordar Frecuentemente}

\firstline{Jarā-dhammomhi jaraṁ anatīto}

\begin{leader}
  [Ha꜓nda mayaṁ abhiṇha-paccave꜕kkhaṇa-pāṭhaṁ bhaṇāmase]
\end{leader}

\sidepar{Hombres}%
[Jarā-dhammomhi꜕] jaraṁ a꜕na꜕tīto

\sidepar{Mujeres}%
[Jarā-dhammāmhi꜕] jaraṁ a꜕na꜕tītā

\begin{english}
  Mi naturaleza es envejecer, aún no estoy más allá del envejecimiento.
\end{english}

\sidepar{h.}%
Byādhi꜓-dhammomhi꜕ byādhiṁ a꜕na꜕tīto

\sidepar{m.}%
Byādhi꜓-dhammāmhi꜕ byādhiṁ a꜕na꜕tītā

\begin{english}
  Mi naturaleza es enfermar, aún no estoy más allá de la enfermedad.
\end{english}

\sidepar{h.}%
Ma꜕raṇa-dhammomhi꜕ ma꜕raṇaṁ a꜕na꜕tīto

\sidepar{m.}%
Ma꜕raṇa-dhammāmhi꜕ ma꜕raṇaṁ a꜕na꜕tītā

\begin{english}
  Mi naturaleza es morir, aún no estoy más allá de la muerte.
\end{english}

Sa꜕bbehi me pi꜕yehi ma꜕nāpehi꜕ nānābhāvo vi꜕nābhāvo

\begin{english}
  Todo aquello que es mío, que amo y aprecio,\\
  se volverá diferente, se separará de mí.
\end{english}

\sidepar{h.}%
Kammassa꜕komhi kamma꜓dāyādo kamma꜕yoni kamma꜕bandhu kammapa꜕ṭisa꜓ra꜕ṇo\\
Yaṁ kammaṁ ka꜕rissāmi, kalyāṇaṁ vā pāpa꜕kaṁ vā, tassa꜕ dāyādo bha꜕vissāmi

\clearpage

\sidepar{m.}%
Kammassa꜕kāmhi kamma꜓dāyādā kamma꜕yoni kamma꜕bandhu kammapa꜕ṭisa꜓ra꜕ṇā\\
Yaṁ kammaṁ ka꜕rissāmi, kalyāṇaṁ vā pāpa꜕kaṁ vā, tassa꜕ dāyādā bha꜕vissāmi

\begin{english}
  Soy el dueño de mi intención, heredero de mi intención, \\
  nacido de mi intención, ligado a mi intención,\\
  permanezco soportado por mi intención; cualquier intención que creo, \\
  Para bien o para mal, de eso seré el heredero.
\end{english}

Evaṁ amhehi꜕ a꜕bhiṇhaṁ pacca꜕vekkhi꜓tabbaṁ

\begin{english}
  Así deberíamos reflexionar frecuentemente.
\end{english}

\chapter[Diez temas]{Diez temas para Recordar Frecuentemente}

\firstline{Dasa ime bhikkhave}

\begin{leader}
  [Ha꜓nda mayaṁ pabbajita\hyp{}abhiṇha\hyp{}paccave꜕kkhaṇa\hyp{}pāṭhaṁ bhaṇāmase]
\end{leader}

[Dasa i꜕me bhikkhave] dhammā pabba꜕jitena a꜕bhiṇhaṁ pacca꜕vekkhi꜓tabbā, \pause\ ka꜕ta꜕me dasa

\begin{english}
  Monjes, existen diez dhammas \pause\ sobre los cuales se debe reflexionar frecuentemente. \pause\ ¿Cuáles son estos diez dhammas?
\end{english}

Vevaṇṇi꜕yamhi ajjhūpa꜕ga꜕to'ti pabba꜕jitena a꜕bhiṇhaṁ pacca꜕vekkhi꜓tabbaṁ

\begin{english}
  ‘Ya no vivo siguiendo los valores y objetivos del mundo.’ \pause\\
  Deber es de renunciante \pause\ reflexionar sobre esto frecuentemente.
\end{english}

Parapaṭi꜕baddhā me jīvi꜓kā'ti pabba꜕jitena a꜕bhiṇhaṁ pacca꜕vekkhi꜓tabbaṁ

\begin{english}
  ‘Mi propia vida es sustentada \pause\ por la generosidad de otros.’ \pause\\
  Deber es de renunciante \pause\ reflexionar sobre esto frecuentemente.
\end{english}

Añño me ākappo ka꜕ra꜕ṇīyo'ti pabba꜕jitena a꜕bhiṇhaṁ pacca꜕vekkhi꜓tabbaṁ

\begin{english}
  ‘Debo esforzarme por abandonar mis hábitos antiguos.’ \pause\\
  Deber es de renunciante \pause\ reflexionar sobre esto frecuentemente.
\end{english}

\clearpage

Kacci nu꜕ kho me attā sīla꜕to na u꜕pavadatī'ti pabba꜕jitena a꜕bhiṇhaṁ pacca꜕vekkhi꜓tabbaṁ

\begin{english}
  ‘¿Surgen remordimientos en mi mente \pause\ en relación con mi conducta?’ \pause\\
  Deber es de renunciante \pause\ reflexionar sobre esto frecuentemente.
\end{english}

Kacci nu꜕ kho maṁ a꜕nuvicca viññū sabrahma꜓cārī sīla꜕to na u꜕pavadantī'ti pabba꜕jitena a꜕bhiṇhaṁ pacca꜕vekkhi꜓tabbaṁ

\begin{english}
  ‘¿Será que mis compañeros espirituales \pause\\
  hallan faltas en mi conducta?’ \pause\\
  Deber es de renunciante \pause\ reflexionar sobre esto frecuentemente.
\end{english}

Sa꜕bbehi me pi꜕yehi ma꜕nāpehi꜕ nānābhāvo vi꜕nābhāvo'ti pabba꜕jitena abhiṇhaṁ pacca꜕vekkhi꜓tabbaṁ

\begin{english}
  ‘Todo aquello que es mío, \pause\ que amo y aprecio, \pause\ se volverá diferente, \pause\ se separará de mí.’ \pause\\
  Deber es de renunciante \pause\ reflexionar sobre esto frecuentemente.
\end{english}

Kammassa꜕komhi kamma꜓dāyādo kamma꜕yoni kamma꜕bandhu kammapa꜕ṭisa꜓raṇo, yaṁ kammaṁ ka꜕rissāmi, kalyāṇaṁ vā pāpa꜕kaṁ vā, tassa꜕ dāyādo bha꜕vissāmī'ti pabba꜕jitena a꜕bhiṇhaṁ pacca꜕vekkhi꜓tabbaṁ

\begin{english}
  ‘Soy el dueño de mi intención, \pause\ heredero de mi intención, \pause\\
  nacido de mi intención, \pause\ ligado a mi intención, \pause\\
  permanezco soportado por mi intención; \pause\ cualquier intención que creo, \pause\\
  para bien o para mal, \pause\ de eso seré el heredero.’ \pause\\
  Deber es de renunciante \pause\ reflexionar sobre esto frecuentemente.
\end{english}

\clearpage

`Kathambhūtassa꜕ me rattindi꜕vā vīti꜕pa꜓tantī'ti pabba꜕jitena a꜕bhiṇhaṁ pacca꜕vekkhi꜓tabbaṁ

\begin{english}
  ‘Los días y las noches pasan continuamente; \pause\\
  ¿Cómo estoy usando mi tiempo?’ \pause\\
  Deber es de renunciante \pause\ reflexionar sobre esto frecuentemente.
\end{english}

Kacci nu꜕ kho'haṁ suññā꜓gāre abhira꜕māmī'ti pabba꜕jitena a꜕bhiṇhaṁ pacca꜕vekkhi꜓tabbaṁ

\begin{english}
  ‘¿Aprecio la soledad o no?’ \pause\\
  Deber es de renunciante \pause\ reflexionar sobre esto frecuentemente.
\end{english}

Atthi nu꜕ kho me uttari-ma꜕nussa-dhammā alamariya꜕-ñāṇa-dassana-viseso adhiga꜕to, so'haṁ pacchi꜓me kāle sa꜕brahmacārīhi꜕ puṭṭho na maṅku bha꜕vissāmī'ti pabba꜕jitena a꜕bhiṇhaṁ pacca꜕vekkhi꜓tabbaṁ

\begin{english}
  ‘¿Ha dado mi práctica frutos de comprensión y libertad, \pause\\ de forma que
  al final de mi vida \pause\ no me sienta avergonzado \pause\\
  cuando sea cuestionado \pause\ por mis compañeros espirituales?’ \pause\\
  Deber es de renunciante \pause\ reflexionar sobre esto frecuentemente.
\end{english}

Ime kho bhikkha꜓ve da꜕sa꜕ dhammā pabba꜕jitena a꜕bhiṇhaṁ pacca꜕vekkhitabbā'ti

\begin{english}
  Monjes, estos son los diez Dhammas \pause\ sobre los cuales se debe reflexionar frecuentemente.
\end{english}

\chapter{Verdaderos y Falsos Refugios}

\firstline{Bahuṁ ve saraṇaṁ yanti}

\begin{leader}
  [Ha꜓nda mayaṁ khemākhema-sa꜕raṇa-gamana-\\
  -pa꜕ridīpikā-gāthā꜓yo bha꜕ṇāmase]
\end{leader}

\begin{twochants}
  Bahuṁ ve sa꜕ra꜓ṇaṁ yanti꜕ & pa꜕bba꜕tāni va꜕nāni꜓ ca \\
  Ārāma-rukkha꜕-cetyāni & manussā꜓ bha꜕ya꜕-tajji꜕tā \\
\end{twochants}

\begin{english}
  A muchos refugios ellos van\\
  a montañas y bosques,\\
  a altares y sitios sagrados,\\
  humanos dominados por miedo.
\end{english}

\begin{twochants}
  N'etaṁ kho sa꜕ra꜓ṇaṁ khemaṁ & n'etaṁ sa꜕raṇam-u꜓tt꜕amaṁ \\
  N'etaṁ sa꜕raṇam-āgamma & sa꜕bba-dukkhā꜓ pa꜕mucca꜕ti \\
\end{twochants}

\begin{english}
  Tales refugios no son seguros,\\
  tales refugios no son supremos,\\
  tales refugios no conducen\\
  a la liberación de todo dukkha.
\end{english}

\begin{twochants}
  Yo ca꜕ Buddhañca꜕ Dhammañca꜕ & sa꜓ṅghañca꜕ sa꜓ra꜕ṇaṁ ga꜕to \\
  Ca꜕ttāri a꜕riya-saccāni & sa꜕mmappaññāya꜓ pa꜕ss꜕ati \\
\end{twochants}

\begin{english}
  El que se refugia\\
  en la Joya Triple\\
  ve claramente\\
  las Cuatro Verdades Nobles:
\end{english}

\begin{twochants}
  Dukkhaṁ dukkha-sa꜕muppādaṁ & dukkhassa ca꜕ a꜕ti꜕kka꜕maṁ \\
  A꜕riyañ-c'a꜕ṭṭh'a꜓ṅgi꜕kaṁ maggaṁ & dukkhūpasa꜕ma꜕-gāmi꜓naṁ \\
\end{twochants}

\begin{english}
  El sufrimento y su origen \\
  y aquello que va más allá,\\
  el Óctuple Noble Sendero\\
  que lleva al fin de dukkha.
\end{english}

\begin{twochants}
  Etaṁ kho sa꜕ra꜓ṇaṁ khemaṁ & etaṁ sa꜕raṇam-u꜓tta꜕maṁ \\
  Etaṁ sa꜕raṇam-āgamma & sa꜕bba-dukkhā꜓ pa꜕mucca꜕ti \\
\end{twochants}

\begin{english}
  Tal refugio es seguro,\\
  tal refugio es supremo,\\
  tal refugio conduce realmente\\
  a la liberación de todo dukkha.
\end{english}

\chapter{Versos sobre la Riqueza de Uno que es Noble}

\firstline{Yassa saddhā tathāgate}

\begin{leader}
  [Ha꜓nda mayaṁ a꜕riya-dhana-gāthā꜓yo bha꜕ṇāmase]
\end{leader}

\begin{twochants}
  Yassa꜕ sa꜕ddhā tathā꜓ga꜕te & a꜕ca꜕lā su꜕pa꜕tiṭṭhi꜓tā \\
  Sī꜓lañca꜕ yassa꜕ kalyāṇaṁ & a꜕riya-kantaṁ pasa꜓ṁsi꜕taṁ \\
\end{twochants}

\begin{english}
  Aquél cuya confianza en el Tathāgata\\
  es inquebrantable y establecida bien,\\
  cuya virtud es admirable,\\
  tiene el regocijo y elogio de los Nobles.
\end{english}

\begin{twochants}
  Sa꜓ṅghe pa꜕sā꜕do yass'atthi & uju-bhūtañca da꜓ss꜕anaṁ \\
  A꜕daliddo't꜕i taṁ āhu꜕ & a꜕moghaṁ ta꜕ssa꜕ jīvi꜓taṁ \\
\end{twochants}

\begin{english}
  Aquél que tiene confianza en la Saṅgha,\\
  que ve directamente la verdadera realidad,\\
  de él se dice que no es `pobre'\\
  y que en vano no es su vida.
\end{english}

\begin{twochants}
  Tasmā sa꜕ddhañca꜕ sī꜓lañca꜕ & pasādaṁ dhamma-da꜓ssa꜕naṁ \\
  A꜕nuyuñjetha medhāvī & sa꜕raṁ buddhāna sā꜓sa꜕naṁ \\
\end{twochants}

\begin{english}
  Por ello, a la virtud y a la fe,\\
  a la confianza y al saber de la verdad ---\\
  a esto los sabios se deben dedicar,\\
  manteniendo las enseñanzas de Buddha en la mente.
\end{english}

\chapter{Versos sobre las Tres Características}

\firstline{Sabbe saṅkhārā aniccā'ti}

\begin{leader}
  [Ha꜓nda mayaṁ ti-lakkhaṇ'ādi-gāthā꜓yo bha꜕ṇāmase]
\end{leader}

\begin{twochants}
  Sa꜕bbe sa꜓ṅkhā꜓rā a꜕ni꜓ccā't꜕i & yadā paññāya꜓ pa꜕ssa꜕ti \\
  Atha nibbinda꜕ti dukkhe & esa꜕ maggo vi꜓su꜕ddh꜓iyā \\
\end{twochants}

\begin{english}
  ‘Todas las formaciones condicionadas son impermanentes’ ---\\
  cuando se ve esto con sabiduría,\\
  uno se desencanta del sufrir.\\
  Este es el camino de la purificación.
\end{english}

\begin{twochants}
  Sa꜕bbe sa꜓ṅkhā꜓rā du꜕kkhā't꜕i & yadā paññāya꜓ pa꜕ssa꜕ti \\
  Atha nibbinda꜕ti dukkhe & esa꜕ maggo vi꜓su꜕ddh꜓iyā \\
\end{twochants}

\begin{english}
  ‘Todas las formaciones condicionadas son dukkha’ ---\\
  cuando se ve esto con sabiduría,\\
  uno se desencanta del sufrir.\\
  Este es el camino de la purificación.
\end{english}

\begin{twochants}
  Sa꜕bbe dhammā ana꜓ttā'ti꜕ & yadā paññāya꜓ pa꜕ssa꜕ti \\
  Atha nibbinda꜕ti dukkhe & esa꜕ maggo vi꜓su꜕ddh꜓iyā \\
\end{twochants}

\begin{english}
  ‘Todas las formaciones no son yo’ ---\\
  cuando se ve esto con sabiduría,\\
  uno se desencanta del sufrir.\\
  Este es el camino de la purificación.
\end{english}

\clearpage

\begin{twochants}
  A꜕ppa꜕kā te manusse꜓su꜕ & ye janā pāra-gāmi꜓no \\
  A꜕thāyaṁ i꜕ta꜕rā pajā & tīram-evānudhā꜓va꜕ti \\
\end{twochants}

\begin{english}
  Pocos son aquellos\\
  que cruzan a la otra orilla.\\
  Sin embargo, muchos son aquellos\\
  que vagan sin rumbo en esta orilla.
\end{english}

\begin{twochants}
  Ye ca꜕ kho sammad-akkhāte & dhamme dhammānuva꜓tt꜕ino \\
  Te ja꜕nā pā꜕ram-essanti & ma꜕ccu-dheyyaṁ sudu꜓tta꜕raṁ \\
\end{twochants}

\begin{english}
  Donde quiera que el Dhamma sea bien enseñado,\\
  los que practican de acuerdo con él\\
  podrán cruzar más allá\\
  de la rueda de la muerte, tan difícil de escapar.
\end{english}

\begin{twochants}
  Kaṇhaṁ dhammaṁ vi꜕ppahā꜓ya & su꜕kkaṁ bhāvetha꜕ paṇḍi꜓to \\
  Okā a꜕noka꜕m-āgamma & viveke ya꜕tth꜕a dūramaṁ \\
  Ta꜕trābh꜕irat꜕im-iccheyya & hi꜕tvā kāme a꜕kiñc꜓ano \\
\end{twochants}

\begin{english}
  Abandonando la oscuridad,\\
  los sabios cultivan la luz;\\
  dejando atrás áreas inundadas,\\
  ellos alcanzan tierra firme.\\
  A pesar de que sea dificil encontrar deleite\\
  en la vida solitaria,\\
  tal deleite debe ser anhelado,\\
  renunciando a los placeres sensuales\\
  habiendo abandonado todo.
\end{english}

\chapter{Versos sobre la Carga}

\firstline{Bhārā have pañcakkhandhā}

\begin{leader}
  [Ha꜓nda mayaṁ bhāra-su꜕tta-gāthā꜓yo bha꜕ṇāmase]
\end{leader}

\begin{twochants}
Bhārā ha꜕ve pañcakkha꜓ndhā & bhāra-hāro ca pu꜓gga꜕lo \\
Bhā꜕r'ādānaṁ du꜕kkhaṁ loke꜓ & bhāra-nikkhe꜓pa꜕naṁ su꜕khaṁ \\
\end{twochants}

\begin{english}
  Las cinco Khandhas cargas son,\\
  y el individuo las carga.\\
  En el mundo, llevar cargas es dukkha;\\
  abandonarlas, felicidad.
\end{english}

\begin{twochants}
Nikkhipi꜕tvā ga꜕ruṁ bhā꜓raṁ & aññaṁ bhāraṁ anā꜓di꜕ya \\
Sa꜕mūlaṁ taṇhaṁ a꜕bbuyha & nicchāto pa꜕ri꜕nibbu꜕to \\
\end{twochants}

\begin{english}
  Abandonando la pesada carga,\\
  y la nueva carga sin tomar,\\
  desraizado el anhelo,\\
  sin más deseo, liberado.
\end{english}

\chapter{Versos sobre una Auspiciosa Noche}

\firstline{Atītaṁ nānvāgameyya}

\begin{leader}
  [Ha꜓nda mayaṁ bhadd'eka-ratta꜕-gāthā꜓yo bha꜕ṇāmase]
\end{leader}

\begin{twochants}
  A꜕tītaṁ nānvāga꜕meyya & nappa꜕ṭikaṅkhe꜓ a꜕nāga꜓taṁ \\
  Ya꜕d'a꜕tītaṁ pa꜕hīnan-taṁ & a꜕ppattañc꜕a a꜕nāga꜕taṁ \\
\end{twochants}

\begin{english}
  No revivir el pasado,\\
  ni anticipar lo que ha de llegar.\\
  El pasado ya pasó\\
  y el futuro aún no llegó.
\end{english}

\begin{twochants}
  Paccu꜕ppannañca꜕ yo dhammaṁ & tattha tattha vi꜓pa꜕ss꜕ati \\
  Asa꜓ṁhi꜕raṁ asa꜓ṅku꜕ppaṁ & taṁ viddhām-a꜕nu꜕brūhaye \\
\end{twochants}

\begin{english}
  Así, viendo claramente,\\
  todo en el presente,\\
  impasible, sin agitar\\
  la realización fortalecerá.
\end{english}

\begin{twochants}
  A꜕jj'eva ki꜕cca꜕m-ātappaṁ & ko jaññā ma꜓ra꜕ṇaṁ su꜕ve \\
  Na hi no sa꜓ṅga꜕ran-tena & mahā-senena ma꜓cc꜕unā \\
\end{twochants}

\begin{english}
  Dedicarse hoy ardientemente a la tarea\\
  pues, ¿quién sabe?, mañana la muerte puede venir.\\
  Y ante el poderoso ejército de la muerte,\\
  no hay negociación posible.
\end{english}

\clearpage

\begin{twochants}
  Evaṁ vihārim-ātāpiṁ & a꜕ho-rattam-a꜕tandi꜓taṁ \\
  Taṁ ve bha꜕dd'eka꜕-ratto'ti & santo ā꜕ci꜕kkha꜕te muni \\
\end{twochants}

\begin{english}
  Aquél que así vive con vigor,\\
  sosteniéndolo día y noche,\\
  tiene de hecho `una noche auspiciosa’ ---\\
  así dice el Pacífico Sabio.
\end{english}

\chapter{Versos de Respeto por el Dhamma}

\firstline{Ye ca atītā sambuddhā}

\begin{leader}
  [Ha꜓nda mayaṁ dhamma-gā꜕rav'ādi꜕-gāthā꜓yo bha꜕ṇāmase]
\end{leader}

\begin{twochants}
  Ye ca꜕ atītā sa꜓mbuddhā & ye ca꜕ buddhā a꜕nāga꜓tā \\
  Yo c'eta꜕rahi sambuddho & ba꜕hunnaṁ so꜕ka꜕-nāsa꜕no \\
\end{twochants}

\begin{english}
  Todos los Buddhas del pasado,\\
  todos los Buddhas por venir,\\
  el Buddha de esta Era actual,\\
  destructores de tanta pena.
\end{english}

\begin{twochants}
  Sa꜕bbe sa꜕ddhamma-gar꜓uno & vi꜕ha꜕riṁsu vi꜕ha꜕ranti ca \\
  A꜕tho pi viha꜕riss꜓anti & esā buddhāna꜓ dha꜕mma꜕tā \\
\end{twochants}

\begin{english}
  Aquellos que vivieron o están vivos,\\
  y aquellos que vivirán,\\
  todos veneran el Verdadero Dhamma,\\
  la naturaleza de los Buddhas es tal.
\end{english}

\begin{twochants}
  Tasmā h꜕i atta-kāmena & mahattam-abhika꜓ṅkh꜕atā \\
  Sa꜕ddhammo ga꜕ru꜓-kāta꜕bbo & s꜕araṁ buddhāna sā꜓sa꜕naṁ \\
\end{twochants}

\begin{english}
  Por ello, anhelando el bienestar\\
  y las aspiraciones más elevadas,\\
  el verdadero Dhamma considerad\\
  evocando las palabras de Buddha.
\end{english}

\clearpage

\begin{twochants}
  Na h꜕i dhammo a꜕dhammo ca & ubho s꜕ama-vipāki꜓no \\
  A꜕dhammo nirayaṁ neti & dh꜕ammo pāpeti꜕ su꜕gga꜕tiṁ \\
\end{twochants}

\begin{english}
  El verdadero  y el falso Dhamma\\
  nunca traerán el mismo efecto:\\
  el Dhamma falso conduce al infierno,\\
  el verdadero Dhamma lleva al buen camino.
\end{english}

Dhammo ha꜕ve rakkha꜕ti꜕ dhamma꜓-cāriṁ\\
Dhammo su꜕ciṇṇo su꜕kham-āvahāti\\
Esā꜓ni꜕saṁso dhamme su꜕ciṇṇe

% The last line is always omitted when chanting this verse in Thai monasteries, for some unknown reason.
% Na duggatiṁ gacchati dhamma-cārī
% (Thag 4.10, Dhammikattheragāthā)

\begin{english}
  El Dhamma protege a quien lo cultiva\\
  y conduce a ser feliz si se practica bien.\\
  Es una bendición del Dhamma bien practicado.
\end{english}

\chapter{Ovāda-Pāṭimokkha}

\firstline{Khantī paramaṁ tapo tītikkhā}

\enlargethispage{\baselineskip}

\begin{leader}
  [Ha꜓nda mayaṁ ovāda-pā꜕ṭi꜕mokkha-gāthā꜓yo bha꜕ṇāmase]
\end{leader}

Sa꜕bb꜕a-pāpa꜕ss꜕a a꜕ka꜕ra꜓ṇaṁ
	
	\begin{english}
		Evitar todo el mal,
	\end{english}
	
	Ku꜕salassūpasa꜓mpa꜕dā
	
	\begin{english}
		cultivar el bien
	\end{english}
	
	Sa꜕ci꜕tta-pa꜕ri꜓yoda꜓pa꜕naṁ
	
	\begin{english}
		y purificar la mente ---
	\end{english}
	
	Etaṁ buddhāna sā꜓sa꜕naṁ
	
	\begin{english}
		Esta es la enseñanza de los Buddhas.
	\end{english}


Kha꜓ntī pa꜕ramaṁ ta꜕po tīti꜕kkhā

\begin{english}
  Permanecer paciente es la mayor austeridad.
\end{english}

Nibbānaṁ pa꜕ramaṁ va꜕dant꜕i buddhā

\begin{english}
  ‘Nibbāna es supremo’, dicen los Buddhas.
\end{english}

Na h꜕i pa꜕bbaji꜕to pa꜕rūpaghātī

\begin{english}
  No es verdaderamente monje \pause\ el que perjudica a alguien,
\end{english}

Sa꜕maṇo ho꜓ti pa꜕raṁ vihe꜓ṭha꜕yanto

\begin{english}
  No se es verdaderamente renunciante \pause\ cuando se oprime a los demás.
\end{english}

A꜕nūpa꜕vādo a꜕nūpa꜕ghāto

\begin{english}
  No ofender, no perjudicar,
\end{english}

Pā꜕ṭimokkhe꜓ ca꜕ sa꜓ṁva꜕ro

\begin{english}
  moderarse de acuerdo con las reglas monásticas.
\end{english}

Mattaññu꜕tā ca꜕ bhatta꜕smiṁ

\begin{english}
  Moderarse en la comida,
\end{english}

Pa꜕ntañca꜕ saya꜓n'āsa꜕naṁ

\begin{english}
  vivir en soledad,
\end{english}

A꜕dhici꜕tte ca꜕ āyogo

\begin{english}
  dedicarse a elevar la conciencia ---
\end{english}

Etaṁ buddhāna sā꜓sa꜕naṁ

\begin{english}
  Esta es la enseñanza de los Buddhas.
\end{english}

\chapter[La Primera Exclamación]{Versos sobre la Primera Exclamación de Buddha}

\firstline{Aneka-jāti-saṁsāraṁ}

\begin{leader}
  [Ha꜓nda mayaṁ paṭhama-bu꜕ddha-bhāsi꜕ta-gāthāyo bh꜕aṇāmase]
\end{leader}

\begin{twochants}
  A꜕neka꜕-jāti꜕-sa꜓ṁsā꜓raṁ & sa꜕ndhāviss꜓aṁ a꜕nibbi꜕saṁ \\
  Ga꜕ha-kā꜕raṁ ga꜕vesa꜓nto & dukkhā jāt꜕i pu꜕nappu꜕naṁ \\
\end{twochants}

\begin{english}
  Durante muchas vidas, en la rueda de vida y muerte\\
  vagué indefinidamente.\\
  Al constructor de esta casa yo buscaba;\\
  ¡cuánta pena da nacer una y otra vez!
\end{english}

\begin{twochants}
  Ga꜕ha-kā꜕raka꜕ diṭṭho꜓'si & pu꜕na gehaṁ na kā꜓hasi \\
  Sa꜕bbā te phāsu꜕kā bhaggā & gaha-kūṭa꜓ṁ vi꜕saṅkh꜕ataṁ \\
  Visa꜓ṅkhā꜕ra-ga꜕taṁ ci꜕ttaṁ & taṇhānaṁ kh꜕aya꜕m-ajjh꜕agā \\
\end{twochants}

\begin{english}
  ¡Oh! Constructor de esta casa, ¡te he visto!\\
  No construirás nuevamente para mí.\\
  Todas tus vigas están partidas\\
  y la cumbrera aplastada.\\
  La mente ha alcanzado lo Incondicionado;\\
  llegando al cese del anhelo.
\end{english}

\chapter[Las Últimas Instrucciones]{Versos sobre las Últimas Instrucciones}

\firstline{Handa dāni bhikkhave āmantayāmi vo}

\begin{leader}
  [Ha꜓nda mayaṁ pacchima-ovāda-gāthā꜓yo bha꜕ṇāmase]
\end{leader}

Handa dāni bhi꜓kkha꜕ve āmant꜕ayāmi꜓ vo

\begin{english}
  Ahora monjes, yo os digo:
\end{english}

Vaya-dhammā sa꜓ṅkhā꜓rā

\begin{english}
  El cambio es la naturaleza de las cosas condicionadas.
\end{english}

A꜕ppamādena sa꜓mpā꜕dethā'ti

\begin{english}
  Practiquen diligentemente --
\end{english}

Ayaṁ tathā꜓ga꜕tassa pa꜕cchi꜓mā vācā

\begin{english}
  estas son las últimas palabras del Tathāgata.
\end{english}

\chapter{Surgen a Partir de una Causa}

\firstline{Ye dhammā hetuppabhavā}

\begin{leader}
  [Ha꜓nda mayaṁ assajithera-gāthā꜓yo bha꜕ṇāmase]
\end{leader}

Ye dhammā hetuppabhavā

\begin{english}
  Todos los fenómenos surgen de una causa:
\end{english}

Tesaṁ hetuṁ tathāgato āha

\begin{english}
  el Tathāgata explicó su causa
\end{english}

Tesañca yo nirodho

\begin{english}
  y también su cese.
\end{english}

Evaṁ-vādī mahāsamaṇo'ti

\begin{english}
  Esta es la enseñanza del Gran Asceta.
\end{english}

% \suttaRef{Mv.1.23.5}

\chapter[Incondicionado]{Reflexión sobre lo Incondicionado}

\firstline{Atthi bhikkhave ajātaṁ abhūtaṁ akataṁ}

\begin{leader}
  [Ha꜓nda mayaṁ nibbāna-sutta-pāṭhaṁ bha꜕ṇāmase]
\end{leader}

Atthi bhi꜓kkha꜕ve a꜕jātaṁ a꜓bhūtaṁ a꜕kataṁ a꜕sa꜓ṅkh꜕ataṁ

\begin{english}
  Existe un no-nacido, no-originado, increado, no-formado.
\end{english}

N꜕o cetaṁ bhi꜓kkha꜕ve a꜕bhavissa a꜕jātaṁ a꜓bhūtaṁ a꜕kataṁ a꜕sa꜓nkh꜕ataṁ

\begin{english}
 Si no existiese este no-nacido, no-originado, increado, no-formado,
\end{english}

Na꜕ yidaṁ jātassa꜕ bhūtassa ka꜕tassa sa꜓ṅkh꜕atassa nissaraṇaṁ paññāye꜓tha

\begin{english}
  La liberación del mundo de lo nacido, originado, creado, formado, no sería posible.
\end{english}

Ya꜕smā ca kho bhi꜓kkh꜕ave atthi a꜕jātaṁ a꜓bhūtaṁ a꜕kataṁ a꜕sa꜓ṅkha꜕taṁ

\begin{english}
  Pero una vez que existe un no-nacido, no-originado, increado, no-formado,
\end{english}

Ta꜕smā jātass꜕a bhūtassa ka꜕tassa sa꜓ṅkha꜕tassa nissaraṇaṁ paññāyati

\begin{english}
  Así es posible la liberación del mundo de lo nacido, originado, creado, formado.
\end{english}

\chapter[Breve Consejo a Gotamī]{Breve Consejo a Gotamī}

\emph{Saṅkhitta-gotamiyovāda Sutta, AN 8.53}

% https://suttacentral.net/an8.53/pli/ms

\begin{leader}
  [Handa mayaṁ saṅkhitta-gotamiyovāda-sutta-pāṭhaṁ bhaṇāmase]
\end{leader}

Ye kho tvaṁ, gotami, dhamme jāneyyāsi:\\
‘ime dhammā sarāgāya saṁvattanti, no virāgāya;\\
saṁyogāya saṁvattanti, no visaṁyogāya;\\
ācayāya saṁvattanti, no apacayāya;\\
mahicchatāya saṁvattanti, no appicchatāya;\\
asantuṭṭhiyā saṁvattanti, no santuṭṭhiyā;\\
saṅgaṇikāya saṁvattanti, no pavivekāya;\\
kosajjāya saṁvattanti, no vīriyārambhāya;\\
dubbharatāya saṁvattanti, no subharatāyā’ti;

\begin{english}
  Gotamī, las cualidades que conozcais,\\
  conducen a la pasión, no al desencanto;\\
  a la obstrucción, no a la desobstrucción;\\
  a acumular, no a desechar;\\
  al engrandecimento personal, no a la modestia;\\
  al descontento, no a la satisfacción;\\
  al enredo, no a la reclusión;\\
  a ser vago, no a la persistencia vivaz;\\
  a ser una carga, no a ser fácil de mantener;
\end{english}

ekaṁsena, gotami, dhāreyyāsi: ‘neso dhammo, neso vinayo, netaṁ satthusāsanan’ti.

\begin{english}
  podeis afirmar categóricamente: ‘Esto no es Dhamma,\\
  esto no es Vinaya, esto no es la instrucción del Maestro.’
\end{english}

Ye ca kho tvaṁ, gotami, dhamme jāneyyāsi:\\
‘ime dhammā virāgāya saṁvattanti, no sarāgāya;\\
visaṁyogāya saṁvattanti, no saṁyogāya;\\
apacayāya saṁvattanti, no ācayāya;\\
appicchatāya saṁvattanti, no mahicchatāya;\\
santuṭṭhiyā saṁvattanti, no asantuṭṭhiyā;\\
pavivekāya saṁvattanti, no saṅgaṇikāya;\\
vīriyārambhāya saṁvattanti, no kosajjāya;\\
subharatāya saṁvattanti, no dubbharatāyā’ti;

\begin{english}
  Cuando las cualidades que conozcais,\\
  conducen al desencanto, no a la pasión;\\
  a la desobstrucción, no a la obstrucción;\\
  a desechar, no a acumular;\\
  a la modestia, no al engrandecimiento personal;\\
  a la satisfacción, no al descontento;\\
  a la reclusión, no al enredo;\\
  a la persistencia vivaz, no a ser vago;\\
  a ser fácil de mantener, no a ser una carga;
\end{english}

ekaṁsena, gotami, dhāreyyāsi: ‘eso dhammo, eso vinayo, etaṁ satthusāsanan’ti.

\begin{english}
  podeis afirmar categóricamente: ‘Esto es Dhamma,\\
  esto es Vinaya, esta es la instrucción del Maestro.’
\end{english}

\chapter{La Raíz de Todas las Cosas}

% AN 10.58
% https://www.accesstoinsight.org/tipitaka/an/an10/an10.058.than.html
% https://www.dhammatalks.org/suttas/AN/AN10_58.html
% https://suttacentral.net/an10.58/pli/ms
% https://suttacentral.net/an10.58/en/sujato
% https://suttacentral.net/an10.58/en/bodhi

\firstline{Kiṁ-mūlakā āvuso sabbe dhammā}

\begin{leader}
  [Ha꜓nda mayam mūlaka-sutta-pāṭhaṁ bha꜕ṇāmase]
\end{leader}

\enlargethispage{2\baselineskip}

Kiṁ-mūlakā āvuso sabbe dhammā\\
kiṁ-sambhavā sabbe dhammā\\
kiṁ-samudayā sabbe dhammā\\
kiṁ-samosaraṇā sabbe dhammā\\
kiṁ-pamukhā sabbe dhammā\\
kiṁ-adhipateyyā sabbe dhammā\\
kiṁ-uttarā sabbe dhammā\\
kiṁ-sārā sabbe dhammā\\
kiṁ-ogadhā sabbe dhammā\\
kiṁ-pariyosānā sabbe dhammā'ti.

Chanda'mūlakā āvuso sabbe dhammā\\
manasikāra'sambhavā sabbe dhammā\\
phassa'samudayā sabbe dhammā\\
vedanā'samosaraṇā sabbe dhammā\\
samādhi'ppamukhā sabbe dhammā\\
satā'dhipateyyā sabbe dhammā\\
paññ'uttarā sabbe dhammā\\
vimutti'sārā sabbe dhammā\\
amat'ogadhā sabbe dhammā\\
nibbāna'pariyosānā sabbe dhammā'ti.

\clearpage

\begin{english}
  ¿Enraizadas en qué, amigo, están todas las cosas?\\
  ¿Nacidas de qué, son todas las cosas?\\
  ¿Provenientes de qué, son todas las cosas?\\
  ¿Convergiendo en qué, son todas las cosas?\\
  ¿Dirigidas por qué, son todas las cosas?\\
  ¿Dominadas por qué, son todas las cosas?\\
  ¿Superadas por qué, son todas las cosas?\\
  ¿Resultan en qué como esencia, todas las cosas?\\
  ¿Se funden en qué, todas las cosas?\\
  ¿Terminan en qué, todas las cosas?

  \bigskip

  Amigo, todas las cosas están enraizadas en el deseo.\\
  Todas las cosas nacen de la atención.\\
  Todas las cosas provienen del contacto.\\
  Todas las cosas convergen en sensación.\\
  Todas las cosas son dirigidas por la concentración.\\
  Todas las cosas son dominadas por la consciencia.\\
  Todas las cosas son superadas por la sabiduría.\\
  Todas las cosas resultan en liberación como esencia.\\
  Todas las cosas se funden en la inmortalidad.\\
  Todas las cosas terminan en Nibbāna.

\end{english}

{\raggedleft
  \emph{Aṅguttara Nikāya 10.58}
\par}

\chapter{Ānāpānassati-sutta}

\firstline{Ānāpānassati bhikkhave bhāvitā bahulī-katā}

\begin{leader}
  [Ha꜓nda mayam ānāpānass꜕ati-sutta-pāṭhaṁ bha꜕ṇāmase]
\end{leader}

Ānāpāna꜓ssa꜕ti bhi꜓kkha꜕ve bhāvi꜓tā bahu꜕līka꜕tā

\begin{english}
  Bhikkhus, cuando ānāpānassati es cultivada y desarrollada,
\end{english}

Mahappha꜕lā ho꜓ti mahā꜓nisa꜓ṁsā

\begin{english}
  da grandes frutos y es de gran beneficio.
\end{english}

Ānāpāna꜓ssa꜕ti bhi꜓kkha꜕ve bhāvi꜓tā bahu꜕līka꜕tā

\begin{english}
  Bhikkhus, cuando cultivada y desarrollada, ānāpānassati
\end{english}

Ca꜕ttāro sati꜓pa꜕ṭṭhāne pa꜕ri꜓pū꜕reti

\begin{english}
  lleva las Cuatro Fundaciones de Sati a su plenitud;
\end{english}

Ca꜕ttāro sa꜕tipa꜕ṭṭhānā bhāvi꜓tā bahu꜕līka꜕tā

\begin{english}
  cuando cultivadas y desarrolladas, las Cuatro Fundaciones de Sati
\end{english}

Sa꜕tta-bojjhaṅge pa꜕ri꜓pū꜕renti

\begin{english}
  llevan los Siete Factores del Despertar a su plenitud;
\end{english}

Sa꜕tta-bojjhaṅgā bhāvi꜓tā bahu꜕līka꜕tā

\begin{english}
  cuando cultivados y desarrollados, los Siete Factores del Despertar
\end{english}

\enlargethispage{\baselineskip}

Vijjā-vimuttiṁ pa꜕ri꜓pū꜕renti

\begin{english}
  llevan al verdadero conocimiento y la liberación a su plenitud.
\end{english}

Kathaṁ bhāvi꜓tā ca bhi꜓kkha꜕ve ānāpāna꜓ss꜕ati ka꜕thaṁ bahu꜕līka꜕tā

\begin{english}
  ¿Y cómo, bhikkhus, ānāpānassati es cultivada y desarrollada
\end{english}

Mahappha꜕lā ho꜓ti mahā꜓nisa꜓ṁsā

\begin{english}
  para dar grandes frutos y ser de gran beneficio?
\end{english}

Idha bhi꜓kkha꜕ve bhikkhu

\begin{english}
  Bhikkhus, aquí un bhikkhu
\end{english}

Arañña꜓-ga꜕to vā

\begin{english}
  habiendo ido al bosque,
\end{english}

Rukkha-mūla꜓-ga꜕to vā

\begin{english}
  hacia la raiz de un árbol
\end{english}

Suññāgāra꜓-ga꜕to vā

\begin{english}
  o hacia una cabaña vacía,
\end{english}

N꜕isīdati pallaṅkaṁ ābhuji꜓tv꜕ā

\begin{english}
  se sienta de piernas cruzadas,
\end{english}

Ujuṁ kāyaṁ pa꜕ṇidhāya pa꜕rimukhaṁ sa꜕tiṁ u꜕paṭṭha꜕petvā

\begin{english}
  con su cuerpo derecho y establece sati en frente suya.
\end{english}

So sa꜕to'va a꜕ssasa꜕ti sa꜕to'va pa꜕ssa꜕sa꜕ti

\begin{english}
  Consciente, él inspira; consciente, él espira.
\end{english}

Dīghaṁ vā assa꜕sa꜓nto dīghaṁ a꜕ssasā꜓mī'ti pa꜕jānāti

\begin{english}
  Al tener una inspiración larga, él sabe: ‘Esta es una inspiración larga’;
\end{english}

Dīghaṁ vā pa꜕ssa꜕santo dīghaṁ pa꜕ssasā꜓mī'ti pa꜕jānāti

\begin{english}
  Al tener una espiración larga, él sabe: ‘Esta es una espiración larga’;
\end{english}

Rassaṁ vā a꜕ssa꜕santo rassaṁ a꜕ssasā꜓mī'ti pa꜕jānāti

\begin{english}
  Al tener una inspiración corta, él sabe: ‘Esta es una inspiración corta’;
\end{english}

Rassaṁ vā pa꜕ssa꜕santo rassaṁ pa꜕ssasā꜓mī'ti pa꜕jānāti

\begin{english}
  Al tener una espiración corta, él sabe: ‘Esta es una espiración corta’;
\end{english}

Sabba꜕-kāya-paṭ꜕isa꜓ṁvedī a꜕ssasi꜕ssāmī'ti si꜕kkh꜕ati

\begin{english}
  Él se entrena: ‘Voy a inspirar experimentando el cuerpo entero’.
\end{english}

Sabba꜕-kāya-paṭ꜕isa꜓ṁvedī pa꜕ssasi꜕ssāmī'ti si꜕kkh꜕ati

\begin{english}
  Él se entrena: ‘Voy a espirar experimentando el cuerpo entero’.
\end{english}

Passa꜕mbhayaṁ kāya꜕-sa꜓ṅkhāraṁ a꜕ssasi꜕ssāmī'ti si꜕kkh꜕ati

\begin{english}
  Él se entrena: ‘Voy a inspirar tranquilizando el cuerpo’.
\end{english}

Passa꜕mbhayaṁ kāya꜕-sa꜓ṅkhāraṁ pa꜕ssasi꜕ssāmī'ti si꜕kkh꜕ati

\begin{english}
  Él se entrena: ‘Voy a espirar tranquilizando el cuerpo’.
\end{english}

Pīti꜕-paṭi꜕sa꜓ṁvedī a꜕ssasi꜕ssāmī'ti si꜕kkh꜕ati

\begin{english}
  Él se entrena: ‘Voy a inspirar experimentando éxtasis’.
\end{english}

Pīti꜕-paṭi꜕sa꜓ṁvedī pa꜕ssasi꜕ssāmī'ti si꜕kkh꜕ati

\begin{english}
  Él se entrena: ‘Voy a espirar experimentando éxtasis’.
\end{english}

Sukh꜕a-paṭi꜕sa꜓ṁvedī a꜕ssasi꜕ssāmī'ti si꜕kkh꜕ati

\begin{english}
  Él se entrena: ‘Voy a inspirar experimentando felicidad’.
\end{english}

Sukh꜕a-paṭi꜕sa꜓ṁvedī pa꜕ssasi꜕ssāmī'ti si꜕kkh꜕ati

\begin{english}
  Él se entrena: ‘Voy a espirar experimentando felicidad’.
\end{english}

Citta꜕-sa꜓ṅkhāra-paṭi꜕sa꜓ṁvedī a꜕ssasi꜕ssāmī'ti si꜕kkh꜕ati

\begin{english}
  Él se entrena: ‘Voy a inspirar experimentando las formaciones mentales’.
\end{english}

Citta꜕-sa꜓ṅkhāra-paṭi꜕sa꜓ṁvedī pa꜕ssasi꜕ssāmī'ti si꜕kkh꜕ati

\begin{english}
  Él se entrena: ‘Voy a espirar experimentando las formaciones mentales’.
\end{english}

Passa꜕mbhayaṁ citta꜕-sa꜓ṅkhāraṁ a꜕ssasi꜕ssāmī'ti si꜕kkh꜕ati

\begin{english}
  Él se entrena: ‘Voy a inspirar tranquilizando las formaciones mentales’.
\end{english}

Passa꜕mbhayaṁ citt꜕a-sa꜓ṅkhāraṁ pa꜕ssasi꜕ssāmī'ti si꜕kkh꜕ati

\begin{english}
  Él se entrena: ‘Voy a espirar tranquilizando las formaciones mentales’.
\end{english}

Citta꜕-paṭi꜕sa꜓ṁvedī a꜕ssasi꜕ssāmī'ti si꜕kkh꜕ati

\begin{english}
  Él se entrena: ‘Voy a inspirar experimentando la mente’.
\end{english}

Citta꜕-paṭi꜕sa꜓ṁvedī pa꜕ssasi꜕ssāmī'ti si꜕kkh꜕ati

\begin{english}
  Él se entrena: ‘Voy a espirar experimentando la mente’.
\end{english}

A꜕bhippa꜕moda꜓yaṁ cittaṁ a꜕ssasi꜕ssāmī'ti si꜕kkh꜕ati

\begin{english}
  Él se entrena: ‘Voy a inspirar alegrando la mente’.
\end{english}

A꜕bhippa꜕moda꜓yaṁ cittaṁ pa꜕ssasi꜕ssāmī'ti si꜕kkh꜕ati

\begin{english}
  Él se entrena: ‘Voy a espirar alegrando la mente’.
\end{english}

Sa꜕māda꜓haṁ cittaṁ a꜕ssasi꜕ssāmī'ti si꜕kkh꜕ati

\begin{english}
  Él se entrena: ‘Voy a inspirar concentrando la mente’.
\end{english}

Sa꜕māda꜓haṁ cittaṁ pa꜕ssasi꜕ssāmī'ti si꜕kkh꜕ati

\begin{english}
  Él se entrena: ‘Voy a espirar concentrando la mente’.
\end{english}

Vimoca꜓yaṁ cittaṁ a꜕ssasi꜕ssāmī'ti si꜕kkh꜕ati

\begin{english}
  Él se entrena: ‘Voy a inspirar liberando la mente’.
\end{english}

Vimoca꜓yaṁ cittaṁ pa꜕ssasi꜕ssāmī'ti si꜕kkh꜕ati

\begin{english}
  Él se entrena: ‘Voy a espirar liberando la mente’.
\end{english}

Aniccānupa꜕ssī a꜕ssasi꜕ssāmī'ti si꜕kkh꜕ati

\begin{english}
  Él se entrena: ‘Voy a inspirar contemplando impermanencia’.
\end{english}

Aniccānupa꜕ssī pa꜕ssasi꜕ssāmī'ti si꜕kkh꜕ati

\begin{english}
  Él se entrena: ‘Voy a espirar contemplando impermanencia’.
\end{english}

Virāgānupa꜕ssī a꜕ssasi꜕ssāmī'ti si꜕kkh꜕ati

\begin{english}
  Él se entrena: ‘Voy a inspirar contemplando el desvanecer de las pasiones’.
\end{english}

Virāgānupa꜕ssī pa꜕ssasi꜕ssāmī'ti si꜕kkh꜕ati

\begin{english}
  Él se entrena: ‘Voy a espirar contemplando el desvanecer de las pasiones’.
\end{english}

Nirodhānupa꜕ssī a꜕ssasi꜕ssāmī'ti si꜕kkh꜕ati

\begin{english}
  Él se entrena: ‘Voy a inspirar contemplando el cese’.
\end{english}

Nirodhānupa꜕ssī pa꜕ssasi꜕ssāmī'ti si꜕kkh꜕ati

\begin{english}
  Él se entrena: ‘Voy a espirar contemplando el cese’.
\end{english}

Pa꜕ṭiniss꜕aggānupa꜕ssī a꜕ssasi꜕ssāmī'ti si꜕kkh꜕ati

\begin{english}
  Él se entrena: ‘Voy a inspirar contemplando la renuncia’.
\end{english}

Pa꜕ṭinissa꜕ggānupa꜕ssī pa꜕ssasi꜕ssāmī'ti si꜕kkh꜕ati

\begin{english}
  Él se entrena: ‘Voy a espirar contemplando la renuncia’.
\end{english}

Evaṁ bhāvi꜓tā kho bhi꜓kkha꜕ve ānāpāna꜓ss꜕ati evaṁ bahu꜕līka꜕tā

\begin{english}
  Bhikkhus, así ānāpānassati es cultivada y desarrollada,
\end{english}

Mahappha꜕lā ho꜓ti mahā꜓nisa꜓ṁsā'ti

\begin{english}
  para dar grandes frutos y ser de gran beneficio.
\end{english}


\chapter[El Óctuple Noble Sendero]{La Enseñanza sobre el Óctuple Noble Sendero}


%SN45.8 Maggavibhanga Sutta

\firstline{Ayam-eva ariyo aṭṭhaṅgiko maggo}

\begin{leader}
	[Handa mayaṁ ariyaṭṭhaṅgika-magga-pāṭham bhaṇāmase]
\end{leader}

Ayam-eva a꜕riyo aṭṭha꜓ṅgi꜕ko maggo

\begin{english}
	Este es el Óc꜕tuple Noble Sende꜓ro,
\end{english}

Se꜓yyathī꜓daṁ

\begin{english}
	Que es como sigue:
\end{english}

Sa꜓mmā-diṭṭhi

\begin{english}
	Visión Co꜕rrecta,
\end{english}

Sa꜓mmā-sa꜓ṅka꜕ppo

\begin{english}
	Correc꜓ta Intención,
\end{english}

Sa꜓mmā-vācā

\begin{english}
	Hablar Correcto,
\end{english}

Sa꜓mmā-kammanto

\begin{english}
	Correcta A꜕cción,
\end{english}

Sa꜓mmā-ājīvo

\begin{english}
	Forma de Vida Correcta,
\end{english}

Sa꜓mmā-vā꜕yāmo

\begin{english}
	Es꜕fuerzo Correcto,
\end{english}

\ifaivedition
\clearpage
\fi

Sa꜓mmā-sa꜕ti

\begin{english}
	Sati Correcto,
\end{english}

Sa꜓mmā-sa꜕mādhi

\begin{english}
	Correcta Concentración.
\end{english}

Ka꜕tamā ca bhi꜓kkh꜕ave sammā-diṭṭhi

\begin{english}
	¿Y qué, bhikkhus, es Visión Correcta?
\end{english}

Yaṁ kho bhi꜓kkh꜕ave dukkhe ñāṇaṁ

\begin{english}
	Conocimiento del sufrimiento;
\end{english}

Dukkha-sa꜕mu꜕daye ñāṇaṁ

\begin{english}
	Conocimiento del origen del sufrimiento;
\end{english}

Dukkha-ni꜓rodhe ñāṇaṁ

\begin{english}
	Conocimiento del cese del sufrimiento;
\end{english}

Dukkha-ni꜓rodha-gāmi꜓ni꜓yā pa꜕ṭipa꜕dāya ñāṇaṁ

\begin{english}
	Conocimiento del camino que lleva al cese del sufrimiento:
\end{english}

A꜕yaṁ vuccati bhi꜓kkh꜕ave sa꜓mmā-diṭṭhi

\begin{english}
	A esto, bhikkhus, se le llama Visión Correcta.
\end{english}

Katamo ca bhi꜓kkh꜕ave sammā-sa꜓ṅka꜕ppo

\begin{english}
	¿Y qué, bhikkhus, es Correcta Intención?
\end{english}

\ifaivedition
\clearpage
\fi

Nekkhamma-sa꜓ṅka꜕ppo

\begin{english}
	La intención de la renuncia;
\end{english}

A꜕byāpāda-sa꜓ṅka꜕ppo

\begin{english}
	La intención de no-mala fe;
\end{english}

A꜕vihiṁsā-sa꜓ṅka꜕ppo

\begin{english}
	La intención de no-crueldad:
\end{english}

Ayaṁ vuccati bhi꜓kkh꜕ave sa꜓mmā-sa꜓ṅka꜕ppo

\begin{english}
	A esto, bhikkhus, se le llama Correcta Intención.
\end{english}

Katamā ca bhi꜓kkh꜕ave sa꜓mmā-vācā

\begin{english}
	¿Y qué, bhikkhus, es Lenguaje Correcto?
\end{english}

Musā-vādā vera꜓ma꜕ṇī

\trline{Abstenerse de mentir;}

Pisuṇāya vācāya vera꜓ma꜕ṇī

\trline{Abstenerse de lenguaje malicioso y dañino;}

Pharusāya vācāya vera꜓ma꜕ṇī

\trline{Abstenerse de lenguaje grosero;}

Sa꜓mphappa꜕lāpā vera꜓ma꜕ṇī.

\trline{Abstenerse de hablar por hablar:}

\ifaivedition
\clearpage
\fi

Ayaṁ vuccati bhi꜓kkh꜕ave sa꜓mmā-vācā

\begin{english}
	A esto, bhikkhus, se le llama Lenguaje Correcto.
\end{english}

Katamo ca bhi꜓kkh꜕ave sa꜓mmā-kammanto

\begin{english}
	¿Y qué, bhikkhus, es Correcta Acción?
\end{english}

Pāṇāti꜕pātā vera꜓ma꜕ṇī

\begin{english}
	Abstenerse de matar seres vivos;
\end{english}

A꜕dinnādānā vera꜓ma꜕ṇī

\begin{english}
	Abstenerse de tomar lo que no se da;
\end{english}

Kāmesu꜕ micchā꜓cārā vera꜓ma꜕ṇī

\begin{english}
	Abstenerse de conducta sexual inapropiada:
\end{english}

Ayaṁ vuccati bhi꜓kkh꜕ave sa꜓mmā-kammanto

\begin{english}
	A esto, bhikkhus, se le llama Correcta Acción.
\end{english}

Katamo ca bhi꜓kkha꜕ve sa꜓mmā-ājīvo

\begin{english}
	¿Y qué, bhikkhus, es Sustento Correcto?
\end{english}

Idha bhi꜓kkh꜕ave a꜕riya-sā꜓va꜕ko micchā-ājīvaṁ pa꜕hāya sammā-ājī꜓vena jīvi꜓taṁ ka꜕ppeti

\begin{english}
	Aquí, bhikkhus, un Noble Discipulo, habiendo abandonado sustento incorrecto, se gana la vida mediante Sustento Correcto:
\end{english}

Ayaṁ vuccati bhi꜓kkh꜕ave sa꜓mmā-ājīvo

\begin{english}
	A esto, bhikkhus, se le llama Sustento Correcto.
\end{english}

Katamo ca bhi꜓kkh꜕ave sa꜓mmā-vāyāmo

\begin{english}
	¿Y qué, bhikkhus, es Correcto Esfuerzo?
\end{english}

Idha bhi꜓kkh꜕ave bhikkhu a꜕nuppannānaṁ pāpa꜕kānaṁ a꜕ku꜕salānaṁ dhammānaṁ anuppādāya

\begin{english}
	Aquí, bhikkhus, un bhikkhu genera interés en la no aparición de estados dañinos que aún no han aparecido;
	
\end{english}

Chandaṁ ja꜕neti vāyama꜓ti vī꜓ri꜓yaṁ ārabha꜕ti ci꜕ttaṁ pa꜕ggaṇhā꜓ti pa꜕daha꜕ti

\begin{english}
	Se esfuerza, despierta energía, empeña su mente y lucha.
\end{english}

U꜕ppannānaṁ pāpa꜕kānaṁ a꜕ku꜕salānaṁ dhammānaṁ pa꜕hānāya

\begin{english}
	Genera interés en el abandono de estados dañinos que han aparecido;
\end{english}

Chandaṁ ja꜕neti vāyama꜓ti vī꜓ri꜓yaṁ ārabha꜕ti ci꜕ttaṁ pa꜕ggaṇhā꜓ti pa꜕daha꜕ti

\begin{english}
	Se esfuerza, despierta energía, empeña su mente y lucha.
\end{english}

Anuppannānaṁ ku꜕salānaṁ dhammānaṁ u꜕ppādāya

\begin{english}
	Genera interés en la aparición de estados beneficiosos que aún no han aparecido;
\end{english}

Chandaṁ ja꜕neti vāyama꜓ti vī꜓ri꜓yaṁ ārabha꜕ti ci꜕ttaṁ pa꜕ggaṇhā꜓ti pa꜕daha꜕ti

\begin{english}
	Se esfuerza, despierta energía, empeña su mente y lucha.
\end{english}

\ifaivedition
\clearpage
\fi

U꜕ppannānaṁ ku꜕salānaṁ dhammānaṁ ṭh꜓iti꜕yā a꜕sa꜕mmosāya bh꜓iyyobhāvāya vepu꜕llāya bhāva꜓nāya pāri꜓pū꜕riyā

\begin{english}
	Genera interés en la continuación, no-desaparición, refuerzo, incremento y plenitud a través del desarrollo de estados beneficiosos que han aparecido;
\end{english}

Chandaṁ ja꜕neti vāyama꜓ti vī꜓ri꜓yaṁ ārabha꜕ti ci꜕ttaṁ pa꜕ggaṇhā꜓ti pa꜕daha꜕ti

\begin{english}
	Se esfuerza, despierta energía, empeña su mente y lucha.
\end{english}

Ayaṁ vuccati bhi꜓kkh꜕ave sa꜓mmā-vāyāmo

\begin{english}
	A esto, bhikkhus, se llama Esfuerzo Correcto.
\end{english}

Katamā ca bhi꜓kkh꜕ave sa꜓mmā-sa꜕ti

\begin{english}
	¿Y qué, bhikkhus, es Sati Correcto?
\end{english}

Idha bhi꜓kkh꜕ave bhikkhu kāye kāyānupa꜕ssī vi꜓ha꜕rati

\begin{english}
	Aquí, bhikkhus, un bhi꜕kkhu꜕ permanece contemplando el cuerpo como cuerpo,
\end{english}

Ātāpī sa꜓mpa꜕jāno sa꜕timā

\begin{english}
	Energético, comprendiendo claramente y consciente,
\end{english}

Vi꜓neyya loke a꜕bhijjhā-domanassaṁ

\begin{english}
	Habiendo abandonado deseo y pesar por el mundo;
\end{english}

Veda꜕nāsu꜕ veda꜕nānu꜓pa꜕ssī vi꜓ha꜕rati

\begin{english}
	Permanece contemplando sensaciones como sensaciones,
\end{english}

\ifaivedition
\clearpage
\fi

Ātāpī sa꜓mpa꜕jāno sa꜕timā

\begin{english}
	Energético, comprendiendo claramente y consciente,
\end{english}

Vi꜓neyya loke a꜕bhijjhā-domanassaṁ

\begin{english}
	Habiendo abandonado deseo y pesar por el mundo;
\end{english}

Ci꜕tte ci꜕ttānu꜓pa꜕ssī vi꜓ha꜕rati

\begin{english}
	Permanece contemplando mente como mente,
\end{english}

Ātāpī sa꜓mpa꜕jāno sa꜕timā

\begin{english}
	Energético, comprendiendo claramente y consciente,
\end{english}

Vi꜓neyya loke a꜕bhijjhā-domanassaṁ

\begin{english}
	Habiendo abandonado deseo y pesar por el mundo;
\end{english}

Dhammesu꜕ dhammānu꜓pa꜕ssī vi꜓ha꜕rati

\begin{english}
	Permanece contemplando fenómenos mentales como fenómenos mentales,
\end{english}

Ātāpī sa꜓mpa꜕jāno sa꜕timā

\begin{english}
	Energético, comprendiendo claramente y consciente,
\end{english}

Vi꜓neyya loke a꜕bhijjhā-domanassaṁ

\begin{english}
	Habiendo abandonado deseo y pesar por el mundo:
\end{english}

Ayaṁ vuccati bhi꜓kkh꜕ave sa꜓mmā-sa꜕ti

\begin{english}
	A esto, bhikkhus, se le  llama Sati Correcto.
\end{english}

\ifaivedition
\clearpage
\fi

Katamo ca bhi꜓kkh꜕ave sa꜓mmā-sa꜕mādhi

\begin{english}
	¿Y qué es, bhikkhus, Correcta Concentración?
\end{english}

Idha bhi꜓kkh꜕ave bhikkhu

\begin{english}
	Aquí, bhikkhus, un bhi꜕kkhu,
\end{english}

Vivicc'eva kāmehi

\begin{english}
	Bien apartado de placeres sensuales,
\end{english}

Vivicca a꜕ku꜕sa꜕lehi dh꜕ammehi

\begin{english}
	Apartado de estados dañinos,
\end{english}

Sa꜕vi꜓ta꜕kkaṁ sa꜕vi꜓cāraṁ viveka꜕-jaṁ pīti꜕-sukhaṁ pa꜕ṭhamaṁ jhānaṁ upasa꜓mpajja vi꜓ha꜕rati

\begin{english}
	El entra y permanece en el primer jhāna --- acompañado de pensamiento aplicado y sostenido, acompañado de éxtasis y placer nacidos de la reclusión.
\end{english}

Vi꜓takka-vicārānaṁ vūpa꜕samā

\begin{english}
	Con el cese de pensamiento aplicado y sostenido,
\end{english}

Ajjhattaṁ sa꜓mpa꜕sādanaṁ ceta꜕so ekodi꜓bhāvaṁ avi꜓ta꜕kkaṁ avi꜓cāraṁ sa꜕mādhi꜓-jaṁ pīti꜕-sukhaṁ du꜕tiyaṁ jhānaṁ upasa꜓mpa꜕jja vi꜓ha꜕rati

\begin{english}
	Él entra y permanece en el segundo jhāna  --- acompañado de firmeza interior y unificación mental, desprovisto de pensamiento aplicado y sostenido, acompañado de éxtasis y placer nacidos de la concentración.
\end{english}

Pītiyā ca꜕ vi꜓rāgā

\begin{english}
	Con el cese del éxtasis
\end{english}

U꜕pekkhako ca vi꜓ha꜕rati

\begin{english}
	Él permanece en ecuanimidad,
\end{english}

Sa꜕to ca꜕ sa꜓mpa꜕jāno

\begin{english}
	Consciente y comprendiendo claramente,
\end{english}

Su꜕khañca kāyena pa꜕ṭisa꜓ṁvedeti

\begin{english}
	Todavía experimentando placer con el cuerpo,
\end{english}

Yaṁ taṁ a꜕riyā āci꜕kkhanti u꜕pekkha꜓ko sa꜕timā su꜕kha-vi꜓hā꜕rī'ti tatiyaṁ jhānaṁ u꜕pasa꜓mpa꜕jja vi꜓ha꜕rati

\begin{english}
	Él entra y permanece en el tercer jhāna --- el cual los Nobles elogian como un estado caracterizado por la presencia de bienestar, ecuanimidad y sati.
\end{english}

Sukhassa ca꜕ pahānā

\begin{english}
	Abandonando el placer
\end{english}

Dukkhassa ca꜕ pahānā

\begin{english}
	y abandonando el dolor,
\end{english}

Pu꜕bb'eva somanassa꜕-domanassā꜓naṁ a꜕tthaṅga꜕mā

\begin{english}
	Con la desaparición previa de deleite y pesar,
\end{english}

Adukkham-asu꜕khaṁ u꜕pekkhā-sa꜕ti-pā꜕ri꜓su꜕ddhiṁ ca꜕tutthaṁ jhānaṁ u꜕pasa꜓mpa꜕jja vi꜓ha꜕rati

\begin{english}
	Él entra y permanece en el cuarto jhāna
	--- desprovisto de dolor o placer, y equipado con puro sati basado en ecuanimidad:
\end{english}

Ayaṁ vuccati bhi꜓kkh꜕ave sa꜓mmā-sa꜕mādhi

\begin{english}
	A esto, bhikkhus, se le llama Correcta Co꜕ncentración.
\end{english}

Ayam-eva a꜕riyo aṭṭha꜓ṅgi꜕ko maggo

\begin{english}
	Este es el Óc꜕tuple Noble Sende꜓ro.
\end{english}

\chapter[Esfuerzo acorde con el Dhamma]{La Enseñanza sobre el Esfuerzo acorde con el Dhamma}

% dutiyadasabalasutta: nt. Saṃyutta Nikāya 12.22 (SN12.22)

\firstline{Evaṁ svākkhāto bhikkhave mayā dhammo}

\begin{leader}
	[Handa mayaṁ dhamma-pahaṁsāna-pāṭham bhaṇāmase]
\end{leader}

Evaṁ svā꜕kkhāto bhi꜓kkh꜕ave mayā dhammo

\begin{english}
	Bhikkhus, el Dhamma ha sido bien expuesto por mí,
\end{english}

Uttāno

\trline{bien elucidado,}

Vi꜓va꜕ṭo

\trline{bien declarado,}

Pa꜕kāsi꜓to

\trline{bien revelado,}

Chi꜓nna-pi꜕loti꜓ko

\trline{desprovisto de remiendos ---}

Alam-eva sa꜕ddhā-pa꜕bbaj꜓itena kula-pu꜕ttena vī꜓riyaṁ ā꜕rabh꜕ituṁ

\begin{english}
	Esto debería ser suficiente para que aquellos que han dejado la vida laica con fe, despierten su energía de la siguiente manera:
\end{english}

Kāmaṁ ta꜕co ca nahā꜓ru c꜕a aṭṭhi c꜕a a꜕vasi꜓ss꜕atu

\begin{english}
	`Que queden solamente mi piel, tendones y huesos.
\end{english}

Sa꜕rīre u꜕pasuss꜓atu maṁsa꜕-lohi꜕taṁ

\begin{english}
	Que se seque mi carne y mi sangre!
\end{english}

Yaṁ taṁ pu꜕risa-thāmena

\trline{Mientras que aquello que puede ser obtenido por fuerza humana,}

Pu꜕risa-vī꜓riyena

\trline{vigor humano,}

Pu꜕risa-pa꜕rakk꜕amena

\trline{energía humana,}

Pa꜕tta꜕bbaṁ na taṁ a꜕pāpu꜕ṇitvā

\trline{no haya sido obtenido,}

Vī꜓riyassa sa꜓ṇṭhānaṁ bha꜕vissa꜕tī'ti

\trline{mi esfuerzo no cesará.'}

Dukkhaṁ bhi꜓kkh꜕ave kusī꜓to vi꜓ha꜕rati

\begin{english}
	Bhikkhus, la persona vaga vive en sufrimiento,
\end{english}

Voki꜕ṇṇo pāpa꜕kehi a꜕ku꜕saleh꜕i dhammehi

\begin{english}
	manchada por estados dañinos y malignos,
\end{english}

Maha꜓ntañca sa꜕da꜕tthaṁ pa꜕ri꜓hāpeti

\begin{english}
	y grande es el bien personal que neglige.
\end{english}

\ifaivedition
\clearpage
\fi

Āraddha-vī꜓riyo c꜕a kho bhi꜓kkh꜕ave su꜕khaṁ vi꜓ha꜕rati

\begin{english}
	La persona energética vive en felicidad,
\end{english}

Pa꜕vivitto pāpa꜕keh꜕i a꜕ku꜕saleh꜕i dhammehi

\begin{english}
	bien apartado de estados dañinos y malignos,
\end{english}

Maha꜓ntañca sa꜕da꜕tthaṁ pa꜕ri꜓pūreti

\begin{english}
	y grande es el bien personal que conquista.
\end{english}

Na bhi꜓kkh꜕ave hī꜕nena a꜕gga꜕ssa꜕ pa꜕tt꜓i hoti

\begin{english}
	Bhikkhus, no es por medio de aquello que es vulgar que lo supremo es alcanzado.
\end{english}

Aggena ca kho bhi꜓kkh꜕ave a꜕gga꜕ssa꜕ pa꜕tt꜓i hoti

\begin{english}
	Sino que, bhikkhus, es por medio de lo supremo que lo supremo es alcanzado.
\end{english}

Maṇḍape꜓yyam-i꜓daṁ bhi꜓kkh꜕ave brahmaca꜕ri꜓yaṁ

\begin{english}
	Bhikkhus, esta vida santa es como la creme de la creme:
\end{english}

Satthā sammukhī꜓-bhū꜕to

\begin{english}
	el Maestro está presente.
\end{english}

Tasmāti꜕ha bhi꜓kkh꜕ave vī꜓riyaṁ ārabha꜕tha

\begin{english}
	Por tanto, bhikkhus, empezad a despertar vuestra energía
\end{english}

A꜕ppa꜕tta꜕ssa꜕ pa꜕tt꜓iyā

\begin{english}
	para la obtención de aquello que aún no ha sido obtenido,
\end{english}

\ifaivedition
\clearpage
\fi

Anadhi꜓ga꜕tassa a꜕dhiga꜕māya

\begin{english}
	para conquistar aquello que aún no ha sido conquistado,
\end{english}

Asa꜕cchi꜕ka꜕tassa sa꜕cchi꜕ki꜕ri꜓yāya

\begin{english}
	para realizar aquello que aún no ha sido realizado.
\end{english}

Evaṁ no ayaṁ amhākaṁ pa꜕bb꜕ajjā a꜕vaṅka꜕tā a꜕vañjhā bha꜕vi꜓ssati

\begin{english}
	Pensando de esta forma: ‘Mi ordenación no será en vano,
\end{english}

Sa꜕phalā s꜕a-u꜕dra꜓yā

\begin{english}
	pero será fértil y fructífera,
\end{english}

Yesa꜓ṁ mayaṁ pa꜕ribhuñjāma cīva꜓ra-piṇḍa꜕pāta-se꜓nāsana-\\
gi꜓lānappa꜕ccaya-bhesa꜕jja-parikkhā꜓raṁ tesaṁ te kārā a꜕mhesu

\begin{english}
	y mi uso de las túnicas, comida, alojamiento y medicinas, dados por otros para mi soporte,
\end{english}

Ma꜕happh꜕alā bhavissanti ma꜕hāni꜕sa꜓ṁsā'ti

\begin{english}
	será de gran recompensa para aquellos que me lo dieron.’
\end{english}

Evaṁ hi꜕ vo bhi꜓kkh꜕ave si꜕kkh꜕it꜕abbaṁ

\begin{english}
	Bhikkhus, debéis entrenaros de esta manera:
\end{english}

A꜕tt'atthaṁ vā hi bhi꜓kk꜕have sa꜓mpassa꜕mānena

\begin{english}
	Considerando vuestro propio bien,
\end{english}

A꜕lam-eva a꜕ppamādena sa꜓mpādetuṁ

\begin{english}
	debería ser suficiente para motivaros a alcanzar el objetivo, libre de negligencia;
\end{english}

Pa꜕r'atthaṁ vā hi bhi꜓kkh꜕ave sa꜓mpass꜕amānena

\begin{english}
	Bhikkhus, considerando el bien de otros,
\end{english}

A꜕lam-eva a꜕ppamādena sa꜓mpāde꜕tuṁ

\begin{english}
	debería ser suficiente para motivaros a alcanzar el objetivo, libre de negligencia;
\end{english}

U꜕bhay'atthaṁ vā hi bhi꜓kkh꜕ave sa꜓mpassa꜕mānena

\begin{english}
	Bhikkhus, considerando el bien de ambos,
\end{english}

Alam-eva a꜕ppamādena sa꜓mpāde꜕tun'ti

\begin{english}
	debería ser suficiente para motivaros a alcanzar el objetivo, libre de negligencia.
\end{english}

\chapter{Los Versos de Tāyana}

\firstline{Chinda sotaṁ parakkamma}

\begin{leader}
	[Handa mayaṁ tāyana-gāthāyo bhaṇāmase]
\end{leader}

\begin{twochants}
	Chi꜓nda so꜕taṁ pa꜕rakkamma & kā꜕me panūda brā꜓hm꜕aṇa \\
	Nappahā꜓ya mu꜕ni kāme & n'ekattam-upa꜕pajja꜕ti \\
\end{twochants}

\begin{english}
	Oh brahmin, esfuérzate en destruir la corriente\\
	Abandona los placeres sensuales\\
	Sin dejar ir de los placeres sensuales,\\
	Un sabio jamás alcanzará unificación.
\end{english}

\begin{twochants}
	Kayirā ce ka꜕yirāthe꜓naṁ & da꜕ḷham-enaṁ pa꜕rakka꜕me \\
	Sithilo hi pa꜕ribbājo & bh꜕iyyo ākira꜕te ra꜕jaṁ \\
\end{twochants}

\begin{english}
	Vigorosamente, con toda tu fuerza,\\
	Haz aquello que debe ser hecho;\\
	Una vida monástica vivida laxamente\\
	agita aún más las impurezas mentales.
\end{english}

\begin{twochants}
	A꜕kataṁ dukkaṭaṁ se꜓yyo & pacchā tappati du꜓kk꜕aṭaṁ \\
	Katañca su꜕ka꜓taṁ seyyo & yaṁ ka꜕tvā nānuta꜕ppa꜕ti \\
\end{twochants}

\begin{english}
	Mejor no realizar malos actos\\
	que más tarde traigan remordimiento;\\
	Mejor es realizar buenas acciones\\
	que hechas, uno no se arrepienta.
\end{english}

\clearpage

\begin{twochants}
	Kuso꜓ ya꜕thā du꜕ggahi꜕to & hattham-evā꜓nu꜕kant꜕ati \\
	Sā꜓maññaṁ du꜕pparāma꜕ṭṭhaṁ & nirayāyūpa꜕kaḍḍh꜕ati \\
\end{twochants}

\begin{english}
	Como el césped cogido incorrectamente,\\
	corta la mano del que lo toma\\
	Así, la vida monástica mal vivida\\
	le lleva a uno al Infierno.
\end{english}

\begin{twochants}
	Yaṁ kiñci si꜕thi꜓laṁ kammaṁ & sa꜓ṅki꜕liṭṭha꜓ñca꜕ yaṁ va꜓taṁ \\
	Sa꜓ṅka꜕ssaraṁ brahma-ca꜕ri꜓yaṁ & na taṁ ho꜓ti ma꜕happh꜕alan'ti \\
\end{twochants}

\begin{english}
	Actos realizados de manera negligente,\\
	cualquier voto mantenido suciamente,\\
	vida monástica deshonesta,\\
	nunca resultarán en gran fruto.
\end{english}

\chapter[Aparihāniya-dhamma-sutta]{Bhikkhu-aparihāniya-dhamma-sutta}

\emph{Siete condiciones para la buenaventura de los bhikkhus, AN 7.23}

\begin{leader}
  [Handa mayaṁ bhikkhu-aparihāniya-dhamma-suttaṁ bhaṇāmase]
\end{leader}

[Evaṁ me sutaṁ.] Ekaṁ samayaṁ bhagavā rājagahe꜔꜒ viharati gijjhakūṭe pabbate.
Tatra kho꜔꜒ bhagavā bhikkhū꜔꜒ āmantesi: Satta vo, bhikkhave, aparihā꜔꜒niye dhamme
desessā꜔꜒mi. Taṁ suṇātha, sā꜔꜒dhukaṁ manasi karotha, bhāsissā꜔꜒mī'ti. Evaṁ, bhante'ti
kho꜔꜒ te bhikkhū꜔꜒ bhagavato paccasso꜔꜒su꜔꜒ṁ. Bhagavā etadavoca:

\begin{english}
  He oído que en cierta ocasión el Excelso estaba en Rajagaha, en el Pico de los Buitres. Allí, él se dirigió a los monjes: ‘Monjes, voy a enseñaros las siete condiciones que llevan a la no-decadencia. Oíd y prestad mucha atención. Voy a hablar.’ ‘Sí, Maestro’, respondieron los monjes. El Excelso dijo:
\end{english}

[1] Katame ca, bhikkhave, satta aparihā꜔꜒niyā dhammā? Yāvakīvañca, bhikkhave, bhikkhū꜔꜒
abhiṇha꜔꜒ṁ sa꜔꜒nnipātā bhavissa꜔꜒nti sa꜔꜒nnipātabahulā; vuddhiyeva, bhikkhave,
bhikkhū꜔꜒naṁ pāṭikaṅkhā꜔꜒, no parihā꜔꜒ni.

\begin{english}
  ‘¿Y cuáles son las siete condiciones que llevan a la no-decadencia? Mientras los monjes
  se reúnan con frecuencia, se reúnan asiduamente; su crecimiento puede ser esperado, no su decadencia.’

\end{english}

[2] Yāvakīvañca, bhikkhave, bhikkhū꜔꜒ samaggā sa꜔꜒nnipatissa꜔꜒nti, samaggā
vuṭṭhahissa꜔꜒nti, samaggā sa꜔꜒ṅghakaraṇīyāni karissa꜔꜒nti; vuddhiyeva, bhikkhave,
bhikkhū꜔꜒naṁ pāṭikaṅkhā꜔꜒, no parihā꜔꜒ni.

\begin{english}
  ‘Mientras los monjes se reúnan en armonía, se dispersen en armonía y conduzcan los asuntos de la Saṅgha en armonía; su crecimiento puede ser
  esperado, no su decadencia.’
\end{english}

[3] Yāvakīvañca, bhikkhave, bhikkhū꜔꜒ apaññattaṁ na paññāpessa꜔꜒nti, paññattaṁ na
samucchi꜔꜒ndissa꜔꜒nti, yathā꜔꜒paññattesu sikkhā꜔꜒padesu samādāya vattissa꜔꜒nti;
vuddhiyeva, bhikkhave, bhikkhū꜔꜒naṁ pāṭikaṅkhā꜔꜒, no parihā꜔꜒ni.

\begin{english}
  ‘Mientras los monjes no decreten lo que no fue decretado, no revoquen lo que
  fue decretado, pero practiquen el cumplimiento de las reglas de entrenamiento conforme
  fueron decretadas; su crecimiento puede ser
  esperado, no su decadencia.’
\end{english}

[4] Yāvakīvañca, bhikkhave, bhikkhū꜔꜒ ye te bhikkhū꜔꜒ the꜔꜒rā rattaññū cirapabbajitā
sa꜔꜒ṅghapitaro sa꜔꜒ṅghapariṇāyakā te sakkarissa꜔꜒nti garuṁ karissa꜔꜒nti mānessa꜔꜒nti
pūjessa꜔꜒nti, tesa꜔꜒ñca so꜔꜒tabbaṁ maññissa꜔꜒nti; vuddhiyeva, bhikkhave, bhikkhū꜔꜒naṁ
pāṭikaṅkhā꜔꜒, no parihā꜔꜒ni.

\begin{english}
  ‘Mientras los monjes honren, respeten, veneren y rindan homenaje a los
  monjes con más antigüedad  -- aquellos que fueron ordenados hace mucho tiempo, los padres de la Saṅgha, los líderes de la Saṅgha -- considerando muy valioso el oírlos; su crecimiento puede ser
  esperado, no su decadencia.’
\end{english}

\enlargethispage{2\baselineskip}

[5] Yāvakīvañca, bhikkhave, bhikkhū꜔꜒ uppannāya taṇhā꜔꜒ya ponobhavikāya na vasa꜔꜒ṁ
gacchissa꜔꜒nti; vuddhiyeva, bhikkhave, bhikkhū꜔꜒naṁ pāṭikaṅkhā꜔꜒, no parihā꜔꜒ni.

\begin{english}
  ‘Mientras los monjes no se sometan al poder de cualquier deseo que surja y
  que lleve a un futuro nacimiento; su crecimiento puede ser
  esperado, no su decadencia.’
\end{english}

[6] Yāvakīvañca, bhikkhave, bhikkhū꜔꜒ āraññakesu se꜔꜒nāsanesu sā꜔꜒pekkhā꜔꜒ bhavissa꜔꜒nti;
vuddhiyeva, bhikkhave, bhikkhū꜔꜒naṁ pāṭikaṅkhā꜔꜒, no parihā꜔꜒ni.

\begin{english}
  ‘Mientras los monjes se regocijen viviendo en el bosque;
  su crecimiento puede ser esperado, no su decadencia.’
\end{english}

[7] Yāvakīvañca, bhikkhave, bhikkhū꜔꜒ paccattaññeva satiṁ upaṭṭhā꜔꜒pessa꜔꜒nti: Kinti
anāgatā ca pesalā sabrahmacārī āgacche꜔꜒yyuṁ, āgatā ca pesalā sabrahmacārī phā꜔꜒su꜔꜒ṁ
vihareyyun'ti; vuddhiyeva, bhikkhave, bhikkhū꜔꜒naṁ pāṭikaṅkhā꜔꜒, no parihā꜔꜒ni.

\begin{english}
  ‘Mientras cada uno de los monjes mantenga firmemente en su mente: ‘Si hubieran
  compañeros con buen comportamiento, seguidores de la vida casta que aún están por
  venir, que puedan ellos venir; y que los compañeros con buen comportamiento de la vida casta
  que vinieron puedan vivir en paz’; su crecimiento puede ser esperado, no su decadencia.’
\end{english}

Yāvakīvañca, bhikkhave, ime satta aparihā꜔꜒niyā dhammā bhikkhū꜔꜒su ṭhassa꜔꜒nti, imesu
ca sattasu aparihā꜔꜒niyesu dhammesu bhikkhū꜔꜒ sa꜔꜒ndississa꜔꜒nti; vuddhiyeva, bhikkhave,
bhikkhū꜔꜒naṁ pāṭikaṅkhā꜔꜒, no parihā꜔꜒nī'ti. Idam-avoca bhagavā. Attamanā te bhikkhū꜔꜒
bhagavato bhāsitaṁ abhinandun'ti.

\begin{english}
  ‘Mientras los monjes permanezcan resueltos en estas siete condiciones, y mientras
  estas siete condiciones persistan entre los monjes, se puede esperar su
  crecimiento, no su decadencia.’ Esto es lo que el Excelso dijo. Satisfechos,
  los monjes se deleitaron en las palabras del Excelso.
\end{english}
