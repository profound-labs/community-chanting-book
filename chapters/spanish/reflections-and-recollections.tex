\chapter*[Compartir Mérito]{Reflexión para compartir Mérito}

\delegateSetUseNext

\begin{leader}
  [Ha꜓nda mayaṁ uddissanādhiṭṭhāna-gāthā꜓yo b꜕haṇāmase]
\end{leader}

\firstline{Iminā puññakammena upajjhāyā guṇuttarā}

[Iminā puñña꜕kammena] u꜕pajjhāyā gu꜕ṇutta꜕rā\\
Ācariyūpa꜕kārā ca꜕ mātāpitā ca꜕ ñāta꜕kā\\
Suriyo candimā rājā gu꜕ṇavantā na꜕rāpi꜕ ca꜕\\
Brahma-mārā ca꜕ indā ca꜕ loka꜕pālā ca꜕ deva꜕tā\\
Yamo mittā ma꜕nussā ca majjhattā veri꜕kāpi꜕ ca꜕\\
Sa꜕bbe sattā sukhī hontu puññāni pa꜕ka꜕tāni꜕ me\\
Sukhañca tividhaṁ dentu꜕ khippaṁ pāpetha꜕ voma꜕taṁ\\
Iminā puññakammena iminā uddi꜕ssena꜕ ca꜕\\
Khipp'āhaṁ su꜕la꜕bhe ceva taṇhūpādāna꜕-cheda꜕naṁ\\
Ye santāne hīnā dhammā yāva꜕ nibbāna꜕to ma꜕maṁ\\
Nassantu sabba꜕dā yeva yattha꜕ jāto bha꜕ve bha꜕ve\\
Ujucittaṁ sa꜕ti꜕paññā sallekho vi꜕ri꜕yamhinā\\
Mārā labhantu nokāsaṁ kātuñca vi꜕ri꜕yes꜕u me\\
Buddhādhipa꜕va꜕ro nātho dhammo nātho va꜕rutta꜕mo\\
Nātho pacceka꜕buddho ca꜕ saṅgho nāthotta꜕ro ma꜕maṁ\\
Tesottamānubhāvena mārokāsaṁ la꜕bhantu꜕ mā

\chapter[Compartir Mérito]{Reflexión para Compartir Mérito}

\enlargethispage{2\baselineskip}

\begin{leader}
  [Cantemos ahora la Reflexión para Compartir Mérito.]
\end{leader}

\firstline{A través del bien que resulta de mi práctica}

A través del bien que resulta de mi práctica,\\
Que mis maestros y guías espirituales de gran virtud,\\
A mi madre, mi padre y mis familiares,\\
El Sol y la Luna, y todos los líderes virtuosos del mundo,\\
Que los Dioses más elevados y las fuerzas del mal,\\
Seres celestiales, espíritus guardianes de la Tierra\\ \vin y el Señor de la Muerte,\\
Aquellos que son amigables, indiferentes u hostiles,\\
Que todos los seres reciban las bendiciones de mi vida.\\
Que brevemente lleguen a la Triple Bendición, y superen la muerte.\\
A través del mérito que resulta de mi práctica,\\
Y a través de este compartir,\\
Que todos los deseos y apegos rápidamente cesen,\\
Así como los estados prejudiciales de la mente.\\
Hasta realizar el Nibbāna,\\
En cualquier tipo de nacimiento, que yo tenga una mente justa,\\
Con consciencia y sabiduría, austeridad y vigor.\\
Que las fuerzas ilusorias no controlen,\\
\vin ni enflaquezcan mi decisión.

El Buddha es mi excelente refugio,\\
Insuperable es la protección del Dhamma,\\
El Buddha solitario es mi Noble ejemplo,\\
La Saṅgha es mi mayor apoyo.

Que a través de esta supremacía\\
Desaparezcan la oscuridad y la ilusión.

\chapter*[Metta Sutta]{Metta Sutta}

\delegateSetUseNext

\firstline{Karaṇīyam-attha-kusalena}

\begin{leader}
  [Ha꜓nda mayaṁ metta-sutta-gāthā꜓yo bha꜕ṇāmase]
\end{leader}

[Karaṇīyam-attha-kusalena]\\
Yan-taṁ santaṁ padaṁ abhisamecca\\
Sakko ujū ca suhujū ca\\
Suvaco c'assa mudu anatimānī

Santussako ca subharo ca\\
Appakicco ca sallahuka-vutti\\
Sant'indriyo ca nipako ca\\
Appagabbho kulesu ananugiddho

Na ca khuddaṁ samācare kiñci\\
Yena viññū pare upavadeyyuṁ\\
Sukhino vā khemino hontu\\
Sabbe sattā bhavantu sukhit'attā

Ye keci pāṇa-bhūt'atthi\\
Tasā vā thāvarā vā anavasesā\\
Dīghā vā ye mahantā vā\\
Majjhimā rassakā aṇuka-thūlā

Diṭṭhā vā ye ca adiṭṭhā\\
Ye ca dūre vasanti avidūre\\
Bhūtā vā sambhavesī vā\\
Sabbe sattā bhavantu sukhit'attā

\chapter[Metta Sutta]{Metta Sutta}

\firstline{Esto es lo que se debe hacer }

\begin{leader}
  [Cantemos ahora las palabras de Buddha \\ sobre el Amor y la Compasión.]
\end{leader}

Esto es lo que se debe hacer\\
para cultivar la bondad\\
y seguir la vía de la paz:\\
Ser capaz y ser honesto,\\
franco y gentil en el hablar.\\
humilde y no arrogante,\\
contento, fácilmente satisfecho,\\
liberado de deberes y frugal en su camino.

Pacífico y sereno, sabio e inteligente,\\
sin orgullo, sin exigencias por naturaleza.\\
Que no haga nada,\\
que los sabios puedan reprender.\\
deseando: En alegría y seguridad,\\
Que todos los seres sean felices.\\
Cualesquieran que sean los seres vivos,\\
débiles, fuertes, sin excepción.\\
del más grande al más pequeño,\\
visibles o invisibles,\\
ya sea cerca o sea lejos,\\
nacidos o por nacer: \\
¡Que todos los seres sean felices!

\clearpage

Na paro paraṁ nikubbetha\\
Nātimaññetha katthaci naṁ kiñci\\
Byārosanā paṭighasaññā\\
Nāññam-aññassa dukkham-iccheyya

Mātā yathā niyaṁ puttaṁ\\
Āyusā eka-puttam-anurakkhe\\
Evam'pi sabba-bhūtesu\\
Mānasam-bhāvaye aparimāṇaṁ

Mettañca sabba-lokasmiṁ\\
Mānasam-bhāvaye aparimāṇaṁ\\
Uddhaṁ adho ca tiriyañca\\
Asambādhaṁ averaṁ asapattaṁ

Tiṭṭhañ-caraṁ nisinno vā\\
Sayāno vā yāvat'assa vigata-middho\\
Etaṁ satiṁ adhiṭṭheyya\\
Brahmam-etaṁ vihāraṁ idham-āhu

Diṭṭhiñca anupagamma\\
Sīlavā dassanena sampanno\\
Kāmesu vineyya gedhaṁ\\
Na hi jātu gabbha-seyyaṁ punaretī'ti

\clearpage

Que no engañe a nadie,\\
ni desprecie a nadie en ninguna condición.\\
Que nadie, por ira o mala fe,\\
desee el mal de otros.

Igual que una madre protege a su hijo,\\
su único hijo, con su vida.\\
Así, de corazón infinito,\\
se debería estimar todo ser vivo;\\
irradiando ternura por todo el mundo:\\
Arriba al más alto cielo,\\
y abajo hasta las profundidades;\\
radiante y sin límites,\\
libre de odio y mala fe.\\
Sea parado o andando,\\
sentado o reclinado,\\
libre de torpor,\\
Esta es una memoria a mantener,\\
la verdadera casa divina.\\

Puro de corazón, con claridad de visión,\\
sin insistir en ideas fijas,\\
liberado de los deseos sensuales,\\
no volverá a nacer en este mundo.

\chapter[Once Beneficios]{Once Beneficios de la Práctica de Metta}

\emph{Mettānisaṁsa Sutta, AN 11.15}

% https://suttacentral.net/an11.15/pli/ms

\begin{leader}
  [Handa mayaṁ mettānisaṁsa-suttaṁ bhaṇāmase]
\end{leader}

Mettāya, bhikkhave, cetovimuttiyā āsevitāya bhāvitāya bahulīkatāya yānīkatāya vatthukatāya anuṭṭhitāya paricitāya susamāraddhāya ekādasānisaṁsā pāṭikaṅkhā. Katame ekādasa?

\begin{english}
  Monjes, hay once beneficios que se pueden esperar como resultado de la liberación del corazón por la práctica de Metta, conociéndola, cultivándola, desarrollándola, teniéndola como guía, estimándola, siguiéndola, consolidándola e implementándola.\\
  Cuales son los once?
\end{english}

Sukhaṁ supati.\\
Sukhaṁ paṭibujjhati.\\
Na pāpakaṁ supinaṁ passati.\\
Manussānaṁ piyo hoti.\\
Amanussānaṁ piyo hoti.\\
Devatā rakkhanti.\\
Nāssa aggi vā visaṁ vā satthaṁ vā kamati.\\
Tuvaṭaṁ cittaṁ samādhiyati.\\
Mukhavaṇṇo vippasīdati.\\
Asammūḷho kālaṁ karoti.\\
Uttari appaṭivijjhanto brahmalokūpago hoti.

\clearpage

\begin{english}
  Dormir bien.\\
  Despertar contento.\\
  No tener pesadillas.\\
  Ser querido por los seres humanos.\\
  Ser querido por los seres no humanos.\\
  Ser protegido por los seres celestiales.\\
  No verse afectado ni por fuego, ni por veneno, ni por armas.\\
  Poder concentrar la mente rápidamente.\\
  Tener un rostro sereno.\\
  Morir sin confusión alguna.\\
  En caso de no obtener un estado superior de iluminación, renacer en el mundo de Brahma.
\end{english}

Mettāya, bhikkhave, cetovimuttiyā āsevitāya bhāvitāya bahulīkatāya yānīkatāya vatthukatāya anuṭṭhitāya paricitāya susamāraddhāya ime ekādasānisaṁsā pāṭikaṅkhā'ti.

\begin{english}
  Monjes, estos once beneficios se pueden esperar como resultado de la liberación del corazón por la práctica de Metta, conociéndola, cultivándola, desarrollándola, teniéndola como guía, estimándola, siguiéndola, consolidándola e implementándola.
\end{english}

\chapter*[Irradiando los Estados Divinos]{Irradiando los Estados Divinos}

\delegateSetUseNext

\firstline{Mettā-sahagatena}

\begin{leader}
  [Ha꜓nda mayaṁ caturappamaññā-obhāsanaṁ karomase]
\end{leader}

[Mettā-sa꜕ha꜕ga꜕tena] cetasā ekaṁ disaṁ pha꜕ri꜕tv꜕ā viha꜕ra꜕ti\\
Ta꜕thā dutiyaṁ ta꜕thā tatiyaṁ ta꜕thā ca꜕tutthaṁ\\
Iti uddhamadho tiriyaṁ sabba꜕dhi꜕ sabbatta꜕tāya\\
Sabbāvantaṁ lokaṁ mettā-sa꜕ha꜕ga꜕tena cetasā\\
Vipulena mahagga꜕tena appa꜕māṇena a꜕verena a꜕byāpajjhena\\
\vin pha꜕ri꜕tv꜕ā viha꜕ra꜕ti

Karuṇā-sa꜕ha꜕ga꜕tena cetasā ekaṁ disaṁ pha꜕ri꜕tv꜕ā viha꜕ra꜕ti\\
Ta꜕thā dutiyaṁ ta꜕thā tatiyaṁ ta꜕thā ca꜕tutthaṁ\\
Iti uddhamadho tiriyaṁ sabba꜕dhi꜕ sabbatta꜕tāya\\
Sabbāvantaṁ lokaṁ ka꜕ru꜕ṇā-sa꜕ha꜕ga꜕tena cetasā\\
Vipulena mahagga꜕tena appa꜕māṇena a꜕verena a꜕byāpajjhena\\
\vin pha꜕ri꜕tv꜕ā viha꜕ra꜕ti

Muditā-sa꜕ha꜕ga꜕tena cetasā ekaṁ disaṁ pha꜕ri꜕tv꜕ā viha꜕ra꜕ti\\
Ta꜕thā dutiyaṁ ta꜕thā tatiyaṁ ta꜕thā ca꜕tutthaṁ\\
Iti uddhamadho tiriyaṁ sabba꜕dhi꜕ sabbatta꜕tāya\\
Sabbāvantaṁ lokaṁ mu꜕di꜕tā-sa꜕ha꜕ga꜕tena cetasā\\
Vipulena mahagga꜕tena appa꜕māṇena a꜕verena a꜕byāpajjhena\\
\vin pha꜕ri꜕tv꜕ā viha꜕ra꜕ti

\chapter[Irradiando los Estados  Divinos]{Irradiando los Estados Divinos}

\enlargethispage{\baselineskip}

\firstline{Yo permaneceré}

\begin{leader}
  \vspace*{-\baselineskip}
  \mbox{[Dejemos ahora irradiar las Cuatro Cualidades Inmensurables.]}
\end{leader}

[Yo permaneceré] permeando un cuadrante del mundo\\
\vin con un corazón saturado de metta; de igual forma\\
\vin el segundo, de igual forma el tercero, de igual forma el cuarto;\\
Tanto hacia arriba, como hacia abajo, alrededor y en todas direcciones;\\
\vin y para todos, tanto como para mí.\\
Yo permaneceré permeando el mundo completo\\
\vin con un corazón saturado de metta; abundante, exaltado,\\
\vin inmensurable, sin hostilidad y sin mal-fé.

[Yo permaneceré] permeando un cuadrante del mundo\\
\vin con un corazón saturado de compasión; de igual forma\\
\vin el segundo, de igual forma el tercero, de igual forma el cuarto;\\
Tanto hacia arriba, como hacia abajo, alrededor y en todas direcciones;\\
\vin y para todos, tanto como para mí.\\
Yo permaneceré permeando el mundo completo\\
\vin con un corazón saturado de compasión; abundante, exaltado,\\
\vin inmensurable, sin hostilidad y sin mal-fé.

[Yo permaneceré] permeando un cuadrante del mundo\\
\vin con un corazón saturado de apreciación empática; de igual forma\\
\vin el segundo, de igual forma el tercero, de igual forma el cuarto;\\
Tanto hacia arriba, como hacia abajo, alrededor y en todas direcciones;\\
\vin y para todos, tanto como para mí.\\
Yo permaneceré permeando el mundo completo\\
\vin con un corazón saturado de apreciación empática; abundante, \vin exaltado,
inmensurable, sin hostilidad y sin mal-fé.

\clearpage

Upekkhā-saha꜕ga꜕te꜕na cetasā ekaṁ disaṁ pha꜕ri꜕tv꜕ā viha꜕ra꜕ti\\
Ta꜕thā dutiyaṁ ta꜕thā tatiyaṁ ta꜕thā ca꜕tutthaṁ\\
Iti uddhamadho tiriyaṁ sabba꜕dhi꜕ sabbatta꜕tāya\\
Sabbāvantaṁ lokaṁ u꜕pe꜕kkhā-sa꜕ha꜕ga꜕tena cetasā\\
Vipulena mahagga꜕tena appa꜕māṇena a꜕verena a꜕byāpajjhena\\
\vin pha꜕ri꜕tv꜕ā viha꜕ra꜕tī'ti

\clearpage

[Yo permaneceré] permeando un cuadrante del mundo\\
\vin con un corazón saturado de equanimidad; de igual forma\\
\vin el segundo, de igual forma el tercero, de igual forma el cuarto;\\
Tanto hacia arriba, como hacia abajo, alrededor y en todas direcciones;\\
\vin y para todos, tanto como para mí.\\
Yo permaneceré permeando el mundo completo\\
\vin con un corazón saturado de equanimidad; abundante, exaltado,\\
\vin inmensurable, sin hostilidad y sin mal-fé.

\chapter{Las Mejores Bendiciones}

\firstline{Así yo oí, que el Señor}

\begin{leader}
  [Cantemos ahora los versos sobre las Mejores Bendiciones.]
\end{leader}

[Así yo oí, que el Señor]\\
Se encontraba en Savatthi,\\
Residiendo en el Bosque de Jeta\\
En el Parque de Anāthapiṇḍika.

Entonces en la oscuridad de la noche, una deva radiante\\
Iluminó todo el Parque de Jeta.\\
Se inclinó rindiendo reverencia al Excelso\\
Y después de ponerse en pie, dijo:

`Los Devas se preocupan por la felicidad\\
Y buscan Paz continuamente.\\
Lo mismo se puede decir de los humanos.\\
Así que, cuales son las mejores Bendiciones?'

`Evitar a los idiotas,\\
Asociarse con los Sabios,\\
Y honrar lo que es digno de ser honrado.\\
Estas son las mejores bendiciones.

`Viver en sitios adecuados,\\
Con los frutos de buenas acciones pasadas,\\
Guiado por el camino correcto.\\
Estas son las mejores bendiciones.

\clearpage

`Competente en estudios y oficios,\\
Con disciplina sublimemente entrenada,\\
Y un hablar agradable al oido.\\
Estas son las mejores bendiciones.

`Cuidar de los padres,\\
Atender a la familia,\\
Y tener una vida inofensiva para los demás.\\
Estas son las mejores bendiciones.

`Generosidad y una vida honesta,\\
Ofrecer ayuda a familiares y amigos,\\
Actuar de forma que no cause remordimientos.\\
Estas son las mejores bendiciones.

`Ser resoluto y dominarse, abandonar los malos caminos,\\
Evitar intoxicantes que apaguen la mente,\\
Y ser diligente en todas las ocasiones.\\
Estas son las mejores bendiciones.

`Respeto y humildad,\\
Estar satisfecho y agradecido,\\
Oir el Dhamma propiamente enseñado.\\
Estas son las mejores bendiciones.

`Paciencia y voluntad para aceptar las propias faltas,\\
Visitar respetables buscadores de la verdad,\\
y compartir el Dhamma adecuadamente.\\
Estas son las mejores bendiciones.

\clearpage

`Dedicarse ardientemente a la Vida Santa,\\
Ver las Nobles Verdades directamente por uno mismo\\
Y realizar el Nibbāna.\\
Estas son las mejores bendiciones.

`Aunque en contacto con el mundo,\\
La mente se mantiene inalterable,\\
Perfectamente segura, más allá de toda aflicción.\\
Estas son las mejores bendiciones.

`Aquellos que siguen este camino,\\
Conecen la Victoria donde quiera que van,\\
Y cualquier lugar para ellos es seguro.\\
Estas son las mejores bendiciones.'

\chapter*[Bienestar Universal]{Reflexión sobre el Bienestar Universal}

\delegateSetUseNext

\firstline{Ahaṁ sukhito homi}

\begin{leader}
  [Ha꜓nda mayam mettāpharaṇaṁ ka꜕romase]
\end{leader}

[Aha꜓ṁ sukhito ho꜓mi]\\
Niddukkho ho꜓mi\\
A꜕vero ho꜓mi\\
A꜕byāpajjho ho꜓mi\\
A꜕nīgho ho꜓mi\\
Sukhī꜓ attānaṁ pa꜕riha꜓rāmi

Sa꜕bbe sa꜕ttā sukhitā ho꜓ntu\\
Sa꜕bbe sa꜕ttā averā ho꜓ntu\\
Sa꜕bbe sa꜕ttā abyāpajjhā ho꜓ntu\\
Sa꜕bbe sa꜕ttā anīghā ho꜓ntu\\
Sa꜕bbe sa꜕ttā sukhī꜓ a꜕ttānaṁ pa꜕riha꜓rantu

Sa꜕bbe sa꜕ttā sabbadukkhā pamucca꜓ntu

Sa꜕bbe sa꜕ttā laddha-sa꜓mpa꜕tti꜓to mā vigaccha꜓ntu

Sa꜕bbe sa꜕ttā kammassa꜕kā kamma꜓dāyādā kamma꜓yonī\\
\vin kamma꜓bandhū kammapa꜕ṭisa꜓ra꜕ṇā\\
Yaṁ kammaṁ ka꜕rissa꜓nti\\
Kalyāṇaṁ vā pāpa꜕kaṁ vā\\
Tassa꜕ dāyādā bha꜕vissa꜓nti

\chapter[Bienestar Universal]{Reflexión sobre el Bienestar Universal}

\firstline{Que yo mantenga bienestar}

\begin{leader}
  [Cantemos ahora la Reflexión sobre el Bienestar Universal.]
\end{leader}

[Que yo pueda vivir en bienestar,]\\
Libre de aflicción,\\
Libre de hostilidad,\\
Libre de mal-fé,\\
Libre de ansiedad,\\
Y que pueda mantener este bienestar.

Que todos los seres vivan en bienestar,\\
Libres de hostilidad,\\
Libres de mal-fé,\\
Libres de ansiedad,\\
Y que ellos puedan mantener su bienestar.

Que todos los seres puedan liberarse de todo sufrimiento.

Y que no se pierdan la buena fortuna que alcanzaron.

Cuando actuan con intención,\\
Todos los seres son los dueños de sus actos y heredarán sus resultados.\\
Su futuro nace de tal acción, compañero de tal acción,\\
Y sus resultados serán su hogar.

Todas las acciones con intención,\\
Sean buenas o malas ---\\
De estos actos, ellos serán los herederos.

\chapter[Cuatro Requisitos]{Reflexión sobre los Cuatro Requisitos}

\firstline{Paṭisaṅkhā yoniso}

\begin{leader}
  [Ha꜓nda mayaṁ taṅkhaṇika-paccave꜕kkhaṇa-pāṭhaṁ bhaṇāmase]
\end{leader}

[Paṭisaṅkhā] yoniso cīva꜕raṁ pa꜕ṭise꜓vāmi, \pause\\
yāvadeva sī꜓tassa꜕ pa꜕ṭighātāya, \pause\ uṇhassa pa꜕ṭighātāya, \pause\\
ḍaṁsa-maka꜕sa꜕-vātāta꜕pa꜕-siriṁsapa-samphassānaṁ pa꜕ṭighātāya, \pause\\
yāvadeva hiri꜓kopina-pa꜕ṭicchāda꜕natthaṁ

\begin{english}
  Reflexionando sabiamente \pause\ yo uso el manto: \pause\ solamente por modestia, \pause\
  para evitar el calor, \pause\ el frio, \pause\ las moscas, \pause\ mosquitos,
  \pause\ bichos que se arrastran, \pause\ el viento y las cosas que arden.
\end{english}

[Paṭisaṅkhā] yoniso piṇḍa꜕pātaṁ pa꜕ṭise꜓vāmi, \pause\\
neva da꜕vāya, na ma꜕dāya, na maṇḍa꜕nāya, na꜕ vi꜓bhūsa꜕nāya, \pause\\
yāvadeva i꜓massa꜕ kāyassa꜕ ṭhi꜕tiyā, \pause\ yāpa꜕nāya, vihiṁsū꜕para꜓ti꜕yā, \pause\\
brahmaca꜕ri꜓yānugga꜕hāya, \pause\ iti purāṇañca꜕ veda꜓naṁ pa꜕ṭiha꜓ṅkhāmi,
navañca꜕ veda꜓naṁ na uppādessāmi, \pause\ yātrā ca꜕ me bhavissati a꜕navajjatā
ca꜕ phāsuvihāro cā'ti

\begin{english}
  Reflexionando sabiamente \pause\ yo uso la comida de mendicancia: \pause\ no por
  diversión, \pause\ no por placer, \pause\ no para engordar, \pause\ no para
  embellecerme, \pause\ pero solamente para sostener y nutrir este cuerpo, \pause\
  para mantenerlo saludable, \pause\ para ayudar a la Vida Santa. \pause\ Pensando
  de esta forma: \pause\ `Saciaré el hambre sin comer demasiado, \pause\ de
  forma a continuar viviendo sereno y sin remordimientos.'
\end{english}

\clearpage

[Paṭisaṅkhā] yoniso senāsa꜕naṁ pa꜕ṭise꜓vāmi, \pause\\
yāvadeva sī꜓tassa꜕ pa꜕ṭighātāya, \pause\ uṇhassa pa꜕ṭighātāya, \pause\\
ḍaṁsa-maka꜕sa꜕-vātāta꜕pa꜕-siriṁsapa-samphassānaṁ pa꜕ṭighātāya, \pause\\
yāvadeva utupa꜕rissaya vi꜕nodanaṁ \pause\ pa꜕ṭisa꜓llānārāmatthaṁ

\begin{english}
  Reflexionando sabiamente \pause\ utilizo la vivienda: \pause\ solamente para evitar el
  frio, \pause\ el calor, \pause\ las moscas, \pause\ mosquitos, \pause\ bichos
  que se arrastran, \pause\ el viento y las cosas que arden. \pause\ Solamente para
  protegerme de los peligros de la naturaleza \pause\ y vivir en recogimiento.
\end{english}

[Paṭisaṅkhā] yoniso gi꜕lāna-pacca꜕ya꜕-bhesajja-pa꜕rikkhāraṁ\\
pa꜕ṭise꜓vāmi, \pause\ yāvadeva uppa꜓nnānaṁ veyyābādhi꜕kānaṁ veda꜕nānaṁ
pa꜕ṭighātāya, \pause\ a꜕byāpajjha-pa꜕ramatāyā'ti

\begin{english}
  Reflexionando sabiamente \pause\ utilizo el apoyo necesario para medicamentos y
  enfermidades: \pause\ solamente para aliviar los dolores que surjan,
  \pause\ de forma a permanecer lo más posible libre de enfermedades.
\end{english}

\chapter[Treinta y dos Partes]{Reflexión sobre las Treinta y dos Partes}

\firstline{Ayaṁ kho me kāyo}

\begin{leader}
  [Ha꜓nda mayaṁ dvattiṁsākāra-pāṭhaṁ bhaṇāmase]
\end{leader}

[Ayaṁ kho] me kāyo uddhaṁ pāda꜕ta꜕lā adho kesamatthakā\\
ta꜕ca꜕pa꜕ri꜕yanto pūro nānappa꜕kārassa꜕ a꜕su꜕ci꜕no

\begin{english}
  Esto, que es mi cuerpo, de la planta de los pies para arriba, y de la coronilla para abajo, es un saco de piel cerrado, lleno de cosas asquerosas.
\end{english}

Atthi imasmi꜔꜒ṁ kāye

\begin{english}
  En este cuerpo existen:
\end{english}

{\centering
\setArrayStrech{1}

\begin{tabular}{ r l }
kesā꜔꜒           & \tr{cabello} \\
lomā            & \tr{bello} \\
nakhā꜔꜒          & \tr{uñas} \\
dantā           & \tr{dientes} \\
taco            & \tr{piel} \\
maṁsa꜔꜒ṁ        & \tr{carne} \\
nahā꜔꜒rū         & \tr{tendones} \\
aṭṭhī꜔꜒           & \tr{huesos} \\
aṭṭhimiñjaṁ     & \tr{medula osea} \\
vakkaṁ          & \tr{riñones} \\
hadayaṁ         & \tr{corazón} \\
yakanaṁ         & \tr{hígado} \\
kilomakaṁ       & \tr{membranas} \\
pihakaṁ         & \tr{bazo} \\
papphā꜔꜒sa꜔꜒ṁ    & \tr{pulmones} \\
\end{tabular}

\clearpage

\begin{tabular}{ r l }
antaṁ           & \tr{intestino grueso} \\
antaguṇaṁ       & \tr{intestino delgado} \\
udariyaṁ        & \tr{comida no digerida} \\
karīsa꜔꜒ṁ        & \tr{excremento} \\
pittaṁ          & \tr{bílis} \\
se꜔꜒mha꜔꜒ṁ       & \tr{moco} \\
pubbo           & \tr{pus} \\
lohitaṁ         & \tr{sangre} \\
se꜔꜒do           & \tr{sudor} \\
medo            & \tr{grasa} \\
assu            & \tr{lágrimas} \\
vasā꜔꜒           & \tr{sebo} \\
khe꜔꜒ḷo          & \tr{saliva} \\
si꜔꜒ṅghāṇikā    & \tr{mocos} \\
lasikā          & \tr{lubricante de las articulaciones} \\
muttaṁ          & \tr{orina} \\
matthaluṅgan’ti & \tr{cerebro} \\
\end{tabular}

\restoreArrayStretch
}

Evam-ayaṁ me kāyo uddhaṁ pāda꜕ta꜕lā adho kesamatthakā\\
ta꜕ca꜕pa꜕ri꜕yanto pūro nānappa꜕kārassa꜕ a꜕su꜕ci꜕no

\begin{english}
  Así, esto que es mi cuerpo, de la planta de los pies para arriba, y de la coronilla para abajo, es un saco de piel cerrado, lleno de cosas asquerosas.
\end{english}

\clearpage

%Se꜔꜒yyathā꜔꜒pi, bhikkhave, ubhatomukhā꜔꜒ putoḷi pūrā nānāvihitassa dhaññassa,
%se꜔꜒yyathī꜔꜒daṁ, sā꜔꜒līnaṁ vīhī꜔꜒naṁ muggānaṁ māsā꜔꜒naṁ tilānaṁ taṇḍulānaṁ. Tamenaṁ
%cakkhumā puriso꜔꜒ muñcitvā paccavekkhe꜔꜒yya:
%
%\begin{english}
%  Bhikkhus, igual que si hubiese un saco, con una abertura en ambos extremos, lleno de varios tipos de granos (tales como arroz de montaña, arroz rojo, judías, guisantes, maiz y arroz blanco) y un hombre con buenos ojos lo abriese y lo describiese de la siguiente forma:
%\end{english}
%
%‘Ime sā꜔꜒lī, ime vīhī꜔꜒ ime muggā ime māsā꜔꜒ ime tilā ime taṇḍulā’ti. Evameva kho꜔꜒,
%bhikkhave, bhikkhu imameva kāyaṁ uddhaṁ pādatalā adho kesamatthakā
%tacapariyantaṁ pūraṁ nānappakārassa asucino paccavekkhati:
%
%\begin{english}
%  ‘Esto es arroz de montaña, esto es arroz rojo, esto son judías, esto son guisantes,
%  esto es maiz, esto es arroz blanco’;
%
%  así de la misma manera, bhikkhus, un bhikkhu describe:
%
 % Esto, que es mi cuerpo, de la planta de los pies para arriba, y de la coronilla para abajo, es un saco de piel cerrado, lleno de cosas asquerosas.
%\end{english}
%
%‘Atthi imasmi꜔꜒ṁ kāye kesā꜔꜒ lomā nakhā꜔꜒ dantā taco, maṁsa꜔꜒ṁ nahā꜔꜒rū aṭṭhī꜔꜒ aṭṭhimiñjaṁ
%vakkaṁ, hadayaṁ yakanaṁ kilomakaṁ pihakaṁ papphā꜔꜒sa꜔꜒ṁ, antaṁ antaguṇaṁ udariyaṁ
%karīsa꜔꜒ṁ, pittaṁ se꜔꜒mha꜔꜒ṁ pubbo lohitaṁ se꜔꜒do medo, assu vasā꜔꜒ khe꜔꜒ḷo si꜔꜒ṅghāṇikā
%lasikā muttaṁ matthaluṅgan’ti.

    %
%\begin{english}
%  En este cuerpo existen:
%  cabello, bello, uñas, dientes, piel,
%  carne, tendones, huesos, medula osea, riñones,
%  corazón, hígado, membranas, bazo, pulmones,
%  intestino grueso, intestino delgado, comida no digerida, excremento,
%  bílis, moco, pus, sangre, sudor, grasa,
%  lágrimas, sebo, saliva, mucosidad, lubricante de las articulaciones, orina, cerebro.
%\end{english}

%Iti ajjhattaṁ vā kāye kāyānupassī꜔꜒ viharati, bahiddhā vā kāye kāyānupassī꜔꜒
%viharati, ajjhatta-bahiddhā vā kāye kāyānupassī꜔꜒ viharati. Samudaya-dhammānupassī꜔꜒
%vā kāyasmi꜔꜒ṁ viharati, vaya-dhammā-\\
%nupassī꜔꜒ vā kāyasmi꜔꜒ṁ viharati, samudaya-vaya-dhammānupassī꜔꜒ vā kāyasmi꜔꜒ṁ viharati.
%‘Atthi kāyo’ti vā panassa sati paccupaṭṭhitā ho꜔꜒ti yāvadeva ñāṇamattāya
%paṭissatimattāya.
%
%Anissito ca viharati, na ca kiñci loke upādiyati. Evampi kho꜔꜒,
%bhikkhave, bhikkhu kāye kāyānupassī꜔꜒ viharati.
%
%\begin{english}
 % De esta forma, considerando el cuerpo, él sigue contemplando el cuerpo internamente;
  %considerando el cuerpo, él sigue contemplando el cuerpo externamente;
  %considerando el cuerpo, él sigue contemplando el cuerpo tanto internamente como externamente.
  %O entonces, él sigue contemplando el cuerpo en su naturaleza de surgir,
  %o sigue contemplando el cuerpo en su naturaleza de cesar, o sigue contemplando el cuerpo en su naturaleza de surgir y cesar. Así, el concepto de que ‘existe
  %un cuerpo’ se establece en él al punto de haber  comprensión directa y consciencia inquebrantable.

  %\bigskip

  %Y él vive de forma independente, sin apegarse a nada en el mundo. Bhikkhus,
  %es de esta forma como un bhikkhu, considerando el cuerpo, sigue contemplando el cuerpo en el cuerpo.
%\end{english}

\chapter[Cinco Temas]{Cinco Temas para Recordar Frequentemente}

\firstline{Jarā-dhammomhi jaraṁ anatīto}

\begin{leader}
  [Ha꜓nda mayaṁ abhiṇha-paccave꜕kkhaṇa-pāṭhaṁ bhaṇāmase]
\end{leader}

\sidepar{Hombres}%
[Jarā-dhammomhi꜕] jaraṁ a꜕na꜕tīto

\sidepar{Mujeres}%
[Jarā-dhammāmhi꜕] jaraṁ a꜕na꜕tītā

\begin{english}
  Mi naturaleza es envejecer, aún no estoy más allá del envejecimiento.
\end{english}

\sidepar{h.}%
Byādhi꜓-dhammomhi꜕ byādhiṁ a꜕na꜕tīto

\sidepar{m.}%
Byādhi꜓-dhammāmhi꜕ byādhiṁ a꜕na꜕tītā

\begin{english}
  Mi naturaleza es enfermar, aún no estoy más allá de la enfermedad.
\end{english}

\sidepar{h.}%
Ma꜕raṇa-dhammomhi꜕ ma꜕raṇaṁ a꜕na꜕tīto

\sidepar{m.}%
Ma꜕raṇa-dhammāmhi꜕ ma꜕raṇaṁ a꜕na꜕tītā

\begin{english}
  Mi naturaleza es morir, aún no estoy más allá de la muerte.
\end{english}

Sa꜕bbehi me pi꜕yehi ma꜕nāpehi꜕ nānābhāvo vi꜕nābhāvo

\begin{english}
  Todo aquello que es mío, que amo y aprecio,\\
  se volverá diferente, se separará de mí.
\end{english}

\sidepar{h.}%
Kammassa꜕komhi kamma꜓dāyādo kamma꜕yoni kamma꜕bandhu kammapa꜕ṭisa꜓ra꜕ṇo\\
Yaṁ kammaṁ ka꜕rissāmi, kalyāṇaṁ vā pāpa꜕kaṁ vā, tassa꜕ dāyādo bha꜕vissāmi

\clearpage

\sidepar{m.}%
Kammassa꜕kāmhi kamma꜓dāyādā kamma꜕yoni kamma꜕bandhu kammapa꜕ṭisa꜓ra꜕ṇā\\
Yaṁ kammaṁ ka꜕rissāmi, kalyāṇaṁ vā pāpa꜕kaṁ vā, tassa꜕ dāyādā bha꜕vissāmi

\begin{english}
  Soy el dueño de mi Kamma, heredero de mi Kamma, \\
  nacido de mi Kamma, ligado a mi Kamma,\\
  permanezco soportado por mi Kamma; cualquier Kamma que creo, \\
  Para bien o para mal, de eso seré el heredero.
\end{english}

Evaṁ amhehi꜕ a꜕bhiṇhaṁ pacca꜕vekkhi꜓tabbaṁ

\begin{english}
  Así deberíamos reflexionar frequentemente.
\end{english}

\chapter[Diez temas]{Diez temas para recordar frequentemente por aquellos que son renunciantes}

\firstline{Dasa ime bhikkhave}

\begin{leader}
  [Ha꜓nda mayaṁ pabbajita\hyp{}abhiṇha\hyp{}paccave꜕kkhaṇa\hyp{}pāṭhaṁ bhaṇāmase]
\end{leader}

[Dasa i꜕me bhikkhave] dhammā pabba꜕jitena a꜕bhiṇhaṁ pacca꜕vekkhi꜓tabbā, \pause\ ka꜕ta꜕me dasa

\begin{english}
  Monjes, existen diez dhammas \pause\ sobre los cuales se debe reflexionar frequentemente. \pause\ Cuales son estos diez dhammas?
\end{english}

Vevaṇṇi꜕yamhi ajjhūpa꜕ga꜕to'ti pabba꜕jitena a꜕bhiṇhaṁ pacca꜕vekkhi꜓tabbaṁ

\begin{english}
  `Ya no vivo siguiendo los valores y objectivos del mundo.' \pause\\
  Aquel que es renunciante \pause\ debe reflexionar sobre esto frequentemente.
\end{english}

Parapaṭi꜕baddhā me jīvi꜓kā'ti pabba꜕jitena a꜕bhiṇhaṁ pacca꜕vekkhi꜓tabbaṁ

\begin{english}
  `Mi propria vida es sostenida \pause\ por la generosidad de los demás.' \pause\\
  Aquel que es renunciante \pause\ debe reflexionar sobre esto frequentemente.
\end{english}

Añño me ākappo ka꜕ra꜕ṇīyo'ti pabba꜕jitena a꜕bhiṇhaṁ pacca꜕vekkhi꜓tabbaṁ

\begin{english}
  `Debo esforzarme por abandonar mis hábitos antiguos.' \pause\\
  Aquel que es renunciante \pause\ debe reflexionar sobre esto frequentemente.
\end{english}

\clearpage

Kacci nu꜕ kho me attā sīla꜕to na u꜕pavadatī'ti pabba꜕jitena a꜕bhiṇhaṁ pacca꜕vekkhi꜓tabbaṁ

\begin{english}
  `Surgen remordimientos en mi mente \pause\ en relación a mi conducta?' \pause\\
  Aquel que es renunciante \pause\ debe reflexionar sobre esto frequentemente.
\end{english}

Kacci nu꜕ kho maṁ a꜕nuvicca viññū sabrahma꜓cārī sīla꜕to na u꜕pavadantī'ti pabba꜕jitena a꜕bhiṇhaṁ pacca꜕vekkhi꜓tabbaṁ

\begin{english}
  `Será que mis compañeros espirituales \pause\\
  hallan faltas en mi conducta?' \pause\\
  Aquel que es renunciante \pause\ debe reflexionar sobre esto frequentemente.
\end{english}

Sa꜕bbehi me pi꜕yehi ma꜕nāpehi꜕ nānābhāvo vi꜕nābhāvo'ti pabba꜕jitena abhiṇhaṁ pacca꜕vekkhi꜓tabbaṁ

\begin{english}
  `Todo aquello que es mío, \pause\ que amo y aprecio, \pause\ se volverá diferente, \pause\ se separará de mí.' \pause\\
  Aquel que es renunciante \pause\ debe reflexionar sobre esto frequentemente.
\end{english}

Kammassa꜕komhi kamma꜓dāyādo kamma꜕yoni kamma꜕bandhu kammapa꜕ṭisa꜓raṇo, yaṁ kammaṁ ka꜕rissāmi, kalyāṇaṁ vā pāpa꜕kaṁ vā, tassa꜕ dāyādo bha꜕vissāmī'ti pabba꜕jitena a꜕bhiṇhaṁ pacca꜕vekkhi꜓tabbaṁ

\begin{english}
  `Soy el dueño de mi Kamma, \pause\ heredero de mi Kamma, \pause\\
  nacido de mi Kamma, \pause\ ligado a mi Kamma, \pause\\
  permanezco soportado por mi Kamma; \pause\ cualquier Kamma que creo, \pause\\
  Para bien o para mal, \pause\ de eso seré el heredero.' \pause\\
  Aquel que es renunciante \pause\ debe reflexionar sobre esto frequentemente.
\end{english}

\clearpage

`Kathambhūtassa꜕ me rattindi꜕vā vīti꜕pa꜓tantī'ti pabba꜕jitena a꜕bhiṇhaṁ pacca꜕vekkhi꜓tabbaṁ

\begin{english}
  `Los dias y las noches pasan continuamente; \pause\\
  Como estoy usando mi tiempo?' \pause\\
  Aquel que es renunciante \pause\ debe reflexionar sobre esto frequentemente.
\end{english}

Kacci nu꜕ kho'haṁ suññā꜓gāre abhira꜕māmī'ti pabba꜕jitena a꜕bhiṇhaṁ pacca꜕vekkhi꜓tabbaṁ

\begin{english}
  `Aprecio la solidad o no?' \pause\\
  Aquel que es renunciante \pause\ debe reflexionar sobre esto frequentemente.
\end{english}

Atthi nu꜕ kho me uttari-ma꜕nussa-dhammā alamariya꜕-ñāṇa-dassana-viseso adhiga꜕to, so'haṁ pacchi꜓me kāle sa꜕brahmacārīhi꜕ puṭṭho na maṅku bha꜕vissāmī'ti pabba꜕jitena a꜕bhiṇhaṁ pacca꜕vekkhi꜓tabbaṁ

\begin{english}
  `Ha dado mi práctica frutos de comprensión y libertad, \pause\\ de forma que
  al final de mi vida \pause\ no me sienta envergonzado \pause\\
  cuando questionado \pause\ por mis compañeros espirituales?' \pause\\
  Aquel que es renunciante \pause\ debe reflexionar sobre esto frequentemente.
\end{english}

Ime kho bhikkha꜓ve da꜕sa꜕ dhammā pabba꜕jitena a꜕bhiṇhaṁ pacca꜕vekkhitabbā'ti

\begin{english}
  Monjes estos son los diez Dhammas \pause\ sobre los cuales se debe reflexionar frequentemente.
\end{english}

\chapter{Verdaderos y Falsos Refugios}

\firstline{Bahuṁ ve saraṇaṁ yanti}

\begin{leader}
  [Ha꜓nda mayaṁ khemākhema-sa꜕raṇa-gamana-\\
  -pa꜕ridīpikā-gāthā꜓yo bha꜕ṇāmase]
\end{leader}

\begin{twochants}
  Bahuṁ ve sa꜕ra꜓ṇaṁ yanti꜕ & pa꜕bba꜕tāni va꜕nāni꜓ ca \\
  Ārāma-rukkha꜕-cetyāni & manussā꜓ bha꜕ya꜕-tajji꜕tā \\
\end{twochants}

\begin{english}
  Hacia muchos refugios ellos van ---\\
  hacia las faldas de las montañas y hacia los claros de los bosques,\\
  hacia parques naturales y sitios sagrados ---\\
  personas dominadas por el miedo.
\end{english}

\begin{twochants}
  N'etaṁ kho sa꜕ra꜓ṇaṁ khemaṁ & n'etaṁ sa꜕raṇam-u꜓tt꜕amaṁ \\
  N'etaṁ sa꜕raṇam-āgamma & sa꜕bba-dukkhā꜓ pa꜕mucca꜕ti \\
\end{twochants}

\begin{english}
  Tales refugios no son seguros,\\
  tales refugios no son supremos,\\
  tales refugios no llevan\\
  a la completa liberación del sufrimiento.
\end{english}

\begin{twochants}
  Yo ca꜕ Buddhañca꜕ Dhammañca꜕ & sa꜓ṅghañca꜕ sa꜓ra꜕ṇaṁ ga꜕to \\
  Ca꜕ttāri a꜕riya-saccāni & sa꜕mmappaññāya꜓ pa꜕ss꜕ati \\
\end{twochants}

\begin{english}
  El que se refugia\\
  en la Joya Triple\\
  ve claramente\\
  las Cuatro Nobles Verdades:
\end{english}

\begin{twochants}
  Dukkhaṁ dukkha-sa꜕muppādaṁ & dukkhassa ca꜕ a꜕ti꜕kka꜕maṁ \\
  A꜕riyañ-c'a꜕ṭṭh'a꜓ṅgi꜕kaṁ maggaṁ & dukkhūpasa꜕ma꜕-gāmi꜓naṁ \\
\end{twochants}

\begin{english}
  El sufrimento, su origen \\
  y la liberación del mismo,\\
  el Noble Óctuple Camino\\
  que lleva al fin del sufrimiento.
\end{english}

\begin{twochants}
  Etaṁ kho sa꜕ra꜓ṇaṁ khemaṁ & etaṁ sa꜕raṇam-u꜓tta꜕maṁ \\
  Etaṁ sa꜕raṇam-āgamma & sa꜕bba-dukkhā꜓ pa꜕mucca꜕ti \\
\end{twochants}

\begin{english}
  Tal refugio es seguro,\\
  tal refugio es supremo,\\
  tal refugio realmente lleva\\
  a la completa liberación del sufrimiento.
\end{english}

\chapter{Versos sobre la Riqueza de Uno que es Noble}

\firstline{Yassa saddhā tathāgate}

\begin{leader}
  [Ha꜓nda mayaṁ a꜕riya-dhana-gāthā꜓yo bha꜕ṇāmase]
\end{leader}

\begin{twochants}
  Yassa꜕ sa꜕ddhā tathā꜓ga꜕te & a꜕ca꜕lā su꜕pa꜕tiṭṭhi꜓tā \\
  Sī꜓lañca꜕ yassa꜕ kalyāṇaṁ & a꜕riya-kantaṁ pasa꜓ṁsi꜕taṁ \\
\end{twochants}

\begin{english}
  Aquél cuya confianza en el Tathāgata\\
  es inquebrantable y bien establecida,\\
  cuya virtud es admirable,\\
  tiene el regocijo y elogio de los Nobles.
\end{english}

\begin{twochants}
  Sa꜓ṅghe pa꜕sā꜕do yass'atthi & uju-bhūtañca da꜓ss꜕anaṁ \\
  A꜕daliddo't꜕i taṁ āhu꜕ & a꜕moghaṁ ta꜕ssa꜕ jīvi꜓taṁ \\
\end{twochants}

\begin{english}
  Aquél que tiene confianza en la Saṅgha,\\
  que ve directamente la verdadera realidad,\\
  de él se dice que no es `pobre'\\
  y que en vano no es su vida.
\end{english}

\begin{twochants}
  Tasmā sa꜕ddhañca꜕ sī꜓lañca꜕ & pasādaṁ dhamma-da꜓ssa꜕naṁ \\
  A꜕nuyuñjetha medhāvī & sa꜕raṁ buddhāna sā꜓sa꜕naṁ \\
\end{twochants}

\begin{english}
  Por ello, a la virtud y a la fé,\\
  a la confianza y al saber de la verdad ---\\
  a esto los sabios se deben dedicar,\\
  manteniendo las enseñanzas de Buddha en la mente.
\end{english}

\chapter{Versos sobre las Tres Características}

\firstline{Sabbe saṅkhārā aniccā'ti}

\begin{leader}
  [Ha꜓nda mayaṁ ti-lakkhaṇ'ādi-gāthā꜓yo bha꜕ṇāmase]
\end{leader}

\begin{twochants}
  Sa꜕bbe sa꜓ṅkhā꜓rā a꜕ni꜓ccā't꜕i & yadā paññāya꜓ pa꜕ssa꜕ti \\
  Atha nibbinda꜕ti dukkhe & esa꜕ maggo vi꜓su꜕ddh꜓iyā \\
\end{twochants}

\begin{english}
  `Todas las formaciones condicionadas son impermanentes' ---\\
  cuando se ve esto con sabiduría,\\
  Uno se desencanta de sufrir.\\
  Este es el camino de la purificación.
\end{english}

\begin{twochants}
  Sa꜕bbe sa꜓ṅkhā꜓rā du꜕kkhā't꜕i & yadā paññāya꜓ pa꜕ssa꜕ti \\
  Atha nibbinda꜕ti dukkhe & esa꜕ maggo vi꜓su꜕ddh꜓iyā \\
\end{twochants}

\begin{english}
  `Todas las formaciones condicionadas son dukkha' ---\\
  cuando se ve esto con sabiduría,\\
  Uno se desencanta de sufrir.\\
  Este es el camino de la purificación.
\end{english}

\begin{twochants}
  Sa꜕bbe dhammā ana꜓ttā'ti꜕ & yadā paññāya꜓ pa꜕ssa꜕ti \\
  Atha nibbinda꜕ti dukkhe & esa꜕ maggo vi꜓su꜕ddh꜓iyā \\
\end{twochants}

\begin{english}
  `Todas las formaciones no son yo' ---\\
  cuando se ve esto con sabiduría,\\
  Uno se desencanta de sufrir.\\
  Este es el camino de la purificación.
\end{english}

\clearpage

\begin{twochants}
  A꜕ppa꜕kā te manusse꜓su꜕ & ye janā pāra-gāmi꜓no \\
  A꜕thāyaṁ i꜕ta꜕rā pajā & tīram-evānudhā꜓va꜕ti \\
\end{twochants}

\begin{english}
  Pocos son aquellos\\
  que cruzan a la otra orilla.\\
  Sin embargo, muchos son aquellos\\
  que en esta orilla vagan, sin rumbo.
\end{english}

\begin{twochants}
  Ye ca꜕ kho sammad-akkhāte & dhamme dhammānuva꜓tt꜕ino \\
  Te ja꜕nā pā꜕ram-essanti & ma꜕ccu-dheyyaṁ sudu꜓tta꜕raṁ \\
\end{twochants}

\begin{english}
  Donde quiera que el Dhamma sea bien enseñado,\\
  los que practican de acuerdo con él\\
  podrán cruzar más allá\\
  del ámbito de vida y muerte, tan dificil de escapar.
\end{english}

\begin{twochants}
  Kaṇhaṁ dhammaṁ vi꜕ppahā꜓ya & su꜕kkaṁ bhāvetha꜕ paṇḍi꜓to \\
  Okā a꜕noka꜕m-āgamma & viveke ya꜕tth꜕a dūramaṁ \\
  Ta꜕trābh꜕irat꜕im-iccheyya & hi꜕tvā kāme a꜕kiñc꜓ano \\
\end{twochants}

\begin{english}
  Abandonando la oscuridad,\\
  los sábios cultivan la luz;\\
  dejando atrás áreas inundadas,\\
  ellos alcanzan tierra firme.\\
  A pesar de que sea dificil encontrar deleite\\
  en la vida solitaria,\\
  tal deleite debe ser anhelado,\\
  renunciando a los placeres sensuales\\
  y habiendo abandonado todo.
\end{english}

\chapter{Versos sobre la Carga}

\firstline{Bhārā have pañcakkhandhā}

\begin{leader}
  [Ha꜓nda mayaṁ bhāra-su꜕tta-gāthā꜓yo bha꜕ṇāmase]
\end{leader}

\begin{twochants}
Bhārā ha꜕ve pañcakkha꜓ndhā & bhāra-hāro ca pu꜓gga꜕lo \\
Bhā꜕r'ādānaṁ du꜕kkhaṁ loke꜓ & bhāra-nikkhe꜓pa꜕naṁ su꜕khaṁ \\
\end{twochants}

\begin{english}
  De hecho, las cinco Khandhas son una carga,\\
  y quien lleva esa carga es el hombre.\\
  En este mundo, llevar cargas es sufrimiento;\\
  abandonarlas, es felicidad.
\end{english}

\begin{twochants}
Nikkhipi꜕tvā ga꜕ruṁ bhā꜓raṁ & aññaṁ bhāraṁ anā꜓di꜕ya \\
Sa꜕mūlaṁ taṇhaṁ a꜕bbuyha & nicchāto pa꜕ri꜕nibbu꜕to \\
\end{twochants}

\begin{english}
  Habiendo abandonado la pesada carga,\\
  no agarrando una carga nueva,\\
  habiendo arrancado el anhelo de raíz,\\
  lo deseos se apaciguan y el ser queda libre.
\end{english}

\chapter{Versos sobre una Auspiciosa Noche}

\firstline{Atītaṁ nānvāgameyya}

\begin{leader}
  [Ha꜓nda mayaṁ bhadd'eka-ratta꜕-gāthā꜓yo bha꜕ṇāmase]
\end{leader}

\begin{twochants}
  A꜕tītaṁ nānvāga꜕meyya & nappa꜕ṭikaṅkhe꜓ a꜕nāga꜓taṁ \\
  Ya꜕d'a꜕tītaṁ pa꜕hīnan-taṁ & a꜕ppattañc꜕a a꜕nāga꜕taṁ \\
\end{twochants}

\begin{english}
  No revivir el pasado,\\
  ni especular sobre el futuro.\\
  El pasado ya pasó\\
  y el futuro aún no llegó.
\end{english}

\begin{twochants}
  Paccu꜕ppannañca꜕ yo dhammaṁ & tattha tattha vi꜓pa꜕ss꜕ati \\
  Asa꜓ṁhi꜕raṁ asa꜓ṅku꜕ppaṁ & taṁ viddhām-a꜕nu꜕brūhaye \\
\end{twochants}

\begin{english}
  Así, todo lo que surge en el presente,\\
  es entonces visto claramente:\\
  impasible, libre de agitación ---\\
  esa realización será su fuerza.
\end{english}

\begin{twochants}
  A꜕jj'eva ki꜕cca꜕m-ātappaṁ & ko jaññā ma꜓ra꜕ṇaṁ su꜕ve \\
  Na hi no sa꜓ṅga꜕ran-tena & mahā-senena ma꜓cc꜕unā \\
\end{twochants}

\begin{english}
  Dedicarse hoy ardentemente a la tarea\\
  pues, quien sabe?, mañana la muerte puede venir.\\
  y ante el poderoso ejército de la muerte,\\
  no hay negociación posible.
\end{english}

\clearpage

\begin{twochants}
  Evaṁ vihārim-ātāpiṁ & a꜕ho-rattam-a꜕tandi꜓taṁ \\
  Taṁ ve bha꜕dd'eka꜕-ratto'ti & santo ā꜕ci꜕kkha꜕te muni \\
\end{twochants}

\begin{english}
  Aquél que así vive con vigor,\\
  sosteniéndolo día y noche,\\
  tiene de hecho `una noche auspiciosa' ---\\
  así dice el Pacífico Sábio.
\end{english}

\chapter{Versos de Respeto por el Dhamma}

\firstline{Ye ca atītā sambuddhā}

\begin{leader}
  [Ha꜓nda mayaṁ dhamma-gā꜕rav'ādi꜕-gāthā꜓yo bha꜕ṇāmase]
\end{leader}

\begin{twochants}
  Ye ca꜕ atītā sa꜓mbuddhā & ye ca꜕ buddhā a꜕nāga꜓tā \\
  Yo c'eta꜕rahi sambuddho & ba꜕hunnaṁ so꜕ka꜕-nāsa꜕no \\
\end{twochants}

\begin{english}
  Todos los Buddhas del pasado,\\
  todos los Buddhas del futuro,\\
  el Buddha actual ---\\
  cuán disipadores de pesar.
\end{english}

\begin{twochants}
  Sa꜕bbe sa꜕ddhamma-gar꜓uno & vi꜕ha꜕riṁsu vi꜕ha꜕ranti ca \\
  A꜕tho pi viha꜕riss꜓anti & esā buddhāna꜓ dha꜕mma꜕tā \\
\end{twochants}

\begin{english}
  Aquellos que vivieron o que están vivos,\\
  y aquellos que vivirán en el futuro,\\
  todos reverencian el Verdadero Dhamma,\\
  tal es la naturaleza de los Buddhas.
\end{english}

\begin{twochants}
  Tasmā h꜕i atta-kāmena & mahattam-abhika꜓ṅkh꜕atā \\
  Sa꜕ddhammo ga꜕ru꜓-kāta꜕bbo & s꜕araṁ buddhāna sā꜓sa꜕naṁ \\
\end{twochants}

\begin{english}
  Así, anhelando el bienestar\\
  y seguiendo las más elevadas aspiraciones,\\
  considere el Verdadero Dhamma\\
  acordandose de las enseñanzas de Buddha.
\end{english}

\clearpage

\begin{twochants}
  Na h꜕i dhammo a꜕dhammo ca & ubho s꜕ama-vipāki꜓no \\
  A꜕dhammo nirayaṁ neti & dh꜕ammo pāpeti꜕ su꜕gga꜕tiṁ \\
\end{twochants}

\begin{english}
  El verdadero Dhamma y el falso\\
  nunca traeran los mismos resultados.\\
  El Dhamma falso conduce al infierno,\\
  el verdadero Dhamma lleva al cielo.
\end{english}

Dhammo ha꜕ve rakkha꜕ti꜕ dhamma꜓-cāriṁ\\
Dhammo su꜕ciṇṇo su꜕kham-āvahāti\\
Esā꜓ni꜕saṁso dhamme su꜕ciṇṇe

% The last line is always omitted when chanting this verse in Thai monasteries, for some unknown reason.
% Na duggatiṁ gacchati dhamma-cārī
% (Thag 4.10, Dhammikattheragāthā)

\begin{english}
  El Dhamma protege a aquél que lo cultiva\\
  y conduce a la felicidad cuando es bien practicado ---\\
  esta es una bendición del Dhamma debidamente cultivado.
\end{english}

\chapter{Ovāda-Pāṭimokkha}

\firstline{Khantī paramaṁ tapo tītikkhā}

\enlargethispage{\baselineskip}

\begin{leader}
  [Ha꜓nda mayaṁ ovāda-pā꜕ṭi꜕mokkha-gāthā꜓yo bha꜕ṇāmase]
\end{leader}

Sa꜕bb꜕a-pāpa꜕ss꜕a a꜕ka꜕ra꜓ṇaṁ
	
	\begin{english}
		Evitar todo el mal,
	\end{english}
	
	Ku꜕salassūpasa꜓mpa꜕dā
	
	\begin{english}
		cultivar el bien
	\end{english}
	
	Sa꜕ci꜕tta-pa꜕ri꜓yoda꜓pa꜕naṁ
	
	\begin{english}
		y purificar la mente ---
	\end{english}
	
	Etaṁ buddhāna sā꜓sa꜕naṁ
	
	\begin{english}
		Esta es la enseñanza de los Buddhas.
	\end{english}


Kha꜓ntī pa꜕ramaṁ ta꜕po tīti꜕kkhā

\begin{english}
  Permanecer paciente es la mayor austeridad.
\end{english}

Nibbānaṁ pa꜕ramaṁ va꜕dant꜕i buddhā

\begin{english}
  “Nibbāna es supremo”, dicen los Buddhas.
\end{english}

Na h꜕i pa꜕bbaji꜕to pa꜕rūpaghātī

\begin{english}
  No se es verdaderamente monje \pause\ cuando se prejudica a alguien,
\end{english}

Sa꜕maṇo ho꜓ti pa꜕raṁ vihe꜓ṭha꜕yanto

\begin{english}
  No se es verdaderamente renunciante \pause\ cuando se oprime a los demás.
\end{english}

A꜕nūpa꜕vādo a꜕nūpa꜕ghāto

\begin{english}
  No ofender, no perjudicar,
\end{english}

Pā꜕ṭimokkhe꜓ ca꜕ sa꜓ṁva꜕ro

\begin{english}
  moderarse de acuerdo con las reglas monásticas.
\end{english}

Mattaññu꜕tā ca꜕ bhatta꜕smiṁ

\begin{english}
  moderarse en la comida,
\end{english}

Pa꜕ntañca꜕ saya꜓n'āsa꜕naṁ

\begin{english}
  vivir en soledad,
\end{english}

A꜕dhici꜕tte ca꜕ āyogo

\begin{english}
  dedicarse a la conciencia elevada ---
\end{english}

Etaṁ buddhāna sā꜓sa꜕naṁ

\begin{english}
  Esta es la enseñanza de los Buddhas.
\end{english}

\chapter[La Primera Exclamación]{Versos sobre la Primera Exclamación de Buddha}

\firstline{Aneka-jāti-saṁsāraṁ}

\begin{leader}
  [Ha꜓nda mayaṁ paṭhama-bu꜕ddha-bhāsi꜕ta-gāthāyo bh꜕aṇāmase]
\end{leader}

\begin{twochants}
  A꜕neka꜕-jāti꜕-sa꜓ṁsā꜓raṁ & sa꜕ndhāviss꜓aṁ a꜕nibbi꜕saṁ \\
  Ga꜕ha-kā꜕raṁ ga꜕vesa꜓nto & dukkhā jāt꜕i pu꜕nappu꜕naṁ \\
\end{twochants}

\begin{english}
  Durante muchas vidas, en la ruleta de vida y muerte\\
  vagué indefinidamente.\\
  Al constructor de esta casa yo buscaba;\\
  cuanta pena da nacer una y otra vez!.
\end{english}

\begin{twochants}
  Ga꜕ha-kā꜕raka꜕ diṭṭho꜓'si & pu꜕na gehaṁ na kā꜓hasi \\
  Sa꜕bbā te phāsu꜕kā bhaggā & gaha-kūṭa꜓ṁ vi꜕saṅkh꜕ataṁ \\
  Visa꜓ṅkhā꜕ra-ga꜕taṁ ci꜕ttaṁ & taṇhānaṁ kh꜕aya꜕m-ajjh꜕agā \\
\end{twochants}

\begin{english}
  Oh! Constructor de esta casa, te he visto!\\
  No construirás nuevamente para mí.\\
  Todas tus vigas están partidas\\
  y la cumbrera aplastada.\\
  La mente ha alcanzado el Incondicionado;\\
  llegando al cese del anhelo.
\end{english}

\chapter[Las Últimas Instrucciones]{Versos sobre las Últimas Instrucciones}

\firstline{Handa dāni bhikkhave āmantayāmi vo}

\begin{leader}
  [Ha꜓nda mayaṁ pacchima-ovāda-gāthā꜓yo bha꜕ṇāmase]
\end{leader}

Handa dāni bhi꜓kkha꜕ve āmant꜕ayāmi꜓ vo

\begin{english}
  Ahora monjes, yo os digo:
\end{english}

Vaya-dhammā sa꜓ṅkhā꜓rā

\begin{english}
  El cambio es la naturaleza de las cosas condicionadas.
\end{english}

A꜕ppamādena sa꜓mpā꜕dethā'ti

\begin{english}
  Practiquen diligentemente --
\end{english}

Ayaṁ tathā꜓ga꜕tassa pa꜕cchi꜓mā vācā

\begin{english}
  estas son las últimas palabras del Tathāgata.
\end{english}

\chapter{Surgen a Partir de una Causa}

\firstline{Ye dhammā hetuppabhavā}

\begin{leader}
  [Ha꜓nda mayaṁ assajithera-gāthā꜓yo bha꜕ṇāmase]
\end{leader}

Ye dhammā hetuppabhavā

\begin{english}
  Todos los fenómenos surgen de una causa:
\end{english}

Tesaṁ hetuṁ tathāgato āha

\begin{english}
  El Tathāgata explicó su causa
\end{english}

Tesañca yo nirodho

\begin{english}
  y también su cese.
\end{english}

Evaṁ-vādī mahāsamaṇo'ti

\begin{english}
  Esta es la enseñanza del Gran Asceta.
\end{english}

% \suttaRef{Mv.1.23.5}

\chapter[Incondicionado]{Reflexión sobre lo Incondicionado}

\firstline{Atthi bhikkhave ajātaṁ abhūtaṁ akataṁ}

\begin{leader}
  [Ha꜓nda mayaṁ nibbāna-sutta-pāṭhaṁ bha꜕ṇāmase]
\end{leader}

Atthi bhi꜓kkha꜕ve a꜕jātaṁ a꜓bhūtaṁ a꜕kataṁ a꜕sa꜓ṅkh꜕ataṁ

\begin{english}
  Existe un No-nacido, No-originado, Increado, No-formado.
\end{english}

N꜕o cetaṁ bhi꜓kkha꜕ve a꜕bhavissa a꜕jātaṁ a꜓bhūtaṁ a꜕kataṁ a꜕sa꜓nkh꜕ataṁ

\begin{english}
 Si no existiese este No-nacido, No-originado, Increado, No-formado,
\end{english}

Na꜕ yidaṁ jātassa꜕ bhūtassa ka꜕tassa sa꜓ṅkh꜕atassa nissaraṇaṁ paññāye꜓tha

\begin{english}
  La liberación del mundo de lo nacido, originado, criado, formado, no sería posible.
\end{english}

Ya꜕smā ca kho bhi꜓kkh꜕ave atthi a꜕jātaṁ a꜓bhūtaṁ a꜕kataṁ a꜕sa꜓ṅkha꜕taṁ

\begin{english}
  Pero una vez que existe un No-nacido, No-originado, Increado, No-formado,
\end{english}

Ta꜕smā jātass꜕a bhūtassa ka꜕tassa sa꜓ṅkha꜕tassa nissaraṇaṁ paññāyati

\begin{english}
  Así es posible la liberación del mundo de lo nacido, originado, criado, formado.
\end{english}

\chapter[Breve Consejo a Gotamī]{Breve Consejo a Gotamī}

\emph{Saṅkhitta-gotamiyovāda Sutta, AN 8.53}

% https://suttacentral.net/an8.53/pli/ms

\begin{leader}
  [Handa mayaṁ saṅkhitta-gotamiyovāda-sutta-pāṭhaṁ bhaṇāmase]
\end{leader}

Ye kho tvaṁ, gotami, dhamme jāneyyāsi:\\
‘ime dhammā sarāgāya saṁvattanti, no virāgāya;\\
saṁyogāya saṁvattanti, no visaṁyogāya;\\
ācayāya saṁvattanti, no apacayāya;\\
mahicchatāya saṁvattanti, no appicchatāya;\\
asantuṭṭhiyā saṁvattanti, no santuṭṭhiyā;\\
saṅgaṇikāya saṁvattanti, no pavivekāya;\\
kosajjāya saṁvattanti, no vīriyārambhāya;\\
dubbharatāya saṁvattanti, no subharatāyā’ti;

\begin{english}
  Gotamī, las cualidades que conozcais\\
  conduzcan a la pasión, no al desencanto;\\
  a la obstrucción, no a la desobstrucción;\\
  a acumular, no a desechar;\\
  al engrandecimento personal, no a la modestia;\\
  al descontento, no a la satisfacción;\\
  al enredo, no a la reclusión;\\
  a ser vago, no a la persistencia vivaz;\\
  a ser una carga, no a ser fácil de mantener;
\end{english}

ekaṁsena, gotami, dhāreyyāsi: ‘neso dhammo, neso vinayo, netaṁ satthusāsanan’ti.

\begin{english}
  podeis afirmar categóricamente: ‘Esto no es Dhamma,\\
  esto no es Vinaya, esto no es la instrucción del Maestro.’
\end{english}

Ye ca kho tvaṁ, gotami, dhamme jāneyyāsi:\\
‘ime dhammā virāgāya saṁvattanti, no sarāgāya;\\
visaṁyogāya saṁvattanti, no saṁyogāya;\\
apacayāya saṁvattanti, no ācayāya;\\
appicchatāya saṁvattanti, no mahicchatāya;\\
santuṭṭhiyā saṁvattanti, no asantuṭṭhiyā;\\
pavivekāya saṁvattanti, no saṅgaṇikāya;\\
vīriyārambhāya saṁvattanti, no kosajjāya;\\
subharatāya saṁvattanti, no dubbharatāyā’ti;

\begin{english}
  Cuando las cualidades que conozcais\\
  conduzcan al desencanto, no a la pasión;\\
  a la desobstrucción, no a la obstrucción;\\
  a desechar, no a acumular;\\
  a la modestia, no al engrandecimiento personal;\\
  a la satisfacción, no al descontento;\\
  a la reclusión, no al enredo;\\
  a la persistencia vivaz, no a ser vago;\\
  a ser fácil de mantener, no a ser una carga;
\end{english}

ekaṁsena, gotami, dhāreyyāsi: ‘eso dhammo, eso vinayo, etaṁ satthusāsanan’ti.

\begin{english}
  podeis afirmar categóricamente: ‘Esto es Dhamma,\\
  esto es Vinaya, esta es la instrucción del Maestro.’
\end{english}

\chapter{La Raíz de Todas las Cosas}

% AN 10.58
% https://www.accesstoinsight.org/tipitaka/an/an10/an10.058.than.html
% https://www.dhammatalks.org/suttas/AN/AN10_58.html
% https://suttacentral.net/an10.58/pli/ms
% https://suttacentral.net/an10.58/en/sujato
% https://suttacentral.net/an10.58/en/bodhi

\firstline{Kiṁ-mūlakā āvuso sabbe dhammā}

\begin{leader}
  [Ha꜓nda mayam mūlaka-sutta-pāṭhaṁ bha꜕ṇāmase]
\end{leader}

\enlargethispage{2\baselineskip}

Kiṁ-mūlakā āvuso sabbe dhammā\\
kiṁ-sambhavā sabbe dhammā\\
kiṁ-samudayā sabbe dhammā\\
kiṁ-samosaraṇā sabbe dhammā\\
kiṁ-pamukhā sabbe dhammā\\
kiṁ-adhipateyyā sabbe dhammā\\
kiṁ-uttarā sabbe dhammā\\
kiṁ-sārā sabbe dhammā\\
kiṁ-ogadhā sabbe dhammā\\
kiṁ-pariyosānā sabbe dhammā'ti.

Chanda'mūlakā āvuso sabbe dhammā\\
manasikāra'sambhavā sabbe dhammā\\
phassa'samudayā sabbe dhammā\\
vedanā'samosaraṇā sabbe dhammā\\
samādhi'ppamukhā sabbe dhammā\\
satā'dhipateyyā sabbe dhammā\\
paññ'uttarā sabbe dhammā\\
vimutti'sārā sabbe dhammā\\
amat'ogadhā sabbe dhammā\\
nibbāna'pariyosānā sabbe dhammā'ti.

\clearpage

\begin{english}
  Enraizadas en qué, amigo, están todas las cosas?\\
  Nacidas de qué, son todas las cosas?\\
  Provenientes de qué, son todas las cosas?\\
  Convergiendo en qué, son todas las cosas?\\
  Dirigidas por qué, son todas las cosas?\\
  Dominadas por qué, son todas las cosas?\\
  Superadas por qué, son todas las cosas?\\
  Resultan en qué como esencia, todas las cosas?\\
  Se funden en qué, todas las cosas?\\
  Terminan en qué, todas las cosas?

  \bigskip

  Amigo, todas las cosas están enraizadas en el deseo.\\
  Todas las cosas nacen de la atención.\\
  Todas las cosas provienen del contacto.\\
  Todas las cosas convergen en sensación.\\
  Todas las cosas son dirigidas por la concentración.\\
  Todas las cosas son dominadas por la consciencia.\\
  Todas las cosas son superadas por la sabiduría.\\
  Todas las cosas resultan en liberación como esencia.\\
  Todas las cosas se funden en la inmortalidad.\\
  Todas las cosas terminan en Nibbāna.

\end{english}

{\raggedleft
  \emph{Aṅguttara Nikāya 10.58}
\par}

\chapter{Ānāpānassati-sutta}

\firstline{Ānāpānassati bhikkhave bhāvitā bahulī-katā}

\begin{leader}
  [Ha꜓nda mayam ānāpānass꜕ati-sutta-pāṭhaṁ bha꜕ṇāmase]
\end{leader}

Ānāpāna꜓ssa꜕ti bhi꜓kkha꜕ve bhāvi꜓tā bahu꜕līka꜕tā

\begin{english}
  Bhikkhus, cuando ānāpānassati es cultivada y desarrollada,
\end{english}

Mahappha꜕lā ho꜓ti mahā꜓nisa꜓ṁsā

\begin{english}
  da grandes frutos y es de gran beneficio.
\end{english}

Ānāpāna꜓ssa꜕ti bhi꜓kkha꜕ve bhāvi꜓tā bahu꜕līka꜕tā

\begin{english}
  Bhikkhus, cuando cultivada y desarrollada, ānāpānassati
\end{english}

Ca꜕ttāro sati꜓pa꜕ṭṭhāne pa꜕ri꜓pū꜕reti

\begin{english}
  lleva las Cuatro Fundaciones de Sati a su plenitud;
\end{english}

Ca꜕ttāro sa꜕tipa꜕ṭṭhānā bhāvi꜓tā bahu꜕līka꜕tā

\begin{english}
  cuando cultivadas y desarrolladas, las Cuatro Fundaciones de Sati
\end{english}

Sa꜕tta-bojjhaṅge pa꜕ri꜓pū꜕renti

\begin{english}
  llevan los Siete Factores del Despertar a su plenitud;
\end{english}

Sa꜕tta-bojjhaṅgā bhāvi꜓tā bahu꜕līka꜕tā

\begin{english}
  cuando cultivados y desarrollados, los Siete Factores del Despertar
\end{english}

\enlargethispage{\baselineskip}

Vijjā-vimuttiṁ pa꜕ri꜓pū꜕renti

\begin{english}
  llevan al verdadero conocimento y la libertación a su plenitud.
\end{english}

Kathaṁ bhāvi꜓tā ca bhi꜓kkha꜕ve ānāpāna꜓ss꜕ati ka꜕thaṁ bahu꜕līka꜕tā

\begin{english}
  Y cómo, bhikkhus, ānāpānassati es cultivada y desarrollada
\end{english}

Mahappha꜕lā ho꜓ti mahā꜓nisa꜓ṁsā

\begin{english}
  para dar grandes frutos y ser de gran beneficio?
\end{english}

Idha bhi꜓kkha꜕ve bhikkhu

\begin{english}
  Bhikkhus, aquí un bhikkhu
\end{english}

Arañña꜓-ga꜕to vā

\begin{english}
  habiendo ido al bosque,
\end{english}

Rukkha-mūla꜓-ga꜕to vā

\begin{english}
  hacia la raiz de un árbol
\end{english}

Suññāgāra꜓-ga꜕to vā

\begin{english}
  o hacia una cabaña vacía,
\end{english}

N꜕isīdati pallaṅkaṁ ābhuji꜓tv꜕ā

\begin{english}
  Se sienta de piernas cruzadas,
\end{english}

Ujuṁ kāyaṁ pa꜕ṇidhāya pa꜕rimukhaṁ sa꜕tiṁ u꜕paṭṭha꜕petvā

\begin{english}
  con su cuerpo derecho y establece sati en frente suya.
\end{english}

So sa꜕to'va a꜕ssasa꜕ti sa꜕to'va pa꜕ssa꜕sa꜕ti

\begin{english}
  Consciente, él inspira; consciente, él expira.
\end{english}

Dīghaṁ vā assa꜕sa꜓nto dīghaṁ a꜕ssasā꜓mī'ti pa꜕jānāti

\begin{english}
  Al tener una inspiración larga, él sabe `Esta es una inspiración larga';
\end{english}

Dīghaṁ vā pa꜕ssa꜕santo dīghaṁ pa꜕ssasā꜓mī'ti pa꜕jānāti

\begin{english}
  Al tener una espiración larga, él sabe `Esta es una espiración larga';
\end{english}

Rassaṁ vā a꜕ssa꜕santo rassaṁ a꜕ssasā꜓mī'ti pa꜕jānāti

\begin{english}
  Al tener una inspiración corta, él sabe `Esta es una inspiración corta';
\end{english}

Rassaṁ vā pa꜕ssa꜕santo rassaṁ pa꜕ssasā꜓mī'ti pa꜕jānāti

\begin{english}
  Al tener una espiración corta, él sabe `Esta es una espiración corta';
\end{english}

Sabba꜕-kāya-paṭ꜕isa꜓ṁvedī a꜕ssasi꜕ssāmī'ti si꜕kkh꜕ati

\begin{english}
  Él se entrena: `Voy a inspirar experimentando el cuerpo entero'.
\end{english}

Sabba꜕-kāya-paṭ꜕isa꜓ṁvedī pa꜕ssasi꜕ssāmī'ti si꜕kkh꜕ati

\begin{english}
  Él se entrena: `Voy a espirar experimentando el cuerpo entero'.
\end{english}

Passa꜕mbhayaṁ kāya꜕-sa꜓ṅkhāraṁ a꜕ssasi꜕ssāmī'ti si꜕kkh꜕ati

\begin{english}
  Él se entrena: `Voy a inspirar tranquilizando el cuerpo'.
\end{english}

Passa꜕mbhayaṁ kāya꜕-sa꜓ṅkhāraṁ pa꜕ssasi꜕ssāmī'ti si꜕kkh꜕ati

\begin{english}
  Él se entrena: `Voy a espirar tranquilizando el cuerpo'.
\end{english}

Pīti꜕-paṭi꜕sa꜓ṁvedī a꜕ssasi꜕ssāmī'ti si꜕kkh꜕ati

\begin{english}
  Él se entrena: `Voy a inspirar experimentando éxtasis'.
\end{english}

Pīti꜕-paṭi꜕sa꜓ṁvedī pa꜕ssasi꜕ssāmī'ti si꜕kkh꜕ati

\begin{english}
  Él se entrena: `Voy a espirar experimentando éxtasis'.
\end{english}

Sukh꜕a-paṭi꜕sa꜓ṁvedī a꜕ssasi꜕ssāmī'ti si꜕kkh꜕ati

\begin{english}
  Él se entrena: `Voy a inspirar experimentando felicidad'.
\end{english}

Sukh꜕a-paṭi꜕sa꜓ṁvedī pa꜕ssasi꜕ssāmī'ti si꜕kkh꜕ati

\begin{english}
  Él se entrena: `Voy a espirar experimentando felicidad'.
\end{english}

Citta꜕-sa꜓ṅkhāra-paṭi꜕sa꜓ṁvedī a꜕ssasi꜕ssāmī'ti si꜕kkh꜕ati

\begin{english}
  Él se entrena: `Voy a inspirar experimentando las formaciones mentales'.
\end{english}

Citta꜕-sa꜓ṅkhāra-paṭi꜕sa꜓ṁvedī pa꜕ssasi꜕ssāmī'ti si꜕kkh꜕ati

\begin{english}
  Él se entrena: `Voy a espirar experimentando las formaciones mentales'.
\end{english}

Passa꜕mbhayaṁ citta꜕-sa꜓ṅkhāraṁ a꜕ssasi꜕ssāmī'ti si꜕kkh꜕ati

\begin{english}
  Él se entrena: `Voy a inspirar tranquilizando las formaciones mentales'.
\end{english}

Passa꜕mbhayaṁ citt꜕a-sa꜓ṅkhāraṁ pa꜕ssasi꜕ssāmī'ti si꜕kkh꜕ati

\begin{english}
  Él se entrena: `Voy a espirar tranquilizando las formaciones mentales'.
\end{english}

Citta꜕-paṭi꜕sa꜓ṁvedī a꜕ssasi꜕ssāmī'ti si꜕kkh꜕ati

\begin{english}
  Él se entrena: `Voy a inspirar experimentando la mente'.
\end{english}

Citta꜕-paṭi꜕sa꜓ṁvedī pa꜕ssasi꜕ssāmī'ti si꜕kkh꜕ati

\begin{english}
  Él se entrena: `Voy a espirar experimentando la mente'.
\end{english}

A꜕bhippa꜕moda꜓yaṁ cittaṁ a꜕ssasi꜕ssāmī'ti si꜕kkh꜕ati

\begin{english}
  Él se entrena: `Voy a inspirar alegrando la mente'.
\end{english}

A꜕bhippa꜕moda꜓yaṁ cittaṁ pa꜕ssasi꜕ssāmī'ti si꜕kkh꜕ati

\begin{english}
  Él se entrena: `Voy a espirar alegrando la mente'.
\end{english}

Sa꜕māda꜓haṁ cittaṁ a꜕ssasi꜕ssāmī'ti si꜕kkh꜕ati

\begin{english}
  Él se entrena: `Voy a inspirar concentrando la mente'.
\end{english}

Sa꜕māda꜓haṁ cittaṁ pa꜕ssasi꜕ssāmī'ti si꜕kkh꜕ati

\begin{english}
  Él se entrena: `Voy a espirar concentrando la mente'.
\end{english}

Vimoca꜓yaṁ cittaṁ a꜕ssasi꜕ssāmī'ti si꜕kkh꜕ati

\begin{english}
  Él se entrena: `Voy a inspirar liberando la mente'.
\end{english}

Vimoca꜓yaṁ cittaṁ pa꜕ssasi꜕ssāmī'ti si꜕kkh꜕ati

\begin{english}
  Él se entrena: `Voy a espirar liberando la mente'.
\end{english}

Aniccānupa꜕ssī a꜕ssasi꜕ssāmī'ti si꜕kkh꜕ati

\begin{english}
  Él se entrena: `Voy a inspirar contemplando impermanencia'.
\end{english}

Aniccānupa꜕ssī pa꜕ssasi꜕ssāmī'ti si꜕kkh꜕ati

\begin{english}
  Él se entrena: `Voy a espirar contemplando impermanencia'.
\end{english}

Virāgānupa꜕ssī a꜕ssasi꜕ssāmī'ti si꜕kkh꜕ati

\begin{english}
  Él se entrena: `Voy a inspirar contemplando el desvanecer de las pasiones'.
\end{english}

Virāgānupa꜕ssī pa꜕ssasi꜕ssāmī'ti si꜕kkh꜕ati

\begin{english}
  Él se entrena: `Voy a espirar contemplando el desvanecer de las pasiones'.
\end{english}

Nirodhānupa꜕ssī a꜕ssasi꜕ssāmī'ti si꜕kkh꜕ati

\begin{english}
  Él se entrena: `Voy a inspirar contemplando el cese'.
\end{english}

Nirodhānupa꜕ssī pa꜕ssasi꜕ssāmī'ti si꜕kkh꜕ati

\begin{english}
  Él se entrena: `Voy a espirar contemplando el cese'.
\end{english}

Pa꜕ṭiniss꜕aggānupa꜕ssī a꜕ssasi꜕ssāmī'ti si꜕kkh꜕ati

\begin{english}
  Él se entrena: `Voy a inspirar contemplando renuncia'.
\end{english}

Pa꜕ṭinissa꜕ggānupa꜕ssī pa꜕ssasi꜕ssāmī'ti si꜕kkh꜕ati

\begin{english}
  Él se entrena: `Voy a espirar contemplando renuncia'.
\end{english}

Evaṁ bhāvi꜓tā kho bhi꜓kkha꜕ve ānāpāna꜓ss꜕ati evaṁ bahu꜕līka꜕tā

\begin{english}
  Bhikkhus, así ānāpānassati es cultivada y desarrollada,
\end{english}

Mahappha꜕lā ho꜓ti mahā꜓nisa꜓ṁsā'ti

\begin{english}
  para dar grandes frutos y ser de gran beneficio.
\end{english}


\chapter[Aparihāniya-dhamma-sutta]{Bhikkhu-aparihāniya-dhamma-sutta}

\emph{Siete condiciones para la buenaventura de los bhikkhus, AN 7.23}

\begin{leader}
  [Handa mayaṁ bhikkhu-aparihāniya-dhamma-suttaṁ bhaṇāmase]
\end{leader}

[Evaṁ me sutaṁ.] Ekaṁ samayaṁ bhagavā rājagahe꜔꜒ viharati gijjhakūṭe pabbate.
Tatra kho꜔꜒ bhagavā bhikkhū꜔꜒ āmantesi: Satta vo, bhikkhave, aparihā꜔꜒niye dhamme
desessā꜔꜒mi. Taṁ suṇātha, sā꜔꜒dhukaṁ manasi karotha, bhāsissā꜔꜒mī'ti. Evaṁ, bhante'ti
kho꜔꜒ te bhikkhū꜔꜒ bhagavato paccasso꜔꜒su꜔꜒ṁ. Bhagavā etadavoca:

\begin{english}
  He oido que en cierta ocasión el Señor estaba en Rajagaha, en el Pico de los
  Buitres. Allí, él se dirigió a los monjes: `Monjes, les voy a enseñar las siete
  condiciones que no llevan al declive. Oigan y presten mucha atención. Voy
  a hablar.' `Si, Señor', respondieron los monjes. El Señor dijo:
\end{english}

[1] Katame ca, bhikkhave, satta aparihā꜔꜒niyā dhammā? Yāvakīvañca, bhikkhave, bhikkhū꜔꜒
abhiṇha꜔꜒ṁ sa꜔꜒nnipātā bhavissa꜔꜒nti sa꜔꜒nnipātabahulā; vuddhiyeva, bhikkhave,
bhikkhū꜔꜒naṁ pāṭikaṅkhā꜔꜒, no parihā꜔꜒ni.

\begin{english}
  `Y cuales son las siete condiciones que no llevan al declive? Mientras los monjes
  se reunan con frequencia, se reunan asiduamente, su crecimiento puede ser
  esperado, no su declive.'

\end{english}

[2] Yāvakīvañca, bhikkhave, bhikkhū꜔꜒ samaggā sa꜔꜒nnipatissa꜔꜒nti, samaggā
vuṭṭhahissa꜔꜒nti, samaggā sa꜔꜒ṅghakaraṇīyāni karissa꜔꜒nti; vuddhiyeva, bhikkhave,
bhikkhū꜔꜒naṁ pāṭikaṅkhā꜔꜒, no parihā꜔꜒ni.

\begin{english}
  `Mientras los monjes se reunan en armonía, se dispersen en
  armonía y conduzcan los asuntos de la Saṅgha en armonía, su crecimiento puede ser
  esperado, no su declive.'
\end{english}

[3] Yāvakīvañca, bhikkhave, bhikkhū꜔꜒ apaññattaṁ na paññāpessa꜔꜒nti, paññattaṁ na
samucchi꜔꜒ndissa꜔꜒nti, yathā꜔꜒paññattesu sikkhā꜔꜒padesu samādāya vattissa꜔꜒nti;
vuddhiyeva, bhikkhave, bhikkhū꜔꜒naṁ pāṭikaṅkhā꜔꜒, no parihā꜔꜒ni.

\begin{english}
  `Mientras los monjes no decreten lo que no fue decretado, no revoquen lo que
  fue decretado, pero practiquen el cumplimiento de las reglas de entrenamiento conforme
  fueron decretadas, su crecimiento puede ser
  esperado, no su declive.'
\end{english}

[4] Yāvakīvañca, bhikkhave, bhikkhū꜔꜒ ye te bhikkhū꜔꜒ the꜔꜒rā rattaññū cirapabbajitā
sa꜔꜒ṅghapitaro sa꜔꜒ṅghapariṇāyakā te sakkarissa꜔꜒nti garuṁ karissa꜔꜒nti mānessa꜔꜒nti
pūjessa꜔꜒nti, tesa꜔꜒ñca so꜔꜒tabbaṁ maññissa꜔꜒nti; vuddhiyeva, bhikkhave, bhikkhū꜔꜒naṁ
pāṭikaṅkhā꜔꜒, no parihā꜔꜒ni.

\begin{english}
  `Mientras los monjes honren, respeten, veneren y rindan homenaje a los
  monjes más viejos -- aquellos con antigüedad que fueron ordenados hace mucho
  tiempo, los padres de la Saṅgha, los líderes de la Saṅgha -- considerando muy valioso
  el oirlos, su crecimiento puede ser
  esperado, no su declive.'
\end{english}

\enlargethispage{2\baselineskip}

[5] Yāvakīvañca, bhikkhave, bhikkhū꜔꜒ uppannāya taṇhā꜔꜒ya ponobhavikāya na vasa꜔꜒ṁ
gacchissa꜔꜒nti; vuddhiyeva, bhikkhave, bhikkhū꜔꜒naṁ pāṭikaṅkhā꜔꜒, no parihā꜔꜒ni.

\begin{english}
  `Mientras los monjes no se submitan al poder de cualquier deseo que surja y
  que lleve a un futuro nacimiento, su crecimiento puede ser
  esperado, no su declive.'
\end{english}

[6] Yāvakīvañca, bhikkhave, bhikkhū꜔꜒ āraññakesu se꜔꜒nāsanesu sā꜔꜒pekkhā꜔꜒ bhavissa꜔꜒nti;
vuddhiyeva, bhikkhave, bhikkhū꜔꜒naṁ pāṭikaṅkhā꜔꜒, no parihā꜔꜒ni.

\begin{english}
  `Mientras los monjes se regocijen viviendo en el bosque,
  su crecimiento puede ser esperado, no su declive.'
\end{english}

[7] Yāvakīvañca, bhikkhave, bhikkhū꜔꜒ paccattaññeva satiṁ upaṭṭhā꜔꜒pessa꜔꜒nti: Kinti
anāgatā ca pesalā sabrahmacārī āgacche꜔꜒yyuṁ, āgatā ca pesalā sabrahmacārī phā꜔꜒su꜔꜒ṁ
vihareyyun'ti; vuddhiyeva, bhikkhave, bhikkhū꜔꜒naṁ pāṭikaṅkhā꜔꜒, no parihā꜔꜒ni.

\begin{english}
  `Mientras cada uno de los monjes mantenga firmemente en mente: ``Si hubiera
  compañeros bien-comportados, seguidores de la vida casta que aún están por
  venir, que puedan ellos venir; y que los compañeros bien-comportados de la vida casta
  que vinieron puedan vivir en paz'', su crecimiento puede ser esperado, no su declive.'
\end{english}

Yāvakīvañca, bhikkhave, ime satta aparihā꜔꜒niyā dhammā bhikkhū꜔꜒su ṭhassa꜔꜒nti, imesu
ca sattasu aparihā꜔꜒niyesu dhammesu bhikkhū꜔꜒ sa꜔꜒ndississa꜔꜒nti; vuddhiyeva, bhikkhave,
bhikkhū꜔꜒naṁ pāṭikaṅkhā꜔꜒, no parihā꜔꜒nī'ti. Idam-avoca bhagavā. Attamanā te bhikkhū꜔꜒
bhagavato bhāsitaṁ abhinandun'ti.

\begin{english}
  `Mientras los monjes permanezcan resueltos en estas siete condiciones, y mientras
  estas siete condiciones persistan entre los monjes, se puede esperar su
  crecimiento, no su declive.' Esto es lo que el Señor dijo. Satisfechos,
  los monjes se deleitaron en las palabras del Señor.
\end{english}
