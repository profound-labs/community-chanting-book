\chapter{Glosario de términos en Pāli }

\enlargethispage{2\baselineskip}

\begin{description}

\item[Anatta] Literalmente, `no-alma,' i.e. impersonal, sin esencia individual;
 ni persona ni perteneciente a persona. Una de las tres características de todo fenómeno condicionado.

\item[Anicca] No permanente, inestable, de la naturaleza de aparecer y extinguirse. Una de las tres características de todo fenomeno condicionado.

\item[Arahaṁ/Arahant] Literalmente, ‘el que se lo merece’ --- un término aplicado a todos los seres iluminados. 

\item[Ariyapuggalā] ‘Seres Nobles’ o ‘Nobles Discipulos’ --- existen ocho tipos: aquellos que trabajan por alcanzar o aquellos que han alcanzado los cuatro diferentes niveles de la iluminación.

\item[Bhagavā] Epíteto reservado al Señor Buddha.

\item[Bhikkhu] Un monje budista que vive como monje mendicante, siguiendo 227 preceptos de entrenamiento que definen una vida de renuncia y simplicidad.

\item[Brahmā] Ser celestial; un dios en uno de los mas altos reinos espirituales.

\item[Buddha] 'El que despierta', uno que conoce las cosas como son; un potencial que existe en todo ser humano. El último Buda conocido,
  Siddhattha Gotama, vivió y enseñó en la India en el 5th siglo BC.

\item[Deva] Ser celestial. Menos refinado que brahmā; parecido al concepto de 'Angel de la guarda'

\item[Dhamma] (Sanskrito: Dharma) The Teaching of the Buddha as contained
  in the scriptures; not dogmatic in character, but more like a raft or
  vehicle to convey the disciple to deliverance. Also, the Truth towards
  which that Teaching points; that which is beyond words, concepts or
  intellectual understanding. When written as ‘\emph{dhamma}’, i.e.
  con `d' minúscula, se refiere a 'cosa’ o 'elemento'.

\item[Dukkha] Literally, ‘hard to bear’ --- dis-ease, restlessness of
  mind, anguish, conflict, unsatisfactoriness, discontent, stress,
  suffering. One of the three characteristics of conditioned phenomena.

\item[Factors of Awakening (bojjhaṅga)] 1.~sati, 2.~investigación de Dhammas, 3.~esfuerzo, 4.~, 5.~tranquilidad, 6.~concentración, 7.~ecuanimidad.

\item[Foundations of Mindfulness (satipaṭṭhāna)] Mindfulness of 1.~\emph{kāya} (body), 2.~\emph{vedanā} (feelings), 3.~\emph{citta} (mind),
4.~\emph{dhamma} (mind-objects).

\item[Vida Santa (brahmacariya)] Literalmente: la conducta de Brahma; usualmente referido a la vida monástica. Usar este término implica voto de celibato.

\item[Jhāna] Absorción mental. Estado de gran concentración enfocado en una simple sensación o noción mental.

\item[Kamma] (Sanskrit: karma) Acción; acciones creadas por impulso habitual, con intención.

\item[Khandhā] Los cinco agregados que componen un ser, tanto fisicos como mentales ---
  estos son is: \emph{rūpa, vedanā, saññā, saṅkhārā, viññāṇa.} Apego a cualquier de estos como, ‘Esto es mio’, ‘Yo soy esto’ or, ‘Esto es mi alma’ is
  \emph{upādāna} --- apego.

\item[Māra] Personificación de las fuerzas del mal. Durante la lucha de Buda por la iluminación, Māra se manisfestó tratando de desviarle de su meta.

\item[Nibbāna] (Sanskrit: Nirvāṇa) Literally, ‘coolness’ --- the state of
  liberation from all suffering and defilements, the goal of the
  Buddhist path.

\item[Paccekabuddha] Solitary Buddha --- someone enlightened by their own
  efforts without relying on a teacher but who, unlike the Buddha, has
  no following of disciples.

\item[Paritta] Verses chanted particularly for blessing and protection.

\item[Parinibbāna] The Buddha’s final passing away, i.e. final entering
  Nibbāna.

\item[Puñña] Mérito, acumulación de buena fortuna, bendiciones o bienestar resultante de la práctica del Dhamma.

\item[Rūpa] Form or matter. The physical elements that make up the body,
  i.e. earth, water, fire and air (solidity, cohesion, temperature and
  vibration).

\item[Saṅgha] The community of those who practise the Buddha’s Way.

  More specifically, those who have formally committed themselves to the
lifestyle of mendicant monks and nuns. The `four pairs, the eight kinds of
noble beings' are those who are on the path to or who have realized the
fruition of the four stages of enlightenment: stream-entry, once-return,
non-return and arahantship.

\item[Saṅkhārā] Formations, constructions, all conditioned things, or volitional
  impulses, that is all mental states apart from feeling and perception
  that colour one’s thoughts and make them either good, bad or neutral.

\item[Saññā] Percepción, la funcion mental de reconocer. Impone valores a las cosas dependiendo de experiencias previas.

\item[Sati] Consciencia. Ser consciente de los estados mentales en cada momento.

\item[Tathāgata] ‘Así ido’ or ‘Así venido’ --- dando el sentido de 'el que viene de vuelta de todo' uno que ha ido más allá del sufrimiento y la muerte; uno que experimenta las cosas como realmente son, sin ilusión. El epíteto que Buda se aplicó a sí mismo después de obtener la Ilumincaión perfecta.

\item[Joya Triple] Buda, Dhamma y Sangha.

\item[Vedanā] Sensación --- Sensaciones mentales o físicas ya sean agradables, desagradables o neutras.

\item[Viññāṇa] Conciencia sensorial --- proceso por el cual puedo existir la visión, oir, oler, gustar, tocar y pensar.

\end{description}

