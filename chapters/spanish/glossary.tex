\chapter{Glosario de términos en Pāli }

\enlargethispage{2\baselineskip}

\begin{description}

\item[Anatta] Literalmente, ‘no-yo’, i.e. impersonal, sin esencia individual;
 ni persona ni perteneciente a persona. Una de las tres características de todo fenómeno condicionado.

\item[Anicca] No permanente, inestable, de la naturaleza de aparecer y extinguirse. Una de las tres características de todo fenómeno condicionado.

\item[Arahaṁ/Arahant] Literalmente, ‘el que se lo merece’ --- un término aplicado a todos los seres iluminados. 

\item[Ariyapuggalā] ‘Seres Nobles’ o ‘Nobles Discípulos’ --- existen ocho tipos: aquellos que trabajan por alcanzar o aquellos que han alcanzado los cuatro diferentes niveles de la iluminación.

\item[Bhagavā] Epíteto reservado al Señor Buddha.

\item[Bhikkhu] Un monje budista que vive como monje mendicante, siguiendo 227 preceptos de entrenamiento que definen una vida de renuncia y simplicidad.

\item[Brahmā] Ser celestial; un dios en uno de los más altos reinos espirituales.

\item[Buddha] ‘El que despierta’, uno que conoce las cosas como son; un potencial que existe en todo ser humano. El último Buddha conocido,
  Siddhattha Gotama, vivió y enseñó en la India en V a.C.

\item[Deva] Ser celestial. Menos refinado que brahmā; parecido al concepto de ‘Ángel de la guarda’

\item[Dhamma] (Sánscrito: Dharma) Las enseñanzas de Buddha contenidas en las escrituras, no como dogma de fe, más como una balsa o vehículo que lleva al discípulo hasta la liberación. También, significa la Verdad a la que se refieren las enseñanzas. Cuando se escribe como ‘\emph{dhamma}’, i.e. con ‘d’ minúscula, se refiere a ‘cosa’ o ‘elemento’.

\item[Dukkha] Literalmente, ‘difícil de soportar’ --- mal-estar, inquietud mental, angustia, insatisfacción, des-contento, estrés, sufrimiento. Una de las tres características de los fenómenos condicionados.

\item[Vida Santa (brahmacariya)] Literalmente: la conducta de Brahma; usualmente referido a la vida monástica. Usar este término implica voto de celibato.

\item[Jhāna] Absorción mental. Estado de gran concentración enfocado en una simple sensación o noción mental.

\item[Kamma] (Sánscrito: karma) Acción intencional; acciones creadas por impulso habitual, con intención.

\item[Khandhā] Los cinco agregados que componen un ser, tanto fisicos como mentales ---
  estos son: \emph{rūpa, vedanā, saññā, saṅkhārā, viññāṇa.} Apego a cualquier de estos como ‘Esto es mio’, ‘Yo soy esto’ o ‘Esto es yo’.

\item[Māra] Personificación de las fuerzas del mal. Durante la lucha de Buddha por la iluminación, Māra se manifestó tratando de desviarle de su meta.

\item[Nibbāna] (Sánscrito: Nirvāṇa) Literalmente, ‘frescor’ --- el estado de liberación de todo sufrimiento e impurezas, la meta del camino budista.

\item[Paccekabuddha] Buddha solitario --- alguien que se ilumina por sus propios medios sin depender de maestro alguno, pero que (a diferencia de Buddha) no posee discípulos.

\item[Paritta] Versos cantados particularmente para bendiciones y protección.

\item[Parinibbāna] El perecimiento definitivo de Buddha, i.e. entrada definitiva en 
  Nibbāna.

\item[Puñña] Mérito, acumulación de buena fortuna, bendiciones o bienestar resultante de la práctica del Dhamma.

\item[Rūpa] Forma o materia. Los elementos físicos que componen el cuerpo,
  i.e. tierra, agua, fuego y aire (solidez, cohesión, temperatura y
  vibración).

\item[Saṅgha] La comunidad de aquellos que practican el camino de Buddha.
  Más específicamente, aquellos que se comprometen formalmente al estilo de vida de monásticos mendicantes.

\item[Saṅkhārā] Formaciones, construcciones, todas las cosas condicionadas, o impulsos voluntarios, es decir, todos los estados mentales con excepción de vedana y sañña, que le dan matiz a los pensamientos volviéndolos en ‘buenos’, ‘malos’ o ‘neutros’.

\item[Saññā] Percepción, la función mental de reconocer. Impone valores a las cosas dependiendo de experiencias previas.

\item[Sati] Este término proviene del verbo \textit{sarati} que significa ‘recordar’. ¿Recordar el qué? Recordar el objetivo de la práctica. 

\item[Tathāgata] ‘Así ido’ o ‘Así venido’ --- dando el sentido de ‘el que viene de vuelta de todo’ uno que ha ido más allá del sufrimiento y la muerte; uno que experimenta las cosas como realmente son, sin ilusión. El epíteto que Buddha se aplicó a sí mismo después de obtener la Iluminación perfecta.

\item[Joya Triple] Buddha, Dhamma y Saṅgha.

\item[Vedanā] Sensación --- Sensaciones mentales o físicas ya sean agradables, desagradables o neutras.

\item[Viññāṇa] Conciencia sensorial --- proceso por el cual puede existir la visión, el oído, el olfato, el gusto, el tocar y el pensar.

\end{description}

