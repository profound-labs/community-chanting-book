\chapter{Invitación a los Devas}

\firstline{Pharitvāna mettaṁ samettā bhadantā}
\firstline{Samantā cakka-vāḷesu}

\begin{paritta}
\sidepar{A.}%
Pharitvāna mettaṁ samettā bhadantā\\
Avikkhitta-cittā parittaṁ bhaṇantu

\sidepar{B.}%
Samantā cakka-vāḷesu\\
Atr'āgacchantu devatā\\
Saddhammaṁ muni-rājassa\\
Suṇantu sagga-mokkha-daṁ

Sagge kāme ca rūpe\\
Giri-sikhara-taṭe c'antalikkhe vimāne\\
Dīpe raṭṭhe ca gāme\\
Taru-vana-gahane geha-vatthumhi khette\\
Bhummā c'āyantu devā\\
Jala-thala-visame yakkha-gandhabba-nāgā\\
Tiṭṭhantā santike yaṁ\\
Muni-vara-vacanaṁ sādhavo me suṇantu

Dhammassavana-kālo ayam-bhadantā

\instr{Tres veces, o la alternativa}

Buddha-dassana-kālo ayam-bhadantā\\
Dhammassavana-kālo ayam-bhadantā\\
Saṅgha-payirūpāsana-kālo ayam-bhadantā
\end{paritta}

\clearpage

\chapter{Pubba-bhāga-nama-kāra-pāṭho}

\firstline{Namo tassa Bhagavato}

\begin{paritta}
Namo tassa bhagavato arahato sammā-sambuddhassa\\
Namo tassa bhagavato arahato sammā-sambuddhassa\\
Namo tassa bhagavato arahato sammā-sambuddhassa
\end{paritta}

\chapter{Saraṇa-gamana-pāṭho}

\firstline{Buddhaṁ saraṇaṁ gacchāmi}

\begin{paritta}
Buddhaṁ saraṇaṁ gacchāmi\\
Dhammaṁ saraṇaṁ gacchāmi\\
Saṅghaṁ saraṇaṁ gacchāmi

Dutiyam pi buddhaṁ saraṇaṁ gacchāmi\\
Dutiyam pi dhammaṁ saraṇaṁ gacchāmi\\
Dutiyam pi saṅghaṁ saraṇaṁ gacchāmi

Tatiyam pi buddhaṁ saraṇaṁ gacchāmi\\
Tatiyam pi dhammaṁ saraṇaṁ gacchāmi\\
Tatiyam pi saṅghaṁ saraṇaṁ gacchāmi
\end{paritta}

\clearpage

\chapter{Nama-kāra-siddhi-gāthā}

\firstline{Yo cakkhumā moha-malāpakaṭṭho}

\begin{paritta}
Yo cakkhumā moha-malāpakaṭṭho\\
Sāmaṁ va buddho sugato vimutto\\
Mārassa pāsā vinimocayanto\\
Pāpesi khemaṁ janataṁ vineyyaṁ\\
Buddhaṁ varan-taṁ sirasā namāmi\\
Lokassa nāthañ-ca vināyakañ-ca\\
Tan-tejasā te jaya-siddhi hotu\\
Sabb'antarāyā ca vināsamentu

Dhammo dhajo yo viya tassa satthu\\
Dassesi lokassa visuddhi-maggaṁ\\
Niyyāniko dhamma-dharassa dhārī\\
Sāt'āvaho santi-karo suciṇṇo\\
Dhammaṁ varan-taṁ sirasā namāmi\\
Mohappadālaṁ upasanta-dāhaṁ\\
Tan-tejasā te jaya-siddhi hotu\\
Sabb'antarāyā ca vināsamentu

Saddhamma-senā sugatānugo yo\\
Lokassa pāpūpakilesa-jetā\\
Santo sayaṁ santi-niyojako ca\\
Svākkhāta-dhammaṁ viditaṁ karoti\\
Saṅghaṁ varan-taṁ sirasā namāmi\\
Buddhānubuddhaṁ sama-sīla-diṭṭhiṁ\\
Tan-tejasā te jaya-siddhi hotu\\
Sabb'antarāyā ca vināsamentu
\end{paritta}

\clearpage

\chapter{Namo-kāra-aṭṭhaka}

\firstline{Namo arahato sammā}

\begin{paritta}
Namo arahato sammā\\
Sambuddhassa mahesino\\
Namo uttama-dhammassa\\
Svākkhātass'eva ten'idha\\
Namo mahā-saṅghassāpi\\
Visuddha-sīla-diṭṭhino\\
Namo omāty-āraddhassa\\
Ratanattayassa sādhukaṁ\\
Namo omakātītassa\\
Tassa vatthuttayassa-pi\\
Namo-kārappabhāvena\\
Vigacchantu upaddavā\\
Namo-kārānubhāvena\\
Suvatthi hotu sabbadā\\
Namo-kārassa tejena\\
Vidhimhi homi tejavā\\
\end{paritta}

\clearpage

\chapter{Ratana-sutta}

\firstline{Yaṅkiñci vittaṁ idha vā huraṁ vā}

\begin{paritta}

Yaṅkiñci vittaṁ idha vā huraṁ vā\\
Saggesu vā yaṁ ratanaṁ paṇītaṁ\\
Na no samaṁ atthi tathāgatena\\
Idam-pi buddhe ratanaṁ paṇītaṁ\\
Etena saccena suvatthi hotu

Khayaṁ virāgaṁ amataṁ paṇītaṁ\\
Yad-ajjhagā sakya-munī samāhito\\
Na tena dhammena sam'atthi kiñci\\
Idam-pi dhamme ratanaṁ paṇītaṁ\\
Etena saccena suvatthi hotu

Yam buddha-seṭṭho parivaṇṇayī suciṁ\\
Samādhim-ānantarikaññam-āhu\\
Samādhinā tena samo na vijjati\\
Idam-pi dhamme ratanaṁ paṇītaṁ\\
Etena saccena suvatthi hotu

Ye puggalā aṭṭha sataṁ pasaṭṭhā\\
Cattāri etāni yugāni honti\\
Te dakkhiṇeyyā sugatassa sāvakā\\
Etesu dinnāni mahapphalāni\\
Idam-pi saṅghe ratanaṁ paṇītaṁ\\
Etena saccena suvatthi hotu

\clearpage

Ye suppayuttā manasā daḷhena\\
Nikkāmino gotama-sāsanamhi\\
Te patti-pattā amataṁ vigayha\\
Laddhā mudhā nibbutiṁ bhuñjamānā\\
Idam-pi saṅghe ratanaṁ paṇītaṁ\\
Etena saccena suvatthi hotu

Khīṇaṁ purāṇaṁ navaṁ n'atthi sambhavaṁ\\
Viratta-citt'āyatike bhavasmiṁ\\
Te khīṇa-bījā aviruḷhi-chandā\\
Nibbanti dhīrā yathā'yam padīpo\\
Idam-pi saṅghe ratanaṁ paṇītaṁ\\
Etena saccena suvatthi hotu

% NOTE: References are commented out, but keep them in the source for information.
%\suttaref{(Sn.vv.224-241; Khp.VI)}

\end{paritta}

\clearpage

\chapter{Maṅgala-sutta}

\firstline{Asevanā ca bālānaṁ}

\begin{paritta}
Asevanā ca bālānaṁ\\
Paṇḍitānañ-ca sevanā\\
Pūjā ca pūjanīyānaṁ\\
Etam maṅgalam-uttamaṁ

Paṭirūpa-desa-vāso ca\\
Pubbe ca kata-puññatā\\
Atta-sammā-paṇidhi ca\\
Etam maṅgalam-uttamaṁ

Bāhu-saccañ-ca sippañ-ca,\\
Vinayo ca susikkhito\\
Subhāsitā ca yā vācā\\
Etam maṅgalam-uttamaṁ

Mātā-pitu-upaṭṭhānaṁ\\
Putta-dārassa saṅgaho\\
Anākulā ca kammantā\\
Etam maṅgalam-uttamaṁ

Dānañ-ca dhamma-cariyā ca\\
Ñātakānañ-ca saṅgaho\\
Anavajjāni kammāni\\
Etam maṅgalam-uttamaṁ

Āratī viratī pāpā\\
Majja-pānā ca saññamo\\
Appamādo ca dhammesu\\
Etam maṅgalam-uttamaṁ

Gāravo ca nivāto ca\\
Santuṭṭhī ca kataññutā\\
Kālena dhammassavanaṁ\\
Etam maṅgalam-uttamaṁ

Khantī ca sovacassatā\\
Samaṇānañ-ca dassanaṁ\\
Kālena dhamma-sākacchā\\
Etam maṅgalam-uttamaṁ

Tapo ca brahma-cariyañ-ca\\
Ariya-saccāna-dassanaṁ\\
Nibbāna-sacchikiriyā ca\\
Etam maṅgalam-uttamaṁ

Phuṭṭhassa loka-dhammehi\\
Cittaṁ yassa na kampati\\
Asokaṁ virajaṁ khemaṁ\\
Etam maṅgalam-uttamaṁ

Etādisāni katvāna\\
Sabbattham-aparājitā\\
Sabbattha sotthiṁ gacchanti\\
Tan-tesaṁ maṅgalam-uttaman'ti

% NOTE: References are commented out, but keep them in the source for information.
%\suttaref{(Sn.vv.258-269; Khp.V)}

\end{paritta}

\clearpage

\chapter{Karaṇīya-metta-sutta}

\firstline{Karaṇīyam-attha-kusalena}

\begin{paritta}
Karaṇīyam-attha-kusalena\\
Yan-taṁ santaṁ padaṁ abhisamecca\\
Sakko ujū ca suhujū ca\\
Suvaco c'assa mudu anatimānī

Santussako ca subharo ca\\
Appakicco ca sallahuka-vutti\\
Sant'indriyo ca nipako ca\\
Appagabbho kulesu ananugiddho

Na ca khuddaṁ samācare kiñci\\
Yena viññū pare upavadeyyuṁ\\
Sukhino vā khemino hontu\\
Sabbe sattā bhavantu sukhit'attā

Ye keci pāṇa-bhūt'atthi\\
Tasā vā thāvarā vā anavasesā\\
Dīghā vā ye mahantā vā\\
Majjhimā rassakā aṇuka-thūlā

Diṭṭhā vā ye ca adiṭṭhā\\
Ye ca dūre vasanti avidūre\\
Bhūtā vā sambhavesī vā\\
Sabbe sattā bhavantu sukhit'attā

Na paro paraṁ nikubbetha\\
Nātimaññetha katthaci naṁ kiñci\\
Byārosanā paṭighasaññā\\
Nāññam-aññassa dukkham-iccheyya

Mātā yathā niyaṁ puttaṁ\\
Āyusā eka-puttam-anurakkhe\\
Evam'pi sabba-bhūtesu\\
Mānasam-bhāvaye aparimāṇaṁ

Mettañ-ca sabba-lokasmiṁ\\
Mānasam-bhāvaye aparimāṇaṁ\\
Uddhaṁ adho ca tiriyañ-ca\\
Asambādhaṁ averaṁ asapattaṁ

Tiṭṭhañ-caraṁ nisinno vā\\
Sayāno vā yāvat'assa vigata-middho\\
Etaṁ satiṁ adhiṭṭheyya\\
Brahmam-etaṁ vihāraṁ idham-āhu

Diṭṭhiñca anupagamma\\
Sīlavā dassanena sampanno\\
Kāmesu vineyya gedhaṁ\\
Na hi jātu gabbha-seyyaṁ punaretī'ti

% NOTE: References are commented out, but keep them in the source for information.
%\suttaref{(Sn.vv.143-152; Khp.IX)}

\end{paritta}

\chapter{Yaṅkiñci ratanaṁ loke}

\firstline{Yaṅkiñci ratanaṁ loke}

\begin{twochants}
Yaṅkiñci ratanaṁ loke & vijjati vividhaṁ puthu\\
Ratanaṁ buddha-samaṁ n'atthi & tasmā sotthī bhavantu te/me\\
Yaṅkiñci ratanaṁ loke & vijjati vividhaṁ puthu\\
Ratanaṁ dhamma-samaṁ n'atthi & tasmā sotthī bhavantu te/me\\
Yaṅkiñci ratanaṁ loke & vijjati vividhaṁ puthu\\
Ratanaṁ saṅgha-samaṁ n'atthi & tasmā sotthī bhavantu te/me\\
\end{twochants}

\chapter{Sambuddhe}

\firstline{Sambuddhe aṭṭhavīsañca}

\begin{twochants}
Sambuddhe aṭṭhavīsañca & dvādasañca sahassake\\
Pañca-sata-sahassāni & namāmi sirasā ahaṁ\\
Tesaṁ dhammañca saṅghañca & ādarena namāmihaṁ\\
Namakārānubhāvena & hantvā sabbe upaddave\\
Anekā antarāyāpi & vinassantu asesato\\
Sambuddhe pañca-paññāsañca & catuvīsati sahassake\\
Dasa-sata-sahassāni & namāmi sirasā ahaṁ\\
Tesaṁ dhammañca saṅghañca & ādarena namāmihaṁ\\
Namakārānubhāvena & hantvā sabbe upaddave\\
Anekā antarāyāpi & vinassantu asesato\\
Sambuddhe navuttarasate & aṭṭhacattāḷīsa sahassake\\
Vīsati-sata-sahassāni & namāmi sirasā ahaṁ\\
Tesaṁ dhammañca saṅghañca & ādarena namāmihaṁ\\
Namakārānubhāvena & hantvā sabbe upaddave\\
Anekā antarāyāpi & vinassantu asesato\\
\end{twochants}

\clearpage

\chapter{Khandha-parittaṁ}

\firstline{Virūpakkhehi me mettaṁ}

% TODO: Virūp_a_kkhehi correct?

\begin{twochants}
Virūpakkhehi me mettaṁ & mettaṁ erāpathehi me\\
Chabyā-puttehi me mettaṁ & mettaṁ kaṇhā-gotamakehi ca\\
Apādakehi me mettaṁ & mettaṁ dipādakehi me\\
Catuppadehi me mettaṁ & mettaṁ bahuppadehi me\\
Mā maṁ apādako hiṁsi & mā maṁ hiṁsi dipādako\\
Mā maṁ catuppado hiṁsi & mā maṁ hiṁsi bahuppado\\
Sabbe sattā sabbe pāṇā & sabbe bhūtā ca kevalā\\
Sabbe bhadrāni passantu & mā kiñci pāpam-āgamā\\
Appamāṇo buddho & appamāṇo dhammo\\
Appamāṇo saṅgho & pamāṇavantāni siriṁsapāni\\
Ahi-vicchikā sata-padī & uṇṇā-nābhī sarabhū mūsikā\\
Katā me rakkhā katā me parittā & paṭikkamantu bhūtāni\\
So'haṁ namo bhagavato & namo sattannaṁ\\
Sammā-sambuddhānaṁ & \\
\end{twochants}

% NOTE: References are commented out, but keep them in the source for information.
%\suttaref{(A.II.72-73; Vin.II.110; J.144)}

\clearpage

\chapter{Buddha-dhamma-saṅgha-guṇā}

\firstline{Iti pi so bhagavā}

\enlargethispage{\baselineskip}

\begin{paritta}
Iti pi so bhagavā arahaṁ sammā-sambuddho\\
Vijjā-caraṇa-sampanno sugato loka-vidū\\
Anuttaro purisa-damma-sārathi\\
Satthā devamanussānaṁ buddho bhagavā'ti

Svākkhāto bhagavatā dhammo sandiṭṭhiko akāliko ehi-passiko\\
Opanayiko paccattaṁ veditabbo viññūhī'ti

Supaṭipanno bhagavato sāvaka-saṅgho\\
Uju-paṭipanno bhagavato sāvaka-saṅgho\\
Ñāya-paṭipanno bhagavato sāvaka-saṅgho\\
Sāmīci-paṭipanno bhagavato sāvaka-saṅgho\\
Yad-idaṁ cattāri purisa-yugāni aṭṭha purisa-puggalā\\
Esa bhagavato sāvaka-saṅgho\\
Āhuneyyo pāhuneyyo dakkhiṇeyyo añjali-karaṇīyo\\
Anuttaraṁ puññakkhettaṁ lokassā'ti
\end{paritta}

\clearpage

\chapter{Mora-parittaṁ}

\firstline{Udet'ayañ-cakkhumā eka-rājā}

\begin{paritta}
Udet'ayañ-cakkhumā eka-rājā\\
Harissa-vaṇṇo paṭhavippabhāso\\
Taṁ taṁ namassāmi\\
Harissa-vaṇṇaṁ paṭhavippabhāsaṁ\\
Tay'ajja guttā viharemu divasaṁ\\
Ye brāhmaṇā vedagu sabba-dhamme

\begin{twochants}
Te me namo & te ca maṁ pālayantu\\
Nam'atthu buddhānaṁ & nam'atthu bodhiyā\\
Namo vimuttānaṁ & namo vimuttiyā\\
Imaṁ so parittaṁ katvā & moro carati esanā'ti\\
\end{twochants}

Apet'ayañ-cakkhumā eka-rājā\\
Harissa-vaṇṇo paṭhavippabhāso\\
Taṁ taṁ namassāmi\\
Harissa-vaṇṇaṁ paṭhavippabhāsaṁ\\
Tay'ajja guttā viharemu rattiṁ\\
Ye brāhmaṇā vedagu sabba-dhamme

\begin{twochants}
Te me namo & te ca maṁ pālayantu\\
Nam'atthu buddhānaṁ & nam'atthu bodhiyā\\
Namo vimuttānaṁ & namo vimuttiyā\\
Imaṁ so parittaṁ katvā & moro vāsam-akappayī'ti
\end{twochants}

% NOTE: References are commented out, but keep them in the source for information.
%\suttaref{(J.159)}

\end{paritta}

\chapter{Vaṭṭaka-parittaṁ}

\firstline{Atthi loke sīla-guṇo}

\begin{twochants}
Atthi loke sīla-guṇo & saccaṁ soceyy'anuddayā\\
Tena saccena kāhāmi & sacca-kiriyam-anuttaraṁ\\
Āvajjitvā dhamma-balaṁ & saritvā pubbake jine\\
Sacca-balam-avassāya & sacca-kiriyam-akās'ahaṁ\\
Santi pakkhā apattanā & santi pādā avañcanā\\
Mātā pitā ca nikkhantā & jāta-veda paṭikkama\\
Saha sacce kate mayhaṁ & mahā-pajjalito sikhī\\
Vajjesi soḷasa karīsāni & udakaṁ patvā yathā sikhī\\
Saccena me samo n'atthi & esā me sacca-pāramī'ti\\
\end{twochants}

% NOTE: References are commented out, but keep them in the source for information.
%\suttaref{(Cariyapiṭaka vv.319-322)}

\chapter{Abhaya-parittaṁ}

\firstline{Yan-dunnimittaṁ avamaṅgalañ-ca}

\enlargethispage{\baselineskip}

\begin{paritta}
Yan-dunnimittaṁ avamaṅgalañ-ca\\
Yo cāmanāpo sakuṇassa saddo\\
Pāpaggaho dussupinaṁ akantaṁ\\
Buddhānubhāvena vināsamentu

Yan-dunnimittaṁ avamaṅgalañ-ca\\
Yo cāmanāpo sakuṇassa saddo\\
Pāpaggaho dussupinaṁ akantaṁ\\
Dhammānubhāvena vināsamentu

Yan-dunnimittaṁ avamaṅgalañ-ca\\
Yo cāmanāpo sakuṇassa saddo\\
Pāpaggaho dussupinaṁ akantaṁ\\
Saṅghānubhāvena vināsamentu
\end{paritta}

\clearpage

\chapter{Āṭānāṭiya-parittaṁ}

\firstline{Vipassissa nam'atthu}

\begin{twochants}
Vipassissa nam'atthu & cakkhumantassa sirīmato\\
Sikhissa pi nam'atthu & sabba-bhūtānukampino\\
Vessabhussa nam'atthu & nhātakassa tapassino\\
Nam'atthu kakusandhassa & māra-senappamaddino\\
Konāgamanassa nam'atthu & brāhmaṇassa vusīmato\\
Kassapassa nam'atthu & vippamuttassa sabbadhi\\
Aṅgīrasassa nam'atthu & sakya-puttassa sirīmato\\
Yo imaṁ dhammam-adesesi & sabba-dukkhāpanūdanaṁ\\
Ye cāpi nibbutā loke & yathā-bhūtaṁ vipassisuṁ\\
Te janā apisuṇā & mahantā vīta-sāradā\\
Hitaṁ deva-manussānaṁ & yaṁ namassanti gotamaṁ\\
Vijjā-caraṇa-sampannaṁ & mahantaṁ vīta-sāradaṁ\\
Vijjā-caraṇa-sampannaṁ & buddhaṁ vandāma gotaman'ti\\
\end{twochants}

\chapter{Aṅguli-māla-parittaṁ}

\firstline{Yato'haṁ bhagini ariyāya jātiyā jāto}

\begin{paritta}
Yato'haṁ bhagini ariyāya jātiyā jāto\\
Nābhijānāmi sañcicca pāṇaṁ jīvitā voropetā\\
Tena saccena sotthi te hotu sotthi gabbhassa

\instr{Tres veces}

% NOTE: References are commented out, but keep them in the source for information.
%\suttaref{(M.II.103)}

\end{paritta}

\clearpage

\chapter{N'atthi me saraṇaṁ aññaṁ}

\firstline{N'atthi me saraṇaṁ aññaṁ}

\begin{paritta}
N'atthi me saraṇaṁ aññaṁ\\
Buddho me saraṇaṁ varaṁ\\
Etena sacca-vajjena\\
Hotu te jaya-maṅgalaṁ\\
N'atthi me saraṇaṁ aññaṁ\\
Dhammo me saraṇaṁ varaṁ\\
Etena sacca-vajjena\\
Hotu te jaya-maṅgalaṁ\\
N'atthi me saraṇaṁ aññaṁ\\
Saṅgho me saraṇaṁ varaṁ\\
Etena sacca-vajjena\\
Hotu te jaya-maṅgalaṁ
\end{paritta}

\chapter{Sakkatvā}

\firstline{Sakkatvā buddha-ratanaṁ}

\begin{twochants}
Sakkatvā buddha-ratanaṁ & osathaṁ uttamaṁ varaṁ\\
Hitaṁ deva-manussānaṁ & buddha-tejena sotthinā\\
Nassant'upaddavā sabbe & dukkhā vūpasamentu te/me\\
Sakkatvā dhamma-ratanaṁ & osathaṁ uttamaṁ varaṁ\\
Pariḷāhūpasamanaṁ & dhamma-tejena sotthinā\\
Nassant'upaddavā sabbe & bhayā vūpasamentu te/me\\
Sakkatvā saṅgha-ratanaṁ & osathaṁ uttamaṁ varaṁ\\
Āhuneyyaṁ pāhuneyyaṁ & saṅgha-tejena sotthinā\\
Nassant'upaddavā sabbe & rogā vūpasamentu te/me
\end{twochants}

\chapter{Bojjh'aṅga-parittaṁ}

\firstline{Bojjh'aṅgo sati-saṅkhāto}

\begin{twochants}
Bojjh'aṅgo sati-saṅkhāto & dhammānaṁ vicayo tathā\\
Viriyam-pīti-passaddhi & bojjh'aṅgā ca tathā'pare\\
Samādh'upekkha-bojjh'aṅgā & satt'ete sabba-dassinā\\
Muninā sammad-akkhātā & bhāvitā bahulīkatā\\
Saṁvattanti abhiññāya & nibbānāya ca bodhiyā\\
Etena sacca-vajjena & sotthi te hotu sabbadā\\
Ekasmiṁ samaye nātho & moggallānañ-ca kassapaṁ\\
Gilāne dukkhite disvā & bojjh'aṅge satta desayi\\
Te ca taṁ abhinanditvā & rogā mucciṁsu taṅ-khaṇe\\
Etena sacca-vajjena & sotthi te hotu sabbadā\\
Ekadā dhamma-rājā pi & gelaññenābhipīḷito\\
Cundattherena tañ-ñeva & bhaṇāpetvāna sādaraṁ\\
Sammoditvā ca ābādhā & tamhā vuṭṭhāsi ṭhānaso\\
Etena sacca-vajjena & sotthi te hotu sabbadā\\
Pahīnā te ca ābādhā & tiṇṇannam-pi mahesinaṁ\\
Magg'āhata-kilesā va & pattānuppatti-dhammataṁ\\
Etena sacca-vajjena & sotthi te hotu sabbadā\\
\end{twochants}

% NOTE: References are commented out, but keep them in the source for information.
%\suttaref{(cf. S.V.80f)}

\clearpage

\chapter{Jaya-maṅgala-aṭṭha-gāthā}

\firstline{Bāhuṁ sahassam-abhinimmita sāvudhan-taṁ}

\begin{paritta}
Bāhuṁ sahassam-abhinimmita sāvudhan-taṁ\\
Grīmekhalaṁ udita-ghora-sasena-māraṁ\\
Dān'ādi-dhamma-vidhinā jitavā mun'indo\\
Tan-tejasā bhavatu te jaya-maṅgalāni

Mārātirekam-abhiyujjhita-sabba-rattiṁ\\
Ghoram-pan'āḷavakam-akkhama-thaddha-yakkhaṁ\\
Khantī-sudanta-vidhinā jitavā mun'indo\\
Tan-tejasā bhavatu te jaya-maṅgalāni

Nāḷāgiriṁ gaja-varaṁ atimatta-bhūtaṁ\\
Dāv'aggi-cakkam-asanīva sudāruṇan-taṁ\\
Mett'ambu-seka-vidhinā jitavā mun'indo\\
Tan-tejasā bhavatu te jaya-maṅgalāni

Ukkhitta-khaggam-atihattha-sudāruṇan-taṁ\\
Dhāvan-ti-yojana-path'aṅguli-mālavantaṁ\\
Iddhī'bhisaṅkhata-mano jitavā mun'indo\\
Tan-tejasā bhavatu te jaya-maṅgalāni

Katvāna kaṭṭham-udaraṁ iva gabbhinīyā\\
Ciñcāya duṭṭha-vacanaṁ jana-kāya majjhe\\
Santena soma-vidhinā jitavā mun'indo\\
Tan-tejasā bhavatu te jaya-maṅgalāni

Saccaṁ vihāya-mati-saccaka-vāda-ketuṁ\\
Vādābhiropita-manaṁ ati-andha-bhūtaṁ\\
Paññā-padīpa-jalito jitavā mun'indo\\
Tan-tejasā bhavatu te jaya-maṅgalāni

Nandopananda-bhujagaṁ vibudhaṁ mah'iddhiṁ\\
Puttena thera-bhujagena damāpayanto\\
Iddhūpadesa-vidhinā jitavā mun'indo\\
Tan-tejasā bhavatu te jaya-maṅgalāni

Duggāha-diṭṭhi-bhujagena sudaṭṭha-hatthaṁ\\
Brahmaṁ visuddhi-jutim-iddhi-bakābhidhānaṁ\\
Ñāṇāgadena vidhinā jitavā mun'indo\\
Tan-tejasā bhavatu te jaya-maṅgalāni

Etā pi buddha-jaya-maṅgala-aṭṭha-gāthā\\
Yo vācano dina-dine saratem-atandī\\
Hitvān'aneka-vividhāni c'upaddavāni\\
Mokkhaṁ sukhaṁ adhigameyya naro sapañño
\end{paritta}

\chapter{Devatā-uyyojana-gāthā}

\firstline{Dukkhappattā ca niddukkhā}

\begin{twochants}
Dukkhappattā ca niddukkhā & bhayappattā ca nibbhayā\\
Sokappattā ca nissokā & hontu sabbe pi pāṇino\\
Ettāvatā ca amhehi & sambhataṁ puñña-sampadaṁ\\
Sabbe devānumodantu & sabba-sampatti-siddhiyā\\
Dānaṁ dadantu saddhāya & sīlaṁ rakkhantu sabbadā\\
Bhāvanābhiratā hontu & gacchantu devatā-gatā\\\relax
[Sabbe buddhā] balappattā & paccekānañ-ca yaṁ balaṁ\\
Arahantānañ-ca tejena & rakkhaṁ bandhāmi sabbaso\\
\end{twochants}

% NOTE: References are commented out, but keep them in the source for information.
%\suttaref{(MJG)}

\clearpage

\chapter{Jaya-parittaṁ}

\firstline{Mahā-kāruṇiko nātho}

\begin{paritta}

Mahā-kāruṇiko nātho\\
Hitāya sabba-pāṇinaṁ\\
Pūretvā pāramī sabbā\\
Patto sambodhim-uttamaṁ\\
Etena sacca-vajjena\\
Hotu te jaya-maṅgalaṁ\\
Jayanto bodhiyā mūle\\
Sakyānaṁ nandi-vaḍḍhano\\
Evaṁ tvaṁ vijayo hohi\\
Jayassu jaya-maṅgale\\
Aparājita-pallaṅke\\
Sīse paṭhavi-pokkhare\\
Abhiseke sabba-buddhānaṁ\\
Aggappatto pamodati\\
Sunakkhattaṁ sumaṅgalaṁ\\
Supabhātaṁ suhuṭṭhitaṁ\\
Sukhaṇo sumuhutto ca\\
Suyiṭṭhaṁ brahma-cārisu\\
Padakkhiṇaṁ kāya-kammaṁ\\
Vācā-kammaṁ padakkhiṇaṁ\\
Padakkhiṇaṁ mano-kammaṁ\\
Paṇidhi te padakkhiṇā\\
Padakkhiṇāni katvāna\\
Labhant'atthe padakkhiṇe

% NOTE: References are commented out, but keep them in the source for information.
%\suttaref{(MJG; A.I.294)}

\end{paritta}

\clearpage

\chapter{Culla-maṅgala-cakka-vāḷa}

\firstline{Sabba-buddh'ānubhāvena}

Sabba-buddh'ānubhāvena, sabba-dhamm'ānubhāvena sabba-saṅgh'ānubhāvena

Buddha-ratanaṁ dhamma-ratanaṁ saṅgha-ratanaṁ

Tiṇṇaṁ ratanānaṁ ānubhāvena,
catur-āsīti-sahassa-dhamma-\\ kkhandh'ānubhāvena,
piṭakattay'ānubhāvena,
jina-sāvak'ānubhāvena

Sabbe te rogā sabbe te bhayā sabbe te antarāyā sabbe te upaddavā sabbe te
dunnimittā sabbe te avamaṅgalā vinassantu

Āyu-vaḍḍhako dhana-vaḍḍhako siri-vaḍḍhako yasa-vaḍḍhako bala-vaḍḍhako
vaṇṇa-vaḍḍhako sukha-vaḍḍhako hotu sabbadā

Dukkha-roga-bhayā verā sokā sattu c'upaddavā, anekā antarāyā pi vinassantu ca tejasā

Jaya-siddhi dhanaṁ lābhaṁ, sotthi bhāgyaṁ sukhaṁ balaṁ

Siri āyu ca vaṇṇo ca bhogaṁ vuḍḍhī ca yasavā\\
Sata-vassā ca āyū ca jīva-siddhī bhavantu te

Bhavatu sabba-maṅgalaṁ\ldots{}

\enlargethispage{\baselineskip}

\chapter{Bhavatu sabba-maṅgalaṁ}

\firstline{Bhavatu sabba-maṅgalaṁ}

\begin{twochants}
Bhavatu sabba-maṅgalaṁ & rakkhantu sabba-devatā\\
Sabba-buddhānubhāvena & sadā sotthī bhavantu te/me\\
Bhavatu sabba-maṅgalaṁ & rakkhantu sabba-devatā\\
Sabba-dhammānubhāvena & sadā sotthī bhavantu te/me\\
Bhavatu sabba-maṅgalaṁ & rakkhantu sabba-devatā\\
Sabba-saṅghānubhāvena & sadā sotthī bhavantu te/me\\
\end{twochants}

% NOTE: References are commented out, but keep them in the source for information.
%\suttaref{(MJG)}

%\cleartoverso

\addtocontents{toc}{%
  \protect \enlargethispage{2\baselineskip}
}

\chapterTocDelegatePageNumber
\chapter{La Protección de los veintiocho Buddhas}

\setTocDelegatedPageNumber

\vspace*{5pt}

{\setlength{\parskip}{0pt}%
\soloinstr{Introducción}

\begin{soloonechants}
	Vamos a recitar el discurso dado por el gran héroe,\\ como protección para los seres humanos que aprecian la virtud,\\ contra el daño de seres malvados no humanos \\que no están contentos con las enseñanzas de Buddha.
\end{soloonechants}%
}

\vspace*{-2pt}

\setEnglishTextSize{12}{21}{\parskip}
\englishText

\begin{onechants}
Homenaje a todos los Buddhas, poderosos que han aparecido:\\
Taṇhaṅkaro, el gran héroe, Medhaṅkaro, de gran fama,\\
Saraṇaṅkaro, guardián del mundo, Dīpaṅkaro, el que trae la luz,\\
Koṇḍañño, libertador de la gente, Maṅgalo, líder de humanos,\\
Sumano, amable y sabio, Revato, el que eleva la alegría,\\
Sobhito, perfecto en virtudes, Anomadassī, el más grande entre el pueblo,\\
Padumo, iluminador del mundo, Nārado, auriga supremo,\\
Padumuttaro, la esencia del ser, Sumedho, aquel sin igual,\\
Sujāto, el pico del mundo,  Piyadassī, el líder de los hombres,\\
Atthadassī, el compasivo, Dhammadassī, destructor de la oscuridad,\\
Siddhattho, sin igual en el mundo, y Tisso, excelente Orador,\\
Phusso, un Buddha dador de excelencia, Vipassī, el incomparable,\\
Sikhī, maestro de bendiciones, Vessabhū, el que da felicidad,\\
Kakusandho, líder de la caravana, Koṇāgamano, que abandona el mal,\\
Kassapo, perfecto en gloria, Gotamo, jefe de los Sakyan.\\
\end{onechants}

\clearpage

\chapterTocSubIndentTrue
\chapter{Āṭānāṭiya Paritta (versión extendida)}

\firstline{Namo me sabbabuddhānaṁ}

\paliText

\begin{leader}
\soloinstr{Solo introduction}

\begin{solotwochants}
Appasannehi nāthassa & sāsane sādhusammate\\
Amanussehi caṇḍehi & sadā kibbisakāribhi\\
Parisānañca-tassannam & ahiṁsāya ca guttiyā\\
Yandesesi mahāvīro & parittan-tam bhaṇāma se.\\
\end{solotwochants}
\end{leader}

\begin{twochants}
[Namo me sabbabuddhānaṁ] & uppannānaṁ mahesinaṁ\\
Taṇhaṅkaro mahāvīro & medhaṅkaro mahāyaso\\
Saraṇaṅkaro lokahito & dīpaṅkaro jutindharo\\
Koṇḍañño janapāmokkho & maṅgalo purisāsabho\\
Sumano sumano dhīro & revato rativaḍḍhano\\
Sobhito guṇasampanno & anomadassī januttamo\\
Padumo lokapajjoto & nārado varasārathī\\
Padumuttaro sattasāro & sumedho appaṭipuggalo\\
Sujāto sabbalokaggo & piyadassī narāsabho\\
Atthadassī kāruṇiko & dhammadassī tamonudo\\
Siddhattho asamo loke & tisso ca vadataṁ varo\\
Phusso ca varado buddho & vipassī ca anūpamo\\
Sikhī sabbahito satthā & vessabhū sukhadāyako\\
Kakusandho satthavāho & koṇāgamano raṇañjaho\\
Kassapo sirisampanno & gotamo sakyapuṅgavo\\
\end{twochants}

\clearpage

\englishText

\begin{onechants}
Estos y otros Buddhas, varios cientos de millones\\
Todos los Buddhas, sin igual, todos los Buddhas de gran poder,\\
Dotados con los Diez Poderes, alcanzaron el conocimiento supremo,\\
A todos ellos se les concede el liderazgo supremo.\\
Rugen con confianza el ‘rugido del león’ entre sus seguidores,\\
observan con el ojo divino, sin obstáculos, el mundo entero.\\
Los líderes dotados de los dieciocho tipos de Buddha-Dhamma,\\
y las treinta y dos marcas mayores y ochenta menores de un gran ser,\\
Brillando con halos de una braza de ancho, todos estos grandes sabios,\\
todos estos Buddhas omniscientes, conquistadores libres de corrupción,\\
de gran brillo, gran poder, de gran sabiduría, gran fuerza,\\
sabios de gran compasión, brindando dicha a todos.\\
Islas, guardianes y apoyos, refugios y cuevas para todos los seres,\\
destino, parientes y consoladores, benevolentes dadores de refugio.\\
Todos ellos son el último lugar de descanso para el mundo con sus dioses.\\
Con mi cabeza a sus pies, saludo a estos grandes seres humanos.\\
De palabra y pensamiento, venero a esos Tathāgatas,\\
ya sea acostado, sentado, de pie o caminando por cualquier lugar.\\
Que ellos, los Buddhas, portadores de paz, protejan siempre tu felicidad,\\
y que tú, protegido por ellos, en paz, libre de temor,\\
libre de toda enfermedad, a salvo de todo tormento\\
habiendo conquistado el odio, que obtengas la extinción.\\

\end{onechants}

\clearpage

\paliText

\begin{twochants}
Ete caññe ca sambuddhā & anekasatakoṭayo\\
Sabbe buddhā asamasamā & sabbe buddhā mahiddhikā\\
Sabbe dasabalūpetā & vesārajjehupāgatā\\
Sabbe te paṭijānanti & āsabhaṇṭhānamuttamaṁ\\
Sīhanādaṁ nadantete & parisāsu visāradā\\
Brahmacakkaṁ pavattenti & loke appaṭivattiyaṁ\\
Upetā buddhadhammehi & aṭṭhārasahi nāyakā\\
Dvattiṁsa-lakkhaṇūpetā & sītyānubyañjanādharā\\
Byāmappabhāya suppabhā & sabbe te munikuñjarā\\
Buddhā sabbaññuno ete & sabbe khīṇāsavā jinā\\
Mahappabhā mahātejā & mahāpaññā mahabbalā\\
Mahākāruṇikā dhīrā & sabbesānaṁ sukhāvahā\\
Dīpā nāthā patiṭṭhā & ca tāṇā leṇā ca pāṇinaṁ\\
Gatī bandhū mahassāsā & saraṇā ca hitesino\\
Sadevakassa lokassa & sabbe ete parāyanā\\
Tesāhaṁ sirasā pāde & vandāmi purisuttame\\
Vacasā manasā ceva & vandāmete tathāgate\\
Sayane āsane ṭhāne & gamane cāpi sabbadā\\
Sadā sukhena rakkhantu & buddhā santikarā tuvaṁ\\
Tehi tvaṁ rakkhito santo & mutto sabbabhayena ca\\
Sabba-rogavinimutto & sabba-santāpavajjito\\
Sabba-veramatikkanto & nibbuto ca tuvaṁ bhava\\
\end{twochants}

\clearpage

\englishText

\begin{onechants}
Por el poder de su verdad, su virtud y su amor,\\
 que te protejan y te guarden con salud y felicidad.\\
En la dirección del Este hay seres de gran poder,\\
 que te protejan y te guarden con salud y felicidad.\\
En la dirección del Sur hay devas de gran poder, \\
 que te protejan y te guarden con salud y felicidad.\\
En la dirección del Oeste hay nāgas de gran     poder,\\
que te protejan y te guarden con salud y felicidad.\\
En la dirección del Norte hay yakkhas de gran poder,\\
 que te protejan y te guarden con salud y felicidad.\\
En el Este está Dhataraṭṭho, en el Sur está Viruḷhako,\\
en el Oeste está Virūpakkho, Kuvero gobierna en el Norte.\\
Estos Cuatro Poderosos Reyes, afamados guardianes del mundo,\\
 que todos sean tus protectores con salud y felicidad.\\
Dioses que habitan en el cielo y en la tierra y nāgas de gran poder,\\
que todos ellos sean tus protectores en salud y felicidad.\\
Para mí no hay otro refugio, el Buddha es mi excelente refugio:\\
por esta declaración de verdad, que bendiciones de victoria sean tuyas.\\
Para mí no hay otro refugio, el Dhamma es mi excelente refugio:\\
por esta declaración de verdad, que bendiciones de victoria sean tuyas.\\
Para mí no hay otro refugio, la Saṅgha es mi excelente refugio:\\
por esta declaración de verdad, que bendiciones de victoria sean tuyas.\\

\end{onechants}

\clearpage

\paliText
\enlargethispage{2\baselineskip}
\savenotes

\begin{twochants}
Tesaṁ saccena sīlena & khantimettābalena ca\\
Tepi tumhe%
\footnote{Si se canta para uno mismo, cambiar \textit{tumhe} por \textit{amhe} aquí y en las lineas siguientes.}
anurakkhantu & ārogyena sukhena ca\\
Puratthimasmiṁ disābhāge & santi bhūtā mahiddhikā\\
Tepi tumhe anurakkhantu & ārogyena sukhena ca\\
Dakkhiṇasmiṁ disābhāge & santi devā mahiddhikā\\
Tepi tumhe anurakkhantu & ārogyena sukhena ca\\
Pacchimasmiṁ disābhāge & santi nāgā mahiddhikā\\
Tepi tumhe anurakkhantu & ārogyena sukhena ca\\
Uttarasmiṁ disābhāge & santi yakkhā mahiddhikā\\
Tepi tumhe anurakkhantu & ārogyena sukhena ca\\
Purimadisaṁ dhataraṭṭho & dakkhiṇena viruḷhako\\
Pacchimena virūpakkho & kuvero uttaraṁ disaṁ\\
Cattāro te mahārājā & lokapālā yasassino\\
Tepi tumhe anurakkhantu & ārogyena sukhena ca\\
Ākāsaṭṭhā ca bhummaṭṭhā & devā nāgā mahiddhikā\\
Tepi tumhe anurakkhantu & ārogyena sukhena ca\\
Natthi me saraṇaṁ aññaṁ & buddho me saraṇaṁ varaṁ\\
Etena saccavajjena & hotu te%
\footnote{Si se canta para uno mismo, cambiar \textit{te} por \textit{me} aquí y en las lineas siguientes.}
jayamaṅgalaṁ\\
Natthi me saraṇaṁ aññaṁ & dhammo me saraṇaṁ varaṁ\\
Etena saccavajjena & hotu te jayamaṅgalaṁ\\
Natthi me saraṇaṁ aññaṁ & saṅgho me saraṇaṁ varaṁ\\
Etena saccavajjena & hotu te jayamaṅgalaṁ\\
\end{twochants}

\spewnotes

\clearpage

\englishText

\begin{onechants}
Cualquier joya que hay en el mundo, por espléndida que pueda ser, \\no hay joya igual al Buddha. Por el poder de esta verdad, bendito seas.\\
Cualquier joya que hay en el mundo, por espléndida que pueda ser, \\no hay joya igual al Dhamma. Por el poder de esta verdad, bendito seas.\\
Cualquier joya que hay en el mundo, por espléndida que pueda ser, \\no hay joya igual a la Saṅgha. Por el poder de esta verdad, bendito seas.\\
Honrando a la joya del Buddha, la suprema y excelente medicina,\\ que beneficia a los dioses y a los humanos. Por el poder y bendición del Buddha,\\ que todos los peligros sean evitados, y tus penas desaparezcan.\\
Honrando a la joya del Dhamma, la suprema y excelente medicina,\\ que calma toda fiebre. Por el poder y bendición del Dhamma,\\ que todos los peligros sean evitados, y tus miedos desaparezcan.\\
Honrando a la joya de la Sangha, la suprema y excelente medicina,\\
digna de presentes y hospitalidad. Por el poder y bendición de la Saṅgha,\\ que todos los peligros sean evitados, y tus enfermedades desaparezcan.\\
Que se eviten todas las calamidades, que pase toda enfermedad,\\
que ningún peligro te amenace, que seas feliz y tengas larga vida,\\
seas recibido con cariño y bienvenido en todas partes.\\
Que cuatro cosas te sean concedidas: larga vida, belleza, felicidad y fuerza.\\
\end{onechants}

\clearpage

\paliText

\begin{twochants}
Yaṅkiñci ratanaṁ loke & vijjati vividhaṁ puthu\\
Ratanaṁ buddhasamaṁ & natthi tasmā sotthī bhavantu te\\
Yaṅkiñci ratanaṁ loke & vijjati vividhaṁ puthu\\
Ratanaṁ dhammasamaṁ & natthi tasmā sotthī bhavantu te\\
Yaṅkiñci ratanaṁ loke & vijjati vividhaṁ puthu\\
Ratanaṁ saṅghasamaṁ & natthi tasmā sotthī bhavantu te\\
Sakkatvā buddharatanaṁ & osathaṁ uttamaṁ varaṁ\\
Hitaṁ devamanussānaṁ & buddhatejena sotthinā\\
Nassantupaddavā sabbe & dukkhā vūpasamentu te\\
Sakkatvā dhammaratanaṁ & osathaṁ uttamaṁ varaṁ\\
Pariḷāhūpasamanaṁ & dhammatejena sotthinā\\
Nassantupaddavā sabbe & bhayā vūpasamentu te\\
Sakkatvā saṅgharatanaṁ & osathaṁ uttamaṁ varaṁ\\
Āhuneyyaṁ pāhuneyyaṁ & saṅghatejena sotthinā\\
Nassantupaddavā sabbe & rogā vūpasamentu te\\
Sabbītiyo vivajjantu & sabbarogo vinassatu\\
Mā te bhavatvantarāyo & sukhī dīghāyuko bhava\\
Abhivādanasīlissa & niccaṁ vuḍḍhāpacāyino\\
Cattāro dhammā vaḍḍhanti & āyu vaṇṇo sukhaṁ balaṁ\\
\end{twochants}

\setEnglishTextSize{12}{18}{9pt}
\resumeNormalText

% End of parittas.tex
