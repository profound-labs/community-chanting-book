\chapter*{Dedicación de Ofrendas}

\delegateSetUseNext

[Yo so] bha꜕gavā a꜕rahaṁ sammāsambuddho\\
Svākkhā꜓to yena bha꜕gava꜓tā dhammo\\
Supaṭi꜕panno yassa bha꜕gava꜕to sāvaka꜕saṅgho\\
Tam-ma꜓yaṁ bha꜕gavantaṁ sa꜕dhammaṁ sa꜕saṅghaṁ\\
Imehi꜓ sakkārehi꜕ yathārahaṁ āropi꜕tehi a꜕bhi꜓pūja꜕yāma\\
Sādhu꜓ no bhante bha꜕gavā su꜕cira-parinibbu꜕topi\\
Pacchi꜓mā-ja꜕na꜓tānu꜓kampa꜕-mānasā\\
Ime sakkāre dugga꜕ta꜕-paṇṇākāra꜓-bhūte pa꜕ṭiggaṇhātu\\
Amhā꜓kaṁ dīgha꜕rattaṁ hi꜕tāya su꜕khāya

Arahaṁ sammāsambuddho bha꜕gavā\\
Buddhaṁ bha꜕gavantaṁ a꜕bhi꜓vādemi \instr{Reverencia}

[Svākkhā꜓to] bha꜕gava꜓tā dhammo\\
Dhammaṁ namassāmi \instr{Reverencia}

[Supaṭi꜕panno] bha꜕gava꜕to sāvaka꜕saṅgho\\
Sa꜓ṅghaṁ na꜕māmi \instr{Reverencia}

\clearpage

\chapter{Dedicación de Ofrendas}

[Al Iluminado,] el Maestro, que totalmente alcanzó\\
\vin la iluminación perfecta,\\
A las enseñanzas, tan bien explicadas por Él,\\
A los discípulos del Maestro, que tan bien practicaron,\\
A estos – al Buddha, al Dhamma y a la Saṅgha ---\\
Presentamos el debido homenaje con ofrendas.\\
Es excelente para nosotros que el Excelso, habiéndose liberado,\\
Haya tenido compasión por las generaciones futuras.\\
Que estas simples ofrendas sean aceptadas\\
Para nuestro duradero beneficio y por la felicidad que nos da.
 
 
\bigskip


Al Maestro, el perfectamente Iluminado y Excelso ---\\
Al Buddha, el Excelso, yo rindo homenaje. \instr{Reverencia}

\bigskip

[A las enseñanzas,] tan bien explicadas por Él ---\\
Al Dhamma, yo rindo homenaje. \instr{Reverencia}

\bigskip

[A los discípulos del Maestro,] que tan bien practicaron\\
A la Saṅgha, yo rindo homenaje. \instr{Reverencia}

\clearpage

\chapter*{Homenaje Preliminar}

\begin{leader}
  [Ha꜓nda mayaṁ buddhassa꜕ bhagavato\\ pubbabhāga-namakā꜕raṁ karomase]
\end{leader}

Namo tassa bha꜕gava꜕to araha꜕to sa꜓mmāsa꜓mbuddha꜕ssa

\instr{Tres veces}

\chapter*{Remembranza del Buddha}

\delegateSetUseNext

\begin{leader}
  [Ha꜓nda mayaṁ buddhānu꜕ssa꜕ti꜕nayaṁ karomase]
\end{leader}

Taṁ kho꜓ pana bha꜕gavantaṁ evaṁ kaly꜓āṇo kitti꜕saddo abbhugga꜕to\\
Itipi so bha꜕gavā a꜕rahaṁ sammāsambuddho\\
Vijjāca꜕raṇa꜓-sampanno su꜕ga꜕to loka꜕vi꜓dū\\
Anu꜓tta꜕ro purisa꜕damma-sārathi satthā deva-ma꜕nussānaṁ\\
\vin buddho bha꜕gavā'ti

\clearpage

\chapter{Homenaje Preliminar}

\begin{leader}
  [Rindamos ahora homenaje preliminar al Buddha.]
\end{leader}

Homenaje al Excelso, Noble y Perfectamente Iluminado.

\instr{Tres veces}

\nextChapterUseDelegatedPageNumber

\chapter{Remembranza del Buddha}

\begin{leader}
  [Cantemos ahora la remembranza del Buddha.]
\end{leader}

Con la buena nueva sobre la reputación del Excelso,\\
\vin se ha oído lo siguiente:\\
Él, el Excelso, es realmente el Puro,\\
\vin el Perfectamente Iluminado;\\
Él es Impecable en conducta y comprensión,\\
\vin el Realizado, el Conocedor de los Mundos;\\
Él adiestra perfectamente aquellos que desean ser adiestrados;\\
Él es Maestro de dioses y humanos;\\
Él es Despierto y Sagrado.

\clearpage

\chapter*{Elogio Supremo al Buddha}

\delegateSetUseNext

\begin{leader}
  [Ha꜓nda mayaṁ buddhābhi꜕gī꜕tiṁ karomase]
\end{leader}

Buddh'vāra꜕ha꜓nta-varatādi꜕gu꜓ṇābhi꜕yutto\\
Suddhābhi꜕ñāṇa-ka꜕ru꜓ṇāhi sa꜓māga꜕tatto\\
Bodhesi꜕ yo su꜕ja꜕na꜓taṁ ka꜕ma꜓laṁ va꜕ sūro\\
Vandām'aha꜓ṁ ta꜕m-ara꜕ṇaṁ si꜕rasā꜓ ji꜕nendaṁ\\
Buddho yo sabba꜕-pāṇīnaṁ sa꜕raṇaṁ khema꜕m-utta꜕maṁ\\
Pa꜕ṭhamānussa꜕tiṭṭhānaṁ vandāmi꜕ taṁ si꜓ren'a꜕haṁ\\
Buddhassā꜓h'a꜕smi dāso/dāsī va buddho me sā꜕mi-ki꜓ssaro\\
Buddho dukkhassa꜕ ghātā ca꜕ vidhātā ca꜕ hi꜓tassa꜕ me\\
Buddhass'āha꜓ṁ niyyādemi sa꜕rīrañ-jīvi꜕tañ-ci꜕daṁ\\
Vandanto'ha꜓ṁ/Vandantī'ha꜓ṁ ca꜕rissāmi buddhass'eva꜕ su꜓bodhi꜕taṁ\\
Natthi me sa꜕ra꜓ṇaṁ aññaṁ buddho me sa꜕ra꜓ṇaṁ va꜕raṁ\\
Etena sacca꜕-vajjena vaḍḍheyyaṁ sa꜕tthu-sā꜓sane\\
Buddhaṁ me vanda꜕mānena/vanda꜕mānāya\\
\vin yaṁ puññaṁ pa꜕su꜓taṁ i꜕dha\\
Sa꜕bbepi anta꜕rāyā me māhe꜓su꜓ṁ ta꜕ssa꜓ teja꜕sā

\begin{instruction}
  Reverencia
\end{instruction}

Kāyena vācāya va ceta꜕sā꜓ vā\\
Bu꜓ddhe ku꜕kammaṁ pa꜕kataṁ ma꜕yā yaṁ\\
Bu꜓ddho pa꜕ṭiggaṇhā꜕tu acca꜕yantaṁ\\
Kālantare sa꜓ṁvarituṁ va꜕ bu꜓ddhe

\clearpage

\chapter{Elogio Supremo al Buddha}

\begin{leader}
  [Cantemos ahora en elogio supremo al Buddha.]
\end{leader}

El Buddha, verdaderamente valorable,\\
\vin dotado de tan excelentes cualidades,\\
Cuyo ser está compuesto de pureza,\\
\vin sabiduría transcendental y compasión,\\
Que iluminó a los sabios, así como el sol despierta la flor del loto ---\\
Yo venero ese pacífico líder de conquistadores.\\
El Buddha, que es el refugio seguro y supremo de todos los seres ---\\
Como primer objeto de remembranza, yo lo venero postrándome.\\
Soy de hecho el sirviente de Buddha, Buddha es mi maestro y guía.\\
Buddha es el destructor de tristeza, que proyecta bendiciones sobre mí.\\
A Buddha, me dedico en cuerpo y alma,\\
Y en devoción caminaré, el Sendero del Despertar de Buddha.\\
Para mí no existe otro refugio, Buddha es mi excelente refugio.\\
Por la afirmación de esta Verdad,\\
\vin que yo crezca en el camino del Maestro.\\
Por mi devoción a Buddha, y por el mérito de esta práctica ---\\
Por su poder, que todos los obstáculos sean vencidos.

\begin{instruction}
  Reverencia
\end{instruction}

De pensamiento, palabra u obra,\\
Por cualquier acción errónea que haya cometido hacia Buddha,\\
Que mi reconocimiento de esa falta sea aceptado,\\
De forma que en el futuro me refrene hacia Buddha.

\clearpage

\chapter*{Remembranza del Dhamma}

\delegateSetUseNext

\begin{leader}
  [Ha꜓nda mayaṁ dhammānu꜕ssa꜕ti꜕nayaṁ karomase]
\end{leader}

Svākkhā꜓to bha꜕gava꜓tā dhammo\\
Sa꜓ndiṭṭhi꜕ko a꜕kāli꜕ko ehi꜕passi꜕ko\\
Opanayi꜕ko pa꜕cca꜕ttaṁ vedi꜓ta꜕bbo viññūhī'ti

\chapter*{Elogio Supremo al Dhamma}

\begin{leader}
  [Ha꜓nda mayaṁ dhammābhi꜕gī꜕tiṁ karomase]
\end{leader}

Svākkhā꜓ta꜕t'ādi꜕guṇa-yoga꜕-va꜓sena꜕ seyyo\\
Yo magga꜕-pāka-pa꜕riyatti꜕-vi꜓mokkha꜕-bhedo\\
Dhammo ku꜕loka-pa꜕ta꜓nā ta꜕da꜓-dhāri꜕-dhārī\\
Vandām'aha꜓ṁ ta꜕ma-ha꜕raṁ va꜕ra-dha꜓mma꜕m-etaṁ\\
Dhammo yo sabba꜕-pāṇīnaṁ sa꜕raṇaṁ khema꜕m-utta꜕maṁ\\
Du꜕tiyānussa꜕tiṭṭhānaṁ vandāmi꜕ taṁ si꜓ren'a꜕haṁ\\
Dhammassā꜓h'a꜕smi dāso/dāsī va dhammo me sā꜕mi-ki꜓ssaro\\
Dhammo dukkhassa꜕ ghātā ca꜕ vidhātā ca꜕ hi꜓tassa꜕ me\\
Dhammass'āha꜓ṁ niyyādemi sa꜕rīrañ-jīvi꜕tañ-ci꜕daṁ\\
Vandantoha꜓ṁ/Vandantīha꜓ṁ ca꜕rissāmi dhammass'eva꜕ su꜓dhamma꜕taṁ\\
Natthi me sa꜕ra꜓ṇaṁ aññaṁ dhammo me sa꜕ra꜓ṇaṁ va꜕raṁ\\
Etena sacca꜕-vajjena vaḍḍheyyaṁ sa꜕tthu-sā꜓sane\\
Dhammaṁ me vanda꜕mānena/vanda꜕mānāya\\
\vin yaṁ puññaṁ pa꜕su꜓taṁ i꜕dha\\
Sa꜕bbepi anta꜕rāyā me māhe꜓su꜓ṁ ta꜕ssa꜓ teja꜕sā

\clearpage

\chapter{Remembranza del Dhamma}

\begin{leader}
  [Cantemos ahora la Remembranza del Dhamma.]
\end{leader}


El Dhamma fue bien explicado por el Excelso,\\
Presente aquí y ahora, intemporal, incentivando a investigar,\\
Guiando al interior, para ser experimentado individualmente\\
\vin por los sabios.


\nextChapterUseDelegatedPageNumber

\chapter{Elogio Supremo al Dhamma}

\begin{leader}
  [Cantemos ahora el elogio supremo al Dhamma.]
\end{leader}

\enlargethispage{\baselineskip}

Es excelente porque está bien explicado,\\
Y puede ser dividido en Camino y Fruto, Aprendizaje y Liberación.\\
El Dhamma protege a quienes lo mantienen, de caer en la ilusión.\\
Yo venero la excelente Enseñanza,\\
\vin que remueve la oscuridad ---\\
El Dhamma es el refugio seguro y supremo de todos los seres ---\\
Como segundo objeto de remembranza, yo lo venero postrándome.\\
Soy de hecho sirviente del Dhamma, el Dhamma es mi maestro y guía.\\
El Dhamma es destructor de tristeza, y proyecta bendiciones sobre mí.\\
Al Dhamma, me dedico en cuerpo y alma,\\
Y en devoción caminaré este excelente Camino de Verdad.\\
Para mí no existe otro refugio,\\
\vin El Dhamma es mi excelente refugio.\\
Por la afirmación de esta Verdad,\\
\vin que yo crezca en el camino del Maestro.\\
Por mi devoción al Dhamma, y por el mérito de esta práctica ---\\
Por su poder, que todos los obstáculos sean vencidos.

\clearpage

\begin{instruction}
  Reverencia
\end{instruction}

Kāyena vācāya va ceta꜕sā꜓ vā\\
Dha꜓mme ku꜕kammaṁ pa꜕kataṁ ma꜕yā yaṁ\\
Dha꜓mmo pa꜕ṭiggaṇhā꜕tu acca꜕yantaṁ\\
Kālantare sa꜓ṁvarituṁ va꜕ dha꜓mme

\chapter*[Remembranza Sangha]{Remembranza de la Saṅgha}

\delegateSetUseNext

\begin{leader}
  [Ha꜓nda mayaṁ saṅghānu꜕ssa꜕ti꜕nayaṁ karomase]
\end{leader}

Supaṭi꜕panno bha꜕gava꜕to sāvaka꜕saṅgho\\
Ujupaṭi꜕panno bha꜕gava꜕to sāvaka꜕saṅgho\\
Ñāyapaṭi꜕panno bha꜕gava꜕to sāvaka꜕saṅgho\\
Sā꜓mīci꜕pa꜕ṭi꜕panno bha꜕gava꜕to sāvaka꜕saṅgho\\
Yadidaṁ cattāri purisa꜕yugāni aṭṭha꜓ purisa꜕pugga꜕lā\\
Esa bha꜕gava꜕to sāvaka꜕saṅgho\\
Āhu꜕neyyo pāhu꜕neyyo dakkhi꜕ṇeyyo añja꜕li-ka꜕ra꜓ṇīyo\\
Anu꜓tta꜕raṁ puññakkhe꜕ttaṁ lokassā'ti

\clearpage

\begin{instruction}
  Reverencia
\end{instruction}

De pensamiento, palabra u obra,\\
Por cualquier acción errónea que haya cometido hacia el Dhamma,\\
Que mi reconocimiento de esa falta sea aceptado,\\
De forma que en el futuro me refrene hacia el Dhamma.

\chapter[Remembranza Sangha]{Remembranza de la Saṅgha}

\begin{leader}
  [Cantemos ahora la Remembranza de la Saṅgha.]
\end{leader}

Son los discípulos del Excelso que practicaron correctamente,\\
Que practicaron directamente,\\
Que practicaron con reflexión,\\
Aquellos que practicaron con integridad ---\\
Es decir, los cuatro pares, los ocho tipos de Seres Nobles ---\\
Estos son los discípulos del Maestro.\\
Tales discípulos son merecedores de presentes,\\
\vin merecedores de hospitalidad, merecedores de ofrendas,\\
\vin merecedores de respeto;\\
Ellos promueven el surgimiento de un bien incomparable en el mundo.

\clearpage

\chapter*[Elogio Sangha]{Elogio Supremo a la Saṅgha}

\delegateSetUseNext

\begin{leader}
  [Ha꜓nda mayaṁ saṅghābhi꜕gī꜕tiṁ karomase]
\end{leader}

Sa꜕ddhammajo supaṭipatti꜕-gu꜓ṇādi꜕yutto\\
Yo'ṭṭhabbi꜕dho ari꜓yapugga꜕la꜓-saṅgha꜕-seṭṭho\\
Sī꜓lādi꜕dhamma-pa꜕varāsa꜕ya꜓-kāya꜕-citto\\
Vandām'aha꜓ṁ ta꜕m-ari꜕yāna꜕-gaṇa꜓ṁ su꜕suddhaṁ\\
Sa꜓ṅgho yo sabba꜕-pāṇīnaṁ sa꜕raṇaṁ khema꜕m-utta꜕maṁ\\
Ta꜕tiyānussa꜕tiṭṭhānaṁ vandāmi꜕ taṁ si꜓ren'a꜕haṁ\\
Saṅghass'ā꜓ha꜕smi dāso/dāsī va saṅgho me sā꜕mi-ki꜓ssaro\\
Sa꜓ṅgho dukkhassa꜕ ghātā ca꜕ vi꜓dhātā ca꜕ hi꜓tassa꜕ me\\
Saṅghass'āha꜓ṁ niyyādemi sa꜕rīrañ-jīvi꜕tañ-ci꜕daṁ\\
Vandanto'ha꜓ṁ/Vandantī'ha꜓ṁ ca꜕rissāmi saṅghassopa꜕ṭi꜓panna꜕taṁ\\
Natthi me sa꜕ra꜓ṇaṁ aññaṁ saṅgho me sa꜕ra꜓ṇaṁ va꜕raṁ\\
Etena sacca꜕-vajjena vaḍḍheyyaṁ sa꜕tthu-sā꜓sane\\
Sa꜓ṅghaṁ me vanda꜕mānena/vanda꜕mānāya\\
\vin yaṁ puññaṁ pa꜕su꜓taṁ i꜕dha\\
Sa꜕bbepi anta꜕rāyā me māhe꜓su꜓ṁ ta꜕ssa꜓ teja꜕sā

\enlargethispage{\baselineskip}

\vfill

\begin{instruction}
  Reverencia
\end{instruction}

Kāyena vācāya va ceta꜕sā꜓ vā\\
Sa꜓ṅghe ku꜕kammaṁ pa꜕kataṁ ma꜕yā yaṁ\\
Sa꜓ṅgho pa꜕ṭiggaṇhā꜕tu acca꜕yantaṁ\\
Kālantare sa꜓ṁvarituṁ va꜕ sa꜓ṅghe

\vfill

\begin{instruction}
  A partir de este momento se practica meditación en silencio, a veces seguida de una enseñanza de Dhamma, y terminando con lo siguiente:
\end{instruction}

\clearpage

\chapter[Elogio Sangha]{Elogio Supremo a la Saṅgha}

\begin{leader}
  [Cantemos ahora el elogio supremo a la Saṅgha.]
\end{leader}

Nacida del Dhamma, esa Saṅgha que practicó bien,\\
El campo de la Saṅgha formado por ocho tipos de Seres Nobles,\\
Guiados en cuerpo y mente por excelente moralidad y virtud.\\
Yo venero esa asamblea de Seres Nobles,\\
\vin perfeccionados en pureza.\\
La Saṅgha, que es el refugio seguro y supremo de todos los seres ---\\
Como tercer objeto de remembranza, yo lo venero postrándome.\\
Soy de hecho sirviente de la Saṅgha, la Saṅgha es mi maestro y guía.\\
La Saṅgha es destructor de tristeza, que proyecta bendiciones sobre mí.\\
A la Saṅgha, me dedico en cuerpo y alma,\\
Y en devoción caminaré, el Camino bien practicado de la Saṅgha.\\
Para mí no existe otro refugio, la Saṅgha es mi excelente refugio.\\
Por la afirmación de esta Verdad,\\
\vin que yo crezca en el camino del Maestro.\\
Por mi devoción a la Saṅgha, y por el mérito de esta práctica ---\\
Por su poder, que todos los obstáculos sean vencidos.

\enlargethispage{2\baselineskip}

\begin{instruction}
  Reverencia
\end{instruction}

De pensamiento, palabra u obra,\\
Por cualquier acción errónea que haya cometido hacia la Saṅgha,\\
Que mi reconocimiento de esa falta sea aceptado,\\
De forma que en el futuro me refrene hacia la Saṅgha.

\begin{instruction}
  A partir de este momento se practica meditación en silencio, a veces seguida de una enseñanza de Dhamma, y terminando con lo siguiente:
\end{instruction}

\clearpage

\chapter*{Homenaje de Cierre}

\delegateSetUseNext

[Arahaṁ] sammāsambuddho bha꜕gavā\\
Buddhaṁ bha꜕gavantaṁ a꜕bhi꜓vādemi \instr{Reverencia}

[Svākkhā꜓to] bha꜕gava꜓tā dhammo\\
Dhammaṁ namassāmi \instr{Reverencia}

[Supaṭi꜕panno] bha꜕gava꜕to sāvaka꜕saṅgho\\
Sa꜓ṅghaṁ na꜕māmi \instr{Reverencia}

\clearpage

\chapter{Homenaje de Cierre}

[Al Maestro,] el perfectamente Iluminado y Excelso \\
Al Buddha, el Excelso, yo rindo homenaje. \instr{Reverencia}

[A las Enseñanzas,] tan bien explicadas por Él \\
Al Dhamma, yo rindo homenaje. \instr{Reverencia}

[A los discípulos del Maestro,] que tan bien practicaron \\
A la Saṅgha, yo rindo homenaje. \instr{Reverencia}
