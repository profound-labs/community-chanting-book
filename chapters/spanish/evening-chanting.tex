\chapter*{Dedicação de Oferendas}

\delegateSetUseNext

[Yo so] bha꜕gavā a꜕rahaṁ sammāsambuddho\\
Svākkhā꜓to yena bha꜕gava꜓tā dhammo\\
Supaṭi꜕panno yassa bha꜕gava꜕to sāvaka꜕saṅgho\\
Tam-ma꜓yaṁ bha꜕gavantaṁ sa꜕dhammaṁ sa꜕saṅghaṁ\\
Imehi꜓ sakkārehi꜕ yathārahaṁ āropi꜕tehi a꜕bhi꜓pūja꜕yāma\\
Sādhu꜓ no bhante bha꜕gavā su꜕cira-parinibbu꜕topi\\
Pacchi꜓mā-ja꜕na꜓tānu꜓kampa꜕-mānasā\\
Ime sakkāre dugga꜕ta꜕-paṇṇākāra꜓-bhūte pa꜕ṭiggaṇhātu\\
Amhā꜓kaṁ dīgha꜕rattaṁ hi꜕tāya su꜕khāya\\
Arahaṁ sammāsambuddho bha꜕gavā\\
Buddhaṁ bha꜕gavantaṁ a꜕bhi꜓vādemi \instr{Vénia}

[Svākkhā꜓to] bha꜕gava꜓tā dhammo\\
Dhammaṁ namassāmi \instr{Vénia}

[Supaṭi꜕panno] bha꜕gava꜕to sāvaka꜕saṅgho\\
Sa꜓ṅghaṁ na꜕māmi \instr{Vénia}

\clearpage

\chapter{Dedicação de Oferendas}

[Ao Excelso,] o Mestre, que totalmente alcançou\\
\vin a iluminação perfeita,\\
Ao ensinamento, tão bem explicado por Ele,\\
E aos discípulos do Excelso, que tão bem praticaram,\\
A estes – ao Buddha, ao Dhamma e ao Saṅgha ---\\
Apresentamos a devida homenagem com oferendas.\\
É benéfico para nós, que tendo o Excelso se libertado,\\
Tenha ainda tido compaixão pelas gerações futuras.\\
Que estas simples oferendas sejam aceites\\
Pelo nosso duradouro benefício e pela felicidade que nos dá.\\
Ao Mestre, O perfeitamente Iluminado e Excelso ---\\
Ao Buddha, o Excelso, eu presto homenagem. \instr{Vénia}

[Ao ensinamento,] tão bem explicado por Ele ---\\
 Ao Dhamma, eu presto homenagem. \instr{Vénia}

[Aos discípulos do Excelso,] que tão bem praticaram ---\\
Ao Saṅgha, eu presto homenagem. \instr{Vénia}

\clearpage

\chapter*{Homenagem Preliminar}

\begin{leader}
  [Ha꜓nda mayaṁ buddhassa꜕ bhagavato\\ pubbabhāga-namakā꜕raṁ karomase]
\end{leader}

Namo tassa bha꜕gava꜕to araha꜕to sa꜓mmāsa꜓mbuddha꜕ssa

\instr{Três vezes}

\chapter*{Recordação do Buddha}

\delegateSetUseNext

\begin{leader}
  [Ha꜓nda mayaṁ buddhānu꜕ssa꜕ti꜕nayaṁ karomase]
\end{leader}

Taṁ kho꜓ pana bha꜕gavantaṁ evaṁ kaly꜓āṇo kitti꜕saddo abbhugga꜕to\\
Itipi so bha꜕gavā a꜕rahaṁ sammāsambuddho\\
Vijjāca꜕raṇa꜓-sampanno su꜕ga꜕to loka꜕vi꜓dū\\
Anu꜓tta꜕ro purisa꜕damma-sārathi satthā deva-ma꜕nussānaṁ\\
\vin buddho bha꜕gavā'ti

\clearpage

\chapter{Homenagem Preliminar}

\begin{leader}
  [Prestemos agora homenagem preliminar ao Buddha.]
\end{leader}

Homenagem ao Excelso, Nobre e Perfeitamente Iluminado.

\instr{Três vezes}

\nextChapterUseDelegatedPageNumber

\chapter{Recordação do Buddha}

\begin{leader}
  [Cantemos agora a recordação do Buddha.]
\end{leader}

A boa nova sobre a reputação do Excelso ouviu-se como se segue:\\
Ele, o Excelso, é realmente O Puro,\\
\vin o Perfeitamente Iluminado;\\
Ele é Impecável na conduta e na compreensão,\\
\vin o Realizado, o Conhecedor dos mundos;\\
Ele treina perfeitamente aqueles que desejam treinar-se;\\
Ele é Professor de deuses e humanos;\\
Ele é Desperto e Sagrado.

\clearpage

\chapter*{O Elogio Supremo ao Buddha}

\delegateSetUseNext

\begin{leader}
  [Ha꜓nda mayaṁ buddhābhi꜕gī꜕tiṁ karomase]
\end{leader}

Buddh'vāra꜕ha꜓nta-varatādi꜕gu꜓ṇābhi꜕yutto\\
Suddhābhi꜕ñāṇa-ka꜕ru꜓ṇāhi sa꜓māga꜕tatto\\
Bodhesi꜕ yo su꜕ja꜕na꜓taṁ ka꜕ma꜓laṁ va꜕ sūro\\
Vandām'aha꜓ṁ ta꜕m-ara꜕ṇaṁ si꜕rasā꜓ ji꜕nendaṁ\\
Buddho yo sabba꜕-pāṇīnaṁ sa꜕raṇaṁ khema꜕m-utta꜕maṁ\\
Pa꜕ṭhamānussa꜕tiṭṭhānaṁ vandāmi꜕ taṁ si꜓ren'a꜕haṁ\\
Buddhassā꜓h'a꜕smi dāso/dāsī va buddho me sā꜕mi-ki꜓ssaro\\
Buddho dukkhassa꜕ ghātā ca꜕ vidhātā ca꜕ hi꜓tassa꜕ me\\
Buddhass'āha꜓ṁ niyyādemi sa꜕rīrañ-jīvi꜕tañ-ci꜕daṁ\\
Vandanto'ha꜓ṁ/Vandantī'ha꜓ṁ ca꜕rissāmi buddhass'eva꜕ su꜓bodhi꜕taṁ\\
Natthi me sa꜕ra꜓ṇaṁ aññaṁ buddho me sa꜕ra꜓ṇaṁ va꜕raṁ\\
Etena sacca꜕-vajjena vaḍḍheyyaṁ sa꜕tthu-sā꜓sane\\
Buddhaṁ me vanda꜕mānena/vanda꜕mānāya\\
\vin yaṁ puññaṁ pa꜕su꜓taṁ i꜕dha\\
Sa꜕bbepi anta꜕rāyā me māhe꜓su꜓ṁ ta꜕ssa꜓ teja꜕sā

\begin{instruction}
  Vénia
\end{instruction}

Kāyena vācāya va ceta꜕sā꜓ vā\\
Bu꜓ddhe ku꜕kammaṁ pa꜕kataṁ ma꜕yā yaṁ\\
Bu꜓ddho pa꜕ṭiggaṇhā꜕tu acca꜕yantaṁ\\
Kālantare sa꜓ṁvarituṁ va꜕ bu꜓ddhe

\clearpage

\chapter{O Elogio Supremo ao Buddha}

\begin{leader}
  [Cantemos agora o elogio supremo ao Buddha.]
\end{leader}

O Buddha, verdadeiramente valoroso, dotado de\\
\vin tais excelentes qualidades,\\
Cujo ser é composto de pureza, sabedoria transcendental,\\
\vin e compaixão,\\
Que iluminou os sábios, tal sol a despertar a flor de lótus ---\\
Eu reverencio esse pacífico líder de conquistadores.\\
O Buddha, que é o refúgio seguro e supremo de todos os seres ---\\
Como Primeiro Objecto de Recordação, eu reverencio-O inclinando-me.\\
Sou de facto o servidor do Buddha, O Buddha é meu mestre e guia.\\
O Buddha é o destruidor da tristeza, que lança bênçãos sobre mim.\\
Ao Buddha dedico este corpo e vida,\\
E em devoção percorrerei, o Caminho de Despertar do Buddha.\\
Para mim não existe outro refúgio, O Buddha é o meu excelente\\
\vin refúgio.\\
Pela afirmação desta Verdade, que eu cresça no caminho do Mestre.\\
Pela minha devoção ao Buddha, e pela bênção desta prática ---\\
Pelo seu poder, que todos os obstáculos sejam vencidos.

\begin{instruction}
  Vénia
\end{instruction}

Quer através do corpo, da fala ou da mente,\\
Por qualquer acção errada que eu tenha cometido\\
\vin para com O Buddha,\\
Que o meu reconhecimento dessa falha seja aceite,\\
De forma a que no futuro haja domínio respeitando O Buddha.

\clearpage

\chapter*{Recordação do Dhamma}

\delegateSetUseNext

\begin{leader}
  [Ha꜓nda mayaṁ dhammānu꜕ssa꜕ti꜕nayaṁ karomase]
\end{leader}

Svākkhā꜓to bha꜕gava꜓tā dhammo\\
Sa꜓ndiṭṭhi꜕ko a꜕kāli꜕ko ehi꜕passi꜕ko\\
Opanayi꜕ko pa꜕cca꜕ttaṁ vedi꜓ta꜕bbo viññūhī'ti

\chapter*{O Elogio Supremo ao Dhamma}

\begin{leader}
  [Ha꜓nda mayaṁ dhammābhi꜕gī꜕tiṁ karomase]
\end{leader}

Svākkhā꜓ta꜕t'ādi꜕guṇa-yoga꜕-va꜓sena꜕ seyyo\\
Yo magga꜕-pāka-pa꜕riyatti꜕-vi꜓mokkha꜕-bhedo\\
Dhammo ku꜕loka-pa꜕ta꜓nā ta꜕da꜓-dhāri꜕-dhārī\\
Vandām'aha꜓ṁ ta꜕ma-ha꜕raṁ va꜕ra-dha꜓mma꜕m-etaṁ\\
Dhammo yo sabba꜕-pāṇīnaṁ sa꜕raṇaṁ khema꜕m-utta꜕maṁ\\
Du꜕tiyānussa꜕tiṭṭhānaṁ vandāmi꜕ taṁ si꜓ren'a꜕haṁ\\
Dhammassā꜓h'a꜕smi dāso/dāsī va dhammo me sā꜕mi-ki꜓ssaro\\
Dhammo dukkhassa꜕ ghātā ca꜕ vidhātā ca꜕ hi꜓tassa꜕ me\\
Dhammass'āha꜓ṁ niyyādemi sa꜕rīrañ-jīvi꜕tañ-ci꜕daṁ\\
Vandantoha꜓ṁ/Vandantīha꜓ṁ ca꜕rissāmi dhammass'eva꜕ su꜓dhamma꜕taṁ\\
Natthi me sa꜕ra꜓ṇaṁ aññaṁ dhammo me sa꜕ra꜓ṇaṁ va꜕raṁ\\
Etena sacca꜕-vajjena vaḍḍheyyaṁ sa꜕tthu-sā꜓sane\\
Dhammaṁ me vanda꜕mānena/vanda꜕mānāya\\
\vin yaṁ puññaṁ pa꜕su꜓taṁ i꜕dha\\
Sa꜕bbepi anta꜕rāyā me māhe꜓su꜓ṁ ta꜕ssa꜓ teja꜕sā

\clearpage

\chapter{Recordação do Dhamma}

\begin{leader}
  [Cantemos agora a Recordação do Dhamma.]
\end{leader}

O Dhamma é bem explicado pelo Excelso,\\
Imanente aqui e agora, intemporal, encorajando investigação,\\
Conduzindo ao interior, para ser experimentado individualmente\\
\vin pelos sábios.

\nextChapterUseDelegatedPageNumber

\chapter{O Elogio Supremo ao Dhamma}

\begin{leader}
  [Cantemos agora o elogio supremo ao Dhamma.]
\end{leader}

É excelente porque é bem explicado,\\
E pode ser dividido em Caminho e Fruto, Aprendizado e Libertação.\\
O Dhamma guarda aqueles que o mantêm, de caírem na ilusão.\\
Eu reverencio o excelente Ensinamento,\\
\vin que remove a escuridão ---\\
O Dhamma que é o refúgio seguro e supremo de todos os seres ---\\
Como segundo objecto de Recordação, eu reverencio-O inclinando-me.\\
Sou de facto o servidor do Dhamma, o Dhamma é meu mestre e guia.\\
O Dhamma é o destruidor da tristeza, e lança bênçãos sobre mim.\\
Ao Dhamma dedico este corpo e vida,\\
E em devoção percorrerei este excelente Caminho da Verdade.\\
Para mim não existe outro refúgio,\\
\vin o Dhamma é o meu excelente refúgio.\\
Pela afirmação desta Verdade, que eu cresça no caminho do Mestre.\\
Pela minha devoção ao Dhamma, e pela bênção desta prática ---\\
Pelo seu poder, que todos os obstáculos sejam vencidos.

\clearpage

\begin{instruction}
  Vénia
\end{instruction}

Kāyena vācāya va ceta꜕sā꜓ vā\\
Dha꜓mme ku꜕kammaṁ pa꜕kataṁ ma꜕yā yaṁ\\
Dha꜓mmo pa꜕ṭiggaṇhā꜕tu acca꜕yantaṁ\\
Kālantare sa꜓ṁvarituṁ va꜕ dha꜓mme

\chapter*{Recordação do Saṅgha}

\delegateSetUseNext

\begin{leader}
  [Ha꜓nda mayaṁ saṅghānu꜕ssa꜕ti꜕nayaṁ karomase]
\end{leader}

Supaṭi꜕panno bha꜕gava꜕to sāvaka꜕saṅgho\\
Ujupaṭi꜕panno bha꜕gava꜕to sāvaka꜕saṅgho\\
Ñāyapaṭi꜕panno bha꜕gava꜕to sāvaka꜕saṅgho\\
Sā꜓mīci꜕pa꜕ṭi꜕panno bha꜕gava꜕to sāvaka꜕saṅgho\\
Yadidaṁ cattāri purisa꜕yugāni aṭṭha꜓ purisa꜕pugga꜕lā\\
Esa bha꜕gava꜕to sāvaka꜕saṅgho\\
Āhu꜕neyyo pāhu꜕neyyo dakkhi꜕ṇeyyo añja꜕li-ka꜕ra꜓ṇīyo\\
Anu꜓tta꜕raṁ puññakkhe꜕ttaṁ lokassā'ti

\clearpage

\begin{instruction}
  Vénia
\end{instruction}

Quer através do corpo, da fala ou da mente,\\
Por qualquer acção errada que eu tenha cometido\\
\vin para com o Dhamma,\\
Que o meu reconhecimento dessa falha seja aceite,\\
De forma a que no futuro haja domínio respeitando o Dhamma.

\chapter{Recordação do Saṅgha}

\begin{leader}
  [Cantemos agora a Recordação do Saṅgha.]
\end{leader}

São os discípulos do Excelso que praticaram correctamente,\\
Que praticaram directamente,\\
Que praticaram reflectidamente,\\
Aqueles que praticam com integridade ---\\
Isso é, os quatro pares, os oito tipos de Seres Nobres ---\\
Estes são os discípulos do Excelso.\\
Tais discípulos são merecedores de presentes,\\
\vin merecedores de hospitalidade, merecedores de oferendas,\\
\vin merecedores de respeito;\\
Eles oferecem oportunidade, para que no mundo surja incomparável bondade.

\clearpage

\chapter*{O Elogio Supremo ao Saṅgha}

\begin{leader}
  [Ha꜓nda mayaṁ saṅghābhi꜕gī꜕tiṁ karomase]
\end{leader}

Sa꜕ddhammajo supaṭipatti꜕-gu꜓ṇādi꜕yutto\\
Yo'ṭṭhabbi꜕dho ari꜓yapugga꜕la꜓-saṅgha꜕-seṭṭho\\
Sī꜓lādi꜕dhamma-pa꜕varāsa꜕ya꜓-kāya꜕-citto\\
Vandām'aha꜓ṁ ta꜕m-ari꜕yāna꜕-gaṇa꜓ṁ su꜕suddhaṁ\\
Sa꜓ṅgho yo sabba꜕-pāṇīnaṁ sa꜕raṇaṁ khema꜕m-utta꜕maṁ\\
Ta꜕tiyānussa꜕tiṭṭhānaṁ vandāmi꜕ taṁ si꜓ren'a꜕haṁ\\
Saṅghass'ā꜓ha꜕smi dāso/dāsī va saṅgho me sā꜕mi-ki꜓ssaro\\
Sa꜓ṅgho dukkhassa꜕ ghātā ca꜕ vi꜓dhātā ca꜕ hi꜓tassa꜕ me\\
Saṅghass'āha꜓ṁ niyyādemi sa꜕rīrañ-jīvi꜕tañ-ci꜕daṁ\\
Vandanto'ha꜓ṁ/Vandantī'ha꜓ṁ ca꜕rissāmi saṅghassopa꜕ṭi꜓panna꜕taṁ\\
Natthi me sa꜕ra꜓ṇaṁ aññaṁ saṅgho me sa꜕ra꜓ṇaṁ va꜕raṁ\\
Etena sacca꜕-vajjena vaḍḍheyyaṁ sa꜕tthu-sā꜓sane\\
Sa꜓ṅghaṁ me vanda꜕mānena/vanda꜕mānāya\\
\vin yaṁ puññaṁ pa꜕su꜓taṁ i꜕dha\\
Sa꜕bbepi anta꜕rāyā me māhe꜓su꜓ṁ ta꜕ssa꜓ teja꜕sā

\enlargethispage{\baselineskip}

\vfill

\begin{instruction}
  Vénia
\end{instruction}

Kāyena vācāya va ceta꜕sā꜓ vā\\
Sa꜓ṅghe ku꜕kammaṁ pa꜕kataṁ ma꜕yā yaṁ\\
Sa꜓ṅgho pa꜕ṭiggaṇhā꜕tu acca꜕yantaṁ\\
Kālantare sa꜓ṁvarituṁ va꜕ sa꜓ṅghe

\vfill

\begin{instruction}
  Nesta altura a meditação é praticada em silêncio, por vezes seguida de uma palestra de Dhamma, e terminando com o seguinte:
\end{instruction}

\clearpage

\nextChapterUseDelegatedPageNumber

\chapter{O Elogio Supremo ao Saṅgha}

\begin{leader}
  [Cantemos agora o elogio supremo ao Saṅgha.]
\end{leader}

Nascido do Dhamma, esse Saṅgha que praticou bem,\\
O campo do Saṅgha formado por oito tipos de Seres Nobres,\\
Guiados em corpo e mente por excelente moralidade e virtude.\\
Eu reverencio essa assembleia de Seres Nobres,\\
\vin aperfeiçoados em pureza.\\
O Saṅgha, que é o refúgio seguro e supremo de todos os seres ---\\
Como terceiro objecto de Recordação, eu reverencio-O inclinando-me.\\
Sou de facto o servidor do Saṅgha, O Saṅgha é meu mestre e guia.\\
O Saṅgha é o destruidor da tristeza, que lança bênçãos sobre mim.\\
Ao Saṅgha dedico este corpo e vida,\\
E em devoção percorrerei, o Caminho bem praticado do Saṅgha.\\
Para mim não existe outro refúgio, o Saṅgha é o meu excelente refúgio.\\
Pela afirmação desta Verdade, que eu cresça no caminho do Mestre.\\
Pela minha devoção ao Saṅgha, e pela bênção desta prática ---\\
Pelo seu poder, que todos os obstáculos sejam vencidos.

\enlargethispage{\baselineskip}

\vfill

\begin{instruction}
  Vénia
\end{instruction}

Quer através do corpo, da fala ou da mente,\\
Por qualquer acção errada que eu tenha cometido\\
\vin para com o Saṅgha,\\
Que o meu reconhecimento dessa falha seja aceite,\\
De forma a que no futuro haja domínio respeitando o Saṅgha.

\begin{instruction}
  Nesta altura a meditação é praticada em silêncio, por vezes seguida de uma palestra de Dhamma, e terminando com o seguinte:
\end{instruction}

\clearpage

\chapter*{Homenagem de Encerramento}

\delegateSetUseNext

[Arahaṁ] sammāsambuddho bha꜕gavā\\
Buddhaṁ bha꜕gavantaṁ a꜕bhi꜓vādemi \instr{Vénia}

[Svākkhā꜓to] bha꜕gava꜓tā dhammo\\
Dhammaṁ namassāmi \instr{Vénia}

[Supaṭi꜕panno] bha꜕gava꜕to sāvaka꜕saṅgho\\
Sa꜓ṅghaṁ na꜕māmi \instr{Vénia}

\clearpage

\chapter{Homenagem de Encerramento}

[Ao Mestre,] O perfeitamente Iluminado e Excelso ---\\
Ao Buddha, o Excelso, eu presto homenagem. \instr{Vénia}

[Ao ensinamento,] tão bem explicado por Ele ---\\
Ao Dhamma, eu presto homenagem. \instr{Vénia}

[Aos discípulos do Excelso,] que tão bem praticaram ---\\
Ao Saṅgha, eu presto homenagem. \instr{Vénia}

