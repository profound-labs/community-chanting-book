\chapter*{Dedicación de Ofrendas}

\delegateSetUseNext

[Yo so] bha꜕gavā a꜕rahaṁ sammāsambuddho\\
Svākkhā꜓to yena bha꜕gava꜓tā dhammo\\
Supaṭi꜕panno yassa bha꜕gava꜕to sāvaka꜕saṅgho\\
Tam-ma꜓yaṁ bha꜕gavantaṁ sa꜕dhammaṁ sa꜕saṅghaṁ\\
Imehi꜓ sakkārehi꜕ yathārahaṁ āropi꜕tehi a꜕bhi꜓pūja꜕yāma\\
Sādhu꜓ no bhante bha꜕gavā su꜕cira-parinibbu꜕topi\\
Pacchi꜓mā-ja꜕na꜓tānu꜓kampa꜕-mānasā\\
Ime sakkāre dugga꜕ta꜕-paṇṇākāra꜓-bhūte pa꜕ṭiggaṇhātu\\
Amhā꜓kaṁ dīgha꜕rattaṁ hi꜕tāya su꜕khāya

[Arahaṁ] sammāsambuddho bha꜕gavā\\
Buddhaṁ bha꜕gavantaṁ a꜕bhi꜓vādemi \instr{Reverencia}

[Svākkhā꜓to] bha꜕gava꜓tā dhammo\\
Dhammaṁ namassāmi \instr{Reverencia}

[Supaṭi꜕panno] bha꜕gava꜕to sāvaka꜕saṅgho\\
Sa꜓ṅghaṁ na꜕māmi \instr{Reverencia}

\clearpage

\chapter{Dedicación de Ofrendas}

[Al Iluminado,] el Maestro, que alcanzó totalmente\\
\vin la iluminación perfecta,\\
A las enseñanzas, explicadas tan bien por Él,\\
A los discípulos del Maestro, que practicaron tan bien,\\
A estos – al Buda, al Dhamma y a la Sangha ---\\
Presentamos el debido homenaje con ofrendas.\\
Para nosotros es beneficioso que el Maestro,\\
\vin habiéndose liberado,\\
Haya tenido compasión por las generaciones futuras.\\
Que estas simples ofrendas sean aceptadas\\
Por nuestro duradero beneficio y por la felicidad que nos da.

[Al Maestro,] el perfectamente Iluminado y Excelso ---\\
Al Buddha, el Excelso, le rindo homenaje. \instr{Reverencia}

[A la enseñanza,] explicada tan bien por Él ---\\
 Al Dhamma, le rindo homenaje. \instr{Reverencia}

[A los discipulos del Excelso,] que practicaron tan bien \\
A la Sangha, le rindo homenaje. \instr{Reverencia}

\clearpage

\chapter*{Homenaje Preliminar}

\begin{leader}
  [Ha꜓nda mayaṁ buddhassa꜕ bhagavato\\ pubbabhāga-namakā꜕raṁ karomase]
\end{leader}

Namo tassa bha꜕gava꜕to araha꜕to sa꜓mmāsa꜓mbuddha꜕ssa

\instr{Tres veces}

\chapter*{Recuerdo del Buddha}

\delegateSetUseNext

\begin{leader}
  [Ha꜓nda mayaṁ buddhānu꜕ssa꜕ti꜕nayaṁ karomase]
\end{leader}

Taṁ kho꜓ pana bha꜕gavantaṁ evaṁ kaly꜓āṇo kitti꜕saddo abbhugga꜕to\\
Itipi so bha꜕gavā a꜕rahaṁ sammāsambuddho\\
Vijjāca꜕raṇa꜓-sampanno su꜕ga꜕to loka꜕vi꜓dū\\
Anu꜓tta꜕ro purisa꜕damma-sārathi satthā deva-ma꜕nussānaṁ\\
\vin buddho bha꜕gavā'ti

\clearpage

\chapter{Homenaje Preliminar}

\begin{leader}
  [Rindamos ahora homenaje preliminar al Buddha.]
\end{leader}

Homenaje al Excelso, Noble y Perfectamente Iluminado.

\instr{Tres veces}

\nextChapterUseDelegatedPageNumber

\chapter{Recuerdo del Buda}

\begin{leader}
  [Cantemos ahora el recuerdo del Buda.]
\end{leader}

Con la buena nueva sobre la reputación del Excelso,\\
\vin se ha oido lo siguiente:\\
Él, el Excelso, es realmente El Puro,\\
\vin el Perfectamente Iluminado;\\
Él es Impecable en conducta y en comprensión,\\
\vin el Realizado, el Conocedor de los mundos;\\
Él entrena perfectamente aquellos que desean entreinarse;\\
Él es Maestro de dioses y humanos;\\
Él es Despierto y Sagrado.

\clearpage

\chapter*{Elogio Supremo al Buda}

\delegateSetUseNext

\begin{leader}
  [Ha꜓nda mayaṁ buddhābhi꜕gī꜕tiṁ karomase]
\end{leader}

Buddh'vāra꜕ha꜓nta-varatādi꜕gu꜓ṇābhi꜕yutto\\
Suddhābhi꜕ñāṇa-ka꜕ru꜓ṇāhi sa꜓māga꜕tatto\\
Bodhesi꜕ yo su꜕ja꜕na꜓taṁ ka꜕ma꜓laṁ va꜕ sūro\\
Vandām'aha꜓ṁ ta꜕m-ara꜕ṇaṁ si꜕rasā꜓ ji꜕nendaṁ\\
Buddho yo sabba꜕-pāṇīnaṁ sa꜕raṇaṁ khema꜕m-utta꜕maṁ\\
Pa꜕ṭhamānussa꜕tiṭṭhānaṁ vandāmi꜕ taṁ si꜓ren'a꜕haṁ\\
Buddhassā꜓h'a꜕smi dāso/dāsī va buddho me sā꜕mi-ki꜓ssaro\\
Buddho dukkhassa꜕ ghātā ca꜕ vidhātā ca꜕ hi꜓tassa꜕ me\\
Buddhass'āha꜓ṁ niyyādemi sa꜕rīrañ-jīvi꜕tañ-ci꜕daṁ\\
Vandanto'ha꜓ṁ/Vandantī'ha꜓ṁ ca꜕rissāmi buddhass'eva꜕ su꜓bodhi꜕taṁ\\
Natthi me sa꜕ra꜓ṇaṁ aññaṁ buddho me sa꜕ra꜓ṇaṁ va꜕raṁ\\
Etena sacca꜕-vajjena vaḍḍheyyaṁ sa꜕tthu-sā꜓sane\\
Buddhaṁ me vanda꜕mānena/vanda꜕mānāya\\
\vin yaṁ puññaṁ pa꜕su꜓taṁ i꜕dha\\
Sa꜕bbepi anta꜕rāyā me māhe꜓su꜓ṁ ta꜕ssa꜓ teja꜕sā

\begin{instruction}
  Reverencia
\end{instruction}

Kāyena vācāya va ceta꜕sā꜓ vā\\
Bu꜓ddhe ku꜕kammaṁ pa꜕kataṁ ma꜕yā yaṁ\\
Bu꜓ddho pa꜕ṭiggaṇhā꜕tu acca꜕yantaṁ\\
Kālantare sa꜓ṁvarituṁ va꜕ bu꜓ddhe

\clearpage

\chapter{Elogio Supremo al Buda}

\begin{leader}
  [Cantemos ahora en elogio supremo al Buddha.]
\end{leader}

El Buddha, verdaderamente valorable,\\
\vin dotado de tales excelentes cualidades,\\
Cuyo ser está compuesto de pureza,\\
\vin sabiduría transcendental y compasión,\\
Que iluminó a los sábios, así como el sol despierta la flor del loto ---\\
Yo venero ese pacífico líder de conquistadores.\\
El Buddha, que es el refugio seguro y supremo de todos los seres ---\\
Como Primero Objecto de Recuerdo, yo lo venero postrándome.\\
Soy de hecho el esclavo de Buddha, Buddha es mi maestro y guía.\\
Buddha es el destructor de tristeza, que proyecta bendiciones sobre mí.\\
A Buddha dedico este cuerpo y vida,\\
Y en devoción caminaré, el Camino de Despertar de Buddha.\\
Para mí no existe otro refugio, Buddha es mi excelente refugio.\\
Por la afirmación de esta Verdad,\\
\vin que yo crezca en el camino del Maestro.\\
Por mi devoción a Buddha, y por el mérito de esta práctica ---\\
Por su poder, que todos los obstáculos sean vencidos.

\begin{instruction}
  Reverencia
\end{instruction}

Ya sea a través del cuerpo, de palabra o de mente,\\
Por cualquier acción erronea que haya cometido\\
\vin hacia el Buddha,\\
Que mi reconocimento de esa falta sea aceptado,\\
De forma que en el futuro modere mi comportamiento hacia Buddha.

\clearpage

\chapter*{Recuerdo del Dhamma}

\delegateSetUseNext

\begin{leader}
  [Ha꜓nda mayaṁ dhammānu꜕ssa꜕ti꜕nayaṁ karomase]
\end{leader}

Svākkhā꜓to bha꜕gava꜓tā dhammo\\
Sa꜓ndiṭṭhi꜕ko a꜕kāli꜕ko ehi꜕passi꜕ko\\
Opanayi꜕ko pa꜕cca꜕ttaṁ vedi꜓ta꜕bbo viññūhī'ti

\chapter*{Elogio Supremo al Dhamma}

\begin{leader}
  [Ha꜓nda mayaṁ dhammābhi꜕gī꜕tiṁ karomase]
\end{leader}

Svākkhā꜓ta꜕t'ādi꜕guṇa-yoga꜕-va꜓sena꜕ seyyo\\
Yo magga꜕-pāka-pa꜕riyatti꜕-vi꜓mokkha꜕-bhedo\\
Dhammo ku꜕loka-pa꜕ta꜓nā ta꜕da꜓-dhāri꜕-dhārī\\
Vandām'aha꜓ṁ ta꜕ma-ha꜕raṁ va꜕ra-dha꜓mma꜕m-etaṁ\\
Dhammo yo sabba꜕-pāṇīnaṁ sa꜕raṇaṁ khema꜕m-utta꜕maṁ\\
Du꜕tiyānussa꜕tiṭṭhānaṁ vandāmi꜕ taṁ si꜓ren'a꜕haṁ\\
Dhammassā꜓h'a꜕smi dāso/dāsī va dhammo me sā꜕mi-ki꜓ssaro\\
Dhammo dukkhassa꜕ ghātā ca꜕ vidhātā ca꜕ hi꜓tassa꜕ me\\
Dhammass'āha꜓ṁ niyyādemi sa꜕rīrañ-jīvi꜕tañ-ci꜕daṁ\\
Vandantoha꜓ṁ/Vandantīha꜓ṁ ca꜕rissāmi dhammass'eva꜕ su꜓dhamma꜕taṁ\\
Natthi me sa꜕ra꜓ṇaṁ aññaṁ dhammo me sa꜕ra꜓ṇaṁ va꜕raṁ\\
Etena sacca꜕-vajjena vaḍḍheyyaṁ sa꜕tthu-sā꜓sane\\
Dhammaṁ me vanda꜕mānena/vanda꜕mānāya\\
\vin yaṁ puññaṁ pa꜕su꜓taṁ i꜕dha\\
Sa꜕bbepi anta꜕rāyā me māhe꜓su꜓ṁ ta꜕ssa꜓ teja꜕sā

\clearpage

\chapter{Recuerdo del Dhamma}

\begin{leader}
  [Cantemos ahora el Recuerdo del Dhamma.]
\end{leader}


El Dhamma fué bien explicado por el Excelso,\\
Presente aquí y ahora, intemporal, incentivando a investigar,\\
Guiando al interior, para ser experimentado individualmente\\
\vin por los sabios.


\nextChapterUseDelegatedPageNumber

\chapter{Elogio Supremo al Dhamma}

\begin{leader}
  [Cantemos ahora el elogio supremo al Dhamma.]
\end{leader}

\enlargethispage{\baselineskip}

Es excelente porque está bien explicado,\\
Y puede ser dividido en Camino y Fruto, Aprendizaje y Liberación.\\
El Dhamma protege aquellos que lo mantienen, de caer en la ilusión.\\
Yo reverencio la excelente Enseñanza,\\
\vin que remueve la oscuridad ---\\
El Dhamma es el refugio seguro y supremo de todos los seres ---\\
Como segundo objecto de Recuerdo, yo lo venero postrándome.\\
Soy de hecho esclavo del Dhamma, el Dhamma es mi maestro y guía.\\
El Dhamma es destructor de tristeza, y proyecta bendiciones sobre mí.\\
Al Dhamma dedico este cuerpo y vida,\\
Y en devoción caminaré este excelente Camino de Verdad.\\
Para mí no existe otro refugio,\\
\vin El Dhamma es mi excelente refugio.\\
Por la afirmación de esta Verdad,\\
\vin que yo crezca en el camino del Maestro.\\
Por mi devoción al Dhamma, y por el mérito de esta práctica ---\\
Por su poder, que todos los obstáculos sean vencidos.

\clearpage

\begin{instruction}
  Reverencia
\end{instruction}

Kāyena vācāya va ceta꜕sā꜓ vā\\
Dha꜓mme ku꜕kammaṁ pa꜕kataṁ ma꜕yā yaṁ\\
Dha꜓mmo pa꜕ṭiggaṇhā꜕tu acca꜕yantaṁ\\
Kālantare sa꜓ṁvarituṁ va꜕ dha꜓mme

\chapter*{Recuerdo de la Sangha}

\delegateSetUseNext

\begin{leader}
  [Ha꜓nda mayaṁ saṅghānu꜕ssa꜕ti꜕nayaṁ karomase]
\end{leader}

Supaṭi꜕panno bha꜕gava꜕to sāvaka꜕saṅgho\\
Ujupaṭi꜕panno bha꜕gava꜕to sāvaka꜕saṅgho\\
Ñāyapaṭi꜕panno bha꜕gava꜕to sāvaka꜕saṅgho\\
Sā꜓mīci꜕pa꜕ṭi꜕panno bha꜕gava꜕to sāvaka꜕saṅgho\\
Yadidaṁ cattāri purisa꜕yugāni aṭṭha꜓ purisa꜕pugga꜕lā\\
Esa bha꜕gava꜕to sāvaka꜕saṅgho\\
Āhu꜕neyyo pāhu꜕neyyo dakkhi꜕ṇeyyo añja꜕li-ka꜕ra꜓ṇīyo\\
Anu꜓tta꜕raṁ puññakkhe꜕ttaṁ lokassā'ti

\clearpage

\begin{instruction}
  Reverencia
\end{instruction}

Ya sea a través del cuerpo, de palabra o de  mente,\\
Por cualquier acción erronea que haya cometido\\
\vin hacia el Dhamma,\\
Que mi reconocimento de esa falta sea aceptado,\\
De forma que en el futuro modere mi comportamiento\\
\vin hacia el Dhamma.

\chapter{Recuerdo de la Sangha}

\begin{leader}
  [Cantemos ahora el Recuerdo de la Saṅgha.]
\end{leader}

Son los discípulos del Excelso que practicaron correctamente,\\
Que practicaron directamente,\\
Que practicaron con reflexión,\\
Aquellos que practicaron con integridad ---\\
Es decir, los cuatro pares, los ocho tipos de Seres Nobles ---\\
Estos son los discípulos del Señor.\\
Tales discípulos son merecedores de presentes,\\
\vin merecedores de hospitalidad, merecedores de ofrendas,\\
\vin merecedores de respeto;\\
Ellos promueven el surgimiento de un bien incomparable en el mundo.

\clearpage

\chapter*{Elogio Supremo a la Sangha}

\begin{leader}
  [Ha꜓nda mayaṁ saṅghābhi꜕gī꜕tiṁ karomase]
\end{leader}

Sa꜕ddhammajo supaṭipatti꜕-gu꜓ṇādi꜕yutto\\
Yo'ṭṭhabbi꜕dho ari꜓yapugga꜕la꜓-saṅgha꜕-seṭṭho\\
Sī꜓lādi꜕dhamma-pa꜕varāsa꜕ya꜓-kāya꜕-citto\\
Vandām'aha꜓ṁ ta꜕m-ari꜕yāna꜕-gaṇa꜓ṁ su꜕suddhaṁ\\
Sa꜓ṅgho yo sabba꜕-pāṇīnaṁ sa꜕raṇaṁ khema꜕m-utta꜕maṁ\\
Ta꜕tiyānussa꜕tiṭṭhānaṁ vandāmi꜕ taṁ si꜓ren'a꜕haṁ\\
Saṅghass'ā꜓ha꜕smi dāso/dāsī va saṅgho me sā꜕mi-ki꜓ssaro\\
Sa꜓ṅgho dukkhassa꜕ ghātā ca꜕ vi꜓dhātā ca꜕ hi꜓tassa꜕ me\\
Saṅghass'āha꜓ṁ niyyādemi sa꜕rīrañ-jīvi꜕tañ-ci꜕daṁ\\
Vandanto'ha꜓ṁ/Vandantī'ha꜓ṁ ca꜕rissāmi saṅghassopa꜕ṭi꜓panna꜕taṁ\\
Natthi me sa꜕ra꜓ṇaṁ aññaṁ saṅgho me sa꜕ra꜓ṇaṁ va꜕raṁ\\
Etena sacca꜕-vajjena vaḍḍheyyaṁ sa꜕tthu-sā꜓sane\\
Sa꜓ṅghaṁ me vanda꜕mānena/vanda꜕mānāya\\
\vin yaṁ puññaṁ pa꜕su꜓taṁ i꜕dha\\
Sa꜕bbepi anta꜕rāyā me māhe꜓su꜓ṁ ta꜕ssa꜓ teja꜕sā

\enlargethispage{\baselineskip}

\vfill

\begin{instruction}
  Reverencia
\end{instruction}

Kāyena vācāya va ceta꜕sā꜓ vā\\
Sa꜓ṅghe ku꜕kammaṁ pa꜕kataṁ ma꜕yā yaṁ\\
Sa꜓ṅgho pa꜕ṭiggaṇhā꜕tu acca꜕yantaṁ\\
Kālantare sa꜓ṁvarituṁ va꜕ sa꜓ṅghe

\vfill

\begin{instruction}
  A partir de ahora meditación es practicada en silencio, a veces seguida de una enseñanza de Dhamma, y terminando con lo siguiente:
\end{instruction}

\clearpage

\nextChapterUseDelegatedPageNumber

\chapter{Elogio Supremo a la Sangha}

\begin{leader}
  [Cantemos ahora el elogio supremo a la Saṅgha.]
\end{leader}

Nacida del Dhamma, esa Saṅgha que practicó bien,\\
El campo de la Saṅgha formado por ocho tipos de Seres Nobles,\\
Guiados en cuerpo y mente por excelente moralidad y virtud.\\
Yo reverencio esa asamblea de Seres Nobles,\\
\vin perfeccionados en pureza.\\
La Saṅgha, que es el refugio seguro y supremo de todos los seres ---\\
Como tercer objecto de Recuerdo, yo lo venero postrándome.\\
Soy de hecho esclavo de la Saṅgha, La Saṅgha es mi maestro y guía.\\
La Saṅgha es destructor de tristeza, que proyecta bendiciones sobre mí.\\
A la Saṅgha dedico este cuerpo y vida,\\
Y en devoción caminaré, el Camino bien practicado de la Saṅgha.\\
Para mí no existe otro refugio, la Saṅgha es mi excelente refugio.\\
Por la afirmación de esta Verdad,\\
\vin que yo crezca en el camino del Maestro.\\
Por mi devoción a la Saṅgha, y por el mérito de esta práctica ---\\
Por su poder, que todos los obstáculos sean vencidos.

\enlargethispage{2\baselineskip}

\vfill

\begin{instruction}
  Reverencia
\end{instruction}

Ya sea a través del cuerpo, de palabra o de mente,\\
Por cualquier acción erronea que haya cometido\\
\vin hacia la Saṅgha,\\
Que mi reconocimento de esa falta sea aceptado,\\
De forma que en el futuro modere mi comportamiento hacia la Saṅgha.

\begin{instruction}
  A partir de ahora meditación es practicada en silencio, a veces seguida de una enseñanza de Dhamma, y terminando con lo siguiente:
\end{instruction}

\clearpage

\chapter*{Homenaje de Cierre}

\delegateSetUseNext

[Arahaṁ] sammāsambuddho bha꜕gavā\\
Buddhaṁ bha꜕gavantaṁ a꜕bhi꜓vādemi \instr{Reverencia}

[Svākkhā꜓to] bha꜕gava꜓tā dhammo\\
Dhammaṁ namassāmi \instr{Reverencia}

[Supaṭi꜕panno] bha꜕gava꜕to sāvaka꜕saṅgho\\
Sa꜓ṅghaṁ na꜕māmi \instr{Reverencia}

\clearpage

\chapter{Homenaje de Cierre}

[Al Maestro,] el Excelso y perfectamente Iluminado \\
Al Buda, el Excelso, le rindo homenaje. \instr{Reverencia}

[A las Enseñanzas,] explicadas tan bien por Él \\
Al Dhamma, le rindo homenaje. \instr{Reverencia}

[A los discípulos del Excelso,] que practicaron tan bien ---\\
A la Sangha, le rindo homenaje. \instr{Reverencia}
