
\thispagestyle{empty}

\newlength\ackWidth
\ifaivedition
\setlength{\ackWidth}{0.77\linewidth}
\else
\setlength{\ackWidth}{0.8\linewidth}
\fi

{\centering

\ifaivedition
\vspace*{9\baselineskip}
\else
\vspace*{6\baselineskip}
\fi

\begin{minipage}{\ackWidth}
\setlength{\parskip}{8pt}

{\instructionFont\color{instruction}Cuidado de los libros de Dhamma:}

\bigskip

Los libros de Dhamma contienen las enseñanzas de Buddha y señalan el camino hacia la liberación de saṁsara. Por lo tanto, deben tratarse con respeto: deben mantenerse fuera del suelo y de lugares donde la gente se sienta o camina, y no deben pasarse por encima. Deben cubrirse o protegerse para su transporte y guardarse en un lugar alto y limpio, separado de materiales más ‘mundanos’. No se deben colocar otros objetos encima de los libros y materiales de Dhamma. Humedecerse los dedos para pasar las páginas se considera de mala educación. Si es necesario deshacerse de materiales de Dhamma, en lugar de tirarlos a la basura, deben quemarse, con sati y una actitud reverencial.

\end{minipage}

}

