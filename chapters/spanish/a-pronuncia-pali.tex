\chapter*{Pronunciación en Pāli}
\label{pron-pali}

\subsection{Las Vocales}

\begin{tabular}{@{} L{5mm} L{\linewidth-15mm}}
\emph{a} & Como el `a' en la palabra inglesa \emph{sug\prul{a}r} o como el `u' en la palabra inglesa \emph{b\prul{u}t}.\\

\emph{ā} & Larga, como en \emph{\prul{á}gua}.\\

\emph{e} & Como en \emph{p\prul{e}ra, m\prul{e}sa, m\prul{e}s}.\\

\emph{i} & Corta, como en \emph{v\prul{i}da, l\prul{i}bro, h\prul{i}jo}.\\

\emph{ī} & Larga, como en \emph{Mar\prul{í}a, polic\prul{í}a}.\\

\emph{o} & Como en \emph{b\prul{o}la, fl\prul{o}r}.\\

\emph{u} & Corta, como en \emph{p\prul{u}reza, m\prul{u}ndo}.\\

\emph{ū} & Larga, como en \emph{m\prul{ú}sica, d\prul{u}da}.\\
\end{tabular}

\subsection{Las Consonantes}

\enlargethispage{\baselineskip}

\begin{tabular}{@{} L{5mm} L{\linewidth-10mm}}
\emph{c} & Es pronunciada como \emph{tch} en \emph{hecho} o \emph{chantaje}, nunca como \emph{c} en \emph{caballo} o \emph{cuando}.\\

\emph{g} & Como en \emph{gusano} o \emph{gato}.\\

\emph{h} & Ya sea colocado inmediatamente después de consonantes o después de consonantes dobles, la \emph{h} es siempre aspirada como un soplo en suspiro gutural, típico de la lengua inglesa;  como en el inglés:\\
\end{tabular}

\begin{tabular}{@{} L{5mm} L{5mm} L{\linewidth-10mm}}
& \emph{bh} & \emph{ca\prul{bh}orse}\\

& \emph{ch} & \emph{ran\prul{ch-h}ouse}\\

& \emph{dh} & \emph{han\prul{dh}old}\\

& \emph{gh} & \emph{ba\prul{g-h}andle}\\

& \emph{jh} & \emph{sle\prul{dgeh}ammer}\\
\end{tabular}

\begin{tabular}{@{} L{5mm} L{\linewidth-15mm}}
\emph{j} & No como en \emph{jarra}, sino como en la palabra inglesa \emph{\prul{j}oy}.\\

\emph{ṁ} & El denominado ‘nasal’ es como \emph{e\prul{ng}añar}, o \emph{so\prul{ng}} en inglés.\\

\emph{s} & Siempre como en \emph{\prul{s}ublimar} o en \emph{ca\prul{s}a}.\\

\emph{ñ} & Como la \emph{ñ} normal en la lengua española; ex: \emph{Espa\prul{ñ}a}.\\

\emph{ph} & Como la \emph{p} seguida de suspiro gutural en las palabras inglesas \emph{ha\prul{ph}azard, a\prul{pp}ear}.\\

\emph{th} & Como la \emph{t} seguida de suspiro gutural en las palabras \emph{\prul{th}at, \prul{th}ese}.\\

\emph{y} & Como la \emph{y} en \emph{ma\prul{y}o, pla\prul{y}a} o la \emph{i} en \emph{\prul{i}nglés}.\\
\end{tabular}

\bigskip

{\raggedright

\emph{ṭ, ṭh, ḍ, ḍh:} Son sonidos producidos con la lengua, denominados cerebrales; al pronunciarlos se debe presionar la lengua contra el paladar.

\emph{Consonantes dobles:} Cada una debe ser pronunciada, como \emph{bb} en el término inglés \emph{subbase}.

}
