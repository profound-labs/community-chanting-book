\chapter*{Pronunciación en Pāli}
\label{pron-pali}

\subsection{Las Vocales}

\begin{tabular}{@{} L{5mm} L{\linewidth-15mm}}
\emph{a} & Como el `a' en la palabra inglesa \emph{sug\prul{a}r} o como el `u' en la palabra inglesa \emph{b\prul{u}t}.\\

\emph{ā} & Larga, como en \emph{\prul{á}gua}.\\

\emph{e} & Como en \emph{p\prul{e}ra, m\prul{e}sa, m\prul{e}s}.\\

\emph{i} & Corta, como en \emph{v\prul{i}da, l\prul{i}bro, h\prul{i}jo}.\\

\emph{ī} & Larga, como en \emph{Mar\prul{í}a, polic\prul{í}a}.\\

\emph{o} & Como en \emph{b\prul{o}la, fl\prul{o}r}.\\

\emph{u} & Corta, como en \emph{p\prul{u}reza, m\prul{u}ndo}.\\

\emph{ū} & Larga, como en \emph{m\prul{ú}sica, d\prul{u}da}.\\
\end{tabular}

\subsection{Las Consonantes}

\enlargethispage{\baselineskip}

\begin{tabular}{@{} L{5mm} L{\linewidth-10mm}}
\emph{c} & Es pronunciada como \emph{tch} en \emph{hecho} o \emph{chantaje}, nunca como \emph{c} en \emph{caballo} o \emph{cuando}.\\

\emph{g} & Como en \emph{gusano} o \emph{gato}.\\

\emph{h} & Ya sea colocada inmediatamente después de consonantes o después de consonantes dobles, la \emph{h} es siempre aspirada como un soplo en suspiro gutural, típico de la lengua inglesa;  como en el inglés:\\
\end{tabular}

\begin{tabular}{@{} L{5mm} L{5mm} L{\linewidth-10mm}}
& \emph{bh} & \emph{ca\prul{bh}orse}\\

& \emph{ch} & \emph{ran\prul{ch-h}ouse}\\

& \emph{dh} & \emph{han\prul{dh}old}\\

& \emph{gh} & \emph{ba\prul{g-h}andle}\\

& \emph{jh} & \emph{sle\prul{dgeh}ammer}\\
\end{tabular}

\begin{tabular}{@{} L{5mm} L{\linewidth-15mm}}
\emph{j} & No como en \emph{jarra}, sino como en la palabra inglesa \emph{\prul{j}oy}.\\

\emph{ṁ} & El denominado ‘nasal’ es como \emph{e\prul{ng}añar}, o \emph{so\prul{ng}} en inglés.\\

\emph{s} & Siempre como en \emph{\prul{s}ublimar} o en \emph{ca\prul{s}a}.\\

\emph{ñ} & Como la \emph{ñ} normal en la lengua española; ex: \emph{Espa\prul{ñ}a}.\\

\emph{ph} & Como la \emph{p} seguida de suspiro gutural en las palabras inglesas \emph{ha\prul{ph}azard, a\prul{pp}ear}.\\

\emph{th} & Como la \emph{t} seguida de suspiro gutural en las palabras \emph{\prul{th}at, \prul{th}ese}.\\

\emph{y} & Como la \emph{y} en \emph{ma\prul{y}o, pla\prul{y}a} o la \emph{i} en \emph{\prul{i}nglés}.\\
\end{tabular}

\bigskip

{\raggedright

\emph{ṭ, ṭh, ḍ, ḍh:} Son sonidos producidos con la lengua, denominados palatales; al pronunciarlos se debe presionar la lengua contra el paladar.

\emph{Consonantes dobles:} Cada una debe ser pronunciada, como \emph{bb} en el término inglés \emph{subbase}.

}

\section{Técnica de Cánticos}

Una vez que haya usted comprendido el sistema de pronunciación pāli y la siguiente técnica de canto, podrá cantar cualquier texto en pāli con el ritmo correcto.

\textbf{Sílabas cortas} terminan en \textbf{a, i} corta o
\textbf{u} corta. Todas las demás sílabas son largas.
Sílabas largas toman el doble de tiempo que las sílabas cortas --- como dos compases en una barra musical (una blanca) en vez de un compas (una negra). Esto es lo que da al cántico su ritmo particular.

\begin{centering}
	
	{\setlength{\tabcolsep}{1.8pt}%
		\begin{tabular}{ccc c ccccc c ccccc c ccccccc}
			BUD & · & DHO & \hsp & SU & · & SUD & · & DHO & \hsp & KA & · & RU & · & ṆĀ & \hsp & MA & · & HAṆ & · & ṆA & · & VO\\
			1  &   & 1   &      & ½  &   & 1   &   & 1   &      & ½  &   & ½  &   & 1  &      & ½  &   & 1   &   & ½  &   & 1\\
		\end{tabular}%
	}
	
\end{centering}

Dos detalles importantes a tener en cuenta cuendo se separan las sílabas:

\textbf{1.} Sílabas con dobles consonantes quedan divididas de esta manera:

\begin{centering}
	
	\begin{minipage}{0.8\linewidth}
		\begin{multicols}{2}
			\setlength{\tabcolsep}{1.8pt}%
			
			\begin{tabular}{rrcccl}
				& A & · & NIC & · & CA   \\
				& ½ &   &  1  &   & ½    \\
				(not & A & · & NI  & · & CCA) \\
				& ½ &   & ½   &   & ½    \\
			\end{tabular}
			
			\columnbreak
			
			\begin{tabular}{rrcccl}
				& PUG & · & GA  & · & LĀ \\
				&  1  &   &  ½  &   &  1 \\
				(not & PU  & · & GGA & · & LĀ)\\
				&  ½  &   &  ½  &   &  1 \\
			\end{tabular}
			
		\end{multicols}
	\end{minipage}
	
\end{centering}

They are always enunciated separately, e.g. \textbf{dd} in ‘uddeso’ as
in ‘mad dog’, or \textbf{gg} in ‘maggo’ as in ‘big gun’.

\textbf{2.} \textbf{Aspirated consonants} like \textbf{bh, dh} etc.
count as single consonant and don't get divided (Therefore
\textbf{am·hā·kaṁ}, but \textbf{sa·dham·maṁ}, not \textbf{sad·ham·maṁ}
or, another example: \textbf{Bud·dho} and not \textbf{Bu·ddho}).

Precise pronunciation and correct separation of the syllables is
especially important when someone is interested in learning Pāli and to
understand and memorize the meaning of Suttas and other chants,
otherwise the meaning of it will get distorted.

\textbf{An example to illustrate this:}

The Pāli word ‘\textbf{sukka}’ means ‘bright’; ‘\textbf{sukkha}’ means
‘dry’; ‘\textbf{sukha}’ --- ‘happiness’; ‘\textbf{suka}’ --- ‘parrot’ and
‘\textbf{sūka}’ --- ‘bristles on an ear of barley’.

So if you chant ‘\textbf{sukha}’ with a ‘\textbf{k}’ instead of a
‘\textbf{kh}’, you would chant ‘parrot’ instead of ‘happiness’.

A general rule of thumb for understanding the practice of chanting is to
listen carefully to what the leader and the group are chanting and to
follow, keeping the same pitch, tempo and speed. All voices should blend
together as one.

\section{Punctuation, tonal marks and pauses in this edition}

[Square brackets] indicate parts usually chanted only by the leader, but
chanting customs differ in the various monasteries.

The slash / indicates variations of male of female forms according to
the person chanting them, or singular and plural forms when chanting
alone or in a group.

The cantillation marks indicate changes in pitch, usually a full tone up or down:

\begin{tabular}{llll}
	High tone: & no꜓ble & Long low tone: & ho꜖mage\\
	Low tone: & ble꜕ssed & Long mid tone: & \prul{guides}\\
\end{tabular}

\section{A note on hyphenation in the text}

As an aid to understanding, some of the longer Pāli words in the text have been
hyphenated into the words from which they are compounded. This does not affect
the pronunciation in any way.
