\chapter*{Pronunciación en Pāli}
\label{pron-pali}

\subsection{Las Vocales}

\begin{tabular}{@{} L{5mm} L{\linewidth-15mm}}
\emph{a} & Como el `a' en la palabra inglesa \emph{sug\prul{a}r} o como el `u' en la palabra inglesa \emph{b\prul{u}t}.\\

\emph{ā} & Larga, como en \emph{\prul{á}gua}.\\

\emph{e} & Como en \emph{p\prul{e}ra, m\prul{e}sa, m\prul{e}s}.\\

\emph{i} & Corta, como en \emph{v\prul{i}da, l\prul{i}bro, h\prul{i}jo}.\\

\emph{ī} & Larga, como en \emph{Mar\prul{í}a, polic\prul{í}a}.\\

\emph{o} & Como en \emph{b\prul{o}la, fl\prul{o}r}.\\

\emph{u} & Corta, como en \emph{p\prul{u}reza, m\prul{u}ndo}.\\

\emph{ū} & Larga, como en \emph{m\prul{ú}sica, d\prul{u}da}.\\
\end{tabular}

\subsection{Las Consonantes}

\enlargethispage{\baselineskip}

\begin{tabular}{@{} L{5mm} L{\linewidth-10mm}}
\emph{c} & Es pronunciada como \emph{tch} en \emph{hecho} o \emph{chantaje}, nunca como \emph{c} en \emph{caballo} o \emph{cuando}.\\

\emph{g} & Como en \emph{gusano} o \emph{gato}.\\

\emph{h} & Ya sea colocada inmediatamente después de consonantes o después de consonantes dobles, la \emph{h} es siempre aspirada como un soplo en suspiro gutural, típico de la lengua inglesa;  como en el inglés:\\
\end{tabular}

\begin{tabular}{@{} L{5mm} L{5mm} L{\linewidth-10mm}}
& \emph{bh} & \emph{ca\prul{bh}orse}\\

& \emph{ch} & \emph{ran\prul{ch-h}ouse}\\

& \emph{dh} & \emph{han\prul{dh}old}\\

& \emph{gh} & \emph{ba\prul{g-h}andle}\\

& \emph{jh} & \emph{sle\prul{dgeh}ammer}\\
\end{tabular}

\begin{tabular}{@{} L{5mm} L{\linewidth-15mm}}
\emph{j} & No como en \emph{jarra}, sino como en la palabra inglesa \emph{\prul{j}oy}.\\

\emph{ṁ} & El denominado ‘nasal’ es como \emph{e\prul{ng}añar}, o \emph{so\prul{ng}} en inglés.\\

\emph{s} & Siempre como en \emph{\prul{s}ublimar} o en \emph{ca\prul{s}a}.\\

\emph{ñ} & Como la \emph{ñ} normal en la lengua española; ex: \emph{Espa\prul{ñ}a}.\\

\emph{ph} & Como la \emph{p} seguida de suspiro gutural en las palabras inglesas \emph{ha\prul{ph}azard, a\prul{pp}ear}.\\

\emph{th} & Como la \emph{t} seguida de suspiro gutural en las palabras \emph{\prul{th}at, \prul{th}ese}.\\

\emph{y} & Como la \emph{y} en \emph{ma\prul{y}o, pla\prul{y}a} o la \emph{i} en \emph{\prul{i}nglés}.\\
\end{tabular}

\bigskip

{\raggedright

\emph{ṭ, ṭh, ḍ, ḍh:} Son sonidos producidos con la lengua, denominados palatales; al pronunciarlos se debe presionar la lengua contra el paladar.

\emph{Consonantes dobles:} Cada una debe ser pronunciada, como \emph{bb} en el término inglés \emph{subbase}.

}

\section{Técnica de Cánticos}

Una vez que haya usted comprendido el sistema de pronunciación pāli y la siguiente técnica de canto, podrá cantar cualquier texto en pāli con el ritmo correcto.

\textbf{Sílabas cortas} terminan en \textbf{a, i} corta o
\textbf{u} corta. Todas las demás sílabas son largas.
Sílabas largas toman el doble de tiempo que las sílabas cortas --- como dos tiempos dentro de un compás (una blanca) en vez de uno (una negra). Esto es lo que da al cántico su ritmo particular.

\begin{centering}
	
	{\setlength{\tabcolsep}{1.8pt}%
		\begin{tabular}{ccc c ccccc c ccccc c ccccccc}
			BUD & · & DHO & \hsp & SU & · & SUD & · & DHO & \hsp & KA & · & RU & · & ṆĀ & \hsp & MA & · & HAṆ & · & ṆA & · & VO\\
			1  &   & 1   &      & ½  &   & 1   &   & 1   &      & ½  &   & ½  &   & 1  &      & ½  &   & 1   &   & ½  &   & 1\\
		\end{tabular}%
	}
	
\end{centering}

Dos detalles importantes a tener en cuenta cuendo se separan las sílabas:

\textbf{1.} Sílabas con dobles consonantes quedan divididas de esta manera:

\begin{centering}
	
	\begin{minipage}{0.8\linewidth}
		\begin{multicols}{2}
			\setlength{\tabcolsep}{1.8pt}%
			
			\begin{tabular}{rrcccl}
				& A & · & NIC & · & CA   \\
				& ½ &   &  1  &   & ½    \\
				(not & A & · & NI  & · & CCA) \\
				& ½ &   & ½   &   & ½    \\
			\end{tabular}
			
			\columnbreak
			
			\begin{tabular}{rrcccl}
				& PUG & · & GA  & · & LĀ \\
				&  1  &   &  ½  &   &  1 \\
				(not & PU  & · & GGA & · & LĀ)\\
				&  ½  &   &  ½  &   &  1 \\
			\end{tabular}
			
		\end{multicols}
	\end{minipage}
	
\end{centering}

Siempre se pronuncian por separado, e.g. \textbf{dd} en ‘uddeso’ como en
 ‘sed-de’, u otros ejemplos: 'pan-negro', 'sal-limpio', 'hay-yerba', 'dad-dinero', 'más-sabor'.

\textbf{2.} \textbf{Consonantes aspiradas} como \textbf{bh, dh} etc.
cuentan como una sola consonante y no son divididas (Así
\textbf{am·hā·kaṁ}, pero \textbf{sa·dham·maṁ}, no \textbf{sad·ham·maṁ}
u otro ejemplo: \textbf{Bud·dho} y no \textbf{Bu·ddho}).

La pronunciación precisa y la correcta separación de las sílabas son especialmente importantes cuando alguien está interesado en aprender pāli y comprender y memorizar el significado de los suttas y otros cantos; de lo contrario, su significado se distorsionará.


\textbf{Un ejemplo para ilustrar esto:}

La palabra Pāli ‘\textbf{sukka}’ significa ‘brillante’; ‘\textbf{sukkha}’ significa
‘seco’; ‘\textbf{sukha}’ --- ‘felicidad’; ‘\textbf{suka}’ --- ‘loro’ and
‘\textbf{sūka}’ --- ‘una espiga de cebada'.
Así, si se canta ‘\textbf{sukha}’ con ‘\textbf{k}’ en lugar de
‘\textbf{kh}’, cantar'ia usted ‘loro’ en lugar de ‘felicidad’.

Una regla general para comprender la práctica del canto es escuchar atentamente lo que canta el líder y el grupo y seguirlo, manteniendo el mismo tono, tempo y velocidad. Todas las voces deben integrarse en una sola.

\section{Puntuación, marcas tonales y pausas en esta edición}

[Los corchetes] indican partes que suelen ser cantadas solo por el líder, pero las costumbres de canto difieren entre los monasterios.

La barra / indica variaciones de las formas masculinas o femeninas según quien las cante, o formas singulares y plurales al cantar solo o en grupo.

Las marcas de tonales indican cambios de tono, generalmente un tono completo hacia arriba o hacia abajo.

\begin{tabular}{llll}
	Tono alto: & no꜓ble \\
	Tono bajo: & excel꜕so \\
\end{tabular}

%\section{A note on hyphenation in the text}
%
%As an aid to understanding, some of the longer Pāli words in the text have been
%hyphenated into the words from which they are compounded. This does not affect
%the pronunciation in any way.
