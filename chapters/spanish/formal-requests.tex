\setlength{\englishIndent}{0pt}

\chapter{Añjali}

Los cánticos y peticiones formales se hacen con las manos en añjali.
Este es un gesto de respeto, ejecutado colocando las palmas de las manos juntas
directamente a la altura del pecho, con los dedos apuntando hacia arriba.

\chapter{Solicitando una Enseñanza de Dhamma}

\begin{instruction}
  Después de hacer tres reverencias, con las manos unidas en añjali, recitar lo seguiente:
\end{instruction}

Brahmā ca꜕ lokādhipa꜕tī sa꜕hampa꜕ti\\
Ka꜕tañja꜕lī a꜕nadhiva꜕raṁ a꜕yāca꜕tha

Santī꜓dha sa꜕ttāppa꜕ra꜕jakkha꜕-jātikā\\
Desetu꜕ dhammaṁ a꜕nu꜕kampi꜕maṁ pa꜕jaṁ

\begin{instruction}
  Hacer tres reverencias de nuevo.
\end{instruction}

\begin{english}
Brahmā Sahampati, el Señor del mundo,\\
con las palmas de las manos unidas en reverencia, pidió un favor:

`Hay seres con apenas `un poco de polvo en los ojos'.\\
Por favor, por compasión, enseñeles el Dhamma.'
\end{english}

\chapter{Reconocimiento de una Enseñanza}

\enlargethispage{2\baselineskip}

\begin{tabular}{@{} ll @{}}
Una persona: & Ha꜓nda mayaṁ dhammakathā꜓ya sā꜓dhukā꜕raṁ dadāmase \\
& \hspace*{1em}\tr{(Expresemos ahora nuestra aprobación}\\
& \hspace*{1em}\tr{por este discurso de Dhamma.)}\\
Respuesta: & Sādhu, sādhu, sādhu, anu꜓modāmi \\
& \hspace*{1em}\tr{(Bien bien bien, yo también me regocijo en el bien creado)} \\
\end{tabular}

\clearpage
\chapter{Petición de Cántico de Parittas}

\begin{instruction}
  Después de hacer reverencia tres veces, con las manos unidas en añjali, recitar lo siguiente:
\end{instruction}

Vipatti-paṭibāhā꜓ya sabba꜕-sampatti꜕-siddhi꜕yā\\
Sabbadukkha-vināsā꜓ya\\
Parittaṁ brūtha꜕ maṅga꜕laṁ

Vipatti-paṭibāhā꜓ya sabba꜕-sampatti꜕-siddhi꜕yā\\
Sabbabhaya-vināsā꜓ya\\
Parittaṁ brūtha꜕ maṅga꜕laṁ

Vipatti-paṭibāhā꜓ya sabba꜕-sampatti꜕-siddhi꜕yā\\
Sabbaroga-vināsā꜓ya\\
Parittaṁ brūtha꜕ maṅga꜕laṁ

\begin{instruction}
  Reverencia tres veces
\end{instruction}

\begin{english}
Para repeler el infortunio, para el surgimiento de buena fortuna,\\
para el desvanecimento de todo dukkha,\\
por favor canten una bendición y protección.

Para repeler el infortunio, para el surgimiento de buena fortuna,\\
para disipar todo miedo,\\
por favor canten una bendición y protección.

Para repeler el infortunio, para el surgimiento de buena fortuna,\\
para alejar toda enfermedad,\\
por favor canten una bendición y protección.
\end{english}

\setlength{\englishIndent}{\leaderIndent}

\clearpage
\chapter[Tres Refugios y Cinco Preceptos]{Solicitud de los Tres Refugios\newline y Cinco Preceptos}

\begin{instruction}
  Después de hacer tres reverencias, con las palmas\\
  de las manos unidas en añjali, se recita la petición:
\end{instruction}

\subsection{En grupo}

Mayaṁ bhante tisaraṇena sa꜕ha pañca sī꜓lāni yā꜕cāma\\
Dutiyampi mayaṁ bhante tisaraṇena sa꜕ha pañca sī꜓lāni yā꜕cāma\\
Tatiyampi mayaṁ bhante tisaraṇena sa꜕ha pañca sī꜓lāni yā꜕cāma

\subsection{Individualmente}

Ahaṁ bhante tisaraṇena sa꜕ha pañca sī꜓lāni yā꜕cāmi\\
Dutiyampi ahaṁ bhante tisaraṇena sa꜕ha pañca sī꜓lāni yā꜕cāmi\\
Tatiyampi ahaṁ bhante tisaraṇena sa꜕ha pañca sī꜓lāni yā꜕cāmi

\subsection{Traducción}

\begin{english}
  Nosotros/Yo, Venerable Señor,\\
  \vin solicitamos/solicito los Tres Refugios y los Cinco Preceptos.
  
  Por segunda vez, Nosotros/Yo, Venerable Señor,\\
  \vin solicitamos/solicito los Tres Refugios y los Cinco Preceptos.
  
  Por tercera vez, Nosotros/Yo, Venerable Señor,\\
  \vin solicitamos/solicito los Tres Refugios y los Cinco Preceptos.
\end{english}

\clearpage
\chapter{Los Tres Refugios}

\begin{instruction}
  Repetir, después de que el lider haya\\
  cantado las primeras tres lineas
\end{instruction}

Namo tassa bhagavato arahato sammāsambuddhassa\\
Namo tassa bhagavato arahato sammāsambuddhassa\\
Namo tassa bhagavato arahato sammāsambuddhassa

\begin{english}
 (Homenaje al Excelso, Noble y Perfectamente Iluminado.)\\
\end{english}

Buddhaṁ saraṇaṁ gacchāmi\\
Dhammaṁ saraṇaṁ gacchāmi\\
Saṅghaṁ saraṇaṁ gacchāmi

\begin{english}
  Voy al Buda en busca de refugio.\\
  Voy al Dhamma en busca de refugio.\\
  Voy a la Sangha en busca de refugio.
\end{english}

Dutiyampi buddhaṁ saraṇaṁ gacchāmi\\
Dutiyampi dhammaṁ saraṇaṁ gacchāmi\\
Dutiyampi saṅghaṁ saraṇaṁ gacchāmi

\begin{english}
  Por segunda vez, Voy al Buda en busca de refugio.\\
  Por segunda vez, Voy al Dhamma en busca de refugio.\\
  Por segunda vez, Voy a la Sangha en busca de refugio.
\end{english}

Tatiyampi buddhaṁ saraṇaṁ gacchāmi\\
Tatiyampi dhammaṁ saraṇaṁ gacchāmi\\
Tatiyampi saṅghaṁ saraṇaṁ gacchāmi

\clearpage

\begin{english}
  Por tercera vez, Voy al Buda en busca de refugio.\\
  Por tercera vez, Voy al Dhamma en busca de refugio.\\
  Por tercera vez, Voy a la Sangha en busca de refugio.
\end{english}

\begin{instruction}
  Líder:
\end{instruction}

[Tisaraṇa-gamanaṁ niṭṭhitaṁ]

\begin{english}
  (Queda así completo el Triple Refugio.)
\end{english}

\begin{instruction}
  Respuesta:
\end{instruction}

Āma bhante

\begin{english}
  (Sí, Venerable Maestro.)
\end{english}

\chapter{Los Cinco Preceptos}

\begin{instruction}
  Repetir cada precepto despues del líder
\end{instruction}

\begin{precept}
  \setcounter{enumi}{0}
  \item Pāṇātipātā vera꜓maṇī sikkhā꜓padaṁ sa꜓mādi꜕yāmi
\end{precept}

\begin{english}
  Tomo el entrenamiento de abstenerme de quitar la vida a cualquier criatura viviente.
\end{english}

\begin{precept}
  \setcounter{enumi}{1}
  \item Adinnādānā vera꜓maṇī sikkhā꜓padaṁ sa꜓mādi꜕yāmi
\end{precept}

\begin{english}
  Tomo el entrenamiento de absternerme de tomar lo que no me es dado.
\end{english}

\begin{precept}
  \setcounter{enumi}{2}
  \item Kāmesu micchā꜓cārā vera꜓maṇī sikkhā꜓padaṁ sa꜓mādi꜕yāmi
\end{precept}

\begin{english}
  Tomo el entrenamiento de abstenerme de toda conducta sexual inapropiada.
\end{english}

\begin{precept}
  \setcounter{enumi}{3}
  \item Musā꜓vādā vera꜓maṇī sikkhā꜓padaṁ sa꜓mādi꜕yāmi
\end{precept}

\enlargethispage{\baselineskip}

\begin{english}
  Tomo el entrenamiento de abstenerme de mentir.
\end{english}

\clearpage

\begin{precept}
  \setcounter{enumi}{4}
  \item Surāmeraya-majja-pamādaṭṭhā꜓nā vera꜓maṇī sikkhā꜓padaṁ sa꜓mādi꜕yāmi
\end{precept}

\begin{english}
  Tomo el entrenamiento de abstenerme de consumir bebidas embriagantes\\
  y drogas que conduzcan al descuido.
\end{english}

\begin{instruction}
  Líder:
\end{instruction}

[Imāni pañca sikkhā꜓padāni\\
Sī꜓lena suga꜕tiṁ yanti\\
Sī꜓lena bhoga꜕sa꜓mpadā\\
Sī꜓lena nibbu꜕tiṁ yanti\\
Tasmā꜓ sī꜓laṁ viso꜓dhaye]

\begin{english}
  (Estos son los Cinco Preceptos;\\
  La virtud es una fuente de felicidad,\\
  La virtud es una fuente de verdadera riqueza,\\
  La virtud es una fuente de paz ---\\
  que la virtud sea así purificada.)
\end{english}

\begin{instruction}
  Respuesta:
\end{instruction}

Sādhu, sādhu, sādhu

\begin{instruction}
  Hacer tres reverencias
\end{instruction}

\clearpage
\chapter[Tres Refugios y Ocho Preceptos]{Petición de los Tres Refugios\newline y Ocho Preceptos}

\begin{instruction}
  Después de hacer tres reverencias, con las palmas\\
  de las manos unidas en añjali, se recita la petición:
\end{instruction}

\subsection{En grupo}

Mayaṁ bhante tisaraṇena sa꜕ha aṭṭha sī꜓lāni yā꜕cāma\\
Dutiyampi mayaṁ bhante tisaraṇena sa꜕ha aṭṭha sī꜓lāni yā꜕cāma\\
Tatiyampi mayaṁ bhante tisaraṇena sa꜕ha aṭṭha sī꜓lāni yā꜕cāma

\subsection{Individualmente}

Ahaṁ bhante tisaraṇena sa꜕ha aṭṭha sī꜓lāni yā꜕cāmi\\
Dutiyampi ahaṁ bhante tisaraṇena sa꜕ha aṭṭha sī꜓lāni yā꜕cāmi\\
Tatiyampi ahaṁ bhante tisaraṇena sa꜕ha aṭṭha sī꜓lāni yā꜕cāmi

\subsection{Traducción}

\begin{english}
  Nosotros/Yo, Venerable Señor,\\
  \vin solicitamos/solicito los Tres Refugios y los Ocho Preceptos.

  Por segunda vez, Nosotros/Yo, Venerable Señor,\\
  \vin solicitamos/solicito los Tres Refugios y los Ocho Preceptos.

  Por tercera vez, Nosotros/Yo, Venerable Señor,\\
  \vin solicitamos/solicito los Tres Refugios y los Ocho Preceptos.
\end{english}

\clearpage
\chapter{Los Tres Refugios}

\begin{instruction}
  Repetir, después de que el líder haya\\
  cantado las primeras tres lineas
\end{instruction}

Namo tassa bhagavato arahato sammāsambuddhassa\\
Namo tassa bhagavato arahato sammāsambuddhassa\\
Namo tassa bhagavato arahato sammāsambuddhassa

\begin{english}
  (Homenaje al Señor, Noble y Perfectamente Iluminado.)\\
\end{english}

Buddhaṁ saraṇaṁ gacchāmi\\
Dhammaṁ saraṇaṁ gacchāmi\\
Saṅghaṁ saraṇaṁ gacchāmi

\begin{english}
  Voy al Buda en busca de refugio.\\
  Voy al Dhamma en busca de refugio.\\
  Voy a la Sangha en busca de refugio.
\end{english}

Dutiyampi buddhaṁ saraṇaṁ gacchāmi\\
Dutiyampi dhammaṁ saraṇaṁ gacchāmi\\
Dutiyampi saṅghaṁ saraṇaṁ gacchāmi

\begin{english}
 Por segunda vez, Voy al Buda en busca de refugio.\\
 Por segunda vez, Voy al Dhamma en busca de refugio.\\
 Por segunda vez, Voy a la Sangha en busca de refugio.
\end{english}

Tatiyampi buddhaṁ saraṇaṁ gacchāmi\\
Tatiyampi dhammaṁ saraṇaṁ gacchāmi\\
Tatiyampi saṅghaṁ saraṇaṁ gacchāmi

\clearpage

\begin{english}
 Por tercera vez, Voy al Buda en busca de refugio.\\
 Por tercera vez, Voy al Dhamma en busca de refugio.\\
 Por tercera vez, Voy a la Sangha en busca de refugio.
\end{english}

\begin{instruction}
  Líder:
\end{instruction}

[Tisaraṇa-gamanaṁ niṭṭhitaṁ]

\begin{english}
  (Queda así completo el Triple Refugio.)
\end{english}

\begin{instruction}
  Respuesta:
\end{instruction}

Āma bhante

\begin{english}
  (Sí, Venerable Maestro.)
\end{english}

\chapter{Los Ocho Preceptos}

\begin{instruction}
  Repetir cada precepto despues del líder
\end{instruction}

\begin{precept}
  \setcounter{enumi}{0}
  \item Pāṇātipātā vera꜓maṇī sikkhā꜓padaṁ sa꜓mādi꜕yāmi
\end{precept}

\begin{english}
  Tomo el entrenamiento de abstenerme de quitar la vida a cualquier criatura viviente.
\end{english}

\begin{precept}
  \setcounter{enumi}{1}
  \item Adinnādānā vera꜓maṇī sikkhā꜓padaṁ sa꜓mādi꜕yāmi
\end{precept}

\begin{english}
  Tomo el entrenamiento de absternerme de tomar lo que no me es dado.
\end{english}

\begin{precept}
  \setcounter{enumi}{2}
  \item Abrahmacariyā vera꜓maṇī sikkhā꜓padaṁ sa꜓mādi꜕yāmi
\end{precept}

\begin{english}
  Tomo el entrenamiento de abstenerme de cualquier tipo de actividad sexual intencional.
\end{english}

\begin{precept}
  \setcounter{enumi}{3}
  \item Musā꜓vādā vera꜓maṇī sikkhā꜓padaṁ sa꜓mādi꜕yāmi
\end{precept}

\begin{english}
  Tomo el entrenamiento de abstenerme de mentir.
\end{english}

\begin{precept}
  \setcounter{enumi}{4}
  \item Surāmeraya-majja-pamādaṭṭhā꜓nā vera꜓maṇī sikkhā꜓padaṁ sa꜓mādi꜕yāmi
\end{precept}

\begin{english}
  Tomo el entrenamiento de abstenerme de consumir bebidas embriagantes\\
  y drogas que conduzcan al descuido.
\end{english}

\begin{precept}
  \setcounter{enumi}{5}
  \item Vikālabhojanā vera꜓maṇī sikkhā꜓padaṁ sa꜓mādi꜕yāmi.
\end{precept}

\begin{english}
  Tomo el entrenamiento de abstenerme de comer en periodos indebidos.
\end{english}

\begin{precept}
  \setcounter{enumi}{6}
  \item Nacca-gīta-vādita-visūkada꜓ssanā mālā-gandha-vilepana-dhāraṇa-maṇḍana-vibhūsanaṭṭhā꜓nā vera꜓maṇī sikkhā꜓padaṁ sa꜓mādi꜕yāmi.
\end{precept}

\begin{english}
  Tomo el entrenamiento de abstenerme de entretenimiento, embellecimiento y adorno.
\end{english}

\begin{precept}
  \setcounter{enumi}{7}
  \item Uccāsayana-mahā꜓sayanā vera꜓maṇī sikkhā꜓padaṁ sa꜓mādi꜕yāmi.
\end{precept}

\begin{english}
  Tomo el entrenamiento de abstenerme de acostarme en un lugar alto o lujoso para dormir.
\end{english}

\begin{instruction}
	Líder:
\end{instruction}

[Imāni aṭṭha sikkhā꜓padāni samādiyāmi]\\

\begin{instruction}
	Respuesta:
\end{instruction}

Imāni aṭṭha sikkhā꜓padāni samādiyāmi (x3)
\begin{instruction}
  Líder:
\end{instruction}

[Imāni aṭṭha sikkhā꜓padāni\\
Sī꜓lena suga꜕tiṁ yanti\\
Sī꜓lena bhoga꜕sa꜓mpadā\\
Sī꜓lena nibbu꜕tiṁ yanti\\
Tasmā꜓ sī꜓laṁ viso꜓dhaye]


\begin{english}
 (Estos son los Ocho Preceptos;\\
 La virtud es una fuente de felicidad,\\
 La virtud es una fuente de verdadera riqueza,\\
 La virtud es una fuente de paz ---\\
 que la virtud sea así purificada.)
\end{english}

\begin{instruction}
  Respuesta:
\end{instruction}

Sādhu, sādhu, sādhu.

\begin{instruction}
  Hacer tres reverencias
\end{instruction}

