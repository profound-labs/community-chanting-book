\addtocontents{toc}{%
  \protect \enlargethispage{2\baselineskip}
}

\clearpage
\chapter*[Bem-Estar Universal]{Reflexão sobre o Bem-Estar Universal}

\delegateSetUseNext

\firstline{Ahaṃ sukhito homi}

\begin{leader}
[Ha꜓nda mayam mettāpharaṇaṃ ka꜕romase]
\end{leader}

[Aha꜓ṃ sukhito ho꜓mi]\\
Niddukkho ho꜓mi\\
A꜕vero ho꜓mi\\
A꜕byāpajjho ho꜓mi\\
A꜕nīgho ho꜓mi\\
Sukhī꜓ attānaṃ pa꜕riha꜓rāmi

Sa꜕bbe sa꜕ttā sukhitā ho꜓ntu\\
Sa꜕bbe sa꜕ttā averā ho꜓ntu\\
Sa꜕bbe sa꜕ttā abyāpajjhā ho꜓ntu\\
Sa꜕bbe sa꜕ttā anīghā ho꜓ntu\\
Sa꜕bbe sa꜕ttā sukhī꜓ a꜕ttānaṃ pa꜕riha꜓rantu

Sa꜕bbe sa꜕ttā sabbadukkhā pamucca꜓ntu

Sa꜕bbe sa꜕ttā laddha-sa꜓mpa꜕tti꜓to mā vigaccha꜓ntu

Sa꜕bbe sa꜕ttā kammassa꜕kā kamma꜓dāyādā kamma꜓yonī\\
\vin kamma꜓bandhū kammapa꜕ṭisa꜓ra꜕ṇā\\
Yaṃ kammaṃ ka꜕rissa꜓nti\\
Kalyāṇaṃ vā pāpa꜕kaṃ vā\\
Tassa꜕ dāyādā bha꜕vissa꜓nti

\clearpage
\chapter[Bem-Estar Universal]{Reflexão sobre o Bem-Estar Universal}

\firstline{Que eu mantenha bem-estar}

\begin{leader}
  [Cantemos agora as Reflexões sobre o Bem-estar Universal.]
\end{leader}

[Que eu mantenha bem-estar,]\\
Livre de aflição,\\
Livre de hostilidade,\\
Livre de má-fé,\\
Livre de ansiedade,\\
E possa eu \prul{manter} em mim bem-estar.

Que todos mantenham bem-estar,\\
Livres de hostilidade,\\
Livres de má-fé,\\
Livres de ansiedade, e possam eles\\
\prul{Manter} bem-estar em si próprios.

Possam \prul{todos} os seres se libertarem de todo o sofrimento.

E que todos não se separarem da \prul{boa fortuna} que alcançaram.

Quando agem com intenção,\\
\prul{Todos} os seres são os donos de sua acção e herdam seus resultados.\\
O seu futuro nasce de tal acção, companheiro de tal acção,\\
E os seus \prul{resultados} serão o seu lar.

\prul{Todas} as acções com intenção,\\
Sejam elas \prul{boas} ou más ---\\
De tais \prul{actos} eles serão os herdeiros.

\clearpage
\chapter[Dez Temas]{Dez Temas para Recordar Frequentemente por Aqueles que Seguem o Caminho}

\firstline{Dasa ime bhikkhave}

\begin{leader}
  [Ha꜓nda mayaṃ pabbajita\hyp{}abhiṇha\hyp{}paccave꜕kkhaṇa\hyp{}pāṭhaṃ bhaṇāmase]
\end{leader}

[Dasa i꜕me bhikkhave] dhammā pabba꜕jitena a꜕bhiṇhaṃ pacca꜕vekkhi꜓tabbā, \pause\ ka꜕ta꜕me dasa

\begin{english}
  Monges, existem dez dhammas \pause\ acerca dos quais se deve reflectir frequentemente. \pause\ \prul{Quais} são estes dez dhammas?
\end{english}

Vevaṇṇi꜕yamhi ajjhūpa꜕ga꜕to'ti pabba꜕jitena a꜕bhiṇhaṃ pacca꜕vekkhi꜓tabbaṃ

\begin{english}
  `Já não vivo segundo os valores e objectivos do mundo.' \pause\\
  Quem perfaz o caminho \pause\ deve reflectir sobre isto frequentemente.
\end{english}

Parapaṭi꜕baddhā me jīvi꜓kā'ti pabba꜕jitena a꜕bhiṇhaṃ pacca꜕vekkhi꜓tabbaṃ

\begin{english}
  `A minha própria vida é sustentada \pause\ pela generosidade dos outros.' \pause\\
  Quem perfaz o caminho \pause\ deve reflectir sobre isto frequentemente.
\end{english}

Añño me ākappo ka꜕ra꜕ṇīyo'ti pabba꜕jitena a꜕bhiṇhaṃ pacca꜕vekkhi꜓tabbaṃ

\begin{english}
  `Devo esforçar-me por abandonar os meus hábitos antigos.' \pause\\
  Quem perfaz o caminho \pause\ deve reflectir sobre isto frequentemente.
\end{english}

\clearpage

Kacci nu꜕ kho me attā sīla꜕to na u꜕pavadatī'ti pabba꜕jitena a꜕bhiṇhaṃ pacca꜕vekkhi꜓tabbaṃ

\begin{english}
  `Surgem remorsos na minha mente \pause\ em relação à minha conduta?' \pause\\
  Quem perfaz o caminho \pause\ deve reflectir sobre isto frequentemente.
\end{english}

Kacci nu꜕ kho maṃ a꜕nuvicca viññū sabrahma꜓cārī sīla꜕to na u꜕pavadantī'ti pabba꜕jitena a꜕bhiṇhaṃ pacca꜕vekkhi꜓tabbaṃ

\begin{english}
  `Será que os meus companheiros espirituais \pause\\\
  acham falhas na minha conduta?' \pause\\
  Quem perfaz o caminho \pause\ deve reflectir sobre isto frequentemente.
\end{english}

Sa꜕bbehi me pi꜕yehi ma꜕nāpehi꜕ nānābhāvo vi꜕nābhāvo'ti pabba꜕jitena abhiṇhaṃ pacca꜕vekkhi꜓tabbaṃ

\begin{english}
  `Tudo aquilo que é meu, \pause\ que amo e prezo, \pause\ tornar-se-á diferente, \pause\ separar-se-á de mim.' \pause\\
  Quem perfaz o caminho \pause\ deve reflectir sobre isto frequentemente.
\end{english}

Kammassa꜕komhi kamma꜓dāyādo kamma꜕yoni kamma꜕bandhu kammapa꜕ṭisa꜓raṇo, yaṃ kammaṃ ka꜕rissāmi, kalyāṇaṃ vā pāpa꜕kaṃ vā, tassa꜕ dāyādo bha꜕vissāmī'ti pabba꜕jitena a꜕bhiṇhaṃ pacca꜕vekkhi꜓tabbaṃ

\begin{english}
  `Sou o dono do meu Kamma, \pause\ herdeiro do meu Kamma, \pause\\
  nascido do meu Kamma, \pause\ ligado ao meu Kamma, \pause\\
  permaneço suportado pelo meu Kamma; \pause\ seja qual Kamma eu criar, \pause\\
  Para o bem ou para o mal, \pause\ \prul{disso} serei o herdeiro.' \pause\\
  Quem perfaz o caminho \pause\ deve reflectir sobre isto frequentemente.
\end{english}

\clearpage

`Kathambhūtassa꜕ me rattindi꜕vā vīti꜕pa꜓tantī'ti pabba꜕jitena a꜕bhiṇhaṃ pacca꜕vekkhi꜓tabbaṃ

\begin{english}
  `Os dias e as noites passam continuamente; \pause\\
  Como estou eu a usar o meu tempo?' \pause\\
  Quem perfaz o caminho \pause\ deve reflectir sobre isto frequentemente.
\end{english}

Kacci nu꜕ kho'haṃ suññā꜓gāre abhira꜕māmī'ti pabba꜕jitena a꜕bhiṇhaṃ pacca꜕vekkhi꜓tabbaṃ

\begin{english}
  `Aprecio a solidão ou não?' \pause\\
  Quem perfaz o caminho \pause\ deve reflectir sobre isto frequentemente.
\end{english}

Atthi nu꜕ kho me uttari-ma꜕nussa-dhammā alamariya꜕-ñāṇa-dassana-viseso adhiga꜕to, so'haṃ pacchi꜓me kāle sa꜕brahmacārīhi꜕ puṭṭho na maṅku bha꜕vissāmī'ti pabba꜕jitena a꜕bhiṇhaṃ pacca꜕vekkhi꜓tabbaṃ

\begin{english}
  `Deu a minha prática frutos de compreensão e liberdade, \pause\\ de forma a que
  no fim da minha vida \pause\ eu não me sinta envergonhado \pause\\
  quando questionado \pause\ pelos meus companheiros espirituais?' \pause\\
  Quem perfaz o caminho \pause\ deve reflectir sobre isto frequentemente.
\end{english}

Ime kho bhikkha꜓ve da꜕sa꜕ dhammā pabba꜕jitena a꜕bhiṇhaṃ pacca꜕vekkhitabbā'ti

\begin{english}
  Monges estes são dez Dhammas \pause\ sobre os quais se deve reflectir frequentemente.
\end{english}

\clearpage
\chapter{Ovāda-Pāṭimokkha}

\firstline{Khantī paramaṃ tapo tītikkhā}

\enlargethispage*{2\baselineskip}

\begin{leader}
  [Handa mayaṃ ovāda-pāṭimokkha-gāthāyo bhaṇāmase]
\end{leader}

\bigskip

{\setlength{\parskip}{0pt plus 0pt minus 1pt}

Khantī paramaṃ tapo tītikkhā

\begin{tightenglish}
  Permanecer paciente é a maior austeridade.
\end{tightenglish}

Nibbānaṃ paramaṃ vadanti buddhā

\begin{tightenglish}
  “Nibbāna é supremo”, dizem os Buddhas.
\end{tightenglish}

Na hi pabbajito parūpaghātī

\begin{tightenglish}
  Não se é verdadeiramente monge quando se prejudica alguém,
\end{tightenglish}

Samaṇo hoti paraṃ viheṭhayanto

\begin{tightenglish}
  nem verdadeiramente renunciante quando se oprime os outros.
\end{tightenglish}

Sabba-pāpassa akaraṇaṃ

\begin{tightenglish}
  Evitar todo o mal,
\end{tightenglish}

Kusalassūpasampadā

\begin{tightenglish}
  cultivar o bem
\end{tightenglish}

Sacitta-pariyodapanaṃ

\begin{tightenglish}
  e purificar a mente --
\end{tightenglish}

Etaṃ buddhāna sāsanaṃ

\begin{tightenglish}
  Este é o ensinamento dos Buddhas.
\end{tightenglish}

Anūpavādo anūpaghāto

\begin{tightenglish}
  Não ofender, não prejudicar,
\end{tightenglish}

Pāṭimokkhe ca saṃvaro

\begin{tightenglish}
  conter-se de acordo com o código monástico,
\end{tightenglish}

Mattaññutā ca bhattasmiṃ

\begin{tightenglish}
  moderar-se na comida,
\end{tightenglish}

Pantañca sayan'āsanaṃ

\begin{tightenglish}
  viver solitário,
\end{tightenglish}

Adhicitte ca āyogo

\begin{tightenglish}
  devotar-se à consciência elevada --
\end{tightenglish}

Etaṃ buddhāna sāsanaṃ ti.

\begin{tightenglish}
  Este é o ensinamento dos Buddhas.
\end{tightenglish}

}

\clearpage
\chapter[Aparihānīyā-dhammā Sutta]{Bhikkhu-aparihānīyā-dhammā Sutta}

\emph{Seven Conditions for the Welfare of the Bhikkhus, AN 7.23}

\begin{leader}
  [Handa mayaṃ bhikkhu-aparihānīyā-dhammā-sutta-pāṭhaṃ bhaṇāmase]
\end{leader}

[Evaṃ me sutaṃ.] Ekaṃ samayaṃ bhagavā rājagahe꜔꜒ viharati gijjhakūṭe pabbate.
Tatra kho꜔꜒ bhagavā bhikkhū꜔꜒ āmantesi: “satta vo, bhikkhave, aparihā꜔꜒niye dhamme
desessā꜔꜒mi. Taṃ suṇātha, sā꜔꜒dhukaṃ manasi karotha, bhāsissā꜔꜒mī”ti. “Evaṃ, bhante”ti
kho꜔꜒ te bhikkhū꜔꜒ bhagavato paccasso꜔꜒su꜔꜒ṃ. Bhagavā etadavoca:

\begin{english}
  I have heard that on one occasion the Blessed One was staying in Rajagaha, on
  Vulture Peak. There he addressed the monks: “Monks, I will teach you the seven
  conditions that lead to no decline. Listen \& pay close attention. I will
  speak.” “Yes, lord,” the monks responded. The Blessed One said:
\end{english}

[1] “Katame ca, bhikkhave, satta aparihā꜔꜒niyā dhammā? Yāvakīvañca, bhikkhave, bhikkhū꜔꜒
abhiṇha꜔꜒ṃ sa꜔꜒nnipātā bhavissa꜔꜒nti sa꜔꜒nnipātabahulā; vuddhiyeva, bhikkhave,
bhikkhū꜔꜒naṃ pāṭikaṅkhā꜔꜒, no parihā꜔꜒ni.

\begin{english}
  “And which seven are the conditions that lead to no decline? As long as the
  monks meet often, meet a great deal, their growth can be expected, not their
  decline.
\end{english}

[2] “Yāvakīvañca, bhikkhave, bhikkhū꜔꜒ samaggā sa꜔꜒nnipatissa꜔꜒nti, samaggā
vuṭṭhahissa꜔꜒nti, samaggā sa꜔꜒ṅghakaraṇīyāni karissa꜔꜒nti; vuddhiyeva, bhikkhave,
bhikkhū꜔꜒naṃ pāṭikaṅkhā꜔꜒, no parihā꜔꜒ni.

\begin{english}
  “As long as the monks meet in harmony, adjourn from their meetings in harmony,
  and conduct Sangha business in harmony, their growth can be expected, not
  their decline.
\end{english}

[3] “Yāvakīvañca, bhikkhave, bhikkhū꜔꜒ apaññattaṃ na paññāpessa꜔꜒nti, paññattaṃ na
samucchi꜔꜒ndissa꜔꜒nti, yathā꜔꜒paññattesu sikkhā꜔꜒padesu samādāya vattissa꜔꜒nti;
vuddhiyeva, bhikkhave, bhikkhū꜔꜒naṃ pāṭikaṅkhā꜔꜒, no parihā꜔꜒ni.

\begin{english}
  “As long as the monks neither decree what has been undecreed nor repeal what
  has been decreed, but practice undertaking the training rules as they have
  been decreed, their growth can be expected, not their decline.
\end{english}

[4] “Yāvakīvañca, bhikkhave, bhikkhū꜔꜒ ye te bhikkhū꜔꜒ the꜔꜒rā rattaññū cirapabbajitā
sa꜔꜒ṅghapitaro sa꜔꜒ṅghapariṇāyakā te sakkarissa꜔꜒nti garuṃ karissa꜔꜒nti mānessa꜔꜒nti
pūjessa꜔꜒nti, tesa꜔꜒ñca so꜔꜒tabbaṃ maññissa꜔꜒nti; vuddhiyeva, bhikkhave, bhikkhū꜔꜒naṃ
pāṭikaṅkhā꜔꜒, no parihā꜔꜒ni.

\begin{english}
  “As long as the monks honor, respect, venerate, and do homage to the elder
  monks — those with seniority who have long been ordained, the fathers of the
  Sangha, leaders of the Sangha — regarding them as worth listening to, their
  growth can be expected, not their decline.
\end{english}

[5] “Yāvakīvañca, bhikkhave, bhikkhū꜔꜒ uppannāya taṇhā꜔꜒ya ponobhavikāya na vasa꜔꜒ṃ
gacchissa꜔꜒nti; vuddhiyeva, bhikkhave, bhikkhū꜔꜒naṃ pāṭikaṅkhā꜔꜒, no parihā꜔꜒ni.

\begin{english}
  “As long as the monks do not submit to the power of any arisen craving that
  leads to further becoming, their growth can be expected, not their decline.
\end{english}

[6] “Yāvakīvañca, bhikkhave, bhikkhū꜔꜒ āraññakesu se꜔꜒nāsanesu sā꜔꜒pekkhā꜔꜒ bhavissa꜔꜒nti;
vuddhiyeva, bhikkhave, bhikkhū꜔꜒naṃ pāṭikaṅkhā꜔꜒, no parihā꜔꜒ni.

\begin{english}
  “As long as the monks see their own benefit in wilderness dwellings, their
  growth can be expected, not their decline.
\end{english}

[7] “Yāvakīvañca, bhikkhave, bhikkhū꜔꜒ paccattaññeva satiṃ upaṭṭhā꜔꜒pessa꜔꜒nti: ‘kinti
anāgatā ca pesalā sabrahmacārī āgacche꜔꜒yyuṃ, āgatā ca pesalā sabrahmacārī phā꜔꜒su꜔꜒ṃ
vihareyyun’ti; vuddhiyeva, bhikkhave, bhikkhū꜔꜒naṃ pāṭikaṅkhā꜔꜒, no parihā꜔꜒ni.

\begin{english}
  “As long as the monks each keep firmly in mind: `If there are any well-behaved
  fellow followers of the chaste life who have yet to come, may they come; and
  may the well-behaved fellow-followers of the chaste life who have come live in
  comfort,' their growth can be expected, not their decline.
\end{english}

“Yāvakīvañca, bhikkhave, ime satta aparihā꜔꜒niyā dhammā bhikkhū꜔꜒su ṭhassa꜔꜒nti, imesu
ca sattasu aparihā꜔꜒niyesu dhammesu bhikkhū꜔꜒ sa꜔꜒ndississa꜔꜒nti; vuddhiyeva, bhikkhave,
bhikkhū꜔꜒naṃ pāṭikaṅkhā꜔꜒, no parihā꜔꜒nī”ti. Idam-avoca Bhagavā. Attamanā te bhikkhū꜔꜒
Bhagavato bhāsitaṃ, abhinandun'ti.

\begin{english}
  “As long as the monks remain steadfast in these seven conditions, and as long
  as these seven conditions endure among the monks, the monks' growth can be
  expected, not their decline.” That is what the Blessed One said. Gratified,
  the monks delighted in the Blessed One's words.
\end{english}
