\addtocontents{toc}{%
  \protect \enlargethispage{2\baselineskip}
}

\clearpage
\chapter*[Bem-Estar Universal]{Reflexão sobre o Bem-Estar Universal}

\delegateSetUseNext

\firstline{Ahaṃ sukhito homi}

\begin{leader}
[Ha꜓nda mayam mettāpharaṇaṃ ka꜕romase]
\end{leader}

[Aha꜓ṃ sukhito ho꜓mi]\\
Niddukkho ho꜓mi\\
A꜕vero ho꜓mi\\
A꜕byāpajjho ho꜓mi\\
A꜕nīgho ho꜓mi\\
Sukhī꜓ attānaṃ pa꜕riha꜓rāmi

Sa꜕bbe sa꜕ttā sukhitā ho꜓ntu\\
Sa꜕bbe sa꜕ttā averā ho꜓ntu\\
Sa꜕bbe sa꜕ttā abyāpajjhā ho꜓ntu\\
Sa꜕bbe sa꜕ttā anīghā ho꜓ntu\\
Sa꜕bbe sa꜕ttā sukhī꜓ a꜕ttānaṃ pa꜕riha꜓rantu

Sa꜕bbe sa꜕ttā sabbadukkhā pamucca꜓ntu

Sa꜕bbe sa꜕ttā laddha-sa꜓mpa꜕tti꜓to mā vigaccha꜓ntu

Sa꜕bbe sa꜕ttā kammassa꜕kā kamma꜓dāyādā kamma꜓yonī\\
\vin kamma꜓bandhū kammapa꜕ṭisa꜓ra꜕ṇā\\
Yaṃ kammaṃ ka꜕rissa꜓nti\\
Kalyāṇaṃ vā pāpa꜕kaṃ vā\\
Tassa꜕ dāyādā bha꜕vissa꜓nti

\clearpage
\chapter[Bem-Estar Universal]{Reflexão sobre o Bem-Estar Universal}

\firstline{Que eu mantenha bem-estar}

\begin{leader}
  [Cantemos agora as Reflexões sobre o Bem-estar Universal.]
\end{leader}

[Que eu mantenha bem-estar,]\\
Livre de aflição,\\
Livre de hostilidade,\\
Livre de má-fé,\\
Livre de ansiedade,\\
E possa eu \prul{manter} em mim bem-estar.

Que todos mantenham bem-estar,\\
Livres de hostilidade,\\
Livres de má-fé,\\
Livres de ansiedade, e possam eles\\
\prul{Manter} bem-estar em si próprios.

Possam \prul{todos} os seres se libertarem de todo o sofrimento.

E que todos não se separarem da \prul{boa fortuna} que alcançaram.

Quando agem com intenção,\\
\prul{Todos} os seres são os donos de sua acção e herdam seus resultados.\\
O seu futuro nasce de tal acção, companheiro de tal acção,\\
E os seus \prul{resultados} serão o seu lar.

\prul{Todas} as acções com intenção,\\
Sejam elas \prul{boas} ou más ---\\
De tais \prul{actos} eles serão os herdeiros.

\clearpage
\chapter[Dez Temas]{Dez Temas para Recordar Frequentemente por Aqueles que Seguem o Caminho}

\firstline{Dasa ime bhikkhave}

\enlargethispage{\baselineskip}

\begin{leader}
  [Ha꜓nda mayaṃ pabbajita\hyp{}abhiṇha\hyp{}paccave꜕kkhaṇa\hyp{}pāṭhaṃ bhaṇāmase]
\end{leader}

[Dasa i꜕me bhikkhave] dhammā pabba꜕jitena a꜕bhiṇhaṃ pacca꜕vekkhi꜓tabbā, ka꜕ta꜕me dasa

\begin{english}
  Monges, existem dez dhammas acerca dos quais se deve reflectir frequentemente. \prul{Quais} são estes dez dhammas?
\end{english}

Vevaṇṇi꜕yamhi ajjhūpa꜕ga꜕to'ti pabba꜕jitena a꜕bhiṇhaṃ pacca꜕vekkhi꜓tabbaṃ

\begin{english}
  `Já não vivo segundo os valores e objectivos do mundo.'\\
  Quem perfaz o caminho\\
  deve reflectir sobre isto frequentemente.
\end{english}

Parapaṭi꜕baddhā me jīvi꜓kā'ti pabba꜕jitena a꜕bhiṇhaṃ pacca꜕vekkhi꜓tabbaṃ

\begin{english}
  `A minha própria vida é sustentada pela generosidade dos outros.'\\
  Quem perfaz o caminho\\
  deve reflectir sobre isto frequentemente.
\end{english}

Añño me ākappo ka꜕ra꜕ṇīyo'ti pabba꜕jitena a꜕bhiṇhaṃ pacca꜕vekkhi꜓tabbaṃ

\begin{english}
  `Devo esforçar-me por abandonar os meus hábitos antigos.'\\
  Quem perfaz o caminho\\
  deve reflectir sobre isto frequentemente.
\end{english}

\clearpage

Kacci nu꜕ kho me attā sīla꜕to na u꜕pavadatī'ti pabba꜕jitena a꜕bhiṇhaṃ pacca꜕vekkhi꜓tabbaṃ

\begin{english}
  `Surgem remorsos na minha mente em relação à minha conduta?'\\
  Quem perfaz o caminho\\
  deve reflectir sobre isto frequentemente.
\end{english}

Kacci nu꜕ kho maṃ a꜕nuvicca viññū sabrahma꜓cārī sīla꜕to na u꜕pavadantī'ti pabba꜕jitena a꜕bhiṇhaṃ pacca꜕vekkhi꜓tabbaṃ

\begin{english}
  `Será que os meus companheiros espirituais acham falhas na minha conduta?'\\
  Quem perfaz o caminho\\
  deve reflectir sobre isto frequentemente.
\end{english}

Sa꜕bbehi me pi꜕yehi ma꜕nāpehi꜕ nānābhāvo vi꜕nābhāvo'ti pabba꜕jitena abhiṇhaṃ pacca꜕vekkhi꜓tabbaṃ

\begin{english}
  `Tudo aquilo que é meu, que amo e prezo, tornar-se-á diferente, separar-se-á de mim.'\\
  Quem perfaz o caminho\\
  deve reflectir sobre isto frequentemente.
\end{english}

Kammassa꜕komhi kamma꜓dāyādo kamma꜕yoni kamma꜕bandhu kammapa꜕ṭisa꜓raṇo, yaṃ kammaṃ ka꜕rissāmi, kalyāṇaṃ vā pāpa꜕kaṃ vā, tassa꜕ dāyādo bha꜕vissāmī'ti pabba꜕jitena a꜕bhiṇhaṃ pacca꜕vekkhi꜓tabbaṃ

\enlargethispage{2\baselineskip}

\begin{english}
  `Sou o dono do meu Kamma, herdeiro do meu Kamma,\\
  nascido do meu Kamma, ligado ao meu Kamma,\\
  permaneço suportado pelo meu Kamma; seja qual Kamma eu criar,\\
  Para o bem ou para o mal, \prul{disso} serei o herdeiro.'\\
  Quem perfaz o caminho\\
  deve reflectir sobre isto frequentemente.
\end{english}

\clearpage

`Kathambhūtassa꜕ me rattindi꜕vā vīti꜕pa꜓tantī'ti pabba꜕jitena a꜕bhiṇhaṃ pacca꜕vekkhi꜓tabbaṃ

\begin{english}
  `Os dias e as noites passam continuamente; Como estou eu a usar\\ o meu tempo?'\\
 Quem perfaz o caminho\\
 deve reflectir sobre isto frequentemente.
\end{english}

Kacci nu꜕ kho'haṃ suññā꜓gāre abhira꜕māmī'ti pabba꜕jitena a꜕bhiṇhaṃ pacca꜕vekkhi꜓tabbaṃ

\begin{english}
  `Aprecio a solidão ou não?'\\
  Quem perfaz o caminho\\
  deve reflectir sobre isto frequentemente.
\end{english}

Atthi nu꜕ kho me uttari-ma꜕nussa-dhammā alamariya꜕-ñāṇa-dassana-viseso adhiga꜕to, so'haṃ pacchi꜓me kāle sa꜕brahmacārīhi꜕ puṭṭho na maṅku bha꜕vissāmī'ti pabba꜕jitena a꜕bhiṇhaṃ pacca꜕vekkhi꜓tabbaṃ

\begin{english}
  `Deu a minha prática frutos de compreensão e liberdade, de forma a que no fim da minha vida eu não me sinta envergonhado quando questionado pelos meus companheiros espirituais?'\\
  Quem perfaz o caminho\\
  deve reflectir sobre isto frequentemente.
\end{english}

Ime kho bhikkha꜓ve da꜕sa꜕ dhammā pabba꜕jitena a꜕bhiṇhaṃ pacca꜕vekkhitabbā'ti

\begin{english}
  Monges estes são dez Dhammas sobre os quais se deve reflectir frequentemente.
\end{english}
