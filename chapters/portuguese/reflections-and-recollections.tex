\chapter*[Partilha de Bençãos]{Reflexões sobre a Partilha de Bençãos}

\delegateSetUseNext

\begin{leader}
  [Ha꜓nda mayaṃ uddissanādhiṭṭhāna-gāthā꜓yo b꜕haṇāmase]
\end{leader}

\firstline{Iminā puññakammena upajjhāyā guṇuttarā}

[Iminā puñña꜕kammena] u꜕pajjhāyā gu꜕ṇutta꜕rā\\
Ācariyūpa꜕kārā ca꜕ mātāpitā ca꜕ ñāta꜕kā\\
Suriyo candimā rājā gu꜕ṇavantā na꜕rāpi꜕ ca꜕\\
Brahma-mārā ca꜕ indā ca꜕ loka꜕pālā ca꜕ deva꜕tā\\
Yamo mittā ma꜕nussā ca majjhattā veri꜕kāpi꜕ ca꜕\\
Sa꜕bbe sattā sukhī hontu puññāni pa꜕ka꜕tāni꜕ me\\
Sukhañca tividhaṃ dentu꜕ khippaṃ pāpetha꜕ voma꜕taṃ\\
Iminā puññakammena iminā uddi꜕ssena꜕ ca꜕\\
Khipp'āhaṃ su꜕la꜕bhe ceva taṇhūpādāna꜕-cheda꜕naṃ\\
Ye santāne hīnā dhammā yāva꜕ nibbāna꜕to ma꜕maṃ\\
Nassantu sabba꜕dā yeva yattha꜕ jāto bha꜕ve bha꜕ve\\
Ujucittaṃ sa꜕ti꜕paññā sallekho vi꜕ri꜕yamhinā\\
Mārā labhantu nokāsaṃ kātuñca vi꜕ri꜕yes꜕u me\\
Buddhādhipa꜕va꜕ro nātho dhammo nātho va꜕rutta꜕mo\\
Nātho pacceka꜕buddho ca꜕ saṅgho nāthotta꜕ro ma꜕maṃ\\
Tesottamānubhāvena mārokāsaṃ la꜕bhantu꜕ mā

\chapter[Partilha de Bençãos]{Reflexões sobre a Partilha de Bençãos}

\enlargethispage{2\baselineskip}

\begin{leader}
  [Cantemos agora as Reflexões sobre a Partilha de Bençãos]
\end{leader}

\firstline{Através do bem que resulta da minha prática}

Através do bem que resulta da minha prática,\\
Que os meus mestres e guias espirituais de grande virtude,\\
A minha mãe, o meu pai e os meus familiares,\\
O Sol e a Lua, e todos os líderes virtuosos do mundo,\\
Que os Deuses mais elevados e as forças do mal,\\
Seres celestiais, espíritos guardiões da Terra e o Senhor da Morte,\\
Aqueles que são amigáveis, indiferentes ou hostis,\\
Que todos os seres recebam as bênçãos da minha vida.\\
Que brevemente cheguem à Tripla Bênção, e superem a morte.

Através do bem que resulta da minha prática,\\
E através desta partilha,\\
Que todos os desejos e apegos rapidamente cessem\\
Assim como os estados prejudiciais da mente.

Até realizar o Nibbana,\\
Em qualquer tipo de nascimento, que eu tenha uma mente justa,\\
Com consciência e sabedoria, austeridade e vigor.\\
Que as forças ilusórias não controlem,\\
nem enfraqueçam a minha decisão.

O Buddha é o meu excelente refúgio,\\
Insuperável é a proteção do Dhamma,\\
O Buddha solitário é o meu Nobre exemplo,\\
O Saṅgha é o meu maior suporte.

Que através desta supremacia\\
Desapareçam a escuridão e a ilusão.

\chapter*[Metta Sutta]{Metta Sutta}

\delegateSetUseNext

\firstline{Karaṇīyam-attha-kusalena}

\begin{leader}
  [Ha꜓nda mayaṃ metta-sutta-gāthā꜓yo bha꜕ṇāmase]
\end{leader}

[Karaṇīyam-attha-kusalena]\\
Yan-taṃ santaṃ padaṃ abhisamecca\\
Sakko ujū ca suhujū ca\\
Suvaco c'assa mudu anatimānī

Santussako ca subharo ca\\
Appakicco ca sallahuka-vutti\\
Sant'indriyo ca nipako ca\\
Appagabbho kulesu ananugiddho

Na ca khuddaṃ samācare kiñci\\
Yena viññū pare upavadeyyuṃ\\
Sukhino vā khemino hontu\\
Sabbe sattā bhavantu sukhit'attā

Ye keci pāṇa-bhūt'atthi\\
Tasā vā thāvarā vā anavasesā\\
Dīghā vā ye mahantā vā\\
Majjhimā rassakā aṇuka-thūlā

Diṭṭhā vā ye ca adiṭṭhā\\
Ye ca dūre vasanti avidūre\\
Bhūtā vā sambhavesī vā\\
Sabbe sattā bhavantu sukhit'attā

\chapter[Metta Sutta]{Metta Sutta}

\firstline{Eis o que se deve fazer}

\begin{leader}
  [Cantemos agora as palavras do Buddha\\ sobre o Amor e a Compaixão]
\end{leader}

Eis o que se deve fazer\\
Para cultivar a bondade\\
E seguir a via da paz:\\
Ser capaz e ser honesto,\\
Franco e gentil no falar.\\
Humilde e não arrogante,\\
Contente, facilmente satisfeito,\\
Aliviado de deveres e frugal no seu caminho.

Pacífico e sereno, sábio e inteligente,\\
Sem orgulho, sem exigência por natureza.\\
Que ele nada faça\\
Que os sábios possam vir a reprovar.\\
Desejando: Na alegria e na segurança,\\
Que todos os seres sejam felizes.\\
Quaisquer que sejam os seres vivos,\\
Fracos, fortes, sem excepção\\
Dos maiores aos mais pequenos,\\
Visíveis ou invisíveis,\\
Estejam perto ou estejam longe,\\
Nascidos ou por nascer ---\\
Que todos os seres sejam felizes!

\clearpage

Na paro paraṃ nikubbetha\\
Nātimaññetha katthaci naṃ kiñci\\
Byārosanā paṭighasaññā\\
Nāññam-aññassa dukkham-iccheyya

Mātā yathā niyaṃ puttaṃ\\
Āyusā eka-puttam-anurakkhe\\
Evam'pi sabba-bhūtesu\\
Mānasam-bhāvaye aparimāṇaṃ

Mettañca sabba-lokasmiṃ\\
Mānasam-bhāvaye aparimāṇaṃ\\
Uddhaṃ adho ca tiriyañca\\
Asambādhaṃ averaṃ asapattaṃ

Tiṭṭhañ-caraṃ nisinno vā\\
Sayāno vā yāvat'assa vigata-middho\\
Etaṃ satiṃ adhiṭṭheyya\\
Brahmam-etaṃ vihāraṃ idham-āhu

Diṭṭhiñca anupagamma\\
Sīlavā dassanena sampanno\\
Kāmesu vineyya gedhaṃ\\
Na hi jātu gabbha-seyyaṃ punaretī'ti

\clearpage

Que ninguém engane ninguém,\\
Ou despreze alguém em que estado fôr.\\
Que ninguém por raiva ou má-fé,\\
Deseje mal a alguém.\\
Assim como uma Mãe protege o filho,\\
Com sua vida, seu único filho,\\
Assim de coração infinito,\\
Se deveria estimar todo o ser vivo;\\
Irradiando ternura por todo o mundo:\\
Acima ao mais alto céu,\\
E abaixo às profundezas;\\
Irradiante e sem limites,\\
Livre de ódio e má-fé.\\
Seja parado ou a andar,\\
Sentado ou deitado,\\
Livre de torpor,\\
Esta é uma lembrança a manter.

Diz-se esta ser a sublime permanência.\\
O puro de coração, com clareza de visão,\\
Ao não insistir em ideias fixas,\\
Liberto dos desejos dos sentidos,\\
Não voltará a nascer neste mundo.

\chapter*[Permanências Divinas]{Difusão Com as Permanências Divinas}

\delegateSetUseNext

\firstline{Mettā-sahagatena}

\begin{leader}
  [Ha꜓nda mayaṃ caturappamaññā-obhāsanaṃ karomase]
\end{leader}

[Mettā-sa꜕ha꜕ga꜕tena] cetasā ekaṃ disaṃ pha꜕ri꜕tv꜕ā viha꜕ra꜕ti\\
Ta꜕thā dutiyaṃ ta꜕thā tatiyaṃ ta꜕thā ca꜕tutthaṃ\\
Iti uddhamadho tiriyaṃ sabba꜕dhi꜕ sabbatta꜕tāya\\
Sabbāvantaṃ lokaṃ mettā-sa꜕ha꜕ga꜕tena cetasā\\
Vipulena mahagga꜕tena appa꜕māṇena a꜕verena a꜕byāpajjhena\\
\vin pha꜕ri꜕tv꜕ā viha꜕ra꜕ti

Karuṇā-sa꜕ha꜕ga꜕tena cetasā ekaṃ disaṃ pha꜕ri꜕tv꜕ā viha꜕ra꜕ti\\
Ta꜕thā dutiyaṃ ta꜕thā tatiyaṃ ta꜕thā ca꜕tutthaṃ\\
Iti uddhamadho tiriyaṃ sabba꜕dhi꜕ sabbatta꜕tāya\\
Sabbāvantaṃ lokaṃ ka꜕ru꜕ṇā-sa꜕ha꜕ga꜕tena cetasā\\
Vipulena mahagga꜕tena appa꜕māṇena a꜕verena a꜕byāpajjhena\\
\vin pha꜕ri꜕tv꜕ā viha꜕ra꜕ti

Muditā-sa꜕ha꜕ga꜕tena cetasā ekaṃ disaṃ pha꜕ri꜕tv꜕ā viha꜕ra꜕ti\\
Ta꜕thā dutiyaṃ ta꜕thā tatiyaṃ ta꜕thā ca꜕tutthaṃ\\
Iti uddhamadho tiriyaṃ sabba꜕dhi꜕ sabbatta꜕tāya\\
Sabbāvantaṃ lokaṃ mu꜕di꜕tā-sa꜕ha꜕ga꜕tena cetasā\\
Vipulena mahagga꜕tena appa꜕māṇena a꜕verena a꜕byāpajjhena\\
\vin pha꜕ri꜕tv꜕ā viha꜕ra꜕ti

\chapter[Permanências Divinas]{Difusão Com as Permanências Divinas}

\enlargethispage{\baselineskip}

\firstline{Eu permanecerei}

\begin{leader}
  [Deixemos irradiar agora as Quatro Qualidades Ilimitadas.]
\end{leader}

[\prul{Eu} permanecerei] permeando um quarto do mundo com um coração\\
\vin imbuído de amável-gentileza;\\
Igualmente o segundo, igualmente o terceiro, igualmente o quarto;\\
Como em cima assim em baixo, à volta e em todo o lado;\\
\vin e para \prul{todos} assim como para mim.\\
\prul{Eu} permanecerei imbuindo todo o mundo circundante com um coração \\
\vin imbuído de amável-gentileza; abundante, exaltado,\\
\vin ime꜕nsurável, sem hostilidade, e sem má-fé.

[\prul{Eu} permanecerei] permeando um quarto do mundo com um coração\\
\vin imbuído de compaixão;\\
Igualmente o segundo, igualmente o terceiro, igualmente o quarto;\\
Como em cima assim em baixo, à volta e em todo o lado;\\
\vin e para \prul{todos} assim como para mim.\\
\prul{Eu} permanecerei imbuindo todo o mundo circundante com um coração\\
\vin imbuído de compaixão; abundante, exaltado,\\
\vin ime꜕nsurável, sem hostilidade, e sem má-fé.

[\prul{Eu} permanecerei] permeando um quarto do mundo com um coração\\
\vin imbuído de alegria;\\
Igualmente o segundo, igualmente o terceiro, igualmente o quarto;\\
Como em cima assim em baixo, à volta e em todo o lado;\\
\vin e para \prul{todos} assim como para mim.\\
\prul{Eu} permanecerei imbuindo todo o mundo circundante com um coração\\
\vin imbuído de alegria; abundante, exaltado,\\
\vin ime꜕nsurável, sem hostilidade, e sem má-fé.

\clearpage

Upekkhā-saha꜕ga꜕te꜕na cetasā ekaṃ disaṃ pha꜕ri꜕tv꜕ā viha꜕ra꜕ti\\
Ta꜕thā dutiyaṃ ta꜕thā tatiyaṃ ta꜕thā ca꜕tutthaṃ\\
Iti uddhamadho tiriyaṃ sabba꜕dhi꜕ sabbatta꜕tāya\\
Sabbāvantaṃ lokaṃ u꜕pe꜕kkhā-sa꜕ha꜕ga꜕tena cetasā\\
Vipulena mahagga꜕tena appa꜕māṇena a꜕verena a꜕byāpajjhena\\
\vin pha꜕ri꜕tv꜕ā viha꜕ra꜕tī'ti

\clearpage

[\prul{Eu} permanecerei] permeando um quarto do mundo com um coração\\
\vin imbuído de equanimidade;\\
Igualmente o segundo, igualmente o terceiro, igualmente o quarto;\\
Como em cima assim em baixo, à volta e em todo o lado;\\
\vin e para \prul{todos} assim como para mim.\\
\prul{Eu} permanecerei imbuindo todo o mundo circundante com um coração \\
\vin imbuído de equanimidade; abundante, exaltado,\\
\vin ime꜕nsurável, sem hostilidade, e sem má-fé.

\chapter{As Bênçãos Maiores}

\firstline{Assim eu ouvi que o Excelso}

\begin{leader}
  [Cantemos agora os versos sobre as Bênção Maiores]
\end{leader}

[\prul{Assim} eu \prul{ouvi} que o Excelso]\\
Se encontrava em Savatthi,\\
\prul{A residir} no Bosque de Jeta\\
No Parque de Anāthapiṇḍika.

\prul{Então} no escuro da noite, uma deva radiante\\
Iluminou \prul{todo} o Parque de Jeta.\\
Inclinou-se prestando reverência ao Excelso\\
E depois colocando-se de pé, disse:

`Os Devas preocupam-se com a felicidade\\
E buscam Paz continuamente.\\
O mesmo se pode dizer da humanidade.\\
Assim, \prul{quais} são as Bênçãos mais elevadas?'

`Evitar os tolos,\\
Associar-se aos Sábios,\\
E honrar quem é digno de honra.\\
\prul{Estas} são as maiores bênçãos.

`Viver em locais adequados,\\
Com os frutos das boas acções passadas,\\
Guiado pelo caminho correcto.\\
\prul{Estas} são as maiores bênçãos.

\clearpage

`Proficiente em estudos e ofícios,\\
Com disciplina sublimemente treinada,\\
E \prul{discurso} verdadeiro agradável ao ouvido.\\
\prul{Estas} são as maiores bênçãos.

`Apoiar os pais,\\
Zelar pela família,\\
E ter uma vida inofensiva para os outros.\\
\prul{Estas} são as maiores bênçãos.

`Generosidade e uma vida honesta,\\
Oferecer ajuda a familiares e amigos,\\
Agir de forma que não promova remorsos.\\
\prul{Estas} são as maiores bênçãos.

`Ser resoluto a controlar-se, a abandonar os caminhos do mal,\\
Evitar intoxicantes que entorpeçam a mente,\\
E ser diligente em todas as ocasiões.\\
\prul{Estas} são as maiores bênçãos.

`Respeito e humildade,\\
Contentamento e gratidão,\\
Ouvir o Dhamma frequentemente ensinado.\\
\prul{Estas} são as maiores bênçãos.

`Paciência e vontade para aceitar as próprias falhas,\\
Visitar respeitáveis buscadores da verdade,\\
e partilhar o Dhamma frequentemente.\\
\prul{Estas} são as maiores bênçãos.

\clearpage

`Dedicar-se ardentemente à Vida Santa,\\
Ver as Nobres Verdades directamente por si\\
E realizar o Nibbana.\\
\prul{Estas} são as maiores bênçãos.

`Ainda que em contacto com o mundo,\\
A mente mantem-se inabalável,\\
Perfeitamente segura além de toda a aflição.\\
\prul{Estas} são as maiores bênçãos.

`Aqueles que seguem este caminho,\\
Conhecem a Victória onde quer que vão,\\
E qualquer \prul{lugar} para eles é seguro.\\
\prul{Estas} são as maiores bênçãos.'

\chapter*[Bem-Estar Universal]{Reflexão sobre o Bem-Estar Universal}

\delegateSetUseNext

\firstline{Ahaṃ sukhito homi}

\begin{leader}
  [Ha꜓nda mayam mettāpharaṇaṃ ka꜕romase]
\end{leader}

[Aha꜓ṃ sukhito ho꜓mi]\\
Niddukkho ho꜓mi\\
A꜕vero ho꜓mi\\
A꜕byāpajjho ho꜓mi\\
A꜕nīgho ho꜓mi\\
Sukhī꜓ attānaṃ pa꜕riha꜓rāmi

Sa꜕bbe sa꜕ttā sukhitā ho꜓ntu\\
Sa꜕bbe sa꜕ttā averā ho꜓ntu\\
Sa꜕bbe sa꜕ttā abyāpajjhā ho꜓ntu\\
Sa꜕bbe sa꜕ttā anīghā ho꜓ntu\\
Sa꜕bbe sa꜕ttā sukhī꜓ a꜕ttānaṃ pa꜕riha꜓rantu

Sa꜕bbe sa꜕ttā sabbadukkhā pamucca꜓ntu

Sa꜕bbe sa꜕ttā laddha-sa꜓mpa꜕tti꜓to mā vigaccha꜓ntu

Sa꜕bbe sa꜕ttā kammassa꜕kā kamma꜓dāyādā kamma꜓yonī\\
\vin kamma꜓bandhū kammapa꜕ṭisa꜓ra꜕ṇā\\
Yaṃ kammaṃ ka꜕rissa꜓nti\\
Kalyāṇaṃ vā pāpa꜕kaṃ vā\\
Tassa꜕ dāyādā bha꜕vissa꜓nti

\chapter[Bem-Estar Universal]{Reflexão sobre o Bem-Estar Universal}

\firstline{Que eu mantenha bem-estar}

\begin{leader}
  [Cantemos agora as Reflexões sobre o Bem-estar Universal.]
\end{leader}

[Que eu mantenha bem-estar,]\\
Livre de aflição,\\
Livre de hostilidade,\\
Livre de má-fé,\\
Livre de ansiedade,\\
E possa eu \prul{manter} em mim bem-estar.

Que todos mantenham bem-estar,\\
Livres de hostilidade,\\
Livres de má-fé,\\
Livres de ansiedade, e possam eles\\
\prul{Manter} bem-estar em si próprios.

Possam \prul{todos} os seres se libertarem de todo o sofrimento.

E que todos não se separarem da \prul{boa fortuna} que alcançaram.

Quando agem com intenção,\\
\prul{Todos} os seres são os donos de sua acção e herdam seus resultados.\\
O seu futuro nasce de tal acção, companheiro de tal acção,\\
E os seus \prul{resultados} serão o seu lar.

\prul{Todas} as acções com intenção,\\
Sejam elas \prul{boas} ou más --\\
De tais \prul{actos} eles serão os herdeiros.

\chapter[Quatro Requisitos]{Reflexão sobre os Quatro Requisitos}

\firstline{Paṭisaṅkhā yoniso}

\begin{leader}
  [Ha꜓nda mayaṃ taṅkhaṇika-paccave꜕kkhaṇa-pāṭhaṃ bhaṇāmase]
\end{leader}

[Paṭisaṅkhā] yoniso cīva꜕raṃ pa꜕ṭise꜓vāmi, \pause\\
yāvadeva sī꜓tassa꜕ pa꜕ṭighātāya, \pause\ uṇhassa pa꜕ṭighātāya, \pause\\
ḍaṃsa-maka꜕sa꜕-vātāta꜕pa꜕-siriṃsapa-samphassānaṃ pa꜕ṭighātāya, \pause\\
yāvadeva hiri꜓kopina-pa꜕ṭicchāda꜕natthaṃ

\begin{english}
  Reflectindo sabiamente eu uso o manto: \pause\ Somente por modéstia, \pause\
  para evitar o calor, \pause\ o frio, \pause\ as moscas, \pause\ mosquitos,
  \pause\ bichos rastejantes, \pause\\ o vento e as coisas que queimam.
\end{english}

[Paṭisaṅkhā] yoniso piṇḍa꜕pātaṃ pa꜕ṭise꜓vāmi, \pause\\
neva da꜕vāya, na ma꜕dāya, na maṇḍa꜕nāya, na꜕ vi꜓bhūsa꜕nāya, \pause\\
yāvadeva i꜓massa꜕ kāyassa꜕ ṭhi꜕tiyā, \pause\ yāpa꜕nāya, vihiṃsū꜕para꜓ti꜕yā, \pause\\
brahmaca꜕ri꜓yānugga꜕hāya, \pause\ iti purāṇañca꜕ veda꜓naṃ pa꜕ṭiha꜓ṅkhāmi,
navañca꜕ veda꜓naṃ na uppādessāmi, \pause\ yātrā ca꜕ me bhavissati a꜕navajjatā
ca꜕ phāsuvihāro cā'ti

\begin{english}
  Reflectindo sabiamente \pause\ eu uso a comida da mendicância: \pause\ Não por
  diversão, \pause\ não por prazer, \pause\ não para engordar, \pause\ não para
  me embelezar, \pause\ mas somente para suster e nutrir este corpo, \pause\
  para o manter saudável, \pause\ para ajudar à Vida Santa. \pause\ Pensando
  desta forma: \pause\ `Saciarei a fome sem comer demasiado, \pause\ de
  forma a~continuar a viver sereno e sem remorsos.'
\end{english}

\clearpage

[Paṭisaṅkhā] yoniso senāsa꜕naṃ pa꜕ṭise꜓vāmi, \pause\\
yāvadeva sī꜓tassa꜕ pa꜕ṭighātāya, \pause\ uṇhassa pa꜕ṭighātāya, \pause\\
ḍaṃsa-maka꜕sa꜕-vātāta꜕pa꜕-siriṃsapa-samphassānaṃ pa꜕ṭighātāya, \pause\\
yāvadeva utupa꜕rissaya vi꜕nodanaṃ \pause\ pa꜕ṭisa꜓llānārāmatthaṃ

\begin{english}
  Reflectindo sabiamente eu uso o alojamento: \pause\ Somente para evitar o
  frio, \pause\ o calor, \pause\ as moscas, \pause\ mosquitos, \pause\ bichos
  rastejantes, \pause\ o vento e as coisas que queimam. \pause\ Somente para me
  abrigar dos perigos da natureza \pause\ e viver em recolhimento.
\end{english}

[Paṭisaṅkhā] yoniso gi꜕lāna-pacca꜕ya꜕-bhesajja-pa꜕rikkhāraṃ\\
pa꜕ṭise꜓vāmi, \pause\ yāvadeva uppa꜓nnānaṃ veyyābādhi꜕kānaṃ veda꜕nānaṃ
pa꜕ṭighātāya, \pause\ a꜕byāpajjha-pa꜕ramatāyā'ti

\begin{english}
  Reflectindo sabiamente \pause\ eu uso o apoio necessário para medicamentos e
  enfermidades: \pause\ Somente para aliviar as dores que tenham surgido,
  \pause\ de forma a ficar o mais possível livre de doenças.
\end{english}

\chapter[Trinta-e-duas-Partes]{Reflexão sobre as Trinta-e-duas-Partes}

\firstline{Ayaṃ kho me kāyo}

\begin{leader}
  [Ha꜓nda mayaṃ dvattiṃsākāra-pāṭhaṃ bhaṇāmase]
\end{leader}

[Ayaṃ kho] me kāyo uddhaṃ pāda꜕ta꜕lā adho kesamatthakā\\
ta꜕ca꜕pa꜕ri꜕yanto pūro nānappa꜕kārassa꜕ a꜕su꜕ci꜕no

\begin{english}
  Isto, que é o meu corpo, das plantas dos pés para cima, e do topo da cabeça para baixo, é um saco de pele fechado cheio de coisas repugnantes.
\end{english}

Atthi imasmiṃ kāye

\begin{english}
  Neste corpo existem:
\end{english}

{\centering
\setArrayStrech{1}

\begin{tabular}{ r l }
kesā            & \tr{cabelo} \\
lomā            & \tr{pêlos} \\
nakhā           & \tr{unhas} \\
dantā           & \tr{dentes} \\
taco            & \tr{pele} \\
maṃsaṃ          & \tr{carne} \\
nahārū          & \tr{tendões} \\
aṭṭhī           & \tr{ossos} \\
aṭṭhimiñjaṃ     & \tr{medula óssea} \\
vakkaṃ          & \tr{rins} \\
hadayaṃ         & \tr{coração} \\
yakanaṃ         & \tr{fígado} \\
kilomakaṃ       & \tr{membranas} \\
pihakaṃ         & \tr{baço} \\
papphāsaṃ       & \tr{pulmões} \\
\end{tabular}

\clearpage

\begin{tabular}{ r l }
antaṃ           & \tr{intestinos} \\
antaguṇaṃ       & \tr{tripas} \\
udariyaṃ        & \tr{comida não digerida} \\
karīsaṃ         & \tr{excremento} \\
pittaṃ          & \tr{bílis} \\
semhaṃ          & \tr{muco} \\
pubbo           & \tr{pus} \\% TODO: is this translated?
lohitaṃ         & \tr{sangue} \\
sedo            & \tr{suor} \\
medo            & \tr{gordura} \\
assu            & \tr{lágrimas} \\
vasā            & \tr{sebo} \\
kheḷo           & \tr{saliva} \\
siṅghāṇikā      & \tr{mucosidade} \\
lasikā          & \tr{lubrificante das articulações} \\
muttaṃ          & \tr{urina} \\
matthaluṅgan'ti & \tr{cérebro} \\
\end{tabular}

\restoreArrayStretch
}

Evam-ayaṃ me kāyo uddhaṃ pāda꜕ta꜕lā adho kesamatthakā\\
ta꜕ca꜕pa꜕ri꜕yanto pūro nānappa꜕kārassa꜕ a꜕su꜕ci꜕no

\begin{english}
  Assim, isto que é o meu corpo, das plantas dos pés para cima, e do topo da cabeça para baixo, é um saco de pele fechado cheio de coisas repugnantes.
\end{english}

\chapter[Cinco Temas]{Cinco Temas para Recordar Frequentemente}

\firstline{Jarā-dhammomhi jaraṃ anatīto}

\begin{leader}
  [Ha꜓nda mayaṃ abhiṇha-paccave꜕kkhaṇa-pāṭhaṃ bhaṇāmase]
\end{leader}

\sidepar{Homens}%
[Jarā-dhammomhi꜕] jaraṃ a꜕na꜕tīto

\sidepar{Mulheres}%
[Jarā-dhammāmhi꜕] jaraṃ a꜕na꜕tītā

\begin{english}
  A minha natureza é envelhecer, ainda não estou para além do envelhecimento.
\end{english}

\sidepar{h.}%
Byādhi꜓-dhammomhi꜕ byādhiṃ a꜕na꜕tīto

\sidepar{m.}%
Byādhi꜓-dhammāmhi꜕ byādhiṃ a꜕na꜕tītā

\begin{english}
  A minha natureza é adoecer, ainda não estou para além da doença.
\end{english}

\sidepar{h.}%
Ma꜕raṇa-dhammomhi꜕ ma꜕raṇaṃ a꜕na꜕tīto

\sidepar{m.}%
Ma꜕raṇa-dhammāmhi꜕ ma꜕raṇaṃ a꜕na꜕tītā

\begin{english}
  A minha natureza é morrer, ainda não estou para além da morte.
\end{english}

Sa꜕bbehi me pi꜕yehi ma꜕nāpehi꜕ nānābhāvo vi꜕nābhāvo

\begin{english}
  Tudo o que é meu, amado e agradável,\\
  tornar-se-á diferente, separar-se-á de mim.
\end{english}

\sidepar{h.}%
Kammassa꜕komhi kamma꜓dāyādo kamma꜕yoni kamma꜕bandhu kammapa꜕ṭisa꜓ra꜕ṇo\\
Yaṃ kammaṃ ka꜕rissāmi, kalyāṇaṃ vā pāpa꜕kaṃ vā, tassa꜕ dāyādo bha꜕vissāmi

\clearpage

\sidepar{m.}%
Kammassa꜕kāmhi kamma꜓dāyādā kamma꜕yoni kamma꜕bandhu kammapa꜕ṭisa꜓ra꜕ṇā\\
Yaṃ kammaṃ ka꜕rissāmi, kalyāṇaṃ vā pāpa꜕kaṃ vā, tassa꜕ dāyādā bha꜕vissāmi

\begin{english}
  Sou o dono do meu Kamma, herdeiro do meu Kamma,\\
  nascido do meu Kamma, ligado ao meu Kamma,\\
  permaneço suportado pelo meu Kamma; seja qual Kamma eu criar,\\
  Para o bem ou para o mal, \prul{disso} serei o herdeiro.
\end{english}

Evaṃ amhehi꜕ a꜕bhiṇhaṃ pacca꜕vekkhi꜓tabbaṃ

\begin{english}
  \prul{Assim} deveríamos frequentemente reflectir.
\end{english}

\chapter[Dez Temas]{Dez Temas para Recordar Frequentemente por Aqueles que Seguem o Caminho}

\firstline{Dasa ime bhikkhave}

\begin{leader}
  [Ha꜓nda mayaṃ pabbajita\hyp{}abhiṇha\hyp{}paccave꜕kkhaṇa\hyp{}pāṭhaṃ bhaṇāmase]
\end{leader}

[Dasa i꜕me bhikkhave] dhammā pabba꜕jitena a꜕bhiṇhaṃ pacca꜕vekkhi꜓tabbā, \pause\ ka꜕ta꜕me dasa

\begin{english}
  Monges, existem dez dhammas \pause\ sobre os quais se deve reflectir frequentemente. \pause\ \prul{Quais} são estes dez dhammas?
\end{english}

Vevaṇṇi꜕yamhi ajjhūpa꜕ga꜕to'ti pabba꜕jitena a꜕bhiṇhaṃ pacca꜕vekkhi꜓tabbaṃ

\begin{english}
  `Já não vivo segundo os valores e objectivos do mundo.' \pause\\
  Quem perfaz o caminho \pause\ deve reflectir sobre isto frequentemente.
\end{english}

Parapaṭi꜕baddhā me jīvi꜓kā'ti pabba꜕jitena a꜕bhiṇhaṃ pacca꜕vekkhi꜓tabbaṃ

\begin{english}
  `A minha própria vida é sustentada \pause\ pela generosidade dos outros.' \pause\\
  Quem perfaz o caminho \pause\ deve reflectir sobre isto frequentemente.
\end{english}

Añño me ākappo ka꜕ra꜕ṇīyo'ti pabba꜕jitena a꜕bhiṇhaṃ pacca꜕vekkhi꜓tabbaṃ

\begin{english}
  `Devo esforçar-me por abandonar os meus hábitos antigos.' \pause\\
  Quem perfaz o caminho \pause\ deve reflectir sobre isto frequentemente.
\end{english}

\clearpage

Kacci nu꜕ kho me attā sīla꜕to na u꜕pavadatī'ti pabba꜕jitena a꜕bhiṇhaṃ pacca꜕vekkhi꜓tabbaṃ

\begin{english}
  `Surgem remorsos na minha mente \pause\ em relação à minha conduta?' \pause\\
  Quem perfaz o caminho \pause\ deve reflectir sobre isto frequentemente.
\end{english}

Kacci nu꜕ kho maṃ a꜕nuvicca viññū sabrahma꜓cārī sīla꜕to na u꜕pavadantī'ti pabba꜕jitena a꜕bhiṇhaṃ pacca꜕vekkhi꜓tabbaṃ

\begin{english}
  `Será que os meus companheiros espirituais \pause\\\
  acham falhas na minha conduta?' \pause\\
  Quem perfaz o caminho \pause\ deve reflectir sobre isto frequentemente.
\end{english}

Sa꜕bbehi me pi꜕yehi ma꜕nāpehi꜕ nānābhāvo vi꜕nābhāvo'ti pabba꜕jitena abhiṇhaṃ pacca꜕vekkhi꜓tabbaṃ

\begin{english}
  `Tudo aquilo que é meu, \pause\ que amo e prezo, \pause\ tornar-se-á diferente, \pause\ separar-se-á de mim.' \pause\\
  Quem perfaz o caminho \pause\ deve reflectir sobre isto frequentemente.
\end{english}

Kammassa꜕komhi kamma꜓dāyādo kamma꜕yoni kamma꜕bandhu kammapa꜕ṭisa꜓raṇo, yaṃ kammaṃ ka꜕rissāmi, kalyāṇaṃ vā pāpa꜕kaṃ vā, tassa꜕ dāyādo bha꜕vissāmī'ti pabba꜕jitena a꜕bhiṇhaṃ pacca꜕vekkhi꜓tabbaṃ

\begin{english}
  `Sou o dono do meu Kamma, \pause\ herdeiro do meu Kamma, \pause\\
  nascido do meu Kamma, \pause\ ligado ao meu Kamma, \pause\\
  permaneço suportado pelo meu Kamma; \pause\ seja qual Kamma eu criar, \pause\\
  Para o bem ou para o mal, \pause\ \prul{disso} serei o herdeiro.' \pause\\
  Quem perfaz o caminho \pause\ deve reflectir sobre isto frequentemente.
\end{english}

\clearpage

`Kathambhūtassa꜕ me rattindi꜕vā vīti꜕pa꜓tantī'ti pabba꜕jitena a꜕bhiṇhaṃ pacca꜕vekkhi꜓tabbaṃ

\begin{english}
  `Os dias e as noites passam continuamente; \pause\\
  Como estou eu a usar o meu tempo?' \pause\\
  Quem perfaz o caminho \pause\ deve reflectir sobre isto frequentemente.
\end{english}

Kacci nu꜕ kho'haṃ suññā꜓gāre abhira꜕māmī'ti pabba꜕jitena a꜕bhiṇhaṃ pacca꜕vekkhi꜓tabbaṃ

\begin{english}
  `Aprecio a solidão ou não?' \pause\\
  Quem perfaz o caminho \pause\ deve reflectir sobre isto frequentemente.
\end{english}

Atthi nu꜕ kho me uttari-ma꜕nussa-dhammā alamariya꜕-ñāṇa-dassana-viseso adhiga꜕to, so'haṃ pacchi꜓me kāle sa꜕brahmacārīhi꜕ puṭṭho na maṅku bha꜕vissāmī'ti pabba꜕jitena a꜕bhiṇhaṃ pacca꜕vekkhi꜓tabbaṃ

\begin{english}
  `Deu a minha prática frutos de compreensão e liberdade, \pause\\ de forma a que
  no fim da minha vida \pause\ eu não me sinta envergonhado \pause\\
  quando questionado \pause\ pelos meus companheiros espirituais?' \pause\\
  Quem perfaz o caminho \pause\ deve reflectir sobre isto frequentemente.
\end{english}

Ime kho bhikkha꜓ve da꜕sa꜕ dhammā pabba꜕jitena a꜕bhiṇhaṃ pacca꜕vekkhitabbā'ti

\begin{english}
  Monges estes são dez Dhammas \pause\ sobre os quais se deve reflectir frequentemente.
\end{english}

\chapter{Ovāda-Pāṭimokkha}

\firstline{Khantī paramaṃ tapo tītikkhā}

\enlargethispage{\baselineskip}

\begin{leader}
  [Ha꜓nda mayaṃ ovāda-pā꜕ṭi꜕mokkha-gāthā꜓yo bha꜕ṇāmase]
\end{leader}

Kha꜓ntī pa꜕ramaṃ ta꜕po tīti꜕kkhā

\begin{english}
  Permanecer paciente é a maior austeridade.
\end{english}

Nibbānaṃ pa꜕ramaṃ va꜕dant꜕i buddhā

\begin{english}
  “Nibbāna é supremo”, dizem os Buddhas.
\end{english}

Na h꜕i pa꜕bbaji꜕to pa꜕rūpaghātī

\begin{english}
  Não se é verdadeiramente monge \pause\ quando se prejudica alguém,
\end{english}

Sa꜕maṇo ho꜓ti pa꜕raṃ vihe꜓ṭha꜕yanto

\begin{english}
  nem verdadeiramente renunciante \pause\ quando se oprime os outros.
\end{english}

Sa꜕bb꜕a-pāpa꜕ss꜕a a꜕ka꜕ra꜓ṇaṃ

\begin{english}
  Evitar todo o mal,
\end{english}

Ku꜕salassūpasa꜓mpa꜕dā

\begin{english}
  cultivar o bem
\end{english}

Sa꜕ci꜕tta-pa꜕ri꜓yoda꜓pa꜕naṃ

\begin{english}
  e purificar a mente --
\end{english}

Etaṃ buddhāna sā꜓sa꜕naṃ

\begin{english}
  Este é o ensinamento dos Buddhas.
\end{english}

A꜕nūpa꜕vādo a꜕nūpa꜕ghāto

\begin{english}
  Não ofender, não prejudicar,
\end{english}

Pā꜕ṭimokkhe꜓ ca꜕ sa꜓ṃva꜕ro

\begin{english}
  conter-se de acordo com o código monástico,
\end{english}

Mattaññu꜕tā ca꜕ bhatta꜕smiṃ

\begin{english}
  moderar-se na comida,
\end{english}

Pa꜕ntañca꜕ saya꜓n'āsa꜕naṃ

\begin{english}
  viver solitário,
\end{english}

A꜕dhici꜕tte ca꜕ āyogo

\begin{english}
  devotar-se à consciência elevada --
\end{english}

Etaṃ buddhāna sā꜓sa꜕naṃ'ti.

\begin{english}
  Este é o ensinamento dos Buddhas.
\end{english}

\chapter[A Primeira Exortação]{Versos sobre a Primeira Exortação do Buddha}

\firstline{Aneka-jāti-saṃsāraṃ}

\begin{leader}
  [Ha꜓nda mayaṃ paṭhama-bu꜕ddha-bhāsi꜕ta-gāthāyo bh꜕aṇāmase]
\end{leader}

\begin{twochants}
  A꜕neka꜕-jāti꜕-sa꜓ṃsā꜓raṃ & sa꜕ndhāviss꜓aṃ a꜕nibbi꜕saṃ \\
  Ga꜕ha-kā꜕raṃ ga꜕vesa꜓nto & dukkhā jāt꜕i pu꜕nappu꜕naṃ \\
\end{twochants}

\begin{english}
  Durante muitas vidas, na roda da vida e da morte\\
  Vagueei infinitamente,\\
  O constructor desta casa eu busquei,\\
  Quão penosos são os repetidos nascimentos.
\end{english}

\begin{twochants}
  Ga꜕ha-kā꜕raka꜕ diṭṭho꜓'si & pu꜕na gehaṃ na kā꜓hasi \\
  Sa꜕bbā te phāsu꜕kā bhaggā & gaha-kūṭa꜓ṃ vi꜕saṅkh꜕ataṃ \\
  Visa꜓ṅkhā꜕ra-ga꜕taṃ ci꜕ttaṃ & taṇhānaṃ kh꜕aya꜕m-ajjh꜕agā \\
\end{twochants}

\begin{english}
  Ó construtor da casa, foste visto!\\
  Não construirás casa novamente.\\
  Todas as tuas vigas estão quebradas\\
  e a cumeeira esmagada.\\
  A mente atingiu o Incondicionado;\\
  chegando à cessação do desejo.
\end{english}

\chapter[As Últimas Instruções]{Versos sobre as Últimas Instruções}

\firstline{Handa dāni bhikkhave āmantayāmi vo}

\begin{leader}
  [Ha꜓nda mayaṃ pacchima-ovāda-gāthā꜓yo bha꜕ṇāmase]
\end{leader}

Handa dāni bhi꜓kkha꜕ve āmant꜕ayāmi꜓ vo

\begin{english}
  Agora monges, eu vos digo,
\end{english}

Vaya-dhammā sa꜓ṅkhā꜓rā

\begin{english}
  A mudança é a natureza das coisas condicionadas;
\end{english}

A꜕ppamādena sa꜓mpā꜕dethā'ti

\begin{english}
  Aperfeiçoem-se, não sendo negligentes:
\end{english}

Ayaṃ tathā꜓ga꜕tassa pa꜕cchi꜓mā vācā

\begin{english}
  Estas são as últimas palavras do Tathāgata.
\end{english}

\chapter{The Root of All Things}

% AN 10.58
% https://www.accesstoinsight.org/tipitaka/an/an10/an10.058.than.html
% https://www.dhammatalks.org/suttas/AN/AN10_58.html
% https://suttacentral.net/an10.58/pli/ms
% https://suttacentral.net/an10.58/en/sujato
% https://suttacentral.net/an10.58/en/bodhi

\firstline{Kiṃ-mūlakā āvuso sabbe dhammā}

\begin{leader}
  [Ha꜓nda mayam mūlaka-sutta-pāṭhaṃ bha꜕ṇāmase]
\end{leader}

\enlargethispage{2\baselineskip}

Kiṃ-mūlakā āvuso sabbe dhammā\\
kiṃ-sambhavā sabbe dhammā\\
kiṃ-samudayā sabbe dhammā\\
kiṃ-samosaraṇā sabbe dhammā\\
kiṃ-pamukhā sabbe dhammā\\
kiṃ-adhipateyyā sabbe dhammā\\
kiṃ-uttarā sabbe dhammā\\
kiṃ-sārā sabbe dhammā\\
kiṃ-ogadhā sabbe dhammā\\
kiṃ-pariyosānā sabbe dhammā'ti.

Evaṃ puṭṭhā tumhe bhikkhave tesaṃ aññatitthiyānaṃ paribbājakānaṃ evaṃ byākareyyātha:

Chanda'mūlakā āvuso sabbe dhammā\\
manasikāra'sambhavā sabbe dhammā\\
phassa'samudayā sabbe dhammā\\
vedanā'samosaraṇā sabbe dhammā\\
samādhi'ppamukhā sabbe dhammā\\
satā'dhipateyyā sabbe dhammā\\
paññ'uttarā sabbe dhammā\\
vimutti'sārā sabbe dhammā\\
amat'ogadhā sabbe dhammā\\
nibbāna'pariyosānā sabbe dhammā'ti.

Evaṃ puṭṭhā tumhe bhikkhave tesaṃ aññatitthiyānaṃ paribbājakānaṃ evaṃ byākareyyāthā'ti.

\begin{english}
  Rooted in what, friend, are all things?\\
  Born of what, are all things?\\
  Arising from what, are all things?\\
  Converging on what, are all things?\\
  Headed by what, are all things?\\
  Dominated by what, are all things?\\
  Surmountable by what, are all things?\\
  Yielding what as essence, are all things?\\
  Merging in what, are all things?\\
  Terminating in what, are all things?

  \bigskip

  When questioned by wanderers, thus you should answer them:

  \bigskip

  Rooted in desire, friend, are all things.\\
  Born of attention, are all things.\\
  Arising from contact, are all things.\\
  Converging on feeling are all things.\\
  Headed by concentration are all things.\\
  Dominated by mindfulness are all things.\\
  Surmountable by wisdom are all things.\\
  Yielding deliverance as essence are all things.\\
  Merging in the Deathless are all things.\\
  Terminating in Nibbāna are all things.

  \bigskip

  When questioned by wanderers, thus you should answer them.
\end{english}

{\raggedleft
  \emph{Aṅguttara Nikāya 10.58}
\par}

\chapter[Incondicionado]{Reflexão sobre o Incondicionado}

\firstline{Atthi bhikkhave ajātaṃ abhūtaṃ akataṃ}

\begin{leader}
  [Ha꜓nda mayaṃ nibbāna-sutta-pāṭhaṃ bha꜕ṇāmase]
\end{leader}

Atthi bhi꜓kkha꜕ve a꜕jātaṃ a꜓bhūtaṃ a꜕kataṃ a꜕sa꜓ṅkh꜕ataṃ

\begin{english}
  Existe um Não-nascido, Não-originado, Incriado, Não-formado.
\end{english}

N꜕o cetaṃ bhi꜓kkha꜕ve a꜕bhavissa a꜕jātaṃ a꜓bhūtaṃ a꜕kataṃ a꜕sa꜓nkh꜕ataṃ

\begin{english}
 Se não existisse este Não-nascido, Não-originado, Incriado, Não-formado,
\end{english}

Na꜕ yidaṃ jātassa꜕ bhūtassa ka꜕tassa sa꜓ṅkh꜕atassa nissaraṇaṃ paññāye꜓tha

\begin{english}
  A libertação do mundo do nascido, originado, criado, formado, não seria possível.
\end{english}

Ya꜕smā ca kho bhi꜓kkh꜕ave atthi a꜕jātaṃ a꜓bhūtaṃ a꜕kataṃ a꜕sa꜓ṅkha꜕taṃ

\begin{english}
  Mas uma vez que existe um Não-nascido, Não-originado, Incriado, Não-formado,
\end{english}

Ta꜕smā jātass꜕a bhūtassa ka꜕tassa sa꜓ṅkha꜕tassa nissaraṇaṃ paññāyati

\begin{english}
  Assim é possível a libertação do mundo do nascido, originado, criado, formado.
\end{english}

\chapter{The Teaching on Mindfulness of Breathing}

\firstline{Ānāpānassati bhikkhave bhāvitā bahulī-katā}

\begin{leader}
  [Ha꜓nda mayam ānāpānass꜕ati-sutta-pāṭhaṃ bha꜕ṇāmase]
\end{leader}

Ānāpāna꜓ssa꜕ti bhi꜓kkha꜕ve bhāvi꜓tā bahu꜕līka꜕tā

\begin{english}
  Bhikkhus, wh꜕en mindfulness of bre꜓athing is de꜕veloped and cu꜕ltiva꜓ted
\end{english}

Mahappha꜕lā ho꜓ti mahā꜓nisa꜓ṃsā

\begin{english}
  It is of gre꜕at fruit and great be꜕nefit;
\end{english}

Ānāpāna꜓ssa꜕ti bhi꜓kkha꜕ve bhāvi꜓tā bahu꜕līka꜕tā

\begin{english}
  Wh꜕en mindfulness of bre꜓athing is de꜕veloped and cu꜕ltiva꜓ted
\end{english}

Ca꜕ttāro sati꜓pa꜕ṭṭhāne pa꜕ri꜓pū꜕reti

\begin{english}
  It fu꜕lfills the Four Foundations of Mi꜕ndfu꜕lness;
\end{english}

Ca꜕ttāro sa꜕tipa꜕ṭṭhānā bhāvi꜓tā bahu꜕līka꜕tā

\begin{english}
  When th꜕e Four Foundations of Mi꜓ndfulness are de꜕veloped and cu꜕ltiva꜓ted
\end{english}

Sa꜕tta-bojjhaṅge pa꜕ri꜓pū꜕renti

\begin{english}
  They fu꜕lfill the Seven Factors of Awa꜕kening;
\end{english}

Sa꜕tta-bojjhaṅgā bhāvi꜓tā bahu꜕līka꜕tā

\begin{english}
  When th꜕e Seven Factors of Awa꜓kening are de꜕veloped and cu꜕ltiva꜓ted
\end{english}

Vijjā-vimuttiṃ pa꜕ri꜓pū꜕renti

\begin{english}
  They fu꜕lfill true knowledge and deli꜕verance.
\end{english}

Kathaṃ bhāvi꜓tā ca bhi꜓kkha꜕ve ānāpāna꜓ss꜕ati ka꜕thaṃ bahu꜕līka꜕tā

\begin{english}
  An꜕d how, bhikkhus, is mindfulness of bre꜓athing de꜕veloped and cu꜕ltiva꜓ted
\end{english}

Mahappha꜕lā ho꜓ti mahā꜓nisa꜓ṃsā

\begin{english}
  So that it is of gre꜕at fruit and great be꜕nefit?
\end{english}

Idha bhi꜓kkha꜕ve bhikkhu

\begin{english}
  Here, bhikkhus, a bhi꜕kkhu,
\end{english}

Arañña꜓-ga꜕to vā

\begin{english}
  Gone to꜕ the fo꜓rest,
\end{english}

Rukkha-mūla꜓-ga꜕to vā

\begin{english}
  To the fo꜕ot o꜕f a꜕ tree
\end{english}

Suññāgāra꜓-ga꜕to vā

\begin{english}
  Or to an em꜓pty꜕ hut.
\end{english}

N꜕isīdati pallaṅkaṃ ābhuji꜓tv꜕ā

\begin{english}
  Si꜕ts down having cro꜕ssed hi꜕s legs,
\end{english}

Ujuṃ kāyaṃ pa꜕ṇidhāya pa꜕rimukhaṃ sa꜕tiṃ u꜕paṭṭha꜕petvā

\begin{english}
  Sets his bo꜕dy꜕ e꜕rect, having established mi꜓ndfulness in fro꜕nt o꜕f him.
\end{english}

So sa꜕to'va a꜕ssasa꜕ti sa꜕to'va pa꜕ssa꜕sa꜕ti

\begin{english}
  Ever mi꜓ndful he bre꜕athes in; mindful h꜕e bre꜕athes out.
\end{english}

Dīghaṃ vā assa꜕sa꜓nto dīghaṃ a꜕ssasā꜓mī'ti pa꜕jānāti

\begin{english}
  Breathing i꜓n long, he꜕ knows `I bre꜕athe i꜕n long';
\end{english}

Dīghaṃ vā pa꜕ssa꜕santo dīghaṃ pa꜕ssasā꜓mī'ti pa꜕jānāti

\begin{english}
  Breathing ou꜕t long, he꜕ knows `I bre꜕athe ou꜕t long';
\end{english}

Rassaṃ vā a꜕ssa꜕santo rassaṃ a꜕ssasā꜓mī'ti pa꜕jānāti

\begin{english}
  Breathing i꜓n short, h꜕e knows `I bre꜕athe i꜕n short';
\end{english}

Rassaṃ vā pa꜕ssa꜕santo rassaṃ pa꜕ssasā꜓mī'ti pa꜕jānāti

\begin{english}
  Breathing ou꜕t short, h꜕e knows `I bre꜕athe ou꜕t short'.
\end{english}

Sabba꜕-kāya-paṭ꜕isa꜓ṃvedī a꜕ssasi꜕ssāmī'ti si꜕kkh꜕ati

\begin{english}
  He tra꜕ins thus: `I shall breathe i꜓n experiencing the whole bo꜕dy'.
\end{english}

Sabba꜕-kāya-paṭ꜕isa꜓ṃvedī pa꜕ssasi꜕ssāmī'ti si꜕kkh꜕ati

\begin{english}
  He tra꜕ins thus: `I shall breathe ou꜕t e꜕xpe꜕ri꜕enci꜕ng th꜕e who꜕le bo꜕dy'.
\end{english}

Passa꜕mbhayaṃ kāya꜕-sa꜓ṅkhāraṃ a꜕ssasi꜕ssāmī'ti si꜕kkh꜕ati

\begin{english}
  He tra꜕ins thus: `I shall breathe i꜓n tranquillizing the bodily forma꜕tions'.
\end{english}

Passa꜕mbhayaṃ kāya꜕-sa꜓ṅkhāraṃ pa꜕ssasi꜕ssāmī'ti si꜕kkh꜕ati

\begin{english}
  He tra꜕ins thus: `I shall breathe ou꜕t tra꜕nqui꜕ll꜕izi꜕ng th꜕e bo꜕dily fo꜕rmations'.
\end{english}

Pīti꜕-paṭi꜕sa꜓ṃvedī a꜕ssasi꜕ssāmī'ti si꜕kkh꜕ati

\begin{english}
  He tra꜕ins thus: `I shall breathe i꜓n experiencing ra꜕pture'.
\end{english}

Pīti꜕-paṭi꜕sa꜓ṃvedī pa꜕ssasi꜕ssāmī'ti si꜕kkh꜕ati

\begin{english}
  He tra꜕ins thus: `I shall breathe ou꜕t e꜕xpe꜕ri꜕enci꜕ng ra꜕pture'.
\end{english}

Sukh꜕a-paṭi꜕sa꜓ṃvedī a꜕ssasi꜕ssāmī'ti si꜕kkh꜕ati

\begin{english}
  He tra꜕ins thus: `I shall breathe i꜓n experiencing ple꜕asure'
\end{english}

Sukh꜕a-paṭi꜕sa꜓ṃvedī pa꜕ssasi꜕ssāmī'ti si꜕kkh꜕ati

\begin{english}
  He tra꜕ins thus: `I shall breathe ou꜕t e꜕xpe꜕ri꜕enci꜕ng ple꜕asure'.
\end{english}

Citta꜕-sa꜓ṅkhāra-paṭi꜕sa꜓ṃvedī a꜕ssasi꜕ssāmī'ti si꜕kkh꜕ati

\begin{english}
  He tra꜕ins thus: `I shall breathe i꜓n experiencing the mental forma꜕tions'.
\end{english}

Citta꜕-sa꜓ṅkhāra-paṭi꜕sa꜓ṃvedī pa꜕ssasi꜕ssāmī'ti si꜕kkh꜕ati

\begin{english}
  He tra꜕ins thus: `I shall breathe ou꜕t e꜕xpe꜕ri꜕enci꜕ng th꜕e me꜕nta꜕l fo꜕rma꜕tions'.
\end{english}

Passa꜕mbhayaṃ citta꜕-sa꜓ṅkhāraṃ a꜕ssasi꜕ssāmī'ti si꜕kkh꜕ati

\begin{english}
  He tra꜕ins thus: `I shall breathe i꜓n tranquillizing the mental forma꜕tions'.
\end{english}

Passa꜕mbhayaṃ citt꜕a-sa꜓ṅkhāraṃ pa꜕ssasi꜕ssāmī'ti si꜕kkh꜕ati

\begin{english}
  He tra꜕ins thus: `I shall breathe ou꜕t tra꜕nqu꜕ill꜕izi꜕ng th꜕e me꜕nta꜕l fo꜕rma꜕tions'.
\end{english}

Citta꜕-paṭi꜕sa꜓ṃvedī a꜕ssasi꜕ssāmī'ti si꜕kkh꜕ati

\begin{english}
  He tra꜕ins thus: `I shall breathe i꜓n experiencing th꜕e mind'.
\end{english}

Citta꜕-paṭi꜕sa꜓ṃvedī pa꜕ssasi꜕ssāmī'ti si꜕kkh꜕ati

\begin{english}
  He tra꜕ins thus: `I shall breathe ou꜕t e꜕xpe꜕ri꜕enci꜕ng th꜕e mind'.
\end{english}

A꜕bhippa꜕moda꜓yaṃ cittaṃ a꜕ssasi꜕ssāmī'ti si꜕kkh꜕ati

\begin{english}
  He tra꜕ins thus: `I shall breathe i꜓n gladdening th꜕e mind'.
\end{english}

A꜕bhippa꜕moda꜓yaṃ cittaṃ pa꜕ssasi꜕ssāmī'ti si꜕kkh꜕ati

\begin{english}
  He tra꜕ins thus: `I shall breathe ou꜕t gl꜕adde꜕ni꜕ng th꜕e mind'.
\end{english}

Sa꜕māda꜓haṃ cittaṃ a꜕ssasi꜕ssāmī'ti si꜕kkh꜕ati

\begin{english}
  He tra꜕ins thus: `I shall breathe i꜓n concentrating th꜕e mind'
\end{english}

Sa꜕māda꜓haṃ cittaṃ pa꜕ssasi꜕ssāmī'ti si꜕kkh꜕ati

\begin{english}
  He tra꜕ins thus: `I shall breathe ou꜕t co꜕nce꜕ntr꜕ati꜕ng th꜕e mind'.
\end{english}

Vimoca꜓yaṃ cittaṃ a꜕ssasi꜕ssāmī'ti si꜕kkh꜕ati

\begin{english}
  He tra꜕ins thus: `I shall breathe i꜓n liberating th꜕e mind'.
\end{english}

Vimoca꜓yaṃ cittaṃ pa꜕ssasi꜕ssāmī'ti si꜕kkh꜕ati

\begin{english}
  He tra꜕ins thus: `I shall breathe ou꜕t li꜕be꜕ra꜕ti꜕ng th꜕e mind'.
\end{english}

Aniccānupa꜕ssī a꜕ssasi꜕ssāmī'ti si꜕kkh꜕ati

\begin{english}
  He tra꜕ins thus: `I shall breathe i꜓n contemplating impe꜕rmanence'.
\end{english}

Aniccānupa꜕ssī pa꜕ssasi꜕ssāmī'ti si꜕kkh꜕ati

\begin{english}
  He tra꜕ins thus: `I shall breathe ou꜕t co꜕nte꜕mpla꜕ti꜕ng i꜕mpe꜕rmanence'.
\end{english}

Virāgānupa꜕ssī a꜕ssasi꜕ssāmī'ti si꜕kkh꜕ati

\begin{english}
  He tra꜕ins thus: `I shall breathe i꜓n contemplating the fading away of pa꜕ssions'.
\end{english}

Virāgānupa꜕ssī pa꜕ssasi꜕ssāmī'ti si꜕kkh꜕ati

\begin{english}
  He tra꜕ins thus: `I shall breathe ou꜕t co꜕nte꜕mpl꜕ati꜕ng th꜕e fa꜕di꜕ng aw꜕ay o꜕f pa꜕ssions'.
\end{english}

Nirodhānupa꜕ssī a꜕ssasi꜕ssāmī'ti si꜕kkh꜕ati

\begin{english}
  He tra꜕ins thus: `I shall breathe i꜓n contemplating cessa꜕tion'.
\end{english}

Nirodhānupa꜕ssī pa꜕ssasi꜕ssāmī'ti si꜕kkh꜕ati

\begin{english}
  He tra꜕ins thus: `I shall breathe ou꜕t co꜕nte꜕mpl꜕ati꜕ng ce꜕ss꜕ation'.
\end{english}

Pa꜕ṭiniss꜕aggānupa꜕ssī a꜕ssasi꜕ssāmī'ti si꜕kkh꜕ati

\begin{english}
  He tra꜕ins thus: `I shall breathe i꜓n contemplating reli꜕nquishment'.
\end{english}

Pa꜕ṭinissa꜕ggānupa꜕ssī pa꜕ssasi꜕ssāmī'ti si꜕kkh꜕ati

\begin{english}
  He tra꜕ins thus: `I shall breathe ou꜕t co꜕nte꜕mpl꜕ati꜕ng re꜕li꜕nquishment'.
\end{english}

Evaṃ bhāvi꜓tā kho bhi꜓kkha꜕ve ānāpāna꜓ss꜕ati evaṃ bahu꜕līka꜕tā

\begin{english}
  Bhikkhus, that is ho꜕w mindfulness of bre꜓athing is de꜕veloped and cu꜕ltiva꜓ted
\end{english}

Mahappha꜕lā ho꜓ti mahā꜓nisa꜓ṃsā'ti

\begin{english}
  So that it is of gr꜕eat fruit and great be꜕nefit.
\end{english}


\chapter[Aparihānīya-dhamma-sutta]{Bhikkhu-aparihānīya-dhamma-sutta}

\emph{Seven Conditions for the Welfare of the Bhikkhus, AN 7.23}

\begin{leader}
  [Handa mayaṃ bhikkhu-aparihānīyā-dhammā-sutta-pāṭhaṃ bhaṇāmase]
\end{leader}

[Evaṃ me sutaṃ.] Ekaṃ samayaṃ bhagavā rājagahe꜔꜒ viharati gijjhakūṭe pabbate.
Tatra kho꜔꜒ bhagavā bhikkhū꜔꜒ āmantesi: Satta vo, bhikkhave, aparihā꜔꜒niye dhamme
desessā꜔꜒mi. Taṃ suṇātha, sā꜔꜒dhukaṃ manasi karotha, bhāsissā꜔꜒mī'ti. Evaṃ, bhante'ti
kho꜔꜒ te bhikkhū꜔꜒ bhagavato paccasso꜔꜒su꜔꜒ṃ. Bhagavā etadavoca:

\begin{english}
  I have heard that on one occasion the Blessed One was staying in Rajagaha, on
  Vulture Peak. There he addressed the monks: “Monks, I will teach you the seven
  conditions that lead to no decline. Listen \& pay close attention. I will
  speak.” “Yes, lord,” the monks responded. The Blessed One said:
\end{english}

[1] Katame ca, bhikkhave, satta aparihā꜔꜒niyā dhammā? Yāvakīvañca, bhikkhave, bhikkhū꜔꜒
abhiṇha꜔꜒ṃ sa꜔꜒nnipātā bhavissa꜔꜒nti sa꜔꜒nnipātabahulā; vuddhiyeva, bhikkhave,
bhikkhū꜔꜒naṃ pāṭikaṅkhā꜔꜒, no parihā꜔꜒ni.

\begin{english}
  “And which seven are the conditions that lead to no decline? As long as the
  monks meet often, meet a great deal, their growth can be expected, not their
  decline.
\end{english}

[2] Yāvakīvañca, bhikkhave, bhikkhū꜔꜒ samaggā sa꜔꜒nnipatissa꜔꜒nti, samaggā
vuṭṭhahissa꜔꜒nti, samaggā sa꜔꜒ṅghakaraṇīyāni karissa꜔꜒nti; vuddhiyeva, bhikkhave,
bhikkhū꜔꜒naṃ pāṭikaṅkhā꜔꜒, no parihā꜔꜒ni.

\begin{english}
  “As long as the monks meet in harmony, adjourn from their meetings in harmony,
  and conduct Sangha business in harmony, their growth can be expected, not
  their decline.
\end{english}

[3] Yāvakīvañca, bhikkhave, bhikkhū꜔꜒ apaññattaṃ na paññāpessa꜔꜒nti, paññattaṃ na
samucchi꜔꜒ndissa꜔꜒nti, yathā꜔꜒paññattesu sikkhā꜔꜒padesu samādāya vattissa꜔꜒nti;
vuddhiyeva, bhikkhave, bhikkhū꜔꜒naṃ pāṭikaṅkhā꜔꜒, no parihā꜔꜒ni.

\begin{english}
  “As long as the monks neither decree what has been undecreed nor repeal what
  has been decreed, but practice undertaking the training rules as they have
  been decreed, their growth can be expected, not their decline.
\end{english}

[4] Yāvakīvañca, bhikkhave, bhikkhū꜔꜒ ye te bhikkhū꜔꜒ the꜔꜒rā rattaññū cirapabbajitā
sa꜔꜒ṅghapitaro sa꜔꜒ṅghapariṇāyakā te sakkarissa꜔꜒nti garuṃ karissa꜔꜒nti mānessa꜔꜒nti
pūjessa꜔꜒nti, tesa꜔꜒ñca so꜔꜒tabbaṃ maññissa꜔꜒nti; vuddhiyeva, bhikkhave, bhikkhū꜔꜒naṃ
pāṭikaṅkhā꜔꜒, no parihā꜔꜒ni.

\begin{english}
  “As long as the monks honor, respect, venerate, and do homage to the elder
  monks — those with seniority who have long been ordained, the fathers of the
  Sangha, leaders of the Sangha — regarding them as worth listening to, their
  growth can be expected, not their decline.
\end{english}

[5] Yāvakīvañca, bhikkhave, bhikkhū꜔꜒ uppannāya taṇhā꜔꜒ya ponobhavikāya na vasa꜔꜒ṃ
gacchissa꜔꜒nti; vuddhiyeva, bhikkhave, bhikkhū꜔꜒naṃ pāṭikaṅkhā꜔꜒, no parihā꜔꜒ni.

\begin{english}
  “As long as the monks do not submit to the power of any arisen craving that
  leads to further becoming, their growth can be expected, not their decline.
\end{english}

[6] Yāvakīvañca, bhikkhave, bhikkhū꜔꜒ āraññakesu se꜔꜒nāsanesu sā꜔꜒pekkhā꜔꜒ bhavissa꜔꜒nti;
vuddhiyeva, bhikkhave, bhikkhū꜔꜒naṃ pāṭikaṅkhā꜔꜒, no parihā꜔꜒ni.

\begin{english}
  “As long as the monks see their own benefit in wilderness dwellings, their
  growth can be expected, not their decline.
\end{english}

[7] Yāvakīvañca, bhikkhave, bhikkhū꜔꜒ paccattaññeva satiṃ upaṭṭhā꜔꜒pessa꜔꜒nti: Kinti
anāgatā ca pesalā sabrahmacārī āgacche꜔꜒yyuṃ, āgatā ca pesalā sabrahmacārī phā꜔꜒su꜔꜒ṃ
vihareyyun'ti; vuddhiyeva, bhikkhave, bhikkhū꜔꜒naṃ pāṭikaṅkhā꜔꜒, no parihā꜔꜒ni.

\begin{english}
  “As long as the monks each keep firmly in mind: `If there are any well-behaved
  fellow followers of the chaste life who have yet to come, may they come; and
  may the well-behaved fellow-followers of the chaste life who have come live in
  comfort,' their growth can be expected, not their decline.
\end{english}

Yāvakīvañca, bhikkhave, ime satta aparihā꜔꜒niyā dhammā bhikkhū꜔꜒su ṭhassa꜔꜒nti, imesu
ca sattasu aparihā꜔꜒niyesu dhammesu bhikkhū꜔꜒ sa꜔꜒ndississa꜔꜒nti; vuddhiyeva, bhikkhave,
bhikkhū꜔꜒naṃ pāṭikaṅkhā꜔꜒, no parihā꜔꜒nī'ti. Idam-avoca Bhagavā. Attamanā te bhikkhū꜔꜒
Bhagavato bhāsitaṃ abhinandun'ti.

\begin{english}
  “As long as the monks remain steadfast in these seven conditions, and as long
  as these seven conditions endure among the monks, the monks' growth can be
  expected, not their decline.” That is what the Blessed One said. Gratified,
  the monks delighted in the Blessed One's words.
\end{english}

