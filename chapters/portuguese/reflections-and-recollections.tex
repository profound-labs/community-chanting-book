\chapter*[Partilha de Bençãos]{Reflexões sobre a Partilha de Bençãos}

\delegateSetUseNext

\begin{leader}
  [Ha꜓nda mayaṁ uddissanādhiṭṭhāna-gāthā꜓yo b꜕haṇāmase]
\end{leader}

\firstline{Iminā puññakammena upajjhāyā guṇuttarā}

[Iminā puñña꜕kammena] u꜕pajjhāyā gu꜕ṇutta꜕rā\\
Ācariyūpa꜕kārā ca꜕ mātāpitā ca꜕ ñāta꜕kā\\
Suriyo candimā rājā gu꜕ṇavantā na꜕rāpi꜕ ca꜕\\
Brahma-mārā ca꜕ indā ca꜕ loka꜕pālā ca꜕ deva꜕tā\\
Yamo mittā ma꜕nussā ca majjhattā veri꜕kāpi꜕ ca꜕\\
Sa꜕bbe sattā sukhī hontu puññāni pa꜕ka꜕tāni꜕ me\\
Sukhañca tividhaṁ dentu꜕ khippaṁ pāpetha꜕ voma꜕taṁ\\
Iminā puññakammena iminā uddi꜕ssena꜕ ca꜕\\
Khipp'āhaṁ su꜕la꜕bhe ceva taṇhūpādāna꜕-cheda꜕naṁ\\
Ye santāne hīnā dhammā yāva꜕ nibbāna꜕to ma꜕maṁ\\
Nassantu sabba꜕dā yeva yattha꜕ jāto bha꜕ve bha꜕ve\\
Ujucittaṁ sa꜕ti꜕paññā sallekho vi꜕ri꜕yamhinā\\
Mārā labhantu nokāsaṁ kātuñca vi꜕ri꜕yes꜕u me\\
Buddhādhipa꜕va꜕ro nātho dhammo nātho va꜕rutta꜕mo\\
Nātho pacceka꜕buddho ca꜕ saṅgho nāthotta꜕ro ma꜕maṁ\\
Tesottamānubhāvena mārokāsaṁ la꜕bhantu꜕ mā

\chapter[Partilha de Bençãos]{Reflexões sobre a Partilha de Bençãos}

\enlargethispage{2\baselineskip}

\begin{leader}
  [Cantemos agora as Reflexões sobre a Partilha de Bençãos.]
\end{leader}

\firstline{Através do bem que resulta da minha prática}

Através do bem que resulta da minha prática,\\
Que os meus mestres e guias espirituais de grande virtude,\\
A minha mãe, o meu pai e os meus familiares,\\
O Sol e a Lua, e todos os líderes virtuosos do mundo,\\
Que os Deuses mais elevados e as forças do mal,\\
Seres celestiais, espíritos guardiões da Terra e o Senhor da Morte,\\
Aqueles que são amigáveis, indiferentes ou hostis,\\
Que todos os seres recebam as bênçãos da minha vida.\\
Que brevemente cheguem à Tripla Bênção, e superem a morte.

Através do bem que resulta da minha prática,\\
E através desta partilha,\\
Que todos os desejos e apegos rapidamente cessem,\\
Assim como os estados prejudiciais da mente.

Até realizar o Nibbana,\\
Em qualquer tipo de nascimento, que eu tenha uma mente justa,\\
Com consciência e sabedoria, austeridade e vigor.\\
Que as forças ilusórias não controlem,\\
nem enfraqueçam a minha decisão.

O Buddha é o meu excelente refúgio,\\
Insuperável é a proteção do Dhamma,\\
O Buddha solitário é o meu Nobre exemplo,\\
O Saṅgha é o meu maior suporte.

Que através desta supremacia\\
Desapareçam a escuridão e a ilusão.

\chapter*[Metta Sutta]{Metta Sutta}

\delegateSetUseNext

\firstline{Karaṇīyam-attha-kusalena}

\begin{leader}
  [Ha꜓nda mayaṁ metta-sutta-gāthā꜓yo bha꜕ṇāmase]
\end{leader}

[Karaṇīyam-attha-kusalena]\\
Yan-taṁ santaṁ padaṁ abhisamecca\\
Sakko ujū ca suhujū ca\\
Suvaco c'assa mudu anatimānī

Santussako ca subharo ca\\
Appakicco ca sallahuka-vutti\\
Sant'indriyo ca nipako ca\\
Appagabbho kulesu ananugiddho

Na ca khuddaṁ samācare kiñci\\
Yena viññū pare upavadeyyuṁ\\
Sukhino vā khemino hontu\\
Sabbe sattā bhavantu sukhit'attā

Ye keci pāṇa-bhūt'atthi\\
Tasā vā thāvarā vā anavasesā\\
Dīghā vā ye mahantā vā\\
Majjhimā rassakā aṇuka-thūlā

Diṭṭhā vā ye ca adiṭṭhā\\
Ye ca dūre vasanti avidūre\\
Bhūtā vā sambhavesī vā\\
Sabbe sattā bhavantu sukhit'attā

\chapter[Metta Sutta]{Metta Sutta}

\firstline{Eis o que se deve fazer}

\begin{leader}
  [Cantemos agora as palavras do Buddha\\ sobre o Amor e a Compaixão.]
\end{leader}

Eis o que se deve fazer\\
Para cultivar a bondade\\
E seguir a via da paz:\\
Ser capaz e ser honesto,\\
Franco e gentil no falar.\\
Humilde e não arrogante,\\
Contente, facilmente satisfeito,\\
Aliviado de deveres e frugal no seu caminho.

Pacífico e sereno, sábio e inteligente,\\
Sem orgulho, sem exigência por natureza.\\
Que ele nada faça\\
Que os sábios possam vir a reprovar.\\
Desejando: na alegria e na segurança,\\
Que todos os seres sejam felizes.\\
Quaisquer que sejam os seres vivos,\\
Fracos, fortes, sem excepção\\
Dos maiores aos mais pequenos,\\
Visíveis ou invisíveis,\\
Estejam perto ou estejam longe,\\
Nascidos ou por nascer ---\\
Que todos os seres sejam felizes.

\clearpage

Na paro paraṁ nikubbetha\\
Nātimaññetha katthaci naṁ kiñci\\
Byārosanā paṭighasaññā\\
Nāññam-aññassa dukkham-iccheyya

Mātā yathā niyaṁ puttaṁ\\
Āyusā eka-puttam-anurakkhe\\
Evam'pi sabba-bhūtesu\\
Mānasam-bhāvaye aparimāṇaṁ

Mettañca sabba-lokasmiṁ\\
Mānasam-bhāvaye aparimāṇaṁ\\
Uddhaṁ adho ca tiriyañca\\
Asambādhaṁ averaṁ asapattaṁ

Tiṭṭhañ-caraṁ nisinno vā\\
Sayāno vā yāvat'assa vigata-middho\\
Etaṁ satiṁ adhiṭṭheyya\\
Brahmam-etaṁ vihāraṁ idham-āhu

Diṭṭhiñca anupagamma\\
Sīlavā dassanena sampanno\\
Kāmesu vineyya gedhaṁ\\
Na hi jātu gabbha-seyyaṁ punaretī'ti

\clearpage

Que ninguém engane ninguém,\\
Ou despreze alguém em que estado for.\\
Que ninguém por raiva ou má-fé,\\
Deseje mal a alguém.

Assim como uma Mãe protege o filho,\\
Com sua vida, seu único filho,\\
Também de coração infinito,\\
Se deveria estimar todo o ser vivo;\\
Irradiando ternura por todo o mundo:\\
Acima ao mais alto céu,\\
E abaixo às profundezas;\\
Irradiante e sem limites,\\
Livre de ódio e má-fé.\\
Seja parado ou a andar,\\
Sentado ou deitado,\\
Livre de torpor,\\
Esta é uma lembrança a manter.

Diz-se esta ser a sublime permanência.\\
O puro de coração, com clareza de visão,\\
Ao não insistir em ideias fixas,\\
Liberto dos desejos dos sentidos,\\
Não voltará a nascer neste mundo.

\chapter[Onze Benefícios]{Onze Benefícios da Prática de Metta}

\emph{Mettānisaṁsa Sutta, AN 11.15}

% https://suttacentral.net/an11.15/pli/ms

\begin{leader}
  [Handa mayaṁ mettānisaṁsa-suttaṁ bhaṇāmase]
\end{leader}

Mettāya, bhikkhave, cetovimuttiyā āsevitāya bhāvitāya bahulīkatāya yānīkatāya vatthukatāya anuṭṭhitāya paricitāya susamāraddhāya ekādasānisaṁsā pāṭikaṅkhā. Katame ekādasa?

\begin{english}
  Monges, há onze benefícios que podem ser esperados como resultado da libertação do coração pela prática da bondade amorosa, conhecendo-a, cultivando-a, desenvolvendo-a, tendo-a como guia, estimando-a, seguindo-a, consolidando-a e implementando-a.\\
  Quais são os onze?
\end{english}

Sukhaṁ supati.\\
Sukhaṁ paṭibujjhati.\\
Na pāpakaṁ supinaṁ passati.\\
Manussānaṁ piyo hoti.\\
Amanussānaṁ piyo hoti.\\
Devatā rakkhanti.\\
Nāssa aggi vā visaṁ vā satthaṁ vā kamati.\\
Tuvaṭaṁ cittaṁ samādhiyati.\\
Mukhavaṇṇo vippasīdati.\\
Asammūḷho kālaṁ karoti.\\
Uttari appaṭivijjhanto brahmalokūpago hoti.

\clearpage

\begin{english}
  Dormir bem.\\
  Acordar satisfeito.\\
  Não ter sonhos pertubadores.\\
  Ser-se querido pelos seres humanos.\\
  Ser-se querido pelos seres não humanos.\\
  Ser-se protegido pelos seres celestiais.\\
  Não ser afectado nem pelo fogo, nem por veneno, nem por armas.\\
  Poder concentrar a mente rapidamente.\\
  Ter um rosto sereno.\\
  Morrer sem qualquer confusão.\\
  Caso não esteja num estágio superior de iluminação, renascer no mundo de brahma.
\end{english}

Mettāya, bhikkhave, cetovimuttiyā āsevitāya bhāvitāya bahulīkatāya yānīkatāya vatthukatāya anuṭṭhitāya paricitāya susamāraddhāya ime ekādasānisaṁsā pāṭikaṅkhā'ti.

\begin{english}
  Monges, estes onze benefícios podem ser esperados como resultado da libertação do coração pela prática da bondade amorosa, conhecendo-a, cultivando-a, desenvolvendo-a, tendo-a como meio de expressão, estimando-a, seguindo-a, empreendendo-a e estabelecendo-a.
\end{english}

\chapter*[Radiância das Estâncias Divinas]{Radiância das Estâncias Divinas}

\delegateSetUseNext

\firstline{Mettā-sahagatena}

\begin{leader}
  [Ha꜓nda mayaṁ caturappamaññā-obhāsanaṁ karomase]
\end{leader}

[Mettā-sa꜕ha꜕ga꜕tena] cetasā ekaṁ disaṁ pha꜕ri꜕tv꜕ā viha꜕ra꜕ti\\
Ta꜕thā dutiyaṁ ta꜕thā tatiyaṁ ta꜕thā ca꜕tutthaṁ\\
Iti uddhamadho tiriyaṁ sabba꜕dhi꜕ sabbatta꜕tāya\\
Sabbāvantaṁ lokaṁ mettā-sa꜕ha꜕ga꜕tena cetasā\\
Vipulena mahagga꜕tena appa꜕māṇena a꜕verena a꜕byāpajjhena\\
\vin pha꜕ri꜕tv꜕ā viha꜕ra꜕ti

Karuṇā-sa꜕ha꜕ga꜕tena cetasā ekaṁ disaṁ pha꜕ri꜕tv꜕ā viha꜕ra꜕ti\\
Ta꜕thā dutiyaṁ ta꜕thā tatiyaṁ ta꜕thā ca꜕tutthaṁ\\
Iti uddhamadho tiriyaṁ sabba꜕dhi꜕ sabbatta꜕tāya\\
Sabbāvantaṁ lokaṁ ka꜕ru꜕ṇā-sa꜕ha꜕ga꜕tena cetasā\\
Vipulena mahagga꜕tena appa꜕māṇena a꜕verena a꜕byāpajjhena\\
\vin pha꜕ri꜕tv꜕ā viha꜕ra꜕ti

Muditā-sa꜕ha꜕ga꜕tena cetasā ekaṁ disaṁ pha꜕ri꜕tv꜕ā viha꜕ra꜕ti\\
Ta꜕thā dutiyaṁ ta꜕thā tatiyaṁ ta꜕thā ca꜕tutthaṁ\\
Iti uddhamadho tiriyaṁ sabba꜕dhi꜕ sabbatta꜕tāya\\
Sabbāvantaṁ lokaṁ mu꜕di꜕tā-sa꜕ha꜕ga꜕tena cetasā\\
Vipulena mahagga꜕tena appa꜕māṇena a꜕verena a꜕byāpajjhena\\
\vin pha꜕ri꜕tv꜕ā viha꜕ra꜕ti

\chapter[Radiância das Estâncias Divinas]{Radiância das Estâncias Divinas}

\enlargethispage{\baselineskip}

\firstline{Eu permanecerei}

\begin{leader}
  \vspace*{-\baselineskip}
  \mbox{[Deixemos agora irradiar as Quatro Qualidades Incomensuráveis.]}
\end{leader}

[Eu permanecerei] permeando um quarto do mundo\\
\vin com um coração imbuído de amável bondade; da mesma forma\\
\vin o segundo, da mesma forma o terceiro, da mesma forma o quarto;\\
Como acima, assim em baixo, à volta e em todo o lado;\\
\vin e para todos como para mim.\\
Eu permanecerei permeando o mundo todo-abrangente\\
\vin com um coração imbuído de amável bondade; abundante, exaltado,\\
\vin imensurável, sem hostilidade e sem má-fé.

[Eu permanecerei] permeando um quarto do mundo\\
\vin com um coração imbuído de compaixão; da mesma forma\\
\vin o segundo, da mesma forma o terceiro, da mesma forma o quarto;\\
Como acima, assim em baixo, à volta e em todo o lado;\\
\vin e para todos como para mim.\\
Eu permanecerei permeando o mundo todo-abrangente\\
\vin com um coração imbuído de compaixão; abundante, exaltado,\\
\vin imensurável, sem hostilidade e sem má-fé.

[Eu permanecerei] permeando um quarto do mundo\\
\vin com um coração imbuído de alegria empática; da mesma forma\\
\vin o segundo, da mesma forma o terceiro, da mesma forma o quarto;\\
Como acima, assim em baixo, à volta e em todo o lado;\\
\vin e para todos como para mim.\\
Eu permanecerei permeando o mundo todo-abrangente\\
\vin com um coração imbuído de alegria empática; abundante, exaltado,\\
\vin imensurável, sem hostilidade e sem má-fé.

\clearpage

Upekkhā-saha꜕ga꜕te꜕na cetasā ekaṁ disaṁ pha꜕ri꜕tv꜕ā viha꜕ra꜕ti\\
Ta꜕thā dutiyaṁ ta꜕thā tatiyaṁ ta꜕thā ca꜕tutthaṁ\\
Iti uddhamadho tiriyaṁ sabba꜕dhi꜕ sabbatta꜕tāya\\
Sabbāvantaṁ lokaṁ u꜕pe꜕kkhā-sa꜕ha꜕ga꜕tena cetasā\\
Vipulena mahagga꜕tena appa꜕māṇena a꜕verena a꜕byāpajjhena\\
\vin pha꜕ri꜕tv꜕ā viha꜕ra꜕tī'ti

\clearpage

[Eu permanecerei] permeando um quarto do mundo\\
\vin com um coração imbuído de equanimidade; da mesma forma\\
\vin o segundo, da mesma forma o terceiro, da mesma forma o quarto;\\
Como acima, assim em baixo, à volta e em todo o lado;\\
\vin e para todos como para mim.\\
Eu permanecerei permeando o mundo todo-abrangente\\
\vin com um coração imbuído de equanimidade; abundante, exaltado,\\
\vin imensurável, sem hostilidade e sem má-fé.

\chapter{As Bênçãos Maiores}

\firstline{Assim eu ouvi que o Excelso}

\begin{leader}
  [Cantemos agora os versos sobre as Bênção Maiores.]
\end{leader}

[Assim eu ouvi que o Excelso]\\
Se encontrava em Savatthi,\\
A residir no Bosque de Jeta\\
No Parque de Anāthapiṇḍika.

Então no escuro da noite, uma deva radiante\\
Iluminou todo o Parque de Jeta.\\
Inclinou-se prestando reverência ao Excelso\\
E depois colocando-se de pé, disse:

`Os Devas preocupam-se com a felicidade\\
E buscam Paz continuamente.\\
O mesmo se pode dizer da humanidade.\\
Assim, quais são as Bênçãos mais elevadas?'

`Evitar os tolos,\\
Associar-se aos Sábios,\\
E honrar quem é digno de honra.\\
Estas são as maiores bênçãos.

`Viver em locais adequados,\\
Com os frutos das boas acções passadas,\\
Guiado pelo caminho correcto.\\
Estas são as maiores bênçãos.

\clearpage

`Proficiente em estudos e ofícios,\\
Com disciplina sublimemente treinada,\\
E discurso verdadeiro, agradável ao ouvido.\\
Estas são as maiores bênçãos.

`Apoiar os pais,\\
Zelar pela família,\\
E ter uma vida inofensiva para os outros.\\
Estas são as maiores bênçãos.

`Generosidade e uma vida honesta,\\
Oferecer ajuda a familiares e amigos,\\
Agir de forma a não causar remorsos.\\
Estas são as maiores bênçãos.

`Ser resoluto a dominar-se, a abandonar os caminhos do mal,\\
Evitar intoxicantes que entorpeçam a mente,\\
E ser diligente em todas as ocasiões.\\
Estas são as maiores bênçãos.

`Respeito e humildade,\\
Contentamento e gratidão,\\
Ouvir o Dhamma devidamente ensinado.\\
Estas são as maiores bênçãos.

`Paciência e vontade para aceitar as próprias falhas,\\
Visitar respeitáveis buscadores da verdade,\\
e partilhar o Dhamma adequadamente.\\
Estas são as maiores bênçãos.

\clearpage

`Dedicar-se ardentemente à Vida Santa,\\
Ver as Nobres Verdades directamente por si\\
E realizar o Nibbana.\\
Estas são as maiores bênçãos.

`Ainda que em contacto com o mundo,\\
A mente mantém-se inabalável,\\
Perfeitamente segura, além de toda a aflição.\\
Estas são as maiores bênçãos.

`Aqueles que seguem este caminho,\\
Conhecem a Victória onde quer que vão,\\
E qualquer lugar para eles é seguro.\\
Estas são as maiores bênçãos.'

\chapter*[Bem-Estar Universal]{Reflexão sobre o Bem-Estar Universal}

\delegateSetUseNext

\firstline{Ahaṁ sukhito homi}

\begin{leader}
  [Ha꜓nda mayam mettāpharaṇaṁ ka꜕romase]
\end{leader}

[Aha꜓ṁ sukhito ho꜓mi]\\
Niddukkho ho꜓mi\\
A꜕vero ho꜓mi\\
A꜕byāpajjho ho꜓mi\\
A꜕nīgho ho꜓mi\\
Sukhī꜓ attānaṁ pa꜕riha꜓rāmi

Sa꜕bbe sa꜕ttā sukhitā ho꜓ntu\\
Sa꜕bbe sa꜕ttā averā ho꜓ntu\\
Sa꜕bbe sa꜕ttā abyāpajjhā ho꜓ntu\\
Sa꜕bbe sa꜕ttā anīghā ho꜓ntu\\
Sa꜕bbe sa꜕ttā sukhī꜓ a꜕ttānaṁ pa꜕riha꜓rantu

Sa꜕bbe sa꜕ttā sabbadukkhā pamucca꜓ntu

Sa꜕bbe sa꜕ttā laddha-sa꜓mpa꜕tti꜓to mā vigaccha꜓ntu

Sa꜕bbe sa꜕ttā kammassa꜕kā kamma꜓dāyādā kamma꜓yonī\\
\vin kamma꜓bandhū kammapa꜕ṭisa꜓ra꜕ṇā\\
Yaṁ kammaṁ ka꜕rissa꜓nti\\
Kalyāṇaṁ vā pāpa꜕kaṁ vā\\
Tassa꜕ dāyādā bha꜕vissa꜓nti

\chapter[Bem-Estar Universal]{Reflexão sobre o Bem-Estar Universal}

\firstline{Que eu mantenha bem-estar}

\begin{leader}
  [Cantemos agora as Reflexões sobre o Bem-estar Universal.]
\end{leader}

[Que eu possa viver em bem-estar,]\\
Livre de aflição,\\
Livre de hostilidade,\\
Livre de má-fé,\\
Livre de ansiedade,\\
E possa eu manter este bem-estar.

Que todos possam viver em bem-estar,\\
Livres de hostilidade,\\
Livres de má-fé,\\
Livres de ansiedade, e possam eles\\
Manter esse bem-estar.

Possam todos os seres libertarem-se de todo o sofrimento.

E que todos não se separarem da boa fortuna que alcançaram.

Quando agem com intenção,\\
Todos os seres são os donos da sua acção e herdam os seus resultados.\\
O seu futuro nasce de tal acção, companheiro de tal acção,\\
E os seus resultados serão o seu lar.

Todas as acções com intenção,\\
Sejam elas boas ou más ---\\
De tais actos eles serão os herdeiros.

\chapter[Quatro Requisitos]{Reflexão sobre os Quatro Requisitos}

\firstline{Paṭisaṅkhā yoniso}

\begin{leader}
  [Ha꜓nda mayaṁ taṅkhaṇika-paccave꜕kkhaṇa-pāṭhaṁ bhaṇāmase]
\end{leader}

[Paṭisaṅkhā] yoniso cīva꜕raṁ pa꜕ṭise꜓vāmi, \pause\\
yāvadeva sī꜓tassa꜕ pa꜕ṭighātāya, \pause\ uṇhassa pa꜕ṭighātāya, \pause\\
ḍaṁsa-maka꜕sa꜕-vātāta꜕pa꜕-siriṁsapa-samphassānaṁ pa꜕ṭighātāya, \pause\\
yāvadeva hiri꜓kopina-pa꜕ṭicchāda꜕natthaṁ

\begin{english}
  Reflectindo sabiamente \pause\ eu uso o manto: \pause\ somente por modéstia, \pause\
  para evitar o calor, \pause\ o frio, \pause\ as moscas, \pause\ mosquitos,
  \pause\ bichos rastejantes, \pause\\ o vento e as coisas que queimam.
\end{english}

[Paṭisaṅkhā] yoniso piṇḍa꜕pātaṁ pa꜕ṭise꜓vāmi, \pause\\
neva da꜕vāya, na ma꜕dāya, na maṇḍa꜕nāya, na꜕ vi꜓bhūsa꜕nāya, \pause\\
yāvadeva i꜓massa꜕ kāyassa꜕ ṭhi꜕tiyā, \pause\ yāpa꜕nāya, vihiṁsū꜕para꜓ti꜕yā, \pause\\
brahmaca꜕ri꜓yānugga꜕hāya, \pause\ iti purāṇañca꜕ veda꜓naṁ pa꜕ṭiha꜓ṅkhāmi,
navañca꜕ veda꜓naṁ na uppādessāmi, \pause\ yātrā ca꜕ me bhavissati a꜕navajjatā
ca꜕ phāsuvihāro cā'ti

\begin{english}
  Reflectindo sabiamente \pause\ eu uso a comida da mendicância: \pause\ não por
  diversão, \pause\ não por prazer, \pause\ não para engordar, \pause\ não para
  me embelezar, \pause\ mas somente para suster e nutrir este corpo, \pause\
  para o manter saudável, \pause\ para ajudar à Vida Santa. \pause\ Pensando
  desta forma: \pause\ `Saciarei a fome sem comer demasiado, \pause\ de
  forma a~continuar a viver sereno e sem remorsos.'
\end{english}

\clearpage

[Paṭisaṅkhā] yoniso senāsa꜕naṁ pa꜕ṭise꜓vāmi, \pause\\
yāvadeva sī꜓tassa꜕ pa꜕ṭighātāya, \pause\ uṇhassa pa꜕ṭighātāya, \pause\\
ḍaṁsa-maka꜕sa꜕-vātāta꜕pa꜕-siriṁsapa-samphassānaṁ pa꜕ṭighātāya, \pause\\
yāvadeva utupa꜕rissaya vi꜕nodanaṁ \pause\ pa꜕ṭisa꜓llānārāmatthaṁ

\begin{english}
  Reflectindo sabiamente \pause\ eu uso o alojamento: \pause\ somente para evitar o
  frio, \pause\ o calor, \pause\ as moscas, \pause\ mosquitos, \pause\ bichos
  rastejantes, \pause\ o vento e as coisas que queimam. \pause\ Somente para me
  abrigar dos perigos da natureza \pause\ e viver em recolhimento.
\end{english}

[Paṭisaṅkhā] yoniso gi꜕lāna-pacca꜕ya꜕-bhesajja-pa꜕rikkhāraṁ\\
pa꜕ṭise꜓vāmi, \pause\ yāvadeva uppa꜓nnānaṁ veyyābādhi꜕kānaṁ veda꜕nānaṁ
pa꜕ṭighātāya, \pause\ a꜕byāpajjha-pa꜕ramatāyā'ti

\begin{english}
  Reflectindo sabiamente \pause\ eu uso o apoio necessário para medicamentos e
  enfermidades: \pause\ somente para aliviar as dores que tenham surgido,
  \pause\ de forma a ficar o mais possível livre de doenças.
\end{english}

\chapter[Trinta-e-duas-Partes]{Reflexão sobre as Trinta-e-duas-Partes}

\firstline{Ayaṁ kho me kāyo}

\begin{leader}
  [Ha꜓nda mayaṁ dvattiṁsākāra-pāṭhaṁ bhaṇāmase]
\end{leader}

[Ayaṁ kho] me kāyo uddhaṁ pāda꜕ta꜕lā adho kesamatthakā\\
ta꜕ca꜕pa꜕ri꜕yanto pūro nānappa꜕kārassa꜕ a꜕su꜕ci꜕no

\begin{english}
  Isto, que é o meu corpo, das plantas dos pés para cima, e do topo da cabeça para baixo, é um saco de pele fechado, cheio de coisas repugnantes.
\end{english}

Atthi imasmi꜔꜒ṁ kāye

\begin{english}
  Neste corpo existem:
\end{english}

{\centering
\setArrayStrech{1}

\begin{tabular}{ r l }
kesā꜔꜒           & \tr{cabelo} \\
lomā            & \tr{pêlos} \\
nakhā꜔꜒          & \tr{unhas} \\
dantā           & \tr{dentes} \\
taco            & \tr{pele} \\
maṁsa꜔꜒ṁ        & \tr{carne} \\
nahā꜔꜒rū         & \tr{tendões} \\
aṭṭhī꜔꜒           & \tr{ossos} \\
aṭṭhimiñjaṁ     & \tr{medula óssea} \\
vakkaṁ          & \tr{rins} \\
hadayaṁ         & \tr{coração} \\
yakanaṁ         & \tr{fígado} \\
kilomakaṁ       & \tr{membranas} \\
pihakaṁ         & \tr{baço} \\
papphā꜔꜒sa꜔꜒ṁ    & \tr{pulmões} \\
\end{tabular}

\clearpage

\begin{tabular}{ r l }
antaṁ           & \tr{intestino grosso} \\
antaguṇaṁ       & \tr{intestino delgado} \\
udariyaṁ        & \tr{comida não digerida} \\
karīsa꜔꜒ṁ        & \tr{excremento} \\
pittaṁ          & \tr{bílis} \\
se꜔꜒mha꜔꜒ṁ       & \tr{muco} \\
pubbo           & \tr{pus} \\
lohitaṁ         & \tr{sangue} \\
se꜔꜒do           & \tr{suor} \\
medo            & \tr{gordura} \\
assu            & \tr{lágrimas} \\
vasā꜔꜒           & \tr{sebo} \\
khe꜔꜒ḷo          & \tr{saliva} \\
si꜔꜒ṅghāṇikā    & \tr{mucosidade} \\
lasikā          & \tr{lubrificante das articulações} \\
muttaṁ          & \tr{urina} \\
matthaluṅgan’ti & \tr{cérebro} \\
\end{tabular}

\restoreArrayStretch
}

Evam-ayaṁ me kāyo uddhaṁ pāda꜕ta꜕lā adho kesamatthakā\\
ta꜕ca꜕pa꜕ri꜕yanto pūro nānappa꜕kārassa꜕ a꜕su꜕ci꜕no

\begin{english}
  Assim, isto que é o meu corpo, das plantas dos pés para cima, e do topo da cabeça para baixo, é um saco de pele fechado, cheio de coisas repugnantes.
\end{english}

\clearpage

Se꜔꜒yyathā꜔꜒pi, bhikkhave, ubhatomukhā꜔꜒ putoḷi pūrā nānāvihitassa dhaññassa,
se꜔꜒yyathī꜔꜒daṁ, sā꜔꜒līnaṁ vīhī꜔꜒naṁ muggānaṁ māsā꜔꜒naṁ tilānaṁ taṇḍulānaṁ. Tamenaṁ
cakkhumā puriso꜔꜒ muñcitvā paccavekkhe꜔꜒yya:

\begin{english}
  Bhikkhus, tal como se houvesse um saco, com uma abertura em ambas as
  extremidades, cheio de vários tipos de grãos (tais como arroz da montanha,
  arroz vermelho, feijões, ervilhas, milho painço e arroz branco) e um homem com
  bons olhos o abrisse e o descrevesse da seguinte forma:
\end{english}

‘Ime sā꜔꜒lī, ime vīhī꜔꜒ ime muggā ime māsā꜔꜒ ime tilā ime taṇḍulā’ti. Evameva kho꜔꜒,
bhikkhave, bhikkhu imameva kāyaṁ uddhaṁ pādatalā adho kesamatthakā
tacapariyantaṁ pūraṁ nānappakārassa asucino paccavekkhati:

\begin{english}
  ‘Isto é arroz da montanha, isto é arroz vermelho, isto são feijões, isto são
  ervilhas, isto é milho painço, isto é arroz branco’;

  assim também, bhikkhus, um bhikkhu descreve:

  Isto, que é o meu corpo, das plantas dos pés para cima, e do topo da cabeça
  para baixo, é um saco de pele fechado cheio de coisas repugnantes.
\end{english}

‘Atthi imasmi꜔꜒ṁ kāye kesā꜔꜒ lomā nakhā꜔꜒ dantā taco, maṁsa꜔꜒ṁ nahā꜔꜒rū aṭṭhī꜔꜒ aṭṭhimiñjaṁ
vakkaṁ, hadayaṁ yakanaṁ kilomakaṁ pihakaṁ papphā꜔꜒sa꜔꜒ṁ, antaṁ antaguṇaṁ udariyaṁ
karīsa꜔꜒ṁ, pittaṁ se꜔꜒mha꜔꜒ṁ pubbo lohitaṁ se꜔꜒do medo, assu vasā꜔꜒ khe꜔꜒ḷo si꜔꜒ṅghāṇikā
lasikā muttaṁ matthaluṅgan’ti.

\begin{english}
  Neste corpo existem:
  cabelo, pêlos, unhas, dentes, pele,
  carne, tendões, ossos, medula óssea, rins,
  coração, fígado, membranas, baço, pulmões,
  intestino grosso, intestino delgado, comida não digerida, excremento,
  bílis, muco, pus, sangue, suor, gordura,
  lágrimas, sebo, saliva, mucosidade, lubrificante das articulações, urina, cérebro.
\end{english}

Iti ajjhattaṁ vā kāye kāyānupassī꜔꜒ viharati, bahiddhā vā kāye kāyānupassī꜔꜒
viharati, ajjhatta-bahiddhā vā kāye kāyānupassī꜔꜒ viharati. Samudaya-dhammānupassī꜔꜒
vā kāyasmi꜔꜒ṁ viharati, vaya-dhammā-\\
nupassī꜔꜒ vā kāyasmi꜔꜒ṁ viharati, samudaya-vaya-dhammānupassī꜔꜒ vā kāyasmi꜔꜒ṁ viharati.
‘Atthi kāyo’ti vā panassa sati paccupaṭṭhitā ho꜔꜒ti yāvadeva ñāṇamattāya
paṭissatimattāya.

Anissito ca viharati, na ca kiñci loke upādiyati. Evampi kho꜔꜒,
bhikkhave, bhikkhu kāye kāyānupassī꜔꜒ viharati.

\begin{english}
  Desta forma, considerando o corpo, ele mantém-se a contemplar o corpo internamente; 
  considerando o corpo, ele mantém-se a~contemplar o corpo externamente;
  considerando o corpo, ele mantém-se a~contemplar o corpo tanto internamente quanto 
  externamente. Ou então, ele mantém-se a~contemplar no corpo a~sua natureza de surgir, 
  ou mantém-se a~contemplar no corpo a~sua natureza de cessar, ou mantém-se a~contemplar 
  no corpo a~sua natureza de surgir e cessar. Assim, a~consciencialização de que ‘existe
  um corpo’ é nele estabelecida ao ponto de haver compreensão directa e consciência inabalável.

  \bigskip

  E ele vive de forma independente, sem se apegar a nada no mundo. Bhikkhus, 
  é desta forma que um bhikkhu, considerando o corpo, mantém-se a contemplar o corpo.
\end{english}

\chapter[Cinco Temas]{Cinco Temas para Recordar Frequentemente}

\firstline{Jarā-dhammomhi jaraṁ anatīto}

\begin{leader}
  [Ha꜓nda mayaṁ abhiṇha-paccave꜕kkhaṇa-pāṭhaṁ bhaṇāmase]
\end{leader}

\sidepar{Homens}%
[Jarā-dhammomhi꜕] jaraṁ a꜕na꜕tīto

\sidepar{Mulheres}%
[Jarā-dhammāmhi꜕] jaraṁ a꜕na꜕tītā

\begin{english}
  A minha natureza é envelhecer, ainda não estou para além do envelhecimento.
\end{english}

\sidepar{h.}%
Byādhi꜓-dhammomhi꜕ byādhiṁ a꜕na꜕tīto

\sidepar{m.}%
Byādhi꜓-dhammāmhi꜕ byādhiṁ a꜕na꜕tītā

\begin{english}
  A minha natureza é adoecer, ainda não estou para além da doença.
\end{english}

\sidepar{h.}%
Ma꜕raṇa-dhammomhi꜕ ma꜕raṇaṁ a꜕na꜕tīto

\sidepar{m.}%
Ma꜕raṇa-dhammāmhi꜕ ma꜕raṇaṁ a꜕na꜕tītā

\begin{english}
  A minha natureza é morrer, ainda não estou para além da morte.
\end{english}

Sa꜕bbehi me pi꜕yehi ma꜕nāpehi꜕ nānābhāvo vi꜕nābhāvo

\begin{english}
  Tudo o que é meu, amado e agradável,\\
  tornar-se-á diferente, separar-se-á de mim.
\end{english}

\sidepar{h.}%
Kammassa꜕komhi kamma꜓dāyādo kamma꜕yoni kamma꜕bandhu kammapa꜕ṭisa꜓ra꜕ṇo\\
Yaṁ kammaṁ ka꜕rissāmi, kalyāṇaṁ vā pāpa꜕kaṁ vā, tassa꜕ dāyādo bha꜕vissāmi

\clearpage

\sidepar{m.}%
Kammassa꜕kāmhi kamma꜓dāyādā kamma꜕yoni kamma꜕bandhu kammapa꜕ṭisa꜓ra꜕ṇā\\
Yaṁ kammaṁ ka꜕rissāmi, kalyāṇaṁ vā pāpa꜕kaṁ vā, tassa꜕ dāyādā bha꜕vissāmi

\begin{english}
  Sou o dono do meu Kamma, herdeiro do meu Kamma,\\
  nascido do meu Kamma, ligado ao meu Kamma,\\
  permaneço suportado pelo meu Kamma; seja qual Kamma eu criar,\\
  para o bem ou para o mal, disso serei o herdeiro.
\end{english}

Evaṁ amhehi꜕ a꜕bhiṇhaṁ pacca꜕vekkhi꜓tabbaṁ

\begin{english}
  Assim deveríamos frequentemente reflectir.
\end{english}

\chapter[Dez Temas]{Dez Temas para Recordar Frequentemente por Aqueles que Seguem o Caminho}

\firstline{Dasa ime bhikkhave}

\begin{leader}
  [Ha꜓nda mayaṁ pabbajita\hyp{}abhiṇha\hyp{}paccave꜕kkhaṇa\hyp{}pāṭhaṁ bhaṇāmase]
\end{leader}

[Dasa i꜕me bhikkhave] dhammā pabba꜕jitena a꜕bhiṇhaṁ pacca꜕vekkhi꜓tabbā, \pause\ ka꜕ta꜕me dasa

\begin{english}
  Monges, existem dez dhammas \pause\ sobre os quais se deve reflectir frequentemente. \pause\ Quais são estes dez dhammas?
\end{english}

Vevaṇṇi꜕yamhi ajjhūpa꜕ga꜕to'ti pabba꜕jitena a꜕bhiṇhaṁ pacca꜕vekkhi꜓tabbaṁ

\begin{english}
  `Já não vivo segundo os valores e objectivos do mundo.' \pause\\
  Quem perfaz o caminho \pause\ deve reflectir sobre isto frequentemente.
\end{english}

Parapaṭi꜕baddhā me jīvi꜓kā'ti pabba꜕jitena a꜕bhiṇhaṁ pacca꜕vekkhi꜓tabbaṁ

\begin{english}
  `A minha própria vida é sustentada \pause\ pela generosidade dos outros.' \pause\\
  Quem perfaz o caminho \pause\ deve reflectir sobre isto frequentemente.
\end{english}

Añño me ākappo ka꜕ra꜕ṇīyo'ti pabba꜕jitena a꜕bhiṇhaṁ pacca꜕vekkhi꜓tabbaṁ

\begin{english}
  `Devo esforçar-me por abandonar os meus hábitos antigos.' \pause\\
  Quem perfaz o caminho \pause\ deve reflectir sobre isto frequentemente.
\end{english}

\clearpage

Kacci nu꜕ kho me attā sīla꜕to na u꜕pavadatī'ti pabba꜕jitena a꜕bhiṇhaṁ pacca꜕vekkhi꜓tabbaṁ

\begin{english}
  `Surgem remorsos na minha mente \pause\ em relação à minha conduta?' \pause\\
  Quem perfaz o caminho \pause\ deve reflectir sobre isto frequentemente.
\end{english}

Kacci nu꜕ kho maṁ a꜕nuvicca viññū sabrahma꜓cārī sīla꜕to na u꜕pavadantī'ti pabba꜕jitena a꜕bhiṇhaṁ pacca꜕vekkhi꜓tabbaṁ

\begin{english}
  `Será que os meus companheiros espirituais \pause\\\
  acham falhas na minha conduta?' \pause\\
  Quem perfaz o caminho \pause\ deve reflectir sobre isto frequentemente.
\end{english}

Sa꜕bbehi me pi꜕yehi ma꜕nāpehi꜕ nānābhāvo vi꜕nābhāvo'ti pabba꜕jitena abhiṇhaṁ pacca꜕vekkhi꜓tabbaṁ

\begin{english}
  `Tudo aquilo que é meu, \pause\ que amo e prezo, \pause\ tornar-se-á diferente, \pause\ separar-se-á de mim.' \pause\\
  Quem perfaz o caminho \pause\ deve reflectir sobre isto frequentemente.
\end{english}

Kammassa꜕komhi kamma꜓dāyādo kamma꜕yoni kamma꜕bandhu kammapa꜕ṭisa꜓raṇo, yaṁ kammaṁ ka꜕rissāmi, kalyāṇaṁ vā pāpa꜕kaṁ vā, tassa꜕ dāyādo bha꜕vissāmī'ti pabba꜕jitena a꜕bhiṇhaṁ pacca꜕vekkhi꜓tabbaṁ

\begin{english}
  `Sou o dono do meu Kamma, \pause\ herdeiro do meu Kamma, \pause\\
  nascido do meu Kamma, \pause\ ligado ao meu Kamma, \pause\\
  permaneço suportado pelo meu Kamma; \pause\ seja qual Kamma eu criar, \pause\\
  Para o bem ou para o mal, \pause\ disso serei o herdeiro.' \pause\\
  Quem perfaz o caminho \pause\ deve reflectir sobre isto frequentemente.
\end{english}

\clearpage

`Kathambhūtassa꜕ me rattindi꜕vā vīti꜕pa꜓tantī'ti pabba꜕jitena a꜕bhiṇhaṁ pacca꜕vekkhi꜓tabbaṁ

\begin{english}
  `Os dias e as noites passam continuamente; \pause\\
  Como estou eu a usar o meu tempo?' \pause\\
  Quem perfaz o caminho \pause\ deve reflectir sobre isto frequentemente.
\end{english}

Kacci nu꜕ kho'haṁ suññā꜓gāre abhira꜕māmī'ti pabba꜕jitena a꜕bhiṇhaṁ pacca꜕vekkhi꜓tabbaṁ

\begin{english}
  `Aprecio a solidão ou não?' \pause\\
  Quem perfaz o caminho \pause\ deve reflectir sobre isto frequentemente.
\end{english}

Atthi nu꜕ kho me uttari-ma꜕nussa-dhammā alamariya꜕-ñāṇa-dassana-viseso adhiga꜕to, so'haṁ pacchi꜓me kāle sa꜕brahmacārīhi꜕ puṭṭho na maṅku bha꜕vissāmī'ti pabba꜕jitena a꜕bhiṇhaṁ pacca꜕vekkhi꜓tabbaṁ

\begin{english}
  `Deu a minha prática frutos de compreensão e liberdade, \pause\\ de forma a que
  no fim da minha vida \pause\ eu não me sinta envergonhado \pause\\
  quando questionado \pause\ pelos meus companheiros espirituais?' \pause\\
  Quem perfaz o caminho \pause\ deve reflectir sobre isto frequentemente.
\end{english}

Ime kho bhikkha꜓ve da꜕sa꜕ dhammā pabba꜕jitena a꜕bhiṇhaṁ pacca꜕vekkhitabbā'ti

\begin{english}
  Monges estes são dez Dhammas \pause\ sobre os quais se deve reflectir frequentemente.
\end{english}

\chapter{Verdadeiros e Falsos Refúgios}

\firstline{Bahuṁ ve saraṇaṁ yanti}

\begin{leader}
  [Ha꜓nda mayaṁ khemākhema-sa꜕raṇa-gamana-\\
  -pa꜕ridīpikā-gāthā꜓yo bha꜕ṇāmase]
\end{leader}

\begin{twochants}
  Bahuṁ ve sa꜕ra꜓ṇaṁ yanti꜕ & pa꜕bba꜕tāni va꜕nāni꜓ ca \\
  Ārāma-rukkha꜕-cetyāni & manussā꜓ bha꜕ya꜕-tajji꜕tā \\
\end{twochants}

\begin{english}
  Para muitos refúgios eles vão ---\\
  para as encostas das montanhas e para as clareiras das florestas,\\
  para parques naturais e sítios sagrados ---\\
  pessoas dominadas pelo medo.
\end{english}

\begin{twochants}
  N'etaṁ kho sa꜕ra꜓ṇaṁ khemaṁ & n'etaṁ sa꜕raṇam-u꜓tt꜕amaṁ \\
  N'etaṁ sa꜕raṇam-āgamma & sa꜕bba-dukkhā꜓ pa꜕mucca꜕ti \\
\end{twochants}

\begin{english}
  Tais refúgios não são seguros,\\
  tais refúgios não são supremos,\\
  tais refúgios não levam\\
  à completa libertação do sofrimento.
\end{english}

\begin{twochants}
  Yo ca꜕ Buddhañca꜕ Dhammañca꜕ & sa꜓ṅghañca꜕ sa꜓ra꜕ṇaṁ ga꜕to \\
  Ca꜕ttāri a꜕riya-saccāni & sa꜕mmappaññāya꜓ pa꜕ss꜕ati \\
\end{twochants}

\begin{english}
  Quem se refugia\\
  na Jóia Tripla\\
  vê claramente\\
  as Quatro Nobres Verdades:
\end{english}

\begin{twochants}
  Dukkhaṁ dukkha-sa꜕muppādaṁ & dukkhassa ca꜕ a꜕ti꜕kka꜕maṁ \\
  A꜕riyañ-c'a꜕ṭṭh'a꜓ṅgi꜕kaṁ maggaṁ & dukkhūpasa꜕ma꜕-gāmi꜓naṁ \\
\end{twochants}

\begin{english}
  O sofrimento, a sua origem \\
  e a libertação deste,\\
  o Nobre Óctuplo Caminho\\
  que leva ao fim do sofrimento.
\end{english}

\begin{twochants}
  Etaṁ kho sa꜕ra꜓ṇaṁ khemaṁ & etaṁ sa꜕raṇam-u꜓tta꜕maṁ \\
  Etaṁ sa꜕raṇam-āgamma & sa꜕bba-dukkhā꜓ pa꜕mucca꜕ti \\
\end{twochants}

\begin{english}
  Tal refúgio é seguro,\\
  tal refúgio é supremo,\\
  tal refúgio realmente leva\\
  à completa libertação do sofrimento.
\end{english}

\chapter{Versos sobre a Riqueza Daquele que é Nobre}

\firstline{Yassa saddhā tathāgate}

\begin{leader}
  [Ha꜓nda mayaṁ a꜕riya-dhana-gāthā꜓yo bha꜕ṇāmase]
\end{leader}

\begin{twochants}
  Yassa꜕ sa꜕ddhā tathā꜓ga꜕te & a꜕ca꜕lā su꜕pa꜕tiṭṭhi꜓tā \\
  Sī꜓lañca꜕ yassa꜕ kalyāṇaṁ & a꜕riya-kantaṁ pasa꜓ṁsi꜕taṁ \\
\end{twochants}

\begin{english}
  Aquele cuja confiança no Tathāgata\\
  é inabalável e bem estabelecida,\\
  cuja virtude é admirável,\\
  tem o regozijo e elogio dos Nobres.
\end{english}

\begin{twochants}
  Sa꜓ṅghe pa꜕sā꜕do yass'atthi & uju-bhūtañca da꜓ss꜕anaṁ \\
  A꜕daliddo't꜕i taṁ āhu꜕ & a꜕moghaṁ ta꜕ssa꜕ jīvi꜓taṁ \\
\end{twochants}

\begin{english}
  Daquele que tem confiança no Sangha,\\
  que vê directamente a verdadeira realidade,\\
  diz-se que não é pobre de espírito\\
  e que a sua vida não é em vão.
\end{english}

\begin{twochants}
  Tasmā sa꜕ddhañca꜕ sī꜓lañca꜕ & pasādaṁ dhamma-da꜓ssa꜕naṁ \\
  A꜕nuyuñjetha medhāvī & sa꜕raṁ buddhāna sā꜓sa꜕naṁ \\
\end{twochants}

\begin{english}
  À virtude e à fé,\\
  à confiança e ao saber da verdade ---\\
  a isto os sábios se devem devotar,\\
  mantendo os ensinamentos do Buddha em mente.
\end{english}

\chapter{Versos sobre as Três Características}

\firstline{Sabbe saṅkhārā aniccā'ti}

\begin{leader}
  [Ha꜓nda mayaṁ ti-lakkhaṇ'ādi-gāthā꜓yo bha꜕ṇāmase]
\end{leader}

\begin{twochants}
  Sa꜕bbe sa꜓ṅkhā꜓rā a꜕ni꜓ccā't꜕i & yadā paññāya꜓ pa꜕ssa꜕ti \\
  Atha nibbinda꜕ti dukkhe & esa꜕ maggo vi꜓su꜕ddh꜓iyā \\
\end{twochants}

\begin{english}
  `Todas as formações condicionadas são impermanentes' ---\\
  quando se vê isto com sabedoria,\\
  desencantamo-nos do sofrimento.\\
  Este é o caminho da purificação.
\end{english}

\begin{twochants}
  Sa꜕bbe sa꜓ṅkhā꜓rā du꜕kkhā't꜕i & yadā paññāya꜓ pa꜕ssa꜕ti \\
  Atha nibbinda꜕ti dukkhe & esa꜕ maggo vi꜓su꜕ddh꜓iyā \\
\end{twochants}

\begin{english}
  `Todas as formações condicionadas são sofrimento' ---\\
  quando se vê isto com sabedoria,\\
  desencantamo-nos do sofrimento.\\
  Este é o caminho da purificação.
\end{english}

\begin{twochants}
  Sa꜕bbe dhammā ana꜓ttā'ti꜕ & yadā paññāya꜓ pa꜕ssa꜕ti \\
  Atha nibbinda꜕ti dukkhe & esa꜕ maggo vi꜓su꜕ddh꜓iyā \\
\end{twochants}

\begin{english}
  `Todas as formações são desprovidas de eu' ---\\
  quando se vê isto com sabedoria,\\
  desencantamo-nos do sofrimento.\\
  Este é o caminho da purificação.
\end{english}

\clearpage

\begin{twochants}
  A꜕ppa꜕kā te manusse꜓su꜕ & ye janā pāra-gāmi꜓no \\
  A꜕thāyaṁ i꜕ta꜕rā pajā & tīram-evānudhā꜓va꜕ti \\
\end{twochants}

\begin{english}
  Poucos são aqueles\\
  que atravessam para a outra margem.\\
  No entanto, muitos são aqueles\\
  que nesta margem vagueiam, sem rumo.
\end{english}

\begin{twochants}
  Ye ca꜕ kho sammad-akkhāte & dhamme dhammānuva꜓tt꜕ino \\
  Te ja꜕nā pā꜕ram-essanti & ma꜕ccu-dheyyaṁ sudu꜓tta꜕raṁ \\
\end{twochants}

\begin{english}
  Onde quer que o Dhamma seja bem ensinado,\\
  os que praticarem de acordo com ele\\
  irão atravessar para além\\
  do âmbito da vida e da morte, tão difícil de escapar.
\end{english}

\begin{twochants}
  Kaṇhaṁ dhammaṁ vi꜕ppahā꜓ya & su꜕kkaṁ bhāvetha꜕ paṇḍi꜓to \\
  Okā a꜕noka꜕m-āgamma & viveke ya꜕tth꜕a dūramaṁ \\
  Ta꜕trābh꜕irat꜕im-iccheyya & hi꜕tvā kāme a꜕kiñc꜓ano \\
\end{twochants}

\begin{english}
  Abandonando a escuridão,\\
  os sábios cultivam a claridade;\\
  deixando para trás áreas inundadas,\\
  eles alcançam terra firme.\\
  Embora seja difícil encontrar deleite\\
  na vida solitária,\\
  tal deleite deve ser almejado,\\
  renunciando aos prazeres sensuais\\
  e tendo tudo abandonado.
\end{english}

\chapter{Versos sobre o Fardo}

\firstline{Bhārā have pañcakkhandhā}

\begin{leader}
  [Ha꜓nda mayaṁ bhāra-su꜕tta-gāthā꜓yo bha꜕ṇāmase]
\end{leader}

\begin{twochants}
Bhārā ha꜕ve pañcakkha꜓ndhā & bhāra-hāro ca pu꜓gga꜕lo \\
Bhā꜕r'ādānaṁ du꜕kkhaṁ loke꜓ & bhāra-nikkhe꜓pa꜕naṁ su꜕khaṁ \\
\end{twochants}

\begin{english}
  De facto, os cinco agregados são um fardo,\\
  e quem carrega esse fardo é o indivíduo.\\
  Neste mundo, agarrar em fardos é sofrimento;\\
  largá-los, é felicidade.
\end{english}

\begin{twochants}
Nikkhipi꜕tvā ga꜕ruṁ bhā꜓raṁ & aññaṁ bhāraṁ anā꜓di꜕ya \\
Sa꜕mūlaṁ taṇhaṁ a꜕bbuyha & nicchāto pa꜕ri꜕nibbu꜕to \\
\end{twochants}

\begin{english}
  Tendo abandonado o pesado fardo,\\
  não agarrando em nova carga,\\
  tendo arrancado o anseio pela raiz,\\
  os desejos apaziguam-se e o ser fica liberto.
\end{english}

\chapter{Versos sobre uma Noite Auspiciosa}

\firstline{Atītaṁ nānvāgameyya}

\begin{leader}
  [Ha꜓nda mayaṁ bhadd'eka-ratta꜕-gāthā꜓yo bha꜕ṇāmase]
\end{leader}

\begin{twochants}
  A꜕tītaṁ nānvāga꜕meyya & nappa꜕ṭikaṅkhe꜓ a꜕nāga꜓taṁ \\
  Ya꜕d'a꜕tītaṁ pa꜕hīnan-taṁ & a꜕ppattañc꜕a a꜕nāga꜕taṁ \\
\end{twochants}

\begin{english}
  Não reviva o passado,\\
  nem especule sobre o futuro.\\
  O passado já aconteceu\\
  e o futuro ainda não se realizou.
\end{english}

\begin{twochants}
  Paccu꜕ppannañca꜕ yo dhammaṁ & tattha tattha vi꜓pa꜕ss꜕ati \\
  Asa꜓ṁhi꜕raṁ asa꜓ṅku꜕ppaṁ & taṁ viddhām-a꜕nu꜕brūhaye \\
\end{twochants}

\begin{english}
  Assim, tudo o que surge no presente,\\
  é então visto por ele, claramente:\\
  inabalado, livre de agitação ---\\
  essa realização será a sua força.
\end{english}

\begin{twochants}
  A꜕jj'eva ki꜕cca꜕m-ātappaṁ & ko jaññā ma꜓ra꜕ṇaṁ su꜕ve \\
  Na hi no sa꜓ṅga꜕ran-tena & mahā-senena ma꜓cc꜕unā \\
\end{twochants}

\begin{english}
  Hoje ardentemente dedicado à sua tarefa\\
  pois, quem sabe, amanhã a morte virá.\\
  E perante o poderoso exército da morte,\\
  não há negociação possível.
\end{english}

\clearpage

\begin{twochants}
  Evaṁ vihārim-ātāpiṁ & a꜕ho-rattam-a꜕tandi꜓taṁ \\
  Taṁ ve bha꜕dd'eka꜕-ratto'ti & santo ā꜕ci꜕kkha꜕te muni \\
\end{twochants}

\begin{english}
  Aquele que assim vive com vigor,\\
  sustendo-o dia e noite,\\
  tem de facto `uma noite auspiciosa' ---\\
  assim disse o Pacífico Sábio.
\end{english}

\chapter{Versos de Respeito pelo Dhamma}

\firstline{Ye ca atītā sambuddhā}

\begin{leader}
  [Ha꜓nda mayaṁ dhamma-gā꜕rav'ādi꜕-gāthā꜓yo bha꜕ṇāmase]
\end{leader}

\begin{twochants}
  Ye ca꜕ atītā sa꜓mbuddhā & ye ca꜕ buddhā a꜕nāga꜓tā \\
  Yo c'eta꜕rahi sambuddho & ba꜕hunnaṁ so꜕ka꜕-nāsa꜕no \\
\end{twochants}

\begin{english}
  Todos os Buddhas do passado,\\
  todos os Buddhas do futuro,\\
  o Buddha actual ---\\
  tais dissipadores do pesar.
\end{english}

\begin{twochants}
  Sa꜕bbe sa꜕ddhamma-gar꜓uno & vi꜕ha꜕riṁsu vi꜕ha꜕ranti ca \\
  A꜕tho pi viha꜕riss꜓anti & esā buddhāna꜓ dha꜕mma꜕tā \\
\end{twochants}

\begin{english}
  Aqueles que viveram ou que estão vivos,\\
  e aqueles que viverão no futuro,\\
  todos reverenciam o Verdadeiro Dhamma,\\
  tal é a natureza dos Buddhas.
\end{english}

\begin{twochants}
  Tasmā h꜕i atta-kāmena & mahattam-abhika꜓ṅkh꜕atā \\
  Sa꜕ddhammo ga꜕ru꜓-kāta꜕bbo & s꜕araṁ buddhāna sā꜓sa꜕naṁ \\
\end{twochants}

\begin{english}
  Assim, almejando o bem-estar\\
  e seguindo as mais elevadas aspirações,\\
  considere o Verdadeiro Dhamma\\
  lembrando-se dos ensinamentos do Buddha.
\end{english}

\clearpage

\begin{twochants}
  Na h꜕i dhammo a꜕dhammo ca & ubho s꜕ama-vipāki꜓no \\
  A꜕dhammo nirayaṁ neti & dh꜕ammo pāpeti꜕ su꜕gga꜕tiṁ \\
\end{twochants}

\begin{english}
  O verdadeiro Dhamma e o falso\\
  nunca trarão os mesmos resultados.\\
  A ausência de Dhamma leva a estados infernais,\\
  o verdadeiro Dhamma leva a estados elevados.
\end{english}

Dhammo ha꜕ve rakkha꜕ti꜕ dhamma꜓-cāriṁ\\
Dhammo su꜕ciṇṇo su꜕kham-āvahāti\\
Esā꜓ni꜕saṁso dhamme su꜕ciṇṇe

\begin{english}
  O Dhamma protege aquele que o cultiva\\
  e conduz à felicidade quando bem praticado ---\\
  esta é a benção do Dhamma devidamente cultivado.
\end{english}

\chapter{Ovāda-Pāṭimokkha}

\firstline{Khantī paramaṁ tapo tītikkhā}

\enlargethispage{\baselineskip}

\begin{leader}
  [Ha꜓nda mayaṁ ovāda-pā꜕ṭi꜕mokkha-gāthā꜓yo bha꜕ṇāmase]
\end{leader}

Kha꜓ntī pa꜕ramaṁ ta꜕po tīti꜕kkhā

\begin{english}
  Permanecer paciente é a maior austeridade.
\end{english}

Nibbānaṁ pa꜕ramaṁ va꜕dant꜕i buddhā

\begin{english}
  “Nibbāna é supremo”, dizem os Buddhas.
\end{english}

Na h꜕i pa꜕bbaji꜕to pa꜕rūpaghātī

\begin{english}
  Não se é verdadeiramente monge \pause\ quando se prejudica alguém,
\end{english}

Sa꜕maṇo ho꜓ti pa꜕raṁ vihe꜓ṭha꜕yanto

\begin{english}
  nem verdadeiramente renunciante \pause\ quando se oprime os outros.
\end{english}

Sa꜕bb꜕a-pāpa꜕ss꜕a a꜕ka꜕ra꜓ṇaṁ

\begin{english}
  Evitar todo o mal,
\end{english}

Ku꜕salassūpasa꜓mpa꜕dā

\begin{english}
  cultivar o bem
\end{english}

Sa꜕ci꜕tta-pa꜕ri꜓yoda꜓pa꜕naṁ

\begin{english}
  e purificar a mente ---
\end{english}

Etaṁ buddhāna sā꜓sa꜕naṁ

\begin{english}
  Este é o ensinamento dos Buddhas.
\end{english}

A꜕nūpa꜕vādo a꜕nūpa꜕ghāto

\begin{english}
  Não ofender, não prejudicar,
\end{english}

Pā꜕ṭimokkhe꜓ ca꜕ sa꜓ṁva꜕ro

\begin{english}
  moderar-se de acordo com o que leva à libertação.
\end{english}

Mattaññu꜕tā ca꜕ bhatta꜕smiṁ

\begin{english}
  moderar-se na alimentação,
\end{english}

Pa꜕ntañca꜕ saya꜓n'āsa꜕naṁ

\begin{english}
  viver solitário,
\end{english}

A꜕dhici꜕tte ca꜕ āyogo

\begin{english}
  devotar-se à consciência elevada ---
\end{english}

Etaṁ buddhāna sā꜓sa꜕naṁ

\begin{english}
  Este é o ensinamento dos Buddhas.
\end{english}

\chapter[A Primeira Exortação]{Versos sobre a Primeira Exortação do Buddha}

\firstline{Aneka-jāti-saṁsāraṁ}

\begin{leader}
  [Ha꜓nda mayaṁ paṭhama-bu꜕ddha-bhāsi꜕ta-gāthāyo bh꜕aṇāmase]
\end{leader}

\begin{twochants}
  A꜕neka꜕-jāti꜕-sa꜓ṁsā꜓raṁ & sa꜕ndhāviss꜓aṁ a꜕nibbi꜕saṁ \\
  Ga꜕ha-kā꜕raṁ ga꜕vesa꜓nto & dukkhā jāt꜕i pu꜕nappu꜕naṁ \\
\end{twochants}

\begin{english}
  Durante muitas vidas, na roda da vida e da morte\\
  vagueei infinitamente.\\
  O construtor desta casa eu busquei;\\
  quão penosos são os repetidos nascimentos.
\end{english}

\begin{twochants}
  Ga꜕ha-kā꜕raka꜕ diṭṭho꜓'si & pu꜕na gehaṁ na kā꜓hasi \\
  Sa꜕bbā te phāsu꜕kā bhaggā & gaha-kūṭa꜓ṁ vi꜕saṅkh꜕ataṁ \\
  Visa꜓ṅkhā꜕ra-ga꜕taṁ ci꜕ttaṁ & taṇhānaṁ kh꜕aya꜕m-ajjh꜕agā \\
\end{twochants}

\begin{english}
  Ó construtor da casa, foste visto!\\
  Não construirás casa novamente.\\
  Todas as tuas vigas estão partidas\\
  e a cumeeira esmagada.\\
  A mente atingiu o Incondicionado;\\
  chegando à cessação do anseio.
\end{english}

\chapter[As Últimas Instruções]{Versos sobre as Últimas Instruções}

\firstline{Handa dāni bhikkhave āmantayāmi vo}

\begin{leader}
  [Ha꜓nda mayaṁ pacchima-ovāda-gāthā꜓yo bha꜕ṇāmase]
\end{leader}

Handa dāni bhi꜓kkha꜕ve āmant꜕ayāmi꜓ vo

\begin{english}
  Agora monges, eu vos digo:
\end{english}

Vaya-dhammā sa꜓ṅkhā꜓rā

\begin{english}
  a mudança é a natureza das coisas condicionadas.
\end{english}

A꜕ppamādena sa꜓mpā꜕dethā'ti

\begin{english}
  Pratiquem diligentemente --
\end{english}

Ayaṁ tathā꜓ga꜕tassa pa꜕cchi꜓mā vācā

\begin{english}
  estas são as últimas palavras do Tathāgata.
\end{english}

\chapter{Surgir a Partir de uma Causa}

\firstline{Ye dhammā hetuppabhavā}

\begin{leader}
  [Ha꜓nda mayaṁ assajithera-gāthā꜓yo bha꜕ṇāmase]
\end{leader}

Ye dhammā hetuppabhavā

\begin{english}
  Todos os fenómenos surgem de uma causa:
\end{english}

Tesaṁ hetuṁ tathāgato āha

\begin{english}
  o Tathāgata explicou a sua causa
\end{english}

Tesañca yo nirodho

\begin{english}
  e também a sua cessação.
\end{english}

Evaṁ-vādī mahāsamaṇo'ti

\begin{english}
  Este é o ensinamento do Grande Asceta.
\end{english}

% \suttaRef{Mv.1.23.5}

\chapter[Incondicionado]{Reflexão sobre o Incondicionado}

\firstline{Atthi bhikkhave ajātaṁ abhūtaṁ akataṁ}

\begin{leader}
  [Ha꜓nda mayaṁ nibbāna-sutta-pāṭhaṁ bha꜕ṇāmase]
\end{leader}

Atthi bhi꜓kkha꜕ve a꜕jātaṁ a꜓bhūtaṁ a꜕kataṁ a꜕sa꜓ṅkh꜕ataṁ

\begin{english}
  Existe um Não-nascido, Não-originado, Incriado, Não-formado.
\end{english}

N꜕o cetaṁ bhi꜓kkha꜕ve a꜕bhavissa a꜕jātaṁ a꜓bhūtaṁ a꜕kataṁ a꜕sa꜓nkh꜕ataṁ

\begin{english}
 Se não existisse este Não-nascido, Não-originado, Incriado, Não-formado,
\end{english}

Na꜕ yidaṁ jātassa꜕ bhūtassa ka꜕tassa sa꜓ṅkh꜕atassa nissaraṇaṁ paññāye꜓tha

\begin{english}
  A libertação do mundo do nascido, originado, criado, formado, não seria possível.
\end{english}

Ya꜕smā ca kho bhi꜓kkh꜕ave atthi a꜕jātaṁ a꜓bhūtaṁ a꜕kataṁ a꜕sa꜓ṅkha꜕taṁ

\begin{english}
  Mas uma vez que existe um Não-nascido, Não-originado, Incriado, Não-formado,
\end{english}

Ta꜕smā jātass꜕a bhūtassa ka꜕tassa sa꜓ṅkha꜕tassa nissaraṇaṁ paññāyati

\begin{english}
  Assim é possível a libertação do mundo do nascido, originado, criado, formado.
\end{english}

\chapter[Breve Conselho a Gotamī]{Breve Conselho a Gotamī}

\emph{Saṅkhitta-gotamiyovāda Sutta, AN 8.53}

% https://suttacentral.net/an8.53/pli/ms

\begin{leader}
  [Handa mayaṁ saṅkhitta-gotamiyovāda-sutta-pāṭhaṁ bhaṇāmase]
\end{leader}

Ye kho tvaṁ, gotami, dhamme jāneyyāsi:\\
‘ime dhammā sarāgāya saṁvattanti, no virāgāya;\\
saṁyogāya saṁvattanti, no visaṁyogāya;\\
ācayāya saṁvattanti, no apacayāya;\\
mahicchatāya saṁvattanti, no appicchatāya;\\
asantuṭṭhiyā saṁvattanti, no santuṭṭhiyā;\\
saṅgaṇikāya saṁvattanti, no pavivekāya;\\
kosajjāya saṁvattanti, no vīriyārambhāya;\\
dubbharatāya saṁvattanti, no subharatāyā’ti;

\begin{english}
  Gotamī, as qualidades que conheceis\\
  que conduzem à paixão, não ao desencantamento;\\
  à obstrução, não à desobstrução;\\
  ao acúmulo, não ao largar;\\
  ao engrandecimento pessoal, não à modéstia;\\
  ao descontentamento, não ao contentamento;\\
  ao enredamento, não à reclusão;\\
  à preguiça, não à persistência vivaz;\\
  a ser um fardo, não a ser fácil de lidar;
\end{english}

ekaṁsena, gotami, dhāreyyāsi: ‘neso dhammo, neso vinayo, netaṁ satthusāsanan’ti.

\begin{english}
  podeis afirmar categoricamente: ‘Isto não é o Dhamma,\\
  isto não é o Vinaya, isto não é a instrução do Professor.’
\end{english}

Ye ca kho tvaṁ, gotami, dhamme jāneyyāsi:\\
‘ime dhammā virāgāya saṁvattanti, no sarāgāya;\\
visaṁyogāya saṁvattanti, no saṁyogāya;\\
apacayāya saṁvattanti, no ācayāya;\\
appicchatāya saṁvattanti, no mahicchatāya;\\
santuṭṭhiyā saṁvattanti, no asantuṭṭhiyā;\\
pavivekāya saṁvattanti, no saṅgaṇikāya;\\
vīriyārambhāya saṁvattanti, no kosajjāya;\\
subharatāya saṁvattanti, no dubbharatāyā’ti;

\begin{english}
  Quanto às qualidades que conheceis\\
  que conduzem ao desencantamento, não à paixão;\\
  à desobstrução, não à obstrução;\\
  ao largar, não ao acúmulo;\\
  à modéstia, não ao engrandecimento pessoal;\\
  ao contentamento, não ao descontentamento;\\
  à reclusão, não ao enredamento;\\
  à persistência vivaz, não à preguiça;\\
  a ser fácil de lidar, a não ser um fardo;
\end{english}

ekaṁsena, gotami, dhāreyyāsi: ‘eso dhammo, eso vinayo, etaṁ satthusāsanan’ti.

\begin{english}
  podeis afirmar categoricamente: ‘Este é o Dhamma,\\
  este é o Vinaya, esta é a instrução do Professor.’
\end{english}

\chapter{A Raiz de Todas as Coisas}

% AN 10.58
% https://www.accesstoinsight.org/tipitaka/an/an10/an10.058.than.html
% https://www.dhammatalks.org/suttas/AN/AN10_58.html
% https://suttacentral.net/an10.58/pli/ms
% https://suttacentral.net/an10.58/en/sujato
% https://suttacentral.net/an10.58/en/bodhi

\firstline{Kiṁ-mūlakā āvuso sabbe dhammā}

\begin{leader}
  [Ha꜓nda mayam mūlaka-sutta-pāṭhaṁ bha꜕ṇāmase]
\end{leader}

\enlargethispage{2\baselineskip}

Kiṁ-mūlakā āvuso sabbe dhammā\\
kiṁ-sambhavā sabbe dhammā\\
kiṁ-samudayā sabbe dhammā\\
kiṁ-samosaraṇā sabbe dhammā\\
kiṁ-pamukhā sabbe dhammā\\
kiṁ-adhipateyyā sabbe dhammā\\
kiṁ-uttarā sabbe dhammā\\
kiṁ-sārā sabbe dhammā\\
kiṁ-ogadhā sabbe dhammā\\
kiṁ-pariyosānā sabbe dhammā'ti.

Chanda'mūlakā āvuso sabbe dhammā\\
manasikāra'sambhavā sabbe dhammā\\
phassa'samudayā sabbe dhammā\\
vedanā'samosaraṇā sabbe dhammā\\
samādhi'ppamukhā sabbe dhammā\\
satā'dhipateyyā sabbe dhammā\\
paññ'uttarā sabbe dhammā\\
vimutti'sārā sabbe dhammā\\
amat'ogadhā sabbe dhammā\\
nibbāna'pariyosānā sabbe dhammā'ti.

\clearpage

\begin{english}
  Enraizadas no quê, amigo, são todas as coisas?\\
  Nascidas do quê, são todas as coisas?\\
  Provenientes de quê, são todas as coisas?\\
  Convergindo no quê, são todas as coisas?\\
  Dirigidas pelo quê, são todas as coisas?\\
  Dominadas pelo quê, são todas as coisas?\\
  Superáveis pelo quê, são todas as coisas?\\
  Resultam em quê como essência, todas as coisas?\\
  Fundem-se em quê, todas as coisas?\\
  Terminam em quê, todas as coisas?

  \bigskip

  Amigo, todas as coisas são enraizadas no desejo.\\
  Todas as coisas nascem da atenção.\\
  Todas as coisas provêm do contacto.\\
  Todas as coisas convergem na sensação.\\
  Todas as coisas são dirigidas pela concentração.\\
  Todas as coisas são dominadas pela consciência.\\
  Todas as coisas são superáveis pela sabedoria.\\
  Todas as coisas resultam em libertação como essência.\\
  Todas as coisas fundem-se na imortalidade.\\
  Todas as coisas terminam em Nibbāna.

\end{english}

{\raggedleft
  \emph{Aṅguttara Nikāya 10.58}
\par}

\chapter{Ānāpānassati-sutta}

\firstline{Ānāpānassati bhikkhave bhāvitā bahulī-katā}

\begin{leader}
  [Ha꜓nda mayam ānāpānass꜕ati-sutta-pāṭhaṁ bha꜕ṇāmase]
\end{leader}

Ānāpāna꜓ssa꜕ti bhi꜓kkha꜕ve bhāvi꜓tā bahu꜕līka꜕tā

\begin{english}
  Bhikkhus, quando ānāpānassati é cultivada e desenvolvida,
\end{english}

Mahappha꜕lā ho꜓ti mahā꜓nisa꜓ṁsā

\begin{english}
  é de grande benefício, de grande fruto.
\end{english}

Ānāpāna꜓ssa꜕ti bhi꜓kkha꜕ve bhāvi꜓tā bahu꜕līka꜕tā

\begin{english}
  Bhikkhus, quando cultivada e desenvolvida, ānāpānassati
\end{english}

Ca꜕ttāro sati꜓pa꜕ṭṭhāne pa꜕ri꜓pū꜕reti

\begin{english}
  leva as Quatro Fundações de Sati à plenitude;
\end{english}

Ca꜕ttāro sa꜕tipa꜕ṭṭhānā bhāvi꜓tā bahu꜕līka꜕tā

\begin{english}
  quando cultivadas e desenvolvidas, as Quatro Fundações de Sati
\end{english}

Sa꜕tta-bojjhaṅge pa꜕ri꜓pū꜕renti

\begin{english}
  levam os Sete Factores do Despertar à plenitude;
\end{english}

Sa꜕tta-bojjhaṅgā bhāvi꜓tā bahu꜕līka꜕tā

\begin{english}
  quando cultivados e desenvolvidos, os Sete Factores do Despertar
\end{english}

\enlargethispage{\baselineskip}

Vijjā-vimuttiṁ pa꜕ri꜓pū꜕renti

\begin{english}
  levam o verdadeiro conhecimento e libertação à plenitude.
\end{english}

Kathaṁ bhāvi꜓tā ca bhi꜓kkha꜕ve ānāpāna꜓ss꜕ati ka꜕thaṁ bahu꜕līka꜕tā

\begin{english}
  E como, bhikkhus, ānāpānassati é cultivada e desenvolvida
\end{english}

Mahappha꜕lā ho꜓ti mahā꜓nisa꜓ṁsā

\begin{english}
  para que seja de grande benefício, de grande fruto?
\end{english}

Idha bhi꜓kkha꜕ve bhikkhu

\begin{english}
  Bhikkhus, aqui um bhikkhu
\end{english}

Arañña꜓-ga꜕to vā

\begin{english}
  tendo ido para a floresta,
\end{english}

Rukkha-mūla꜓-ga꜕to vā

\begin{english}
  para a raiz de uma árvore
\end{english}

Suññāgāra꜓-ga꜕to vā

\begin{english}
  ou para uma cabana vazia,
\end{english}

N꜕isīdati pallaṅkaṁ ābhuji꜓tv꜕ā

\begin{english}
  senta-se de pernas cruzadas,
\end{english}

Ujuṁ kāyaṁ pa꜕ṇidhāya pa꜕rimukhaṁ sa꜕tiṁ u꜕paṭṭha꜕petvā

\begin{english}
  com o seu corpo direito e estabelece sati à sua frente.
\end{english}

So sa꜕to'va a꜕ssasa꜕ti sa꜕to'va pa꜕ssa꜕sa꜕ti

\begin{english}
  Consciente, ele inspira; consciente, ele expira.
\end{english}

Dīghaṁ vā assa꜕sa꜓nto dīghaṁ a꜕ssasā꜓mī'ti pa꜕jānāti

\begin{english}
  Ao ter uma inspiração longa, ele sabe `Esta é uma inspiração longa';
\end{english}

Dīghaṁ vā pa꜕ssa꜕santo dīghaṁ pa꜕ssasā꜓mī'ti pa꜕jānāti

\begin{english}
  Ao ter uma expiração longa, ele sabe `Esta é uma expiração longa';
\end{english}

Rassaṁ vā a꜕ssa꜕santo rassaṁ a꜕ssasā꜓mī'ti pa꜕jānāti

\begin{english}
  Ao ter uma inspiração curta, ele sabe `Esta é uma inspiração curta';
\end{english}

Rassaṁ vā pa꜕ssa꜕santo rassaṁ pa꜕ssasā꜓mī'ti pa꜕jānāti

\begin{english}
  Ao ter uma expiração curta, ele sabe `Esta é uma expiração curta';
\end{english}

Sabba꜕-kāya-paṭ꜕isa꜓ṁvedī a꜕ssasi꜕ssāmī'ti si꜕kkh꜕ati

\begin{english}
  Ele treina-se: `Irei inspirar experienciando o corpo inteiro'.
\end{english}

Sabba꜕-kāya-paṭ꜕isa꜓ṁvedī pa꜕ssasi꜕ssāmī'ti si꜕kkh꜕ati

\begin{english}
  Ele treina-se: `Irei expirar experienciando o corpo inteiro'.
\end{english}

Passa꜕mbhayaṁ kāya꜕-sa꜓ṅkhāraṁ a꜕ssasi꜕ssāmī'ti si꜕kkh꜕ati

\begin{english}
  Ele treina-se: `Irei inspirar pacificando o corpo'.
\end{english}

Passa꜕mbhayaṁ kāya꜕-sa꜓ṅkhāraṁ pa꜕ssasi꜕ssāmī'ti si꜕kkh꜕ati

\begin{english}
  Ele treina-se: `Irei expirar pacificando o corpo'.
\end{english}

Pīti꜕-paṭi꜕sa꜓ṁvedī a꜕ssasi꜕ssāmī'ti si꜕kkh꜕ati

\begin{english}
  Ele treina-se: `Irei inspirar experienciando êxtase'.
\end{english}

Pīti꜕-paṭi꜕sa꜓ṁvedī pa꜕ssasi꜕ssāmī'ti si꜕kkh꜕ati

\begin{english}
  Ele treina-se: `Irei expirar experienciando êxtase'.
\end{english}

Sukh꜕a-paṭi꜕sa꜓ṁvedī a꜕ssasi꜕ssāmī'ti si꜕kkh꜕ati

\begin{english}
  Ele treina-se: `Irei inspirar experienciando felicidade'.
\end{english}

Sukh꜕a-paṭi꜕sa꜓ṁvedī pa꜕ssasi꜕ssāmī'ti si꜕kkh꜕ati

\begin{english}
  Ele treina-se: `Irei expirar experienciando felicidade'.
\end{english}

Citta꜕-sa꜓ṅkhāra-paṭi꜕sa꜓ṁvedī a꜕ssasi꜕ssāmī'ti si꜕kkh꜕ati

\begin{english}
  Ele treina-se: `Irei inspirar experienciando as formações mentais'.
\end{english}

Citta꜕-sa꜓ṅkhāra-paṭi꜕sa꜓ṁvedī pa꜕ssasi꜕ssāmī'ti si꜕kkh꜕ati

\begin{english}
  Ele treina-se: `Irei expirar experienciando as formações mentais'.
\end{english}

Passa꜕mbhayaṁ citta꜕-sa꜓ṅkhāraṁ a꜕ssasi꜕ssāmī'ti si꜕kkh꜕ati

\begin{english}
  Ele treina-se: `Irei inspirar pacificando as formações mentais'.
\end{english}

Passa꜕mbhayaṁ citt꜕a-sa꜓ṅkhāraṁ pa꜕ssasi꜕ssāmī'ti si꜕kkh꜕ati

\begin{english}
  Ele treina-se: `Irei expirar pacificando as formações mentais'.
\end{english}

Citta꜕-paṭi꜕sa꜓ṁvedī a꜕ssasi꜕ssāmī'ti si꜕kkh꜕ati

\begin{english}
  Ele treina-se: `Irei inspirar experienciando a mente'.
\end{english}

Citta꜕-paṭi꜕sa꜓ṁvedī pa꜕ssasi꜕ssāmī'ti si꜕kkh꜕ati

\begin{english}
  Ele treina-se: `Irei expirar experienciando a mente'.
\end{english}

A꜕bhippa꜕moda꜓yaṁ cittaṁ a꜕ssasi꜕ssāmī'ti si꜕kkh꜕ati

\begin{english}
  Ele treina-se: `Irei inspirar alegrando a mente'.
\end{english}

A꜕bhippa꜕moda꜓yaṁ cittaṁ pa꜕ssasi꜕ssāmī'ti si꜕kkh꜕ati

\begin{english}
  Ele treina-se: `Irei expirar alegrando a mente'.
\end{english}

Sa꜕māda꜓haṁ cittaṁ a꜕ssasi꜕ssāmī'ti si꜕kkh꜕ati

\begin{english}
  Ele treina-se: `Irei inspirar concentrando a mente'.
\end{english}

Sa꜕māda꜓haṁ cittaṁ pa꜕ssasi꜕ssāmī'ti si꜕kkh꜕ati

\begin{english}
  Ele treina-se: `Irei expirar concentrando a mente'.
\end{english}

Vimoca꜓yaṁ cittaṁ a꜕ssasi꜕ssāmī'ti si꜕kkh꜕ati

\begin{english}
  Ele treina-se: `Irei inspirar libertando a mente'.
\end{english}

Vimoca꜓yaṁ cittaṁ pa꜕ssasi꜕ssāmī'ti si꜕kkh꜕ati

\begin{english}
  Ele treina-se: `Irei expirar libertando a mente'.
\end{english}

Aniccānupa꜕ssī a꜕ssasi꜕ssāmī'ti si꜕kkh꜕ati

\begin{english}
  Ele treina-se: `Irei inspirar contemplando a impermanência'.
\end{english}

Aniccānupa꜕ssī pa꜕ssasi꜕ssāmī'ti si꜕kkh꜕ati

\begin{english}
  Ele treina-se: `Irei expirar contemplando a impermanência'.
\end{english}

Virāgānupa꜕ssī a꜕ssasi꜕ssāmī'ti si꜕kkh꜕ati

\begin{english}
  Ele treina-se: `Irei inspirar contemplando o desvanecer das paixões'.
\end{english}

Virāgānupa꜕ssī pa꜕ssasi꜕ssāmī'ti si꜕kkh꜕ati

\begin{english}
  Ele treina-se: `Irei expirar contemplando o desvanecer das paixões'.
\end{english}

Nirodhānupa꜕ssī a꜕ssasi꜕ssāmī'ti si꜕kkh꜕ati

\begin{english}
  Ele treina-se: `Irei inspirar contemplando cessação'.
\end{english}

Nirodhānupa꜕ssī pa꜕ssasi꜕ssāmī'ti si꜕kkh꜕ati

\begin{english}
  Ele treina-se: `Irei expirar contemplando cessação'.
\end{english}

Pa꜕ṭiniss꜕aggānupa꜕ssī a꜕ssasi꜕ssāmī'ti si꜕kkh꜕ati

\begin{english}
  Ele treina-se: `Irei inspirar contemplando a renúncia'.
\end{english}

Pa꜕ṭinissa꜕ggānupa꜕ssī pa꜕ssasi꜕ssāmī'ti si꜕kkh꜕ati

\begin{english}
  Ele treina-se: `Irei expirar contemplando a renúncia'.
\end{english}

Evaṁ bhāvi꜓tā kho bhi꜓kkha꜕ve ānāpāna꜓ss꜕ati evaṁ bahu꜕līka꜕tā

\begin{english}
  Bhikkhus, assim ānāpānassati é desenvolvida e cultivada,
\end{english}

Mahappha꜕lā ho꜓ti mahā꜓nisa꜓ṁsā'ti

\begin{english}
  de forma a dar grandes frutos e a ser de grande benefício.
\end{english}


\chapter[Aparihāniya-dhamma-sutta]{Bhikkhu-aparihāniya-dhamma-sutta}

\emph{Sete condições para a bem-aventurança dos bhikkhus, AN 7.23}

\begin{leader}
  [Handa mayaṁ bhikkhu-aparihāniya-dhamma-suttaṁ bhaṇāmase]
\end{leader}

[Evaṁ me sutaṁ.] Ekaṁ samayaṁ bhagavā rājagahe꜔꜒ viharati gijjhakūṭe pabbate.
Tatra kho꜔꜒ bhagavā bhikkhū꜔꜒ āmantesi: Satta vo, bhikkhave, aparihā꜔꜒niye dhamme
desessā꜔꜒mi. Taṁ suṇātha, sā꜔꜒dhukaṁ manasi karotha, bhāsissā꜔꜒mī'ti. Evaṁ, bhante'ti
kho꜔꜒ te bhikkhū꜔꜒ bhagavato paccasso꜔꜒su꜔꜒ṁ. Bhagavā etadavoca:

\begin{english}
  Eu ouvi que em certa ocasião o Excelso estava em Rajagaha, no Pico dos
  Abutres. Ali, ele dirigiu-se aos monges: `Monges, eu irei ensinar-vos as sete
  condições que não levam ao declínio. Ouçam e prestem muita atenção. Eu vou
  falar.' `Sim, senhor', responderam os monges. O Excelso disse:
\end{english}

[1] Katame ca, bhikkhave, satta aparihā꜔꜒niyā dhammā? Yāvakīvañca, bhikkhave, bhikkhū꜔꜒
abhiṇha꜔꜒ṁ sa꜔꜒nnipātā bhavissa꜔꜒nti sa꜔꜒nnipātabahulā; vuddhiyeva, bhikkhave,
bhikkhū꜔꜒naṁ pāṭikaṅkhā꜔꜒, no parihā꜔꜒ni.

\begin{english}
  `E quais são as sete condições que não levam ao declínio? Desde que os monges
  se reúnam com frequência, se reúnam assiduamente, o seu crescimento pode ser
  esperado, não o seu declínio.'

\end{english}

[2] Yāvakīvañca, bhikkhave, bhikkhū꜔꜒ samaggā sa꜔꜒nnipatissa꜔꜒nti, samaggā
vuṭṭhahissa꜔꜒nti, samaggā sa꜔꜒ṅghakaraṇīyāni karissa꜔꜒nti; vuddhiyeva, bhikkhave,
bhikkhū꜔꜒naṁ pāṭikaṅkhā꜔꜒, no parihā꜔꜒ni.

\begin{english}
  `Enquanto os monges se reunirem em harmonia, terminarem as suas reuniões em
  harmonia e conduzirem os assuntos da Saṅgha em harmonia, o seu crescimento
  pode ser esperado, não o seu declínio.'
\end{english}

[3] Yāvakīvañca, bhikkhave, bhikkhū꜔꜒ apaññattaṁ na paññāpessa꜔꜒nti, paññattaṁ na
samucchi꜔꜒ndissa꜔꜒nti, yathā꜔꜒paññattesu sikkhā꜔꜒padesu samādāya vattissa꜔꜒nti;
vuddhiyeva, bhikkhave, bhikkhū꜔꜒naṁ pāṭikaṅkhā꜔꜒, no parihā꜔꜒ni.

\begin{english}
  `Enquanto os monges nem decretarem o que não foi decretado, nem revogarem o que
  foi decretado, mas praticarem o cumprimento das regras de treino conforme
  foram decretadas, o seu crescimento pode ser esperado, não o seu declínio.'
\end{english}

[4] Yāvakīvañca, bhikkhave, bhikkhū꜔꜒ ye te bhikkhū꜔꜒ the꜔꜒rā rattaññū cirapabbajitā
sa꜔꜒ṅghapitaro sa꜔꜒ṅghapariṇāyakā te sakkarissa꜔꜒nti garuṁ karissa꜔꜒nti mānessa꜔꜒nti
pūjessa꜔꜒nti, tesa꜔꜒ñca so꜔꜒tabbaṁ maññissa꜔꜒nti; vuddhiyeva, bhikkhave, bhikkhū꜔꜒naṁ
pāṭikaṅkhā꜔꜒, no parihā꜔꜒ni.

\begin{english}
  `Enquanto os monges honrarem, respeitarem, venerarem e prestarem homenagem aos
  monges mais velhos -- aqueles com antiguidade que foram ordenados há muito
  tempo, os pais do Sangha, os líderes do Sangha -- considerando muito valioso
  ouvi-los, o seu crescimento pode ser esperado, não o seu declínio.'
\end{english}

\enlargethispage{2\baselineskip}

[5] Yāvakīvañca, bhikkhave, bhikkhū꜔꜒ uppannāya taṇhā꜔꜒ya ponobhavikāya na vasa꜔꜒ṁ
gacchissa꜔꜒nti; vuddhiyeva, bhikkhave, bhikkhū꜔꜒naṁ pāṭikaṅkhā꜔꜒, no parihā꜔꜒ni.

\begin{english}
  `Enquanto os monges não se submeterem ao poder de qualquer desejo que surja e
  que leve a um futuro nascimento, o seu crescimento pode ser esperado, não o
  seu declínio.'
\end{english}

[6] Yāvakīvañca, bhikkhave, bhikkhū꜔꜒ āraññakesu se꜔꜒nāsanesu sā꜔꜒pekkhā꜔꜒ bhavissa꜔꜒nti;
vuddhiyeva, bhikkhave, bhikkhū꜔꜒naṁ pāṭikaṅkhā꜔꜒, no parihā꜔꜒ni.

\begin{english}
  `Enquanto os monges regozijarem em viver nas florestas,
  o seu crescimento pode ser esperado, não o seu declínio.'
\end{english}

[7] Yāvakīvañca, bhikkhave, bhikkhū꜔꜒ paccattaññeva satiṁ upaṭṭhā꜔꜒pessa꜔꜒nti: Kinti
anāgatā ca pesalā sabrahmacārī āgacche꜔꜒yyuṁ, āgatā ca pesalā sabrahmacārī phā꜔꜒su꜔꜒ṁ
vihareyyun'ti; vuddhiyeva, bhikkhave, bhikkhū꜔꜒naṁ pāṭikaṅkhā꜔꜒, no parihā꜔꜒ni.

\begin{english}
  `Enquanto cada um dos monges mantiver firmemente em mente: ``Se houver
  companheiros bem-comportados, seguidores da vida casta que ainda estão por
  vir, que possam eles vir; e que os companheiros bem-comportados da vida casta
  que vierem possam viver à vontade'', pode-se esperar o seu crescimento, não o
  seu declínio.'
\end{english}

Yāvakīvañca, bhikkhave, ime satta aparihā꜔꜒niyā dhammā bhikkhū꜔꜒su ṭhassa꜔꜒nti, imesu
ca sattasu aparihā꜔꜒niyesu dhammesu bhikkhū꜔꜒ sa꜔꜒ndississa꜔꜒nti; vuddhiyeva, bhikkhave,
bhikkhū꜔꜒naṁ pāṭikaṅkhā꜔꜒, no parihā꜔꜒nī'ti. Idam-avoca bhagavā. Attamanā te bhikkhū꜔꜒
bhagavato bhāsitaṁ abhinandun'ti.

\begin{english}
  `Enquanto os monges permanecerem resolutos nestas sete condições, e enquanto
  estas sete condições persistirem entre os monges, pode-se esperar o seu
  crescimento, não o seu declínio.' Isto é o que o Excelso disse. Satisfeitos,
  os monges deleitaram-se com as palavras do Excelso.
\end{english}
