% vim: foldmethod=marker foldlevel=0 foldtext=FoldText()

\chapter{Dedicação de ofertas}   % {{{1

[Yo so] bha꜕gavā a꜕rahaṃ sammāsambuddho\\
Svākkhā꜓to yena bha꜕gava꜓tā dhammo\\
Supaṭi꜕panno yassa bha꜕gava꜕to sāvaka꜕saṅgho\\
Tam-ma꜓yaṃ bha꜕gavantaṃ sa꜕dhammaṃ sa꜕saṅghaṃ\\
Imehi꜓ sakkārehi꜕ yathārahaṃ āropi꜕tehi a꜕bhi꜓pūja꜕yāma\\
Sādhu꜓ no bhante bha꜕gavā su꜕cira-parinibbu꜕topi\\
Pacchi꜓mā-ja꜕na꜓tānu꜓kampa꜕-mānasā\\
Ime sakkāre dugga꜕ta꜕-paṇṇākāra꜓-bhūte pa꜕ṭiggaṇhātu\\
Amhā꜓kaṃ dīgha꜕rattaṃ hi꜕tāya su꜕khāya

Arahaṃ sammāsambuddho bha꜕gavā\\
Buddhaṃ bha꜕gavantaṃ a꜕bhi꜓vādemi \instr{Curve-se}

[Svākkhā꜓to] bha꜕gava꜓tā dhammo\\
Dhammaṃ namassāmi \instr{Curve-se}

[Supaṭi꜕panno] bha꜕gava꜕to sāvaka꜕saṅgho\\
Sa꜓ṅghaṃ na꜕māmi \instr{Curve-se}

\chapter{Homenagem Preliminar}

\begin{leader}
  [Ha꜓nda mayaṃ buddhassa꜕ bha꜕gavato pubbabhāga-namakā꜕raṃ karomase]
\end{leader}

Namo tassa bha꜕gava꜕to araha꜕to sa꜓mmāsa꜓mbuddha꜕ssa

\instr{Três tempos}

%}}}1
\clearpage

\chapter{Lembrança do Buddha}     % {{{1

\begin{leader}
  [Ha꜓nda mayaṃ buddhābhi꜕tthu꜕tiṃ karomase]
\end{leader}

Yo so tathā꜓ga꜕to a꜕rahaṃ sammāsambuddho\\
Vijjāca꜕raṇa꜓-sampanno Su꜕ga꜕to Loka꜕vi꜓dū\\
Anu꜓tta꜕ro purisa꜕damma-sārathi\\
Satthā deva-ma꜕nussānaṃ Buddho bha꜕gavā

Yo imaṃ lokaṃ sa꜕devakaṃ sa꜕mārakaṃ sa꜕brahma꜕kaṃ\\
Sassa꜓maṇa-brāhmaṇiṃ pa꜕jaṃ sa꜕deva-ma꜕nussa꜓ṃ sa꜕yaṃ a꜕bhiññā sacchika꜕tv꜓ā pa꜕vedesi\\
Yo dhammaṃ dese꜓si ā꜕di꜓-kalyāṇaṃ majjhe꜓-ka꜕lyāṇaṃ\\
pa꜕riyosāṇa-k꜕alyāṇaṃ\\
Sāttha꜓ṃ sa꜕byañjaṇaṃ kevala-pa꜕ripuṇṇaṃ pa꜕risuddhaṃ\\
brahma-ca꜕ri꜓yaṃ pa꜕kāsesi\\

Tam-aha꜓ṃ bha꜕gavantaṃ a꜕bhi꜓pūja꜕yāmi tam-aha꜓ṃ bha꜕gavantaṃ\\
si꜕rasā꜓ na꜕māmi \instr{Curve-se}

\chapter{Lembrança do Dhamma}     % {{{1

\begin{leader}
  [Ha꜓nda mayaṃ dhammābhi꜕tthu꜕tiṃ karomase]
\end{leader}

Yo so svākkhā꜓to bha꜕gava꜓tā dhammo\\
Sa꜓ndiṭṭhi꜕ko a꜕kāli꜕ko ehi꜕passi꜕ko opanayi꜕ko pa꜕cca꜕ttaṃ vedi꜓ta꜕bbo viññūhi

Tam-aha꜓ṃ dhammaṃ a꜕bhi꜓pūja꜕yāmi\\
\vin tam-aha꜓ṃ dhammaṃ si꜕rasā꜓ na꜕māmi \instr{Curve-se}

%}}}1
\enlargethispage{\baselineskip}
\clearpage

\chapter{Lembrança do Sangha}     % {{{1

\begin{leader}
  [Ha꜓nda mayaṃ saṅghābhi꜕tthu꜕tiṃ karomase]
\end{leader}

Yo so supaṭi꜕panno bha꜕gava꜕to sāvaka꜕saṅgho\\
Ujupaṭi꜕panno bha꜕gava꜕to sāvaka꜕saṅgho\\
Ñāyapaṭi꜕panno bha꜕gava꜕to sāvaka꜕saṅgho\\
Sā꜓mīci꜕pa꜕ṭi꜕panno bha꜕gava꜕to sāvaka꜕saṅgho\\
Yadidaṃ cattāri purisa꜕yugāni aṭṭha꜓ purisa꜕pugga꜕lā\\
Esa bha꜕gava꜕to sāvaka꜕saṅgho

Āhu꜕ṇeyyo pāhu꜕ṇeyyo dakkhi꜕ṇeyyo añja꜕li-ka꜕ra꜓ṇīyo\\
Anu꜓tta꜕raṃ puññakkhe꜕ttaṃ lokassa

Tam-aha꜓ṃ saṅghaṃ a꜕bhi꜓pūja꜕yāmi\\
\vin tam-aha꜓ṃ saṅghaṃ si꜕rasā꜓ na꜕māmi \instr{Curve-se}

\chapter{Saudação ao Tripla Gema}% {{{1

\begin{leader}
  [Ha꜓nda mayaṃ ratanattaya-paṇāma-gāthā꜓yo ceva\\
  sa꜓ṃvega-parikittana-pāṭhañca꜕ bhaṇāmase]
\end{leader}

Buddho su꜕suddho ka꜕ruṇāmaha꜓ṇṇavo\\
Yoccanta꜕-suddhabba꜕ra-ñāṇa꜕-loca꜕no\\
Lokassa꜕ pāpūpa꜕ki꜓lesa꜕-ghāta꜕ko\\
Vandāmi꜓ buddhaṃ a꜕ha꜓m-āda꜕rena꜕ taṃ\\
Dhammo pa꜕dīpo vi꜕ya tassa꜕ satthu꜕no\\
Yo magga꜓pākāma꜕ta꜕-bheda꜕-bhinna꜕ko\\
Lokuttaro yo ca꜕ ta꜕dattha꜕-dīpa꜕no\\
Vandāmi꜓ dhammaṃ a꜕ha꜓m-āda꜕rena꜕ taṃ\\
Sa꜓ṅgho su꜕khettābhyati-khe꜕tta-sa꜓ññito\\
Yo diṭṭha꜓santo su꜕ga꜕tānu꜕bodha꜕ko\\
Lolappa꜕hīno a꜕ri꜓yo su꜕medha꜕so\\
Vandāmi꜓ saṅghaṃ a꜕ha꜓m-āda꜕rena꜕ taṃ\\
Iccevam-ekanta꜕bhi꜓pūja꜕-neyya꜕kaṃ vatthuttayaṃ\\
vanda꜕ya꜕tābhi꜕saṅkha꜕taṃ\\
Puññaṃ ma꜕yā yaṃ ma꜕ma꜕ sabbu꜕padda꜕vā\\
\vin mā ho꜓ntu꜕ ve tassa꜕ pa꜕bhāva꜕siddhi꜕yā

Idha tathā꜓ga꜕to loke u꜕ppanno a꜕rahaṃ sammāsambuddho\\
Dhammo ca꜕ desi꜕to niyyāni꜕ko u꜕pa꜕sa꜕miko pa꜕rinibbāni꜕ko\\
\vin sa꜓mbodha꜕gāmī su꜕ga꜕tappa꜕vedi꜕to

Ma꜓yantaṃ dhammaṃ su꜕tvā evaṃ jānāma

Jātipi꜕ dukkhā jarāpi꜕ dukkhā ma꜕raṇampi꜕ dukkhaṃ\\
So꜓ka-pa꜕rideva-dukkha꜕-domanassu꜕pāyāsā꜓pi꜕ dukkhā\\
Appiyehi꜕ sa꜓mpa꜕yogo dukkho\\
Piyehi꜕ vi꜓ppa꜕yogo dukkho\\
Yampiccha꜓ṃ na꜕ labhati tampi꜕ dukkhaṃ\\
Sa꜓ṅkhittena pañcu꜕pādānakkha꜓ndhā dukkhā

Seyya꜕thīdaṃ

\begin{twochants}
Rūpūpādāna꜕kkha꜓ndho & Vedanūpādāna꜕kkha꜓ndho\\
Sa꜓ññūpādāna꜕kkha꜓ndho & Sa꜓ṅkhā꜓rūpādāna꜕kkha꜓ndho\\
Viññāṇūpādāna꜕kkha꜓ndho & \\
\end{twochants}

Yesaṃ pa꜕riññāya\\
Dha꜕ramāno so꜓ bha꜕gavā evaṃ ba꜕hulaṃ sā꜓va꜕ke vi꜕neti\\
Evaṃ bhāgā ca꜕ panassa bha꜕gava꜕to sā꜓va꜕kesu a꜕nusā꜓sa꜕nī ba꜕hulā pa꜕vatta꜕ti

\clearpage

\begin{threechants}
Rūpaṃ a꜕niccaṃ & Vedanā a꜕niccā & Sa꜓ññā a꜕niccā\\
Sa꜓ṅkhā꜓rā a꜕niccā & Viññāṇaṃ a꜕niccaṃ & \\
Rūpaṃ a꜕nattā & Vedanā a꜕nattā & Sa꜓ññā a꜕nattā\\
Sa꜓ṅkhā꜓rā a꜕nattā & Viññāṇaṃ a꜕nattā & \\
\end{threechants}

Sa꜕bbe sa꜓ṅkhā꜓rā a꜕niccā\\
Sa꜕bbe dhammā a꜕nattā'ti

Te ma꜓yaṃ otiṇṇāmha-jāti꜕yā ja꜕rāmaraṇena\\
So꜓kehi꜕ pa꜕ridevehi꜕ dukkhe꜓hi꜕ domanassehi꜕ u꜕pāyāsehi\\
Dukkho꜓tiṇṇā dukkha꜕pa꜕retā\\
Appevanāmi꜓massa꜕ kevalassa꜕ dukkhakkha꜓ndhassa꜕\\
\vin anta꜕kiri꜓yā paññāyethā'ti

\begin{instruction}
  O que se segue é entoado apenas pelos monges e monjas.
\end{instruction}

Ci꜓ra꜓pari꜕nibbutampi꜓ taṃ bha꜕gava꜓ntaṃ uddissa\\
\vin a꜕raha꜓ntaṃ sammāsambuddhaṃ\\
Saddhā a꜕gārasmā anagāri꜓yaṃ pabba꜕ji꜕tā\\
Tasmi꜓ṃ bha꜕gavati brahma-ca꜕ri꜓yaṃ ca꜕rāma\\
Bhikkhū꜓naṃ/Sīladharī꜓naṃ si꜓kkhāsā꜕jīva꜕-samāpannā\\
Taṃ no brahma-ca꜕ri꜓yaṃ imassa꜕ kevalassa꜕ dukkhakkha꜓ndhassa꜕\\
\vin anta꜕kiri꜓yāya sa꜓ṃva꜓tta꜕tu

\clearpage

\begin{instruction}
  Uma versão alternativa da secção precedente, o que pode ser recitado por leigos bem.
\end{instruction}

Ci꜓ra꜓pari꜕nibbutampi꜓ taṃ bha꜕gava꜓ntaṃ saraṇaṃ ga꜕tā\\
Dha꜓mmañca Sa꜓ṅghañca\\
Tassa bha꜕gavato sā꜓sanaṃ yathā꜓sati yathā꜓balaṃ\\
\vin manasika꜕roma a꜕nupaṭipa꜓jjāma\\
Sā꜓ sā꜓ no pa꜕ṭi꜓patti\\
Imassa꜕ kevalassa꜕ dukkhakkha꜓ndhassa꜕\\
\vin anta꜕kiri꜓yāya sa꜓ṃva꜓tta꜕tu

\chapter{Fechando homenagem}       % {{{1

[Arahaṃ] sammāsambuddho bha꜕gavā\\
Buddhaṃ bha꜕gavantaṃ a꜕bhi꜓vādemi \instr{Curve-se}

[Svākkhā꜓to] bha꜕gava꜓tā dhammo\\
Dhammaṃ namassāmi \instr{Curve-se}

[Supaṭi꜕panno] bha꜕gava꜕to sāvaka꜕saṅgho\\
Sa꜓ṅghaṃ na꜕māmi \instr{Curve-se}

%}}}1

% End of morning-chanting.tex
