\chapterTocDelegatePageNumber
\chapter{Setting in Motion the Wheel of Dhamma}

\setTocDelegatedPageNumber
\englishText
\renewcommand{\englishTitle}{Setting in Motion the Wheel of Dhamma}

\begin{leader}
\soloinstr{Solo introduction}

This is the first teaching of the Tathāgata on attaining to unexcelled,
perfect enlightenment.

\ldots{}

\end{leader}

Thus have I heard.

Once when the Blessed One was staying in the deer sanctuary at
Isipatana, near Benares, he spoke to the group of five bhikkhus:

\ldots{}

\chapterTocSubIndentTrue
\chapter{Dhammacakkappavattana Sutta}

\paliText
\renewcommand{\paliTitle}{Dhammacakkappavattana Sutta}

\begin{leader}
\soloinstr{Solo introduction}

\begin{solotwochants}
Anuttaraṃ abhisambodhiṃ & sambujjhitvā tathāgato\\
Pathamaṃ yaṃ adesesi & dhammacakkaṃ anuttaraṃ\\
\ldots{} & \\
\end{solotwochants}
\end{leader}

[Evaṃ me sutaṃ]

Ekaṃ samayaṃ bhagavā bārāṇasiyaṃ viharati isipatane migadāye. Tatra kho
bhagavā pañcavaggiye bhikkhū āmantesi:

\ldots{}

\clearpage

\englishText
\markboth{\englishTitle}{\rightmark}

‘It is just this Noble Eightfold Path, namely:

‘Right View, Right Intention, Right Speech, Right Action, Right
Livelihood, Right Effort, Right Mindfulness, and Right Concentration.

\ldots{}

\clearpage

\paliText
\markboth{\paliTitle}{\rightmark}

Ayam-eva ariyo aṭṭhaṅgiko maggo seyyathīdaṃ:

Sammā-diṭṭhi, sammā-saṅkappo, sammā-vācā, sammā-kammanto, sammā-ājīvo,
sammā-vāyāmo, sammā-sati, sammā-samādhi.

\ldots{}

\chapterTocDelegatePageNumber
\chapter{The Characteristic of Not-Self}

\setTocDelegatedPageNumber
\englishText
\renewcommand{\englishTitle}{The Characteristic of Not-Self}

\begin{leader}
\soloinstr{Solo introduction}

All beings should take pains to understand the characteristic of
not-self, which provides matchless deliverance from self-view and
self-perception, as taught by the supreme Buddha.

\ldots{}

\end{leader}

Thus have I heard.

At one time the Blessed One was dwelling at Benares in the deer park.
There he addressed the group of five bhikkhus:

\ldots{}

\chapterTocSubIndentTrue
\chapter{Anatta-lakkhaṇa Sutta}

\paliText
\renewcommand{\paliTitle}{Anatta-lakkhaṇa Sutta}

\begin{leader}
\soloinstr{Solo introduction}

{\setlength{\tabcolsep}{0.9em}
\begin{solotwochants}
Yantaṃ sattehi dukkhena & ñeyyaṃ anattalakkhaṇaṃ\\
Attavādattasaññāṇaṃ  & sammadeva vimocanaṃ\\
\ldots{} & \\
\end{solotwochants}
}
\end{leader}

[Evaṃ me sutaṃ]

Ekaṃ samayaṃ bhagavā bārāṇasiyaṃ viharati isipatane migadāye. Tatra kho
bhagavā pañcavaggiye bhikkhū āmantesi:

\ldots{}

\clearpage

\englishText
\markboth{\englishTitle}{\rightmark}

‘Perception is not-self. If, bhikkhus, perception were self, perception
would not lead to affliction, and one might be able to say in regard to
perception, “Let my perception be thus, let my perception not be thus.”
But since, bhikkhus, perception is not-self, perception therefore leads
to affliction, and one is not able to say in regard to perception, “Let
my perception be thus, let my perception not be thus.”

\ldots{}

\clearpage

\paliText
\markboth{\paliTitle}{\rightmark}

Saññā anattā, saññā ca hidaṃ bhikkhave attā abhavissa, nayidaṃ saññā
ābādhāya saṃvatteyya, labbhetha ca saññāya, evaṃ me saññā hotu, evaṃ me
saññā mā ahosī ti.

Yasmā ca kho bhikkhave saññā anattā, tasmā, saññā ābādhāya saṃvattati,
na ca labbhati saññāya, evaṃ me saññā hotu, evaṃ me saññā mā ahosī ti.

\ldots{}

\clearpage

\chapterTocDelegatePageNumber
\chapter{The Fire Sermon}

\setTocDelegatedPageNumber
\englishText
\renewcommand{\englishTitle}{The Fire Sermon}

\begin{leader}
\soloinstr{Solo introduction}

With his skill in training the trainable, the All-transcendent Buddha,
lucid speaker, teacher of the highest knowledge,

\ldots{}

\end{leader}

Thus have I heard.

At one time the Blessed One was staying near Gayā at Gayā Head together
with a thousand bhikkhus. There the Blessed One addressed the bhikkhus
thus:

\ldots{}

\chapterTocSubIndentTrue
\chapter{Āditta-pariyāya Sutta}

\paliText
\renewcommand{\paliTitle}{Āditta-pariyāya Sutta}

\begin{leader}
\soloinstr{Solo introduction}

\begin{solotwochants}
Veneyyadamanopāye  & sabbaso pāramiṃ gato\\
Amoghavacano buddho & abhiññāyānusāsako\\
\ldots{} & \\
\end{solotwochants}
\end{leader}

[Evaṃ me sutaṃ]

Ekaṃ samayaṃ bhagavā gayāyaṃ viharati gayāsīse saddhiṃ bhikkhu-sahassena.
Tatra kho bhagavā bhikkhū āmantesi:

\ldots{}

\clearpage

\englishText
\markboth{\englishTitle}{\rightmark}

‘The nose is burning, odours are burning, nose consciousness is burning,
nose contact is burning, the feeling that arises from nose contact,
whether it is pleasant, painful, or neutral, that too is burning. With
what is it burning? I declare that it is burning with the fires of
passion, hatred, and delusion; it is burning with birth, ageing, and
death, with sorrow, lamentation, pain, grief, and despair.

\ldots{}

\clearpage

\paliText
\markboth{\paliTitle}{\rightmark}

Ghānaṃ ādittaṃ, gandhā ādittā, ghānaviññāṇaṃ ādittaṃ, ghānasamphasso
āditto, yampidaṃ ghānasamphassapaccayā uppajjati vedayitaṃ sukhaṃ vā
dukkhaṃ vā adukkhamasukhaṃ vā tam pi ādittaṃ. Kena ādittaṃ. Ādittaṃ
rāgagginā dosagginā mohagginā, ādittaṃ jātiyā jarāmaraṇena sokehi
paridevehi dukkhehi domanassehi upāyāsehi ādittan'ti vadāmi.

\ldots{}

% End of suttas.tex
