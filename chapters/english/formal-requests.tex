% vim: foldmethod=marker foldlevel=0 foldtext=FoldText()

\setlength{\englishIndent}{0pt}

\chapter{Añjali}%{{{1

Chanting and making formal requests is done with the hands in añjali.
This is a gesture of respect, made by placing the palms together
directly in front of the chest, with the fingers aligned and pointing
upwards.

\chapter{Requesting a Dhamma Talk}%{{{1

\begin{instruction}
  After bowing three times, with hands joined in añjali,\\
  recite the following:
\end{instruction}

Brahmā ca꜕ lokādhipa꜕tī sa꜕hampa꜕ti\\
Ka꜕tañja꜕lī a꜕nadhiva꜕raṁ a꜕yāca꜕tha

Santī꜓dha sa꜕ttāppa꜕ra꜕jakkha꜕-jātikā\\
Desetu꜕ dhammaṁ a꜕nu꜕kampi꜕maṁ pa꜕jaṁ

\begin{instruction}
  Bow three times again
\end{instruction}

\begin{english}
The Brahmā god Sahampati, Lord of the world,\\
With palms joined in reverence, requested a favour:

`Beings are here with but little dust in their eyes,\\
Pray, teach the Dhamma out of compassion for them.'
\end{english}

\chapter{Acknowledging the Teaching}%{{{1

\enlargethispage{\baselineskip}

\begin{tabular}{@{} ll @{}}
One person: & Ha꜓nda mayaṁ dhammakathā꜓ya sā꜓dhukā꜕raṁ dadāmase \\
& \hspace*{1em}\tr{Now let us express our approval of this Dhamma Teaching.} \\
Response: & Sādhu, sādhu, sādhu, anu꜓modāmi \\
& \hspace*{1em}\tr{It is well, I appreciate it.} \\
\end{tabular}

\clearpage
\chapter{Requesting Paritta Chanting}%{{{1

\begin{instruction}
  After bowing three times, with hands joined in añjali,\\
  recite the following
\end{instruction}

Vipatti-paṭibāhā꜓ya sabba꜕-sampatti꜕-siddhi꜕yā\\
Sabbadukkha-vināsā꜓ya\\
Parittaṁ brūtha꜕ maṅga꜕laṁ

Vipatti-paṭibāhā꜓ya sabba꜕-sampatti꜕-siddhi꜕yā\\
Sabbabhaya-vināsā꜓ya\\
Parittaṁ brūtha꜕ maṅga꜕laṁ

Vipatti-paṭibāhā꜓ya sabba꜕-sampatti꜕-siddhi꜕yā\\
Sabbaroga-vināsā꜓ya\\
Parittaṁ brūtha꜕ maṅga꜕laṁ

\begin{instruction}
  Bow three times
\end{instruction}

\begin{english}
For warding off misfortune, for the arising of good fortune,\\
For the dispelling of all dukkha,\\
May you chant a blessing and protection.

For warding off misfortune, for the arising of good fortune,\\
For the dispelling of all fear,\\
May you chant a blessing and protection.

For warding off misfortune, for the arising of good fortune,\\
For the dispelling of all sickness,\\
May you chant a blessing and protection.
\end{english}

\setlength{\englishIndent}{\leaderIndent}

\clearpage
\chapter[Three Refuges \& the Five Precepts]{Requesting the Three Refuges\newline \& the Five Precepts}%{{{1

\begin{instruction}
  After bowing three times, with hands joined in añjali,\\
  recite the appropriate request.
\end{instruction}

\section{For a group from a monk}

\begin{twochants}
Mayaṁ bhante tisaraṇena sa꜕ha & pañca sī꜓lāni yā꜕cāma\\
Dutiyampi mayaṁ bhante tisaraṇena sa꜕ha & pañca sī꜓lāni yā꜕cāma\\
Tatiyampi mayaṁ bhante tisaraṇena sa꜕ha & pañca sī꜓lāni yā꜕cāma\\
\end{twochants}

\section{For oneself from a monk}

\begin{twochants}
Ahaṁ bhante tisaraṇena sa꜕ha & pañca sī꜓lāni yā꜕cāmi\\
Dutiyampi ahaṁ bhante tisaraṇena sa꜕ha & pañca sī꜓lāni yā꜕cāmi\\
Tatiyampi ahaṁ bhante tisaraṇena sa꜕ha & pañca sī꜓lāni yā꜕cāmi
\end{twochants}

\section{For a group from a nun}

\begin{twochants}
Mayaṁ ayye tisaraṇena sa꜕ha & pañca sī꜓lāni yā꜕cāma\\
Dutiyampi mayaṁ ayye tisaraṇena sa꜕ha & pañca sī꜓lāni yā꜕cāma\\
Tatiyampi mayaṁ ayye tisaraṇena sa꜕ha & pañca sī꜓lāni yā꜕cāma\\
\end{twochants}

\section{For oneself from a nun}

\begin{twochants}
Ahaṁ ayye tisaraṇena sa꜕ha & pañca sī꜓lāni yā꜕cāmi\\
Dutiyampi ahaṁ ayye tisaraṇena sa꜕ha & pañca sī꜓lāni yā꜕cāmi\\
Tatiyampi ahaṁ ayye tisaraṇena sa꜕ha & pañca sī꜓lāni yā꜕cāmi\\
\end{twochants}

\section{For a group from a layperson}

\begin{twochants}
Mayaṁ mitta tisaraṇena sa꜕ha & pañca sī꜓lāni yā꜕cāma\\
Dutiyampi mayaṁ mitta tisaraṇena sa꜕ha & pañca sī꜓lāni yā꜕cāma\\
Tatiyampi mayaṁ mitta tisaraṇena sa꜕ha & pañca sī꜓lāni yā꜕cāma\\
\end{twochants}

\section{For oneself from a layperson}

\begin{twochants}
Ahaṁ mitta tisaraṇena sa꜕ha & pañca sī꜓lāni yā꜕cāmi\\
Dutiyampi ahaṁ mitta tisaraṇena sa꜕ha & pañca sī꜓lāni yā꜕cāmi\\
Tatiyampi ahaṁ mitta tisaraṇena sa꜕ha & pañca sī꜓lāni yā꜕cāmi\\
\end{twochants}

\section{Translation}

\begin{english}
  We/I, Venerable Sir/Sister/Friend,\\
  request the Three Refuges and the Five Precepts.

  For the second time,\\
  we/I, Venerable Sir/Sister/Friend,\\
  request the Three Refuges and the Five Precepts.

  For the third time,\\
  we/I, Venerable Sir/Sister/Friend,\\
  request the Three Refuges and the Five Precepts.
\end{english}

\clearpage
\chapter{Taking the Three Refuges}%{{{1

\begin{instruction}
  Repeat, after the leader has chanted the first three lines
\end{instruction}

Namo tassa bhagavato arahato sammāsambuddhassa\\
Namo tassa bhagavato arahato sammāsambuddhassa\\
Namo tassa bhagavato arahato sammāsambuddhassa

\begin{english}
  Homage to the Blessed, Noble, and Perfectly Enlightened One.\\
  Homage to the Blessed, Noble, and Perfectly Enlightened One.\\
  Homage to the Blessed, Noble, and Perfectly Enlightened One.
\end{english}

Buddhaṁ saraṇaṁ gacchāmi\\
Dhammaṁ saraṇaṁ gacchāmi\\
Saṅghaṁ saraṇaṁ gacchāmi

\begin{english}
  To the Buddha I go for refuge.\\
  To the Dhamma I go for refuge.\\
  To the Saṅgha I go for refuge.
\end{english}

Dutiyampi buddhaṁ saraṇaṁ gacchāmi\\
Dutiyampi dhammaṁ saraṇaṁ gacchāmi\\
Dutiyampi saṅghaṁ saraṇaṁ gacchāmi

\begin{english}
  For the second time, to the Buddha I go for refuge.\\
  For the second time, to the Dhamma I go for refuge.\\
  For the second time, to the Saṅgha I go for refuge.
\end{english}

Tatiyampi buddhaṁ saraṇaṁ gacchāmi\\
Tatiyampi dhammaṁ saraṇaṁ gacchāmi\\
Tatiyampi saṅghaṁ saraṇaṁ gacchāmi

\clearpage

\begin{english}
  For the third time, to the Buddha I go for refuge.\\
  For the third time, to the Dhamma I go for refuge.\\
  For the third time, to the Saṅgha I go for refuge.
\end{english}

\begin{instruction}
  Leader:
\end{instruction}

[Tisaraṇa-gamanaṁ niṭṭhitaṁ]

\begin{english}
  This completes the going to the Three Refuges.
\end{english}

\begin{instruction}
  Response:
\end{instruction}

Āma bhante / Āma ayye / Āma mitta

\begin{english}
  Yes, Venerable Sir/Sister/Friend.
\end{english}

\chapter{The Five Precepts}%{{{1

\begin{instruction}
  Repeat each precept after the leader
\end{instruction}

\begin{precept}
  \setcounter{enumi}{0}
  \item Pāṇātipātā vera꜓maṇī sikkhā꜓padaṁ sa꜓mādi꜕yāmi
\end{precept}

\begin{english}
  I undertake the precept to refrain from taking the life of any living~creature.
\end{english}

\begin{precept}
  \setcounter{enumi}{1}
  \item Adinnādānā vera꜓maṇī sikkhā꜓padaṁ sa꜓mādi꜕yāmi
\end{precept}

\begin{english}
  I undertake the precept to refrain from taking that which is not given.
\end{english}

\begin{precept}
  \setcounter{enumi}{2}
  \item Kāmesu micchā꜓cārā vera꜓maṇī sikkhā꜓padaṁ sa꜓mādi꜕yāmi
\end{precept}

\begin{english}
  I undertake the precept to refrain from sexual misconduct.
\end{english}

\clearpage

\begin{precept}
  \setcounter{enumi}{3}
  \item Musā꜓vādā vera꜓maṇī sikkhā꜓padaṁ sa꜓mādi꜕yāmi
\end{precept}

\begin{english}
  I undertake the precept to refrain from lying.
\end{english}

\begin{precept}
  \setcounter{enumi}{4}
  \item Surāmeraya-majja-pamādaṭṭhā꜓nā vera꜓maṇī sikkhā꜓padaṁ sa꜓mādi꜕yāmi
\end{precept}

\begin{english}
  I undertake the precept to refrain from consuming intoxicating drink and drugs which lead to carelessness.
\end{english}

\begin{instruction}
  Leader:
\end{instruction}

[Imāni pañca sikkhā꜓padāni\\
Sī꜓lena suga꜕tiṁ yanti\\
Sī꜓lena bhoga꜕sa꜓mpadā\\
Sī꜓lena nibbu꜕tiṁ yanti\\
Tasmā꜓ sī꜓laṁ viso꜓dhaye]

\begin{english}
  These are the Five Precepts;\\
  virtue is the source of happiness,\\
  virtue is the source of true wealth,\\
  virtue is the source of peacefulness ---\\
  Therefore let virtue be purified.
\end{english}

\begin{instruction}
  Response:
\end{instruction}

Sādhu, sādhu, sādhu

\begin{instruction}
  Bow three times
\end{instruction}

\clearpage
\chapter[Three Refuges \& the Eight Precepts]{Requesting the Three Refuges\newline \& the Eight Precepts}% {{{1

\begin{instruction}
  After bowing three times, with hands joined in añjali,\\
  recite the appropriate request.
\end{instruction}

\section{For a group from a monk}

\begin{twochants}
Mayaṁ bhante tisaraṇena sa꜕ha & aṭṭha sī꜓lāni yā꜕cāma\\
Dutiyampi mayaṁ bhante tisaraṇena sa꜕ha & aṭṭha sī꜓lāni yā꜕cāma\\
Tatiyampi mayaṁ bhante tisaraṇena sa꜕ha & aṭṭha sī꜓lāni yā꜕cāma\\
\end{twochants}

\section{For oneself from a monk}

\begin{twochants}
Ahaṁ bhante tisaraṇena sa꜕ha & aṭṭha sī꜓lāni yā꜕cāmi\\
Dutiyampi ahaṁ bhante tisaraṇena sa꜕ha & aṭṭha sī꜓lāni yā꜕cāmi\\
Tatiyampi ahaṁ bhante tisaraṇena sa꜕ha & aṭṭha sī꜓lāni yā꜕cāmi
\end{twochants}

\section{For a group from a nun}

\begin{twochants}
Mayaṁ ayye tisaraṇena sa꜕ha & aṭṭha sī꜓lāni yā꜕cāma\\
Dutiyampi mayaṁ ayye tisaraṇena sa꜕ha & aṭṭha sī꜓lāni yā꜕cāma\\
Tatiyampi mayaṁ ayye tisaraṇena sa꜕ha & aṭṭha sī꜓lāni yā꜕cāma\\
\end{twochants}

\section{For oneself from a nun}

\begin{twochants}
Ahaṁ ayye tisaraṇena sa꜕ha & aṭṭha sī꜓lāni yā꜕cāmi\\
Dutiyampi ahaṁ ayye tisaraṇena sa꜕ha & aṭṭha sī꜓lāni yā꜕cāmi\\
Tatiyampi ahaṁ ayye tisaraṇena sa꜕ha & aṭṭha sī꜓lāni yā꜕cāmi\\
\end{twochants}

\section{For a group from a layperson}

\begin{twochants}
Mayaṁ mitta tisaraṇena sa꜕ha & aṭṭha sī꜓lāni yā꜕cāma\\
Dutiyampi mayaṁ mitta tisaraṇena sa꜕ha & aṭṭha sī꜓lāni yā꜕cāma\\
Tatiyampi mayaṁ mitta tisaraṇena sa꜕ha & aṭṭha sī꜓lāni yā꜕cāma\\
\end{twochants}

\section{For oneself from a layperson}

\begin{twochants}
Ahaṁ mitta tisaraṇena sa꜕ha & aṭṭha sī꜓lāni yā꜕cāmi\\
Dutiyampi ahaṁ mitta tisaraṇena sa꜕ha & aṭṭha sī꜓lāni yā꜕cāmi\\
Tatiyampi ahaṁ mitta tisaraṇena sa꜕ha & aṭṭha sī꜓lāni yā꜕cāmi\\
\end{twochants}

\section{Translation}

\begin{english}
  We/I, Venerable Sir/Sister/Friend,\\
  request the Three Refuges and the Eight Precepts.

  For the second time,\\
  We/I, Venerable Sir/Sister/Friend,\\
  request the Three Refuges and the Eight Precepts.

  For the third time,\\
  We/I, Venerable Sir/Sister/Friend,\\
  request the Three Refuges and the Eight Precepts.
\end{english}

\clearpage
\chapter{Taking the Three Refuges}%{{{1

\begin{instruction}
  Repeat, after the leader has chanted the first three lines
\end{instruction}

Namo tassa bhagavato arahato sammāsambuddhassa\\
Namo tassa bhagavato arahato sammāsambuddhassa\\
Namo tassa bhagavato arahato sammāsambuddhassa

\begin{english}
  Homage to the Blessed, Noble, and Perfectly Enlightened One.\\
  Homage to the Blessed, Noble, and Perfectly Enlightened One.\\
  Homage to the Blessed, Noble, and Perfectly Enlightened One.
\end{english}

Buddhaṁ saraṇaṁ gacchāmi\\
Dhammaṁ saraṇaṁ gacchāmi\\
Saṅghaṁ saraṇaṁ gacchāmi

\begin{english}
  To the Buddha I go for refuge.\\
  To the Dhamma I go for refuge.\\
  To the Saṅgha I go for refuge.
\end{english}

Dutiyampi buddhaṁ saraṇaṁ gacchāmi\\
Dutiyampi dhammaṁ saraṇaṁ gacchāmi\\
Dutiyampi saṅghaṁ saraṇaṁ gacchāmi

\begin{english}
  For the second time, to the Buddha I go for refuge.\\
  For the second time, to the Dhamma I go for refuge.\\
  For the second time, to the Saṅgha I go for refuge.
\end{english}

Tatiyampi buddhaṁ saraṇaṁ gacchāmi\\
Tatiyampi dhammaṁ saraṇaṁ gacchāmi\\
Tatiyampi saṅghaṁ saraṇaṁ gacchāmi

\clearpage

\begin{english}
  For the third time, to the Buddha I go for refuge.\\
  For the third time, to the Dhamma I go for refuge.\\
  For the third time, to the Saṅgha I go for refuge.
\end{english}

\begin{instruction}
  Leader:
\end{instruction}

[Tisaraṇa-gamanaṁ niṭṭhitaṁ]

\begin{english}
  This completes the going to the Three Refuges.
\end{english}

\begin{instruction}
  Response:
\end{instruction}

Āma bhante / Āma ayye / Āma mitta

\begin{english}
  Yes, Venerable Sir/Sister/Friend.
\end{english}

\chapter{The Eight Precepts}%{{{1

\begin{instruction}
  Repeat each precept after the leader
\end{instruction}

\begin{precept}
  \setcounter{enumi}{0}
  \item Pāṇātipātā vera꜓maṇī sikkhā꜓padaṁ sa꜓mādi꜕yāmi
\end{precept}

\begin{english}
  I undertake the precept to refrain from taking the life of any living~creature.
\end{english}

\begin{precept}
  \setcounter{enumi}{1}
  \item Adinnādānā vera꜓maṇī sikkhā꜓padaṁ sa꜓mādi꜕yāmi
\end{precept}

\begin{english}
  I undertake the precept to refrain from taking that which is not given.
\end{english}

\begin{precept}
  \setcounter{enumi}{2}
  \item Abrahmacariyā vera꜓maṇī sikkhā꜓padaṁ sa꜓mādi꜕yāmi
\end{precept}

\begin{english}
  I undertake the precept to refrain from any intentional sexual activity.
\end{english}

\clearpage

\begin{precept}
  \setcounter{enumi}{3}
  \item Musā꜓vādā vera꜓maṇī sikkhā꜓padaṁ sa꜓mādi꜕yāmi
\end{precept}

\begin{english}
  I undertake the precept to refrain from lying.
\end{english}

\begin{precept}
  \setcounter{enumi}{4}
  \item Surāmeraya-majja-pamādaṭṭhā꜓nā vera꜓maṇī sikkhā꜓padaṁ sa꜓mādi꜕yāmi
\end{precept}

\begin{english}
  I undertake the precept to refrain from consuming intoxicating drink and drugs which lead to carelessness.
\end{english}

\begin{precept}
  \setcounter{enumi}{5}
  \item Vikālabhojanā vera꜓maṇī sikkhā꜓padaṁ sa꜓mādi꜕yāmi.
\end{precept}

\begin{english}
  I undertake the precept to refrain from eating at inappropriate times.
\end{english}

\begin{precept}
  \setcounter{enumi}{6}
  \item Nacca-gīta-vādita-visūkada꜓ssanā mālā-gandha-vilepana-dhāraṇa-maṇḍana-vibhūsanaṭṭhā꜓nā vera꜓maṇī sikkhā꜓padaṁ sa꜓mādi꜕yāmi.
\end{precept}

\begin{english}
  I undertake the precept to refrain from entertainment, beautification, and~adornment.
\end{english}

\begin{precept}
  \setcounter{enumi}{7}
  \item Uccāsayana-mahā꜓sayanā vera꜓maṇī sikkhā꜓padaṁ sa꜓mādi꜕yāmi.
\end{precept}

\begin{english}
  I undertake the precept to refrain from lying on a high or luxurious sleeping place.
\end{english}

\begin{instruction}
  Leader:
\end{instruction}

[Imāni aṭṭha sikkhā꜓padāni sa꜓mādi꜕yāmi]

\begin{instruction}
  Response:
\end{instruction}

\clearpage

Imāni aṭṭha sikkhā꜓padāni sa꜓mādi꜕yāmi\\
Imāni aṭṭha sikkhā꜓padāni sa꜓mādi꜕yāmi\\
Imāni aṭṭha sikkhā꜓padāni sa꜓mādi꜕yāmi

\begin{english}
  I undertake these Eight Precepts.\\
  I undertake these Eight Precepts.\\
  I undertake these Eight Precepts.
\end{english}

\begin{instruction}
  Leader:
\end{instruction}

[Imāni aṭṭha sikkhā꜓padāni\\
Sī꜓lena suga꜕tiṁ yanti\\
Sī꜓lena bhoga꜕sa꜓mpadā\\
Sī꜓lena nibbu꜕tiṁ yanti\\
Tasmā꜓ sī꜓laṁ viso꜓dhaye]

\begin{english}
  These are the Eight Precepts;\\
  virtue is the source of happiness,\\
  virtue is the source of true wealth,\\
  virtue is the source of peacefulness ---\\
  Therefore let virtue be purified.
\end{english}

\begin{instruction}
  Response:
\end{instruction}

Sādhu, sādhu, sādhu.

\begin{instruction}
  Bow three times
\end{instruction}

