% vim: foldmethod=marker foldlevel=0 foldtext=FoldText()

\setlength{\englishIndent}{0pt}

\chapter{Añjali}

% TODO: add illustration

\begin{english}
Chanting and making formal requests is done with the hands in añjali.
This is a gesture of respect, made by placing the palms together
directly in front of the chest, with the fingers aligned and pointing
upwards.
\end{english}

\chapter{Requesting a Dhamma talk}

\begin{instruction}
  After bowing three times, with hands joined in añjali,\\
  recite the following:
\end{instruction}

Brahmā ca꜕ lokādhipa꜕tī sa꜕hampa꜕ti\\
Ka꜕tañja꜕lī a꜕ndhiva꜕raṃ a꜕yāca꜕tha

Santī꜓dha sa꜕ttāppa꜕ra꜕jakkha꜕-jātikā\\
Desetu꜕ dhammaṃ a꜕nu꜕kampi꜕maṃ pa꜕jaṃ

\begin{instruction}
  Bow three times again
\end{instruction}

\begin{english}
The Brahma god Sahampati, Lord of the world,\\
With palms joined in reverence, requested a favour:

``Beings are here with but little dust in their eyes,\\
Pray, teach the Dhamma out of compassion for them.''
\end{english}

\chapter{Acknowledging the teaching}

\begin{tabular}{@{} ll @{}}
One person: & Ha꜓nda mayaṃ dhammakathā꜓ya sā꜓dhukā꜕raṃ dadāmase. \\
& \hspace*{1em}\tr{Now let us express our approval of this Dhamma Teaching.} \\
Response: & Sādhu, sādhu, sādhu, anu꜓modāmi. \\
& \hspace*{1em}\tr{It is well, I appreciate it.} \\
\end{tabular}

\clearpage
\chapter{Requesting paritta chanting}

\begin{instruction}
  After bowing three times, with hands joined in añjali, recite the following
\end{instruction}

Vipatti-paṭibāhā꜓ya sabba꜕-sampatti꜕-siddhi꜕yā\\
Sabbadukkha-vināsā꜓ya\\
Parittaṃ brūtha꜕ maṅga꜕laṃ

Vipatti-paṭibāhā꜓ya sabba꜕-sampatti꜕-siddhi꜕yā\\
Sabbabhaya-vināsā꜓ya\\
Parittaṃ brūtha꜕ maṅga꜕laṃ

Vipatti-paṭibāhā꜓ya sabba꜕-sampatti꜕-siddhi꜕yā\\
Sabbaroga-vināsā꜓ya\\
Parittaṃ brūtha꜕ maṅga꜕laṃ

\begin{instruction}
  Bow three times
\end{instruction}

\begin{english}
For warding off misfortune, for the arising of good fortune,\\
For the dispelling of all dukkha,\\
May you chant a blessing and protection.

For warding off misfortune, for the arising of good fortune,\\
For the dispelling of all fear,\\
May you chant a blessing and protection.

For warding off misfortune, for the arising of good fortune,\\
For the dispelling of all sickness,\\
May you chant a blessing and protection.
\end{english}

\setlength{\englishIndent}{\leaderIndent}

\clearpage
\chapter{Requesting the Three Refuges \& the Five Precepts}

\begin{instruction}
  After bowing three times, with hands joined in añjali, recite
\end{instruction}

Mayaṃ/Ahaṃ bhante/ayye* tisaraṇena sa꜕ha\\
pañca sī꜓lāni yā꜕cāma/yā꜕cāmi

Dutiyampi mayaṃ/ahaṃ bhante/ayye* tisaraṇena sa꜕ha\\
pañca sī꜓lāni yā꜕cāma/yā꜕cāmi

Tatiyampi mayaṃ/ahaṃ bhante/ayye* tisaraṇena sa꜕ha\\
pañca sī꜓lāni yā꜕cāma/yā꜕cāmi

\begin{english}
We/I, Venerable Sir/Sister**, request the Three Refuges and the Five Precepts.

For the second time, we/I, Venerable Sir/Sister**, request the Three Refuges and the Five Precepts.

For the third time, we/I, Venerable Sir/Sister**, request the Three Refuges and the Five Precepts.
\end{english}

\chapter{Taking the Three Refuges}

\begin{instruction}
  Repeat, after the leader has chanted the first three lines
\end{instruction}

Namo tassa bhagavato arahato sammāsambuddhassa\\
Namo tassa bhagavato arahato sammāsambuddhassa\\
Namo tassa bhagavato arahato sammāsambuddhassa

\begin{english}
  Homage to the Blessed, Noble, and Perfectly Enlightened One.\\
  Homage to the Blessed, Noble, and Perfectly Enlightened One.\\
  Homage to the Blessed, Noble, and Perfectly Enlightened One.
\end{english}

Buddhaṃ saraṇaṃ gacchāmi\\
Dhammaṃ saraṇaṃ gacchāmi\\
Saṅghaṃ saraṇaṃ gacchāmi

\begin{english}
  To the Buddha I go for refuge.\\
  To the Dhamma I go for refuge.\\
  To the Sangha I go for refuge.
\end{english}

Dutiyampi Buddhaṃ saraṇaṃ gacchāmi\\
Dutiyampi Dhammaṃ saraṇaṃ gacchāmi\\
Dutiyampi Saṅghaṃ saraṇaṃ gacchāmi

\begin{english}
  For the second time, to the Buddha I go for refuge.\\
  For the second time, to the Dhamma I go for refuge.\\
  For the second time, to the Sangha I go for refuge.
\end{english}

Tatiyampi Buddhaṃ saraṇaṃ gacchāmi\\
Tatiyampi Dhammaṃ saraṇaṃ gacchāmi\\
Tatiyampi Saṅghaṃ saraṇaṃ gacchāmi

\begin{english}
  For the third time, to the Buddha I go for refuge.\\
  For the third time, to the Dhamma I go for refuge.\\
  For the third time, to the Sangha I go for refuge.
\end{english}

\begin{leader}
  Leader:\\
  [Tisaraṇa-gamanaṃ niṭṭhitaṃ]
\end{leader}

\begin{english}
  This completes the going to the Three Refuges.
\end{english}

Response: Āma bhante/ayye*

\begin{english}
  Yes, Venerable Sir/Sister**.
\end{english}

\chapter{The Five Precepts}

\begin{instruction}
  To undertake the precepts, repeat each precept after the leader
\end{instruction}

1. Pāṇātipātā vera꜓maṇī sikkhā꜓padaṃ sa꜓mādi꜕yāmi.

I undertake the precept to refrain from taking the life of any \\
living creature.

2. Adinnādānā vera꜓maṇī sikkhā꜓padaṃ sa꜓mādi꜕yāmi.

I undertake the precept to refrain from taking that which is not given.

3. Kāmesu micchā꜓cārā vera꜓maṇī sikkhā꜓padaṃ sa꜓mādi꜕yāmi.

I undertake the precept to refrain from sexual misconduct.

4. Musā꜓vādā vera꜓maṇī sikkhā꜓padaṃ sa꜓mādi꜕yāmi.

I undertake the precept to refrain from lying.

5. Surāmeraya-majja-pamādaṭṭhā꜓nā vera꜓maṇī sikkhā꜓padaṃ

sa꜓mādi꜕yāmi.

I undertake the precept to refrain from consuming intoxicating \\
drink and drugs which lead to carelessness.

Leader: Imāni pañca sikkhā꜓padāni

Sī꜓lena suga꜕tiṃ yanti

Sī꜓lena bhoga꜕sa꜓mpadā

Sī꜓lena nibbu꜕tiṃ yanti

Tasmā꜓ sī꜓laṃ viso꜓dhaye

These are the Five Precepts;

virtue is the source of happiness,

virtue is the source of true wealth,

virtue is the source of peacefulness -

Therefore let virtue be purified.

Response: Sādhu, sādhu, sādhu

[ Bow three times ]

REQUESTING THE THREE REFUGES

AND THE EIGHT PRECEPTS

[ After bowing three times, with hands joined in añjali, recite ]

Mayaṃ/Ahaṃ bhante/ayye* tisaraṇena sa꜕ha \\
aṭṭha sī꜓lāni yā꜕cāma/yā꜕cāmi

Dutiyampi mayaṃ/ahaṃ bhante/ayye* tisaraṇena sa꜕ha \\
aṭṭha sī꜓lāni yā꜕cāma/yā꜕cāmi

Tatiyampi mayaṃ/ahaṃ bhante/ayye* tisaraṇena sa꜕ha \\
aṭṭha sī꜓lāni yā꜕cāma/yā꜕cāmi

We/I, Venerable Sir/Sister**, request the Three Refuges and \\
the Eight Precepts.

For the second time, we/I, Venerable Sir/Sister**, request the \\
Three Refuges and the Eight Precepts.

For the third time, we/I, Venerable Sir/Sister**, request the \\
Three Refuges and the Eight Precepts.

TAKING THE THREE REFUGES

[ Repeat, after the leader has chanted the first three lines ]

Namo tassa bhagavato arahato sammāsambuddhassa

Namo tassa bhagavato arahato sammāsambuddhassa

Namo tassa bhagavato arahato sammāsambuddhassa

Homage to the Blessed, Noble, and Perfectly Enlightened One.

Homage to the Blessed, Noble, and Perfectly Enlightened One.

Homage to the Blessed, Noble, and Perfectly Enlightened One.

Buddhaṃ saraṇaṃ gacchāmi

Dhammaṃ saraṇaṃ gacchāmi

Saṅghaṃ saraṇaṃ gacchāmi

To the Buddha I go for refuge.

To the Dhamma I go for refuge.

To the Sangha I go for refuge.

Dutiyampi Buddhaṃ saraṇaṃ gacchāmi

Dutiyampi Dhammaṃ saraṇaṃ gacchāmi

Dutiyampi Saṅghaṃ saraṇaṃ gacchāmi

For the second time, to the Buddha I go for refuge.

For the second time, to the Dhamma I go for refuge.

For the second time, to the Sangha I go for refuge.

Tatiyampi Buddhaṃ saraṇaṃ gacchāmi

Tatiyampi Dhammaṃ saraṇaṃ gacchāmi

Tatiyampi Saṅghaṃ saraṇaṃ gacchāmi

For the third time, to the Buddha I go for refuge.

For the third time, to the Dhamma I go for refuge.

For the third time, to the Sangha I go for refuge.

Leader: Tisaraṇa-gamanaṃ niṭṭhitaṃ

This completes the going to the Three Refuges.

Response: Ama bhante/ayye*

Yes, Venerable Sir/Sister**

THE EIGHT PRECEPTS

[ To undertake the precepts, repeat each precept after the leader ]

1. Pāṇātipātā vera꜓maṇī sikkhā꜓padaṃ sa꜓mādi꜕yāmi.

I undertake the precept to refrain from taking the life of any living creature.

2. Adinnādānā vera꜓maṇī sikkhā꜓padaṃ sa꜓mādi꜕yāmi.

I undertake the precept to refrain from taking that which is not given.

3. Abrahmacariyā vera꜓maṇī sikkhā꜓padaṃ sa꜓mādi꜕yāmi.

I undertake the precept to refrain from any intentional sexual activity.

4. Musā꜓vādā vera꜓maṇī sikkhā꜓padaṃ sa꜓mādi꜕yāmi.

I undertake the precept to refrain from lying.

5. Surāmeraya-majja-pamādaṭṭhā꜓nā vera꜓maṇī sikkhā꜓padaṃ

sa꜓mādi꜕yāmi.

I undertake the precept to refrain from consuming intoxicating \\
drink and drugs which lead to carelessness.

6. Vikālabhojanā vera꜓maṇī sikkhā꜓padaṃ sa꜓mādi꜕yāmi.

I undertake the precept to refrain from eating at inappropriate times.

7. Nacca-gīta-vādita-visūkada꜓ssanā mālā-gandha-vilepana

dhāraṇa maṇḍana-vibhūsanaṭṭhā꜓nā vera꜓maṇī sikkhā꜓padaṃ

sa꜓mādi꜕yāmi.

I undertake the precept to refrain from entertainment, \\
beautification, and adornment.

8.Uccāsayana-mahā꜓sayanā vera꜓maṇī sikkhā꜓padaṃ sa꜓mādi꜕yāmi.

I undertake the precept to refrain from lying on a high or \\
luxurious sleeping place.

Leader: Imāni aṭṭha sikkhā꜓padāni sa꜓mādi꜕yāmi

Response: Imāni aṭṭha sikkhā꜓padāni sa꜓mādi꜕yāmi

Imāni aṭṭha sikkhā꜓padāni sa꜓mādi꜕yāmi

Imāni aṭṭha sikkhā꜓padāni sa꜓mādi꜕yāmi

I undertake these Eight Precepts.

I undertake these Eight Precepts.

I undertake these Eight Precepts.

Leader: Imāni aṭṭha sikkhā꜓padāni

Sī꜓lena suga꜕tiṃ yanti

Sī꜓lena bhoga꜕sa꜓mpadā

Sī꜓lena nibbu꜕tiṃ yanti

Tasmā꜓ sī꜓laṃ viso꜓dhaye

These are the Eight Precepts;

virtue is the source of happiness,

virtue is the source of true wealth,

virtue is the source of peacefulness -

Therefore let virtue be purified

Response: Sādhu, sādhu, sādhu

[ Bow three times ]

%PART 4
%
%FORMAL REQUESTS
%
%*When requesting in Pāli from a layperson, ``mitta'' replaces ``bhante'' or ``ayye''.
%
%**When requesting in English from a layperson, ``Friend'' replaces ``Venerable Sir'' or ``Sister''.
%
%*When requesting in Pāli from a layperson, ``mitta'' replaces ``bhante'' or ``ayye''.
%
%**When requesting in English from a layperson, ``Friend'' replaces ``Venerable Sir'' or ``Sister''.
%
%*When requesting in Pāli from a layperson, ``mitta'' replaces ``bhante'' or ``ayye''.
%
%**When requesting in English from a layperson, ``Friend'' replaces ``Venerable Sir'' or ``Sister''.
%
%*When requesting in Pāli from a layperson, ``mitta'' replaces ``bhante'' or ``ayye''.
%
%**When requesting in English from a layperson, ``Friend'' replaces ``Venerable Sir'' or ``Sister''.
