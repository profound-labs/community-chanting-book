\newcommand\englishPage{%
  \clearpage%
  \englishText%
  %\markboth{\englishTitle}{\rightmark}%
}

\newcommand\paliPage{%
  \clearpage%
  \paliText%
  %\markboth{\paliTitle}{\rightmark}%
}

\renewcommand{\englishTitle}{The Foundations of Mindfulness}
\renewcommand{\paliTitle}{Mahāsatipaṭṭhāna Sutta}

\englishPage
\chapter{Introduction}

Thus have I heard.

On one occasion the Blessed One was in the Kuru country where there was a town
of the Kurus named Kammāsadhamma. There the Blessed One addressed the bhikkhus
thus: “Bhikkhus.” “Bhante,” the bhikkhus replied to the Blessed One. The Blessed
One said this:

“Bhikkhus, this is the one-way path for the purification of beings, for the
surmounting of sorrow and lamentation, for the passing away of pain and
dejection, for the attainment of the true way, for the realisation of Nibbāna,
namely, the four foundations of mindfulness. What are the four?

Here, bhikkhus, a bhikkhu dwells contemplating the body in the body, ardent,
clearly comprehending, and mindful, having subdued longing and dejection in
regard to the world. He dwells contemplating feelings in feelings, ardent,
clearly comprehending, and mindful, having subdued longing and dejection in
regard to the world. He dwells contemplating mind in mind, ardent, clearly
comprehending, and mindful, having subdued longing and dejection in regard to
the world. He dwells contemplating phenomena in phenomena, ardent, clearly
comprehending, and mindful, having subdued longing and dejection in regard to
the world.

\instr{The Introduction is finished.}

\paliPage
\chapter*{Uddeso}

[Evaṃ me sutaṃ]

Ekaṃ samayaṃ bhagavā kurūsu viharati kammāsadhammaṃ nāma kurūnaṃ nigamo. Tatra
kho bhagavā bhikkhū āmantesi: “bhikkhavo”ti. “Bhaddante”ti te bhikkhū bhagavato
paccassosuṃ. Bhagavā etadavoca:

“Ekāyano ayaṃ, bhikkhave, maggo sattānaṃ visuddhiyā, soka-paridevānaṃ
samatikkamāya dukkha-domanassānaṃ atthaṅgamāya ñāyassa adhigamāya nibbānassa
sacchikiriyāya, yadidaṃ cattāro satipaṭṭhānā.

Katame cattāro? Idha, bhikkhave, bhikkhu kāye kāyānupassī viharati ātāpī
sampajāno satimā vineyya loke abhijjhā-domanassaṃ, vedanāsu vedanānupassī
viharati ātāpī sampajāno satimā vineyya loke abhijjhā-domanassaṃ, citte
cittānupassī viharati ātāpī sampajāno satimā vineyya loke abhijjhā-domanassaṃ,
dhammesu dhammānupassī viharati ātāpī sampajāno satimā vineyya loke
abhijjhā-domanassaṃ.

\instr{Uddeso niṭṭhito.}

\englishPage
\chapter{Contemplation of the Body}

\section{Mindfulness of Breathing}

And how, bhikkhus, does a bhikkhu dwell contemplating\\
the body in the body?

Here, bhikkhus, a bhikkhu, gone to the forest, to the foot of a tree, or to an
empty hut, sits down; having folded his legs crosswise, straightened his body,
and established mindfulness in front of him.

Just mindful he breathes in, mindful he breathes out.\\
Breathing in long, he understands: ‘I breathe in long’;\\
or breathing out long, he understands: ‘I breathe out long.’\\
Breathing in short, he understands: ‘I breathe in short’;\\
or breathing out short, he understands: ‘I breathe out short.’\\
He trains thus: ‘I will breathe in experiencing the whole body’;\\
he trains thus:‘I will breathe out experiencing the whole body.’\\
He trains thus: ‘I will breathe in tranquilising the bodily formation’;\\
he trains thus: ‘I will breathe out tranquilising the bodily formation.’

Just as, bhikkhus, a skilled lathe-worker or his apprentice,\\
when making a long turn, understands: ‘I make a long turn’;\\
or, when making a short turn, understands: ‘I make a short turn’;\\
so too, bhikkhus, a bhikkhu\\
breathing in long, he understands: ‘I breathe in long’;\\
or breathing out long, he understands: ‘I breathe out long.’\\
Breathing in short, he understands: ‘I breathe in short’;\\
or breathing out short, he understands: ‘I breathe out short.’\\
He trains thus: ‘I will breathe in experiencing the whole body’;\\
he trains thus: ‘I will breathe out experiencing the whole body.’\\
He trains thus: ‘I will breathe in tranquilising the bodily formation’;\\
he trains thus: ‘I will breathe out tranquilising the bodily formation.’

\paliPage
\chapter*{Kāyānupassanā}

\section*{Ānāpānapabbaṃ}

Kathañca pana, bhikkhave, bhikkhu kāye kāyānupassī viharati?

Idha, bhikkhave, bhikkhu araññagato vā rukkhamūlagato vā suññāgāragato vā
nisīdati pallaṅkaṃ ābhujitvā ujuṃ kāyaṃ paṇidhāya parimukhaṃ satiṃ upaṭṭhapetvā.
so satova assasati, satova passasati.

Dīghaṃ vā assasanto ‘dīghaṃ assasāmī’ti pajānāti,\\
dīghaṃ vā passasanto ‘dīghaṃ passasāmī’ti pajānāti.\\
rassaṃ vā assasanto ‘rassaṃ assasāmī’ti pajānāti,\\
rassaṃ vā passasanto ‘rassaṃ passasāmī’ti pajānāti.\\
‘sabbakāya-paṭisaṃvedī assasissāmī’ti sikkhati,\\
‘sabbakāya-paṭisaṃvedī passasissāmī’ti sikkhati.\\
‘passambhayaṃ kāyasaṅkhāraṃ assasissāmī’ti sikkhati,\\
‘passambhayaṃ kāyasaṅkhāraṃ passasissāmī’ti sikkhati.

Seyyathāpi, bhikkhave, dakkho bhamakāro vā bhamakārantevāsī vā\\
dīghaṃ vā añchanto ‘dīghaṃ añchāmī’ti pajānāti,\\
rassaṃ vā añchanto ‘rassaṃ añchāmī’ti pajānāti;\\
evameva kho, bhikkhave, bhikkhu\\
dīghaṃ vā assasanto ‘dīghaṃ assasāmī’ti pajānāti,\\
dīghaṃ vā passasanto ‘dīghaṃ passasāmī’ti pajānāti,\\
rassaṃ vā assasanto ‘rassaṃ assasāmī’ti pajānāti,\\
rassaṃ vā passasanto ‘rassaṃ passasāmī’ti pajānāti.\\
‘sabbakāya-paṭisaṃvedī assasissāmī’ti sikkhati,\\
‘sabbakāya-paṭisaṃvedī passasissāmī’ti sikkhati,\\
‘passambhayaṃ kāyasaṅkhāraṃ assasissāmī’ti sikkhati,\\
‘passambhayaṃ kāyasaṅkhāraṃ passasissāmī’ti sikkhati.

\englishPage

In this way he dwells contemplating the body in the body internally, or he
dwells contemplating the body in the body externally, or he dwells contemplating
the body in the body both internally and externally. Or else he dwells
contemplating in the body its nature of arising, or he dwells contemplating in
the body its nature of vanishing, or he dwells contemplating in the body its
nature of both arising and vanishing. Or else mindfulness that ‘there is a body’
is simply established in him to the extent necessary for bare knowledge and
repeated mindfulness.

And he dwells independent, not clinging to anything in the world. That is how,
bhikkhus, a bhikkhu dwells contemplating the body in the body.

\instr{The section on Mindfulness of Breathing is finished.}

\section{The Four Postures}

Again, bhikkhus, a bhikkhu when walking, understands: ‘I am walking’; when
standing, he understands: ‘I am standing’; when sitting, he understands: ‘I am
sitting’; when lying down, he understands: ‘I am lying down’; or however his
body is disposed, he understands it accordingly.

In this way he dwells contemplating the body in the body internally, or he
dwells contemplating the body in the body externally, or he dwells contemplating
the body in the body both internally and externally. Or else he dwells
contemplating in the body its nature of arising, or he dwells contemplating in
the body its nature of vanishing, or he dwells contemplating in the body its
nature of both arising and vanishing. Or else mindfulness that ‘there is a body’
is simply established in him to the extent necessary for bare knowledge and
repeated mindfulness.

And he dwells independent, not clinging to anything in the world. That is how,
bhikkhus, a bhikkhu dwells contemplating the body in the body.

\instr{The section on the Four Postures is finished.}

\paliPage

Iti ajjhattaṃ vā kāye kāyānupassī viharati, bahiddhā vā kāye kāyānupassī
viharati, ajjhatta-bahiddhā vā kāye kāyānupassī viharati. samudaya-dhammānupassī
vā kāyasmiṃ viharati, vaya-dhammā-\\
nupassī vā kāyasmiṃ viharati, samudaya-vaya-dhammānupassī vā kāyasmiṃ viharati.
‘atthi kāyo’ti vā panassa sati paccupaṭṭhitā hoti yāvadeva ñāṇamattāya
paṭissatimattāya anissito ca viharati, na ca kiñci loke upādiyati. evampi kho,
bhikkhave, bhikkhu kāye kāyānupassī viharati.

\instr{Ānāpānapabbaṃ niṭṭhitaṃ.}

\section*{Iriyāpathapabba}

Puna caparaṃ, bhikkhave, bhikkhu gacchanto vā ‘gacchāmī’ti pajānāti, ṭhito vā
‘ṭhitomhī’ti pajānāti, nisinno vā ‘nisinnomhī’ti pajānāti, sayāno vā
‘sayānomhī’ti pajānāti, yathā yathā vā panassa kāyo paṇihito hoti tathā tathā
naṃ pajānāti.

Iti ajjhattaṃ vā kāye kāyānupassī viharati, bahiddhā vā kāye kāyānupassī
viharati, ajjhatta-bahiddhā vā kāye kāyānupassī viharati. samudaya-dhammānupassī
vā kāyasmiṃ viharati, vaya-dhammā-\\
nupassī vā kāyasmiṃ viharati, samudaya-vaya-dhammānupassī vā kāyasmiṃ viharati.
‘atthi kāyo’ti vā panassa sati paccupaṭṭhitā hoti yāvadeva ñāṇamattāya
paṭissatimattāya anissito ca viharati, na ca kiñci loke upādiyati. evampi kho,
bhikkhave, bhikkhu kāye kāyānupassī viharati.

\instr{Iriyāpathapabbaṃ niṭṭhitaṃ.}

\englishPage
\section{Clear Comprehension}

Again, bhikkhus, a bhikkhu is one who acts with clear comprehension when going
forward and returning, who acts with clear comprehension when looking ahead and
looking away; who acts with clear comprehension when bending and stretching his
limbs; who acts with clear comprehension when wearing his robes and carrying his
outer robe and bowl; who acts with clear comprehension when eating, drinking,
chewing, and tasting; who acts with clear comprehension when defecating and
urinating; who acts with clear comprehension when walking, standing, sitting,
falling asleep, waking up, talking and keeping silent.

In this way he dwells contemplating the body in the body internally, or he
dwells contemplating the body in the body externally, or he dwells contemplating
the body in the body both internally and externally. Or else he dwells
contemplating in the body its nature of arising, or he dwells contemplating in
the body its nature of vanishing, or he dwells contemplating in the body its
nature of both arising and vanishing. Or else mindfulness that ‘there is a body’
is simply established in him to the extent necessary for bare knowledge and
repeated mindfulness.

And he dwells independent, not clinging to anything in the world. That is how,
bhikkhus, a bhikkhu dwells contemplating the body in the body.

\instr{The section on Clear Comprehension is finished.}

\section{Unattractiveness of the Body}

Again, bhikkhus, a bhikkhu reviews this same body up from the soles of the feet
and down from the top of the hair, bounded by skin, as full of many kinds of
impurity thus:

\enlargethispage{2\baselineskip}

‘In this body there are head-hairs, body hairs, nails, teeth, skin, flesh,
sinews, bones, bone-marrow, kidneys, heart, liver, diaphragm, spleen, lungs,
intestines, mesentery, stomach, feces, bile, phlegm, pus, blood, sweat, fat,
tears, grease, spittle, snot, oil of the joints, and urine.’

\paliPage
\section*{Sampajānapabba}

Puna caparaṃ, bhikkhave, bhikkhu abhikkante paṭikkante sampajānakārī hoti,
ālokite vilokite sampajānakārī hoti, samiñjite pasārite sampajānakārī hoti,
saṅghāṭi-patta-cīvara-dhāraṇe sampajānakārī hoti, asite pīte khāyite sāyite
sampajānakārī hoti, uccāra-passāva-kamme sampajānakārī hoti, gate ṭhite nisinne
sutte jāgarite bhāsite tuṇhībhāve sampajānakārī hoti.

Iti ajjhattaṃ vā kāye kāyānupassī viharati, bahiddhā vā kāye kāyānupassī
viharati, ajjhatta-bahiddhā vā kāye kāyānupassī viharati. samudaya-dhammānupassī
vā kāyasmiṃ viharati, vaya-dhammā-\\
nupassī vā kāyasmiṃ viharati, samudaya-vaya-dhammānupassī vā kāyasmiṃ viharati.
‘atthi kāyo’ti vā panassa sati paccupaṭṭhitā hoti yāvadeva ñāṇamattāya
paṭissatimattāya anissito ca viharati, na ca kiñci loke upādiyati. evampi kho,
bhikkhave, bhikkhu kāye kāyānupassī viharati.

\instr{Sampajānapabbaṃ niṭṭhitaṃ.}

\section*{Paṭikūla-manasikārapabba}

Puna caparaṃ, bhikkhave, bhikkhu imameva kāyaṃ uddhaṃ pādatalā adho kesamatthakā
tacapariyantaṃ pūraṃ nānappakārassa asucino paccavekkhati:

‘Atthi imasmiṃ kāye kesā lomā nakhā dantā taco, maṃsaṃ nahārū aṭṭhī aṭṭhimiñjaṃ
vakkaṃ, hadayaṃ yakanaṃ kilomakaṃ pihakaṃ papphāsaṃ, antaṃ antaguṇaṃ udariyaṃ
karīsaṃ, pittaṃ semhaṃ pubbo lohitaṃ sedo medo, assu vasā kheḷo siṅghāṇikā
lasikā muttan’ti.

\englishPage

Just as though, bhikkhus, there were a bag with an opening at both ends full of
many sorts of grain, such as hill rice, red rice, beans, peas, millet, and white
rice, and a man with good eyes were to open it and review it thus:

‘This is hill rice, this is red rice, these are beans, these are peas, this is
millet, this is white rice’; so too, bhikkhus, a bhikkhu reviews this same body
up from the soles of the feet and down from the top of the hair, bounded by
skin, as full of many kinds of impurity thus:

'In this body there are head-hairs, body hairs, nails, teeth, skin, flesh,
sinews, bones, bone-marrow, kidneys, heart, liver, diaphragm, spleen, lungs,
intestines, mesentery, stomach, feces, bile, phlegm, pus, blood, sweat, fat,
tears, grease, spittle, snot, oil of the joints, and urine.'

In this way he dwells contemplating the body in the body internally, or he
dwells contemplating the body in the body externally, or he dwells contemplating
the body in the body both internally and externally. Or else he dwells
contemplating in the body its nature of arising, or he dwells contemplating in
the body its nature of vanishing, or he dwells contemplating in the body its
nature of both arising and vanishing. Or else mindfulness that ‘there is a body’
is simply established in him to the extent necessary for bare knowledge and
repeated mindfulness.

And he dwells independent, not clinging to anything in the world. That is how,
bhikkhus, a bhikkhu dwells contemplating the body in the body.

\instr{The section on Unattractiveness of the Body is finished.}

\paliPage

Seyyathāpi, bhikkhave, ubhatomukhā putoḷi pūrā nānāvihitassa dhaññassa,
seyyathīdaṃ, sālīnaṃ vīhīnaṃ muggānaṃ māsānaṃ tilānaṃ taṇḍulānaṃ. Tamenaṃ
cakkhumā puriso muñcitvā paccavekkheyya:

‘Ime sālī, ime vīhī ime muggā ime māsā ime tilā ime taṇḍulā’ti. Evameva kho,
bhikkhave, bhikkhu imameva kāyaṃ uddhaṃ pādatalā adho kesamatthakā
tacapariyantaṃ pūraṃ nānappakārassa asucino paccavekkhati:

‘Atthi imasmiṃ kāye kesā lomā nakhā dantā taco, maṃsaṃ nahārū aṭṭhī aṭṭhimiñjaṃ
vakkaṃ, hadayaṃ yakanaṃ kilomakaṃ pihakaṃ papphāsaṃ, antaṃ antaguṇaṃ udariyaṃ
karīsaṃ, pittaṃ semhaṃ pubbo lohitaṃ sedo medo, assu vasā kheḷo siṅghāṇikā
lasikā muttan’ti.

Iti ajjhattaṃ vā kāye kāyānupassī viharati, bahiddhā vā kāye kāyānupassī
viharati, ajjhatta-bahiddhā vā kāye kāyānupassī viharati. samudaya-dhammānupassī
vā kāyasmiṃ viharati, vaya-dhammā-\\
nupassī vā kāyasmiṃ viharati, samudaya-vaya-dhammānupassī vā kāyasmiṃ viharati.
‘atthi kāyo’ti vā panassa sati paccupaṭṭhitā hoti yāvadeva ñāṇamattāya
paṭissatimattāya anissito ca viharati, na ca kiñci loke upādiyati. evampi kho,
bhikkhave, bhikkhu kāye kāyānupassī viharati.

\instr{Paṭikūla-manasikārapabbaṃ niṭṭhitaṃ.}

\englishPage
\section{Elements}

Again, bhikkhus, a bhikkhu reviews this same body, however it is placed, however
disposed, as consisting of elements thus: `In this body there are the earth
element, the water element, the fire element, and the air element.'

Just as though, bhikkhus, a skilled butcher or his apprentice had killed a cow
and were seated at the crossroads with it cut up into pieces; so too, bhikkhus,
a bhikkhu reviews this same body, however it is placed, however disposed, as
consisting of elements thus: `In this body there are the earth element, the
water element, the fire element, and the air element.'

In this way he dwells contemplating the body in the body internally, or he
dwells contemplating the body in the body externally, or he dwells contemplating
the body in the body both internally and externally. Or else he dwells
contemplating in the body its nature of arising, or he dwells contemplating in
the body its nature of vanishing, or he dwells contemplating in the body its
nature of both arising and vanishing. Or else mindfulness that ‘there is a body’
is simply established in him to the extent necessary for bare knowledge and
repeated mindfulness.

And he dwells independent, not clinging to anything in the world. That is how,
bhikkhus, a bhikkhu dwells contemplating the body in the body.

\instr{The section on Elements is finished.}

\section{Nine Charnel Ground Contemplations}

[1] Again, bhikkhus, as though he were to see a corpse thrown aside in a charnel
ground, one, two, or three days dead, bloated, livid, and oozing matter, a
bhikkhu compares this same body with it thus: 'This body too is of the same
nature, it will be like that, it is not exempt from that fate.'

\paliPage
\section*{Dhātu-manasikārapabba}

Puna caparaṃ, bhikkhave, bhikkhu imameva kāyaṃ yathāṭhitaṃ yathāpaṇihitaṃ
dhātuso paccavekkhati: ‘atthi imasmiṃ kāye pathavīdhātu āpodhātu tejodhātu
vāyodhātū’ti.

Seyyathāpi, bhikkhave, dakkho goghātako vā goghāta-kantevāsī vā gāviṃ vadhitvā
cātummahāpathe bilaso vibhajitvā nisinno assa; evameva kho, bhikkhave, bhikkhu
imameva kāyaṃ yathāṭhitaṃ yathāpaṇihitaṃ dhātuso paccavekkhati: ‘atthi imasmiṃ
kāye pathavīdhātu āpodhātu tejodhātu vāyodhātū’ti.

Iti ajjhattaṃ vā kāye kāyānupassī viharati, bahiddhā vā kāye kāyānupassī
viharati, ajjhatta-bahiddhā vā kāye kāyānupassī viharati. samudaya-dhammānupassī
vā kāyasmiṃ viharati, vaya-dhammā-\\
nupassī vā kāyasmiṃ viharati, samudaya-vaya-dhammānupassī vā kāyasmiṃ viharati.
‘atthi kāyo’ti vā panassa sati paccupaṭṭhitā hoti yāvadeva ñāṇamattāya
paṭissatimattāya anissito ca viharati, na ca kiñci loke upādiyati. evampi kho,
bhikkhave, bhikkhu kāye kāyānupassī viharati.

\instr{Dhātu-manasikārapabbaṃ niṭṭhitaṃ.}

\section*{Navasivathikapabba}

[1] Puna caparaṃ, bhikkhave, bhikkhu seyyathāpi passeyya sarīraṃ sivathikāya
chaḍḍitaṃ ekāhamataṃ vā dvīhamataṃ vā tīhamataṃ vā uddhumātakaṃ vinīlakaṃ
vipubbakajātaṃ. So imameva kāyaṃ upasaṃharati: ‘ayampi kho kāyo evaṃ-dhammo
evaṃ-bhāvī evaṃ-anatīto’ti.

\englishPage

In this way he dwells contemplating the body in the body internally, or he
dwells contemplating the body in the body externally, or he dwells contemplating
the body in the body both internally and externally. Or else he dwells
contemplating in the body its nature of arising, or he dwells contemplating in
the body its nature of vanishing, or he dwells contemplating in the body its
nature of both arising and vanishing. Or else mindfulness that ‘there is a body’
is simply established in him to the extent necessary for bare knowledge and
repeated mindfulness.

And he dwells independent, not clinging to anything in the world. That is how,
bhikkhus, a bhikkhu dwells contemplating the body in the body.

[2] Again, bhikkhus, as though he were to see a corpse thrown aside in a charnel
ground, being devoured by crows, being devoured by vultures, being devoured by
hawks, being devoured by dogs, being devoured by jackals, or being devoured by
various kinds of worms, a bhikkhu compares this same body with it thus: 'This
body too is of the same nature, it will be like that, it is not exempt from that
fate.'

In this way he dwells contemplating the body in the body internally, or he
dwells contemplating the body in the body externally, or he dwells contemplating
the body in the body both internally and externally. Or else he dwells
contemplating in the body its nature of arising, or he dwells contemplating in
the body its nature of vanishing, or he dwells contemplating in the body its
nature of both arising and vanishing. Or else mindfulness that ‘there is a body’
is simply established in him to the extent necessary for bare knowledge and
repeated mindfulness.

And he dwells independent, not clinging to anything in the world. That is how,
bhikkhus, a bhikkhu dwells contemplating the body in the body.

\paliPage

Iti ajjhattaṃ vā kāye kāyānupassī viharati, bahiddhā vā kāye kāyānupassī
viharati, ajjhatta-bahiddhā vā kāye kāyānupassī viharati. samudaya-dhammānupassī
vā kāyasmiṃ viharati, vaya-dhammā-\\
nupassī vā kāyasmiṃ viharati, samudaya-vaya-dhammānupassī vā kāyasmiṃ viharati.
‘atthi kāyo’ti vā panassa sati paccupaṭṭhitā hoti yāvadeva ñāṇamattāya
paṭissatimattāya anissito ca viharati, na ca kiñci loke upādiyati. evampi kho,
bhikkhave, bhikkhu kāye kāyānupassī viharati.

[2] Puna caparaṃ, bhikkhave, bhikkhu seyyathāpi passeyya sarīraṃ sivathikāya
chaḍḍitaṃ kākehi vā khajjamānaṃ kulalehi vā khajjamānaṃ gijjhehi vā khajjamānaṃ
kaṅkehi vā khajjamānaṃ sunakhehi vā khajjamānaṃ byagghehi vā khajjamānaṃ dīpīhi
vā khajjamānaṃ siṅgālehi vā khajjamānaṃ vividhehi vā pāṇakajātehi khajjamānaṃ.
So imameva kāyaṃ upasaṃharati: ‘ayampi kho kāyo evaṃ-dhammo evaṃ-bhāvī
evaṃ-anatīto’ti.

Iti ajjhattaṃ vā kāye kāyānupassī viharati, bahiddhā vā kāye kāyānupassī
viharati, ajjhatta-bahiddhā vā kāye kāyānupassī viharati. samudaya-dhammānupassī
vā kāyasmiṃ viharati, vaya-dhammā-\\
nupassī vā kāyasmiṃ viharati, samudaya-vaya-dhammānupassī vā kāyasmiṃ viharati.
‘atthi kāyo’ti vā panassa sati paccupaṭṭhitā hoti yāvadeva ñāṇamattāya
paṭissatimattāya anissito ca viharati, na ca kiñci loke upādiyati. evampi kho,
bhikkhave, bhikkhu kāye kāyānupassī viharati.

\englishPage

[3]~Again, bhikkhus, as though he were to see a corpse thrown aside in a
charnel ground, a skeleton with flesh and blood, held together with sinews~\ldots{}

[4]~a fleshless skeleton smeared with blood, held together with sinews~\ldots{}

[5]~a skeleton without flesh and blood, held together with sinews~\ldots{}

[6]~disconnected bones not held together with sinews scattered in all directions
-- here a hand-bone, there a foot bone, here a shin-bone, there a thigh-bone,
here a hip-bone, there a back-bone, here a rib-bone, there a chest-bone, here an
arm-bone, there a shoulder-bone, here a neck-bone, there a jaw-bone, here a
tooth-bone, there the skull -- a bhikkhu compares this same body with it thus:
`This body too is of the same nature, it will be like that, it is not exempt
from that fate.'

In this way he dwells contemplating the body in the body internally, or he
dwells contemplating the body in the body externally, or he dwells contemplating
the body in the body both internally and externally. Or else he dwells
contemplating in the body its nature of arising, or he dwells contemplating in
the body its nature of vanishing, or he dwells contemplating in the body its
nature of both arising and vanishing. Or else mindfulness that ‘there is a body’
is simply established in him to the extent necessary for bare knowledge and
repeated mindfulness.

And he dwells independent, not clinging to anything in the world. That is how,
bhikkhus, a bhikkhu dwells contemplating the body in the body.

[7] Again, bhikkhus, as though he were to see a corpse thrown aside in a charnel
ground, bones bleached white, the colour of shells~\ldots{}

[8]~bones heaped up, more than a year old~\ldots{}

\paliPage

[3]~Puna caparaṃ, bhikkhave, bhikkhu seyyathāpi passeyya sarīraṃ sivathikāya
chaḍḍitaṃ aṭṭhikasaṅkhalikaṃ samaṃsalohitaṃ nahārusambandhaṃ~\ldots{}

[4]~Aṭṭhikasaṅkhalikaṃ nimaṃsalohitamakkhitaṃ nahārusambandhaṃ~\ldots{}

[5]~Aṭṭhikasaṅkhalikaṃ apagatamaṃsalohitaṃ nahārusambandhaṃ~\ldots{}

[6]~Aṭṭhikāni apagatasambandhāni disā vidisā vikkhittāni, aññena hatthaṭṭhikaṃ
aññena pādaṭṭhikaṃ aññena gopphakaṭṭhikaṃ aññena jaṅghaṭṭhikaṃ aññena ūruṭṭhikaṃ
aññena kaṭiṭṭhikaṃ aññena phāsukaṭṭhikaṃ aññena piṭṭhiṭṭhikaṃ aññena
khandhaṭṭhikaṃ aññena gīvaṭṭhikaṃ aññena hanukaṭṭhikaṃ aññena dantaṭṭhikaṃ
aññena sīsakaṭāhaṃ. So imameva kāyaṃ upasaṃharati: ‘ayampi kho kāyo evaṃdhammo
evaṃbhāvī evaṃanatīto’ti.

Iti ajjhattaṃ vā kāye kāyānupassī viharati, bahiddhā vā kāye kāyānupassī
viharati, ajjhatta-bahiddhā vā kāye kāyānupassī viharati. samudaya-dhammānupassī
vā kāyasmiṃ viharati, vaya-dhammā-\\
nupassī vā kāyasmiṃ viharati, samudaya-vaya-dhammānupassī vā kāyasmiṃ viharati.
‘atthi kāyo’ti vā panassa sati paccupaṭṭhitā hoti yāvadeva ñāṇamattāya
paṭissatimattāya anissito ca viharati, na ca kiñci loke upādiyati. evampi kho,
bhikkhave, bhikkhu kāye kāyānupassī viharati.

[7]~Puna caparaṃ, bhikkhave, bhikkhu seyyathāpi passeyya sarīraṃ sivathikāya
chaḍḍitaṃ aṭṭhikāni setāni saṅkhavaṇṇapaṭibhāgāni~\ldots{}

[8]~Aṭṭhikāni puñjakitāni terovassikāni~\ldots{}

\englishPage

[9]~bones rotted and crumbled to dust, a bhikkhu compares this same
body with it thus: ‘This body too is of the same nature, it will be like that,
it is not exempt from that fate.’

In this way he dwells contemplating the body in the body internally, or he
dwells contemplating the body in the body externally, or he dwells contemplating
the body in the body both internally and externally. Or else he dwells
contemplating in the body its nature of arising, or he dwells contemplating in
the body its nature of vanishing, or he dwells contemplating in the body its
nature of both arising and vanishing. Or else mindfulness that ‘there is a body’
is simply established in him to the extent necessary for bare knowledge and
repeated mindfulness.

And he dwells independent, not clinging to anything in the world. That is how,
bhikkhus, a bhikkhu dwells contemplating the body in the body.

\instr{The section on the Nine Charnel Ground Contemplations is finished.}

\instr{Contemplation of the Body is finished.}

\paliPage

[9]~Aṭṭhikāni pūtīni cuṇṇakajātāni. so imameva kāyaṃ upasaṃharati: ‘ayampi kho
kāyo evaṃ-dhammo evaṃ-bhāvī evaṃ-anatīto’ti.

Iti ajjhattaṃ vā kāye kāyānupassī viharati, bahiddhā vā kāye kāyānupassī
viharati, ajjhatta-bahiddhā vā kāye kāyānupassī viharati. samudaya-dhammānupassī
vā kāyasmiṃ viharati, vaya-dhammā-\\
nupassī vā kāyasmiṃ viharati, samudaya-vaya-dhammānupassī vā kāyasmiṃ viharati.
‘atthi kāyo’ti vā panassa sati paccupaṭṭhitā hoti yāvadeva ñāṇamattāya
paṭissatimattāya anissito ca viharati, na ca kiñci loke upādiyati. evampi kho,
bhikkhave, bhikkhu kāye kāyānupassī viharati.

\instr{Navasivathikapabbaṃ niṭṭhitaṃ.}

\instr{Kāyānupassanā niṭṭhitā.}

\englishPage
\chapter{Contemplation of Feelings}

And how, bhikkhus, does a bhikkhu dwell contemplating feelings in feelings?

Here, bhikkhus, when feeling a pleasant feeling, a bhikkhu understands:
`I~feel a pleasant feeling';
when feeling a painful feeling, he understands:
`I~feel a painful feeling';
when feeling a neither-painful-nor-pleasant feeling, he understands:
`I~feel a neither-painful-nor-pleasant feeling.'

When feeling a carnal pleasant feeling, he understands:
`I~feel a carnal pleasant feeling';
when feeling a spiritual pleasant feeling, he understands:
`I~feel a spiritual pleasant feeling';
when feeling a carnal painful feeling, he understands:
`I~feel a carnal painful feeling';
when feeling a spiritual painful feeling, he understands:
`I~feel a spiritual painful feeling';
when feeling a carnal neither-painful-nor-pleasant feeling, he understands:
`I~feel a carnal neither-painful-nor-pleasant feeling';
when feeling a spiritual neither-painful-nor-pleasant feeling, he understands:
`I~feel a spiritual neither-painful-nor-pleasant feeling.'

\paliPage
\chapter*{Vedanānupassanā}

Kathañca pana, bhikkhave, bhikkhu vedanāsu vedanānupassī viharati?

Idha, bhikkhave, bhikkhu\\
sukhaṃ vā vedanaṃ vedayamāno\\
‘sukhaṃ vedanaṃ vedayāmī’ti pajānāti.\\
dukkhaṃ vā vedanaṃ vedayamāno\\
‘dukkhaṃ vedanaṃ vedayāmī’ti pajānāti.\\
adukkhamasukhaṃ vā vedanaṃ vedayamāno\\
‘adukkhamasukhaṃ vedanaṃ vedayāmī’ti pajānāti.

Sāmisaṃ vā sukhaṃ vedanaṃ vedayamāno\\
‘sāmisaṃ sukhaṃ vedanaṃ vedayāmī’ti pajānāti.\\
nirāmisaṃ vā sukhaṃ vedanaṃ vedayamāno\\
‘nirāmisaṃ sukhaṃ vedanaṃ vedayāmī’ti pajānāti.

Sāmisaṃ vā dukkhaṃ vedanaṃ vedayamāno\\
‘sāmisaṃ dukkhaṃ vedanaṃ vedayāmī’ti pajānāti.\\
nirāmisaṃ vā dukkhaṃ vedanaṃ vedayamāno\\
‘nirāmisaṃ dukkhaṃ vedanaṃ vedayāmī’ti pajānāti.

Sāmisaṃ vā adukkhamasukhaṃ vedanaṃ vedayamāno\\
‘sāmisaṃ adukkhamasukhaṃ vedanaṃ vedayāmī’ti pajānāti.\\
nirāmisaṃ vā adukkhamasukhaṃ vedanaṃ vedayamāno\\
‘nirāmisaṃ adukkhamasukhaṃ vedanaṃ vedayāmī’ti pajānāti.

\englishPage

In this way he dwells contemplating feelings in feelings internally, or he
dwells contemplating feelings in feelings externally, or he dwells contemplating
feelings in feelings both internally and externally. Or else he dwells
contemplating in feelings their nature of arising, or he dwells contemplating in
feelings their nature of vanishing, or he dwells contemplating in feelings their
nature of both arising and vanishing. Or else mindfulness that ‘there is
feeling’ is simply established in him to the extent necessary for bare knowledge
and repeated mindfulness.

And he dwells independent, not clinging to anything in the world. That is how,
bhikkhus, a bhikkhu dwells contemplating feelings in feelings.

\instr{The Contemplation of Feelings is finished.}

\chapter{Contemplation of Mind}

And how, bhikkhus, does a bhikkhu dwell contemplating mind in mind?

Here, bhikkhus, a bhikkhu\\
understands a mind with lust as a mind with lust,\\
and a mind without lust as a mind without lust.\\
He understands a mind with hatred as a mind with hatred,\\
and a mind without hatred as a mind without hatred.\\
He understands a mind with delusion as a mind with delusion,\\
and a mind without delusion as a mind without delusion.\\
He understands a contracted mind as contracted,\\
and a distracted mind as distracted.\\
He understands an exalted mind as exalted,\\
and an unexalted mind as unexalted.

\paliPage

Iti ajjhattaṃ vā vedanāsu vedanānupassī viharati, bahiddhā vā vedanāsu
vedanānupassī viharati, ajjhatta-bahiddhā vā vedanāsu vedanānupassī viharati.
samudaya-dhammānupassī vā vedanāsu viharati, vaya-dhammānupassī vā vedanāsu
viharati, samudaya-vaya-\\
dhammānupassī vā vedanāsu viharati. ‘atthi vedanā’ti vā panassa sati
paccupaṭṭhitā hoti yāvadeva ñāṇamattāya paṭissatimattāya anissito ca viharati,
na ca kiñci loke upādiyati. evampi kho, bhikkhave, bhikkhu vedanāsu
vedanānupassī viharati.

\instr{Vedanānupassanā niṭṭhitā.}

\chapter*{Cittānupassanā}

Kathañca pana, bhikkhave, bhikkhu citte cittānupassī viharati?

Idha, bhikkhave, bhikkhu\\
sarāgaṃ vā cittaṃ ‘sarāgaṃ cittan’ti pajānāti.\\
vītarāgaṃ vā cittaṃ ‘vītarāgaṃ cittan’ti pajānāti.\\
sadosaṃ vā cittaṃ ‘sadosaṃ cittan’ti pajānāti.\\
vītadosaṃ vā cittaṃ ‘vītadosaṃ cittan’ti pajānāti.\\
samohaṃ vā cittaṃ ‘samohaṃ cittan’ti pajānāti.\\
vītamohaṃ vā cittaṃ ‘vītamohaṃ cittan’ti pajānāti.\\
saṅkhittaṃ vā cittaṃ ‘saṅkhittaṃ cittan’ti pajānāti.\\
vikkhittaṃ vā cittaṃ ‘vikkhittaṃ cittan’ti pajānāti.\\
mahaggataṃ vā cittaṃ ‘mahaggataṃ cittan’ti pajānāti.\\
amahaggataṃ vā cittaṃ ‘amahaggataṃ cittan’ti pajānāti.

\englishPage

He understands a surpassable mind as surpassable,\\
and an unsurpassable mind as unsurpassable.\\
He understands a concentrated mind as concentrated,\\
and an unconcentrated mind as unconcentrated.\\
He understands a liberated mind as liberated,\\
and an unliberated mind as unliberated.

In this way he dwells contemplating mind in mind internally, or he dwells
contemplating mind in mind externally, or he dwells contemplating mind in mind
both internally and externally. Or else he dwells contemplating in mind its
nature of arising, or he dwells contemplating in mind its nature of vanishing,
or he dwells contemplating in mind its nature of both arising and vanishing. Or
else mindfulness that ‘there is mind’ is simply established in him to the extent
necessary for bare knowledge and repeated mindfulness.

And he dwells independent, not clinging to anything in the world. That is how,
bhikkhus, a bhikkhu dwells contemplating mind in mind.

\instr{The Contemplation of Mind is finished.}

\paliPage

Sauttaraṃ vā cittaṃ ‘sauttaraṃ cittan’ti pajānāti.\\
anuttaraṃ vā cittaṃ ‘anuttaraṃ cittan’ti pajānāti.\\
samāhitaṃ vā cittaṃ ‘samāhitaṃ cittan’ti pajānāti.\\
asamāhitaṃ vā cittaṃ ‘asamāhitaṃ cittan’ti pajānāti.\\
vimuttaṃ vā cittaṃ ‘vimuttaṃ cittan’ti pajānāti.\\
avimuttaṃ vā cittaṃ ‘avimuttaṃ cittan’ti pajānāti.

Iti ajjhattaṃ vā citte cittānupassī viharati, bahiddhā vā citte cittānupassī
viharati, ajjhatta-bahiddhā vā citte cittānupassī viharati.
samudaya-dhammānupassī vā cittasmiṃ viharati, vaya-dhammā-\\
nupassī vā cittasmiṃ viharati, samudaya-vaya-dhammānupassī vā cittasmiṃ
viharati, ‘atthi cittan’ti vā panassa sati paccupaṭṭhitā hoti yāvadeva
ñāṇamattāya paṭissatimattāya anissito ca viharati, na ca kiñci loke upādiyati.
evampi kho, bhikkhave, bhikkhu citte cittānupassī viharati.

\instr{Cittānupassanā niṭṭhitā.}

\englishPage

\paliPage

\englishPage
\chapter{Contemplation of Phenomena}

\section{The Five Hindrances}

And how, bhikkhus, does a bhikkhu dwell contemplating phenomena in phenomena?

Here, bhikkhus, a bhikkhu dwells contemplating phenomena in phenomena in terms
of the five hindrances.

And how, bhikkhus, does a bhikkhu dwell contemplating phenomena in phenomena in
terms of the five hindrances?

Here, bhikkhus, a bhikkhu,
when there is sensual desire in him, understands:
`There is sensual desire in me';
or when there is no sensual desire in him, he understands:
`There is no sensual desire in me';
and he also understands how unarisen sensual desire arises,
and how arisen sensual desire is abandoned,
and how abandoned sensual desire does not arise again in the future.

When there is ill will in him, a bhikkhu understands:
`There is ill will in me';
or when there is no ill will in him, he understands:
`There is no ill will in me';
and he also understands how unarisen ill will arises,
and how arisen ill will is abandoned,
and how abandoned ill will does not arise again in the future.

When there is dullness and drowsiness in him, a bhikkhu understands:
`There is dullness and drowsiness in me';
or when there is no dullness and drowsiness in him, he understands:
`There is no dullness and drowsiness in me';
and he also understands how unarisen dullness and drowsiness arises,
and how arisen dullness and drowsiness is abandoned, and
how abandoned dullness and drowsiness does not arise again in the future.'

\paliPage
\chapter*{Dhammānupassanā}

\section*{Nīvaraṇapabba}

Kathañca pana, bhikkhave, bhikkhu dhammesu dhammānupassī viharati?

Idha, bhikkhave, bhikkhu dhammesu dhammānupassī viharati pañcasu nīvaraṇesu.

Kathañca pana, bhikkhave, bhikkhu dhammesu dhammānupassī viharati pañcasu
nīvaraṇesu?

Idha, bhikkhave, bhikkhu
santaṃ vā ajjhattaṃ kāmacchandaṃ ‘atthi me ajjhattaṃ kāmacchando’ti pajānāti,
asantaṃ vā ajjhattaṃ kāmacchandaṃ ‘natthi me ajjhattaṃ kāmacchando’ti pajānāti,
yathā ca anuppannassa kāmacchandassa uppādo hoti tañca pajānāti,
yathā ca uppannassa kāmacchandassa pahānaṃ hoti tañca pajānāti,
yathā ca pahīnassa kāmacchandassa āyatiṃ anuppādo hoti tañca pajānāti.

Santaṃ vā ajjhattaṃ byāpādaṃ ‘atthi me ajjhattaṃ byāpādo’ti pajānāti,
asantaṃ vā ajjhattaṃ byāpādaṃ ‘natthi me ajjhattaṃ byāpādo’ti pajānāti,
yathā ca anuppannassa byāpādassa uppādo hoti tañca pajānāti,
yathā ca uppannassa byāpādassa pahānaṃ hoti tañca pajānāti,
yathā ca pahīnassa byāpādassa āyatiṃ anuppādo hoti tañca pajānāti.

\enlargethispage{\baselineskip}

Santaṃ vā ajjhattaṃ thīnamiddhaṃ ‘atthi me ajjhattaṃ thīnamiddhan’ti pajānāti,
asantaṃ vā ajjhattaṃ thīnamiddhaṃ ‘natthi me ajjhattaṃ thīnamiddhan’ti pajānāti,
yathā ca anuppannassa thīnamiddhassa uppādo hoti tañca pajānāti,
yathā ca uppannassa thīnamiddhassa pahānaṃ hoti tañca pajānāti,
yathā ca pahīnassa thīnamiddhassa āyatiṃ anuppādo hoti tañca pajānāti.

\englishPage

When there is restlessness and remorse in him, a bhikkhu understands:
`There is restlessness and remorse in me';
or when there is no restlessness and remorse in him, he understands:
`There is no restlessness and remorse in me';
and he also understands how unarisen restlessness and remorse arises,
and how arisen restlessness and remorse is abandoned,
and how abandoned restlessness and remorse does not arise again in the future.

When there is doubt in him, a bhikkhu understands:
`There is doubt in me';
or when there is no doubt in him, he understands:
`There is no doubt in me';
and he also understands how unarisen doubt arises,
and how arisen doubt is abandoned,
and how abandoned doubt does not arise again in the future.

In this way he dwells contemplating phenomena in phenomena internally, or he
dwells contemplating phenomena in phenomena externally, or he dwells
contemplating phenomena in phenomena both internally and externally. Or else he
dwells contemplating in phenomena its nature of arising, or he dwells
contemplating in phenomena its nature of vanishing, or he dwells contemplating
in phenomena its nature of both arising and vanishing. Or else mindfulness that
‘there are phenomena’ is simply established in him to the extent necessary for
bare knowledge and repeated mindfulness.

And he dwells independent, not clinging to anything in the world. That is how,
bhikkhus, a bhikkhu dwells contemplating phenomena in phenomena in terms of five
hindrances.

\instr{The section on the Five Hindrances is finished.}

\paliPage

Santaṃ vā ajjhattaṃ uddhacca-kukkuccaṃ ‘atthi me ajjhattaṃ uddhacca-kukkuccan’ti pajānāti,
asantaṃ vā ajjhattaṃ uddhacca-kukkuccaṃ ‘natthi me ajjhattaṃ uddhacca-kukkuccan’ti pajānāti,
yathā ca anuppannassa uddhacca-kukkuccassa uppādo hoti tañca pajānāti,
yathā ca uppannassa uddhacca-kukkuccassa pahānaṃ hoti tañca pajānāti,
yathā ca pahīnassa uddhacca-kukkuccassa āyatiṃ anuppādo hoti tañca pajānāti.

Santaṃ vā ajjhattaṃ vicikicchaṃ ‘atthi me ajjhattaṃ vicikicchā’ti pajānāti,
asantaṃ vā ajjhattaṃ vicikicchaṃ ‘natthi me ajjhattaṃ vicikicchā’ti pajānāti,
yathā ca anuppannāya vicikicchāya uppādo hoti tañca pajānāti,
yathā ca uppannāya vicikicchāya pahānaṃ hoti tañca pajānāti,
yathā ca pahīnāya vicikicchāya āyatiṃ anuppādo hoti tañca pajānāti.

Iti ajjhattaṃ vā dhammesu dhammānupassī viharati,
bahiddhā vā dhammesu dhammānupassī viharati,
ajjhatta-bahiddhā vā dhammesu dhammānupassī viharati.
samudaya-dhammānupassī vā dhammesu viharati,
vaya-dhammānupassī vā dhammesu viharati,
samudaya-vaya-\\ dhammānupassī vā dhammesu viharati.
‘atthi dhammā’ti vā panassa sati paccupaṭṭhitā hoti
yāvadeva ñāṇamattāya paṭissatimattāya, anissito ca viharati,
na ca kiñci loke upādiyati. evampi kho, bhikkhave, bhikkhu
dhammesu dhammānupassī viharati pañcasu nīvaraṇesu.

\instr{Nīvaraṇapabbaṃ niṭṭhitaṃ.}

\englishPage
\section{The Five Aggregates}

Again, bhikkhus, a bhikkhu dwells contemplating phenomena in phenomena in terms
of the five aggregates subject to clinging.

And how, bhikkhus, does a bhikkhu dwell contemplating phenomena in phenomena in
terms of the five aggregates affected by clinging?

Here, bhikkhus, a bhikkhu understands:
`Such is form, such its origin, such its passing away;
such is feeling, such its origin, such its passing away;
such is perception, such its origin, such its passing away;
such are the volitional formations, such their origin, such their passing away;
such is consciousness, such its origin, such its passing away.'

In this way he dwells contemplating phenomena in phenomena internally, or he
dwells contemplating phenomena in phenomena externally, or he dwells
contemplating phenomena in phenomena both internally and externally. Or else he
dwells contemplating in phenomena its nature of arising, or he dwells
contemplating in phenomena its nature of vanishing, or he dwells contemplating
in phenomena its nature of both arising and vanishing. Or else mindfulness that
‘there are phenomena’ is simply established in him to the extent necessary for
bare knowledge and repeated mindfulness.

And he dwells independent, not clinging to anything in the world. That is how,
bhikkhus, a bhikkhu dwells contemplating phenomena in phenomena in terms of the
five aggregates subject to clinging.

\instr{The section on the Five Aggregates is finished.}

\paliPage
\section*{Khandhapabba}

Puna caparaṃ, bhikkhave, bhikkhu dhammesu dhammānupassī viharati pañcasu
upādāna-kkhandhesu.

Kathañca pana, bhikkhave, bhikkhu dhammesu dhammānupassī viharati pañcasu
upādāna-kkhandhesu?

Idha, bhikkhave, bhikkhu:
‘iti rūpaṃ, iti rūpassa samudayo, iti rūpassa atthaṅgamo;
iti vedanā, iti vedanāya samudayo, iti vedanāya atthaṅgamo;
iti saññā, iti saññāya samudayo, iti saññāya atthaṅgamo;
iti saṅkhārā, iti saṅkhārānaṃ samudayo, iti saṅkhārānaṃ atthaṅgamo;
iti viññāṇaṃ, iti viññāṇassa samudayo, iti viññāṇassa atthaṅgamo’ti.

Iti ajjhattaṃ vā dhammesu dhammānupassī viharati,
bahiddhā vā dhammesu dhammānupassī viharati,
ajjhatta-bahiddhā vā dhammesu dhammānupassī viharati.
samudaya-dhammānupassī vā dhammesu viharati,
vaya-dhammānupassī vā dhammesu viharati,
samudaya-vaya-\\ dhammānupassī vā dhammesu viharati.
‘atthi dhammā’ti vā panassa sati paccupaṭṭhitā hoti
yāvadeva ñāṇamattāya paṭissatimattāya, anissito ca viharati,
na ca kiñci loke upādiyati. evampi kho, bhikkhave, bhikkhu
dhammesu dhammānupassī viharati pañcasu upādāna-kkhandhesu.

\instr{Khandhapabbaṃ niṭṭhitaṃ.}

\englishPage
\section{The Six Sense Bases}

Again, bhikkhus, a bhikkhu dwells contemplating phenomena in phenomena in terms
of the six internal and external sense bases.

And how, bhikkhus, does a bhikkhu dwell contemplating phenomena in phenomena in
terms of the six internal and external sense bases?

Here, bhikkhus, a bhikkhu understands the eye, he understands forms, and he
understands the fetter that arises dependent on both; and he also understands
how the unarisen fetter arises, and how the arisen fetter is abandoned, and how
the abandoned fetter does not arise again in the future.

He understands the ear, he understands sounds, and he understands the fetter
that arises dependent on both; and he also understands how the unarisen fetter
arises, and how the arisen fetter is abandoned, and how the abandoned fetter
does not arise again in the future.

He understands the nose, he understands odours, and he understands the fetter
that arises dependent on both; and he also understands how the unarisen fetter
arises, and how the arisen fetter is abandoned, and how the abandoned fetter
does not arise again in the future.

He understands the tongue, he understands flavours, and he understands the
fetter that arises dependent on both; and he also understands how the unarisen
fetter arises, and how the arisen fetter is abandoned, and how the abandoned
fetter does not arise again in the future.

\paliPage
\section*{Āyatanapabba}

Puna caparaṃ, bhikkhave, bhikkhu dhammesu dhammānupassī viharati chasu
ajjhattika-bāhiresu āyatanesu.

Kathañca pana, bhikkhave, bhikkhu dhammesu dhammānupassī viharati chasu
ajjhattika-bāhiresu āyatanesu?

Idha, bhikkhave, bhikkhu
cakkhuñca pajānāti,
rūpe ca pajānāti,
yañca tadubhayaṃ paṭicca uppajjati saṃyojanaṃ tañca pajānāti,
yathā ca anuppannassa saṃyojanassa uppādo hoti tañca pajānāti,
yathā ca uppannassa saṃyojanassa pahānaṃ hoti tañca pajānāti,
yathā ca pahīnassa saṃyojanassa āyatiṃ anuppādo hoti tañca pajānāti.

Sotañca pajānāti,
sadde ca pajānāti,
yañca tadubhayaṃ paṭicca uppajjati saṃyojanaṃ tañca pajānāti,
yathā ca anuppannassa saṃyojanassa uppādo hoti tañca pajānāti,
yathā ca uppannassa saṃyojanassa pahānaṃ hoti tañca pajānāti,
yathā ca pahīnassa saṃyojanassa āyatiṃ anuppādo hoti tañca pajānāti.

Ghānañca pajānāti,
gandhe ca pajānāti,
yañca tadubhayaṃ paṭicca uppajjati saṃyojanaṃ tañca pajānāti,
yathā ca anuppannassa saṃyojanassa uppādo hoti tañca pajānāti,
yathā ca uppannassa saṃyojanassa pahānaṃ hoti tañca pajānāti,
yathā ca pahīnassa saṃyojanassa āyatiṃ anuppādo hoti tañca pajānāti.

Jivhañca pajānāti,
rase ca pajānāti,
yañca tadubhayaṃ paṭicca uppajjati saṃyojanaṃ tañca pajānāti,
yathā ca anuppannassa saṃyojanassa uppādo hoti tañca pajānāti,
yathā ca uppannassa saṃyojanassa pahānaṃ hoti tañca pajānāti,
yathā ca pahīnassa saṃyojanassa āyatiṃ anuppādo hoti tañca pajānāti.

\englishPage

He understands the body, he understands tactile objects, and he understands the
fetter that arises dependent on both; and he also understands how the unarisen
fetter arises, and how the arisen fetter is abandoned, and how the abandoned
fetter does not arise again in the future.

He understands the mind, he understands phenomena, and he understands the fetter
that arises dependent on both; and he also understands how the unarisen fetter
arises, and how the arisen fetter is abandoned, and how the abandoned fetter
does not arise again in the future.

In this way he dwells contemplating phenomena in phenomena internally, or he
dwells contemplating phenomena in phenomena externally, or he dwells
contemplating phenomena in phenomena both internally and externally. Or else he
dwells contemplating in phenomena its nature of arising, or he dwells
contemplating in phenomena its nature of vanishing, or he dwells contemplating
in phenomena its nature of both arising and vanishing. Or else mindfulness that
‘there are phenomena’ is simply established in him to the extent necessary for
bare knowledge and repeated mindfulness.

And he dwells independent, not clinging to anything in the world. That is how,
bhikkhus, a bhikkhu dwells contemplating phenomena in phenomena in terms of the
six internal and external sense bases.

\instr{The section on the six sense bases is finished.}

\section{The Seven Enlightenment Factors}

Again, bhikkhus, a bhikkhu dwells contemplating phenomena in phenomena in terms
of the seven enlightenment factors.

And how, bhikkhus, does a bhikkhu dwell contemplating phenomena in phenomena in
terms of the seven enlightenment factors?

\paliPage

Kāyañca pajānāti,
phoṭṭhabbe ca pajānāti,
yañca tadubhayaṃ paṭicca uppajjati saṃyojanaṃ tañca pajānāti,
yathā ca anuppannassa saṃyojanassa uppādo hoti tañca pajānāti,
yathā ca uppannassa saṃyojanassa pahānaṃ hoti tañca pajānāti,
yathā ca pahīnassa saṃyojanassa āyatiṃ anuppādo hoti tañca pajānāti.

Manañca pajānāti,
dhamme ca pajānāti,
yañca tadubhayaṃ paṭicca uppajjati saṃyojanaṃ tañca pajānāti,
yathā ca anuppannassa saṃyojanassa uppādo hoti tañca pajānāti,
yathā ca uppannassa saṃyojanassa pahānaṃ hoti tañca pajānāti,
yathā ca pahīnassa saṃyojanassa āyatiṃ anuppādo hoti tañca pajānāti.

Iti ajjhattaṃ vā dhammesu dhammānupassī viharati,
bahiddhā vā dhammesu dhammānupassī viharati,
ajjhatta-bahiddhā vā dhammesu dhammānupassī viharati.
samudaya-dhammānupassī vā dhammesu viharati,
vaya-dhammānupassī vā dhammesu viharati,
samudaya-vaya-\\ dhammānupassī vā dhammesu viharati.
‘atthi dhammā’ti vā panassa sati paccupaṭṭhitā hoti
yāvadeva ñāṇamattāya paṭissatimattāya, anissito ca viharati,
na ca kiñci loke upādiyati. evampi kho, bhikkhave, bhikkhu
dhammesu dhammānupassī viharati chasu ajjhattika-bāhiresu āyatanesu.

\instr{Āyatanapabbaṃ niṭṭhitaṃ.}

\section*{Bojjhaṅgapabba}

Puna caparaṃ, bhikkhave, bhikkhu dhammesu dhammānupassī viharati sattasu
bojjhaṅgesu.

Kathañca pana, bhikkhave, bhikkhu dhammesu dhammānupassī viharati sattasu
bojjhaṅgesu?

\englishPage

Here, bhikkhus, when there is the mindfulness enlightenment factor in him, a
bhikkhu understands: `There is the mindfulness enlightenment factor in me'; or
when there is no mindfulness enlightenment factor in him, he understands: `There
is no mindfulness enlightenment factor in me'; and he also understands how the
unarisen mindfulness enlightenment factor arises, and how the arisen mindfulness
enlightenment factor comes to fulfillment by deveopment.

When there is the discrimination of phenomena enlightenment factor in him, a
bhikkhu understands: `There is the discrimination of phenomena enlightenment
factor in me'; or when there is no discrimination of phenomena enlightenment
factor in him, he understands: `There is no discrimination of phenomena
enlightenment factor in me'; and he also understands how the unarisen
discrimination of phenomena enlightenment factor arises, and how the arisen
discrimination of phenomena enlightenment factor comes to fulfillment by
deveopment.

When there is the energy enlightenment factor in him, a bhikkhu understands:
`There is the energy enlightenment factor in me'; or when there is no energy
enlightenment factor in him, he understands: `There is no energy enlightenment
factor in me'; and he also understands how the unarisen energy enlightenment
factor arises, and how the arisen energy enlightenment factor comes to
fulfillment by deveopment.

When there is the rapture enlightenment factor in him, a bhikkhu understands:
`There is the rapture enlightenment factor in me'; or when there is no rapture
enlightenment factor in him, he understands: `There is no rapture enlightenment
factor in me'; and he also understands how the unarisen rapture enlightenment
factor arises, and how the arisen rapture enlightenment factor comes to
fulfillment by deveopment.

\paliPage

Idha, bhikkhave, bhikkhu
santaṃ vā ajjhattaṃ sati-sambojjhaṅgaṃ ‘atthi me ajjhattaṃ sati-sambojjhaṅgo’ti pajānāti,
asantaṃ vā ajjhattaṃ sati-sambojjhaṅgaṃ ‘natthi me ajjhattaṃ sati-sambojjhaṅgo’ti pajānāti,
yathā ca anuppannassa sati-sambojjhaṅgassa uppādo hoti tañca pajānāti,
yathā ca uppannassa sati-sambojjhaṅgassa bhāvanāya pāripūrī hoti tañca pajānāti.

Santaṃ vā ajjhattaṃ dhammavicaya-sambojjhaṅgaṃ ‘atthi me ajjhattaṃ dhammavicaya-sambojjhaṅgo’ti pajānāti,
asantaṃ vā ajjhattaṃ dhammavicaya-sambojjhaṅgaṃ ‘natthi me ajjhattaṃ dhammavicaya-sambojjhaṅgo’ti pajānāti,
yathā ca anuppannassa dhammavicaya-sambojjhaṅgassa uppādo hoti tañca pajānāti,
yathā ca uppannassa dhammavicaya-sambojjhaṅgassa bhāvanāya pāripūrī hoti tañca pajānāti.

Santaṃ vā ajjhattaṃ viriya-sambojjhaṅgaṃ ‘atthi me ajjhattaṃ viriya-sambojjhaṅgo’ti pajānāti,
asantaṃ vā ajjhattaṃ viriya-sambojjhaṅgaṃ ‘natthi me ajjhattaṃ viriya-sambojjhaṅgo’ti pajānāti,
yathā ca anuppannassa viriya-sambojjhaṅgassa uppādo hoti tañca pajānāti,
yathā ca uppannassa viriya-sambojjhaṅgassa bhāvanāya pāripūrī hoti tañca pajānāti.

Santaṃ vā ajjhattaṃ pīti-sambojjhaṅgaṃ ‘atthi me ajjhattaṃ pīti-sambojjhaṅgo’ti pajānāti,
asantaṃ vā ajjhattaṃ pīti-sambojjhaṅgaṃ ‘natthi me ajjhattaṃ pīti-sambojjhaṅgo’ti pajānāti,
yathā ca anuppannassa pīti-sambojjhaṅgassa uppādo hoti tañca pajānāti,
yathā ca uppannassa pīti-sambojjhaṅgassa bhāvanāya pāripūrī hoti tañca pajānāti.

\englishPage

When there is the tranquillity enlightenment factor in him, a bhikkhu
understands: `There is the tranquillity enlightenment factor in me'; or when
there is no tranquillity enlightenment factor in him, he understands: `There is
no tranquillity enlightenment factor in me'; and he also understands how the
unarisen tranquillity enlightenment factor arises, and how the arisen
tranquillity enlightenment factor comes to fulfillment by deveopment.

When there is the concentration enlightenment factor in him, a bhikkhu
understands: `There is the concentration enlightenment factor in me'; or when
there is no concentration enlightenment factor in him, he understands: `There is
no concentration enlightenment factor in me'; and he also understands how the
unarisen concentration enlightenment factor arises, and how the arisen
concentration enlightenment factor comes to fulfillment by deveopment.

When there is the equanimity enlightenment factor in him, a bhikkhu understands:
`There is the equanimity enlightenment factor in me'; or when there is no
equanimity enlightenment factor in him, he understands: `There is no equanimity
enlightenment factor in me'; and he also understands how the unarisen equanimity
enlightenment factor arises, and how the arisen equanimity enlightenment factor
comes to fulfillment by deveopment.

\enlargethispage{2\baselineskip}

In this way he dwells contemplating phenomena in phenomena \ldots{} or else
mindfulness that ‘there are phenomena’ is simply established in him to the
extent necessary for bare knowledge and repeated mindfulness.

And he dwells independent, not clinging to anything in the world. That is how,
bhikkhus, a bhikkhu dwells contemplating phenomena in phenomena in terms of the
seven enlightenment factors.

\instr{The section on the seven enlightenment factors is finished.}

\paliPage

Santaṃ vā ajjhattaṃ passaddhi-sambojjhaṅgaṃ ‘atthi me ajjhattaṃ passaddhi-sambojjhaṅgo’ti pajānāti,
asantaṃ vā ajjhattaṃ passaddhi-sambojjhaṅgaṃ ‘natthi me ajjhattaṃ passaddhi-\\ sambojjhaṅgo’ti pajānāti,
yathā ca anuppannassa passaddhi-\\ sambojjhaṅgassa uppādo hoti tañca pajānāti,
yathā ca uppannassa passaddhi-sambojjhaṅgassa bhāvanāya pāripūrī hoti tañca pajānāti.

Santaṃ vā ajjhattaṃ samādhi-sambojjhaṅgaṃ ‘atthi me ajjhattaṃ samādhi-sambojjhaṅgo’ti pajānāti,
asantaṃ vā ajjhattaṃ samādhi-sambojjhaṅgaṃ ‘natthi me ajjhattaṃ samādhi-\\ sambojjhaṅgo’ti pajānāti,
yathā ca anuppannassa samādhi-\\ sambojjhaṅgassa uppādo hoti tañca pajānāti,
yathā ca uppannassa samādhi-sambojjhaṅgassa bhāvanāya pāripūrī hoti tañca pajānāti.

Santaṃ vā ajjhattaṃ upekkhā-sambojjhaṅgaṃ ‘atthi me ajjhattaṃ upekkhā-sambojjhaṅgo’ti pajānāti,
asantaṃ vā ajjhattaṃ upekkhā-sambojjhaṅgaṃ ‘natthi me ajjhattaṃ upekkhā-\\ sambojjhaṅgo’ti pajānāti,
yathā ca anuppannassa upekkhā-\\ sambojjhaṅgassa uppādo hoti tañca pajānāti,
yathā ca uppannassa upekkhā-sambojjhaṅgassa bhāvanāya pāripūrī hoti tañca pajānāti.

Iti ajjhattaṃ vā dhammesu dhammānupassī viharati,
bahiddhā vā dhammesu dhammānupassī viharati,
ajjhatta-bahiddhā vā dhammesu dhammānupassī viharati.
samudaya-dhammānupassī vā dhammesu viharati,
vaya-dhammānupassī vā dhammesu viharati,
samudaya-vaya-\\ dhammānupassī vā dhammesu viharati.
‘atthi dhammā’ti vā panassa sati paccupaṭṭhitā hoti
yāvadeva ñāṇamattāya paṭissatimattāya, anissito ca viharati,
na ca kiñci loke upādiyati. evampi kho, bhikkhave, bhikkhu
dhammesu dhammānupassī viharati sattasu bojjhaṅgesu.

\enlargethispage{\baselineskip}

\instr{Bojjhaṅgapabbaṃ niṭṭhitaṃ.}

\englishPage
\section{The Truths}

Again, bhikkhus, a bhikkhu dwells contemplating phenomena in phenomena in terms
of the Four Noble Truths.

And how, bhikkhus, does a bhikkhu dwell contemplating phenomena in phenomena in
terms of the Four Noble Truths?

Here, bhikkhus, a bhikkhu understands as it really is: `This is suffering. The
is the origin of suffering. This is the cessation of suffering. This is the way
leading to the cessation of suffering.'

\section{Suffering}

And what, bhikkhus, is the noble truth of suffering? Birth is suffering; ageing
is suffering; death is suffering; sorrow, lamentation, pain, grief, and despair
are suffering; union with what is displeasing is suffering; separation from what
is pleasing is suffering; not to get what one wants is suffering; in brief, the
five aggregates subject to clinging are suffering.

And what, bhikkhus, is birth? The birth of beings into the vaious orders of
beings, their coming to birth, precipitation [in a womb], generation, the
manifestation of the aggregates, obtaining the bases for contact -- this is
called birth.

And what, bhikkhus, is ageing? The ageing of beings in the various orders of
beings, their old age, brokenness of teeth, greyness of hair, wrinkling of skin,
deline of life, weakness of faculties -- this is called ageing.

\paliPage

\section*{Saccapabba}

Puna caparaṃ, bhikkhave, bhikkhu dhammesu dhammānupassī viharati catūsu
ariyasaccesu.

Kathañca pana, bhikkhave, bhikkhu dhammesu dhammānupassī viharati catūsu
ariyasaccesu?

Idha, bhikkhave, bhikkhu ‘idaṃ dukkhan’ti yathābhūtaṃ pajānāti, ‘ayaṃ
dukkha-samudayo’ti yathābhūtaṃ pajānāti, ‘ayaṃ dukkha-nirodho’ti yathābhūtaṃ
pajānāti, ‘ayaṃ dukkha-nirodha-gāminī paṭipadā’ti yathābhūtaṃ pajānāti.

\section*{Dukkha-sacca}

Katamañca, bhikkhave, dukkhaṃ ariyasaccaṃ? Jātipi dukkhā, jarāpi dukkhā,
maraṇampi dukkhaṃ, soka-parideva-dukkha-\\ domanass'upāyāsāpi dukkhā, appiyehi
sampayogo dukkho, piyehi vippayogo dukkho, yampicchaṃ na labhati tampi dukkhaṃ,
saṅkhittena pañcupādāna-kkhandhā dukkhā.

Katamā ca, bhikkhave, jāti? Yā tesaṃ tesaṃ sattānaṃ tamhi tamhi sattanikāye jāti
sañjāti okkanti abhinibbatti khandhānaṃ pātubhāvo āyatanānaṃ paṭilābho, ayaṃ
vuccati, bhikkhave, jāti.

Katamā ca, bhikkhave, jarā? Yā tesaṃ tesaṃ sattānaṃ tamhi tamhi sattanikāye jarā
jīraṇatā khaṇḍiccaṃ pāliccaṃ valittacatā āyuno saṃhāni indriyānaṃ paripāko, ayaṃ
vuccati, bhikkhave, jarā.

\englishPage

And what, bhikkhus, is death? The passing of beings out of the various orders of
beings, their passing away, dissolution, disappearance, dying, death, completion
of time, dissolution of aggregates, laying down of the body, the cutting off of
the life faculty -- this is called death.

And what, bhikkhus, is sorrow? Bhikkhus, for one who has encountered some
misfortune or is affected by some painful state, there is sorrow, sorrowing,
sorrowfulness, inner sorrow, inner sorriness -- this is called sorrow.

And what, bhikkhus, is lamentation? Bhikkhus, for one who has encountered some
misfortune or is affected by some painful state, there is wail and lament,
wailing and lamenting, bewailing and lamentation -- this is called lamentation.

And what, bhikkhus, is pain? That, bhikkhus, which is bodily pain, bodily
discomfort, painful, uncomfortable feeling born of bodily contact -- this is
called pain.

And what, bhikkhus, is grief? That, bhikkhus, which is mental pain, mental
discomfort, painful, uncomfortable feeling born of mental contact -- this is
called grief.

And what, bhikkhus, is despair? Bhikkhus, for one who has encountered some
misfortune or is affected by some painful state, there is the trouble and
despair, the tribulation and desperation -- this is called despair.

And what, bhikkhus, is `union with what is displeasing is suffering'? Here,
bhikkhus, for one who has undesireable, unlovely, disagreeable forms, sounds,
ordours, tastes, and tactile objects;

\paliPage

Katamañca, bhikkhave, maraṇaṃ? Yaṃ tesaṃ tesaṃ sattānaṃ tamhā tamhā sattanikāyā
cuti cavanatā bhedo antaradhānaṃ maccu maraṇaṃ kālakiriyā khandhānaṃ bhedo
kaḷevarassa nikkhepo jīvitindriyass'upacchedo, idaṃ vuccati, bhikkhave, maraṇaṃ.

Katamo ca, bhikkhave, soko? Yo kho, bhikkhave, aññata'raññatarena byasanena
samannāgatassa aññata'raññatarena dukkha-dhammena phuṭṭhassa soko socanā
socitattaṃ antosoko antoparisoko, ayaṃ vuccati, bhikkhave, soko.

Katamo ca, bhikkhave, paridevo? Yo kho, bhikkhave, aññata'raññatarena
byasanena samannāgatassa aññata'raññatarena dukkha-dhammena phuṭṭhassa
ādevo paridevo ādevanā paridevanā ādevitattaṃ paridevitattaṃ, ayaṃ
vuccati, bhikkhave, paridevo.

Katamañca, bhikkhave, dukkhaṃ? Yaṃ kho, bhikkhave, kāyikaṃ dukkhaṃ kāyikaṃ
asātaṃ kāya-samphassajaṃ dukkhaṃ asātaṃ vedayitaṃ, idaṃ vuccati, bhikkhave,
dukkhaṃ.

Katamañca, bhikkhave, domanassaṃ? Yaṃ kho, bhikkhave, cetasikaṃ dukkhaṃ
cetasikaṃ asātaṃ mano-samphassajaṃ dukkhaṃ asātaṃ vedayitaṃ, idaṃ vuccati,
bhikkhave, domanassaṃ.

Katamo ca, bhikkhave, upāyāso? Yo kho, bhikkhave, aññata'raññatarena byasanena
samannāgatassa aññata'raññatarena dukkha-dhammena phuṭṭhassa āyāso upāyāso
āyāsitattaṃ upāyāsitattaṃ, ayaṃ vuccati, bhikkhave, upāyāso.

Katamo ca, bhikkhave, appiyehi sampayogo dukkho? Idha yassa te honti aniṭṭhā
akantā amanāpā rūpā saddā gandhā rasā phoṭṭhabbā dhammā,

\englishPage

or for one who has those who do not desire his welfare, his benefit, his
comfort, and his security -- (and then) having meetings, assembly, connection,
and mixing with them: this, bhikkhus, is called `union with what is displeasing
is suffering'.

And what, bhikkhus, is `separation from what is pleasing is suffering'? Here,
bhikkhus, for one who has desirable, lovely, agreeable forms, sounds, ordours,
tastes, and tactile objects; or, for one who has those who do desire his
welfare, his benefit, his comfort and his security -- mothers, fathers,
brothers, or sisters; friends, companions, or blood relatives -- (and then) not
having meetings, assembly, connection, and mixing with them: this, bhikkhus, is
called `separation from what is pleasing is suffering'?

And what, bhikkhus, is `not to get what one wants is suffeing'? To beings
subject to birth there comes the wish: `Oh, that we were not subject to birth!
That birth would not come to us!' But this is not to be obtained by wishing;
this is `not to get what one wants is suffering.'

To beings subject to ageing there comes the wish: `Oh, that we were not subject
to ageing! That ageing would not come to us!' But this is not to be obtained by
wishing; this is `not to get what one wants is suffering.'

To beings subject to sickness there comes the wish: `Oh, that we were not
subject to sickness! That sickness would not come to us!' But this is not to be
obtained by wishing; this is `not to get what one wants is suffering.'

To beings subject to death there comes the wish: `Oh, that we were not subject
to death! That death would not come to us!' But this is not to be obtained by
wishing; this is `not to get what one wants is suffering.'

\paliPage

ye vā panassa te honti anatthakāmā ahitakāmā aphāsukakāmā ayogakkhemakāmā, yā
tehi saddhiṃ saṅgati samāgamo samodhānaṃ missībhāvo, ayaṃ vuccati, bhikkhave,
appiyehi sampayogo dukkho.

Katamo ca, bhikkhave, piyehi vippayogo dukkho? Idha yassa te honti iṭṭhā kantā
manāpā rūpā saddā gandhā rasā phoṭṭhabbā dhammā, ye vā panassa te honti
atthakāmā hitakāmā phāsukakāmā yogakkhemakāmā mātā vā pitā vā bhātā vā bhaginī
vā mittā vā amaccā vā ñātisālohitā vā, yā tehi saddhiṃ asaṅgati asamāgamo
asamodhānaṃ amissībhāvo, ayaṃ vuccati, bhikkhave, piyehi vippayogo dukkho.

Katamañca, bhikkhave, yampicchaṃ na labhati tampi dukkhaṃ? Jātidhammānaṃ,
bhikkhave, sattānaṃ evaṃ icchā uppajjati: ‘aho vata mayaṃ na jātidhammā assāma,
na ca vata no jāti āgaccheyyā’ti. Na kho panetaṃ icchāya pattabbaṃ, idampi
yampicchaṃ na labhati tampi dukkhaṃ.

Jarādhammānaṃ, bhikkhave, sattānaṃ evaṃ icchā uppajjati: ‘aho vata mayaṃ na
jarādhammā assāma, na ca vata no jarā āgaccheyyā’ti. Na kho panetaṃ icchāya
pattabbaṃ, idampi yampicchaṃ na labhati tampi dukkhaṃ.

Byādhidhammānaṃ, bhikkhave, sattānaṃ evaṃ icchā uppajjati ‘aho vata mayaṃ na
byādhidhammā assāma, na ca vata no byādhi āgaccheyyā’ti. Na kho panetaṃ icchāya
pattabbaṃ, idampi yampicchaṃ na labhati tampi dukkhaṃ.

Maraṇadhammānaṃ, bhikkhave, sattānaṃ evaṃ icchā uppajjati ‘aho vata mayaṃ na
maraṇadhammā assāma, na ca vata no maraṇaṃ āgaccheyyā’ti. Na kho panetaṃ icchāya
pattabbaṃ, idampi yampicchaṃ na labhati tampi dukkhaṃ.

\englishPage

To beings subject to sorrow, lamentation, pain, grief, and despair, there comes
the wish: `Oh, that we were not subject to sorrow, lamentation, pain, grief, and
despair! That sorrow, lamentation, pain, grief, and despair would not come to
us!' But this is not to be obtained by wishing; this is `not to get what one
wants is suffering.'

And what, bhikkhus, are the five aggregates subject to clinging that, in brief,
are suffering? They are: the material form aggregate subject to clinging, the
feeling aggregate subject to clinging, the perception aggregate subject to
clinging, the volitional formations aggregate subject to clinging, the
consciousness aggregate subject to clinging. These are the five aggregates
subject to clinging that, in brief, are suffering.

This, bhikkhus, is called the noble truth of suffering.

\section{Origin}

And what, bhikkhus, is the noble truth of the origin of suffering? It is this
craving, which leads to renewed existence, accompanied by delight and lust,
seeking delight in this and that; that is, craving for sensual pleasures,
craving for existence, and craving for non-existence.

Now where, bhikkhus, does that craving when it is arising arise? When settling
where does it settle? That in the world which is pleasant and agreeable -- here
this craving when it is arising arises, here when settling it settles.

In the world what is pleasant and agreeable? In the world the eye \ldots{} the
ear \ldots{} the nose \ldots{} the tongue \ldots{} the body \ldots{} the mind is
likeable and pleasing -- here this craving when it is arising arises, here when
settling it settles.

In the world the forms \ldots{} the sounds \ldots{} the ordours \ldots{} the
tastes \ldots{} the tactile objects \ldots{} the mental phenomena is likeable
and pleasing -- here this craving when it is arising arises, here when settling
it settles.

\paliPage

Soka-parideva-dukkha-domanass'upāyāsa-dhammānaṃ, bhikkhave, sattānaṃ evaṃ icchā
uppajjati ‘aho vata mayaṃ na soka-parideva-\\
dukkha-domanass'upāyāsa-dhammā assāma, na ca vata no soka-\\
parideva-dukkha-domanass'upāyāsā āgaccheyyun’ti. Na kho panetaṃ icchāya
pattabbaṃ, idampi yampicchaṃ na labhati tampi dukkhaṃ.

Katame ca, bhikkhave, saṅkhittena pañcupādāna-kkhandhā dukkhā? Seyyathīdaṃ,
rūpūpādāna-kkhandho, vedanūpādāna-kkhandho, saññūpādāna-kkhandho,
saṅkhārūpādāna-kkhandho, viññāṇūpādāna-kkhandho. ime vuccanti, bhikkhave,
saṅkhittena pañcupādāna-kkhandhā dukkhā.

Idaṃ vuccati, bhikkhave, dukkhaṃ ariyasaccaṃ.

\section*{Samudaya-sacca}

Katamañca, bhikkhave, dukkha-samudayaṃ ariyasaccaṃ? Yāyaṃ taṇhā ponobbhavikā
nandi-rāga-sahagatā tatra-tatrābhinandinī, seyyathīdaṃ, kāmataṇhā bhavataṇhā
vibhavataṇhā.

Sā kho panesā, bhikkhave, taṇhā kattha uppajjamānā uppajjati, kattha nivīsamānā
nivīsati? Yaṃ loke piyarūpaṃ sātarūpaṃ, etthesā taṇhā uppajjamānā uppajjati,
ettha nivīsamānā nivīsati.

\enlargethispage{\baselineskip}

Kiñca loke piyarūpaṃ sātarūpaṃ? Cakkhu loke piyarūpaṃ sātarūpaṃ, etthesā taṇhā
uppajjamānā uppajjati, ettha nivīsamānā nivīsati. Sotaṃ loke
\ldots{} ghānaṃ loke \ldots{} jivhā loke \ldots{} kāyo loke \ldots{} mano loke
piyarūpaṃ sātarūpaṃ, etthesā taṇhā uppajjamānā uppajjati, ettha nivīsamānā
nivīsati.

Rūpā loke \ldots{} saddā loke \ldots{} gandhā loke \ldots{} rasā loke \ldots{}
phoṭṭhabbā loke \ldots{} dhammā loke piyarūpaṃ sātarūpaṃ, etthesā taṇhā
uppajjamānā uppajjati, ettha nivīsamānā nivīsati.

\englishPage

In the world the eye-consciousness \ldots{} ear-consciousness \ldots{} nose-consciousness
 \ldots{} tongue-consciousness \ldots{} body-consciousness \ldots{} mind-consciousness is likeable
and pleasing -- here this craving when it is arising arises, here when settling
it settles.

In the world the eye-contact \ldots{} ear-contact \ldots{} nose-contact \ldots{}
tongue-contact \ldots{} body-contact \ldots{} mind-contact is likeable and
pleasing -- here this craving when it is arising arises, here when settling it
settles.

In the world feeling born of eye-contact \ldots{} feeling born of ear-contact
\ldots{} feeling born of nose-contact \ldots{} feeling born of tongue-contact
\ldots{} feeling born of body-contact \ldots{} feeling born of mind-contact is
likeable and pleasing -- here this craving when it is arising arises, here when
settling it settles.

In the world perception of forms \ldots{} perception of sounds \ldots{}
perception of odours \ldots{} perception of tastes \ldots{} perception of
tactile objects \ldots{} perception of mental phenomena is likeable and pleasing
-- here this craving when it is arising arises, here when settling it settles.

In the world volition regarding forms \ldots{} volition regarding sounds
\ldots{} volition regarding odours \ldots{} volition regarding tastes \ldots{}
volition regarding tactile objects \ldots{} volition regarding mental phenomena
is likeable and pleasing -- here this craving when it is arising arises, here
when settling it settles.

In the world craving for forms \ldots{} craving for sounds \ldots{} craving for
odours \ldots{} craving for tastes \ldots{} craving for tactile objects \ldots{}
craving for mental phenomena is likeable and pleasing -- here this craving when
it is arising arises, here when settling it settles.

\paliPage

Cakkhu-viññāṇaṃ loke \ldots{} sota-viññāṇaṃ loke \ldots{} ghāna-viññāṇaṃ loke
\ldots{} jivhā-viññāṇaṃ loke \ldots{} kāya-viññāṇaṃ loke \ldots{} mano-viññāṇaṃ
loke piyarūpaṃ sātarūpaṃ, etthesā taṇhā uppajjamānā uppajjati, ettha nivīsamānā
nivīsati.

Cakkhu-samphasso loke \ldots{} sota-samphasso loke \ldots{} ghāna-samphasso loke
\ldots{} jivhā-samphasso loke \ldots{} kāya-samphasso loke \ldots{} mano-samphasso
loke piyarūpaṃ sātarūpaṃ, etthesā taṇhā uppajjamānā uppajjati, ettha nivīsamānā
nivīsati.

Cakkhu-samphassajā vedanā loke \ldots{} sota-samphassajā vedanā loke \ldots{}
ghāna-samphassajā vedanā loke \ldots{} jivhā-samphassajā vedanā loke \ldots{}
kāya-samphassajā vedanā loke \ldots{} mano-samphassajā vedanā loke piyarūpaṃ
sātarūpaṃ, etthesā taṇhā uppajjamānā uppajjati, ettha nivīsamānā nivīsati.

Rūpa-saññā loke \ldots{} sadda-saññā loke \ldots{} gandha-saññā loke \ldots{}
rasa-saññā loke \ldots{} phoṭṭhabba-saññā loke \ldots{} dhamma-saññā loke piyarūpaṃ
sātarūpaṃ, etthesā taṇhā uppajjamānā uppajjati, ettha nivīsamānā nivīsati.

Rūpa-sañcetanā loke \ldots{} sadda-sañcetanā loke \ldots{} gandha-sañcetanā loke
\ldots{} rasa-sañcetanā loke \ldots{} phoṭṭhabba-sañcetanā loke \ldots{}
dhamma-sañcetanā loke piyarūpaṃ sātarūpaṃ, etthesā taṇhā uppajjamānā uppajjati,
ettha nivīsamānā nivīsati.

Rūpa-taṇhā loke \ldots{} sadda-taṇhā loke \ldots{} gandha-taṇhā loke \ldots{}
rasa-taṇhā loke \ldots{} phoṭṭhabba-taṇhā loke \ldots{} dhamma-taṇhā loke piyarūpaṃ
sātarūpaṃ, etthesā taṇhā uppajjamānā uppajjati, ettha nivīsamānā nivīsati.

\englishPage

In the world thought about forms \ldots{} thought about sounds \ldots{} thought
about odours \ldots{} thought about tastes \ldots{} thought about tactile
objects \ldots{} thought about mental phenomena is likeable and pleasing -- here
this craving when it is arising arises, here when settling it settles.

In the world examination of forms \ldots{} examination of sounds \ldots{}
examination of odours \ldots{} examination of tastes \ldots{} examination of
tactile objects \ldots{} examination of mental phenomena is likeable and
pleasing -- here this craving when it is arising arises, here when settling it
settles.

This, bhikkhus, is called the noble truth of the origin of suffering.

\section{Cessation}

And what, bhikkhus, is the noble truth of the cessation of suffering? It is the
remainderless fading away and cessation of that same craving, the giving up and
relinquishing of it, freedom from it, non-reliance on it.

Now where, bhikkhus, is that craving when it is being abandoned, abandoned? When
ceasing where does it cease? That in the world which is pleasant and agreeable
-- here this craving when it is being abandoned, abandoned, here when ceasing it
ceases.

In the world what is pleasant and agreeable? In the world the eye \ldots{} the
ear \ldots{} the nose \ldots{} the tongue \ldots{} the body \ldots{} the mind is
likeable and pleasing -- here this craving when it is being abandoned, abandoned,
here when ceasing it ceases.

In the world the forms \ldots{} the sounds \ldots{} the ordours \ldots{} the
tastes \ldots{} the tactile objects \ldots{} the mental phenomena is likeable
and pleasing -- here this craving when it is being abandoned, abandoned, here
when ceasing it ceases.

\paliPage

Rūpa-vitakko loke \ldots{} sadda-vitakko loke \ldots{} gandha-vitakko loke \ldots{}
rasa-vitakko loke \ldots{} phoṭṭhabba-vitakko loke \ldots{} dhamma-vitakko loke
piyarūpaṃ sātarūpaṃ, etthesā taṇhā uppajjamānā uppajjati, ettha nivīsamānā
nivīsati.

Rūpa-vicāro loke \ldots{} sadda-vicāro loke \ldots{} gandha-vicāro loke \ldots{}
rasa-vicāro loke \ldots{} phoṭṭhabba-vicāro loke \ldots{} dhamma-vicāro loke
piyarūpaṃ sātarūpaṃ, etthesā taṇhā uppajjamānā uppajjati, ettha nivīsamānā
nivīsati.

Idaṃ vuccati, bhikkhave, dukkha-samudayaṃ ariyasaccaṃ.

\section*{Nirodha-sacca}

Katamañca, bhikkhave, dukkha-nirodhaṃ ariyasaccaṃ? Yo tassāyeva taṇhāya
asesa-virāga-nirodho cāgo paṭinissaggo mutti anālayo.

Sā kho panesā, bhikkhave, taṇhā kattha pahīyamānā pahīyati, kattha nirujjhamānā
nirujjhati? Yaṃ loke piyarūpaṃ sātarūpaṃ, etthesā taṇhā pahīyamānā pahīyati,
ettha nirujjhamānā nirujjhati.

Kiñca loke piyarūpaṃ sātarūpaṃ? Cakkhu loke piyarūpaṃ sātarūpaṃ, etthesā taṇhā
pahīyamānā pahīyati, ettha nirujjhamānā nirujjhati. Sotaṃ loke \ldots{}
\ldots{} ghānaṃ loke \ldots{} jivhā loke \ldots{} kāyo loke \ldots{} mano loke
piyarūpaṃ sātarūpaṃ, etthesā taṇhā pahīyamānā pahīyati, ettha nirujjhamānā
nirujjhati.

Rūpā loke \ldots{} saddā loke \ldots{} gandhā loke \ldots{} rasā loke \ldots{}
phoṭṭhabbā loke \ldots{} dhammā loke piyarūpaṃ sātarūpaṃ, etthesā taṇhā
pahīyamānā pahīyati, ettha nirujjhamānā nirujjhati.

\englishPage

In the world the eye-consciousness \ldots{} ear-consciousness \ldots{}
nose-consciousness \ldots{} tongue-consciousness \ldots{} body-consciousness
\ldots{} mind-consciousness is likeable and pleasing -- here this craving when
it is being abandoned, abandoned, here when ceasing it ceases.

In the world the eye-contact \ldots{} ear-contact \ldots{} nose-contact \ldots{}
tongue-contact \ldots{} body-contact \ldots{} mind-contact is likeable and
pleasing -- here this craving when it is being abandoned, abandoned, here when
ceasing it ceases.

In the world feeling born of eye-contact \ldots{} feeling born of ear-contact
\ldots{} feeling born of nose-contact \ldots{} feeling born of tongue-contact
\ldots{} feeling born of body-contact \ldots{} feeling born of mind-contact is
likeable and pleasing -- here this craving when it is being abandoned, abandoned,
here when ceasing it ceases.

In the world perception of forms \ldots{} perception of sounds \ldots{}
perception of odours \ldots{} perception of tastes \ldots{} perception of
tactile objects \ldots{} perception of mental phenomena is likeable and pleasing
-- here this craving when it is being abandoned, abandoned, here when ceasing it
ceases.

In the world volition regarding forms \ldots{} volition regarding sounds
\ldots{} volition regarding odours \ldots{} volition regarding tastes \ldots{}
volition regarding tactile objects \ldots{} volition regarding mental phenomena
is likeable and pleasing -- here this craving when it is being abandoned,
abandoned, here when ceasing it ceases.

In the world craving for forms \ldots{} craving for sounds \ldots{} craving for
odours \ldots{} craving for tastes \ldots{} craving for tactile objects \ldots{}
craving for mental phenomena is likeable and pleasing -- here this craving when
it is being abandoned, abandoned, here when ceasing it ceases.

\paliPage

Cakkhu-viññāṇaṃ loke \ldots{} sota-viññāṇaṃ loke \ldots{} ghāna-viññāṇaṃ loke
\ldots{} jivhā-viññāṇaṃ loke \ldots{} kāya-viññāṇaṃ loke \ldots{} mano-viññāṇaṃ
loke piyarūpaṃ sātarūpaṃ, etthesā taṇhā pahīyamānā pahīyati, ettha nirujjhamānā
nirujjhati.

Cakkhu-samphasso loke \ldots{} sota-samphasso loke \ldots{} ghāna-samphasso loke
\ldots{} jivhā-samphasso loke \ldots{} kāya-samphasso loke \ldots{} mano-samphasso
loke piyarūpaṃ sātarūpaṃ, etthesā taṇhā pahīyamānā pahīyati, ettha nirujjhamānā
nirujjhati.

Cakkhu-samphassajā vedanā loke \ldots{} sota-samphassajā vedanā loke \ldots{}
ghāna-samphassajā vedanā loke \ldots{} jivhā-samphassajā vedanā loke \ldots{}
kāya-samphassajā vedanā loke \ldots{} mano-samphassajā vedanā loke piyarūpaṃ
sātarūpaṃ, etthesā taṇhā pahīyamānā pahīyati, ettha nirujjhamānā nirujjhati.

Rūpa-saññā loke \ldots{} sadda-saññā loke \ldots{} gandha-saññā loke \ldots{}
rasa-saññā loke \ldots{} phoṭṭhabba-saññā loke \ldots{} dhamma-saññā loke piyarūpaṃ
sātarūpaṃ, etthesā taṇhā pahīyamānā pahīyati, ettha nirujjhamānā nirujjhati.

Rūpa-sañcetanā loke \ldots{} sadda-sañcetanā loke \ldots{} gandha-sañcetanā loke
\ldots{} rasa-sañcetanā loke \ldots{} phoṭṭhabba-sañcetanā loke \ldots{}
dhamma-sañcetanā loke piyarūpaṃ sātarūpaṃ, etthesā taṇhā pahīyamānā pahīyati,
ettha nirujjhamānā nirujjhati.

Rūpa-taṇhā loke \ldots{} sadda-taṇhā loke \ldots{} gandha-taṇhā loke \ldots{}
rasa-taṇhā loke \ldots{} phoṭṭhabba-taṇhā loke \ldots{} dhamma-taṇhā loke piyarūpaṃ
sātarūpaṃ, etthesā taṇhā pahīyamānā pahīyati, ettha nirujjhamānā nirujjhati.

\englishPage

In the world thought about forms \ldots{} thought about sounds \ldots{} thought
about odours \ldots{} thought about tastes \ldots{} thought about tactile
objects \ldots{} thought about mental phenomena is likeable and pleasing -- here
this craving when it is being abandoned, abandoned, here when ceasing it ceases.

In the world examination of forms \ldots{} examination of sounds \ldots{}
examination of odours \ldots{} examination of tastes \ldots{} examination of
tactile objects \ldots{} examination of mental phenomena is likeable and
pleasing -- here this craving when it is being abandoned, abandoned, here when
ceasing it ceases.

This, bhikkhus, is called the noble truth of the cessation of suffering.

\section{The Way}

And what, bhikkhus, is the noble truth of the way leading to the cessation of
suffering? It is this Noble Eightfold Path; that is right view, right intention,
right speech, right action, right livelihood, right effort, right mindfulness,
right concentration.

And what, bhikkhus, is right view? Bhikkhus, the knowledge of suffering,
knowledge of the origin of suffering, knowledge of the cessation of suffering,
knowledge of the way leading to the cessation of suffering: this, bhikkhus, is
called right view.

And what, bhikkhus, is right intention? Intention of renunciation, intention of
non-ill will, intention of harmlessness: this, bhikkhus, is called right
intention.

And what, bhikkhus, is right speech? Abstinence from false speech, abstinence
from divisive sppech, abstinence from harsh speech, abstinence from idle
chatter: this, bhikkhus, is called right speech.

\paliPage

Rūpa-vitakko loke \ldots{} sadda-vitakko loke \ldots{} gandha-vitakko loke \ldots{}
rasa-vitakko loke \ldots{} phoṭṭhabba-vitakko loke \ldots{} dhamma-vitakko loke
piyarūpaṃ sātarūpaṃ, etthesā taṇhā pahīyamānā pahīyati, ettha nirujjhamānā
nirujjhati.

Rūpa-vicāro loke \ldots{} sadda-vicāro loke \ldots{} gandha-vicāro loke \ldots{}
rasa-vicāro loke \ldots{} phoṭṭhabba-vicāro loke \ldots{} dhamma-vicāro loke
piyarūpaṃ sātarūpaṃ, etthesā taṇhā pahīyamānā pahīyati, ettha nirujjhamānā
nirujjhati.

Idaṃ vuccati, bhikkhave, dukkhanirodhaṃ ariyasaccaṃ.

\section*{Magga-sacca}

Katamañca, bhikkhave, dukkha-nirodha-gāminī paṭipadā ariyasaccaṃ? ayameva ariyo
aṭṭhaṅgiko maggo seyyathīdaṃ, sammā-diṭṭhi sammā-saṅkappo sammā-vācā
sammā-kammanto sammā-ājīvo sammā-vāyāmo sammā-sati sammā-samādhi.

Katamā ca, bhikkhave, sammā-diṭṭhi? yaṃ kho, bhikkhave, dukkhe ñāṇaṃ,
dukkha-samudaye ñāṇaṃ, dukkha-nirodhe ñāṇaṃ, dukkha-nirodha-gāminiyā paṭipadāya
ñāṇaṃ. ayaṃ vuccati, bhikkhave, sammā-diṭṭhi.

Katamo ca, bhikkhave, sammā-saṅkappo? nekkhamma-saṅkappo abyāpāda-saṅkappo
avihiṃsā-saṅkappo. ayaṃ vuccati, bhikkhave, sammā-saṅkappo.

Katamā ca, bhikkhave, sammā-vācā? musāvādā veramaṇī pisuṇāya vācāya veramaṇī
pharusāya vācāya veramaṇī samphappalāpā veramaṇī. ayaṃ vuccati, bhikkhave,
sammā-vācā.

\englishPage

And what, bhikkhus, is right action? Abstinence from the destruction of life,
abstinence from taking what is not given, abstinence from sexual misconduct:
this, bhikkhus, is called right action.

And what, bhikkhus, is right livelihood? Here, bhikkhus, a noble disciple,
having abandoned a wrong mode of livelihood, earns his living by a right
livelihood: this, bhikkhus, is called right livelihood.

And what, bhikkhus, is right effort? Here, bhikkhus, a bhikkhu, for the
nonarising of unarisen evil unwholesome states; he generates desire, makes an
effort, arouses energy, applies his mind, and strives. For the abandoning of
arisen evil unwholesome states; he generates desire, makes an effort, arouses
energy, applies his mind, and strives. For the arising of unarisen wholesome
states; he generates desire, makes an effort, arouses energy, applies his mind,
and strives. For the maintenance of arisen wholesome states, for their nondecay,
increase, expansion, and fulfilment by development; he generates desire, makes
an effort, arouses energy, applies his mind, and strives. This, bhikkhus, is
called right effort.

And what, bhikkhus, is right mindfulness? Here, bhikkhus, a bhikkhu dwells
contemplating the body in the body, ardent, clearly comprehending, and mindful,
having subdued longing and dejection in regard to the world.

He dwells contemplating feelings in feelings, ardent, clearly comprehending, and
mindful, having subdued longing and dejection in regard to the world.

He dwells contemplating mind in mind, ardent, clearly comprehending, and
mindful, having subdued longing and dejection in regard to the world.

He dwells contemplating phenomena in phenomena, ardent, clearly comprehending,
and mindful, having subdued longing and dejection in regard to the world. This,
bhikkhus, is called right mindfulness.

\paliPage

Katamo ca, bhikkhave, sammā-kammanto? pāṇātipātā veramaṇī adinnādānā veramaṇī
kāmesu-micchācārā veramaṇī. ayaṃ vuccati, bhikkhave, sammā-kammanto.

Katamo ca, bhikkhave, sammā-ājīvo? idha, bhikkhave, ariyasāvako micchā-ājīvaṃ
pahāya sammā-ājīvena jīvitaṃ kappeti. ayaṃ vuccati, bhikkhave, sammā-ājīvo.

Katamo ca, bhikkhave, sammā-vāyāmo? Idha, bhikkhave, bhikkhu
anuppannānaṃ pāpakānaṃ akusalānaṃ dhammānaṃ
anuppādāya chandaṃ janeti vāyamati viriyaṃ ārabhati cittaṃ paggaṇhāti padahati;

uppannānaṃ pāpakānaṃ akusalānaṃ dhammānaṃ
pahānāya chandaṃ janeti vāyamati viriyaṃ ārabhati cittaṃ paggaṇhāti padahati;

anuppannānaṃ kusalānaṃ dhammānaṃ
uppādāya chandaṃ janeti vāyamati viriyaṃ ārabhati cittaṃ paggaṇhāti padahati;

uppannānaṃ kusalānaṃ dhammānaṃ
ṭhitiyā asammosāya bhiyyobhāvāya vepullāya
bhāvanāya pāripūriyā chandaṃ janeti vāyamati viriyaṃ ārabhati cittaṃ paggaṇhāti
padahati. ayaṃ vuccati, bhikkhave, sammā-vāyāmo.

Katamā ca, bhikkhave, sammā-sati? Idha, bhikkhave, bhikkhu kāye kāyānupassī
viharati ātāpī sampajāno satimā vineyya loke abhijjhā-domanassaṃ;

vedanāsu vedanānupassī viharati ātāpī sampajāno satimā vineyya loke
abhijjhā-domanassaṃ;

\enlargethispage{\baselineskip}

citte cittānupassī viharati ātāpī sampajāno satimā vineyya loke
abhijjhā-domanassaṃ;

dhammesu dhammānupassī viharati ātāpī sampajāno satimā vineyya loke
abhijjhā-domanassaṃ. ayaṃ vuccati, bhikkhave, sammā-sati.

\englishPage

And what, bhikkhus, is right concentration? Here, bhikkhus, secluded from
sensual pleasures, secluded from unwholesome states, accompanied by thought and
examination, with rapture and happiness born of seclusion, a bhikkhu enters and
dwells in the first jhāna.

With the subsiding of thought and examination, with internal confidence and
unification of mind, being without thought and examination, having the rapture
and happiness born of concentration, he enters and dwells in the second jhāna.

With the fading away as well of rapture, he dwells equanimous and, mindful and
clearly comprehending, he experiences happiness with the body; that which the
noble ones declare: `He is equanimous, mindul, one who dwells happily', he
enters and dwells in the third jhāna.

With the abandoning of pleasure, with the abandoning of pain, with the previous
passing away of joy and displeasure, which is neither painful nor pleasant and
includes the purification of mindfulness by equanimity, he enters and dwells in
the fourth jhāna. This, bhikkhus, is called right concentration.

This, bhikkhus, is called the noble truth of the way leading to the cessation of
suffering.

In this way he dwells contemplating phenomena in phenomena \ldots{} or else
mindfulness that ‘there are phenomena’ is simply established in him to the
extent necessary for bare knowledge and repeated mindfulness.

And he dwells independent, not clinging to anything in the world. That is how,
bhikkhus, a bhikkhu dwells contemplating phenomena in phenomena in terms of the
Four Noble Truths.

\instr{The section on Truths is finished.}

\instr{The Contemplation of Phenomena is finished.}

\paliPage

Katamo ca, bhikkhave, sammā-samādhi? Idha, bhikkhave, bhikkhu vivicceva kāmehi
vivicca akusalehi dhammehi savitakkaṃ savicāraṃ vivekajaṃ pītisukhaṃ paṭhamaṃ
jhānaṃ upasampajja viharati.

Vitakka-vicārānaṃ vūpasamā ajjhattaṃ sampasādanaṃ cetaso ekodibhāvaṃ avitakkaṃ
avicāraṃ samādhijaṃ pītisukhaṃ dutiyaṃ jhānaṃ upasampajja viharati.

Pītiyā ca virāgā upekkhako ca viharati, sato ca sampajāno, sukhañca kāyena
paṭisaṃvedeti, yaṃ taṃ ariyā ācikkhanti ‘upekkhako satimā sukhavihārī’ti tatiyaṃ
jhānaṃ upasampajja viharati.

Sukhassa ca pahānā dukkhassa ca pahānā pubbeva somanassa-domanassānaṃ atthaṅgamā
adukkhamasukhaṃ upekkhā-satipārisuddhiṃ catutthaṃ jhānaṃ upasampajja viharati.
ayaṃ vuccati, bhikkhave, sammā-samādhi.

Idaṃ vuccati, bhikkhave, dukkha-nirodha-gāminī paṭipadā ariyasaccaṃ.

Iti ajjhattaṃ vā dhammesu dhammānupassī viharati,
bahiddhā vā dhammesu dhammānupassī viharati,
ajjhatta-bahiddhā vā dhammesu dhammānupassī viharati.
samudaya-dhammānupassī vā dhammesu viharati,
vaya-dhammānupassī vā dhammesu viharati,
samudaya-vaya-\\ dhammānupassī vā dhammesu viharati.
‘atthi dhammā’ti vā panassa sati paccupaṭṭhitā hoti
yāvadeva ñāṇamattāya paṭissatimattāya anissito ca viharati,
na ca kiñci loke upādiyati. evampi kho, bhikkhave, bhikkhu
dhammesu dhammānupassī viharati catūsu ariyasaccesu.

\instr{Saccapabbaṃ niṭṭhitaṃ.}

\instr{Dhammānupassanā niṭṭhitā.}

\englishPage
\chapter{Conclusion}

Bhikkhus, if anyone should develop these four foundations of mindfulness in
such a way for seven years, one of two fruits could be expected for him: either
final knowledge here and now, or if there is a trace of clinging left,
nonreturning.

Let alone seven years, bhikkhus. If anyone should develop these four
foundations of mindfulness in such a way for six years \ldots{} five years
\ldots{} four years \ldots{} three years \ldots{} two years \ldots{} one year,
Let alone one year, bhikkhus. If anyone should develop these four foundations
of mindfulness in such a way for seven months, one of two fruits could be
expected for him: either final knowledge here and now, or if there is a trace of
clinging left, nonreturning. Let alone seven months, bhikkhus. If anyone should
develop these four foundations of mindfulness in such a way for six months
\ldots{} five months \ldots{} four months \ldots{} three months \ldots{} two
months \ldots{} one month \ldots{} half a month, Let alone half a month,
bhikkhus.

If anyone should develop these four foundations of mindfulness in such a way
for seven days, one of two fruits could be expected for him: eitherfinal
knowledge here and now, or if there is a trace of clinging left, nonreturning.

`Bhikkhus, this is the direct path for the purification of beings, for the
surmounting of sorrow and lamentation, for the passing away of pain and
dejection, for the attainment of the true way, for the realisation of Nibbāna,
namely, the four foundations of mindfulness.'

It was with reference to this that it was said. This is what the Blessed One
said. The bhikkhus were satisfied and delighted in the Blessed One's words.

\bigskip

{\centering\instructionFont\color{instruction}\upshape

  The Greater Discourse on the\\
  Foundations of Mindfulness is finished.

}

\paliPage
\chapter*{Conclusion}

Yo hi koci, bhikkhave, ime cattāro satipaṭṭhāne evaṃ bhāveyya sattavassāni,
tassa dvinnaṃ phalānaṃ aññataraṃ phalaṃ pāṭikaṅkhaṃ diṭṭheva dhamme aññā; sati
vā upādisese anāgāmitā.

Tiṭṭhantu, bhikkhave, sattavassāni. yo hi koci, bhikkhave, ime cattāro
satipaṭṭhāne evaṃ bhāveyya cha vassāni \ldots{} pañca vassāni
\ldots{} cattāri vassāni \ldots{} tīṇi vassāni \ldots{} dve vassāni \ldots{}
ekaṃ vassaṃ \ldots{} tiṭṭhatu, bhikkhave, ekaṃ vassaṃ. yo hi koci, bhikkhave,
ime cattāro satipaṭṭhāne evaṃ bhāveyya sattamāsāni, tassa dvinnaṃ phalānaṃ
aññataraṃ phalaṃ pāṭikaṅkhaṃ diṭṭheva dhamme aññā; sati vā upādisese anāgāmitā.

Tiṭṭhantu, bhikkhave, satta māsāni. yo hi koci, bhikkhave, ime cattāro
satipaṭṭhāne evaṃ bhāveyya cha māsāni \ldots{} pañca māsāni \ldots{}
cattāri māsāni \ldots{} tīṇi māsāni \ldots{} dve māsāni \ldots{} ekaṃ māsaṃ
\ldots{} aḍḍhamāsaṃ \ldots{} tiṭṭhatu, bhikkhave, aḍḍhamāso. yo hi koci,
bhikkhave, ime cattāro satipaṭṭhāne evaṃ bhāveyya sattāhaṃ, tassa dvinnaṃ
phalānaṃ aññataraṃ phalaṃ pāṭikaṅkhaṃ diṭṭheva dhamme aññā; sati vā upādisese
anāgāmitā.

Ekāyano ayaṃ, bhikkhave, maggo sattānaṃ visuddhiyā soka-paridevānaṃ samatikkamāya
dukkha-domanassānaṃ atthaṅgamāya ñāyassa adhigamāya nibbānassa sacchikiriyāya
yadidaṃ cattāro satipaṭṭhānāti. Iti yaṃ taṃ vuttaṃ, idametaṃ paṭicca vuttan'ti.

Idamavoca bhagavā. Attamanā te bhikkhū bhagavato bhāsitaṃ abhinandunti.

\bigskip

{\centering\instructionFont\color{instruction}\upshape

  Mahāsatipaṭṭhānasuttaṃ niṭṭhitaṃ.

}

\resumeNormalText
