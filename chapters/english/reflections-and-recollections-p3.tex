\chapter[The Noble Eightfold Path]{The Teaching on the Noble Eightfold Path}

\firstline{Ayam-eva ariyo aṭṭhaṅgiko maggo}

\begin{leader}
  [Handa mayaṁ ariyaṭṭhaṅgika-magga-pāṭham bhaṇāmase]
\end{leader}

Ayam-eva a꜕riyo aṭṭha꜓ṅgi꜕ko maggo

\begin{english}
  This is the No꜕bl꜕e E꜕ightfo꜕ld Path,
\end{english}

Se꜓yyathī꜓daṁ

\begin{english}
  Which is as fo꜕llows:
\end{english}

Sa꜓mmā-diṭṭhi

\begin{english}
  Ri꜕ght View,
\end{english}

Sa꜓mmā-sa꜓ṅka꜕ppo

\begin{english}
  Right Inte꜕ntion,
\end{english}

Sa꜓mmā-vācā

\begin{english}
  Ri꜕ght Speech,
\end{english}

Sa꜓mmā-kammanto

\begin{english}
  Right A꜕ction,
\end{english}

Sa꜓mmā-ājīvo

\begin{english}
  Right Li꜓vel꜕ihood,
\end{english}

Sa꜓mmā-vā꜕yāmo

\begin{english}
  Right E꜕ffort,
\end{english}

\ifaivedition
\clearpage
\fi

Sa꜓mmā-sa꜕ti

\begin{english}
  Right Mi꜓ndfu꜕lness,
\end{english}

Sa꜓mmā-sa꜕mādhi

\begin{english}
  Ri꜕ght Co꜕nce꜕ntr꜕ation.
\end{english}

Ka꜕tamā ca bhi꜓kkh꜕ave sammā-diṭṭhi

\begin{english}
  And what, bhikkhus, i꜕s Ri꜕ght View?
\end{english}

Yaṁ kho bhi꜓kkh꜕ave dukkhe ñāṇaṁ

\begin{english}
  Knowledge of su꜕ffering;
\end{english}

Dukkha-sa꜕mu꜕daye ñāṇaṁ

\begin{english}
  Knowledge of the o꜓rigin of su꜕ffering;
\end{english}

Dukkha-ni꜓rodhe ñāṇaṁ

\begin{english}
  Knowledge of the cessa꜕tio꜕n o꜕f su꜕ffe꜕ring;
\end{english}

Dukkha-ni꜓rodha-gāmi꜓ni꜓yā pa꜕ṭipa꜕dāya ñāṇaṁ

\begin{english}
  Knowledge of th꜓e path leading to the cess꜕ati꜕on o꜕f su꜕ffering:
\end{english}

A꜕yaṁ vuccati bhi꜓kkh꜕ave sa꜓mmā-diṭṭhi

\begin{english}
  This, bhikkhus, is ca꜕lled Ri꜕ght View.
\end{english}

Katamo ca bhi꜓kkh꜕ave sammā-sa꜓ṅka꜕ppo

\begin{english}
  And what, bhikkhus, is Ri꜕ght I꜕nte꜕ntion?
\end{english}

\ifaivedition
\clearpage
\fi

Nekkhamma-sa꜓ṅka꜕ppo

\begin{english}
  The intention of renu꜕nc꜕ia꜕tion;
\end{english}

A꜕byāpāda-sa꜓ṅka꜕ppo

\begin{english}
  The intention of no꜕n-il꜕l-will;
\end{english}

A꜕vihiṁsā-sa꜓ṅka꜕ppo

\begin{english}
  The intention of non-cru꜓e꜕lty:
\end{english}

Ayaṁ vuccati bhi꜓kkh꜕ave sa꜓mmā-sa꜓ṅka꜕ppo

\begin{english}
  This, bhikkhus, is ca꜕lled Ri꜕ght I꜕nte꜕ntion.
\end{english}

Katamā ca bhi꜓kkh꜕ave sa꜓mmā-vācā

\begin{english}
  And what, bhikkhus, i꜕s Ri꜕ght Speech?
\end{english}

Musā-vādā vera꜓ma꜕ṇī

\trline{Abstaining fro꜕m fa꜕lse speech;}

Pisuṇāya vācāya vera꜓ma꜕ṇī

\trline{Abstaini꜓ng from mali꜓cio꜕us speech;}

Pharusāya vācāya vera꜓ma꜕ṇī

\trline{Abstaining fro꜕m ha꜕rsh speech;}

Sa꜓mphappa꜕lāpā vera꜓ma꜕ṇī.

\trline{Abstaining from i꜕dl꜕e cha꜕tter:}

\ifaivedition
\clearpage
\fi

Ayaṁ vuccati bhi꜓kkh꜕ave sa꜓mmā-vācā

\begin{english}
  This, bhikkhus, is ca꜕lled Ri꜕ght Speech.
\end{english}

Katamo ca bhi꜓kkh꜕ave sa꜓mmā-kammanto

\begin{english}
  And what bhikkhus i꜕s Ri꜕ght A꜕ction?
\end{english}

Pāṇāti꜕pātā vera꜓ma꜕ṇī

\begin{english}
  Abstaini꜓ng from ki꜕lli꜕ng li꜕vi꜕ng be꜕ings;
\end{english}

A꜕dinnādānā vera꜓ma꜕ṇī

\begin{english}
  Abstaini꜓ng from ta꜕ki꜕ng wh꜕at i꜕s no꜕t gi꜕ven;
\end{english}

Kāmesu꜕ micchā꜓cārā vera꜓ma꜕ṇī

\begin{english}
  Abstaini꜓ng from se꜕xu꜕al mi꜓sco꜕nduct:
\end{english}

Ayaṁ vuccati bhi꜓kkh꜕ave sa꜓mmā-kammanto

\begin{english}
  This, bhikkhus, is ca꜕lled Ri꜕ght Ac꜕tion.
\end{english}

Katamo ca bhi꜓kkha꜕ve sa꜓mmā-ājīvo

\begin{english}
  And what, bhikkhus, is Right L꜓ivel꜕ihood?
\end{english}

Idha bhi꜓kkh꜕ave a꜕riya-sā꜓va꜕ko micchā-ājīvaṁ pa꜕hāya sammā-ājī꜓vena jīvi꜓taṁ ka꜕ppeti

\begin{english}
  Here, bhikkhus, a Nobl꜕e Di꜕sc꜕iple, having a꜓bandoned wrong li꜓vel꜕ihood, earns h꜓is living by ri꜕ght li꜕vel꜕ihood:
\end{english}

Ayaṁ vuccati bhi꜓kkh꜕ave sa꜓mmā-ājīvo

\begin{english}
  This, bhikkhus, is ca꜕lled Ri꜕ght Li꜕vel꜕ihood.
\end{english}

Katamo ca bhi꜓kkh꜕ave sa꜓mmā-vāyāmo

\begin{english}
  And what, bhikkhus, i꜕s Ri꜕ght E꜕ffort?
\end{english}

Idha bhi꜓kkh꜕ave bhikkhu a꜕nuppannānaṁ pāpa꜕kānaṁ a꜕ku꜕salānaṁ dhammānaṁ anuppādāya

\begin{english}
  Here, bhikkhus, a꜓ bhikkhu awa꜕ke꜕ns zeal for the non-a꜓rising of unari꜕sen, evil unwho꜓leso꜕me states;
\end{english}

Chandaṁ ja꜕neti vāyama꜓ti vī꜓ri꜓yaṁ ārabha꜕ti ci꜕ttaṁ pa꜕ggaṇhā꜓ti pa꜕daha꜕ti

\begin{english}
  He puts forth e꜕ffort, arouses e꜓ne꜕rgy, exerts h꜓is mind an꜕d strives.
\end{english}

U꜕ppannānaṁ pāpa꜕kānaṁ a꜕ku꜕salānaṁ dhammānaṁ pa꜕hānāya

\begin{english}
  He awake꜓ns zeal for the aba꜕ndoning of a꜓risen, evil unwho꜓leso꜕me states;
\end{english}

Chandaṁ ja꜕neti vāyama꜓ti vī꜓ri꜓yaṁ ārabha꜕ti ci꜕ttaṁ pa꜕ggaṇhā꜓ti pa꜕daha꜕ti

\begin{english}
  He puts forth e꜕ffort, arouses e꜓ne꜕rgy, exerts h꜓is mind an꜕d strives.
\end{english}

Anuppannānaṁ ku꜕salānaṁ dhammānaṁ u꜕ppādāya

\begin{english}
  He awake꜓ns zeal for the ari꜕sing of una꜓risen who꜓leso꜕me states;
\end{english}

Chandaṁ ja꜕neti vāyama꜓ti vī꜓ri꜓yaṁ ārabha꜕ti ci꜕ttaṁ pa꜕ggaṇhā꜓ti pa꜕daha꜕ti

\begin{english}
  He puts forth e꜕ffort, arouses e꜓ne꜕rgy, exerts h꜓is mind an꜕d strives.
\end{english}

\ifaivedition
\clearpage
\fi

U꜕ppannānaṁ ku꜕salānaṁ dhammānaṁ ṭh꜓iti꜕yā a꜕sa꜕mmosāya bh꜓iyyobhāvāya vepu꜕llāya bhāva꜓nāya pāri꜓pū꜕riyā

\begin{english}
  He awakens zeal for the conti꜕nuance, non-disa꜓ppearance, stre꜕ngthening, increase and fu꜓lfilment by deve꜓lo꜕pment of ari꜕sen who꜕leso꜕me states;
\end{english}

Chandaṁ ja꜕neti vāyama꜓ti vī꜓ri꜓yaṁ ārabha꜕ti ci꜕ttaṁ pa꜕ggaṇhā꜓ti pa꜕daha꜕ti

\begin{english}
  He puts forth e꜕ffort, arouses e꜓ne꜕rgy, exerts h꜓is mind an꜕d strives:
\end{english}

Ayaṁ vuccati bhi꜓kkh꜕ave sa꜓mmā-vāyāmo

\begin{english}
  This bhikkhus is ca꜕lled Ri꜕ght E꜕ffort.
\end{english}

Katamā ca bhi꜓kkh꜕ave sa꜓mmā-sa꜕ti

\begin{english}
  And what, bhikkhus, is Right Mi꜓ndfu꜕lness?
\end{english}

Idha bhi꜓kkh꜕ave bhikkhu kāye kāyānupa꜕ssī vi꜓ha꜕rati

\begin{english}
  Here, bhikkhus, a bhi꜕kkhu꜕ a꜕bides conte꜓mplating the bo꜕dy a꜕s a꜕ bo꜕dy,
\end{english}

Ātāpī sa꜓mpa꜕jāno sa꜕timā

\begin{english}
  Ardent, fully꜓ a꜕ware and mi꜕ndful,
\end{english}

Vi꜓neyya loke a꜕bhijjhā-domanassaṁ

\begin{english}
  Having pu꜕t a꜕way co꜕ve꜕to꜕usn꜕ess an꜕d gri꜕ef fo꜕r th꜕e world;
\end{english}

Veda꜕nāsu꜕ veda꜕nānu꜓pa꜕ssī vi꜓ha꜕rati

\begin{english}
  He a꜕bides conte꜓mplating fe꜕eli꜕ngs a꜕s fe꜕elings,
\end{english}

\ifaivedition
\clearpage
\fi

Ātāpī sa꜓mpa꜕jāno sa꜕timā

\begin{english}
  Ardent, fully꜓ a꜕ware and mi꜕ndful,
\end{english}

Vi꜓neyya loke a꜕bhijjhā-domanassaṁ

\begin{english}
  Having pu꜕t a꜕way co꜕ve꜕to꜕usn꜕ess an꜕d gri꜕ef fo꜕r th꜕e world;
\end{english}

Ci꜕tte ci꜕ttānu꜓pa꜕ssī vi꜓ha꜕rati

\begin{english}
  He a꜕bides conte꜓mplating mi꜕nd a꜕s mind,
\end{english}

Ātāpī sa꜓mpa꜕jāno sa꜕timā

\begin{english}
  Ardent, fully꜓ a꜕ware and mi꜕ndful,
\end{english}

Vi꜓neyya loke a꜕bhijjhā-domanassaṁ

\begin{english}
  Having pu꜕t a꜕way co꜕ve꜕to꜕usne꜕ss an꜕d gri꜕ef fo꜕r th꜕e world;
\end{english}

Dhammesu꜕ dhammānu꜓pa꜕ssī vi꜓ha꜕rati

\begin{english}
  He a꜕bides conte꜓mplating mind-o꜕bje꜕cts a꜕s mi꜕nd-o꜕bjects,
\end{english}

Ātāpī sa꜓mpa꜕jāno sa꜕timā

\begin{english}
  Ardent, fully꜓ a꜕ware and mi꜕ndful,
\end{english}

Vi꜓neyya loke a꜕bhijjhā-domanassaṁ

\begin{english}
  Having pu꜕t a꜕way co꜕ve꜕to꜕usn꜕ess an꜕d gri꜕ef fo꜕r th꜕e world:
\end{english}

Ayaṁ vuccati bhi꜓kkh꜕ave sa꜓mmā-sa꜕ti

\begin{english}
  This, bhikkhus, is ca꜕lled Ri꜕ght Mi꜕ndfu꜕lness.
\end{english}

\ifaivedition
\clearpage
\fi

Katamo ca bhi꜓kkh꜕ave sa꜓mmā-sa꜕mādhi

\begin{english}
  And what, bhikkhus, is Ri꜕ght Co꜕nce꜕ntr꜕ation?
\end{english}

Idha bhi꜓kkh꜕ave bhikkhu

\begin{english}
  Here, bhikkhus, a bhi꜕kkhu,
\end{english}

Vivicc'eva kāmehi

\begin{english}
  Quite se꜓cluded from se꜕nsu꜕al pl꜕easures,
\end{english}

Vivicca a꜕ku꜕sa꜕lehi dh꜕ammehi

\begin{english}
  Secluded from unwho꜓leso꜕me states,
\end{english}

Sa꜕vi꜓ta꜕kkaṁ sa꜕vi꜓cāraṁ viveka꜕-jaṁ pīti꜕-sukhaṁ pa꜕ṭhamaṁ jhānaṁ upasa꜓mpajja vi꜓ha꜕rati

\begin{english}
  Enters u꜓pon and a꜕bides in꜕ th꜕e fi꜕rst jhā꜕na --- accompa꜓nied by
  appli꜕ed an꜕d su꜕stai꜕ned thought, with raptu꜓re and ple꜕asure bo꜕rn
  o꜕f se꜕clu꜕sion.
\end{english}

Vi꜓takka-vicārānaṁ vūpa꜕samā

\begin{english}
  With the stilling of appli꜕ed an꜕d su꜕stai꜕ned thought,
\end{english}

Ajjhattaṁ sa꜓mpa꜕sādanaṁ ceta꜕so ekodi꜓bhāvaṁ avi꜓ta꜕kkaṁ avi꜓cāraṁ sa꜕mādhi꜓-jaṁ pīti꜕-sukhaṁ du꜕tiyaṁ jhānaṁ upasa꜓mpa꜕jja vi꜓ha꜕rati

\begin{english}
  He enters u꜓pon and a꜕bides in꜕ th꜕e se꜕co꜕nd jhā꜕na --- accompa꜓nied
  by self-co꜓nf꜕idence and si꜕ngle꜕ne꜕ss o꜕f mind, without applie꜕d an꜕d
  su꜕stai꜕ned thought, with raptu꜓re and ple꜕asure bo꜕rn o꜕f
  co꜕nce꜕ntr꜕ation.
\end{english}

\clearpage

Pītiyā ca꜕ vi꜓rāgā

\begin{english}
  With the fadi꜓ng a꜕way as we꜕ll o꜕f ra꜕pture
\end{english}

U꜕pekkhako ca vi꜓ha꜕rati

\begin{english}
  He abides in equani꜓mi꜕ty,
\end{english}

Sa꜕to ca꜕ sa꜓mpa꜕jāno

\begin{english}
  Mindful and fully꜓ a꜕ware,
\end{english}

Su꜕khañca kāyena pa꜕ṭisa꜓ṁvedeti

\begin{english}
  Still fee꜕li꜕ng ple꜕asu꜕re wi꜕th th꜕e bo꜕dy,
\end{english}

Yaṁ taṁ a꜕riyā āci꜕kkhanti u꜕pekkha꜓ko sa꜕timā su꜕kha-vi꜓hā꜕rī'ti tatiyaṁ jhānaṁ u꜕pasa꜓mpa꜕jja vi꜓ha꜕rati

\begin{english}
  He enters u꜓pon and a꜕bides in꜕ th꜕e thi꜕rd jh꜕āna --- on account o꜓f
  which the No꜕bl꜕e O꜓nes a꜕nnounce, `He has a꜓ pleasant abi꜕ding, with
  equani꜓mi꜕ty and\\ is mi꜕ndful.'
\end{english}

Sukhassa ca꜕ pahānā

\begin{english}
  With the aba꜓ndoning of ple꜕asure
\end{english}

Dukkhassa ca꜕ pahānā

\begin{english}
  And the aba꜕ndo꜕ni꜕ng o꜕f pain,
\end{english}

Pu꜕bb'eva somanassa꜕-domanassā꜓naṁ a꜕tthaṅga꜕mā

\begin{english}
  With the previous disa꜓ppearance of jo꜕y an꜕d grief,
\end{english}

Adukkham-asu꜕khaṁ u꜕pekkhā-sa꜕ti-pā꜕ri꜓su꜕ddhiṁ ca꜕tutthaṁ jhānaṁ u꜕pasa꜓mpa꜕jja vi꜓ha꜕rati

\begin{english}
  He enters u꜓pon and a꜕bides i꜕n th꜕e fou꜕rth jh꜕āna --- accompa꜓nied
  by neither pa꜕in no꜕r pl꜕easure, and purity of mi꜓ndfu꜕lness du꜕e to꜕
  e꜕qu꜕an꜕imity:
\end{english}

Ayaṁ vuccati bhi꜓kkh꜕ave sa꜓mmā-sa꜕mādhi

\begin{english}
  This, bhikkhus, is ca꜕lled Ri꜕ght Co꜕nce꜕ntr꜕ation.
\end{english}

Ayam-eva a꜕riyo aṭṭha꜓ṅgi꜕ko maggo

\begin{english}
  This is the No꜕bl꜕e E꜕ightfo꜕ld Path.
\end{english}

\chapter[The Wheel of Dhamma]{Teachings from the Discourse on Setting in Motion the Wheel of Dhamma}

\firstline{Dve me bhikkhave antā}

\begin{leader}
  [Ha꜓nda mayaṁ dhammacakkappavattana-su꜕tta-pāṭhaṁ bha꜕ṇāmase]
\end{leader}

Dve me bhi꜓kkha꜕ve antā

\begin{english}
  Bhikkhus, there are these tw꜕o ex꜕tremes
\end{english}

Pabbaji꜓tena na sevi꜓ta꜕bbā

\begin{english}
  That shou꜕ld no꜕t b꜕e pu꜕rsued by one who ha꜓s go꜕ne forth:
\end{english}

Yo cāyaṁ kāmesu꜕ kāma-su꜕kh'alli꜓kānu꜓yogo

\begin{english}
  That is, whatever is tied u꜕p t꜕o se꜕nse pl꜕easures, within the re꜕alm\\ o꜕f se꜕nsu꜕a꜕li꜕ty,
\end{english}

Hīno

\begin{english}
  Whi꜕ch i꜕s low,
\end{english}

Gammo

\begin{english}
  Co꜕mmon,
\end{english}

Pothujj꜓ani꜕ko

\begin{english}
  The way of the co꜕mmo꜕n folks,
\end{english}

Anar꜓iyo

\begin{english}
  Not the wa꜕y o꜕f th꜕e No꜓bl꜕e Ones
\end{english}

\clearpage

Anattha-sa꜓ñh꜕ito

\begin{english}
  And po꜕intless;
\end{english}

Yo cāyaṁ atta-kilama꜓thānu꜓yogo

\begin{english}
  Then there is whate꜕ve꜕r i꜕s t꜕ied up with se꜕lf-de꜕pr꜕iva꜕tion,
\end{english}

Dukkho

\begin{english}
  Which is pa꜕inful,
\end{english}

Anar꜓iyo

\begin{english}
  Not the wa꜕y o꜕f th꜕e No꜓bl꜕e Ones
\end{english}

A꜕nattha꜕-sa꜓ñh꜕ito

\begin{english}
  And po꜕intless.
\end{english}

Ete te bhi꜓kkha꜕ve ub꜕ho꜕ ante a꜕nupa꜕gamma majjhi꜓mā pa꜕ṭi꜕pa꜕dā tathā꜓ga꜕tena a꜕bhis꜓ambuddhā

\begin{english}
  Bhikkhus, without go꜕ing t꜕o e꜕ith꜕er o꜕f th꜕ese e꜕xtremes, the
  Tathā꜓ga꜕ta has u꜕lt꜕ima꜕tel꜕y a꜕wa꜕kened to a꜓ middle wa꜕y o꜕f
  pr꜕actice,
\end{english}

Cakkhu-ka꜕ra꜓ṇī

\begin{english}
  Givi꜓ng rise to v꜕ision,
\end{english}

Ñāṇa-ka꜕ra꜓ṇī

\begin{english}
  Ma꜕ki꜕ng fo꜕r i꜕nsight,
\end{english}

U꜕pasa꜕māya

\begin{english}
  Leadi꜓ng t꜕o calm,
\end{english}

A꜕bhiññāya

\begin{english}
  To he꜕ight꜕ened kn꜕owing,
\end{english}

Sa꜓mbodhāya

\begin{english}
  Awa꜕ke꜕ning,
\end{english}

Ni꜓bbānāya sa꜓ṁvat꜕tati

\begin{english}
  An꜕d t꜕o Nibbā꜕na.
\end{english}

Katamā ca sā bhi꜓kkh꜕ave majjhi꜕mā p꜕aṭ꜕ip꜕adā

\begin{english}
  And what, bhikkhus, i꜕s tha꜕t mi꜕ddl꜕e wa꜕y o꜕f pra꜕ctice?
\end{english}

Ayam-eva a꜕riyo aṭṭha꜓ṅgi꜕ko maggo

\begin{english}
  It is this No꜕bl꜕e Ei꜕ghtfo꜕ld Path,
\end{english}

Se꜓yyathī꜓daṁ

\begin{english}
  Which is as fo꜕llows:
\end{english}

Sa꜓mmā-diṭṭhi

\begin{english}
  Ri꜕ght View,
\end{english}

Sa꜓mmā-sa꜓ṅka꜕ppo

\begin{english}
  Right Inte꜕ntion,
\end{english}

Sa꜓mmā-vācā

\begin{english}
  Ri꜕ght Speech,
\end{english}

\ifaivedition
\clearpage
\fi

Sa꜓mmā-kammanto

\begin{english}
  Right A꜕ction,
\end{english}

Sa꜓mmā-ājīvo

\begin{english}
  Right Li꜓vel꜕ihood,
\end{english}

Sa꜓mmā-vā꜕yāmo

\begin{english}
  Right E꜕ffort,
\end{english}

Sa꜓mmā-sa꜕ti

\begin{english}
  Right Mi꜓ndfu꜕lness,
\end{english}

Sa꜓mmā-sa꜕mādhi

\begin{english}
  Ri꜕ght Co꜕nce꜕ntr꜕ation.
\end{english}

Ayaṁ kho sā bhi꜓kkh꜕ave majjh꜕imā p꜕aṭ꜕ip꜕adā tathā꜓ga꜕tena abhisa꜓mbuddhā

\begin{english}
  This, bhikkhus, is the mi꜕ddl꜕e wa꜕y o꜕f pr꜕actice that the
  Tathā꜓ga꜕ta has u꜕lti꜕ma꜕tel꜕y a꜕wa꜕ke꜕ned to,
\end{english}

Cakkhu-ka꜕ra꜓ṇī

\begin{english}
  Givi꜓ng rise to v꜕ision,
\end{english}

Ñāṇa-ka꜕ra꜓ṇī

\begin{english}
  Ma꜕ki꜕ng fo꜕r i꜕nsight,
\end{english}

U꜕pasa꜕māya

\begin{english}
  Leadi꜓ng t꜕o calm,
\end{english}

A꜕bhiññāya

\begin{english}
  To he꜕ight꜕ened kn꜕owing,
\end{english}

Sa꜓mbodhāya

\begin{english}
  Awa꜕ke꜕ning,
\end{english}

Ni꜓bbānāya sa꜓ṁvat꜕tati

\begin{english}
  An꜕d t꜕o Nibbā꜕na.
\end{english}

Idaṁ kho pana bhi꜓kkh꜕ave dukkhaṁ a꜕riya꜓-s꜕accaṁ

\begin{english}
  This bhikkhus is the No꜕ble꜕ Tr꜕uth o꜕f du꜕kkha:
\end{english}

Jātipi꜕ dukkhā

\begin{english}
  Birth is du꜕kkha,
\end{english}

Jarāpi꜕ dukkhā

\begin{english}
  Ageing is du꜕kkha
\end{english}

Maraṇampi꜕ dukkhaṁ

\begin{english}
  And death is du꜕kkha;
\end{english}

So꜓ka-pa꜕rideva-dukkha꜕-domanass'u꜕pāyāsā꜓pi꜕ dukkhā

\begin{english}
  So꜓rrow, lamenta꜕tion, pain, grief and de꜕spair are du꜕kkha,
\end{english}

Appiyehi꜕ sa꜓mpa꜕yogo dukkho

\begin{english}
  Association with the di꜕sliked is du꜕kkha,
\end{english}

\ifaivedition
\clearpage
\fi

Piyehi꜕ vi꜓ppa꜕yogo dukkho

\begin{english}
  Separa꜓tion from th꜕e liked is du꜕kkha,
\end{english}

Yampiccha꜓ṁ na꜕ labhati꜕ tampi꜕ dukkhaṁ

\begin{english}
  Not attaining one's wi꜓shes is du꜕kkha;
\end{english}

Sa꜓ṅkhi꜕ttena pañcu꜕pādānakkha꜓ndhā dukkhā

\begin{english}
  In brief, th꜕e five focuses of iden꜓tity are du꜕kkha.
\end{english}

Idaṁ kho pa꜕na bhi꜓kkh꜕ave dukkha-sa꜕mu꜕dayo a꜕riya꜓-sa꜕ccaṁ

\begin{english}
  This bhikkhus is the No꜕bl꜕e Tr꜕uth o꜕f th꜕e cau꜕se o꜕f du꜕kkha:
\end{english}

Yā'yaṁ taṇhā

\begin{english}
  It is this cra꜕ving
\end{english}

Ponobbha꜓vi꜕kā

\begin{english}
  Which lea꜕ds t꜕o re꜕birth,
\end{english}

Nandi꜓-rāga-sa꜕ha꜕ga꜕tā

\begin{english}
  Accompanied by deli꜓ght a꜕nd lust,
\end{english}

Ta꜕tra-ta꜕trābhi꜓nandi꜕nī

\begin{english}
  Delighting now he꜕re, no꜕w there,
\end{english}

Se꜓yyathī꜓daṁ

\begin{english}
  Na꜕mely:
\end{english}

\ifaivedition
\clearpage
\fi

Kāma-taṇhā

\begin{english}
  Craving fo꜕r se꜕nsu꜕a꜕lity,
\end{english}

Bhava-taṇhā

\begin{english}
  Craving t꜓o be꜕come,
\end{english}

Vi꜓bhava-taṇhā

\begin{english}
  Craving no꜕t t꜕o be꜕come.
\end{english}

Idaṁ kho pa꜕na bhi꜓kkh꜕ave dukkha-nirodho a꜕riya꜓-sa꜕ccaṁ

\begin{english}
  This bhikkhus is the No꜕bl꜕e Tr꜕uth o꜕f th꜕e ce꜕ssa꜕ti꜕on o꜕f du꜕kkha:
\end{english}

Yo tassā yeva taṇhāya a꜕sesa-vi꜓rāga-nirodho

\begin{english}
  It is the remainderless fa꜕di꜕ng a꜕wa꜕y an꜕d ce꜕ssa꜕tion of th꜓at very cr꜕aving,
\end{english}

Cāgo

\trline{Its reli꜓nqu꜕ishment,}

Pa꜕ṭini꜓ssa꜕ggo

\trline{Le꜕tti꜕ng go,}

Mutti

\trline{Re꜕lease,}

A꜕nāla꜓yo

\trline{Without a꜕ny꜕ a꜕tta꜕chment.}

\ifaivedition
\clearpage
\fi

Idaṁ kho pa꜕na bhi꜓kkh꜕ave dukkha-nirodha꜕-gāmi꜓nī pa꜕ṭi꜕pa꜕dā a꜕riya꜓-sa꜕ccaṁ

\begin{english}
  This bhikkhus is the No꜕ble꜕ Tru꜕th o꜕f th꜕e wa꜕y o꜕f pra꜕ctice leading to the ce꜓ssation of du꜕kkha:
\end{english}

Ayam-eva a꜕riyo aṭṭh'a꜓ṅgi꜕ko maggo

\begin{english}
  It is just this No꜕ble꜕ E꜕ightfo꜕ld Path,
\end{english}

Se꜓yyathī꜓daṁ

\trline{Which is as fo꜕llows:}

Sa꜓mmā-diṭṭhi

\trline{Ri꜕ght View,}

Sa꜓mmā-sa꜓ṅka꜕ppo

\trline{Right Inte꜕ntion,}

Sa꜓mmā-vācā

\trline{Ri꜕ght Speech,}

Sa꜓mmā-kammanto

\trline{Right A꜕ction,}

Sa꜓mmā-ājīvo

\trline{Right Li꜓vel꜕ihood,}

Sa꜓mmā-vā꜕yāmo

\trline{Right E꜕ffort,}

Sa꜓mmā-sa꜕ti

\trline{Right Mi꜓ndfu꜕lness,}

Sa꜓mmā-sa꜕mādhi

\trline{Ri꜕ght Co꜕nce꜕ntr꜕ation.}

Idaṁ dukkhaṁ a꜕riya-sa꜕ccan't꜕i me bhi꜓kkh꜕ave\\
Pubbe a꜕nanussu꜕tesu꜕ dhammesu\\
Cakkhuṁ u꜕da꜓pādi\\
Ñāṇaṁ u꜕da꜓pādi\\
Paññā u꜕da꜓pādi\\
Vijjā u꜕da꜓pādi\\
Āloko u꜕da꜓pādi

\begin{english}
  Bhikkhus, in rega꜕rd t꜕o thi꜕ngs u꜕nhe꜕ard-o꜕f be꜕fore,\\
  Visio꜓n a꜕rose,\\
  I꜕ns꜕ight a꜕rose,\\
  Disce꜕rnm꜕ent a꜕rose,\\
  Knowle꜓dge a꜕rose,\\
  L꜕ight a꜕rose:\\
  This is the No꜕bl꜕e Tru꜕th o꜕f du꜕kkha;
\end{english}

Taṁ kho pa꜕n'idaṁ dukkhaṁ a꜕riya꜓-sa꜕ccaṁ pa꜕riññeyyan'ti

\begin{english}
  Now this No꜕ble꜕ Tr꜕uth o꜕f du꜕kkha should be completely u꜓nde꜕rstood;
\end{english}

Taṁ kho pa꜕n'idaṁ dukkhaṁ a꜕riya꜓-sa꜕ccaṁ pa꜕riññātan'ti

\begin{english}
  Now this No꜕ble꜕ Tr꜕uth o꜕f du꜕kkha has be꜕en co꜕mple꜕tel꜕y u꜕nde꜕rstood.
\end{english}

\ifaivedition
\clearpage
\fi

Idaṁ dukkha-sa꜕mu꜕dayo a꜕riya꜓-sa꜕ccan't꜕i me bhi꜓kkh꜕ave\\
Pubbe a꜕nanussu꜕tesu꜕ dhammesu\\
Cakkhuṁ u꜕da꜓pādi\\
Ñāṇaṁ u꜕da꜓pādi\\
Paññā u꜕da꜓pādi\\
Vijjā u꜕da꜓pādi\\
Āloko u꜕da꜓pādi

\begin{english}
  Bhikkhus, in rega꜕rd t꜕o thi꜕ngs u꜕nhe꜕ard-o꜕f be꜕fore,\\
  Visio꜓n a꜕rose,\\
  I꜕ns꜕ight a꜕rose,\\
  Disce꜕rnm꜕ent a꜕rose,\\
  Knowle꜓dge a꜕rose,\\
  L꜕ight a꜕rose:\\
  This is the No꜕ble꜕ Tru꜕th o꜕f th꜕e ca꜕use o꜕f du꜕kkha.
\end{english}

Taṁ kho pa꜕n'idaṁ dukkha-sa꜕mu꜕dayo a꜕riya꜓-sa꜕ccaṁ pa꜕hāta꜕bban'ti

\begin{english}
  Now this ca꜕use o꜕f du꜕kkha sho꜕uld b꜕e a꜕ba꜕ndoned;
\end{english}

Taṁ kho pa꜕n'idaṁ dukkha-sa꜕mu꜕dayo a꜕riya꜓-sa꜕ccaṁ pa꜕hīnan'ti

\begin{english}
  Now this ca꜕use o꜕f du꜕kkha ha꜕s be꜕en a꜕ba꜕ndoned.
\end{english}

Idaṁ dukkha꜕-nirodho a꜕riya꜓-sa꜕ccan't꜕i me bhi꜓kkh꜕ave\\
Pubbe a꜕nanussu꜕tesu꜕ dhammesu\\
Cakkhuṁ u꜕da꜓pādi\\
Ñāṇaṁ u꜕da꜓pādi\\
Paññā u꜕da꜓pādi\\
Vijjā u꜕da꜓pādi\\
Āloko u꜕da꜓pādi

\begin{english}
  Bhikkhus, in rega꜕rd t꜕o thi꜕ngs u꜕nhe꜕ard-o꜕f be꜕fore,\\
  Visio꜓n a꜕rose,\\
  I꜕ns꜕ight a꜕rose,\\
  Disce꜕rnm꜕ent a꜕rose,\\
  Knowle꜓dge a꜕rose,\\
  L꜕ight a꜕rose:\\
  This is the No꜕ble꜕ Tr꜕uth o꜕f th꜕e ce꜕ssa꜕ti꜕on o꜕f du꜕kkha;
\end{english}

Taṁ kho pa꜕n'idaṁ dukkha-nirodho a꜕riya꜓-sa꜕ccaṁ sacch꜕ikāta꜓bban'ti

\begin{english}
  Now the ce꜓ssation o꜕f du꜕kkha should be expe꜕rie꜕nced di꜕re꜕ctly;
\end{english}

Taṁ kho pa꜕n'idaṁ dukkha-nirodho a꜕riya꜓-sa꜕ccaṁ sacch꜕ika꜕tan'ti

\begin{english}
  Now the ce꜓ssation o꜕f du꜕kkha ha꜕s be꜕en e꜕xpe꜕rie꜕nced di꜕re꜕ctly.
\end{english}

Idaṁ dukkha꜕-nirodha꜕-gāmi꜓nī pa꜕ṭi꜕pa꜕dā a꜕riya꜓-sa꜕ccan't꜕i me bhi꜓kkh꜕ave\\
Pubbe a꜕nanussu꜕tesu꜕ dhammesu\\
Cakkhuṁ u꜕da꜓pādi\\
Ñāṇaṁ u꜕da꜓pādi\\
Paññā u꜕da꜓pādi\\
Vijjā u꜕da꜓pādi\\
Āloko u꜕da꜓pādi

\begin{english}
  Bhikkhus, in rega꜕rd t꜕o thi꜕ngs u꜕nhe꜕ard-o꜕f be꜕fore,\\
  Visio꜓n a꜕rose,\\
  I꜕ns꜕ight a꜕rose,\\
  Disce꜕rnm꜕ent a꜕rose,\\
  Knowle꜓dge a꜕rose,\\
  L꜕ight a꜕rose:\\
  \ifaivedition
  \clearpage
  \fi
  This is the No꜕ble꜕ Tru꜕th o꜕f th꜕e wa꜕y o꜕f pra꜕ctice\\
  leading to the ce꜓ssation of du꜕kkha;
\end{english}

Taṁ kho pa꜕n'idaṁ dukkha-nirodha-gāmi꜓nī pa꜕ṭi꜕pa꜕dā a꜕riya꜓-sa꜕ccaṁ bhāvetabban'ti

\begin{english}
  Now this wa꜕y o꜕f pra꜕ctice leading to the ce꜓ssation of du꜕kkha\\
  sho꜕uld be꜕ de꜕ve꜕loped;
\end{english}

Taṁ kho pa꜕n'idaṁ dukkha-nirodha-gāmi꜓nī pa꜕ṭi꜕pa꜕dā a꜕riya꜓-sa꜕ccaṁ bhāvi꜓tan'ti

\begin{english}
  Now this wa꜕y o꜕f pra꜕ctice leading to the ce꜓ssation of du꜕kkha\\
  ha꜕s be꜕en de꜕ve꜕loped.
\end{english}

Yāva kī꜕vañca꜕ me bhi꜓kkh꜕ave i꜕mesu꜕ ca꜕tūsu a꜕riya꜓-sa꜕ccesu\\
Evan-t꜕i-pa꜕rivaṭṭaṁ dvādas'ā꜓kā꜓raṁ yathā꜓-bhūtaṁ ñāṇa-dassa꜕naṁ na su꜕vi꜓su꜕ddhaṁ a꜕hosi

\ifaivedition\relax\else
\clearpage
\fi

\begin{english}
  As long, bhi꜕kkhus, as my knowledge and understa꜕nding,\\
  As it ac꜕tua꜕lly꜕ is,\\
  Of these Four No꜓bl꜕e Truths,\\
  With their three pha꜕se꜕s an꜕d twe꜕lve a꜕spects,\\
  Was no꜕t e꜕nt꜓irel꜕y pure,
\end{english}

N'eva tāv'āhaṁ bhi꜓kkh꜕ave sa꜕deva꜓ke loke sa꜕māra꜓ke sa꜕brahma꜓ke\\
Sassamaṇa-brāhmaṇiyā pa꜕jāya sa꜕deva-ma꜕nussā꜓ya\\
Anu꜓tta꜕raṁ sa꜓mmā-sa꜓mbodhiṁ a꜕bhisa꜓mbuddho pa꜕ccaññāsiṁ

\begin{english}
  Did I not cla꜕im, bhi꜕kkhus,\\
  In this world of de꜕vas, Mā꜕ra꜕ an꜕d Br꜕ahmā,\\
  \ifaivedition
  \clearpage
  \fi
  Amongst ma꜕nkind with its priests and renu꜓nci꜕ants,\\
  Kings and co꜓mmo꜕ners,\\
  An u꜕lti꜕ma꜕te a꜕wa꜕ke꜕ning\\
  To unsu꜓rpassed, pe꜕rfe꜕ct e꜕nli꜓ghte꜕nment.
\end{english}

Ya꜕to ca꜕ kho me bhi꜓kkh꜕ave i꜕mesu꜕ ca꜕tūsu a꜕riya꜓-sa꜕ccesu\\
Evan-t꜕i-pa꜕rivaṭṭaṁ dvādas'ā꜓kā꜓raṁ yathā꜓-bhūtaṁ ñāṇa-dassanaṁ su꜕vi꜓su꜕ddhaṁ ahosi

\begin{english}
  But when, bhi꜕kkhus, my knowledge and understa꜕nding\\
  As it ac꜕tua꜕lly꜕ is,\\
  Of these Four No꜓bl꜕e Truths,\\
  With their three pha꜕se꜕s an꜕d twe꜕lve a꜕spects,\\
  Was inde꜕ed e꜕nt꜓irel꜕y pure,
\end{english}

Ath'āhaṁ bhi꜓kkh꜕ave sa꜕deva꜓ke loke sa꜕mār꜓ake sa꜕brahma꜓ke\\
Sassamaṇa-brāhmaṇiyā pa꜕jāya sa꜕deva-ma꜕nussā꜓ya\\
Anu꜓tta꜕raṁ sa꜓mmā-sa꜓mbodhiṁ a꜕bhisa꜓mbuddho pa꜕cca꜕ññā꜕siṁ

\begin{english}
  Th꜕en i꜕ndeed did I cla꜕im, bhi꜕kkhus,\\
  In this world of de꜕vas, Mā꜕ra꜕ an꜕d Br꜕ahmā,\\
  Amongst ma꜕nkind with its priests and renu꜓nci꜕ants,\\
  Kings and co꜓mmo꜕ners,\\
  An u꜕lti꜕ma꜕te a꜕wa꜕ke꜕ning\\
  To unsu꜓rpassed, pe꜕rfe꜕ct e꜕nli꜓ghte꜕nment.
\end{english}

Ñāṇañca pana me dass꜕anaṁ u꜕da꜓pādi

\begin{english}
  Now kno꜕wle꜕dge an꜕d u꜕nde꜕rst꜕anding aro꜕se i꜕n me:
\end{english}

\ifaivedition
\clearpage
\fi

A꜕kuppā me vi꜓mutti a꜕yam-ant꜕imā jāti natthi꜓ dāni pu꜕nabbh꜕avo'ti

\begin{english}
  My release i꜕s unsh꜓akeable,\\
  This is my la꜕st birth,\\
  There won't be a꜕ny꜕ fu꜕rth꜕er be꜕co꜕ming.
\end{english}

\chapter[Striving According to Dhamma]{The Teaching on Striving According to Dhamma}

\firstline{Evaṁ svākkhāto bhikkhave mayā dhammo}

\begin{leader}
  [Handa mayaṁ dhamma-pahaṁsāna-pāṭham bhaṇāmase]
\end{leader}

Evaṁ svā꜕kkhāto bhi꜓kkh꜕ave mayā dhammo

\begin{english}
  Bhikkhus, th꜕e Dhamma has thus been we꜕ll expo꜓unded by me,
\end{english}

Uttāno

\trline{Elu꜕ci꜕da꜕ted,}

Vi꜓va꜕ṭo

\trline{Di꜕sclosed,}

Pa꜕kāsi꜓to

\trline{Re꜕vealed,}

Chi꜓nna-pi꜕loti꜓ko

\trline{An꜕d str꜕ipped o꜕f pa꜕tchwork ---}

Alam-eva sa꜕ddhā-pa꜕bbaj꜓itena kula-pu꜕ttena vī꜓riyaṁ ā꜕rabh꜕ituṁ

\begin{english}
  This is enou꜕gh fo꜕r a꜕ cl꜕ansman, who has go꜕ne forth out o꜕f faith,
  to arou꜕se h꜕is e꜓ne꜓rgy꜕ thus:
\end{english}

Kāmaṁ ta꜕co ca nahā꜓ru c꜕a aṭṭhi c꜕a a꜕vasi꜓ss꜕atu

\begin{english}
  `Willingly let o꜕nly꜕ my꜕ skin, si꜕ne꜕ws a꜕nd bo꜕nes re꜕main,
\end{english}

Sa꜕rīre u꜕pasuss꜓atu maṁsa꜕-lohi꜕taṁ

\begin{english}
  And let th꜓e flesh and blo꜕od i꜕n th꜕is bo꜕dy wi꜕th꜕er a꜕way.
\end{english}

Yaṁ taṁ

\trline{As long as whate꜕ve꜕r i꜕s t꜕o b꜕e a꜕ttained}

Pu꜕risa-thāmena

\trline{By huma꜓n strength,}

Pu꜕risa-vī꜓riyena

\trline{By human e꜓ne꜕rgy,}

Pu꜕risa-pa꜕rakk꜕amena

\trline{B꜕y hu꜕ma꜕n e꜕ffort,}

Pa꜕tta꜕bbaṁ na taṁ a꜕pāpu꜕ṇitvā

\trline{Has not be꜓en a꜕ttained,}

Vī꜓riyassa sa꜓ṇṭhānaṁ bha꜕vissa꜕tī'ti

\trline{Let no꜕t m꜕y e꜕ff꜕orts st꜕and still.'}

Dukkhaṁ bhi꜓kkh꜕ave kusī꜓to vi꜓ha꜕rati

\begin{english}
  Bhikkhus, the laz꜓y person dwe꜕lls i꜕n su꜓ffe꜕ring,
\end{english}

Voki꜕ṇṇo pāpa꜕kehi a꜕ku꜕saleh꜕i dhammehi

\begin{english}
  Soiled by e꜕vi꜕l, u꜕nwho꜕leso꜕me states
\end{english}

Maha꜓ntañca sa꜕da꜕tthaṁ pa꜕ri꜓hāpeti

\begin{english}
  And great is th꜓e personal go꜕od tha꜕t h꜕e ne꜕glects.
\end{english}

\ifaivedition
\clearpage
\fi

Āraddha-vī꜓riyo c꜕a kho bhi꜓kkh꜕ave su꜕khaṁ vi꜓ha꜕rati

\begin{english}
  The ene꜓rgetic pe꜕rs꜕on tho꜕ugh dw꜕ells ha꜕ppi꜕ly,
\end{english}

Pa꜕vivitto pāpa꜕keh꜕i a꜕ku꜕saleh꜕i dhammehi

\begin{english}
  Well withdrawn from unwho꜓leso꜕me states
\end{english}

Maha꜓ntañca sa꜕da꜕tthaṁ pa꜕ri꜓pūreti

\begin{english}
  And great is th꜓e personal go꜕od tha꜕t h꜕e a꜕chieves.
\end{english}

Na bhi꜓kkh꜕ave hī꜕nena a꜕gga꜕ssa꜕ pa꜕tt꜓i hoti

\begin{english}
  Bhikkhus, it i꜓s not by lo꜕we꜕r means that the supre꜕me i꜕s a꜕ttained
\end{english}

Aggena ca kho bhi꜓kkh꜕ave a꜕gga꜕ssa꜕ pa꜕tt꜓i hoti

\begin{english}
  But, bhikkhus, it is by th꜓e su꜕preme that the supre꜕me i꜕s a꜕ttained.
\end{english}

Maṇḍape꜓yyam-i꜓daṁ bhi꜓kkh꜕ave brahmaca꜕ri꜓yaṁ

\begin{english}
  Bhikkhus, this h꜓ol꜕y life is like the cre꜕am o꜕f t꜕he milk:
\end{english}

Satthā sammukhī꜓-bhū꜕to

\begin{english}
  The Te꜕ach꜕er i꜕s pr꜕esent,
\end{english}

Tasmāti꜕ha bhi꜓kkh꜕ave vī꜓riyaṁ ārabha꜕tha

\begin{english}
  Therefore, bh꜕ikkhus, sta꜕rt t꜕o a꜕rou꜕se your e꜓ne꜕rgy
\end{english}

A꜕ppa꜕tta꜕ssa꜕ pa꜕tt꜓iyā

\begin{english}
  For the a꜓ttainment of the as ye꜕t u꜕na꜕ttained,
\end{english}

\ifaivedition
\clearpage
\fi

Anadhi꜓ga꜕tassa a꜕dhiga꜕māya

\begin{english}
  For the a꜓chievement of the as ye꜕t u꜕na꜕chieved,
\end{english}

Asa꜕cchi꜕ka꜕tassa sa꜕cchi꜕ki꜕ri꜓yāya

\begin{english}
  For the reali꜓zation of the as ye꜕t u꜕nre꜕alized.
\end{english}

Evaṁ no ayaṁ amhākaṁ pa꜕bb꜕ajjā a꜕vaṅka꜕tā a꜕vañjhā bha꜕vi꜓ssati

\begin{english}
  Thinking, in su꜕ch a꜕ way: `Our Go꜓i꜕ng Forth will no꜕t b꜕e ba꜕rren
\end{english}

Sa꜕phalā s꜕a-u꜕dra꜓yā

\begin{english}
  But will be꜓come fru꜕itfu꜕l an꜕d fe꜕rtile,
\end{english}

Yesa꜓ṁ mayaṁ pa꜕ribhuñjāma cīva꜓ra-piṇḍa꜕pāta-se꜓nāsana-\\
gi꜓lānappa꜕ccaya-bhesa꜕jja-parikkhā꜓raṁ tesaṁ te kārā a꜕mhesu

\begin{english}
  And all our us꜕e o꜕f robes, a꜕lmsfood, l꜕odgings, and medici꜕nal
  re꜓qui꜕sites, given by o꜕th꜕ers fo꜕r ou꜕r su꜕pport,
\end{english}

Ma꜕happh꜕alā bhavissanti ma꜕hāni꜕sa꜓ṁsā'ti

\begin{english}
  Will rewa꜕rd th꜕em wi꜕th gre꜕at fruit and great be꜓ne꜕fit.'
\end{english}

Evaṁ hi꜕ vo bhi꜓kkh꜕ave si꜕kkh꜕it꜕abbaṁ

\begin{english}
  Bhikkhus, you should tra꜕in yo꜕urse꜕lves thus:
\end{english}

A꜕tt'atthaṁ vā hi bhi꜓kk꜕have sa꜓mpassa꜕mānena

\begin{english}
  Co꜓nsidering your ow꜕n good,
\end{english}

A꜕lam-eva a꜕ppamādena sa꜓mpādetuṁ

\begin{english}
  It is e꜓nough to str꜕ive fo꜕r th꜕e go꜕al wi꜕tho꜕ut ne꜕gligence;
\end{english}

Pa꜕r'atthaṁ vā hi bhi꜓kkh꜕ave sa꜓mpass꜕amānena

\begin{english}
  Bhikkhus, co꜓nsidering the go꜕od o꜕f o꜕thers,
\end{english}

A꜕lam-eva a꜕ppamādena sa꜓mpāde꜕tuṁ

\begin{english}
  It is e꜓nough to str꜕ive fo꜕r th꜕e go꜕al wi꜕tho꜕ut ne꜕gligence;
\end{english}

U꜕bhay'atthaṁ vā hi bhi꜓kkh꜕ave sa꜓mpassa꜕mānena

\begin{english}
  Bhikkhus, co꜓nsidering the go꜕od o꜕f both,
\end{english}

Alam-eva a꜕ppamādena sa꜓mpāde꜕tun'ti

\begin{english}
  It is e꜓nough to str꜕ive fo꜕r th꜕e go꜕al wi꜕tho꜕ut ne꜕gligence.
\end{english}

\chapter{The Verses of Tāyana}

\firstline{Chinda sotaṁ parakkamma}

\begin{leader}
  [Handa mayaṁ tāyana-gāthāyo bhaṇāmase]
\end{leader}

\begin{twochants}
  Chi꜓nda so꜕taṁ pa꜕rakkamma & kā꜕me panūda brā꜓hm꜕aṇa \\
  Nappahā꜓ya mu꜕ni kāme & n'ekattam-upa꜕pajja꜕ti \\
\end{twochants}

\begin{english}
  Exert yourself a꜕nd cut t꜕he stream.\\
  Discard sense-pl꜓easu꜕res, Holy꜕ Man;\\
  Not letting sensu꜕al pleasu꜕res go,\\
  A sage will no꜓t re꜕ach uni꜕ty.
\end{english}

\begin{twochants}
  Kayirā ce ka꜕yirāthe꜓naṁ & da꜕ḷham-enaṁ pa꜕rakka꜕me \\
  Sithilo hi pa꜕ribbājo & bh꜕iyyo ākira꜕te ra꜕jaṁ \\
\end{twochants}

\begin{english}
  Vigorously, wi꜕th all on꜕e's strength,\\
  It should be do꜓ne, wh꜕at should b꜕e done;\\
  A lax monast꜕ic life sti꜕rs up\\
  The dust of pa꜓ssio꜕ns all th꜕e more.
\end{english}

\begin{twochants}
  A꜕kataṁ dukkaṭaṁ se꜓yyo & pacchā tappati du꜓kk꜕aṭaṁ \\
  Katañca su꜕ka꜓taṁ seyyo & yaṁ ka꜕tvā nānuta꜕ppa꜕ti \\
\end{twochants}

\begin{english}
  Better is not t꜕o do ba꜕d deeds\\
  That afterwa꜓rds wo꜕uld bring re꜕morse;\\
  It's rather go꜓od de꜕eds one sho꜕uld do\\
  Which having done on꜕e won't re꜕gret.
\end{english}

\clearpage

\begin{twochants}
  Kuso꜓ ya꜕thā du꜕ggahi꜕to & hattham-evā꜓nu꜕kant꜕ati \\
  Sā꜓maññaṁ du꜕pparāma꜕ṭṭhaṁ & nirayāyūpa꜕kaḍḍh꜕ati \\
\end{twochants}

\begin{english}
  As Kusa-grass, wh꜕en wrongly꜕ grasped,\\
  Will only cu꜓t i꜕nto on꜕e's hand\\
  So does th꜕e monk's lif꜕e wrongl꜕y led\\
  Indeed drag on꜓e t꜕o hell꜕ish states.
\end{english}

\begin{twochants}
  Yaṁ kiñci si꜕thi꜓laṁ kammaṁ & sa꜓ṅki꜕liṭṭha꜓ñca꜕ yaṁ va꜓taṁ \\
  Sa꜓ṅka꜕ssaraṁ brahma-ca꜕ri꜓yaṁ & na taṁ ho꜓ti ma꜕happh꜕alan'ti \\
\end{twochants}

\begin{english}
  Whateve꜕r deed tha꜕t's slackl꜕y done,\\
  Whatever vo꜓w co꜕rruptl꜕y kept,\\
  The Holy Life le꜕d in꜕ doubtf꜕ul ways ---\\
  All these will ne꜓ve꜕r bear gr꜕eat fruit.
\end{english}

% End of reflections-and-recollections-p3.tex
