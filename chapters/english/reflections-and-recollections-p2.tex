% vim: foldmethod=marker foldlevel=0 foldtext=FoldText()

\chapter[Off-Putting Qualities of the Requisites]{Reflection on the Off-Putting Qualities of the Requisites}% {{{1

\firstline{Yathā paccayaṁ pavattamānaṁ}

\begin{leader}
  [Ha꜓nda mayaṁ dhātu-paṭikūla-paccavekkhaṇa-pāṭhaṁ bhaṇāmase]
\end{leader}

%\suttaref{Trad.}%
[Yathā꜓ pa꜕ccayaṁ] pava꜓tt꜕amānaṁ dhātu꜕-ma꜓tta꜕m-ev'etaṁ

\trline{Composed of only e꜓lements acco꜕rdin꜕g to꜕ ca꜕use꜕s an꜕d co꜕ndi꜕tions}

Yad i꜓daṁ cī꜓varaṁ ta꜕d upa꜕bhuñja꜓ko c꜕a pu꜕gga꜕lo

\trline{Are these ro꜕bes an꜕d so꜕ is꜕ th꜕e pe꜕rso꜕n we꜕aring them;}

Dhātu-ma꜓tta꜕ko

\trline{Merely e꜕lements,}

Ni꜓ssa꜕tto

\trline{Not a be꜕ing,}

Ni꜓jjīvo

\trline{Without a꜕ soul}

Su꜓ñño

\trline{And e꜕mpty꜕ o꜕f self.}

S꜕abbāni pa꜕na imāni cī꜓varāni a꜕jigu꜓ccha꜕nīyāni

\trline{None of th꜓ese robes ar꜕e inna꜕tel꜕y re꜕pu꜕lsive}

\clearpage

Imaṁ pūti꜓-kāyaṁ pa꜕tvā

\trline{But touching this u꜓nclean bo꜕dy}

A꜕tiviya jigu꜓ccha꜕nīyāni jāyanti

\trline{They beco꜕me di꜕sgu꜕sti꜕ng in꜕deed.}

Yathā꜓ pa꜕ccayaṁ pava꜓tt꜕amānaṁ dhātu꜕-ma꜓tta꜕m-ev'etaṁ

\trline{Composed of only e꜓lements acco꜕rdin꜕g to꜕ ca꜕use꜕s an꜕d co꜕ndi꜕tions}

Yad i꜓daṁ piṇḍa꜓pāto ta꜕d upa꜕bhuñja꜓ko c꜕a pu꜕gga꜕lo

\trline{Is this a꜕lmsfo꜕od an꜕d s꜕o i꜕s th꜕e pe꜕rso꜕n ea꜕ting it;}

Dhātu-ma꜓tta꜕ko

\trline{Merely e꜕lements,}

Ni꜓ssa꜕tto

\trline{Not a be꜕ing,}

Ni꜓jjīvo

\trline{Without a꜕ soul}

Su꜓ñño

\trline{And e꜕mpty꜕ o꜕f self.}

S꜕abbo pa꜕nāyaṁ piṇḍa꜓pāto a꜕jigu꜓ccha꜕nīyo

\trline{None of th꜓is almsfood is inna꜕tel꜕y re꜕pu꜕lsive}

\clearpage

Imaṁ pūti꜓-kāyaṁ pa꜕tvā

\trline{But touching this u꜓nclean bo꜕dy}

A꜕tiviya jigu꜓ccha꜕nīyo jāyati

\trline{It beco꜕mes di꜕sgu꜕sti꜕ng in꜕deed.}

Yathā꜓ pa꜕ccayaṁ pava꜓tt꜕amānaṁ dhātu꜕-ma꜓tta꜕m-ev'etaṁ

\trline{Composed of only e꜓lements acco꜕rdin꜕g to꜕ ca꜕use꜕s an꜕d co꜕ndi꜕tions}

Yad i꜓daṁ senā꜓sanaṁ ta꜕d upa꜕bhuñja꜓ko c꜕a pu꜕gga꜕lo

\trline{Is this dwe꜕lli꜕ng an꜕d s꜕o i꜕s th꜕e pe꜕rso꜕n u꜕sing it;}

Dhātu-ma꜓tta꜕ko

\trline{Merely e꜕lements,}

Ni꜓ssa꜕tto

\trline{Not a be꜕ing,}

Ni꜓jjīvo

\trline{Without a꜕ soul}

Su꜓ñño

\trline{And e꜕mpty꜕ o꜕f self.}

S꜕abbāni pa꜕na imāni senā꜓sanāni a꜕jigu꜓ccha꜕nīyāni

\trline{None of the꜓se dwellings are inna꜕tel꜕y re꜕pu꜕lsive}

\clearpage

Imaṁ pūti꜓-kāyaṁ pa꜕tvā

\trline{But touching this u꜓nclean bo꜕dy}

A꜕tiviya jigu꜓ccha꜕nīyāni jāyanti

\trline{They beco꜕me di꜕sgu꜕sti꜕ng in꜕deed.}

Yathā꜓ pa꜕ccayaṁ pava꜓tt꜕amānaṁ dhātu꜕-ma꜓tta꜕m-ev'etaṁ

\trline{Composed of only e꜓lements acco꜕rdin꜕g to꜕ ca꜕use꜕s an꜕d co꜕ndi꜕tions}

Yad i꜓daṁ gi꜕lāna-pacca꜕ya꜕-bhesajja-pa꜕rikkhāro ta꜕d upa꜕bhuñja꜓ko c꜕a pu꜕gga꜕lo

\trline{Is this m꜕e꜕di꜕ci꜕na꜕l requ꜕is꜕ite an꜕d s꜕o i꜕s th꜕e pe꜕rso꜕n tha꜕t ta꜕kes it;}

Dhātu-ma꜓tta꜕ko

\trline{Merely e꜕lements,}

Ni꜓ssa꜕tto

\trline{Not a be꜕ing,}

Ni꜓jjīvo

\trline{Without a꜕ soul}

Su꜓ñño

\trline{And e꜕mpty꜕ o꜕f self.}

S꜕abbo pa꜕nāyaṁ gi꜕lāna-pacca꜕ya꜕-bhesajja-pa꜕rikkhāro a꜕jigu꜓ccha꜕nīyo

\trline{None of th꜓is medicinal re꜕qui꜕si꜕te is inna꜕tel꜕y re꜕pu꜕lsive}

Imaṁ pūti꜓-kāyaṁ pa꜕tvā

\trline{But touching this u꜓nclean bo꜕dy}

A꜕tiviya jigu꜓ccha꜕nīyo jāyati

\trline{It beco꜕mes di꜕sgu꜕sti꜕ng in꜕deed.}

\artopttrue

\chapter{Reflection on Impermanence}% {{{1

\firstline{Sabbe saṅkhārā aniccā}

\begin{leader}
  [Handa mayaṁ aniccānussati-pāṭhaṁ bhaṇāmase]
\end{leader}

[Sa꜕bbe sa꜓ṅkhā꜓rā a꜕ni꜓ccā]

\trline{All conditioned things are impe꜕rmanent;}

Sa꜕bbe sa꜓ṅkhā꜓rā du꜕kkhā

\trline{All conditioned things are du꜕kkha;}

Sa꜕bbe dhammā a꜕na꜓ttā

\trline{Everything is vo꜕id o꜕f self.}

A꜕ddhuvaṁ jīvi꜓taṁ

\trline{Life is no꜕t fo꜕r sure;}

Dhuvaṁ ma꜓ra꜕ṇaṁ

\trline{Dea꜕th i꜕s fo꜕r sure;}

A꜕vassaṁ mayā mari꜓ta꜕bbaṁ

\trline{It is i꜓nevitable tha꜕t I꜕'ll die;}

Ma꜕raṇa-pa꜕riyosā꜓naṁ me jīvi꜓taṁ

\trline{Death is th꜓e culmina꜕ti꜕on o꜕f my꜕ life;}

Jīvitaṁ me ani꜓ya꜕taṁ

\trline{My life is unce꜕rtain;}

Maraṇaṁ me ni꜓ya꜕taṁ

\trline{My dea꜕th is꜕ ce꜕rtain.}

Vata

\trline{I꜕ndeed,}

A꜕yaṁ kāyo

\trline{This bo꜕dy}

A꜕ciraṁ

\trline{Wi꜕ll soon}

A꜕peta꜕-viññāṇo

\trline{Be void of co꜓nsci꜕ousness}

Chu꜕ddho

\trline{And ca꜕st a꜕way.}

A꜕dhise꜓ssa꜕ti

\trline{I꜕t wi꜕ll lie}

\ifaivedition
\clearpage
\fi

Pa꜕ṭha꜕viṁ

\trline{On꜕ th꜕e ground}

Ka꜕liṅga꜓raṁ i꜕va

\trline{Just like a ro꜓tten꜕ log,}

Ni꜕ratthaṁ

\trline{Comple꜕tel꜕y vo꜕id o꜕f use.}

Aniccā vata sa꜓ṅkhā꜓rā

\trline{Truly co꜓nditioned thin꜕gs ca꜕nno꜕t last,}

U꜕ppāda-vaya-dha꜓mmi꜕no

\trline{Their nature is to ri꜕se an꜕d fall,}

U꜕ppajjitvā nirujjh꜓anti

\trline{Having a꜓risen thi꜕ngs mu꜕st cease,}

Tesa꜓ṁ vūpa꜕sa꜕mo sukho

\trline{Their st꜕illin꜕g i꜕s tr꜕ue ha꜕ppiness.}

\artoptfalse

\chapter{True and False Refuges}% {{{1

\firstline{Bahuṁ ve saraṇaṁ yanti}

\begin{leader}
  [Ha꜓nda mayaṁ khemākhema-sa꜕raṇa-gamana-\\
  -pa꜕ridīpikā-gāthā꜓yo bha꜕ṇāmase]
\end{leader}

\begin{twochants}
  Bahuṁ ve sa꜕ra꜓ṇaṁ yanti꜕ & pa꜕bba꜕tāni va꜕nāni꜓ ca \\
  Ārāma-rukkha꜕-cetyāni & manussā꜓ bha꜕ya꜕-tajji꜕tā \\
\end{twochants}

\begin{english}
  To many re꜕fu꜕ge꜕s th꜕ey go ---\\
  To mountain slopes and fo꜓re꜕st glades,\\
  To pa꜕rkla꜕nd shri꜕nes an꜕d sa꜕cre꜕d sites ---\\
  People overco꜓me by꜕ fear.
\end{english}

\begin{twochants}
  N'etaṁ kho sa꜕ra꜓ṇaṁ khemaṁ & n'etaṁ sa꜕raṇam-u꜓tt꜕amaṁ \\
  N'etaṁ sa꜕raṇam-āgamma & sa꜕bba-dukkhā꜓ pa꜕mucca꜕ti \\
\end{twochants}

\begin{english}
  Such a refuge is no꜕t se꜕cure,\\
  Such a refuge is no꜓t su꜕preme,\\
  Such a꜓ refuge do꜕es no꜕t bring\\
  Complete release from suf꜕fe꜕ring.
\end{english}

\begin{twochants}
  Yo ca꜕ Buddhañca꜕ Dhammañca꜕ & sa꜓ṅghañca꜕ sa꜓ra꜕ṇaṁ ga꜕to \\
  Ca꜕ttāri a꜕riya-saccāni & sa꜕mmappaññāya꜓ pa꜕ss꜕ati \\
\end{twochants}

\begin{english}
  Whoe꜕ve꜕r go꜕es t꜕o r꜕efuge\\
  In the Tr꜓iple꜕ Gem\\
  Sees with ri꜓ght disce꜕rnment\\
  The Fo꜕ur No꜕bl꜕e Truths:
\end{english}

\begin{twochants}
  Dukkhaṁ dukkha-sa꜕muppādaṁ & dukkhassa ca꜕ a꜕ti꜕kka꜕maṁ \\
  A꜕riyañ-c'a꜕ṭṭh'a꜓ṅgi꜕kaṁ maggaṁ & dukkhūpasa꜕ma꜕-gāmi꜓naṁ \\
\end{twochants}

\begin{english}
  Suffering an꜕d it꜕s o꜕rigin\\
  And that which li꜓es be꜕yond ---\\
  The Nob꜕le E꜕ightfo꜕ld Path\\
  That leads th꜓e way to su꜕ff'r꜕ing's end.
\end{english}

\begin{twochants}
  Etaṁ kho sa꜕ra꜓ṇaṁ khemaṁ & etaṁ sa꜕raṇam-u꜓tta꜕maṁ \\
  Etaṁ sa꜕raṇam-āgamma & sa꜕bba-dukkhā꜓ pa꜕mucca꜕ti \\
\end{twochants}

\begin{english}
  Such a꜓ refuge i꜕s se꜕cure,\\
  Such a refuge i꜓s su꜕preme,\\
  Such a refuge tr꜕uly꜕ brings\\
  Complete r꜓elease from all su꜕ffe꜕ring.
\end{english}

\chapter{Verses on the Riches of a Noble One}% {{{1

\firstline{Yassa saddhā tathāgate}

\begin{leader}
  [Ha꜓nda mayaṁ a꜕riya-dhana-gāthā꜓yo bha꜕ṇāmase]
\end{leader}

\begin{twochants}
  Yassa꜕ sa꜕ddhā tathā꜓ga꜕te & a꜕ca꜕lā su꜕pa꜕tiṭṭhi꜓tā \\
  Sī꜓lañca꜕ yassa꜕ kalyāṇaṁ & a꜕riya-kantaṁ pasa꜓ṁsi꜕taṁ \\
\end{twochants}

\begin{english}
  One whose faith in the Tathā꜕gata\\
  Is unshaken and esta꜓bli꜕shed well,\\
  Whose virtue is be꜕autiful,\\
  The Noble Ones enjo꜓y an꜕d praise;
\end{english}

\begin{twochants}
  Sa꜓ṅghe pa꜕sā꜕do yass'atthi & uju-bhūtañca da꜓ss꜕anaṁ \\
  A꜕daliddo't꜕i taṁ āhu꜕ & a꜕moghaṁ ta꜕ssa꜕ jīvi꜓taṁ \\
\end{twochants}

\begin{english}
  Whose trust is in꜕ th꜕e Sa꜕ṅgha,\\
  Who sees things rightly a꜓s th꜕ey are,\\
  It is sa꜕id th꜕at no꜕t i꜕n vain\\
  And undeluded i꜓s th꜕eir life.
\end{english}

\begin{twochants}
  Tasmā sa꜕ddhañca꜕ sī꜓lañca꜕ & pasādaṁ dhamma-da꜓ssa꜕naṁ \\
  A꜕nuyuñjetha medhāvī & sa꜕raṁ buddhāna sā꜓sa꜕naṁ \\
\end{twochants}

\begin{english}
  To virtu꜓e and to꜕ faith,\\
  To trust to se꜓ein꜕g truth,\\
  To these the wise devo꜕te th꜕emselves,\\
  The Buddha꜓'s teaching in꜕ th꜕eir mind.
\end{english}

\chapter{Verses on the Three Characteristics}% {{{1

\firstline{Sabbe saṅkhārā aniccā'ti}

\begin{leader}
  [Ha꜓nda mayaṁ ti-lakkhaṇ'ādi-gāthā꜓yo bha꜕ṇāmase]
\end{leader}

\begin{twochants}
  Sa꜕bbe sa꜓ṅkhā꜓rā a꜕ni꜓ccā't꜕i & yadā paññāya꜓ pa꜕ssa꜕ti \\
  Atha nibbinda꜕ti dukkhe & esa꜕ maggo vi꜓su꜕ddh꜓iyā \\
\end{twochants}

\begin{english}
  `Impermanent are all condi꜕tio꜕ned things' ---\\
  When with wisdom th꜓is i꜕s seen\\
  One feels we꜕ary꜕ o꜕f a꜕ll du꜕kkha;\\
  This is the path to pu꜓r꜕ity.
\end{english}

\begin{twochants}
  Sa꜕bbe sa꜓ṅkhā꜓rā du꜕kkhā't꜕i & yadā paññāya꜓ pa꜕ssa꜕ti \\
  Atha nibbinda꜕ti dukkhe & esa꜕ maggo vi꜓su꜕ddh꜓iyā \\
\end{twochants}

\begin{english}
  `Dukkha are all condi꜕tio꜕ned things' ---\\
  When with wisdom th꜓is i꜕s seen\\
  One feels we꜕ary꜕ o꜕f a꜕ll du꜕kkha;\\
  This is the path to pu꜓r꜕ity.
\end{english}

\begin{twochants}
  Sa꜕bbe dhammā ana꜓ttā'ti꜕ & yadā paññāya꜓ pa꜕ssa꜕ti \\
  Atha nibbinda꜕ti dukkhe & esa꜕ maggo vi꜓su꜕ddh꜓iyā \\
\end{twochants}

\begin{english}
  `There is no self in a꜕nything' ---\\
  When with wisdom th꜓is i꜕s seen\\
  One feels we꜕ary꜕ o꜕f a꜕ll du꜕kkha;\\
  This is the path to pu꜓ri꜕ty.
\end{english}

\clearpage

\begin{twochants}
  A꜕ppa꜕kā te manusse꜓su꜕ & ye janā pāra-gāmi꜓no \\
  A꜕thāyaṁ i꜕ta꜕rā pajā & tīram-evānudhā꜓va꜕ti \\
\end{twochants}

\begin{english}
  Few amongst huma꜕nkind\\
  Are those who g꜓o b꜕eyond,\\
  Yet there are the ma꜕ny folks\\
  Ever wand'ring o꜕n th꜕is shore.
\end{english}

\begin{twochants}
  Ye ca꜕ kho sammad-akkhāte & dhamme dhammānuva꜓tt꜕ino \\
  Te ja꜕nā pā꜕ram-essanti & ma꜕ccu-dheyyaṁ sudu꜓tta꜕raṁ \\
\end{twochants}

\begin{english}
  Wherever Dha꜕mm꜕a i꜕s we꜕ll-taught,\\
  Those who train in li꜓ne wi꜕th it\\
  Are the ones who wi꜕ll cr꜕oss o꜕ver\\
  The realm o꜓f death so ha꜕rd t꜕o flee.
\end{english}

\begin{twochants}
  Kaṇhaṁ dhammaṁ vi꜕ppahā꜓ya & su꜕kkaṁ bhāvetha꜕ paṇḍi꜓to \\
  Okā a꜕noka꜕m-āgamma & viveke ya꜕tth꜕a dūramaṁ \\
  Ta꜕trābh꜕irat꜕im-iccheyya & hi꜕tvā kāme a꜕kiñc꜓ano \\
\end{twochants}

\begin{english}
  Abandoning the da꜕rke꜕r states,\\
  The wise pursu꜓e th꜕e bright;\\
  From the floo꜕ds dr꜕y la꜕nd th꜕ey reach\\
  Living wi꜓thdrawn so ha꜕rd t꜕o do.\\
  Such rare de꜓light on꜕e sho꜕uld de꜕si꜕re,\\
  Sense pleasu꜓res cast awa꜕y,\\
  No꜕t ha꜕vin꜕g a꜕nything.
\end{english}

\chapter{Verses on the Burden}% {{{1

\firstline{Bhārā have pañcakkhandhā}

\begin{leader}
  [Ha꜓nda mayaṁ bhāra-su꜕tta-gāthā꜓yo bha꜕ṇāmase]
\end{leader}

\begin{twochants}
Bhārā ha꜕ve pañcakkha꜓ndhā & bhāra-hāro ca pu꜓gga꜕lo \\
Bhā꜕r'ādānaṁ du꜕kkhaṁ loke꜓ & bhāra-nikkhe꜓pa꜕naṁ su꜕khaṁ \\
\end{twochants}

\begin{english}
  The five aggregates inde꜕ed ar꜕e bu꜕rdens,\\
  The beast of burden tho꜓ugh i꜕s man.\\
  In this world to ta꜕ke u꜕p bu꜕rde꜕ns i꜕s du꜕kkha.\\
  Putting the꜓m down brings ha꜓ppi꜕ness.
\end{english}

\begin{twochants}
Nikkhipi꜕tvā ga꜕ruṁ bhā꜓raṁ & aññaṁ bhāraṁ anā꜓di꜕ya \\
Sa꜕mūlaṁ taṇhaṁ a꜕bbuyha & nicchāto pa꜕ri꜕nibbu꜕to \\
\end{twochants}

\begin{english}
  A heavy burden ca꜕st a꜕way,\\
  Not taking on ano꜓th꜕er load,\\
  With cravi꜓ng pulled out fro꜕m th꜕e root,\\
  Desire꜓s stilled, on꜕e i꜕s re꜕leased.
\end{english}

\chapter{Verses on a Shining Night of Prosperity}% {{{1

\firstline{Atītaṁ nānvāgameyya}

\begin{leader}
  [Ha꜓nda mayaṁ bhadd'eka-ratta꜕-gāthā꜓yo bha꜕ṇāmase]
\end{leader}

\begin{twochants}
  A꜕tītaṁ nānvāga꜕meyya & nappa꜕ṭikaṅkhe꜓ a꜕nāga꜓taṁ \\
  Ya꜕d'a꜕tītaṁ pa꜕hīnan-taṁ & a꜕ppattañc꜕a a꜕nāga꜕taṁ \\
\end{twochants}

\begin{english}
  One should not revi꜕ve th꜕e past\\
  Nor speculate on wha꜓t's t꜕o come;\\
  The past is l꜕eft be꜕hind,\\
  The futu꜓re is un-r꜓ea꜕lized.
\end{english}

\begin{twochants}
  Paccu꜕ppannañca꜕ yo dhammaṁ & tattha tattha vi꜓pa꜕ss꜕ati \\
  Asa꜓ṁhi꜕raṁ asa꜓ṅku꜕ppaṁ & taṁ viddhām-a꜕nu꜕brūhaye \\
\end{twochants}

\begin{english}
  In every presently ari꜕se꜕n state\\
  There, just there, one cle꜓arly꜕ sees;\\
  Unmoved, una꜕gi꜕ta꜕ted,\\
  Such insight i꜓s on꜕e's strength.
\end{english}

\begin{twochants}
  A꜕jj'eva ki꜕cca꜕m-ātappaṁ & ko jaññā ma꜓ra꜕ṇaṁ su꜕ve \\
  Na hi no sa꜓ṅga꜕ran-tena & mahā-senena ma꜓cc꜕unā \\
\end{twochants}

\begin{english}
  Ardently doing one's ta꜕sk t꜕oday,\\
  Tomorrow, who knows, de꜓ath ma꜕y come;\\
  Facing the mighty ho꜕rdes o꜕f death,\\
  Indeed o꜓ne cannot str꜕ike a꜕ deal.
\end{english}

\clearpage

\begin{twochants}
  Evaṁ vihārim-ātāpiṁ & a꜕ho-rattam-a꜕tandi꜓taṁ \\
  Taṁ ve bha꜕dd'eka꜕-ratto'ti & santo ā꜕ci꜕kkha꜕te muni \\
\end{twochants}

\begin{english}
  To dwell with e꜕ne꜕rgy꜕ a꜕roused\\
  Thus for a night of no꜓n-de꜕cline,\\
  That is a `night of shi꜕ni꜕ng pr꜕osperity.'\\
  So it was taught by the Peacefu꜕l Sage.
\end{english}

\chapter{Verses on Respect for the Dhamma}% {{{1

\firstline{Ye ca atītā sambuddhā}

\begin{leader}
  [Ha꜓nda mayaṁ dhamma-gā꜕rav'ādi꜕-gāthā꜓yo bha꜕ṇāmase]
\end{leader}

\begin{twochants}
  Ye ca꜕ atītā sa꜓mbuddhā & ye ca꜕ buddhā a꜕nāga꜓tā \\
  Yo c'eta꜕rahi sambuddho & ba꜕hunnaṁ so꜕ka꜕-nāsa꜕no \\
\end{twochants}

\begin{english}
  All the Buddhas o꜕f th꜕e past,\\
  All the Buddhas ye꜓t to꜕ come,\\
  The Buddha of this cu꜕rre꜕nt age ---\\
  Dispe꜓llers of mu꜕ch so꜕rrow.
\end{english}

\begin{twochants}
  Sa꜕bbe sa꜕ddhamma-gar꜓uno & vi꜕ha꜕riṁsu vi꜕ha꜕ranti ca \\
  A꜕tho pi viha꜕riss꜓anti & esā buddhāna꜓ dha꜕mma꜕tā \\
\end{twochants}

\begin{english}
  Those having lived or li꜕vi꜕ng now,\\
  Those livi꜓ng in the fu꜕ture,\\
  All do reve꜕re th꜕e Tru꜕e Dha꜕mma ---\\
  That is th꜓e nature o꜕f a꜕ll Bu꜕ddhas.
\end{english}

\begin{twochants}
  Tasmā h꜕i atta-kāmena & mahattam-abhika꜓ṅkh꜕atā \\
  Sa꜕ddhammo ga꜕ru꜓-kāta꜕bbo & s꜕araṁ buddhāna sā꜓sa꜕naṁ \\
\end{twochants}

\begin{english}
  Therefore de꜓siring on꜕e's ow꜕n welfare,\\
  Pursui꜓ng greatest a꜕spi꜕ra꜕tions,\\
  One should reve꜕re th꜕e Tr꜕ue Dha꜕mma ---\\
  Reco꜓llecting th꜕e Bu꜕ddha꜕'s te꜕aching.
\end{english}

\clearpage

\begin{twochants}
  Na h꜕i dhammo a꜕dhammo ca & ubho s꜕ama-vipāki꜓no \\
  A꜕dhammo nirayaṁ neti & dh꜕ammo pāpeti꜕ su꜕gga꜕tiṁ \\
\end{twochants}

\begin{english}
  What is true Dhamma an꜕d wh꜕at not\\
  Will never have the sa꜓me re꜕sults,\\
  While lack of Dha꜕mma꜕ le꜕ads t꜕o he꜕ll-realms ---\\
  True Dhamma꜓ takes one o꜕n a꜕ go꜕od course.
\end{english}

Dhammo ha꜕ve rakkha꜕ti꜕ dhamma꜓-cāriṁ\\
Dhammo su꜕ciṇṇo su꜕kham-āvahāti\\
Esā꜓ni꜕saṁso dhamme su꜕ciṇṇe\\
Na du꜕gga꜕tiṁ gaccha꜕ti꜕ dhamma꜓-cārī

\begin{english}
  The Dhamma guards who li꜕ves i꜕n li꜓ne wi꜕th it\\
  And leads to ha꜕ppi꜕ne꜕ss whe꜕n pra꜕cti꜕sed well ---\\
  This is th꜓e blessing of we꜕ll-pr꜕acti꜕sed Dha꜕mma.
\end{english}

\chapter{Verses on the Training Code}% {{{1

\firstline{Sabba-pāpassa akaraṇaṁ}

\begin{leader}
  [Ha꜓nda mayaṁ ovāda-pā꜕ṭi꜕mokkha-gāthā꜓yo bha꜕ṇāmase]
\end{leader}

\begin{instruction}
  Version One
\end{instruction}

Sa꜕bb꜕a-pāpa꜕ss꜕a a꜕ka꜕ra꜓ṇaṁ

\begin{english}
  Avoidance of all e꜕vil ways;
\end{english}

Ku꜕salassūpasa꜓mpa꜕dā

\begin{english}
  Commitment to what's wh꜓olly good;
\end{english}

Sa꜕ci꜕tta-pa꜕ri꜓yoda꜓pa꜕naṁ

\begin{english}
  Purifica꜕tion of one's mind:
\end{english}

Etaṁ buddhāna sā꜓sa꜕naṁ

\begin{english}
  Just this is what the Bu꜓ddhas teach.
\end{english}

Kha꜓ntī pa꜕ramaṁ ta꜕po tīti꜕kkhā

\begin{english}
  Pa꜕tience is the cl꜕eansing flame;
\end{english}

Nibbānaṁ pa꜕ramaṁ va꜕dant꜕i buddhā

\begin{english}
  Nibbāna's supre꜓me, the Bu꜓ddhas say.
\end{english}

Na h꜕i pa꜕bbaji꜕to pa꜕rūpaghātī

\begin{english}
  Ha꜕rming others, you're n꜓o recluse;
\end{english}

Sa꜕maṇo ho꜓ti pa꜕raṁ vihe꜓ṭha꜕yanto

\begin{english}
  A trouble-maker's no꜕ samana.
\end{english}

A꜕nūpa꜕vādo a꜕nūpa꜕ghāto

\begin{english}
  To neither insult nor cau꜕se wounds;
\end{english}

Pā꜕ṭimokkhe꜓ ca꜕ sa꜓ṁva꜕ro

\begin{english}
  To live restrai꜓ned by training rules;
\end{english}

Mattaññu꜕tā ca꜕ bhatta꜕smiṁ

\begin{english}
  To know what's eno꜓ugh when taking food;
\end{english}

Pa꜕ntañca꜕ saya꜓n'āsa꜕naṁ

\begin{english}
  To dw꜕ell alone in a qu꜓iet place;
\end{english}

A꜕dhici꜕tte ca꜕ āyogo

\begin{english}
  And devo꜕tion to the hi꜓gher mind:
\end{english}

Etaṁ buddhāna sā꜓sa꜕naṁ

\begin{english}
  Every Buddha te꜓aches this.
\end{english}

\clearpage

\begin{instruction}
  Version Two
\end{instruction}

\firstline{Sabba-pāpassa akaraṇaṁ}

Sabba-pāpa꜕ss꜕a a꜕ka꜕ra꜓ṇaṁ

\begin{english}
  Not do꜕in꜕g a꜕ny꜕ e꜕vil;
\end{english}

Kusalassūpasa꜓mpa꜕dā

\begin{english}
  To be committed to꜕ th꜕e good;
\end{english}

Sa꜕citta-pa꜕ri꜓yoda꜓pa꜕naṁ

\begin{english}
  To pu꜕ri꜕fy꜕ on꜕e's mind:
\end{english}

Etaṁ buddhāna sā꜓sa꜕naṁ

\begin{english}
  These are th꜓e teachings o꜕f al꜕l Bu꜕ddhas.
\end{english}

Kha꜓ntī pa꜕ramaṁ ta꜕po tīti꜕kkhā

\begin{english}
  Patient e꜓ndurance is the highest pra꜕cti꜕ce, bu꜕rni꜕ng ou꜕t de꜕fi꜕lements;
\end{english}

Nibbānaṁ pa꜕ramaṁ vadant꜕i buddhā

\begin{english}
  The Buddha꜓s say Nibbā꜕na꜕ i꜕s su꜕preme.
\end{english}

Na h꜕i pa꜕bbaji꜕to pa꜕rūpaghātī

\begin{english}
  Not a renu꜕nci꜕ant is꜕ on꜕e wh꜕o in꜕ju꜕res o꜕thers;
\end{english}

Sa꜕maṇo ho꜓ti pa꜕raṁ vihe꜓ṭha꜕yanto

\begin{english}
  Whoever troubl꜓es others ca꜕n't b꜕e ca꜓lled a꜕ monk.
\end{english}

\clearpage

A꜕nūpa꜕vādo a꜕nūpa꜕ghāto

\begin{english}
  Not to insu꜕lt an꜕d no꜕t t꜕o i꜕njure;
\end{english}

Pāṭimokkhe꜓ ca꜕ sa꜓ṁva꜕ro

\begin{english}
  To live restrained by tra꜕ini꜕ng rules;
\end{english}

Mattaññu꜕tā ca꜕ bhatta꜕smiṁ

\begin{english}
  Knowing one's me꜕asure a꜕t t꜕he meal;
\end{english}

Pantañca꜕ saya꜓n'āsa꜕naṁ

\begin{english}
  Retreating to a lo꜓ne꜕ly place;
\end{english}

A꜕dhici꜕tte ca꜕ āyogo

\begin{english}
  Devoti꜓on to the hi꜕ghe꜕r mind:
\end{english}

Etaṁ buddhāna sā꜓sa꜕naṁ

\begin{english}
  These are the tea꜕chi꜕ngs o꜕f al꜕l Bu꜕ddhas.
\end{english}

\chapter{Verses on the Buddha's First Exclamation}% {{{1

\firstline{Aneka-jāti-saṁsāraṁ}

\begin{leader}
  [Ha꜓nda mayaṁ paṭhama-bu꜕ddha-bhāsi꜕ta-gāthāyo bh꜕aṇāmase]
\end{leader}

\begin{twochants}
  A꜕neka꜕-jāti꜕-sa꜓ṁsā꜓raṁ & sa꜕ndhāviss꜓aṁ a꜕nibbi꜕saṁ \\
  Ga꜕ha-kā꜕raṁ ga꜕vesa꜓nto & dukkhā jāt꜕i pu꜕nappu꜕naṁ \\
\end{twochants}

\begin{english}
  For many lifetimes in the ro꜕und o꜕f birth,\\
  Wandering on e꜓ndle꜕ssly,\\
  For the bu꜕ilde꜕r o꜕f th꜕is ho꜕use I꜕ searched ---\\
  How painful is repe꜓ated꜕ birth.
\end{english}

\begin{twochants}
  Ga꜕ha-kā꜕raka꜕ diṭṭho꜓'si & pu꜕na gehaṁ na kā꜓hasi \\
  Sa꜕bbā te phāsu꜕kā bhaggā & gaha-kūṭa꜓ṁ vi꜕saṅkh꜕ataṁ \\
  Visa꜓ṅkhā꜕ra-ga꜕taṁ ci꜕ttaṁ & taṇhānaṁ kh꜕aya꜕m-ajjh꜕agā \\
\end{twochants}

\begin{english}
  House-builder yo꜕u've be꜕en seen,\\
  Another home you w꜓ill no꜕t build,\\
  All your ra꜕fte꜕rs ha꜕ve be꜕en snapped,\\
  Dismantle꜓d is your ri꜕dge-pole;\\
  The non-co꜓nstructing mi꜕nd\\
  Ha꜕s co꜕me to꜕ cra꜕vi꜕ng's end.
\end{english}

\chapter{Verses on the Last Instructions}% {{{1

\firstline{Handa dāni bhikkhave āmantayāmi vo}

\begin{leader}
  [Ha꜓nda mayaṁ pacchima-ovāda-gāthā꜓yo bha꜕ṇāmase]
\end{leader}

Handa dāni bhi꜓kkha꜕ve āmant꜕ayāmi꜓ vo

\begin{english}
  Now bhikkhus I decl꜕are t꜕o you,
\end{english}

Vaya-dhammā sa꜓ṅkhā꜓rā

\begin{english}
  Change is th꜓e nature o꜕f co꜕ndi꜕tio꜕ned things;
\end{english}

A꜕ppamādena sa꜓mpā꜕dethā'ti

\begin{english}
  Perfect yo꜓urselves, no꜕t be꜕i꜕ng ne꜕gligent:
\end{english}

Ayaṁ tathā꜓ga꜕tassa pa꜕cchi꜓mā vācā

\begin{english}
  These are the Tathā꜓ga꜕ta's fi꜕na꜕l words.
\end{english}

\chapter{The Teaching on Mindfulness of Breathing}% {{{1

\firstline{Ānāpānassati bhikkhave bhāvitā bahulī-katā}

\begin{leader}
  [Ha꜓nda mayam ānāpānass꜕ati-sutta-pāṭhaṁ bha꜕ṇāmase]
\end{leader}

Ānāpāna꜓ssa꜕ti bhi꜓kkha꜕ve bhāvi꜓tā bahu꜕līka꜕tā

\begin{english}
  Bhikkhus, wh꜕en mindfulness of bre꜓athing is de꜕veloped and cu꜕ltiva꜓ted
\end{english}

Mahappha꜕lā ho꜓ti mahā꜓nisa꜓ṁsā

\begin{english}
  It is of gre꜕at fruit and great be꜕nefit;
\end{english}

Ānāpāna꜓ssa꜕ti bhi꜓kkha꜕ve bhāvi꜓tā bahu꜕līka꜕tā

\begin{english}
  Wh꜕en mindfulness of bre꜓athing is de꜕veloped and cu꜕ltiva꜓ted
\end{english}

Ca꜕ttāro sati꜓pa꜕ṭṭhāne pa꜕ri꜓pū꜕reti

\begin{english}
  It fu꜕lfills the Four Foundations of Mi꜕ndfu꜕lness;
\end{english}

Ca꜕ttāro sa꜕tipa꜕ṭṭhānā bhāvi꜓tā bahu꜕līka꜕tā

\begin{english}
  When th꜕e Four Foundations of Mi꜓ndfulness are de꜕veloped and cu꜕ltiva꜓ted
\end{english}

Sa꜕tta-bojjhaṅge pa꜕ri꜓pū꜕renti

\begin{english}
  They fu꜕lfill the Seven Factors of Awa꜕kening;
\end{english}

Sa꜕tta-bojjhaṅgā bhāvi꜓tā bahu꜕līka꜕tā

\begin{english}
  When th꜕e Seven Factors of Awa꜓kening are de꜕veloped and cu꜕ltiva꜓ted
\end{english}

Vijjā-vimuttiṁ pa꜕ri꜓pū꜕renti

\begin{english}
  They fu꜕lfill true knowledge and deli꜕verance.
\end{english}

\ifaivedition
\clearpage
\fi

Kathaṁ bhāvi꜓tā ca bhi꜓kkha꜕ve ānāpāna꜓ss꜕ati ka꜕thaṁ bahu꜕līka꜕tā

\begin{english}
  An꜕d how, bhikkhus, is mindfulness of bre꜓athing de꜕veloped and cu꜕ltiva꜓ted
\end{english}

Mahappha꜕lā ho꜓ti mahā꜓nisa꜓ṁsā

\begin{english}
  So that it is of gre꜕at fruit and great be꜕nefit?
\end{english}

Idha bhi꜓kkha꜕ve bhikkhu

\begin{english}
  Here, bhikkhus, a bhi꜕kkhu,
\end{english}

Arañña꜓-ga꜕to vā

\begin{english}
  Gone to꜕ the fo꜓rest,
\end{english}

Rukkha-mūla꜓-ga꜕to vā

\begin{english}
  To the fo꜕ot o꜕f a꜕ tree
\end{english}

Suññāgāra꜓-ga꜕to vā

\begin{english}
  Or to an em꜓pty꜕ hut.
\end{english}

N꜕isīdati pallaṅkaṁ ābhuji꜓tv꜕ā

\begin{english}
  Si꜕ts down having cro꜕ssed hi꜕s legs,
\end{english}

Ujuṁ kāyaṁ pa꜕ṇidhāya pa꜕rimukhaṁ sa꜕tiṁ u꜕paṭṭha꜕petvā

\begin{english}
  Sets his bo꜕dy꜕ e꜕rect, having established mi꜓ndfulness in fro꜕nt o꜕f him.
\end{english}

So sa꜕to'va a꜕ssasa꜕ti sa꜕to'va pa꜕ssa꜕sa꜕ti

\begin{english}
  Ever mi꜓ndful he bre꜕athes in; mindful h꜕e bre꜕athes out.
\end{english}

\ifaivedition
\clearpage
\fi

Dīghaṁ vā assa꜕sa꜓nto dīghaṁ a꜕ssasā꜓mī'ti pa꜕jānāti

\begin{english}
  Breathing i꜓n long, he꜕ knows `I bre꜕athe i꜕n long';
\end{english}

Dīghaṁ vā pa꜕ssa꜕santo dīghaṁ pa꜕ssasā꜓mī'ti pa꜕jānāti

\begin{english}
  Breathing ou꜕t long, he꜕ knows `I bre꜕athe ou꜕t long';
\end{english}

Rassaṁ vā a꜕ssa꜕santo rassaṁ a꜕ssasā꜓mī'ti pa꜕jānāti

\begin{english}
  Breathing i꜓n short, h꜕e knows `I bre꜕athe i꜕n short';
\end{english}

Rassaṁ vā pa꜕ssa꜕santo rassaṁ pa꜕ssasā꜓mī'ti pa꜕jānāti

\begin{english}
  Breathing ou꜕t short, h꜕e knows `I bre꜕athe ou꜕t short'.
\end{english}

Sabba꜕-kāya-paṭ꜕isa꜓ṁvedī a꜕ssasi꜕ssāmī'ti si꜕kkh꜕ati

\begin{english}
  He tra꜕ins thus: `I shall breathe i꜓n experiencing the whole bo꜕dy'.
\end{english}

Sabba꜕-kāya-paṭ꜕isa꜓ṁvedī pa꜕ssasi꜕ssāmī'ti si꜕kkh꜕ati

\begin{english}
  He tra꜕ins thus: `I shall breathe ou꜕t e꜕xpe꜕ri꜕enci꜕ng th꜕e who꜕le bo꜕dy'.
\end{english}

Passa꜕mbhayaṁ kāya꜕-sa꜓ṅkhāraṁ a꜕ssasi꜕ssāmī'ti si꜕kkh꜕ati

\begin{english}
  He tra꜕ins thus: `I shall breathe i꜓n tranquillizing the bodily forma꜕tions'.
\end{english}

Passa꜕mbhayaṁ kāya꜕-sa꜓ṅkhāraṁ pa꜕ssasi꜕ssāmī'ti si꜕kkh꜕ati

\begin{english}
  He tra꜕ins thus: `I shall breathe ou꜕t tra꜕nqui꜕ll꜕izi꜕ng th꜕e bo꜕dily fo꜕rmations'.
\end{english}

Pīti꜕-paṭi꜕sa꜓ṁvedī a꜕ssasi꜕ssāmī'ti si꜕kkh꜕ati

\begin{english}
  He tra꜕ins thus: `I shall breathe i꜓n experiencing ra꜕pture'.
\end{english}

Pīti꜕-paṭi꜕sa꜓ṁvedī pa꜕ssasi꜕ssāmī'ti si꜕kkh꜕ati

\begin{english}
  He tra꜕ins thus: `I shall breathe ou꜕t e꜕xpe꜕ri꜕enci꜕ng ra꜕pture'.
\end{english}

Sukh꜕a-paṭi꜕sa꜓ṁvedī a꜕ssasi꜕ssāmī'ti si꜕kkh꜕ati

\begin{english}
  He tra꜕ins thus: `I shall breathe i꜓n experiencing ple꜕asure'
\end{english}

Sukh꜕a-paṭi꜕sa꜓ṁvedī pa꜕ssasi꜕ssāmī'ti si꜕kkh꜕ati

\begin{english}
  He tra꜕ins thus: `I shall breathe ou꜕t e꜕xpe꜕ri꜕enci꜕ng ple꜕asure'.
\end{english}

Citta꜕-sa꜓ṅkhāra-paṭi꜕sa꜓ṁvedī a꜕ssasi꜕ssāmī'ti si꜕kkh꜕ati

\begin{english}
  He tra꜕ins thus: `I shall breathe i꜓n experiencing the mental forma꜕tions'.
\end{english}

Citta꜕-sa꜓ṅkhāra-paṭi꜕sa꜓ṁvedī pa꜕ssasi꜕ssāmī'ti si꜕kkh꜕ati

\begin{english}
  He tra꜕ins thus: `I shall breathe ou꜕t e꜕xpe꜕ri꜕enci꜕ng th꜕e me꜕nta꜕l fo꜕rma꜕tions'.
\end{english}

Passa꜕mbhayaṁ citta꜕-sa꜓ṅkhāraṁ a꜕ssasi꜕ssāmī'ti si꜕kkh꜕ati

\begin{english}
  He tra꜕ins thus: `I shall breathe i꜓n tranquillizing the mental forma꜕tions'.
\end{english}

Passa꜕mbhayaṁ citt꜕a-sa꜓ṅkhāraṁ pa꜕ssasi꜕ssāmī'ti si꜕kkh꜕ati

\begin{english}
  He tra꜕ins thus: `I shall breathe ou꜕t tra꜕nqu꜕ill꜕izi꜕ng th꜕e me꜕nta꜕l fo꜕rma꜕tions'.
\end{english}

Citta꜕-paṭi꜕sa꜓ṁvedī a꜕ssasi꜕ssāmī'ti si꜕kkh꜕ati

\begin{english}
  He tra꜕ins thus: `I shall breathe i꜓n experiencing th꜕e mind'.
\end{english}

Citta꜕-paṭi꜕sa꜓ṁvedī pa꜕ssasi꜕ssāmī'ti si꜕kkh꜕ati

\begin{english}
  He tra꜕ins thus: `I shall breathe ou꜕t e꜕xpe꜕ri꜕enci꜕ng th꜕e mind'.
\end{english}

A꜕bhippa꜕moda꜓yaṁ cittaṁ a꜕ssasi꜕ssāmī'ti si꜕kkh꜕ati

\begin{english}
  He tra꜕ins thus: `I shall breathe i꜓n gladdening th꜕e mind'.
\end{english}

A꜕bhippa꜕moda꜓yaṁ cittaṁ pa꜕ssasi꜕ssāmī'ti si꜕kkh꜕ati

\begin{english}
  He tra꜕ins thus: `I shall breathe ou꜕t gl꜕adde꜕ni꜕ng th꜕e mind'.
\end{english}

Sa꜕māda꜓haṁ cittaṁ a꜕ssasi꜕ssāmī'ti si꜕kkh꜕ati

\begin{english}
  He tra꜕ins thus: `I shall breathe i꜓n concentrating th꜕e mind'
\end{english}

Sa꜕māda꜓haṁ cittaṁ pa꜕ssasi꜕ssāmī'ti si꜕kkh꜕ati

\begin{english}
  He tra꜕ins thus: `I shall breathe ou꜕t co꜕nce꜕ntr꜕ati꜕ng th꜕e mind'.
\end{english}

Vimoca꜓yaṁ cittaṁ a꜕ssasi꜕ssāmī'ti si꜕kkh꜕ati

\begin{english}
  He tra꜕ins thus: `I shall breathe i꜓n liberating th꜕e mind'.
\end{english}

Vimoca꜓yaṁ cittaṁ pa꜕ssasi꜕ssāmī'ti si꜕kkh꜕ati

\begin{english}
  He tra꜕ins thus: `I shall breathe ou꜕t li꜕be꜕ra꜕ti꜕ng th꜕e mind'.
\end{english}

Aniccānupa꜕ssī a꜕ssasi꜕ssāmī'ti si꜕kkh꜕ati

\begin{english}
  He tra꜕ins thus: `I shall breathe i꜓n contemplating impe꜕rmanence'.
\end{english}

Aniccānupa꜕ssī pa꜕ssasi꜕ssāmī'ti si꜕kkh꜕ati

\begin{english}
  He tra꜕ins thus: `I shall breathe ou꜕t co꜕nte꜕mpla꜕ti꜕ng i꜕mpe꜕rmanence'.
\end{english}

Virāgānupa꜕ssī a꜕ssasi꜕ssāmī'ti si꜕kkh꜕ati

\begin{english}
  He tra꜕ins thus: `I shall breathe i꜓n contemplating the fading away of pa꜕ssions'.
\end{english}

Virāgānupa꜕ssī pa꜕ssasi꜕ssāmī'ti si꜕kkh꜕ati

\begin{english}
  He tra꜕ins thus: `I shall breathe ou꜕t co꜕nte꜕mpl꜕ati꜕ng th꜕e fa꜕di꜕ng aw꜕ay o꜕f pa꜕ssions'.
\end{english}

Nirodhānupa꜕ssī a꜕ssasi꜕ssāmī'ti si꜕kkh꜕ati

\begin{english}
  He tra꜕ins thus: `I shall breathe i꜓n contemplating cessa꜕tion'.
\end{english}

Nirodhānupa꜕ssī pa꜕ssasi꜕ssāmī'ti si꜕kkh꜕ati

\begin{english}
  He tra꜕ins thus: `I shall breathe ou꜕t co꜕nte꜕mpl꜕ati꜕ng ce꜕ss꜕ation'.
\end{english}

Pa꜕ṭiniss꜕aggānupa꜕ssī a꜕ssasi꜕ssāmī'ti si꜕kkh꜕ati

\begin{english}
  He tra꜕ins thus: `I shall breathe i꜓n contemplating reli꜕nquishment'.
\end{english}

Pa꜕ṭinissa꜕ggānupa꜕ssī pa꜕ssasi꜕ssāmī'ti si꜕kkh꜕ati

\begin{english}
  He tra꜕ins thus: `I shall breathe ou꜕t co꜕nte꜕mpl꜕ati꜕ng re꜕li꜕nquishment'.
\end{english}

Evaṁ bhāvi꜓tā kho bhi꜓kkha꜕ve ānāpāna꜓ss꜕ati evaṁ bahu꜕līka꜕tā

\begin{english}
  Bhikkhus, that is ho꜕w mindfulness of bre꜓athing is de꜕veloped and cu꜕ltiva꜓ted
\end{english}

Mahappha꜕lā ho꜓ti mahā꜓nisa꜓ṁsā'ti

\begin{english}
  So that it is of gr꜕eat fruit and great be꜕nefit.
\end{english}

% }}}1

% End of reflections-and-recollections-p2.tex
