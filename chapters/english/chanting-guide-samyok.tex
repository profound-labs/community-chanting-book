\cleartorecto
\chapter{Chanting Guide}

The syllables are chanted in either mid (unmarked), high (puna꜓), low
(ca꜕paraṃ), or rising tone (kho꜔꜒), according to the tonal rules of Thai
syllables. This is known as the \emph{saṃyok} style.

In the main text only the rising tone is marked to serve as a reminder. A
complete example can be seen below.

The syllable has rising tone (kho꜔꜒), if:

\begin{itemize}[itemsep=0pt, parsep=0pt, topsep=0pt]
  \item it begins with either: s, h, ch, th, ṭh, kh, ph, and
  \item is a long syllable, and
  \item is not stopped.
\end{itemize}

`sm-' syllables are an exception: tasmā꜔꜒, āyasmā꜔꜒, yasmi꜔꜒ṃ, \ldots{}

A syllable is `stopped' when it ends with either of:\\
k, c, t, ṭ, p, g, j, d, ḍ, b, or s.

\enlargethispage{2\baselineskip}

\section*{Iriyāpathapabba}

Puna꜓ ca꜕paraṃ, bhi꜓kkha꜕ve, bhi꜓kkhu꜕ ga꜕cchanto꜔꜒ vā ‘ga꜕cchā꜔꜒mī’ti꜕ pajānāti꜕, ṭhi꜕to vā
‘ṭhi꜕tomhī꜔꜒’ti pajānāti꜕, ni꜓si꜔꜒nno vā ‘ni꜓si꜔꜒nnomhī꜔꜒’ti꜕ pajānāti꜕, sa꜕yāno vā
‘sayānomhī꜔꜒’ti꜕ pajānāti꜕, ya꜓thā꜔꜒ ya꜓thā꜔꜒ vā pana꜓ssa꜕ kāyo paṇi꜓hi꜕to ho꜔꜒ti꜓ ta꜕thā꜔꜒ ta꜕thā꜔꜒
naṃ pajānāti꜕.

Iti ajjhattaṃ vā kāye kāyānupassī꜔꜒ viharati, bahiddhā vā kāye kāyānupassī꜔꜒
viharati, ajjhatta-bahiddhā vā kāye kāyānupassī꜔꜒ viharati. samudaya-dhammānupassī꜔꜒
vā kāyasmiṃ viharati, vaya-dhammā-\\
nupassī꜔꜒ vā kāyasmi꜔꜒ṃ viharati, samudaya-vaya-dhammānupassī꜔꜒ vā kāyasmi꜔꜒ṃ viharati.
‘atthi kāyo’ti vā panassa sati paccupaṭṭhitā ho꜔꜒ti yāvadeva ñāṇamattāya
paṭissatimattāya anissito ca viharati, na ca kiñci loke upādiyati. evampi kho꜔꜒,
bhikkhave, bhikkhu kāye kāyānupassī꜔꜒ viharati.
