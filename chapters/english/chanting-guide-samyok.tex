\cleartorecto
\chapter{Chanting Guide}

The syllables are chanted in either mid (unmarked), high (puna꜓), low
(ca꜕paraṁ), or rising tone (kho꜔꜒), according to the tonal rules of Thai
syllables. This is known as the \emph{saṁyok} style.

In the main text only the rising tone is marked to serve as a reminder. A
complete example can be seen below.

The syllable has rising tone (kho꜔꜒), if:

\begin{itemize}[itemsep=0pt, parsep=0pt, topsep=0pt]
  \item it begins with either: s, h, ch, th, ṭh, kh, ph, and
  \item is a long syllable, and
  \item is not stopped.
\end{itemize}

`sm-' syllables are an exception: tasmā꜔꜒, āyasmā꜔꜒, yasmi꜔꜒ṁ, \ldots{}

A syllable is `stopped' when it ends with either of:\\
k, c, t, ṭ, p, g, j, d, ḍ, b, or s.

\enlargethispage{2\baselineskip}

\section*{Iriyāpathapabba}

Pu꜕na꜓ ca꜕pa꜕raṁ, bhi꜓kkha꜕ve, bhi꜓kkhu꜕ ga꜓ccha꜔꜒nto vā ‘ga꜓cchā꜔꜒mī’ti꜕ pa꜕jānāti꜕, ṭhi꜕to vā
‘ṭhi꜕tomhī꜔꜒’ti꜕ pa꜕jānāti꜕, ni꜓si꜔꜒nno vā ‘ni꜓si꜔꜒nnomhī꜔꜒’ti꜕ pa꜕jānāti꜕, sa꜕yāno vā
‘sa꜕yānomhī꜔꜒’ti꜕ pa꜕jānāti꜕, ya꜓thā꜔꜒ ya꜓thā꜔꜒ vā pa꜕na꜓ssa꜕ kāyo pa꜕ṇi꜓hi꜕to ho꜔꜒ti꜕ ta꜕thā꜔꜒ ta꜕thā꜔꜒
naṁ pa꜕jānāti꜕.

I꜕ti꜕ a꜕jjha꜓ttaṁ vā kāye kāyānu꜓pa꜕ssī꜔꜒ vi꜓ha꜕ra꜓ti꜕, ba꜓hi꜕ddhā vā kāye kāyānu꜓pa꜕ssī꜔꜒
vi꜓ha꜕ra꜓ti꜕, a꜕jjha꜓tta꜕-ba꜓hi꜕ddhā vā kāye kāyānu꜓pa꜕ssī꜔꜒ vi꜓ha꜕ra꜓ti꜕. sa꜕mu꜓da꜓ya꜓-dhammānu꜓pa꜕ssī꜔꜒
vā kāya꜓smi꜔꜒ṁ vi꜓ha꜕ra꜓ti꜕, va꜓ya꜓-dhammā-\\
nu꜓pa꜕ssī꜔꜒ vā kāya꜓smi꜔꜒ṁ vi꜓ha꜕ra꜓ti꜕, sa꜕mu꜓da꜓ya꜓-va꜓ya꜓-dhammānu꜓pa꜕ssī꜔꜒ vā kāya꜓smi꜔꜒ṁ vi꜓ha꜕ra꜓ti꜕.
‘a꜕tthi꜕ kāyo’ti꜕ vā pa꜕na꜓ssa꜕ sa꜕ti꜕ pa꜕ccu꜕pa꜕ṭṭhi꜕tā ho꜔꜒ti꜕ yāva꜓deva꜓ ñāṇa꜓ma꜓ttāya꜓
pa꜕ṭi꜕ssa꜕ti꜕ma꜓ttāya꜓ a꜕ni꜓ssi꜕to ca꜕ vi꜓ha꜕ra꜓ti꜕, na꜓ ca꜕ kiñci꜕ loke u꜕pādi꜓ya꜓ti꜕. evampi꜕ kho꜔꜒,
bhi꜓kkha꜕ve, bhi꜓kkhu꜕ kāye kāyānu꜓pa꜕ssī꜔꜒ vi꜓ha꜕ra꜓ti꜕.
